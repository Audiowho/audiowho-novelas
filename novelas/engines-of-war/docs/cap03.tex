\chapter*{Capítulo Tres}
\addcontentsline{toc}{chapter}{Capítulo Tres}

La TARDIS se sumergió en la atmósfera superior del planeta cayendo como una piedra, dando tumbos y dejando una ondulante estela de humo negro a su paso. 

En el interior, el Doctor se aferraba a la barandilla metálica que rodeaba los bordes de la tarima central. Los motores chirriaban y traqueteaban mientras la nave intentaba enderezarse, pero la trayectoria era demasiado pronunciada y estaban cayendo demasiado rápido. 

El techo mostraba aún una proyección de la vista del exterior, pero ahora no era más que un desorientador revoltijo de imágenes: instantáneas de un cielo púrpura amoratado, panorámicas de continentes repletos de ruinas erizadas, llamas lamiendo furiosamente los bordes de la cubierta exterior de la nave. 

Con un esfuerzo descomunal, el Doctor se soltó de la barandilla y fue tambaleándose hacia la consola, agarrándose a un cable de aros en un esfuerzo para evitar caerse despatarrado en el suelo. Tiro de él para apoyarse, pero, para su consternación, se le desprendió de la mano, uno de los extremos se separó de su alojamiento y le hizo oscilar salvajemente, agitó la otra mano como un molinete hasta que la nave se inclinó hacia delante de nuevo y pudo agarrarse a una palanca cercana. 

Se enderezó como buenamente pudo, balanceándose con el movimiento de la bamboleante nave. 



—Bien, veamos si esto funciona… —dijo, tirando el extremo suelto del cable y golpeando una serie de botones e interruptores del panel de control. 

Los motores gritaron en señal de protesta, y la TARDIS hizo un tembloroso intento de desmaterializarse. En el exterior, visible por el techo transparente, el mundo pareció desvanecerse hacia la no-existencia, sustituido por las turbulentas tonalidades del Vórtice del Tiempo. 

Justo cuando el Doctor estaba a punto de emitir un profundo suspiro de alivio, la vista vaciló como si estuviera fuera de alcance, y volvió a centellear imágenes del mundo desolado y ruinoso de abajo, viéndose solo en fragmentos como si el suelo se apresurarase a encontrarse con la descendiente TARDIS. 

Golpeó los controles con furia, en vano. Incluso la columna central había cesado ahora de subir y bajar, como si la TARDIS misma hubiera previsto lo que venía después y se encerrase en sí misma, apagando sus sistemas vitales. 



—Lo siento, vieja amiga —dijo el Doctor, agarrándose a la consola con toda su alma—. Creo que vamos a tener un aterrizaje algo brusco… 



Tenía la boca llena de tierra, la mejilla izquierda le escocía y estaba segura de que se había roto al menos una costilla. No podía recordar dónde estaba, ni lo que había estado haciendo. Una negrura reconfortante le ofrecía consumirla. Le dio la bienvenida. Dormir, dormir era lo que ella necesitaba. Dormir... 



—Localizad al otro humano. 



El áspero sonido metálico de una Degradación la agitó hasta desvelarla. ¡Por supuesto! El acantilado. El corrimiento de tierra. Las Degradaciones. Debían de haber pasado sólo unos pocos segundos. Se quedó rígida e inmóvil. ¿La creerían muerta? 

Estaba parcialmente cubierta por tierra suelta. Podía sentir el peso en sus piernas. Eso era bueno, al menos todavía podía sentir las piernas. El lodo debió de haber amortiguado su caída. Movió un pie, muy ligeramente, y sintió que la tierra amontonada cedía. Podría liberarse. No estaba enterrada a mucha profundidad. 

Todavía seguía agarrando el arma Dalek robada. La notaba suave y fría en la palma de su mano, y zumbaba llena de energía. No sólo eso, sino que tenía el factor sorpresa. No esperarían que comenzase a disparar otra vez. Y por lo visto no habían encontrado a Finch. No habían... 



—¡Cinder! —el grito de preocupación de Finch resonó por las ruinas. Cinder quería gritar de frustración. ¡Qué estaba haciendo! Estaba revelando su posición, convirtiéndose en un blanco fácil.

Bien, supuso que había sido obligado a hacerlo... 

Con un jadeo, Cinder se desenterró de la tierra amontonada, girando mientras se levantaba, escupiendo tierra. No tuvo tiempo de evaluar lo que las Degradaciones estaban haciendo. Vio uno de los Planeadores, flotando a unos pocos metros del suelo dándole la espalda, y apuntó, disparando dos tiros. Aún girando, tuvo a otro Planeador en su punto de mira y disparó otros dos tiros. 

Estallaron en bolas brillantes de fuego, uno tras otro, regando la tierra con escombros en llamas, y Cinder se puso a cubierto, rodando detrás del armazón del Dalek que había eliminado desde arriba. Todavía quedaban dos Degradaciones a las que enfrentarse y no tenía muchas posibilidades contra el cañón. 



—¡Cinder! 



Se puso en pie para ver la alta y ancha silueta de Finch delante, saliendo desde detrás de una pared rota y corriendo hacia la calle. Vestía un sucio mono negro y llevaba una ametralladora anticuada, con la que regaba a balazos a las restantes criaturas Dalek mientras corría. Las balas golpearon inútilmente sus armazones, pero su plan, si es que era un plan, había funcionado y les había distraído el tiempo suficiente para que Cinder se pusiese a cubierto. 



—¡Cinder, ponte a salvo, ahora! —rugió Finch. Roció a las Degradaciones con otra ráfaga de munición inútil, después se volvió y echó a correr. 

—¡Erradicar! —se erizo el Dalek del cañón, girando su sección media para rastrearlo mientras corría. 

—¡Finch! —gritó Cinder—. ¡No!

 

El cañón disparó, emitiendo un pulso de una espeluznante luz color rubí. Golpeó a Finch en la espalda y pareció engullirlo por completo, rodeando su cuerpo, susurrando a su alrededor como si buscara una forma de entrar. Dejó de correr, retorciéndose en clara agonía y forcejeando como si tratara de liberarse del abrazo mortal del rayo. No había escapatoria. 

Abrió la boca para gritar y el haz de luz se precipitó a través del orificio bucal, inundando su cuerpo, asfixiándole. Se agarró la garganta con ambas manos, buscando aire. 

Mientras miraba con las lágrimas picándole los ojos, la carne de Finch comenzó a brillar, cogiendo el mismo extraño tono rosado que la luz. Parecía desintegrarse ante ella, desapareciendo de la existencia, como si la luz de su interior estuviera empujando y expandiéndose hacia afuera, disolviéndolo desde dentro. 

En unos pocos segundos, ya no quedaba nada de él, salvo un tenue jirón de luz que se desvanecía lentamente. 

Agazapándose detrás del quemado Dalek, Cinder sintió una extraña sensación. Sabía que acababa de presenciar algo horrible, pero, por alguna razón, no podía entender totalmente el qué. Su memoria parecía de repente borrosa, confusa. 

Tenía la inquietante sensación de que había algo que no podía recordar, arañando el fondo de su mente. Podía haber jurado que las Degradaciones acababan de exterminar a alguien, quizá incluso a alguien que conocía, pero no podía imaginar a quién podía haber sido. Después de todo, ella había planeado la emboscada sola, sin ayuda. ¿Verdad? 

No obstante, no podía negar la abrumadora sensación de vacío, como si estuviera experimentando la ausencia de una emoción similar a la pena. Sin embargo no tenía tiempo para pensar en ello, cuando las dos Degradaciones que quedaban se movían, volviéndose hacia ella… 

Miró hacia atrás, buscando algún lugar hacia el que correr. No había ningún sitio, sólo las ruinas del otro lado de la calle y no tenía muchas posibilidades en campo abierto. Por otra parte, el armazón destrozado de un Dalek tampoco le iba a hacer de escudo durante mucho tiempo. 

Cinder alzó la vista hacia un silbido agudo encima de su cabeza, quedándose boquiabierta de asombro. Algo estaba cayendo del cielo, una gran caja azul con iluminadas ventanas apaneladas y una luz parpadeante en la parte superior. Estaba acercándose a bastante velocidad, brillando al rojo vivo por los bordes y dejando una larga y oscura mancha en el cielo a su paso. Fuera lo que fuese, estaba claramente fuera de control e iba a aterrizar en cualquier momento… 



—¡Evadir! ¡Evadir! —la Degradación en forma de huevo se volvió y se deslizó hacia las ruinas, sus extremidades arácnidas se clavaban en el destrozado suelo para asirse. 

Cinder se encogió, cayendo de rodillas y hundiendo la cara en los brazos. Había poco más que pudiera hacer. El rugido de la caja que caía había aumentado en tal intensidad que era lo único que podía oír. No había tiempo para correr a buscar refugio. Estaba viniendo, y estaba viniendo ahora. 

Impactó con un tremendo crujido, levantando un borbotón de tierra desplazada que lanzó a Cinder y al armazón del Dalek muerto en el que había estado acurrucada, al menos dos metros por el aire. Aterrizó de espaldas, quedándose sin aire en los pulmones, justo cuando la caja, que había rebotado en el borde del acantilado y se lanzaba a toda velocidad hacia la calle, se estrelló por segunda vez, causando esta vez una explosión colosal. Por segunda vez en el día, fue rociada con un chorro de tierra suelta y escombros. 

La caja azul chirrió contra el asfalto, desgarrando lo que parecía ser madera, hasta que golpeó en los restos de una pared de ladrillo y llegó a un alto repentino y estridente. 

Cinder respiró hondo y abrió los ojos. Lo primero que le llamó la atención fue el hecho de que aún seguía viva. Lo segundo fue el extraño silencio que se había apoderado del momento. El único sonido era el siseo de la chamuscada caja derritiendo el asfalto en la superficie de la calle donde se había detenido. No tenía ni idea de cómo una caja de madera podría haber sobrevivido a la violencia de la reentrada a la atmósfera del planeta. 

Cinder se levantó, se quitó los polvorientos fragmentos de carcasa Dalek y la suciedad de la ropa. Abrió la boca para respirar, forzando al aire a bajar a sus pulmones. Los oídos le zumbaban. Se tambaleó hacia delante unos pocos pasos, pero después se lo pensó mejor y decidió que esperaría hasta que su cabeza dejase de dar vueltas. 

Trató de orientarse. 

Toda la escena era un caos. El impacto inicial había abierto un cráter en un lado del acantilado, la fuerza con la que había ondulado, había arrugado la superficie de la calle y destrozado un área del tamaño de una casa. 

El armazón del Dalek yacía a un lado a unos tres metros de distancia, meciéndose todavía suavemente con el movimiento del impacto. 

El humo ondulaba desde donde la caja azul se había detenido finalmente, tumbada de lado. Una escotilla estaba abierta en la parte superior, pero no podía ver el interior. Las luces aún brillaban suavemente en las ventanas, aunque la lámpara de la parte superior se había apagado. Se preguntó si esa era la señal de emergencia o un dispositivo de rastreo. 

Al parecer la caja también le había salvado la vida. La mitad de una carcasa Dalek, presumiblemente perteneciente a la Degradación del cañón, seguía de pie junto a la caja volcada, pero a la mitad superior no se la veía por ningún lado. Al parecer la caja había decapitado a la cosa antes de que hubiese tenido la oportunidad de quitarse de enmedio. 

De los mutantes achaparrados en forma de araña, no había ni rastro. 

Cinder se arrastró hacia delante, observando la caja. Lo único que podía ver era una cortina de humo espeso y la impresión de alguna brillante iluminación interior. Pensó en llamar para ver si alguien seguía vivo en el interior, pero le preocupaba llamar la atención. Y además no tenía ni idea de quién, o qué, podría estar ahí. No, se acercaría un poco más y miraría dentro… 

Se paralizó al oír el sonido de un hombre farfullando. Había venido del interior de la caja. Así que el ocupante seguía vivo. 

Rápidamente, se lanzó a por su arma. Sobresalía de la tierra húmeda cercana y apresuradamente la desenterró con las manos, consiguiendo que la gruesa y mugrienta arcilla se le metiera debajo de las uñas rotas. Tiro de ella para liberarla, arrastrando cables, y después la desempolvó y la revisó. 

La luz de la fuente de alimentación se había atenuado y vuelto de color rojo, lo que indicaba que toda la energía almacenada se había descargado. Claramente había sido dañada en la explosión. Maldijo entre dientes. Aún así, quienquiera que fuese el que había bajado en esa caja azul no tenía por qué saberlo. El arma aún podía ser un eficaz medio de disuasión. 

Blandiéndola como si fuera un escudo, avanzó lentamente hacia la caja, atenta a cualquier repentina señal de movimiento que indicase hostilidad. ¿Era una cápsula de escape? Desde luego no parecía muy grande, y la forma en la que había caído del cielo sugería que había sido expulsada de una nave en órbita. Los bordes de la caja aún brillaban a causa de su abrasiva entrada a la atmósfera y una raya negra de hollín cruzaba su carcasa exterior indicando que había sido alcanzada por un arma de energía. ¿Un platillo Dalek había derribado la nave? Se preguntó incluso si el ocupante de la cápsula de escape sería humano. Pero, ¿por qué estaban escritas las palabras “CABINA DE POLICÍA” en un lateral en letras grandes y gruesas? Nada de lo que estaba pasando parecía tener sentido. 

El hombre tosió de nuevo, esta vez más fuerte. Cinder notó un movimiento. Se detuvo y apuntó su arma en dirección a la caja, justo a tiempo de ver emerger una cabeza de la escotilla abierta. 

Con un fuerte resoplido, el hombre alzó los brazos por los lados de la caja y se aupó hacia arriba, de manera que la cabeza y los hombros le sobresalían por encima del borde. 

Cinder le fulminó con la mirada sin saber qué decir ni qué hacer. Era un hombre mayor con un anguloso rostro agobiado y llamativos ojos marrón verdosos. Su cabello era de color gris plateado, cepillado en una cresta por la parte delantera, y llevaba una tupida barba blanca y bigote. Frunció el ceño, perplejo. Parecía llevar un maltrecho abrigo de cuero y una bufanda estampada de espiga. 



—¿Y bien? —dijo, como si estuviera esperando una respuesta a una pregunta no formulada. 

—¿Y bien qué? —respondió Cinder, agitando el arma para asegurarse de que él la había visto. 

Arqueó las cejas como sorprendiéndose por su insolencia. 

—Oh, así que agitar un arma ante mi es la mejor cosa que puedes hacer en las actuales circunstancias, ¿no? 

—Bueno… —Cinder lo consideró por un instante, confusa—. ¡Oye, eres tu el que acaba de caer del cielo! 

—Y menos mal que lo hice —dijo—. Diría que mi precisión es impecable. 

—¿De qué estás hablando? —dijo Cinder, no pudiendo reprimir su exasperación. 

—Mirate —dijo—. Es evidente que necesitas mi ayuda. 

Cinder sintió una oleada de indignación. 

—Oh, ¿en serio? —sacudió la cabeza ante la arrogancia del hombre—. ¿Necesito tu ayuda? 

—Diría que sí —respondió el hombre. 

—¿Y qué te hace pensar eso? —preguntó Cinder. Estaba cansándose de este irritante recién llegado y de su ridícula pose. 

El hombre hizo un gesto que podría haber sido un encogimiento de hombros, si no hubiera sido por el hecho de que estaba colgado en el borde de la caja con ambos brazos. Ahora que lo pensaba, la posición parecía un poco extraña, teniendo en cuenta lo poco profunda que era en realidad la caja. El hombre suspiró. 

—Si no quieres acabar exterminada, te sugiero que te muevas y saltes dentro. 

—¿Qué? —dijo—. ¿Quieres que me meta contigo en esa caja? Le puso su mejor expresión de ”jamás en la vida”. 

—No quiero que hagas nada —dijo el hombre—, pero a no ser que seas tan estúpida como pareces, harás lo que te digo. 

Cinder tuvo que luchar contra la tentación de apretar el gatillo del arma con la esperanza de que hubiera la suficiente carga residual en la fuente de alimentación como para volarle en pedazos. 

—Bien —dijo—. Conmigo no cuentes. 

Se dio la vuelta para alejarse. 

—¡AHORA! —bramó el hombre. Había una sensación de urgencia en su voz que no había estado ahí antes, un estímulo que le hizo de repente decidir prestar atención. 

—¡Ex-ter-mi-nar! 

Cinder se giró al mismo tiempo que veía a la cosa en forma de araña emergiendo de las ruinas a su derecha. Maldijo en voz alta. Había estado tan concentrada en su discusión con el hombre que no había estado prestando atención. Debería haberlo sabido. Apuntó su arma a la Degradación y apretó el gatillo, pero como había esperado, no pasó nada. Estaba completamente descargada. 

A Cinder se le agotaban rápidamente las opciones. Podía quedarse aquí y tratar de luchar contra una Degradación con un arma que resultaría tan útil como un palo de madera, intentar huir y exponerse a recibir un disparo en la espalda o lanzarse dentro de una caja azul con un anciano que acababa de caer del cielo. 



—Salir de la sartén para caer en las brasas —murmuró. 



Mientras la Degradación llegaba trepando a los restos de una pared, soltando una ráfaga de ladrillos sueltos, ella retrocedió, tomó carrerilla y saltó a la escotilla de la cápsula de escape. Contrajo las rodillas contra el pecho mientras saltaba, preparándose para caer de cuclillas cuando aterrizase dentro de la poco profunda caja. 



—¡Entrando! —gritó, para dar al hombre la oportunidad de ponerse a cubierto antes de que aterrizase sobre él. 



Cayó de espaldas, golpeándose dolorosamente en lo que parecían placas de suelo metálicas, y rodó hacia su izquierda, poniendo una mano para detenerse. Con la otra aún agarraba cerca del pecho el arma Dalek. 

El ímpetu la llevó a un lado y acabó con la cara estampada contra frío metal, que parecía ronronear suavemente con la vibración de un motor al ralentí. 

Algo no estaba bien. 

Había mantenido los ojos cerrados durante la caída. Los abrió, esperando ver al anciano apretujado contra ella en el reducido espacio, a cubierto de la Degradación de afuera. En cambio la recibió la visión de una gran sala circular. 

Se sentó, agarrando el arma contra el pecho. 

La habitación era totalmente incongruente con lo que había esperado. Las paredes resplandecían con una serie de extrañas marcas redondas, luces ocultas, quizá, y ásperos pilares de piedra se arqueaban hacia arriba para sostener el techo. 

Una tarima elevada alojaba lo que parecía un panel de control, o algo parecido, aunque los controles en cuestión parecían ser remendados e improvisados a partir de componentes rescatados que habían sido adaptados para hacer que encajaran. Nidos de cables colgaban del techo. 

El lugar entero tenía un cierto estilo embarullado, como si constantemente estuviera siendo transformado por un pillastre empedernido, o reparado por alguien que nunca fue capaz de conseguir las piezas correctas. Era la sala de control de una nave. Supuso que podría haberse golpeado durante el salto a la cápsula de escape y haber vuelto en sí, horas después, en un sitio distinto. Pero aunque trataba de convencerse a sí misma, no creía que fuese eso ni por asomo. 

El hombre cuya cabeza y hombros había visto sobresalir de la caja estaba ahora de pie junto al panel de control, tratando de ajustar la imagen en una pequeña pantalla de ordenador. Estaba de espaldas a ella, pero definitivamente era el mismo hombre, llevaba la misma chaqueta marrón y su cabello era del mismo color gris plateado. 

Miró hacia atrás. Extrañamente, estaba sentada de espaldas a la escotilla. La estudió por un momento, evaluando el tamaño y la forma de la abertura. Supuso, tras reflexionar, que técnicamente era más que una puerta, pero parecía la correcta. Era sin duda la escotilla por la que había saltado. 

—Es… es… —tartamudeó. 

El hombre dejó de hacer lo que estaba haciendo y la miró. 



—Más grande por dentro. Sí, lo sé. Superemos eso y hagámoslo rápido, ¿eh? —dijo. 

—…está en la posición correcta, —terminó Cinder—. La caja estaba volcada de lado y ahora estoy en la posición correcta. 

—Oh. Bien. Ummm. No esperaba eso —dijo—. Sí, supongo que sí. Eso serán los estabilizadores relativos dimensionales. Hace que dejes de, bueno…caerte —la miró y levantó una ceja con ironía—. El interior se puede orientar de distinta manera que el exterior —agitó la mano como si explicase un milagro nada más que como un juego de manos. 

—Y es más grande —dijo Cinder. 

El hombre se echó a reír. 

—Y ahí lo tenemos. Eso es lo que estaba esperando. 

—Lo que significa… —la expresión de Cinder se oscureció—. ¿Esto es una TARDIS?

—Lo es —dijo el hombre. Volvió su atención a la consola y empezó a examinar las lecturas en la pantalla de ordenador. Se veía anticuado y un poco decrépito. Dio unos golpecitos en el teclado, como si intentase que algo funcionase. 

Cinder se asomó por encima de su hombro para mirar lo que estaba viendo, pero todo lo que pudo ver en la pantalla fue una masa de extraños pictogramas, desplazándose y cambiando en un aparente baile aleatorio. 

—¡Maldita sea! —ladró de repente en respuesta a algo que había leído y Cinder se sobresaltó, su dedo rozó el gatillo de su arma. 

—Si esto es una TARDIS —dijo—. Entonces significa que tu eres un… 

—Un Señor del Tiempo —dijo interrumpiéndola—. Sí, es correcto. Bien hecho —su tono era condescendiente. 

Cinder respiró hondo. Retrocedió, arrastrándose de espaldas. Alzó el cañón de su arma hasta apuntar al Señor del Tiempo. Estaba empezando a pensar que tendría mejor suerte ahí fuera con los mutantes Dalek. Podía oir a una de las Degradaciones golpear la puerta  tratando de entrar. Afortunadamente, las puertas de la TARDIS parecían contenerla. 

—¿Qué vas a hacer conmigo? —dijo con voz temblorosa. 

El Señor del Tiempo suspiró. 

—Te dejaré en lugar seguro tan pronto me sea posible —dijo—. Así podré conseguir un poco de paz y tranquilidad —la miró como sopesando su respuesta. 

—¿Dime por qué no debería matarte ahora? —dijo blandiendo el arma Dalek. No tenía forma de saber que estaba dañada, que toda la carga que tenía se había agotado. 

—Porque te he salvado la vida —dijo, razonablemente—. Porque no pareces una asesina y porque la fuente de alimentación de tu arma recuperada está completamente muerta. Alcanzó el panel de control y comenzó a pulsar interruptores. 

—¡Me has salvado la vida! —le espetó indignada—. ¡Casi me matas aplastándome, precipitándote a toda velocidad por el cielo en tu.. tu… caja! —maldijo en voz baja por la frustración. Debía de haberla visto tratar de disparar a la Degradación y comprender que no le quedaba energía. Eso significaba que estaba expuesta. Sin embargo, todavía sería capaz de ganarle en una pelea si intentaba algo. Después de todo era mucho más joven que él. 

—Oh, ya veo. ¿Así que hubiera sido la simplicidad misma librarte tu sola de esa patrulla Dalek? 

No se creía mucho ese tono condescendiente, dado que básicamente había estrellado su nave. Sacó algo de dentro del pliegue de su chaqueta, pero no pudo ver lo que era. 

—No eran Dalek —alegó—. Ya me había ocupado del Dalek. Eran mutantes. Degradaciones. 

El Señor del Tiempo se encogió de hombros. 

—Un Dalek es un Dalek —dijo—. Cualquiera que sea su forma y en cualquier época o permutación de la realidad del que procedan. 

—¿No es eso verdad también para los Señores del Tiempo? —preguntó Cinder. Su voz iba cargada de sarcasmo. 

—Tristemente, creo que así es —respondió. 

—Pero, ¿eres un Señor del Tiempo? —dijo, agitando el arma para asegurarse de que no la había olvidado. No estaba mirando. Había vuelto a jugar con el objeto que tenía en la mano, un delgado cilindro de metal con un extremo brillante que hacía un zumbido exasperante cada vez que se pulsaba un botón que tenía. 

—Sí —dijo, arrastrando la palabra, como indicando su impaciencia. Se llevó el dispositivo a la oreja y presionó el botón, escuchando con atención el sonido. Después, con el ceño fruncido como si se frustrara con el objeto, lo golpeó repetidas veces contra la palma de la mano. 

—Entonces, ¿dónde están el gorro y los ropajes? —dijo Cinder—. No te pareces mucho a un Señor del Tiempo. 

—Me han dicho que hay excepciones para cada regla —respondió. Levantó el dispositivo hacia su oreja de nuevo, escuchó el sonido y después, aparentemente satisfecho, colocó el dispositivo en un aro de cuero de la correa de munición que llevaba y se sacudió las manos. 

—¿Qué es esa cosa? ¿Un arma? —dijo. 

Le ofreció una mirada de impaciencia. 

—No, es un destornillador. Ahora, ¿por qué no bajas ese arma? Estas molestando a mi vieja amiga —le dio una palmadita a la consola de la TARDIS con cariño—. Y para ser absolutamente francos, me estás molestando a mi también. 

Cinder ignoró la última parte de la burla. 

—¿Quieres decir más de lo que acabas de molestarla estrellándola contra un planeta? —replicó. De todos modos, bajó el cañón del arma, aunque se negó a renunciar por completo a sujetarla. 

—Eso es —dijo el Señor del Tiempo—.¿No te sientes mejor? 

Cinder soltó un suspiro de exasperación. 

—Oye, ¿qué estás haciendo aquí en Moldox? 

—Ah, así que es así como se llama este planeta espantoso, ¿eh? Moldox. —dijo la palabra como si estuviera probando la medida, después sacudió la cabeza como si decidiese que no iba con él—. La cuestión es, ¿qué estabas haciendo ahí fuera, frente a frente contra los Dalek? 

—Una emboscada —dijo. 

El Señor del tiempo la miró con aprobación. 

—¿Una emboscada? —repitió—. Solo tú, tu amigo y una única arma de energía Dalek rescatada. Estoy impresionado —momentáneamente la miró con tristeza—. Lo siento, no pude salvarlo. 

Cinder le miró, confusa. 

—¿Mi amigo? Estaba sola. 

El Señor del Tiempo frunció el ceño. 

—La TARDIS registró dos signos de vida humana en la zona del accidente. Uno de ellos desapareció justo después de una descarga masiva de energía proveniente de uno de los Dalek. Había asumido que estabais juntos. 

De nuevo esa extraña comezón en la parte posterior de su mente como si hubiera algo que debiera ser capaz de recordar, pero no pudo. 

—Yo… —vaciló—. Creo que no —dijo. 

El Señor del Tiempo asintió, pero era evidente que le preocupaba su respuesta. 

—Bueno, puedes sentirte como en tu casa por un minuto o dos —dijo, dando una vuelta a la consola, ajustando los controles—. Sólo voy a ponerla en marcha de nuevo. Agarró una palanca con un desgastado mango de madera y tiro de ella La alta cámara de vidrio en el centro de la consola parpadeó brevemente con una brillante luz blanca y el conjunto de tubos en su corazón comenzó a levantarse en el interior de la columna. Pero entonces la luz se atenuó y se oyó un profundo e inquietante gemido debajo del suelo. 

—¡Maldita sea! —dijo el Señor del Tiempo golpeando furioso el panel de control con el puño—. Está inservible. Va a necesitar algún tiempo para sanarse antes de que pueda llevarla fuera de este mundo. 

—¿Fuera de este mundo? —dijo Cinder. Un repentino pensamiento había entrado sin querer en su cabeza. ¿Era esto? ¿Era esta la oportunidad para escapar que había estado buscando? ¿Podría viajar fuera del planeta con este excéntrico y viejo Señor del Tiempo? La idea era atractiva. Había jugado con la idea de abandonar Moldox cientos de veces a lo largo de los años, pero nunca se había presentado la oportunidad. ¿Podría ser esta? Su oportunidad para un nuevo comienzo, un lugar donde la guerra no fuera más que un recuerdo lejano, un cuento de hadas que se contaba a los jóvenes para fomentar el buen comportamiento. Lugares como ese debían de existir en algún lugar del cosmos. 

—Bueno, no es que tengamos una particular prisa —dijo finalmente poniéndose en pie. Apoyó la pistola en la barandilla de metal, pero se aseguró de permanecer a una distancia en que la pudiese coger. No serviría de mucho en una situación difícil, al menos hasta que encontrase otra fuente de alimentación, pero si las cosas se pusieran feas, era todo lo que tenía. 

—¿Nosotros? —dijo el Señor del Tiempo. 

—Dijiste que ibas a llevarme a un lugar seguro —dijo Cinder—. Y te puedo asegurar que Moldox no es seguro. Ya es bastante difícil eludir las patrullas Dalek. Prefiero morir que dejar que me lleven prisionera. 

—¿Prisionera? —dijo el Señor del Tiempo—. Eso no parece propio de los Dalek. No a menos que tengan planes para este planeta. ¿Qué les ha sucedido a las personas que se han llevado? 

Cinder se encogió de hombros. 

—Lo único que sé es que son retenidos en las ciudades. Para eso son las patrullas Dalek, para reunir a la gente. Sólo te exterminan si intentas correr o luchar. 

—¿Están perforando pozos en la tierra? ¿Excavando minas? 

Cinder se encogió de hombros. No tenía ni idea. 

—Creo que será mejor que me lo enseñes —dijo el Señor del Tiempo. 

A Cinder se le cayó el alma a los pies. 

—¿Qué pasa con los Dalek? Se dio cuenta de que los martillazos en la puerta habían cesado. Quizá la Degradación se había dado por vencida y se había escabullido para informar. Sin embargo prefería evitar volver allí fuera para averiguarlo. 

—Cruzaremos ese puente cuando lleguemos a él —dijo—. ¿Cuál es la ciudad más cercana?

—Andor —dijo—. A unos diez kilómetros de aquí. 

—¿Conoces el camino?

Cinder asintió 

—Es peligroso —dijo—. Allí hay miles de ellos. Cuentan historias… sobre los mutantes y las nuevas armas que están desarrollando. 

—Eso es lo que me temo —dijo el Señor del Tiempo. Echó un ultimo vistazo al monitor y después se dirigió a la puerta—. Vamos. No hay mejor momento que el presente. 

—Si hago esto. Si te llevo a Andor y te muestro a los Dalek, ¿me llevarás después lejos de aquí en tu TARDIS, a un lugar seguro? —se le quebró la voz al decir las palabras. Se metió las manos en los bolsillos para que no viese que estaban temblando. 

—Sí —dijo—. Lo haré. Te lo prometo. 

—¿Cómo sé que puedo confiar en ti? 

Sus ojos se encontraron con los de ella, antes de que él se volviese y cruzase la puerta. —No lo sabes—le gritó a su espalda. 

Pensando que no tenía nada que perder, Cinder agarró su arma y corrió tras él.



