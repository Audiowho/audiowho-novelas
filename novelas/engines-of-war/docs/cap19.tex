\chapter*{Capítulo Diecinueve}
\addcontentsline{toc}{chapter}{Capítulo Diecinueve}

La TARDIS se materializó en un risco sobre un desolado paisaje salvaje.
Tras un momento, la puerta se abrió y Cinder emergió, preparándose para la fina mordedura del viento. El pelo le azotaba en su cara, sus ojos le lloraban, y se encontró envolviendo sus brazos alrededor del cuerpo, tratando de retener el calor.
Oyó al Doctor cerrar la puerta de la TARDIS tras ella y se giró para verle estar allí de pie, inspeccionando el páramo a sus pies. Tan lejos como alcanzaba la vista, podían ver campos de hierba pajiza y brezo, interrumpidos por ocasionales grupos de árboles que dominaban el paisaje. El cielo era de un nítido azul pálido atravesado por mechones de nubes.

--¿Creía que habías dicho que íbamos a volver a Gallifrey? --dijo.
--Ah --respondió el Doctor--. Sí, debería explicarlo.

Cinder levantó una ceja expectante, poniendo las manos en las caderas. 

--¿Y bien?.
--Esto es Gallifrey --dijo--. Al menos, en cierto modo. Es un pequeño bolsillo de naturaleza gallifreyana, acordonada en una burbuja temporal. Los Señores del Tiempo lo conocen cariñosamente como la Zona Muerta --sonrió--. Un lugar bastante poco hospitalario --dijo.
--Maravilloso --dijo Cinder--. La Zona Muerta --pisoteó el suelo, sintiéndose expuesta en la ladera-- ¿Recuérdame de nuevo por qué estamos aquí?.
--La Zona Muerta solía ser el lugar en que desafortunados participantes eran invitados a jugar Los Juegos de Rassilon, enfrentados en una batalla a vida o muerte contra una variedad de especies alien, sacados forzosamente de sus hábitats naturales --dijo el Doctor ignorando la pregunta.
--Y aquí estoy yo, pensando que los Señores del Tiempo eran buenos --dijo Cinder sarcásticamente.
--Fue hace mucho tiempo, en la primera Edad de Rassilon. Construyó su tumba aquí --el Doctor se giró, señalando una aguja negra en la distancia, juzgando por la tierra a los pies de la montaña--. Ahí --dijo-- La Torre Oscura.
--¿Su tumba? --dijo Cinder-- Pero no está muerto. Lo conocí. Aunque me gustaría no haberlo hecho.
--Es complicado --dijo el Doctor--. Esencialmente Rassilon es inmortal. En la antigüedad abandonó su forma corpórea, y durante milenios residió en su tumba, adorado como antiguo y futuro rey. Sin embargo fue resucitado en los primeros días de la Guerra, cuando los Señores del Tiempo se dieron cuenta de que necesitaban un tipo diferente de líder.
--Puedo ver que le ha hecho bien al mundo --dijo Cinder.
--Cierto --respondió el Doctor. Miró apesadumbrado.
--Aún no has respondido mi pregunta --dijo.
--¿Qué pregunta?.
--¿Por qué estamos aquí?.
--Estamos aquí para cometer una fuga --dijo el Doctor--. Rassilon tiene... a alguien atrapado en la Torre. Alguien cuya ayuda necesitamos. Su nombre es Borusa. Vamos a ayudarle a escapar.
--¿No podías haber aparcado un poco más cerca? --dijo Cinder, soplando en sus manos ahuecadas.

El Doctor negó con la cabeza. 

--Ese es el problema con la Zona Muerta. A menos que tengas acceso a un dispositivo transmat en las cámaras del Alto Consejo, tienes que hacer una caminata a través de la tierra salvaje para llegar cerca de la torre. Es un vestigio de los juegos, pero también sirve como una protección excelente para Rassilon y lo que sea que quiera levantar en su vieja tumba.

Cinder asintió. 

--Bueno, veamos si podemos bajar de aquí --dijo, caminando hacia el otro borde de la ladera y mirando hacia abajo. Cuando vio lo que había en los pies de la ladera, emitió un sonido que era algo entre un chillido y un grito aterrorizado.
--¿Qué pasa? --llamó el Doctor corriendo hacia ella.
--Eso --dijo, señalando a la cosa en la base de la colina. Una enorme criatura semejante a un lagarto estaba recostada sobre el brezo, masticando felizmente la flora local. Era al menos de veinte metros de largo, con cuatro piernas achaparradas de elefante y un gran y serpenteante cuello. Su piel era verde y su espalda estaba adornada con gruesas escamas de quitino como armadura. La cabeza era pequeña, con pequeños y brillantes ojos negros y una mandibula llena de serrados y afilados dientes. Su cola se agitaba indiferente de lado a lado, arrancando la maleza--. ¿Qué es?.
--Una bestia primitiva --dijo el Doctor-- del sombrío y distante pasado de Gallifrey. Los Señores del Tiempo cesaron de sacar incautos alien hace muchos años, pero claramente no les había impedido intervenir con su propio pasado. Esta es una criatura fuera de su tiempo, una especie hace tiempo extinguida antes de que los Señores del Tiempo caminaran por el planeta.

Cinder miró al lagarto gigante, que aún estaba mascando seriamente las frondosas hojas de un árbol caído. 

--Mientras que no tenga ningún hermano mayor carnívoro --dijo.
--Ah... --dijo de nuevo el Doctor.

Cinder se dio la vuelta. 

--¿En serio? --dijo.
--No debes de preocuparte excesivamente --respondió--. Son lo suficientemente grandes y los veremos con tiempo de sobra. Son las hormigas carnívoras de las que tienes que preocuparte. Pueden darte un buen mordisco.

Cinder se miró las botas, acurrucándose en el brezo, e inmediatamente empezó a rascarse las piernas. El Doctor se rió, y ella se giró y le pegó enfadada en el brazo. 

--Eres incorregible --dijo, permitiendo que una sonrisa apareciera. Después de todo, un poco de frivolidad era precisamente lo que necesitaba.
--Entonces, ¿por qué es tan especial el tal Borusa?. ¿Cómo va a ser capaz de ayudarnos contra los Dalek?.

El Doctor parecía avergonzado. 

--Borusa puede ver el futuro --dijo-- y el pasado. La red de todos los tiempos. Toda y cada una de las posibilidades, como se interrelacionan, como cada decisión causa una fractura en el universo, abriendo nuevos camino a través del tiempo. El Vórtice Temporal fluye por su mente. Rasilon utiliza a Borusa para navegar por las lineas temporales, para discernir el curso de acción más efectivo para sus ataques contra los Dalek.
--¿Nació así? --dijo Cinder.

Fueron caminando dificultosamente por un campo vacío, sus botas enredándose en la larga hierba con capa paso. Una ocasional ráfaga de viento amenazó con hacerles volcar, pero hasta ahora no había ninguna señal de bestias carnívoras y, afortunadamente, nada de hormigas.

--No --dijo el Doctor--.Él no era así. Su línea temporal fue retro-manipulada. Fue forjado por Rassilon en un intento de encontrar la solución al fin de la Guerra.

Cinder llegó a entender las implicaciones de esto, así que siguió su propio consejo. ¿Cómo retro-manipulas a una persona?. El tiempo, supuso, lo dirá. 

--Así que, este amigo tuyo, ¿va a decirnos como derrotar a los Dalek?.
--En cierto modo --dijo el Doctor--. Vamos a llevarlo al Ojo de Tantalus. Nos va a ayudar a cambiar el futuro.
--Bien --dijo Cinder--, por supuesto. Al Ojo. Eso tiene mucho sentido --miró a la Torre, amenazante en la distancia.

Desde algún lugar cercano hubo un sonido como una sirena, el trompeteo de una enorme bestia. Los ojos de Cinder se ensancharon aterrorizados.

--No te preocupes --dijo el Doctor--.Es solo otro de los amigables, un herbívoro como el que vimos antes, llamando a los suyos.

Cinder suspiró aliviada. 

--¿Por qué hace eso? --dijo Cinder--, ¿una llamada de apareamiento?.

El Doctor negó con la cabeza. 

--No --dijo encogiéndose de hombros--. Más bien una señal de adver...--frenó en seco, mirando a Cinder, con comprensión en sus ojos-- Ah. Supongo que esta bastante cerca, ¿cierto?.

Hubo un rugido atronador que pareció sacudir la tierra bajo sus pies. Cinder sintió sus piernas temblar. Los pelos de la nuca se le erizaron.
Tras ellos vino un pisotón, seguido por otro, y luego otro, incrementando el ritmo. Cinder tragó saliva, y se quedó en el sitio. La expresión en el rostro del Doctor era de sorprendida sorpresa.
Lentamente, giró la cabeza.
La criatura era como algo nacido de la pesadilla de un niño. Era titánica, dos veces el tamaño de un hab-block de Moldox, e incluso más desagradable de observar. Se puso de pie en dos fornidas patas traseras, con una pequeña, gorda cola arrastrándose por el suelo para mantener el equilibrio. Donde podría haber esperado que sobresalieran dos piernas frontales, a cada lado de su cuerpo la bestia tenía alas, cortas, no voladoras cubiertas en aterciopeladas plumas púrpuras y blancas. Su enorme cabeza era en su mayoría dientes, contenidas en un enorme desfiladero de boca. Por encima de esto, una fila de cuatro pequeños y brillantes ojos parpadearon en una rápida sucesión. Ahora, los cuatro estaban fijados en Cinder y el Doctor mientras cargaba por el campo hacia ellos.

--Creía que habías dicho que los veríamos venir con tiempo de sobra --dijo Cinder.

El Doctor se encogió de hombros. 

--Trataba de hacerte sentir mejor --dijo. La cogió del brazo--. ¿Ves esos acantilados de allí?.

Cinder asintió.

--¡Corre!.

Cinder corrió con todas sus fuerzas. Los músculos de los muslos quemaban mientras se sumergía a través de la desigual tierra hacia la pared del acantilado. Estaba zancadas por delante del Doctor, quien, a pesar de su sorprendente agilidad, era incapaz de mantener el ritmo.
La bestia andaba torpemente tras ellos, sus extrañas y no voladoras alas batiendose excitadamente mientras corría. Hilos de baba goteaban de sus terroríficas fauces.
El pie de Cinder quedó atrapado en una raíz o una irregular mata de hierba y cayó, extendiendo sus manos y golpeando el duro y seco suelo. Inmediatamente echo a rodar, negándose a permitir que la detuviera en su ímpetu, y con un pequeño movimiento se puso de pie de un salto y continuó su frenética carrera en busca de cobertura.
Tras ella la bestia rugió, un primitivo, estruendoso grito que le golpeó como un puñetazo en el estomago. Cualquier numero de veces en el pasado que había escuchado el sonido de un disparo o una mina explotando lo describió como ensordecedor, y nunca se había parado a pensar mucho en como realmente podía ser un ruido ensordecedor. El ruido que salió de la bestia, sin embargo, era literalmente eso. Dejó sus oídos pitando, como si alguien hubiera puesto bolas de algodón en ellos, y ahogó el mundo.
Todo parecía hiper-real, onírico. ¿Estaba realmente corriendo por un campo en un planeta lejano perseguida por una criatura carnivora parecida a un dinosaurio?.
La pared del acantilado se le acercaba, negra y en bloque, una pared sobre el mundo. Se dio cuenta con terror de que no tenía ni idea de lo que tenía que hacer cuando llegara allí. Estaba corriendo hacia un callejón sin salida. La criatura les había acorralado contra la pared con ningún sitio al que ir.

--¡Doctor --gimió, sabiendo por completo que sería incapaz de oír su respuesta.

Los pasos de la criatura se estaban acercando, físicamente haciéndola botar con cada atronador paso. Su audición estaba empezando a volver en una serie de tartamudeantes, desorientantes episodios. Comenzó a reducir la velocidad.
Alcanzó el pie de la escarpada ladera, mirando de izquierda a derecha. Todo lo que podía ver era una solitaria peña, entre una pila de rocas sueltas. ¿Pretendía el Doctor que se escondieran tras eso?.
De repente sintió como tomaba su muñeca y, sorprendida, le permitió tirar de ella, corriendo paralelamente a la pared. La criatura trató de disminuir la velocidad e, incapaz de guardar su ímpetu, patinó en la pared de roca.
Cinder miró por encima del hombro mientras corría y vio que no había sido disuadida, sin embargo, a medida que continuaba tambaleándose tras ellos, balanceando su cabeza de lado a lado.

--¡Aqui! --dijo el Doctor, señalando una estrecha fisura en la roca-- ¡Entra!.
--¡Ahi!-- gritó Cinder.
--¡Sencillamente hazlo! --vociferó el Doctor, liberando su agarre del brazo y empujándola en la dirección de la gran grieta. Corrió hacia ella, girándose para que pudiera contorsionarse, esperando que se ampliara una vez que estuviera dentro para crear un espacio en que pudieran resguardarse hasta que el monstruo se fuera.
Las rocas eran dentadas y arañando su espalda y manos forzó su entrada. A través de la estrecha abertura podía ver al Doctor retrocediendo hacia ella, agitando su destornillador sónico a la criatura. No parecía estar teniendo ningún efecto. 

--¡Doctor! --gritó.
--Enseguida estoy ahí --gritó sobre su hombro.

Desapareció de vista durante un momento, y entonces estaba en la entrada, forzando su entrada tras ella.
La criatura rugió de nuevo, dando cabezazos contra la pared de la ladera, como si estuviera tratando de destrozarla para atraparlos. Resoplando, bajó su cabeza a la grieta, mirando con sus múltiples ojos. Podía verlos retorcerse en la grieta de la roca y no estaba feliz. Resopló de nuevo, enviando trozos de baba al interior de la fisura, salpicando el pelo y la cara de Cinder. Hizo un sonido de asco mientras trataba de limpiarlo, y solo consiguió embadurnar su mejilla. Les gruñó, y la estela de su aliento era caliente y nauseabundo con el hedor de carne podrida.

--Aquí estaremos a salvo --dijo el Doctor--.Podemos esperar aquí hasta que se vaya --puso una mano tranquilizadora en su hombro. Ella le miró. Estaba al menos a un metro, todavía tratando negociar su entrada a la cueva, con sus hombros jadeando a cada respiración. ¿Así que, de quién era la mano?.
Cinder se retorció frenéticamente, atrapando su otro hombro en un afilado afloramiento de una roca y chillando en asustada sorpresa por el hombre que estaba tras ella. Un hombre del que no estaba completamente segura de que fuera un hombre.
Parecía estar vestido con una túnica de lino, y era delgado y pálido. Sin embargo, su cara, era casi imposible centrarse en ella. Las características parpadeaban y cambiaban cuando la miraba, como si una cara se estuviera mutando a otra, y luego a otra, y otra, encerrada en un constante ciclo de cambio. Parecía brillar con una suave luz ámbar, como si estuviera impregnada con una rara titilante energía, similar a la forma en que la cara de Karlax había empezado a brillar, justo antes de que el Doctor le explicara que el Señor del Tiempo estaba a punto de regenerarse.
Sin embargo, esto era diferente. Era como si el hombre no estuviera allí realmente, como si sus rasgos fueran fantasmales e incorpóreos, a pesar de la cara podía sentir el peso de su mano en el hombro. No podía leer ninguna expresión en su cara. Todo era demasiado fluido. No tenía ni idea de si pretendí hacerle daño o no.

--No pasa nada --dijo el Doctor tras ella--. No te hará daño.

El recién llegado retiró la mano y se movió lentamente por el camino por el que había venido, hacia el interior de la cueva. Ahora que tenía su atención, agito su mano silenciosamente para que le siguieran.

--¿Qué es? --susurró Cinder.
--Es un Señor del Tiempo --dijo el Doctor, con su voz cargada de tristeza.
--¿Un Señor del Tiempo?.
--Sí. ¿Recuerdas lo que te conté sobre Borusa?.

Cinder asintió. 

--¿Es ese?.
--No, ese no es él, pero eso es lo que es ahora. Esta pobre alma debe de haber sido uno de los anteriores experimentos de Rassilon, desterrado a la Zona Muerta cuando el experimento falló.
--¿Qué experimento?. No eres coherente --dijo Cinder.
--Rassilon, al igual que los Dalek, ha estado jugando con la evolución de los Señores del Tiempo, cogiendo sujetos y retro-evolucionando su linea temporal personal, alterando su composición para que evolucionen en otra cosa. En eso --el Doctor le hizo una señal con la mano para dirigirse al túnel, instándola a seguir al extraño medio hombre.
--¿Qué le pasa en la cara? --dijo.
--Esas son todas sus encarnaciones pasadas y futuras --dijo el Doctor--.Las diferentes caras que hubiera llevado. Están atrapadas en un ciclo de flujo constante. No es ni una ni otra de la gente que podría haber sido. Está encerrado en un estado constante de metacrisis entre caras, su mente expuesta a la pura energía del Vórtice Temporal.

Cinder no sabía qué decir. Sonaba terrible, y todavía no estaba segura de si podía confiar en el extraño Señor del Tiempo mutante, ofreciéndose a llevarlos al interior de su cueva. Sin embargo sus opciones parecían estar entre eso y el enfadado mastodonte que aún estaba paseándose fuera, esperando comérsela. Supuso que no había otra opción. Fue tras él, el Doctor le seguía.
A los pocos metros la cavidad se ensanchó, y el asqueante sentido de claustrofobia que había estado amenazando con superar a Cinder empezó a desaparecer. Fue capaz de girarse y caminar normalmente. A lo lejos, guiado por el suave brillo de la carne del medio-hombre guiado por la luz suave de la carne del medio-hombre pudo ver que la grieta se abria a un completo sistema de cuevas, conectado por una red de túneles. Era como un laberinto, tallado profundamente en la parte inferior de la montaña.
A medida que se abrieron paso hacia el interior de la montaña, Cinder captó más en la ladera de la montaña, Cinder captó el aroma espeso del humo de la madera a la deriva a través de los túneles, y arrugó la nariz. Se dio cuenta de que probablemente esto significaba que el medio-hombre les estaba llevando a su campamento, escondido en una de las cavernas cercanas. Con suerte eso significaría que habría una oportunidad de que se tomaran un pequeño descanso, los músculos de los muslos la estaban matando, aunque la idea de estar atrapada en un espacio confinado con ese extraño individuo no le atraía especialmente, a pesar de las garantías del Doctor.
Supuso que debería de haber sido duro vivir allí. Desterrado a la tierra salvaje de la Zona Muerta, probablemente encontró este sitio mientras buscaba refugio de los monstruos, tal y como el Doctor y ella habían hecho, y estableció su residencia. Cinder consideró que probablemente ella hubiera hecho lo mismo, si se le hubiera presentado la decisión entre un frío, húmedo y oscuro sistema de cuevas y las babeantes mandíbulas de un hambriento dinosaurio.
El olor del fuego iba en aumento, guiándolos como un faro. El pasaje a través de las cuevas se bifurcó y se retorció, y se dio cuenta de que debía de haberse adentrado muy al interior de la montaña. Sin embargo no parecían estar viajando hacia abajo, así que asumió que tenían que estar pasando a través, y que quizás, en algún momento aclararía que había una salida en el lado opuesto. Ciertamente esperaba que así fuera, no le gustaban sus opciones de retraer todos esos pasos sin perderse. Y, por supuesto, aún había que considerar al monstruo.
Caminaron arduamente en silencio durante un rato, hasta que, finalmente, el crujiente sonido del fuego se hizo audible sobre el eco de los pasos. El medio-hombre tomó una curva, desapareciendo súbitamente de vista, y Cinder dudó, deteniéndose en el pasillo, insegura de si continuar o no. El Doctor puso una mano sobre su hombro y silenciosamente le instó a continuar.
Alrededor del túnel encontró una gran e irregular cueva. Otros dos pasajes salían a las sombras, y un fuego ardía en un hoyo superficial en el suelo. El medio-hombre estaba de pie frente a ellos, como si les diera la bienvenida a su casa, y dos personas más se sentaron en rocas junto al fuego, calentando sus manos. Uno era una mujer, el otro un hombre más joven, y ambos compartían el mismo estado, de aflicción, como el primer hombre. Sus rostros cambiaban constantemente, patrones titilantes, cambiando de forma desconcertante a través de sus encarnaciones perdidas.
El primer hombre, el que les había conducido allí, hizo señas para que entraran en la cueva y tomaran asiento junto al fuego.

--Vamos --dijo el Doctor. 

Cuando dudó de nuevo, él se abrió camino empujándola suavemente, sonriendo a las dos nuevas figuras, y rodeó el fuego, buscando otra piedra. Se sentó, estirando sus músculos del cuello y de los hombros.
Decidió que no tenía nada que perder, Cinder se unió, tomando asiento tras él, mientras que el medio-hombre se ocupó echando más leños al fuego.

--¿Ves la forma en que el humo se riza? --preguntó el Doctor.

Miró por un momento como el espeso, aceitoso humo se arremolinaba, como si se arremolinara por una brisa. 

--Debe de haber otra salida --dijo, aliviada. Claramente habían pasado bajo la montaña y estaban cerca de una ruta alternativa de salida. Más cerca de la Torre.
--Exactamente --dijo el Doctor--. Pero vamos a sentarnos un rato y recuperemos el aliento.

Cinder observó a los curiosos, mudos Señores del Tiempo. No podía decir si la estaban mirando o no. 

--Sí--dijo--. Pero no durante mucho tiempo.

El Doctor asintió.
Cinder miró la cueva. Había poco para definirlo como un hogar. Algunas esterillas de paja y tarros de arcilla. No había señales de ningún alimento. Se preguntó si esta gente siquiera necesitaba comer, atrapados como estaban en un raro ciclo de vida y muerte.
Miró con curiosidad a la pared. Parecía haber machas de color en la roca desnuda, pero era difícil de discernir en la tenue luz. ¿Era liquen o moho?.

--¿Qué es eso? --dijo, dando un codazo al Doctor y señalando a la pared.

El Doctor siguió su mirada. 

--Oh... son pinturas --dijo, su ojos de repente se iluminaron. Se levantó, se encaminó hacia la pared, alcanzando un palo del fuego. Eligió uno y lo sacó, elevándolo ante él como una antorcha. Cinder, poniéndose de pie, esquivó la trayectoria de su punta llameante mientras se giraba. Se acercó a la pared, sosteniendo la antorcha en alto--. Son marivillosos --dijo--. Completamente maravillosos.

Ella se le unió, consciente de las tres medias-personas, que permanecían sentadas impasivas alrededor del fuego.
Las pinturas eran primitivas y claramente enlucidas con dedos embadurnados en pigmentos vibrantes. Cubrían una pared entera de la cueva, se desparramaban en otra, una serie de pequeñas escenas, aparentemente independientes, cada una representando una historia diferente. A Cinder le parecían pictogramas dentro de una tumba antigua, hablándole a través de incalculables siglos. No había indicios de lo viejas que realmente eran.
Se quedó tras el Doctor mientras caminaba de un lado a otro, sosteniendo la antorcha en alto. En el cálido y naranja brillo, los dibujos parecían tomar vida propia, moviéndose bajo las parpadeantes sombras. Apenas sabía dónde mirar. Había tantos. Siguió uno con el dedo, tratando de interpretar lo que estaba viendo.
En él, una figura que claramente intentaba ser el Doctor, con una mata de pelo gris y una chaqueta de cuero oscura, estaba junto a una mujer rubia vestida con harapos. Una alta flor roja estaba entre ellos.
En otro, cinco Dalek formaban un amplio círculo alrededor de un sexto Dalek, más grande.
Un tercero mostraba un enorme ojo y una caja azul que, Cinder se dio cuenta, tenía la intención de representar la TARDIS en vuelo alrededor del Ojo de Tantalus.
También había otros: una delgada figura con pelo largo y rizado, un hombre larguirucho en un traje azul, un tercero con abombado pelo blanco y una capa siendo perseguido por un robot plateado, una mujer pelirroja yaciendo en el suelo junto a lo que parecía ser la consola de la TARDIS,...
Cinder tragó saliva. No quería ni imaginarse lo que ese podría significar. 

--¿Qué son?. ¿Quién es esta gente? --dijo.
--Son yo --respondió el Doctor. Parecía completamente embrujado por las pinturas primitivas, siguiéndolas a través de la pared con la punta de los dedos, moviendo la antorcha de un lado a otro para ver. Se inclinó más cerca, estudiándolas atentamente--. Al menos creo que lo son. No reconozco a todos. Algunas de estas cosas aun no han sucedido. Puede que haya cambiado de cara.

Cinder frunció el ceño. Aún no había sucedido. Entonces, si la mujer en los dibujos se suponía que era ella, eso no era buena señal. Consideró señalárselo, pero decidió que no. Existía el riesgo de que si lo hacía, pudiera de alguna forma hacerlo real. Decidió que era mejor ignorarlo. 

--¿Por qué hay dibujos de ti en la pared de una cueva, aquí, en el medio de la nada?.
--Están pintando lo que ven --dijo el Doctor--. Y por alguna razón, me ven a mi. Pasado, presente y futuro. Esta es mi historia.
--Entonces no deberías leerla --dijo Cinder--. Nadie debería conocer su futuro --sintió un escalofrío recorrer su columna.
--No --coincidió el Doctor--.Tienes razón. 

Sin embargo, no se apartó y continuó allí, estudiando las imágenes.

--¿Doctor?.
--Oh, de acuerdo --dijo, alejándose de mala gana de la pared--. Es que es tan tentador echar una pequeña ojeada a como mis futuros yo podrían ser.
--No estoy segura de cuánto vas a averiguar de los dibujos de una cueva --dijo Cinder-- y supón que ves algo que no deberías, como la forma en que vas a morir. ¿Qué harías entonces?.

Él la miró con recelo. 

--Nada es fijo. No importa lo que vi, si no ha sucedido aún puede ser cambiado.
Suspiró de alivio. Bueno, al menos era bueno saberlo. 

--Deberíamos irnos --dijo--. Necesitaremos volver de la Torre antes de que oscurezca.

El Doctor parecía un poco alicaído, como si prefiriera estar en la cueva estudiando las pinturas. 

--¿Tienes que ser siempre tan sensata? --dijo.
--Me temo que sí --respondió--. En Moldox aprendes a no quedarte quieta. Tienes que seguir en movimiento para permanecer por delante de los Dalek.
--Cuando acabemos con esto --dijo--, voy a enseñarte algunas de las mejores cosas de la vida. Libros, malvaviscos, té Earl Grey, las vistas desde la orilla del Rin, los océanos de ceniza de Astragard, el placer de la corte de Cleopatra.
--Te tomo la palabra--dijo con una sonrisa. Echó un vistazo a la pintura de la mujer pelirroja, y luego se aclaró la garganta--. Razón de más para seguir encontrando a este Borusa. ¡Vamos! --se dirigió al pasadizo de la izquierda, y sintió una ligera brisa en su mejilla--. Por aquí.

Una vistazo hacia atrás le dijo que el Doctor y las tres medio-personas la estaban siguiendo.
La Torre permaneció oscura y presagiante ante ellos. Cinder sintió un involuntario escalofrío en su columna ante la vista de ella. 

--No parece mucho una tumba --dijo al Doctor, que estaba caminando fatigosamente junto a ella--, se parece más a una fortaleza.
--Mmmm, hmmm --masculló el Doctor, sin comprometerse.

Cerca, las tres medias-personas de la cueva les seguían. Cinder se pateó a si misma por seguir pensando de ellos así. Por supuesto, estas no eran medio-personas. ¿Intersticiales?. Bastaría por ahora.
Deseó saber cómo comunicarse con ellos, si serían capaces de entenderla. No estaba muy segura de cómo actuar a su alrededor.
El Doctor la vio mirando. 

--Quieren ayudar --dijo--. Ya han visto todo esto. Están preparados para la tormenta que se avecina.

Cinder asintió, pero no dijo nada, preguntándose qué más habrían visto. Esa era la verdadera raíz de su inquietud. No sabía si el Doctor había visto la pintura en la pared de la cueva de la chica pelirroja, acostada en el suelo de lo que parecía la sala de la consola de la TARDIS. ¿Era Cinder, o alguna otra chica de cabellos llameantes del futuro o del pasado del Doctor?. La pintura era demasiado primitiva para decirlo, pero algo al respecto la preocupaba. No podía sacarse esa imagen de la cabeza.
Conforme se acercaban a la Torre, podía ver braseros llameantes que flanqueaban la entrada elevada. Había medio esperado que el lugar pareciera abandonado, o estuviera en un estado de ruina o deterioro, pero los braseros mostraron claramente que todavía estaba habitado.
--Espera --dijo el Doctor, señalándole hacia una arboleda cercana. Saltó hacia ellos, notando que los Intersticiales habían parado, permaneciendo donde estaban en el camino a plena vista, ignorando al Doctor. No tenía ni idea de si era estúpida obstinación, o simplemente porque ya eran conscientes de que no había nadie en la Torre.

El Doctor, sin embargo, no quería arriesgarse, y Cinder le vio arrastrarse hacia la entrada. Merodeó por allí un momento, evidentemente escuchando cualquier ruido del interior, y entonces entró, desapareciendo de la vista.
Un momento más tarde regresó, agitando ambos brazos para indicar que estaba despejado.
Cinder y los Intersticiales trotaron para unirse.

--Tenemos que actuar con rapidez --dijo el Doctor mientras les guiaba hacia el débilmente iluminado interior de la Torre--. Rassilon podría volver en cualquier momento, y eso haría las cosas sumamente difíciles.

Cinder consideró esto por un momento. 

--Oh, no sé. Él es uno, nosotros dos... --dijo.

El Doctor negó con la cabeza. 

--¿El guantelete que lleva?. Tiene el mismo efecto que las armas temporales de los Dalek, entre otras cosas. No sería bueno entrar en una disputa con él.

Cinder frunció el ceño. Cuanto más descubría sobre Rassilon, más convencida estaba de que el universo estaría mejor sin él.
Miró a su alrededor, tratando de dar sentido al lugar. En el interior parecía más un mausoleo, un hall cavernoso, con estandartes hechos jirones cayendo del techo, un pedestal de piedra que parecía una fuente, y la propia tumba, que parecía un vasta cama con dosel desprovista de dosel.
Todo el lugar tenía un aura de abandono, un aire deprimente que la hizo querer salir de allí todo lo rápido que pudiera.

--Está por aquí --dijo el Doctor, conduciéndola al corte tramo de escaleras que llevaban a la tumba. Los Intersticiales estaban esperando silenciosamente en la puerta.
--¿Borusa? --dijo el Doctor-- ¿Estás ahí?.

Hubo un zumbido, como de engranajes girando, y la plataforma de metal que descansaba en lo alto de la tumba empezó a girar en trinquetes de radios. Para su horror, Cinder se dio cuenta de que había una persona atada a la estructura. Sus pies y manos estaban atados, y cables parecían brotar de su cavidad torácica y la parte posterior de su cráneo, dejando un rastro hasta la parte más lejana de la tumba.

--Doctor --dijo la cosa-hombre. Al igual que los Intersticiales, su cara estaba en constante flujo, titilando entre ese pálido viejo hombre, a un bronceado joven, a una mujer de media edad y muchas más. Sin embargo, sus ojos, no eran como los de los otros, y titilaban con danzantes luces azules, como si una corriente eléctrica estuviera corriendo de su cabeza y sus ojos fueran pequeñas ventanas, permitiéndole echar un vistazo al interior. Era la cosa más espantosa que nunca había visto.
--Borusa, estoy aquí para ayudarte --dijo el Doctor--. Pero primero, necesito que me ayudes.

Borusa rió, y era una húmeda, convulsa asfixia.

--El factor aleatorio --dijo--. El nudo de posibilidad. Siempre fuiste alguien muy difícil de achantar, Doctor. Lord Rassilon no estará complacido.

El Doctor le ignoró. 

--Borusa, voy a liberarte y llevarte conmigo al Ojo. Tengo un plan para derrotar a los Dalek --dudó--. ¿Puedes verlo?. ¿Puedes ver ese futuro desarrollándose?.
--Puedo --dijo Borusa.
--¿Entonces, me ayudarás?.

Hubo una larga pausa, mientras Borusa parecía considerar la petición del Doctor. Cinder se preguntó si estaba, de hecho, tratando de mirar hacia delante, de ver lo que podría ser de ellos si lo hacía.

--Te ayudaré --dijo tras un momento--, pero hay una condición.
--Nombrala --dijo el Doctor.
--Más tarde, cuando esté hecho, terminarás con mi sufrimiento. Me liberarás.

El Doctor inclinó la cabeza, claramente dolido por la idea.

--El Vórtice Temporal --dijo Borusa--. Se desenmaraña en mi cabeza. Es hermoso, exquisito. Veo el mapa de toda la eternidad, cada momento, cada delicada decisión que es tomada, y como altera el curso del futuro, cambia las posibilidades. Pero duele, Doctor. Es más de lo que puedo soportar. Ningún ser vivo debería de tener esta carga.

Cinder miró al Doctor. Estaba cabizbajo.

--¿Me ayudarás? --dijo Borusa-- ¿Me liberarás del motor de posibilidad?.
--Lo haré --dijo el Doctor voz quebrada.
--Entonces tenemos un acuerdo --dijo Borusa--. Libérame. Corta mis ataduras de la Matriz. Me uniré a tu TARDIS para un último viaje.
--Ayúdame --dijo el Doctor a Cinder mientras subía los escalones--. Necesitamos liberarlo.

El Doctor se precipitó a la alzada plataforma y se acercó a la estructura de metal a la que Borusa estaba atado, hasta que fue capaz de agacharse torpemente tras la cabeza de Borusa. Se puso a trabajar, descorchando cables de enchufes tras la calavera de Borusa. Un fluido blanco brotó de los puertos abiertos. Cinder decidió no mirar.

--¿Qué necesitas que haga? --dijo-- ¿Lo vamos a cortar de la estructura de metal? --sacó las anudadas cuerdas que ataban el pie izquierdo de Borusa. Podía ver donde habían mordido en su carne, con llagas debido a la fricción alrededor de su tobillo.
--¡No! --dijo el Doctor, inmediatamente, sacando la cabeza tras la estructura. Cuando vio que aún no había empezado, se relajó--. No, no hagas eso. Dudo mucho que pueda soportarlo tras todo este tiempo atado. Solo necesitamos liberar la estructura de su lugar.

Cinder asintió. Se dejó caer de cuclillas, mirando bajo la estructura. Era un mecanismo primitivo,realmente, como si hubiera sido construido en el ultimo minuto como una ocurrencia tardía, una vez que el verdadero trabajo, la retro-ingeniería, como la había llamado el Doctor, se había realizado.
Borusa gimió cuando el Doctor sacó el último de los cables de la parte posterior de su cráneo, cerrando la conexión a la Matriz, fuera lo que fuera. 

--Por aquí --llamó al Doctor--. He encontrado los tornillos correctos, pero vas a necesitar tu sonicky-do-dah.
--¿Mi qué?-
--Oh, ya sabes a lo que me refiero.

El Doctor se acercó a ella. 

--Allí abajo --dijo, indicándole el primer tornillo.
--Está bien --dijo el Doctor--.Asegura la estructura mientras lo miro.

Cinder se quedó de pie, considerando cómo hacer lo que el Doctor le había pedido, solo para encontrar que los tres Intersticiales se le habían unido y cada uno estaba sosteniendo una esquina de la estructura, sujetándola mientras el Doctor presionaba el botón de su sónico y los tornillo empezaban a aflojarse y a liberarse.
Momentos después, el motor de posibilidad era liberado de su carcasa, y los tres Intersticiales estaban bajándolo cuidadosamente, sosteniéndolo sobre su cabezas. Desde su punto de vista en lo alto de la tumba, Cinder pudo ver su cara. Sus ojos eléctricos parecían mirar directamente en su alma.

--Bien --dijo el Doctor--. Volvamos a la TARDIS antes de que Rassilon se de cuenta de que su juguete favorito ha sido incautado --saltó de la tumba, metiendo su destornillador sónico en su cinturón de munición. Ofreció a Cinder su mano, y la tomó, saltando junto a él.
--Pregúntale si nos vamos a encontrar con otro de esos dinosaurios --dijo, señalando con la cabeza a Borusa--. Porque si lo vamos a hacer, me quedo aquí, con Rassilon o sin él.

El Doctor se rió, dirigiéndose a la puerta.

--No estoy bromeando --dijo tras él--. ¿Doctor?. ¡Doctor!.

Los tres Intersticiales cargaron con el motor de posibilidad sobre sus hombros como antiguos hombres, llevando a su rey en una litera. Borusa permaneció inmóvil sobre la estructura de hierro mientras marchaban por la tierra salvaje hacia la TARDIS. Cinder y el Doctor los seguían por detrás en silenciosa contemplación.
Durante casi una hora caminaron por campos barridos por el viento, y mientras tanto Cinder nerviosamente continuó en guardia en busca de cualquier señal de la bestia que anteriormente había ido tras ellos, o cualquiera de su clase infernal.
Cuando se acercaron al risco, siguiendo a los Intersticiales, que parecían instintivamente saber la ruta a tomar, Cinder vio a la TARDIS posada en la ladera. La luz del sol estaba empezando a desvanecerse, y cuando miró hacia delante, vio que marcas de pequeñas luces habían sido puestas en el camino que llevaba de la base de la colina a la TARDIS.

--¿Qué es eso? --dijo.

El Doctor sonrió. 

--Ya verás --respondió, riendo.

A medida que se acercaban y la vista se fijaba adecuadamente, se quedó sin aliento ante la vista de cincuenta o más figuras, todas alineadas como una extraña procesión, mostrándoles el camino a la TARDIS. Todo eran Intersticiales, cada uno de ellos. Contó docenas de ellos, filas de profundidad, antes de que se rindiera, dándose cuenta de que era una tarea inútil.
Las luces que había visto desde la distancia era el ligero brillo de su carne, brillando en el crepúsculo.

--Están aquí para desearnos lo mejor --dijo el Doctor--. Para guiarnos en nuestro camino.
--Hay tantos de ellos --dijo Cinder--. Un pensamiento le vino a la cabeza--. ¿Cómo es que no están atrayendo a las bestias carnívoras?.
--Pueden ver el futuro, Cinder --dijo el Doctor.
--Ah, sí. Supongo que eso sería de ayuda --respondió.

Habían llegado al pie del peñasco, y los tres Intersticiales frente a ellos empezaron lentamente a subir a la cima, elevando el motor de posibilidad por encima de sus cabezas para que los otros pudieran ver. Al pasar, los miembros de la multitud inclinaron la cabeza en señal de reverencia.
Sintiéndose un poco vergonzosa, Cinder les siguió, no segura de cómo actuar, y si se suponía que tenía que hacer o decir algo. En lugar de cualquier futura orientación, simplemente copió al Doctor, que mantenía sus ojos bajos y lentamente seguía los tres tres portadores de la litera.
Cuando llegaron a la TARDIS los tres Intersticiales se hicieron a un lado mientras el Doctor abría la puerta. Entonces, con una rápida mirada sobre su hombro a Cinder, los guió a todos al interior.
Cinder observó como desparecían por la pequeña puerta, y entonces, contenta de alejarse de la Zona Muerta tanto como le fuera posible, se apresuró tras ellos.

