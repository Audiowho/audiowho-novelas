\chapter*{Capítulo Trece}
\addcontentsline{toc}{chapter}{Capítulo Trece}

Cinder había estado caminando de un lado a otro por la Sala de Observación durante al menos media hora y con cada paso su frustración iba en aumento. Quería hacer algo. El Doctor le había dicho que permaneciese aquí y no se metiese en problemas, pero simplemente ese no era el estilo de Cinder. Nunca se había sentido cómoda permaneciendo quieta por mucho tiempo y dudaba de que alguna vez pasase. Con un gruñido de irritación se dirigió a la ventana y miró hacia la ciudad. La noche había caído y con ella aparecieron las estrellas. 
Las estrellas. Había leído sobre ellas, por supuesto, comprendía exactamente lo que eran, pero nunca las había visto. No hasta ahora. Durante todos estos años, creciendo en Moldox, el cielo de la noche siempre había estado contaminado por fluctuantes auroras que, para ella, se describían como coloridos sueños a la deriva en el éter mientras la gente dormía. Por supuesto ahora sabía que estaban causadas por la fuga de radiación temporal desde el Ojo de Tantalus, la misma radiación que estaban empleando los Dalek para dar energía a sus armas. 
De alguna manera eso las hacía parecer menos bellas. Esto, en sí mismo, no era nada nuevo. Todo lo que había amado alguna vez, los Dalek se lo habían arrebatado. Todo lo bello, lo habían estropeado. Eso era lo que los Dalek hacían. Conquistaban. Saqueaban. Y ahora incluso estaban contaminando el cielo. 
Sin embargo, Cinder había acabado con todo eso. Había encontrado una manera de salir de Moldox con el Doctor y no permitiría ser menospreciada nunca más ni por los Dalek ni por nadie más. 
Levantó la vista hacia las titilantes constelaciones. Las estrellas eran como puntos de luz, agujeros en el tejido del cielo, a través de los que se podía observar el brillo de universos distantes. Nunca había imaginado que habría tantos de ellos. Comprendía, intelectualmente, que el Universo estaba poblado por incontables billones de estrellas, pero viéndolas brillar sobre su cabeza realmente era algo más. Era asombrosamente bello. 
Se preguntó cuántas personas estarían ahí fuera, como ella, mirando hacia el cielo y sintiéndose esperanzados. Tal vez conseguiría visitar algunos de esos lugares con el Doctor, una vez que la Guerra hubiera acabado. Le gustaría eso, y se dio cuenta de que él lo necesitaba también. Sería bueno para él alejarse de todo, recordar quién era realmente. Se dio cuenta de que la Guerra le estaba minando. Se estaba encalleciendo, se estaba acostumbrando a ella, pero estaba segura que había mucho más en él que eso, otro hombre enterrado en algún lugar bajo el exterior cascarrabias. 
Suspiró, mirando hacia la puerta. ¿Cuánto tiempo había dicho que estaría fuera?. Seguramente no haría ningún mal si echaba un pequeño vistazo. ¿Cuántos problemas podría realmente causar?. 
Cinder tomó una decisión. Quizá incluso podría descubrir algo útil, algo que ayudase al Doctor a persuadir a los Señores del Tiempo de que no usasen su dispositivo del fin del mundo. Sabía que él podía. Tenía que tener fe en él. La alternativa era inimaginable. 
Cruzó la puerta, medio esperando que estuviese bloqueada. No lo estaba. La abrió lo suficiente para echar una mirada afuera. El pasillo estaba desierto. No quería ir muy lejos. Probablemente podría encontrar el camino de vuelta a la Sala de Guerra o la Sala del Consejo, pero no más lejos para arriesgarse en incurrir en la ira del Gobernador y sus guardias. Atravesó la puerta, cerrándola tras ella, después gritó cuando una mano la agarró firmemente por el hombro. Karlax chasqueo la lengua y ella se retorció tratando de liberarse. Era demasiado fuerte y la había pillado con la guardia baja. 

--Pensabas que podías ir a dar una pequeña vuelta, ¿verdad? --dijo. Sentía su aliento caliente en la parte posterior del cuello--. Bueno, no podemos hacer eso. ¿Qué diría el Gobernador?. ¿Ummm? --se movió detrás de ella, sujetando sus brazos por detrás de su espalda--. De hecho --continuó--, creo que deberíamos ir y averiguarlo, ¿no?.
 
--Déjame marchar, Karlax --dijo--. El Doctor volverá en cualquier momento. 

Karlax se echó a reír. 

--Oh, me temo que eso no me preocupa lo más mínimo, jovencita. Ni un ápice. No está aquí ahora y eso es todo lo que importa. Obtendré todo lo que necesito para cuando te encuentre. 

--¿Qué quieres decir? --dijo, comenzando a sentir miedo. ¿Qué tenía el odioso hombrecillo en mente?.
 
--Oh, no es nada de lo que preocuparse --susurró, apretando una mano sobre su boca para ahogar cualquier grito--. Sólo una pequeña prueba que necesito hacer. El Gobernador tiene una maquina, sabes, conocida como sonda de mentes… 

Presa del pánico, Cinder dio una patada hacia atrás, golpeando con los talones la espinilla de Karlax. Gritó, pero no cejó en su agarre. En represalia, le retorció el brazo más arriba en la espalda hasta que amenazaba con romperse y la intensidad del dolor la hizo desmayarse. Cuando cayó en sus brazos sin fuerzas y delirante, la arrastró hasta la sala de al lado, donde estaba esperando el Gobernador. 

--¿Está seguro que quiere seguir adelante con esto, Karlax? --dijo el Gobernador. Estaba inclinado sobre ella, sujetándola en una silla dura de metal y ciñendo las correas alrededor del casco que le habían puesto en la cabeza--. Es solo… Ella es solo una humana. Existe el riesgo de que muera. 

--Irrelevante --dijo Karlax--. Siempre y cuando me consiga la información que quiero, no podría importarme menos lo que le pase. De hecho, le podría enseñar al Doctor una valiosa lección el que ella muriese. 

Ante esto Cinder luchó violentamente contra las ataduras sacudiéndose en la silla, pero el Gobernador había hecho bien su trabajo y no había manera de liberarse. Ni siquiera podía gritar pidiendo ayuda, ya que la habían amordazado tan pronto como Karlax la había metido a empujones en la sala. 
Había conseguido arañar la cara de Karlax con las uñas durante la forcejeante extracción de sangre que siguió, pero había sido sólo una pequeña victoria, un momento fugaz de satisfacción antes de que realmente arraigase el horror de su situación. Estaba atrapada en la sala con esos dos hombres y su máquina, y nadie sabía ni siquiera que estaba allí. Le gustase o no, iban a utilizar en ella su sonda de mentes. 
Cinder se encontró preguntándose con qué frecuencia tenían la ocasión de usarla. A juzgar por el aspecto de la cara de anticipación de Karlax, no se sorprendió saber que la usaba en cualquier oportunidad que tenía. Claramente, entre sus muchas virtudes tenía una vena sádica bien desarrollada. 
Estaba atada a una silla de respaldo alto frente a un montón de monitores de cristal. En estos momentos solo mostraban nieve estática, ruido blanco, pero supuso que ahí sería donde cualquier recuerdo que consiguieran extraer de su mente se reproduciría para que los demás lo viesen . 
Podía ver su propio reflejo en el cristal pulido. Se veía empequeñecida por la silla y los cables que se elevaban desde el casco hasta el techo podrían haber sido mechones de pelo fibroso, erizado como si estuviese cargado de electricidad estática. 
Le recordaban a las cámaras de incubación de cristal que había visto en la nave Dalek, y solo deseaba tener la misma oportunidad ahora de sabotear la máquina antes de que tuvieran la oportunidad de activarla. 

--Acabemos de una vez --dijo Karlax. Estaba mirando hacia la puerta, claramente preocupado de que el Doctor pudiese irrumpir dentro en cualquier momento e interrumpir el proceso. 

--Trabajo lo más rápido que puedo --respondió el Gobernador--. Si no consigo los niveles correctos, le freiremos el cerebro antes de que consiga sacar algo de su cabeza. 

Karlax caminaba de un lado a otro con las manos en la espalda. Parecía autoritario, lleno de presunción, y Cinder sonrío al lugar de las tres inflamadas marcas en la mejilla izquierda. Con un poco de suerte, sería capaz de ofrecerle otro conjunto de ellas que hiciesen juego en la otra mejilla cuando esto terminase. El Gobernador dio un paso atrás. 

--Estoy listo --dijo. 

Karlax dejó de pasearse y se colocó detrás de la silla, fuera de la vista. Por primera vez el Gobernador, de pie junto a la silla, miró hacia abajo y se encontró con su mirada. 

--Lo siento --dijo--. Esto va a dolerte. 

Pulsó el interruptor. Al principio no pasó nada. Oyó un leve zumbido que venía de detrás de su oreja izquierda, y todo lo que podía sentir era un cálido cosquilleo en la parte frontal del cráneo. Era incómodo, pero no doloroso. Echó un vistazo a los monitores, pero seguían sin mostrar nada excepto danzante estática.
Se concentró en el zumbido, ya que parecía crecer en intensidad. Con él la presión dentro de su cabeza comenzó a aumentar. El dolor estalló y mordió la mordaza. Aun así el calor y la presión continuaron creciendo hasta que ella estuvo segura de que en cualquier momento haría que su cráneo se rompiese. 
Se balanceó hacia delante en la silla, la visión se le nublaba. El dolor era como una luz blanca, abrasadora y brillante y no había forma de apagarla o escapar de ella. Trató de gritar, pero se ahogó con la mordaza de la boca. 
Los recuerdos llegaron en un repentino torbellino, en cascada a través de su mente como una serie de imágenes vacilantes. Curiosamente, estaban desprovistas de color, como las antiguas fotografías en blanco y negro siendo ordenadas por el ojo de su mente. Se registraban sin orden alguno: un trozo de aquí, un trozo de allí, fragmentos de su infancia, de su época con los rebeldes en Moldox, de su reciente encuentro con el Doctor. 
Se forzó a mantener los ojos abiertos para ver esas escenas desarrollándose en los monitores, la historia de su vida siendo reproducida en una extraña secuencia en bucle. 
Vio caras, gente hablándole, y aunque no podía oír nada, los gustos y olores eran frescos como si estuviera experimentándolos de nuevo por primera vez. 
Vio a su hermano, brincando como un mono, haciéndole muecas. Vio a su madre servir la cena en su finca, su padre leerle un cuento antes de dormir. Y después les vio morir a todos de nuevo, exterminados por los monstruos de metal que parecían surgir de la nada, cayendo desde el cielo en unos platillos brillantes, encendiendo hogueras con sus chillonas armas. 
Habían entrado a través de la pared de la cocina, cinco de ellos, hablando ásperamente con sus empalagosas voces metálicas, todos de oro y bronce y ladrando órdenes. No había entendido ni una palabra, pero cuando empezaron a disparar y su padre se desplomó en el suelo del salón, el vapor subiendo de su cuerpo sin vida, había entendido lo suficiente como para correr y esconderse. 
Momentos antes de que hubiesen llegado los Dalek, la madre de Cinder había estado vaciando el cubo de la basura de la cocina y, en el caos, Cinder se coló en el porche, rápidamente lo volcó y se agachó dentro. Se encogió allí dentro mientras los Dalek arrasaban su finca hasta los cimientos. No había vuelto a ver a su familia de nuevo, ni siquiera sus cuerpos. 
Los recuerdos siguieron creciendo de forma espontánea en su conciencia. Ahora venían con una claridad sorprendente y un dolor insoportable: 

--Coyne enseñándole cómo apuntar con un rifle, usado como blanco el armazón calcinado de una Degradación de Skaro que él había destruido ese mismo día en una emboscada. 

--Aprendiendo a abrir una cerradura con Ash, un niño de 12 años con el pelo rubio rojizo que había sido asesinado esa noche durante una redada Dalek. 

--Tumbada encima de un edificio durante una tormenta esperando que pasase una patrulla Dalek por debajo, así podría hacer estallar la mina que había enterrado en la calle esa mañana. 

--Su primer beso con otra chica del campamento rebelde, Stephanie la de pelo moreno, que le había enseñado cosas que nunca podría haber imaginado. 

--Y Finch, de quien se había olvidado de alguna manera. Finch, su compañero de fechorías, su amigo. Finch que había muerto durante la emboscada en la que había venido al Doctor cayendo desde el cielo, que había sido borrado de la existencia por el arma temporal del nuevo Dalek… 

Cinder sintió las lágrimas resbalando de sus ojos, corriendo por sus mejillas, pero no eran lágrimas de dolor. Eran lágrimas de tristeza. 
Imágenes de la base Dalek parpadearon por su mente: de ir corriendo por los pasillos detrás del Doctor, de explosiones de Dalek y criaderos obscenos. Del laboratorio donde los Dalek estaban diseccionando a las Degradaciones y de su vuelo a través de las ruinas, todo el camino de vuelta a la TARDIS. 
Cinder no fue consciente de que el Gobernador apagaba la máquina, pero sintió que el fuego de su cabeza comenzaba a sofocarse. El zumbido cesó de repente. Se dejó caer en la silla con nauseas y mareos. Su respiración venía en irregulares e intermitentes jadeos. 
Sintió que alguien le tomaba el pulso en la garganta. 

--Sobrevivirá --dijo el Gobernador. 
--Una lástima --dijo Karlax--. Estaba deseando ver la expresión en la cara del Doctor cuando le contase la noticia. 

El Gobernador le quitó la mordaza de la boca. Abrió la boca en busca de aire. 

--Te matará --dijo entre respiraciones superficiales--. Te matará por esto. 

Karlax se rió. 

--Oh no, no el Doctor --dijo--. El Doctor y yo somos antiguos compañeros de juegos. A él no le gusta ensuciarse las manos. 

Cinder cerró los ojos. El mundo le daba vueltas. No podía arriesgarse a caer inconsciente junto a estos hombres. Si lo hacía, había muchas probabilidades de que nunca volviera a despertar otra vez. 

--Agua --dijo, su voz era un graznido seco. Estaba reseca y tenía un extraño sabor en la lengua como a aluminio. 

--Karlax, dele un poco de agua mientras le quito estas correas, ¿quiere? --dijo el Gobernador--. Consiguió lo que quería. Ha visto la evidencia que respalda las afirmaciones del Doctor y sabe lo que estaba haciendo en Moldox. Es hora de dejar a la chica en paz. 

--Si debo hacerlo --respondió Karlax con veneno y abandonó la sala. 

--Bien --dijo el Gobernador una vez la puerta se cerró detrás de Karlax--. Vamos a sacarte de aquí --empezó a desabrochar correas--. Rápido ahora, ayúdame si tienes fuerzas. Quiero sacarte de aquí antes de que vuelva.
 
Cinder vio como el hombre, con la cara roja, se daba prisa por liberarla. No le quedaban fuerzas con las que ayudarle. Era todo demasiado poco, demasiado tarde. Claramente era el tipo más débil de hombre, cómplice en su tortura y ahora arrepentido. Había conocido antes a gente así. En Moldox no sobrevivían por mucho tiempo. 
El Gobernador había terminado de desatarla y se inclinó, sacándola de la silla y levantándola entre sus brazos. 

--Te llevaré a algún lugar para que puedas dormir la mona --dijo--, mientras esperas que el Doctor vuelva --se tambaleó hacia la puerta, abriéndola de una patada--. Por si sirve de algo, creo que tienes razón. El Doctor es un hombre diferente estos días. Si agarra a Karlax después de esto, creo que podría matarlo. 

Sin embargo Cinder sólo oyó un vago murmullo mientras finalmente se dejaba caer hacia un pacifico olvido.

