\chapter*{Capítulo Veintiuno}

\addcontentsline{toc}{chapter}{Capítulo Veintiuno}



Un grupo de Dalek los condujo a punta de pistola por los silenciosos y cavernosos pasillos de la estación de mando. Las paredes aquí eran similares a las que había a bordo de los platillos Dalek en Moldox: un entramado hexagonal de cristal, latiendo con el paso de gases de colores y fluidos.

Pasaron por pasillos ramificados y habitaciones que formaban centros, como puntos de unión en un extraño y sobrenatural laberinto. Puertas selladas sugerían habitaciones y celdas, pero ninguna de ellas se abrió. Otros Dalek se deslizaban en silencio a lo largo de los pasillos, como solemnes monjes en los pasillos de una abadía, ni siquiera se saludaban unos a otros cuando pasaban. Aquí todos los Dalek parecían ser de la variedad típica de latón y oro. Ninguno de los mutantes o Degradaciones parecía estar presente.

Poco después, los guardias Dalek se detuvieron ante un gran arco abierto.



--Esperad --dijo uno de ellos en un profundo y mecánico tono, antes de entrar deslizándose en la cámara.



Cinder no podía ver mucho desde donde estaba de pie detrás de otro de los Dalek, aparte del hecho de que las paredes de la habitación cambiaron de apariencia convirtiéndose en blancas y opacas. El suelo también parecía estar hecho de paneles de metal blanco y liso.

El Dalek regresó momentos después, claramente habiendo comprobado algo. 



--Proceded.

--Me alegra ver que estás tan comunicativo --murmuró Cinder.



El Doctor, que hasta entonces había seguido su propio consejo mientras habían sido conducidos por la estación, se volvió hacia ella. 



--Bien, vamos a ver con qué nos enfrentamos, ¿de acuerdo? --se acarició la barba con nerviosismo, tirándose de la punta del bigote.



Cinder dio un grito de sorpresa cuando uno de los Dalek la tocó entre los omóplatos con su brazo manipulador, empujándola hacia delante. 



--Muy bien, muy bien --dijo--. Ya voy.



El Doctor miró al Dalek y la agarró por el brazo, tirando de ella hacia su lado. Juntos, cogidos del brazo, entraron en la habitación a encontrarse con el Circulo de la Eternidad.

Cinder y el Doctor se encontraron entrando en una gran cámara de audiencias, dispuesta en torno a un vestíbulo hexagonal.

En cada lado del hexágono, a excepción de la entrada, había un pedestal elevado, y encima de cada uno se sentaba un Dalek, mirando majestuosamente hacia abajo sobre el auditorio.

Eran similares en tamaño y apariencia a los Dalek de bronce que había visto en todas partes en la estación y tan frecuentemente en Moldox, el mismo brazo manipulador y arma de energía, el mismo ojo amenazante, salvo por los colores, que, así lo suponía Cinder, era lo que los distinguía. La carcasa de todos esos cinco Dalek “Circulo de la Eternidad” era de un profundo azul metálico con globos sensores de plata y cabezas abovedadas. Parecían ser idénticos entre sí, aunque sabía, al examinar las carcasas de Dalek muertos en Moldox, que usualmente llevaban pequeñas marcas bajo los ojos para ser más fáciles de identificar.

Otros dos pasillos se introducían en la habitación y una serie de sombríos huecos existían entre cada uno de los pedestales de los Dalek.



--Bienvenido Doctor --declaró el Dalek del pedestal central.

--Así que esto es el llamado Circulo de la Eternidad --dijo el Doctor con una sonrisa--. Os dais cuenta de que en realidad no es un círculo, ¿verdad? --trazó un círculo en el aire con el dedo para enfatizar su observación.



Los Dalek le miraron en silencio. Cinder notó que los guardias se habían retirado a la sombra de la arcada, mirando desde las líneas laterales.



--¿Así que sois los responsables de todo esto?. ¿Para utilizar la energía del Ojo de Tantalus? --dijo el Doctor--.Os lo reconozco, es ciertamente original.

--Es un arma digna de los Dalek --replicó el Dalek del pedestal central. Cinder decidió que tenía que ser el líder.

--¿Entonces, es eso lo que hacéis? --dijo el Doctor--.¿Os sentáis ahí en vuestros pedestales sintiéndoos superiores y soñando con nuevas maneras de torturar a cualquier otra forma de vida en el universo?.

--Ese es, efectivamente, nuestro propósito --dijo el Dalek, y otra vez, Cinder percibió una inteligencia profunda y perturbadora en acción. Este era el Dalek que habían oído a través del sistema de comunicación de la TARDIS. Era diferente a los otros, y no sólo en virtud del color de su carcasa. Parecía tener un sentido de la ironía--. Nosotros, el Círculo de la Eternidad somos los encargados de asegurar la proliferación de la raza Dalek a través del tiempo. Nos encargamos de la invasión de la historia con el fin de asegurar el futuro, y la erradicación de todas las demás formas de vida.

--Criaturas nacidas del odio --escupió el Doctor en respuesta--. Me dáis asco.

--Semejante furia. Semejante ira pura y ardiente. Es algo de una rara belleza para la vista. Eres tan digno como esperábamos, Doc-tor --el Dalek parecía impresionado.

--¿Digno? --dijo el Doctor--. ¿De exterminio?. Tenía la impresión de que simplemente uno tenía que tener la osadía de estar vivo para justificar semejante respuesta de tu especie.



El Dalek hizo un sonido como si se estuviera ahogando, un extraño grito ahogado que Cinder se dio cuenta, con disgusto, de que en realidad era una carcajada ronca. El Dalek de hecho se estaba riendo.



--Eres aún más iluso de lo que me había imaginado --dijo el Doctor, señalando a cada uno de los cinco Dalek azules por turno--. Sentados aquí en vuestra torre de marfil, incubando planes y construyendo vuestras superarmas.

--El Cañón Temporal es sólo una pequeña pieza, Doctor. Un medio para un fin. Gallifrey será destruido de todas formas. La ambición Dalek conoce límites mucho más grandes --el Dalek hizo una pausa, como si sopesara sus palabras--. Tu, Doctor. Vas a ser nuestro salvador. Vas a garantizar la supervivencia de la especie Dalek.



El Doctor entrecerró los ojos. 



--No lo haré --dijo--. Cometí ese error una vez, en Skaro, cuando no pude poner fin al trabajo de vuestro creador.

--Ah --dijo el Dalek--. El comienzo de la Guerra del Tiempo. El momento en que tu, Doctor, enseñaste a los Dalek la lección más valiosa de todas, que la emoción es una debilidad que debe ser erradicada. Que la misericordia no tiene cabida en la victoria.

--No es una debilidad --dijo el Doctor--, sino una fuerza.

--Si no hubiera sido por tu vacilación --dijo el Dalek, con tono burlón--, por tu incapacidad de hacer lo que era necesario, toda la Guerra se podría haber evitado. Los Dalek habrían dejado de existir.

--¿Es eso cierto? --dijo Cinder, asombrada--. ¿Tuviste la oportunidad de matarlos a todos ellos y les dejaste vivir?.

--Continúa --dijo el Dalek--. Cuéntaselo a tu compañera. Cuéntale cómo fallaste.

--Es cierto --dijo el Doctor, bajando la cabeza--. Mucho antes de que comenzase la Guerra, en una vida diferente, tuve la oportunidad de impedir la aparición de la raza Dalek, de asesinarles en su cuna antes de que el universo conociese su terror --suspiró--. Pero dudé. Todavía tenía la esperanza de que pudiesen ser salvados. Estaba equivocado y cuando volví, cuando me di cuenta de mi error y traté de destruirlos, era demasiado tarde. Ya habían empezado la línea de producción.



Cinder no sabía qué decir. La idea de que podría haber evitado todo esto, lo que le había sucedido a su familia, a sus amigos, a los trillones de vidas que se habían perdido por todo el cosmos, ¿cómo pudo haber permitido que todo eso sucediera?.

El Doctor había tenido ese poder en sus manos. Sin embargo también le había hablado de la responsabilidad, de que ejercer tal poder nunca debería ser la carga de una sola persona. ¿Realmente tenía el derecho de destruir a una raza en su infancia, antes de que realmente entendiese de lo que era capaz de hacer, de cómo podría evolucionar?. Por supuesto que no. No se le podía culpar por eso. 



--No es malo --dijo ella, en voz baja--. Sólo humano --fue el mayor cumplido que se le ocurrió ofrecerle. Él sonrió con gratitud.

--Ya veo, Doc-tor, que lo entiendes --dijo el Dalek--. Y ahora te ofreceremos un regalo, un papel en el próximo inicio del nuevo imperio Dalek. Te convertirás en nuestro instrumento de extinción. Concebirás nuevos e inventivos medios de difundir nuestro regalo de muerte a través del cosmos. Tu ira reavivará las llamas de la guerra, llevando a los Dalek a la victoria en un billón de mundos. Será un reino hermoso y aterrador, y los Dalek te adorarán por ello --el Dalek se quedó en silencio, a la espera de la respuesta del Doctor.

--Moriré antes de mover un dedo para ayudaros --respondió el Doctor.

--La entidad conocida como el Doctor, efectivamente, será exterminada --dijo el Dalek--. Los centros emocionales de tu cerebro serán neutralizados. Se te extirparán todos los pensamientos de tus vidas anteriores. Tu mente, sin embargo, será cosechada. Tu creatividad tendrá un uso maravilloso apoyando la causa Dalek.

--No lo entiendes, ¿verdad? --dijo el Doctor, riendo--. No tienes ni idea. Es a causa de mis emociones que soy el que soy. Sin ellas no sería nada más que un zángano, como el resto de su patética raza.

--Lo veremos, Doctor --dijo el Dalek. Su ojo giró de repente a la derecha--. Ha llegado el momento. Comenzad el proceso.

--Obedezco --llegó la metálica respuesta desde algún lugar fuera de la vista.



Cinder sintió un movimiento en uno de los huecos de debajo del Círculo de la Eternidad, y algo comenzó a salir de entre las sombras. Era la silueta de un Dalek, pero mucho más grande.



--Contempla, Doctor, al Depredador Dalek.



Cinder observaba con una mezcla de asombro y horror como el nuevo Dalek avanzaba rodando hacia la luz. Tenía el doble de tamaño de un Dalek estándar, aunque construido con ese mismo familiar diseño. Su falda era de un profundo bermellón metálico, con los globos sensores negros y rejillas, y aunque estaba en su mayor parte inanimado, su aspecto, sin embargo, llenaba a Cinder de pavor. Los Dalek claramente habían estado planeando esto durante algún tiempo, y estaba empezando a tener la sensación de que el Doctor había entrado sin querer en su trampa.



--Esta es nuestra verdadera victoria, Doctor --dijo el Dalek en el pedestal--. El arma que va a ganar la guerra. El Ojo de Tantalus no es más que un medio para un fin, la eliminación de la distracción que son los Señores del Tiempo. El Depredador Dalek será el heraldo de una nueva era. La hora de los Dalek se acerca.



La carcasa del Depredador Dalek se separó como puertas abriéndose simultáneamente hacia afuera, revelando una gran cavidad en su interior. Dentro, había un asiento de metal bruñido, rodeado de diales y monitores, asemejándose a la cabina de un vehículo pequeño. Estaba claramente vacío y a la espera de un ocupante, pero a diferencia de las carcasas Dalek estándar que había visto en Moldox, éste no fue diseñado para albergar en la silla a mutantes Kaled, sino a una figura humanoide.

Cables fibrosos y racimos de sondas en forma de aguja luchaban por el espacio a cada lado de la silla, y una púa de metal afilado, fijada a la parte posterior del reposacabezas, brillaba por el lubricante. Era claramente la versión Dalek de una interfaz neuronal, para insertarse en el tejido blando de la base del cráneo del ocupante.

Más agujas de aspecto depravado estaban fijadas en el interior de las puertas con bisagras como una Doncella de Hierro, a la espera de ser clavadas en la carne del ocupante una vez que estaba en su interior.

Una vez emparedado, era evidente que no había escapatoria. Las máquinas de la carcasa se unirían con la biología del ocupante, fusionándose para convertirse en una sola entidad simbiótica. Un Dalek.



--Esto, Doctor, es tu futuro --dijo el líder de los Dalek.



Desde la puerta, los tres guardias Dalek avanzaron hacia adelante, formando un amplio círculo alrededor de Cinder y el Doctor.



--¿Doctor? --dijo, preocupada. Él la miró, y vio verdadero terror en sus ojos. No había previsto esto, las verdaderas intenciones del Círculo de la Eternidad. Es más, no parecía que fuesen a sortear el asunto o a ofrecerle ninguna posibilidad de escapar. La carcasa del Depredador Dalek se quedó de pie esperando para aceptar a su nuevo huésped.



--¡Vamos! --gritó el Doctor, mirando a su alrededor frenéticamente--. ¡Vamos!.

--No luches contra él, Doctor --dijo el Dalek.

--Cinder... yo... --no parecía saber qué decir.

--Tu compañera tendrá el privilegio de ser el primer humano en ser exterminado por el Depredador Dalek.



Los guardias se acercaron más, sus brazos manipuladores levantados como pinchos para ganado. Cinder quería gritar. Hubiera querido tener su arma, tener algo, pero no parecía haber ninguna forma de defenderse, y ningún lugar hacia el que escapar. El Doctor les había llevado sin querer a una trampa.

Corrió hacia delante, agarrando al Doctor frenéticamente por las solapas de su chaqueta de cuero. Él la envolvió en sus brazos, besándola en la parte superior de la cabeza. 



--Lo siento, Cinder --dijo.

--Doc-tor, es el momento --dijo el Dalek con voz áspera.



Se oyó un sonido como el retumbo distante de un trueno. Al principio, Cinder pensó que los Dalek habían hecho algo, que de alguna manera habían activado el Depredador Dalek. A medida que el sonido se hacía más fuerte, sin embargo, reconoció el familiar y chirriante resoplido. Desgarró el aire de la cámara de audiencias.



--¡Explicar! --gritó el Dalek--. ¡Explicar!.



Cinder sintió que el Doctor le agarraba con más fuerza por los hombros. 



--¡Agarrate! --exclamó.

--¡No! --bramó el Dalek, cuando el mundo alrededor de Cinder empezó a cambiar de repente. Ella vio como iluminados redondeles aparecían parpadeando, sustituyendo su visión de las paredes blancas y de los guardias Dalek.

--¡Exterminar! --La voz del Dalek sonaba distante, y ella se estremeció al oír cerca el sonido de un arma de energía disparando.



El disparo nunca la alcanzó, puesto que las paredes de la TARDIS se cerraron milagrosamente en torno a ellos, arrancándoles hábilmente de las garras de los guardias Dalek.



