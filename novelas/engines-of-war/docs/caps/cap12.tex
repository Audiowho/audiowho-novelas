\chapter*{Capítulo Doce}

\addcontentsline{toc}{chapter}{Capítulo Doce}



El Doctor volvió de nuevo a la vida con un centelleo en el tempestuoso páramo que una vez fuera la Zona de la Muerte. 

Le costó un poco recuperar la orientación. Ya no usaba el teletransporte con tanta frecuencia y durante un momento se sintió desorientado. 

El paisaje era salvaje, inmenso y peligroso. Los afloramientos rocosos habían estado expuestos a los elementos, los bosques habían crecido sin control, y el cuidado de toda clase de bestia salvaje formaba parte de lo que qeudaba de los días de los juegos. El Doctor sabía que, ocultas bajo la hierba, había ciénagas y pantanos mortales. Peor aún, había sistemas de cuevas que eran propensos a colapsar, y algunos de ellos eran la única forma de pasar de una parte de la zona a otra. 

Aquí, los Señores del Tiempo de la antigüedad habían llevado a cabo los brutales juegos de la vida y la muerte, durante los cuales especies exóticas eran sacadas de sus hábitats naturales y obligadas a enfrentarse entre sí en este rincón salvaje de Gallifrey. Era un deporte de espectadores, parecido a los de los antiguos romanos de la Tierra, un pasatiempo sanguinario y violento propio de una época menos ilustrada y que muchos Señores del Tiempo deseaban olvidar. 

El Doctor había jugado una vez, hacía mucho tiempo, cuando un loco arrancó a cinco de las encarnaciones del Doctor de sus líneas temporales y trató de depositarlos aquí con la esperanza de que le llevarían a la tumba y al secreto de la longevidad de Rassilon. El hombre responsable había sido un viejo amigo y mentor, Borusa, que se había obsesionado con alcanzar la inmortalidad y tenía la necesidad de mantener un control sobre su menguante poder. 

Al seguir al Doctor a la tumba se le había concedido su deseo, ya que el Señor del Tiempo Presidente lo había engañado y había sepultado su conciencia en el relieve de una piedra para toda la eternidad. 

Rassilon había sido resucitado en los primeros días de la Guerra, y había sido alentado a tomar forma corpórea una vez más para liderar a los Señores del Tiempo en su cruzada contra los Dalek.

El Doctor una vez lo había imaginado como un líder benigno, un innovador y un gran estadista, tal y como lo pintaban las antiguas leyendas, pero ahora sabía que Rassilon era tan imperfecto como cualquier otro Señor del Tiempo. Peor aún, sus ideales estaban anticuados, y su inflado sentimiento de importancia era lo que movía su política. En su propia mente, Rassilon se había convertido en un dios, capaz de hacer lo que quisiera. 

Los Señores del Tiempo siempre había preferido una sociedad autocrática, les gustaba mucho que les dijeran lo que tenían que hacer, pero lo que había visto durante la reunión del Consejo Superior había convencido al Doctor más que nunca de que las cosas iban muy mal. La única pregunta que se planteaba era cómo iba a interceder con éxito. 

Se subió el cuello de la chaqueta y se mesó pensativamente la barba. Había emergido del teletransporte al pié de la Torre, una alta estructura que estaba en el corazón de la Zona Muerta. La Torre estaba tallada en imponentes y oscuras losas de granito coronadas con un remate en forma de un globo de oro atravesado por una media luna. Debía de ser uno de los lugares más inhóspitos que el Doctor jamás había visto, su arquitectura brutal pertenecía a aquellos primitivos días en los que Gallifrey hacía sus primeras incursiones en la ingeniería estelar que más tarde les permitiría adentrarse en el Tiempo. Dentro de la Torre estaba la antigua tumba de Rassilon. 

Este debía de ser el lugar al que había ido el Presidente. ¿Pero que pretendía?. ¿Por qué volver a este lugar, donde había dormido en paz durante tantos milenios, que ahora estaba, sin duda, en desuso?. ¿Había encontrado Rassilon otro uso para el lugar?. ¿Algo que no quisiera compartir con el resto del Consejo Superior?. 

El Doctor supuso que sólo había una forma de averiguarlo. Se arriesgaba a incurrir en la ira de Rassilon pero llegó a la conclusión de que ya era un poco tarde para eso, y ahora tenía curiosidad por descubrir exactamente de qué se trataba ese ``motor de posibilidad''. 

Se encaminó hacia la Torre, acercándose a las puertas principales. En este punto dos inmensos pilares flanqueaban la entrada y había braseros de hierro montados en unos picos profusamente ornamentados, sus cuencos brillaban con una llama de color naranja brillante. 

Esperaba que Cinder se mantuviera alejada de los líos en el Capitolio, aunque lo dudaba. Probablemente se las estaba haciendo pasar canutas a Karlax. Tampoco era que no se lo mereciera, aunque estaría bien que, por lo menos, uno de ellos no acabara en una celda. 

El Doctor tomó aliento y se adentró en la Torre. 

El interior de la Torre era cavernoso y estaba iluminado por más braseros que arrojaban largas y parpadeantes sombras, dando al lugar un tono sombrío, lo que, en opinión del Doctor, era lo normal para una tumba. 

El espacio estaba dominado por la tumba en sí, asentada en un gran zócalo en el centro de la cámara, con impresionantes columnas de mármol en cada esquina y un corto tramo de escaleras que conducían a la tarima. Banderas en grises y azules colgaban, hechas colgajos, del techo. Tiempo atrás debían haber sido espléndidas, pero ahora estaban llenas de polvo y estaban podridas, como una representación, decidió el Doctor, de la desvanecida gloria de los propios Señores del Tiempo. 

El Doctor observó cómo Rassilon entró en la gran sala, sus ondulantes ropas arrastradas por el suelo de mármol, levantando remolinos de polvo a su paso. Su bastón personal golpeando ligeramente con cada paso, haciendo eco en el desolado y abandonado lugar. 

Se acercó a una pequeña consola hexagonal y pasó las manos sobre ella, despertando el sistema de modo que una serie de runas iluminaron toda su superficie. A continuación se volvió y se acercó a la tumba vacía, sobre la cual su cuerpo, o mejor dicho, el cuerpo de su encarnación anterior, había permanecido una vez. 



--¡Borusa! --llamó con voz potente, como si fuera a despertar a los muertos--. ¡Borusa!. Te necesito.



¿Estaba Borusa todavía aquí, su esencia atrapada en el relieve tallado en el lateral del ataúd de granito de Rassilon?. Durante aquel fatídico episodio, en que el Doctor se había visto obligado a soportar la Zona de la Muerte, este era el lugar en el que Rassilon había encarcelado a Borusa. 

Se produjo una especie de zumbido desde lo alto de la tumba, y mientras el Doctor miraba, una plataforma comenzó a enderezarse, girando de modo que la figura que estuviera encima de la tumba quedaría vertical para cualquiera que estuviera abajo. 

Rassilon apareció en la parte inferior de la escalera, mirando hacia arriba en el momento en que la máquina completó su ciclo. El Doctor, aún de pie en la puerta, se deslizó hacia delante para ver mejor, y lo rodeó desde la distancia, a la sombra de un contrafuerte. Se encogió cuando sus desgastadas botas arañaron el mármol pulido, pero por suerte el sonido chirriante del mecanismo amortiguó sus pasos y Rassilon no miró hacia atrás. 

La visión de aquella cosa en la tumba era algo que el Doctor nunca olvidaría, no en todas sus vidas. Era completamente monstruoso. Borusa había sido atado a la estructura de acero por las muñecas y los tobillos, de modo que su cuerpo quedaba en forma de cruz. Todavía llevaba sus ropas ceremoniales, pero quedaban abiertas sobre su pecho, estaba claro que escondían multitud de pecados. 

Su cuerpo era un desastre que descansaba en el corazón de un nido de alambres y cables. Su pálida piel estaba agujereada en los puntos donde se habían realizado incisiones en el pecho, en los que a su vez se habían insertado tubos que se habían empujado profundamente en la cavidad, presuntamente para inflar sus pulmones y mantener, por lo menos, a uno de sus corazones latiendo. 

Su cabeza estaba coronada por un casquete de metal de cuya parte posterior surgía un nudo de cables que se dirigía hacia a la parpadeante caja de un relé neuronal. 

Lo más aterrador de todo era su rostro, que parecía estar atrapado en un ciclo sin fin de transición, acompañado por el suave resplandor de la energía regenerativa. Se retorció y reformó mientras el Doctor miraba, pasando por un reflejo de todas sus encarnaciones anteriores, o al menos las que el Doctor reconoció. Sus ojos titilaron una brillante energía mientras miraba fijamente, aunque sin ver, a Rassilon. 

Así que esto era el ``motor la posibilidad'' de Rassilon. 



--Dime, Borusa, ¿qué ves? --dijo Rassilon, casi respetuosamente. 

--Veo Gallifrey ardiendo --graznó Borusa,con una voz que era apenas un susurro--. Veo el final de todas las cosas, la oscuridad que limpia. Veo el momento, el preciso momento en que todas las cosas dejarán de ser. 



Rassilon cerró su enguantado puño. 



--Pues eso es lo que tenemos que cambiar --dijo. --¿Y el Ojo Tantalus?. El Doctor nos trae noticias de un arma Dalek, un dispositivo del juicio final que podría suponer nuestra destrucción. 



Por un momento Borusa no respondió, pero volvió la cabeza como si estuviera mirando a lo lejos buscando una visión del futuro. 



--El Doctor dice la verdad --dijo--. El plan de los Dalek se acerca a su cenit, y si no se controla, erradicará a Gallifrey y a todos sus hijos de la historia. 

--Entonces tenemos que actuar --dijo Rassilon--. No tenemos otra opción.

 

Borusa volvió la cabeza y sus extraños, crepitantes, ojos parecían fijados en el Doctor, que seguía al acecho entre las sombras, junto a la puerta. El Doctor se estremeció con nerviosa expectación. ¿Había visto Borusa que estaba aquí?. ¿Lo delataría?.

 

--Hay otra persona aquí --dijo Borusa, respondiendo a la pregunta mental del Doctor--. Adelántate. Doctor.

 

Bueno, supuso que era demasiado tarde para salir corriendo así que dio un paso hacia la luz para ver como Rassilon se giraba, levantando su mano enguantada. El puño de metal comenzó a brillar, adquiriendo un brillo azulado y zumbando con la energía. Lo tendió extendiendo sus dedos, como si fuera a cerrarlo en cualquier momento, aplastando al Doctor. En cambio la bajó a su costado, y el resplandor azul comenzó a desaparecer. 



--Podría haberte matado, Doctor, por tu intrusión. Aquí no eres bienvenido --dijo. 



El Doctor le ignoró y miró a Borusa, atado al monstruoso dispositivo. 



--Rassilon, ¿qué has hecho?. 

--Una cosa espléndida, ¿no? --graznó Rassilon, incapaz de resistir la tentación de presumir--. Este, Doctor, es mi motor de posibilidad. 

--Es terrible --dijo el Doctor--. Monstruoso.  

--Es un regalo. Borusa trae entendimiento. Su recompensa es ver todas las maravillas del universo, en todas sus múltiples formas.  

--Y todos los horrores, también, teniendo en cuenta como suena --dijo el Doctor--. ¿Qué le has hecho?.

--Su línea de tiempo ha sido retro-evolucionada --dijo Rassilon--. Está atrapado en un ciclo regenerativo iterativo, siempre cambiando, convirtiéndose cada vez más. 

--¿Más? --dijo el Doctor--. A mi me suena terriblemente a una prisión.  

--Te falta imaginación, Doctor. Esta máquina, que ha liberado a Borusa de la prisión de la carne, ha desbloqueado su verdadero potencial. Es libre para pasear por todo el Tiempo y el Espacio dentro de su propia mente. Puede navegar por cada cambio de la realidad, cada posibilidad.  

--¿Borusa?. ¿Qué hay de ti? --preguntó el Doctor. 

--Veo --respondió Borusa, simplemente--. Lo veo todo.  

--Es una abominación --dijo el Doctor--. Lo que has hecho aquí, nos deprecia a todos.  

--Te equivocas, Doctor. Borusa ha trascendido. Él representa el futuro. Él es el afortunado, el primero de nosotros que es verdaderamente libre. 



El Doctor negó con la cabeza. 



--¿No lo ves, Rassilon?. ¿No ves lo que esto significa?. Al hacerle esto a Borusa, nos estás poniendo al mismo nivel que los Dalek. Nos estás alterando, transformándonos en algo menos de lo que somos. Está a un sólo paso de la eliminación de nuestra capacidad de empatía, de la emoción. ¿Dónde nos llevará?. ¿Soldados sin conciencia?. ¿Unidades de transporte de metal?.

 -- Estás siendo melodramático, Doctor --dijo Rassilon--. Es un hombre. 

--Siempre comienza con un hombre, Rassilon --contestó el Doctor solemnemente. 

--He hecho lo que era necesario --dijo Rassilon--. Lo que nadie más estaba dispuesto a hacer. Borusa entendió lo que necesitaba de él. El motor de posibilidad representa nuestra salvación. Con él, podremos ver el tejido de todos los futuros posibles. Podemos elegir nuestro propio destino, siguiendo sólo los caminos más beneficiosos. Podemos medir los potenciales resultados de todas nuestras ofensivas contra los Dalek, asegurando la victoria a cada paso. ¡Podemos poner fin a la guerra!. 

--Adelante, entonces --dijo el Doctor--. Pregúntale. Pregúntale si hay una manera de desplegar la Lágrima de Isha sin asesinar a toda esa gente.

 

Rassilon lo miró, desafiante. 



--Muy bien --dijo y se volvió hacia Borusa--. Borusa, el plan del Gobernador para desplegar la Lágrima de Isha en el Ojo de Tantalus, ¿funcionará?. ¿Pondrá fin a la amenaza Dalek en ese sector, y destruirá su nueva arma temporal?. ¿Nos salvará?.

 

Borusa giró la cabeza de lado a lado, emitiendo un suave gemido. Tras un momento habló. 



--Funcionará. La Lágrima cerrará el Ojo, y el arma Dalek será neutralizada. 

--Excelente --dijo Rassilon. 

--Sin embargo el aplazamiento será sólo temporal --continuó Borusa--. La oscuridad vendrá igualmente y ahogará todas las cosas. La edad de los Señores del Tiempo llega a su fin. 

--¿Lo ves? -- dijo el Doctor--. Incluso tu propia máquina te advierte de que esta no es la solución. La Lágrima no es la respuesta, no los detendrá.  

--¡Nos dará tiempo! --dijo Rassilon--. Un tiempo muy valioso para preparar, elaborar estrategias, y poder consultar aún más el motor de la posibilidad.  

--¿A qué coste? --dijo el Doctor--. Afirmas ser mejor que los Dalek, que es nuestro deber sobrevivir a esta guerra, traer la paz y la estabilidad al Universo, y sin embargo te parece bien rediseñar la esencia misma de tu propia gente para convertirlos en activos estratégicos, para destruir civilizaciones enteras, para conseguir lo que quieres. ¿En que sentido es eso mejor?. ¿Cómo lo consideras diferente?. 

--Hablas como si prefierieras los Dalek a tu propia gente, Doctor --dijo Rassilon--. ¿Voy a tener que considerarte un traidor a partir de ahora?. 

--Odio a los Dalek por todo lo que representan --respondió el Doctor, su tono de voz calmado. Estaba tratando de no perder la calma, a pesar del hecho de que cada fibra de su ser le gritaba que tirara a Rassilon al suelo y tratara de meter algo de sentido en aquél hombre idiota antes de que fuera demasiado tarde--. No quiero terminar odiando a mi propia gente por las mismas razones --añadió. 



Rassilon permaneció en silencio, como pensando las palabras del Doctor. El Doctor miró hacia arriba, hacia la desorientada y cambiante cara de Borusa. 



--Borusa, ¿existe alguna forma de implementar la Lágrima en el Ojo de Tantalus sin causar la muerte de las personas que habitan en los doce mundos de la Espiral?. 

--No --dijo Borusa sin dudarlo--. No veo ninguna posibilidad en la que los colonos humanos sobrevivan si la Lágrima se despliega. 



El Doctor se volvió hacia Rassilon. 



--Ya tienes tu respuesta, ¿no?. 

--Sólo tengo conciencia de las consecuencias de mis acciones, Doctor. Esto no cambia nada --dijo Rassilon. Su tono era firme, final--. Ya se ha tomado la decisión. La Lágrima se desplegará. Había pensado salvar a tus preciados humanos si Borusa nos hubiera enseñado una manera de hacerlo, pero por desgracia no lo ha hecho. Ha llegado el momento de actuar --se acercó a la consola e inició el comando para bajar la cuna de Borusa de nuevo a la tumba. 



El Doctor lo siguió. 



--No puedes hacer esto, Rassilon. Esto lo cambia todo. Te lo advierto, nunca podrás volver atrás. 

--Ya está hecho --fue la respuesta, severa y final--. Ven, voy a hablar de nuevo con el Consejo Superior --se dirigió hacia la puerta.



Con los corazones hundidos, el Doctor siguió a Rassilon de vuelta a la estación de transmisiones.



