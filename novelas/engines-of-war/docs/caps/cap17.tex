\part*{Parte 3 \\ \vspace{2 mm} {\large Dentro del ojo}}
\addcontentsline{toc}{part}{Parte 3}
\chapter*{Capítulo Diecisiete}
\addcontentsline{toc}{chapter}{Capítulo Diecisiete}



La TARDIS colgaba todavía en el Vórtice del Tiempo en medio de un caos espiral de púrpuras y azules. Cinder podía ver los enfurecidos colores a través del techo transparente de la sala de la consola, como una tormenta tempestuosa, nubes amoratadas preñadas de crepitante energía. Por lo que sabía, la capa exterior de la nave estaba siendo zarandeada y golpeada, pero en el interior todo estaba en calma.

Se sentó en el borde de la tarima central, de espaldas a la consola de la nave. Le parecía extraño no estar corriendo, perseguida a través de pasillos serpenteantes o tratando desesperadamente de escapar de una nave Dalek, de una ciudad en ruinas o de una prisión de los Señores del Tiempo. De cualquier lugar. Lo consideró por un momento. Toda su vida había estado tratando de escapar de donde estaba, siempre dando por sentado que, si tan sólo pudiera escapar, habría un lugar mejor esperándola allí fuera en algún lugar en el cosmos. Un lugar al que pudiera llamar hogar.

Ahora, no estaba tan segura. No podía dejar de preguntárselo, mientras observaba la tormenta ondulante del Vórtice del Tiempo soplando a su alrededor, si el universo entero estaba así de enojado, si era así de violento. Ciertamente lo parecía.

De todos modos, le extrañaba que siguiera sentada.

El Doctor iba de un lado a otro de la sala de la consola como si hubiera perdido algo.



--¿Qué vamos a hacer ahora? --preguntó Cinder. Apoyó la barbilla en la palma de la mano--. No podemos volver a Moldox, y no podemos volver a Gallifrey. No somos muy populares, ¿verdad? --murmuró.



El Doctor dejó lo que estaba haciendo por un momento. 



--Vamos a impedir que los Señores del Tiempo utilicen la Lágrima de Isha --dijo--. Eso es lo que vamos a hacer.



``No será nada fácil'', pensó Cinder, ``para dos renegados en una cabina azul''. Se dio cuenta de que se estaba poniendo sensiblera y decidió animarse. 



--Supongamos que tenemos éxito --dijo--, y que somos capaces de evitar que los Señores del Tiempo colapsen el Ojo de Tantalus...

--Si --dijo el Doctor.

--Entonces, ¿qué pasa con los Dalek?. ¿Qué vamos a hacer con ellos?. Seguramente no vas a permitirles terminar lo que han empezado, borrar Gallifrey de la existencia.

--¿No debería? --dijo, pero ella se dio cuenta de que en su corazón no lo creía. Estaba furioso con su pueblo y con razón. No sólo se habían negado a escucharle, sino que se habían vuelto contra él cuando estaba tratando de ayudarlos, cuando más lo necesitaban. Le habían mostrado sus verdaderas caras.

Todo lo que había oído acerca de los Señores del Tiempo, todos los rumores, ahora suponía que debían ser verdad, que en realidad no había mucha diferencia entre ellos y los Dalek.

``Todos, excepto uno'', pensó con una sonrisa. ``Él no era tan malo''. Y no creyó ni por un momento que iba a esperar a ver cómo eran borrados del tiempo.



--¡Ah, lo tengo! --exclamó el Doctor, y Cinder se acercó arrastrando los pies para ver lo que estaba haciendo, levantando las rodillas hasta el pecho y enroscando los brazos a su alrededor.

El Doctor estaba de rodillas, preocupándose por algo que estaba debajo de la consola en forma de hongo. 



--¿Qué es eso? --preguntó.

--Espera ... --fue la respuesta amortiguada. Tenía una mirada de intensa concentración y su lengua le sobresalía cómicamente de la esquina de la boca--. ¡Ya está! --declaró. Una pequeña vaina negra, que evidentemente había sido colocada debajo del panel de control, se desprendió en su mano. La arrojó en el aire y la atrapó de nuevo--. Eso les enseñará --dijo, poniéndose de pie.

--¿Qué es?. ¿Qué has encontrado?.



El Doctor se puso en pie y se acercó a donde estaba sentada. Extendió la mano. El objeto negro era un ovoide delgado, hecho de lo que parecía ser cerámica brillante. No revelaba mucho. 



--Un dispositivo de seguimiento --dijo--. Sabía que no serían capaces de resistirse. Los hombres del Gobernador lo deben haber colocado antes de llevar la TARDIS al desguace.

--¿Así que entonces crees que vendrán tras nosotros? --preguntó Cinder. Era lo que les faltaba. Había tenido más que suficiente de los Señores del Tiempo para toda una vida. Tal vez dos.

--No me sorprendería nada --dijo el Doctor, con resignación--. Si nos pueden encontrar, lo harán. Sabrán que vamos detrás de la Lágrima. Probablemente enviarán una fuerza de ataque para tratar de detenernos.



Cinder suspiró. ``¿Cuándo terminará todo esto?'' 



--Así que, entonces, los Dalek... --dijo.

--¿Qué pasa con ellos?.

--Bueno, sé que no lo decías en serio. Pero ¿qué vamos a hacer para detenerlos? --dijo--. Si usan su arma en Gallifrey, sólo será cuestión de tiempo que la usen también en otra parte. No habrá nadie para detenerlos.

--Sí, aún no he pensado mucho en eso --dijo el Doctor. Sus pobladas cejas se movieron.

--Entonces, ¿has considerado la alternativa? --dijo. Se odiaba a sí misma por tan siquiera sacar el tema, pero tenía que decirlo.

--No hay alternativa --respondió el Doctor.



Cinder negó con la cabeza. 



--Podrías permitir que los Señores del Tiempo utilizasen la Lágrima. ¿Y si Rassilon tiene razón?. Las vidas de unos pocos millones de esclavos humanos para garantizar la seguridad de todos los demás en el universo...



El Doctor parecía furioso. 

--No es así como esto funciona, Cinder. No podemos tomar esa decisión. Nadie debería ejercer esa clase de poder.

--Pero si no lo hacemos, ¿no estamos entregando ese mismo poder a los Dalek?.

--Encontraré una manera --dijo el Doctor--. Siempre lo hago. Pero sin duda no permitiré que los Señores del Tiempo cometan un genocidio para hacerlo.



Cinder asintió. Despreocupadamente, cogió el cañón Dalek, que yacía en el suelo junto a sus pies. No había nada estéticamente agradable en esa cosa. Era un arma de puro odio, funcional y mortal.



--¿De dónde has sacado eso? --dijo el Doctor. Parecía receloso.

--De aquí, justo de donde lo dejaste--respondió Cinder.



El Doctor se puso de cuclillas a su lado y le tendió la mano. Ella se lo pasó. 



--Dejé esto en Gallifrey --dijo--. En la Sala del Consejo.

--Deben haber decidido devolverlo --dijo Cinder.

--Ummm --murmuró el Doctor, girando el arma en las manos--. Sí, como yo pensaba --le dio la vuelta para mostrarle el pequeño nódulo negro que había sido escondido debajo del cañón. Lo arrancó.

--Parece que no corren riesgos --dijo--. Tengo la reputación de ser difícil de mantener dentro de una celda.



Se puso de pie, dejando los dos dispositivos de localización uno al lado del otro en el suelo. Después los pisoteó repetidamente, aplastándolos bajo su talón hasta que no quedó nada excepto una pasta de cerámica y diodos rotos. 



--Ya está, eso debería evitarlo --dijo.



Le ofreció la mano, tirando para levantarla. 



--Ahora, es hora de que descanses un poco.



Cinder se frotó los ojos. 



--No, estoy bien --dijo.



El Doctor negó con la cabeza. 



--Necesitas dormir.



Ahora que lo había mencionado, se dio cuenta realmente de lo cerca que estaba de la extenuación. Sus ojos estaban calientes y pesados y tenía un dolor sordo en la parte posterior del cráneo, que había estado allí desde su experiencia con la sonda mental. Las extremidades le pesaban como el plomo. 



--Está bien --dijo--. Un ratito.

--Sube esas escaleras --dijo el Doctor--.La primera puerta a la izquierda. Encontrarás un sitio para dormir un poco --Cinder se alisó la túnica. Estaba sucia--. Debería haber algo de ropa limpia en el armario, también. Coge lo que quieras.

--Gracias --dijo ella. Se dirigió hacia las escaleras. Quizá por primera vez en su vida, se sentía tranquila. Sin embargo, no pudo evitar preguntarse si esto era la calma que precede a la tormenta, el sobrecogedor silencio en la noche antes de la batalla. De todos modos, tardaría un rato en reunir fuerzas.

--¿Sabes ya lo que vas a hacer? --dijo--. ¿Cómo vas a detenerlos?. Quiero decir a los Señores del Tiempo.



El Doctor sonrió. 



--Siempre he sido único para... bueno, improvisar --respondió.



Cinder se echó a reír. 



--Yo también.



Subió las escaleras en busca de una cama.

Cinder se precipitó por las escaleras para encontrar al Doctor aún de pie en la consola, jugando con los controles. 



--He estado durmiendo durante horas --dijo--. ¿Por qué no me has despertado?.



El Doctor levantó la mirada, impasible. 



--Necesitabas descansar --dijo.

--¡Pero la Lágrima!. ¿No llegaremos demasiado tarde?.



El Doctor se echó a reír. 



--Esto es una máquina del tiempo --dijo--. Aquí, en el Vórtice estamos a un paso de lo que está pasando.

--No lo entiendo.

--Imagínatelo como un río --dijo el Doctor--, siempre fluyendo, siempre corriendo. Eso es el tiempo, y la TARDIS se cierne por encima de ese río. Siguiéndolo aguas arriba podemos sumergirnos en el futuro, volviendo por la dirección opuesta, y aunque estemos nadando contra corriente, podemos llegar a cualquier punto en el pasado.

Cinder negó con la cabeza, bajando de un salto el último de los escalones. 



--Aceptaré tu palabra --dijo.



Recorrer la TARDIS no había sido tan simple como ``la primera puerta a la izquierda'', que en realidad la había llevado a un patio palaciego lleno de olivos y bancos de parque, completado con una fuente de mármol esculpido para parecerse a una mujer desnuda, vertiendo agua desde una jarra. Desde aquí, al menos otras cinco puertas llevaban a habitaciones contiguas. Probó cada una, descubriendo todo tipo de entornos extraños: una selva chillona, cargada con el olor de lluvia fresca, una enorme pajarera lleno de coloridos pájaros cantando, un laboratorio de química con antiguos bancos de madera, llaves de gas y quemadores Bunsen y estanterías llenas de innumerables frascos. Finalmente había encontrado un dormitorio, evidentemente todavía repleto con el desorden de un ocupante anterior. Cinder no recogió la mayor parte, sino que simplemente se derrumbó sobre la cama y se quedó dormida en un largo y lujoso sueño.

Al despertar, había encontrado un par de ajustados pantalones vaqueros negros y una camiseta de Greenpeace en el armario, aunque no tenía ni idea de lo que significaba el eslogan.



--Así que, ¿cuál es el plan?.

--Dirigirnos a la Espiral Tantalus --dijo el Doctor--. La flota de los Señores del Tiempo va a tener que acercarse si tienen la intención de utilizar la Lágrima. Ahí es donde nos los encontraremos.

--¿Y después? --dijo Cinder.

--Y después nos lo inventaremos sobre la marcha --respondió--. ¡Agarrate!.



Hizo lo que le dijo, agarrándose al borde de la consola mientras él devolvía a la TARDIS a la vida. Los motores rugieron cuando se zambuyeron en la Espiral Tantalus. En el lugar que una vez había sido su hogar.

El Doctor había dejado el techo transparente, y Cinder veía como los remolinos de niebla del Vórtice del Tiempo cambiaban de repente, dando paso a un nítido campo de estrellas.



--Ahora, sólo tenemos que esperar no atraer la atención de ningún Dal... --el Doctor se detuvo en seco cuando la TARDIS se estremeció como cogida por un disparo de rebote.

--¿Qué fue eso? --dijo Cinder.



El Doctor cogió un mando de la consola y lo retorcio haciendo un círculo, provocando que la vista a través de la cubierta se deslizase vertiginosamente, ofreciéndoles una perspectiva alternativa del espacio local. Cinco TARDISes de batalla blancas, similares a las que habían visto en el cementerio, pero repletas de un arsenal de armamento de un aspecto brutal, habían formado un anillo a su alrededor.



--Una emboscada --dijo el Doctor, sobriamente--. Nos estaban esperando, en cuanto nos materializamos. Debe haber otro dispositivo de seguimiento --miró a Cinder--. ¡Por supuesto! --dijo, recorriendo su frente con la parte inferior de la palma de la mano--. Debería haberlo visto.

--¿Visto el qué? --dijo Cinder, mirando el despliegue de TARDISes en la pantalla.

--¡Tú!. ¡Eres tú!.



Cinder dio un paso atrás, sintiéndose insegura. 



--¿Qué?. ¿Qué he hecho?.



El Doctor negó con la cabeza. 



--No, te deben haber plantado el rastreador durante ese asunto de la sonda mental.



Cinder no estaba segura de que ``ese asunto'' decribiera adecuadamente el tortuoso episodio al que había sido sometida por Karlax y el Gobernador. Tampoco le gustaba la insinuación de que de alguna manera la seguían utilizando para llegar hasta el Doctor. No tuvo tiempo de considerarlo, sin embargo, porque una voz familiar crepitó por el enlace de comunicación.



--Muy perspicaz, Doctor --dijo la voz fina y aflautada.

--Karlax --escupió el Doctor--. Tendría que haberlo sabido. Con unos pocos amigos de la CIA, sin duda.

--Naturalmente --respondió Karlax--. Debo decir, Doctor, que nos quedamos todos muy impresionados por la manera en la que pudo librarse de nuestros guardias. Tengo entendido que siempre ha sido muy difícil mantenerle en una celda.



El Doctor miró a Cinder con una expresión que decía ``te lo dije''.



--Aún así, poco importa --continuó Karlax--. El Comandante Partheus pronto llevará la Lágrima de Isha al Ojo. Usted y su compañera, por desgracia, se contarán entre los miles de millones de muertos.



En el monitor Cinder vio que una de las TARDISes de batalla sacaba lo que parecía un tubo lanzatorpedos. Les señalaba directamente a ellos. 



--Doctor --dijo.

--Lo sé --estaba de espaldas a ella.

--No, Doctor, realmente creo que necesitas...

--Lo sé --dijo, con más fuerza.

--¡Entonces haz algo!.



Hubo un estallido de luz en el extremo del tubo lanzatorpedos cuando la otra TARDIS disparó. En respuesta, el Doctor se tiró contra los controles, y la TARDIS descendió, cayendo en picado hacia abajo y dejando a las cinco TARDISes de batalla colgando en un ordenado círculo.

El torpedo se alejó nadando en el vacío, perdiendo luz. Momentos más tarde hubo un destello cuando detonó en el vacío sin causar daños. Por encima de ellos, el anillo de las TARDISes se agitó.



--Encuentra algo a lo que agarrarte --dijo el Doctor--. Esto va a ser un viaje lleno de baches.



Hubo una sacudida repentina cuando el Doctor manipuló los controles y la TARDIS cesó su caída libre y se hizo a un lado, girando como un sacacorchos, que hacía sentir a Cinder como si el corazón se le fuese a salir por la boca y el estómago estuviera en su cavidad torácica. Cerró los ojos, pero eso no hizo que el giro se sintiese mejor.

Dieron la vuelta, dejándose caer de nuevo, esta vez al revés, para esquivar la trayectoria de otro torpedo. El Doctor tiró de una palanca e hicieron tambaleándose un rizo, escalando hacia arriba, en un esfuerzo para sacudirse de encima a una de las TARDISes de batalla que habían caído tras ellos, pisándoles los talones a toda velocidad.



--Sabe que está perdiendo el tiempo --dijo Karlax por el enlace de comunicación--. Piense en ello. ¿No es mejor irse con elegancia, con dignidad, sabiendo que su tiempo ha acabado?.

--Eso suena más propio de usted, Karlax --respondió el Doctor--.Siempre dispuesto a darse por vencido cuando las cosas se ponen difíciles. Si me voy, me iré luchando.

--Que así sea --dijo Karlax, cortando la conexión.



La TARDIS de batalla que estaba detrás de ellos les estaba ganando terreno. Estaban dentro del alcance de disparo de sus armas, pero el Doctor estaba zigzagueando de un lado a otro, claramente haciendo que les resultase difícil el conseguir fijarles.



--¡Dispara! --bramó Cinder.

--¡No puedo!.

--¿Qué quieres decir con qué no puedes? --dijo incrédula.

--No tenemos armas --gritó el Doctor. El ruido de los motores chillando estaba ahogando todo lo demás.

--¿Por qué no?.

--La TARDIS, a ella no le gustan --respondió el Doctor. Estaba colgado en la consola con las dos manos, pero se inclinaba hacia atrás, como si tratara de empujar físicamente a la nave en una dirección diferente.

--¡Que no le gustan! --Cinder se habría llevado las manos a la cabeza si no hubiera estado agarrándose por su vida--. ¿Qué puedo hacer? --preguntó--. ¡Dime!.

--Sólo espera --dijo el Doctor--. Voy a intentar desmaterializarnos y escapar, ganar un poco de tiempo --tiró de los controles, justo cuando la TARDIS intentaba librarse de otro torpedo.



Esta vez el Doctor no tuvo oportunidad de esquivarlo y el delgado cilindro de plata se estrelló contra el costado de la cabina de policía, detonando con un halo de intensa luz blanca.

La sala de la consola se sacudió, haciendo que Cinder cayese sobre una rodilla, y después, de repente, todo se detuvo. Los motores suspiraron, las luces se apagaron, y por la vibración de las placas del suelo y la vista del techo, podía decir que se habían parado inmediata y completamente.



--¿Qué fue eso? --dijo.



El Doctor dio un puñetazo a la consola. 



--Un Torpedo de Tiempo. Estamos paralizados temporalmente en una burbuja de éxtasis. No nos podemos mover.

--Perfecto --dijo Cinder--. Ojalá me hubiera quedado en la cama.



Mientras miraban, una de los TARDISes de batalla se deslizó quedando a la vista, acercándose con la clara intención de abordarlos. 



--Este será Karlax --dijo el Doctor--. Queriendo pavonearse.

--¿No podemos impedir que suba a bordo? --preguntó Cinder.

--Podemos intentarlo --dijo el Doctor.



Cinder sintió un movimiento por el rabillo del ojo, y una fracción de segundo después la TARDIS de batalla explotó, detonando de repente, como si fuese alcanzada por un disparo desde atrás. La sala de la consola se estremeció con la réplica de la explosión. No pudo ver nada, ninguna señal de lo que había causado la explosión, mientras apresuradamente examinaba las vistas.

La TARDIS en ruinas parecía descomprimirse en el espacio delante de sus ojos, su interior se abría como las que había visto en el cementerio, hinchándose hasta abarcar toda su visión. Algunos objetos se alejaban en el vacío: monitores rotos, trajes espaciales, sillas.

El Doctor movió el mando de la consola y la vista cambió. Una formación de naves negras y elegantes se estaban ocupando de las otras cuatro TARDISes restantes, y los dos bandos estaban inmersos en una batalla campal, intercambiando disparos mientras volaban en círculos entre si en un baile rápido y violento.

Las naves negras parecían haber salido de la nada. 



--¿Qué son? --preguntó Cinder.

--Naves sigilosas Dalek --dijo el Doctor--. No aparecen en ninguno de los sistemas de monitorización de los Señores del Tiempo. Están al acecho en el Vórtice del Tiempo como cazadores acechando a sus presas, después atacan en el momento más oportuno.



Una de las TARDISes pareció disparar un misil en el flanco de una de las sigilosas naves, y detonó con una brillante explosión, barriendo el negro caparazón de la nave. La TARDIS intentó sacar provecho de su ataque, balanceándose a su alrededor para un segundo disparo, pero otra de las naves Dalek la barrió, desencadenando una lluvia de energía sobrecargada, que trituró a la TARDIS, aniquilándola en cuestión de segundos.



--Aquí somos un blanco fácil --dijo Cinder, con creciente pánico--. ¿No puedes hacer algo?.



El Doctor negó con la cabeza. 



--Aún tenemos la esperanza de que no les importe hacer frente primero a los blancos en movimiento --dijo, aunque se dio cuenta de que sus manos no se habían alejado mucho de los controles.



Una de las sigilosas naves estalló en una bola de fuego, atrapada en una lluvia entre las dos restantes TARDISes de batalla, pero no iba a ser suficiente. Simplemente había demasiadas naves Dalek. Cinder no había sido capaz de contarlas, pero el número era de dos dígitos, más del doble de las de los Señores del Tiempo. Fueron superados en todos los aspectos.

Casi al mismo tiempo, vio cómo morían las restantes TARDISes, sus dimensiones interiores dramáticamente al descubierto .

A Cinder le sudaban las manos. Sabía lo que vendría después. Las sigilosas naves convergerían en la TARDIS del Doctor, y en un momento, también serían reducidos a nada más que un cadáver hinchado, a la deriva en el vacío.

Observó que una de las naves Dalek se deslizaba por encima. El Doctor accionó un interruptor en la consola y el motor volvió silbando a la vida.



--Creia que habías dicho que estabamos congelados en una burbuja temporal --exclamó.



El Doctor se encogió del hombros. 



--No voy a caer en esa vieja trampa otra vez --dijo--. He actualizado el blindaje.

--Entonces... ¿acabas de ganar tiempo?.

--Precisamente --dijo el Doctor, golpeando el puño contra los controles. La TARDIS ascendió en espiral a una velocidad increíble, estrellándose contra la parte inferior de la sigilosa nave.



La puntería del Doctor no había sido del todo exacta, y le dieron en un lateral, haciendole un enorme agujero mientras la cruzaban. En el monitor vio a la otra nave girando fuera de control, una masa retorcida de metal torturado. Restos de gas ondulaban en el vacío, congelados instantáneamente formando nubes de hielo a la deriva.



--¡Rápido!. Salgamos de aquí --gritó Cinder--. Desmaterializate. Hay demasiados.



El Doctor tocó la pantalla con su dedo índice. 



--Todavía hay alguien vivo allí abajo.



Ella se movió, sin soltar la barandilla. En el monitor se podía ver el armazón de una de las TARDISes devastadas. Una figura diminuta se retorcía en agonía entre las ruinas de la sala de la consola.



--No me digas que estás pensando...

--Oh,claro que sí --respondió el Doctor. Tiró de una palanca y la TARDIS se desmaterializó por un breve segundo, volviéndose a formar entre los escombros de la TARDIS de batalla derribada. Las sigilosas naves se acercaban desde todas las direcciones.



Cinder intentó darle sentido a lo que estaba pasando. De repente, había restos por todo el suelo: trozos de coral, de pilar roto, fragmentos de una pared de color gris oscuro, la mitad de una consola rota, todavía burbujeando y explotando como descargas eléctricas. Entre todo eso, enclavado en un pozo de cables, yacía un Señor del Tiempo.

Iba vestido con una túnica escarlata y casquete, y se agarraba el cuello con ambas manos, luchando por respirar. De sus ojos, nariz y labios borboteaba sangre brillante, goteandole por la barbilla. Su carne estaba quemada y con ampollas, pero sus rasgos eran inconfundibles. 



--Es Karlax --dijo.

--Haz que esté cómodo --dijo el Doctor.

--Pero yo... --empezó.

--¡Sólo hazlo! --gritó, interrumpiéndola.



La TARDIS tembló mientras esquivaba haciendo piruetas otra descarga Dalek. 



--¡Maldita sea! --dijo el Doctor. Golpeó los controles y los motores gimieron, trasportándolos gradualmente dentro del Vortice--. ¡Maldita sea! --repitió.



Cinder estaba de rodillas, sosteniendo la cabeza de Karlax con sus manos. Estaba muy mal. Su respiración venía en cortas y sibilantes bocanadas. La exposición al vacío casi lo había matado, e incluso ahora, no estaba segura de que no fuera a hacerlo. Su piel había adquirido un brillo extraño, que parecía ir y venir, como si la luz estuviera moviéndose de alguna manera por debajo de la superficie de su carne.

No tenía ni idea de qué hacer. Ni siquiera sabía si los Señores del Tiempo tenían la misma fisiología que los seres humanos.

Cinder sintió al Doctor sobre su hombro. 



--¿Los Dalek? --dijo, sin levantar la vista.

--No nos encontrarán aquí --respondió el Doctor. Se agachó a su lado, poniendo una mano en la garganta de Karlax, tomándole el pulso--. Llegamos demasiado tarde --dijo--. Ha comenzado a regenerarse.



Karlax tosió, y una gota de sangre espesa y oscura le salió de la boca, derramándose en sus ropas.



--Ayúdame con él --dijo el Doctor. Deslizó las manos por debajo de los brazos de Karlax y 

lo incorporó hasta dejarle sentado, instigando una ronda explosiva de tos--. Cógele de los pies.



Cinder hizo lo que el Doctor le pidió y lo levantó, arrastrando los pies torpemente hacia los escalones. Karlax estaba sin fuerzas y pesaba más de lo que parecía. 



--¿A dónde le estamos llevando?. ¿A una sala de asistencia médica?.

--No --dijo el Doctor--.A la Habitación Zero.

--¿La Habitación Zero? --preguntó Cinder sin aliento mientras se esforzaba por impedir que los cuartos traseros de Karlax golpeasen el suelo. El Doctor subió los escalones hacia atrás, levantando la cabeza y los hombros de Karlax más alto.

--Un lugar en el que se puede regenerar en paz --dijo el Doctor--, y quizá lo más importante, donde no estorbará. Tiene una puerta que se puede cerrar con llave.

--¿Por qué le ayudas? --dijo Cinder--. ¿Después de todo?. Estaba tratando de matarnos. No se merece nuestra ayuda. Deberíamos haberlo dejado morir.

--Cuando nos conocimos, allí en Moldox --dijo el Doctor--, ¿te acuerdas de lo que estabas haciendo?.



Cinder frunció el ceño. 

--Luchando contra los Dalek --dijo.

--No, después de eso, cuando llegué.

--No sabía si confiar en ti --dijo--. Te amenacé con mi arma.

--Precisamente --dijo el Doctor--.Y no te dejé morir.



Cinder suspiró. 



--¿En serio me estás diciendo que es un malentendido?. Doctor, realmente trataba de matarnos.

--Sea como fuere, Cinder, todo el mundo merece una segunda oportunidad. Y Karlax está a punto de obtener una perspectiva completamente nueva. 



Llegaron al final de las escaleras y el Doctor les llevó por un pasillo hasta una puerta. La abrió de una patada y metieron a Karlax dentro.

La habitación estaba vacía, desprovista de muebles. Las paredes estaban cubiertas de los mismos redondeles brillantes que adornaban la sala de la consola. 



--Siéntalo aquí --dijo el Doctor. Lo tendieron en el suelo. Desconcertantemente, su pálida piel ahora estaba brillando con mayor intensidad que antes en las manos y la cara.

--¿Eso es la regeneración? --dijo Cinder.

--Sí, está llegando --dijo el Doctor--. Será mejor que le dejemos --la condujo fuera de la puerta, sacando inesperadamente una llave del bolsillo del pantalón y cerrando la puerta tras ellos--. Ya está --dijo--. Eso lo mantendrá ocupado durante un rato. Ahora, ¿dónde estábamos?.

--A punto de impedir que una flotilla de los Señores del Tiempo cometa un genocidio --dijo Cinder.

--¡Ah, sí! --dijo el Doctor, como si le hubiera recordado donde había dejado sus gafas de lectura--. ¡Será mejor que vuelva a ello!.



