\chapter*{Capítulo Catorce}
\addcontentsline{toc}{chapter}{Capítulo Catorce}

El Doctor y Rassilon volvieron a la cámara del Alto Consejo con el transmat, encontrándose a Karlax esperándoles. Estaba sentado en la mesa con una expresión de preocupación, las manos pinzandose bajo la barbilla.

--¡Ah, Doctor!. Estábamos preocupados por su paradero. Nadie parecía saber dónde estaba.

--Preocupados --repitió el Doctor--. Sí, me creo que estuviera preocupado, Karlax.

El hombre sonrió con una sonrisa forzada. 

--Ya veo que no teníamos necesidad de preocuparnos, dado que estaba en compañía del Lord Presidente.

Rassilon se bajó del podio del transmat. Su rostro era impasible. 

--Karlax, reúna al Consejo. Daré mis instrucciones inmediatamente --se giró hacia el Doctor--. Ya no se requiere su presencia, Doctor. Encuentre a su asistente y váyase.

--Está cometiendo un grave error, Rassilon --dijo el Doctor.

--Estoy tomando la única opción que puedo. No escucharé más sus insolencias. Ya me cansa. Váyase ahora, antes de que me vea obligado a silenciarle --dirigió una inmutable mirada y sus dedos se cernieron visiblemente sobre el mango de su bastón, subrayando sus palabras. El Doctor sabía que no era una amenaza vanal. Rassilon se enfadaba rápido y era aun más rápido en reaccionar.

Desafiante, el Doctor le devolvió la mirada. Después, con reticencia, dio la espalda al hombre. Parecía que se estaba quedando sin opciones. Evidentemente, ninguno de los miembros del Alto Consejo estaba preparado para atender a razones. Decidió que iba a tener que encontrar otro enfoque, otra forma de detenerlos. Pasara lo que pasara, no podía permitirles utilizar la Lágrima, incluso si significaba actuar contra ellos e intervenir en sus planes.

Sin decir nada más, se retiró de la habitación, dirigiéndose a la Sala de Observación para encontrar a Cinder.

--¿Qué habéis hecho con ella? -- gruñó el Doctor, irrumpiendo por las puertas de la Cámara del Alto Consejo. Su mandíbula estaba tensa, y estaba lleno de indignación e ira-- ¿Dónde está, Karlax?.

El Alto Consejo estaba, de nuevo, en sesión plenaria, y los Señores del Tiempo reunidos cesaron su parloteo para mirar al Doctor viniendo hacia ellos, mirando a Karlax, esperando una respuesta.

El asistente estaba de pie en el lado opuesto de la sala, contra la pared, justo detrás del hombro izquierdo de Rassilon.

--¿Su compañera, Doctor? --dijo Karlax, con fingida inocencia--. ¿No la dejó en la Sala de Observación esperándole mientras se ocupaba de su pequeño, y me siento obligado a añadir, no autorizado paseo?.

El Doctor golpeó la mesa. Había buscado en la galería de observación y en las habitaciones circundantes, y a Cinder no se la veía por ningún lado. Claramente le había pasado algo mientras había estado en la Zona Muerta. 

--No se haga el inocente, Karlax. Sé que está tramando algo. Ahora dígame, ¿dónde está?.

--Puedo decir honestamente, Doctor, que no tengo ni idea --dijo Karlax, con una sonrisa satisfecha. Se cruzó de brazos--. Si la ha extraviado, tal vez pueda considerar invertir en una correa más eficaz.

El Doctor respiró hondo. Sabía que Karlax estaba tras la desaparición de Cinder. Lo sabía. Estaba furioso consigo mismo por dejarla tan expuesta mientras había salido corriendo tras Rassilon a la Zona Muerta. Estúpidamente pensó que aquí había menos posibilidades de que resultara herida, aquí en el Capitolio de los Señores del Tiempo. Se suponía que eran civilizados. Era por eso por lo que viajaba solo hoy en día. La Guerra lo había cambiado todo, los había cambiado a todos, y no quería la responsabilidad. Ya no estaba seguro de poder protegerles.

Sin embargo, justamente así era Karlax. Era un oportunista. Vio su oportunidad y la cogió, llevándose rápidamente a Cinder como una forma de llegar al Doctor.

Apretó los puños, con tanta fuerza que sintió que sus uñas se clavaban en las palmas. 

--Se lo advierto, Karlax... --dijo.

El Gobernador indeciso se puso de pie. Tosió nerviosamente en su puño. 

--Yo sé dónde está, Doctor --dijo llanamente--. Se la mostraré -- empujó la silla hacia atrás, con sus piernas raspando toscamente sobre el mármol, y rodeó la mesa hasta que estuvo junto al arpa. Todos los ojos de la sala estaban puestos sobre él, y el Doctor notó que Karlax había dirigido al Gobernador una mirada particularmente amenazante.

El Gobernador se detuvo, miró a Rassilon, cuyas facciones permanecían impasibles, y entonces llegó al arpa. Sus dedos puntearon torpemente las cuerdas, sus manos temblaban. Estaba leyendo las notas detalladas en la pintura de la pared, recreándolas con la verdadera arpa. El Doctor comprendió lo que iba a suceder, lo había visto antes.

Después de un momento la melodía llegó a su fin, y el panel en la pared, justo detrás del pedestal en que el arpa descansaba se abrió para revelar una sala de control oculta. De un surtido de viejos paneles polvorientos y consolas parpadearon luces. Y allí, tendida sobre una silla, estaba Cinder.

El Doctor corrió hacia ella, pasando al Gobernador e irrumpiendo en la pequeña antecámara. Apenas estaba consciente, su cabeza se dejaba caer sobre su hombro izquierdo, de forma que su brillante pelo naranja caía en mechones sobre su cara. Tenía los ojos cerrados y su respiración era irregular.

Suavemente, el Doctor reposicionó su cabeza, quitando el pelo de su frente. Estaba pálida y fría, su piel estaba fría y húmeda al tacto. Sus párpados se agitaron, tratando de abrirse. Le comprobó el pulso, y suspiró de alivio cuando se dio cuenta de que todavía era fuerte y regular.

--¿Qué ha pasado? --dijo tiernamente--. ¿Qué te han hecho Cinder?.

Abrió la boca, pero todo lo que salió fue un murmullo indescifrable. 

--Ss... sss...

Se acercó más, poniendo el oído tan cerca de la boca que podía sentir su cálido aliento en la mejilla.

--S... sonda... mental --dijo, en lo que pareció un esfuerzo descomunal. Volvió a recostarse en la silla, con lo que parecían ser sus últimas energías.

El Doctor se enderezó, girándose lentamente para enfrentarse a los rostros expectantes en la otra habitación. Sintió el calor blanco de la furia creciendo dentro de su pecho. 

--¡Sonda Mental! --gritó, causando que el Gobernador, que seguía de pie junto al arpa, hiciera una mueca de dolor.

El Doctor irrumpió a la sala, yendo directamente hacia Karlax, que, al ver lo que venía, comenzó a rodear la mesa, tratando de poner una barrera entre él y el Doctor.

El Doctor no estaba interesado en jugar al ratón y al gato con aquel tonto servil, y así, en lugar de intentar perseguirlo alrededor de la mesa, tiró la silla del Gobernador fuera de su camino y saltó sobre la mesa, provocando jadeos sobresaltados del resto del Alto Consejo.

Enviando con cada paso documentos revoloteando al suelo, atravesó la mesa hacia Karlax, que ahora estaba atrapado en la esquina, sin ningún lugar para escapar. Se encogió cuando el Doctor saltó de la mesa.

Dos zancadas le pusieron directamente en frente del asistente y, sin perder su ímpetu, el Doctor extendió la mano, agarró al hombre por el cuello y lo empujó contra la pared, con tanta fuerza que gritó de dolor cuando la cabeza le golpeó contra el yeso.

--Digame por qué no debería estrangularle ahora, Karlax --ladró el Doctor. La saliva salpicaba el rostro de Karlax, y él se estremeció, sus ojos parpadearon de pánico.

--¿L ... Lord... Presidente ...? --tartamudeó, retorciéndose en las garras del Doctor.

El Doctor posó la mirada en Rassilon para evaluar su reacción. El Lord Presidente parecía totalmente desinteresado en lo que estaba sucediendo, como si estuviera simplemente esperando a que todo se calmase. Esto, en sí mismo, no hacía sino aumentar la ira del Doctor.

--Suplicando a su amo, ¿eh? --se rió el Doctor, volviendo su atención a Karlax--. ¿Ahora quien se está asfixiando con una correa?. Podría matarle antes de que tan siquiera le mirara de nuevo, sapo llorica.

--Pero, por supuesto, no lo hará --dijo Rassilon, a su espalda Oyó el tap-tap-tap de los dedos del guantelete de Rassilon en la superficie de la mesa. Una advertencia, sabía el Doctor. Ese guantelete tenía un poder inimaginable, incluyendo la capacidad de desmaterializar a una persona, igual que la nueva arma Dalek. Rassilon le estaba recordando dónde estaba.

El Doctor suspiró. 

--No. No lo haré --soltó a Karlax empujándolo al suelo, donde cayó de rodillas, agarrándose la garganta--. Pero confía en mí, Karlax, no dejaría una mancha en mi conciencia.

--Ahora, ¿ha terminado con su pequeña rebelión, Doctor? --dijo Rassilon--. Se está haciendo tedioso.

El Doctor se volvió hacia el presidente. 

--¿Sabía algo acerca de esto, Rassilon?. ¿Sabía lo que iban a hacer?.

Los labios de Rassilon se curvaron en una leve sonrisa. 

--Oh, no, Doctor. Todo fue asunto de aquí Karlax y el Gobernador, todo fue iniciativa suya. Los resultados, sin embargo, han sido de lo más esclarecedores.

El Doctor miró al Gobernador, que no quiso mirarlo a los ojos. 

--¡Podríais haberla matado! --dijo--. Es humana. Su mente no es lo suficientemente fuerte como para soportar la sonda. ¿Qué esperabais conseguir?.

--Es como usted ha dicho, Doctor --maulló Karlax, poniéndose de pie y quitándose el polvo de su túnica--. La perspectiva de ella resultó más valiosa. Ahora hemos podido corroborar su historia. Somos totalmente conscientes de la amenaza Dalek.

--¿Qué está diciendo? --dijo el Doctor.

--Que el Alto Consejo ha respaldado mi recomendación, Doctor --dijo Rassilon, poniéndose de pie--. Está a tiempo de ver cómo se da la orden --se volvió hacia su asistente--. Karlax, puede dar la orden. La Lágrima de Isha será utilizada.

--Sí, Lord Presidente --dijo Karlax, mirando al Doctor.

--Gobernador, diga al Comandante Partheus que prepare su flota --continuó Rassilon--. Él tendrá el honor. Llevará la Lágrima a las profundidades del Ojo de Tantalus, y con él, las esperanzas de todos los Señores del Tiempo, vivos, muertos y aún por existir. Vamos a asestar un duro golpe hoy a los Dalek. Conocerán la furia de los Señores del Tiempo.

--Y vosotros conoceréis la mía --dijo el Doctor en voz baja. No podía, no debía, permitir que esto sucediera. Tantas vidas, en tantos mundos. Tenía que haber una forma mejor.

--¿Doctor? --dijo Rassilon--. ¿Tiene algo que añadir?.

--Os voy a detener --dijo--. Entienda esto, Rassilon. Me niego a permitir que utilicéis la Lágrima de Isha.

--¿Se niega? --dijo Rassilon, con tono incrédulo--. Va a desobedecer directamente una decisión del Alto Consejo, del Lord Presidente?.

--Nada que no haya hecho antes --dijo el Doctor--. No significa nada --los miró uno a uno por turnos--. Estáis todos locos --dijo, exasperado--. Habéis olvidado quienes sois. Habéis permitido que la Guerra os convierta en seres desesperados y ciegos. Miráos todos, ocultos aquí con vuestras túnicas y tocados de fantasía, fingiendo que sabéis lo que realmente está pasando ahí fuera, diciéndoos lo tan condenadamente importantes que sois. Bueno, dejadme deciros la verdad: ¡os equivocáis!. Os equivocáis.

Señaló con el dedo a Rassilon. 

--Si le permitís hacer esto, cometer genocidio a esta escala, entonces somos tan terribles como los Dalek. ¿No lo véis?. Estáis tan obsesionados con vuestra propia y pequeña supervivencia que habéis perdido de vista el cuadro general. Si esto es en lo que los Señores del Tiempo se han convertido, entonces no merecemos sobrevivir.

La sala se quedó en silencio por un momento. El Doctor trató de recuperar el aliento. Rassilon fue el primero en hablar. 

--¿Debo entender, Doctor, que tiene la intención de actuar contra nosotros?.

El Doctor le miró a los ojos. Podía sentir todos los ojos de la habitación fijos en él. Echó un vistazo a Cinder, todavía semiinconsciente en la silla. 

--Sí --dijo, con una determinación de acero--. Si es lo que hace falta. Lo hago por vuestro propio bien, por el bien de los Señores del Tiempo. Estoy tratando de salvaros de vosotros mismos. El camino que está tomando, Rassilon, no conduce a la victoria. Si hace esto, será el final de los Señores del Tiempo. Pregunte a Borusa si duda de mí --añadió, amargamente.

Cruzó la habitación hacia Cinder. Era el momento de dejar Gallifrey, y dudaba que alguna vez volviera. No había vuelta atrás. Ya había tenido suficiente.

--Detenedle --dijo Rassilon--. A él y a la chica. Arrojadles a una celda y confiscadle su TARDIS.

El Doctor sintió que unas manos lo agarraban por detrás, retorciéndole el brazo detrás de la espalda. Luchó, pero fue en vano. El Gobernador era más joven y más fuerte, y era experto en obedecer órdenes, sin importar cuán desagradables fuesen. 

--Mejor aún --continuó Rassilon--, desmanteladla. Es una cosa vieja y decrépita y de ninguna utilidad para nosotros. El Doctor es un renegado y no se le permitirá interferir en nuestros planes. Una vez que la Lágrima de Isha haya neutralizado la amenaza Dalek, será juzgado y hallado culpable.

El Doctor oyó a Karlax llamar a más guardias. Era inútil oponer resistencia, al menos por ahora. Ya llegaría su oportunidad. Tenía que creer eso.

Mientras el Gobernador lo arrastraba lejos del sonido de la risa burlona de Karlax, el Doctor echó una última mirada a Cinder. No había querido la responsabilidad, pero a pesar de todo ahora había asumido esa carga. No sólo por Cinder, sino por toda su raza, todos esos billones de personas que se encontraban presos en los mundos ocupados de la Espiral. A juzgar por el actual estado de las cosas, estaban mejor con los Dalek que con su propio pueblo.

Los Señores del Tiempo estaban a punto de cruzar una línea de la que nunca podrían volver atrás y sólo él, un guerrero viejo y cansado,se interponía en su camino.

No iba a poder hacer mucho desde el interior de una celda.


