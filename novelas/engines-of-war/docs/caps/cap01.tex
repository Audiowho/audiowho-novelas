\chapter*{Capítulo Uno}
\addcontentsline{toc}{chapter}{Capítulo Uno}

Las patrullas Dalek habían sido esporádicas en los últimos tiempos, como si ya no se molestasen en explorar las ruinas periféricas. Se estaban concentrando en la ciudad, acorralando a cualquier superviviente que encontraban. Sus planes habían cambiado. Algo nuevo está ocurriendo.

Tal vez tendría que plantearse ponerse de nuevo en marcha. Y justo cuando empezaba a estar cómoda.

Cinder yacía boca abajo sobre el polvo y la suciedad, completamente inmóvil, observando el camino. Había oído que una patrulla Dalek iba a pasar por ahí, pero eso había sido hacía más de una hora. ¿Habría sido atacada por alguna célula de resistencia? Parecía improbable. Si lo hubieran hecho, ya lo sabría. Un mensaje habría zumbado a través del inter-comunicador. No, era más probable que los Dalek hubieran encontrado otro grupo de supervivientes y los estuvieran procesando para la esclavitud o “exterminándolos”, o, como ella prefería llamarlo, ``asesinándolos en el acto''. Cinder agarró el arma con fuerza, sintiendo una chispa de rabia ante esa imagen. Si se les ocurría venir por ahí...

Se apartó el flequillo de los ojos. Tenía el pelo castaño cortado de forma irregular hasta los hombros. Fue por esto por lo que se había ganado el sobrenombre de “Ascua”. Bueno, por eso y por el hecho de que había sido encontrada en las ruinas todavía ardientes de su casa, lo único que quedo vivo después de que los Dalek la hubieran atravesado.

Parecía que había pasado mucho tiempo desde que el planeta se había quemado. Cuando todos habían sido quemados. Cinder había visto como cada uno de los mundos de la Espiral había estallado hasta ser incandescente, iluminando el cielo por encima de Moldox, una hélice rotativa de orbes llameantes, una espiral de estrellas recién bautizadas.

Entonces no era más que una niña muy pequeña. Sin embargo, incluso a esa temprana edad había aprendido lo que significaba el fuego en el cielo para ella y su especie: los Dalek habían llegado. Toda esperanza estaba perdida.

Moldox había caído poco después, y la vida, si se podía llamar así, no volvió a ser la misma.

Su familia murió en los primeros días de la invasión, incinerada por una patrulla Dalek cuando trataban de huir. Cinder sobrevivió escondiéndose en un cubo de basura de metal volcado, mirando a través de un agujero pequeño de la carcasa oxidada, con miedo hasta de respirar. Tardó casi un año antes de sentirse lo suficientemente segura como para emitir otro sonido.

Días después, confundida y traumatizada, fue encontrada vagando entre las ruinas de su antigua casa y fue recogida por un grupo itinerante de combatientes de la resistencia. Sin embargo, esto no se trató de un acto de bondad por parte de sus compañeros humanos, sino que fue simplemente un medio para un fin. Necesitaban un niño entre sus filas para ayudar a poner trampas a los Dalek, para esconderse y escabullirse en los pequeños lugares donde los Dalek no podían entrar. Pasó los siguientes catorce años aprendiendo cómo luchar y cómo ganarse la vida en las ruinas. Más enfadada a cada día que pasaba.

Todo lo que había hecho desde ese momento había sido promovido por esa furia ardiente, por ese deseo de venganza.

Ella sabía que los años que había vivido no le habían sentado bien. Estaba delgada, a pesar de ser musculosa, su piel era pálida y perpetuamente manchada de tierra, y cada vez que encontraba tiempo para mirarse en un espejo roto o en un destrozado panel de vidrio, todo lo que veía reflejado era el dolor y el pesar en sus oscuros ojos de color verde oliva. Sin embargo, esta era su vida ahora: sobrevivir día a día buscando alimento y cazando Dalek siempre que se presentaba la oportunidad.

Al mismo tiempo, en el Universo, la guerra entre los Señores del Tiempo y los Dalek transcurría independientemente, destrozando todo el tiempo y el espacio a su paso.

Cinder había oído decir que en términos simples y lineales la guerra había comenzado hacía más de cuatrocientos años. Esto por supuesto, era mentira, o al menos, un dato irrelevante. Las zonas de guerra temporales habían penetrado tan profundamente en la estructura misma del Universo que el conflicto se había estado librando, literalmente, por toda la eternidad. No hubo época que hubiera permanecido indemne, ni historia que no hubiera sido reescrita.

Para muchos había llegado a ser conocida, tal vez irónicamente, como la Gran Guerra del Tiempo. Para Cinder, era simplemente el Infierno.

Ella cambió su peso corporal de un codo al otro, sin apartar la vista en ningún momento de la carretera de asfalto agrietado, en busca de signos, esperando. Ellos vendrían pronto, estaba segura de ello. Aquel mismo día había destruido otro de sus transpondedores y la patrulla que los otros habían descubierto debía haber sido enviada a investigar. Los Dalek eran tremendamente predecibles.

Recorrió con la mirada la hilera de edificios irregulares y destruidos que bordeaban el lado opuesto de la carretera en busca de Finch. Era su turno para abrir fuego contra los Dalek mientras ella los exterminaba desde detrás. No podía ver entre las ruinas. Bien. Eso significaba que estaba manteniendo la cabeza agachada. Odiaría si le ocurriese algo. Él era uno de los buenos. De los pocos que podía considerar un ``amigo''.

Las fachadas de los edificios en ruinas a lo largo de la carretera estaban ennegrecidas y astilladas, resultado tanto de los rayos de energía Dalek como de las bombas incendiarias de las fuerzas de defensa humana que trataban de mantener a raya a los invasores. En última instancia, habían fracasado frente a los enormes obstáculos y a un enemigo inquebrantable. Los Dalek eran absolutamente implacables, y en pocos días todo el planeta había quedado reducido a unas ruinas humeantes.

Cinder apenas podía recordar la época anterior a la llegada de los Dalek a Moldox. Tenía vagos recuerdos de relucientes torres y ciudades en expansión, bosques salvajes y cielos llenos de veloces naves de transporte. Aquí, en la Espiral Tantalus, los humanos habían alcanzado su cenit colonizando un vasto sacacorchos de mundos que rodeaban una inmensa y fantasmal estructura en el espacio, el Ojo Tantalus. En aquel preciso momento la miró, estudiando siniestramente los acontecimientos que se estaban produciendo.

Supuso que debió de haber sido testigo de algunos horrores en la última década y media. Moldox había sido una vez majestuosa, pero ahora no era más que un mundo moribundo que se aferraba miserablemente a los últimos vestigios de vida.

Escuchó un ruido en la carretera debajo de ella. Cinder se hundió aún más en la tierra y se arrastró hacia adelante unos centímetros, mirando por encima del borde del saliente con el fin de ver un poco más del camino. El asa de su mochila se le clavaba incómodamente en el hombro, pero ella lo ignoró.

Los Dalek llegaron finalmente, tal y como lo había previsto. Su pulso se aceleró. Ella entrecerró los ojos, tratando de discernir cuántos había. Pudo ver cinco formas distintas, pero su corazón se encogió mientras se acercaban, al conseguir distinguirlas.

Sólo uno de ellos era un Dalek, flotando en la parte trasera del pequeño grupo como si guiara al resto. Su carcasa de bronce brillaba en el sol menguante de la tarde, y su dispositivo ocular giraba de lado a lado, observando el camino que tenía delante.

El resto eran Kaled mutantes, Dalek en especie pero evolucionados en una nueva e inquitante forma por interferencia de los Señores del Tiempo. Eran Degradaciones de Skaro, el resultado de los esfuerzos de los Señores del Tiempo por reescribir la historia Dalek y jugar con la evolución de su especie originaria, probablemente en un intento de evitar el desarrollo de la raza Dalek. Sin embargo, los resultados fueron catastróficos, y en cada permutación de la realidad, en cada posibilidad, los Dalek habían sobrevivido. Nada les iba a detener. Daba igual lo mucho que lo analizase. Parecía que el universo quería a los Dalek.

Muchas de estas degradaciones eran inestables, impredecibles, y en opinión de Cinder, los hacía aún más peligrosos que a los Dalek. Y ahora estaban de servicio aquí, en Moldox.

Cinder preparó su arma, una pistola de energía arrancada de la carcasa rota de un Dalek moribundo y acoplado en un paquete de energía, y luchó contra el impulso de huir. Era demasiado tarde. Tenían que hacerlo Sólo esperaba que ninguna de las Degradaciones llevara un arma desconocida.

Cuando la patrulla estaba cerca, Cinder pudo verles con claridad. Dos de las Degradaciones eran casi idénticas y de un tipo que había visto muchas veces antes: un torso humanoide en una cámara de vidrio reforzado suspendido bajo una cabeza Dalek y un dispositivo ocular. Tres paneles alargados sobre brazos de metal negro flanqueaban esta columna central a los lados y atrás. Estaban salpicados con los mismos sensores semiesféricos de las carcasas Dalek estándar, y de cada lado sobresalían armas de energía montadas sobre bases estrechas.

Los torsos sin extremidades dentro de las cámaras de vidrio temblaron nerviosamente mientras entes monstruosos se deslizaban a lo largo, impulsándolos por el aire en columnas de luz azul. Finch les había apodado “Planeadores”.

Los otros, sin embargo, no se parecían a nada que hubiera visto antes. Uno de ellos tenía forma de huevo y se erigía sobre tres extremidades en forma de araña, arrastrándolos a lo largo de la carretera como un enorme y terrorífico insecto. Una vez más, su carcasa tenía las mismas y familiares semiesferas, aunque en este caso eran del color del carbón y estaban engarzadas en paneles de un profundo rojo metálico. El dispositivo ocular era más grueso y de su carcasa salían cuatro bases para armas.

El mutante final parecía ser casi idéntico a un Dalek, exceptuando que su sección central, que normalmente alojaba el brazo manipulador y la pistola, había sido reemplazada por una torreta giratoria, sobre la cual se montaba un único cañón de energía masiva.

Cinder intentó tragar saliva, pero tenía la boca seca. De ninguna manera podía correr el riesgo de que el cañón disparase. Los resultados serían devastadores y Finch no tendría posibilidad de conseguir ángulo de tiro. Ese tenía que ser su primer objetivo.

Sintió un movimiento en las ruinas, y con un vistazo rápido supo que Finch ya estaba en movimiento, corriendo de escondite en escondite para llamar la atención del Dalek. El Dalek lo sintió también, y su dispositivo ocular giró en su dirección.



—¡Detente! ¡Muéstrate! ¡Ríndete y no serás ex-ter-mi-na-do! —la voz áspera y metálica del Dalek y su eco a lo largo de la desierta carretera produjeron un escalofrío en la espalda de Cinder. Ella buscó a Finch con la mirada, tratando de encontrarle entre las ruinas y anticipar su próximo movimiento. Por ningún motivo obedecería la orden del Dalek. Aunque no estuviera mintiendo, el exterminio era una mejor alternativa a ser esclavizados por esos monstruos.



¡Ahí! Le vio moverse de nuevo, cerca de los restos de una casa quemada, y el Dalek se giró, disparando tres veces con su arma. El estruendo de la descarga de energía fue ensordecedora. Se produjo un intenso destello de luz blanca, seguido por una explosión, y restos de una pared dañada cayeron en un montón, cerca de donde Finch se había escondido sólo unos segundos antes. El humo se elevó perezosamente entre las ruinas.



—Buscar. Localizar. ¡Destruir! —ordenó el Dalek—. Buscar al humano y ex-ter-mi-nar-lo.

—Obedecemos —entonaron a coro las Degradaciones con sus voces sintéticas. Los dos planeadores se elevaron sobre haces de luz mientras que los demás se dispersaron cubriendo las ruinas con sus armas.

La patrulla se había separado y Cinder vio su oportunidad. Se levantó sobre sus rodillas, levantando el arma Dalek sobre su hombro y mirando a lo largo del cañón dentado. Apuntó al jefe de la Degradación con el cañón, respiró hondo, y disparó.

El arma emitió una breve ráfaga de energía de gran alcance, y la fuerza de la descarga estuvo a punto de tirarla al suelo. Mantuvo su hombro en posición, estabilizándose. El aire se llenó del olor del ozono quemado.

Su puntería fue certera, y el haz de energía atravesó del caparazón de bronce del mutante, marcando un surco negro profundo y detonando una de sus válvulas de radiación. No obstante, no tuvo el efecto deseado de hacer explotar la cabeza de manera espectacular. En su lugar, provocó una respuesta menos bienvenida.



—¡Nos atacan! ¡Nos atacan! —bramó la Degradación, girando su cabeza 180º grados para escanear la zona donde se encontraba ella—. Mujer humana armada con neutralizador Dalek. ¡Exterminar! ¡Exterminar! 

Presa del pánico, Cinder miró el arma en sus manos. ¿Qué había salido mal? Nunca había visto a un Dalek sobrevivir a una explosión de energía de una de sus propias armas. ¿Tenía esta nueva especie de mutante armadura reforzada? En cualquier caso, todo lo que había logrado estaba revelando su propia ubicación.

Tenía que actuar con rapidez y eliminar al líder. Se giró, levantando el arma y cerró el ojo izquierdo, dibujando una línea de visión en el Dalek, mientras éste cambiaba de posición, preparándose para responder al fuego. Apretó el gatillo y el arma improvisada escupió otro rayo de energía ardiente.

El tiro dio en el blanco, golpeando al Dalek justo debajo del dispositivo ocular. La carcasa detonó de manera satisfactoria, rompiendo las rejillas del sensor y derramando la biomasa del Dalek muerto. Las llamas lamían los bordes de la herida abierta mientras una pulpa verde burbujeaba y rezumaba con un grotesco siseo.

Cinder no tuvo tiempo para celebrarlo, ya que la Degradación en forma de huevo abrió fuego de nuevo. Sus cuatro armas dispararon en rápida sucesión, como piezas charlatanas de artillería, batiendo el suelo cerca de su ubicación. Se echó hacia atrás, rodando por la cubierta, pero ya era demasiado tarde. El impacto había desestabilizado el suelo y el borde del acantilado se derrumbó en un corrimiento de barro y tierra.

Cinder sintió que el mundo cedía bajo sus pies. Gritó mientras agarraba con fuerza su arma al tiempo que caía de cabeza hacia las Degradaciones reunidas abajo.



