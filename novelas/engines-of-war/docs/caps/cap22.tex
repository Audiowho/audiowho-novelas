\chapter*{Capítulo Veintidos}

\addcontentsline{toc}{chapter}{Capítulo Veintidos}



Cinder recorrió con una mirada amplia y atónita todo a su alrededor. Después, sin demasiada convicción, se dio palmaditas en el pecho para asegurarse de que era real y que esto no era, de hecho, un sueño extraño pero enormemente creíble.



Estaba de pie en la TARDIS junto al Doctor, justo en la entrada. Un minuto antes estaba rodeada de Dalek, preparándose para morir y al minuto siguiente la nave simplemente se materializó a su alrededor, arrancándolos de las garras de los Dalek. Desde donde estaba sólo podía distinguir el  monitor, que mostraba a los Dalek pululando por el exterior, sin duda ladrándose ordenes unos a otros, tratando de determinar con precisión lo que el Doctor había hecho.



Se sobresaltó cuando se dio cuenta de que un hombre desconocido estaba de pie en la consola. Era moreno y musculoso, con el pelo rapado y llamativos ojos azules. Iba vestido con las ropas de un Señor del Tiempo, pero le estaban holgadas y estaban cubiertas de manchas oscuras de sangre.



--Pero… pero… ¿cómo? --dijo, volviendo la vista hacia el Doctor.



--Karlax --respondió el Doctor.



El hombre de la consola se burló.



“¿Karlax?. ¿Entonces ese era Karlax?.” Conocía la regeneración de un Señor del Tiempo, por supuesto, pero verlo por sí misma, ver la evidencia de un cambio tan profundo, hacía que su mente diera vueltas. Le miró boquiabierta. 



--Entonces, ¿fue usted quién nos salvó? --se preguntó si quizá el cambio había sido más profundo que meramente estético. ¿Ese nuevo Karlax había desarrollado una conciencia, así como un nuevo rostro?.



--No por propia elección --dijo Karlax, disipando inmediatamente esa teoría.



--Abrí la puerta de la Habitación Zero antes de dejar la TARDIS --dijo el Doctor--. Sabía que, dándole esa oportunidad, Karlax intentaría escapar. Así que me aseguré de que la trayectoria de vuelo estableciese como casa el dispositivo de rastreo que te había plantado allí en Gallifrey. En cuanto la puso en marcha…



--Se abalanzó a nuestro rescate --terminó Cinder.



--Y ya es hora de terminar el trabajo --dijo el Doctor--. De coger a Borusa en el Ojo --empezó a avanzar, pero se detuvo cuando Karlax dio la vuelta desde detrás de la consola, agarrando firmemente una pistola.



--No es así como va a acabar, Doctor --dijo.



El Doctor dejó caer los hombros bruscamente. Más que otra cosa, se le veía abatido, como si hubiera esperado más de Karlax, como si hubiese tenido la esperanza de que tal vez las cosas no fueran a desarrollarse por ese camino.



--Te salvó la vida, Karlax --dijo Cinder--. Te arrastró de entre los escombros de tu dañada TARDIS cuando podía haberte abandonado para que murieses allí.



--Siempre fue un viejo tonto sentimental --dijo Karlax--. Nunca fue capaz de comprender cuando tenía una ventaja táctica.



--Karlax. Baja el arma --dijo el Doctor razonablemente--. Tengo un plan para detener a los Dalek. Podemos terminar con todo esto una vez eso esté fuera de aquí.



Karlax negó con la cabeza. 



--Es un criminal fugitivo --dijo--. Ha traicionado a los Señores del Tiempo. Es un traidor. No puedo confiar en nada de lo que diga.



Cinder podía oírlo en la voz de Karlax, había perdido totalmente la razón. Si se trataba de un síntoma de su reciente regeneración, o, simplemente, que finalmente tenía la oportunidad de lograr vengarse del Doctor, no podía asegurarlo. De todas maneras, el temblor agudo de su voz, la emocionada mirada que sus ojos estaban lanzando, las gotas de sudor que le caían por la frente. Todo eso señalaba el hecho de que no estaba en su sano juicio. Lo que, para opinión de Cinder, le hacía incluso más peligroso de lo habitual. Era impredecible.



--Esperando complacer a su amo, ¿verdad? --dijo el Doctor con la voz llena de cinismo--. ¿Esperando que le dará palmaditas en la cabeza y le dirá lo bien que lo ha hecho?. Le contaré un secreto, Karlax, no le importa. No está interesado en usted y su pequeña y llorica existencia. Le encuentra útil. Eso es todo. Cuando se muera, le reemplazará sin dudarlo un instante. Probablemente ya lo habrá hecho.



--Cállese --dijo Karlax amargamente.



El Doctor apuntó con un dedo a Karlax, como si le regañase. 



--Es más, incluso si obtiene lo que quiere, va a ser una vana victoria. Los Dalek están yendo hacia Gallifrey. Rassilon no le agradecerá haberles permitido utilizar su planeta.



Cinder vio que el dedo de Karlax se contraía en el gatillo de la pistola. Podía ver lo que el Doctor estaba tratando de hacer, socavar la determinación de Karlax, pero no funcionaba. Ya había ido demasiado lejos.



--Creo que ya ha habido suficiente charla por ahora --dijo Karlax agitando la pistola.



--Bien --dijo el Doctor--. Es hora de continuar --se movió hacia la consola. Cinder vio la mirada de puro odio en los ojos de Karlax, el repentino y brusco movimiento cuando levantó la pistola y apretó el gatillo.



--¡No! --gritó, saltando hacia el Doctor y derribándolo, quedó tendido en el suelo, maldiciendo por la sorpresa.



El rayo de energía le dio justo debajo de las costillas y cayó pesadamente al suelo, gritando por el dolor punzante. Rodó sobre su espalda, buscando con desesperación la herida con las dos manos, tratando de aplicar presión. La sangre palpitaba, caliente y húmeda, brotando entre sus dedos mientras hacía todo lo posible por mantenerla dentro. Se quedó sin aliento y después se estremeció por el lacerante dolor. Su pulmón derecho se estaba llenando de sangre.



El Doctor se puso de rodillas y de repente estaba a su lado, sosteniéndole la cabeza en el regazo. 



--¡Cinder!. ¡Cinder!. Aguanta. Va a estar bien. Todo va a estar bien.



Cinder abrió la boca para hablar, pero brotó sangre de sus labios, goteándole por la barbilla. El dolor la estaba consumiendo. Mordió con fuerza, apretando los dientes, obligándose a mantenerse despierta, para luchar contra la paulatina oscuridad que le nublaba la vista.



Oyó reír a Karlax y levantó la vista para verlo de pie sobre ellos con su pistola ahora colgando lánguidamente entre sus dedos. 



--Oh querido, Doctor. Su mascota parece haberse herido. Siempre ha sido demasiado aficionado a estas criaturas humanas.



El Doctor gruñó con una inarticulada ira y Karlax dio un paso atrás, sorprendido, pareciendo momentáneamente inseguro, antes de recuperar la compostura. 



--Bueno, supongo que he comenzado el trabajo --dijo. Levantó su arma, mirándola, como cautivado por su poder.



--¿Por qué Karlax? --gruñó el Doctor.



--Porque tengo mis órdenes --respondió Karlax.



El Doctor negó con la cabeza. 



--No es suficientemente bueno --dijo. Mirando hacia arriba con ojos de sueño, Cinder vio que el Doctor miraba la consola a pocos metros de distancia.



La respiración de Cinder era ahora superficial y cada inspiración sacudía su cuerpo de dolor. Podía sentir como se iba desvaneciendo.



--Es lo suficientemente bueno para mí --dijo Karlax.



Lo que sucedió a continuación a Cinder le pareció que ocurría como una serie de imágenes congeladas y vacilantes. Karlax levanto el arma para apuntar, mientras el Doctor torció bruscamente a la izquierda, alcanzando con los dedos los controles de la TARDIS. Se giró, golpeando con el puño hacia abajo la palanca de desmaterialización, justo cuando Karlax conseguía  hacer su segundo disparo.



Los motores aullaron y la columna central se iluminó y se impulsó. El haz de energía del arma de Karlax golpeó la consola, duchando al Doctor con chispas y haciendo que se tambaleara hacia atrás, maldiciendo cuando las diminutas y brillantes estrellas le quemaron el dorso de la mano.



La vista de Cinder iba y venía. No podía estar segura de lo que estaba sucediendo. El dolor estaba remitiendo ahora, convirtiéndose en una niebla distante. Se sentía entumecida y fría, pero Karlax parecía estar… desvaneciéndose de alguna manera, como si se estuviera desmaterializando en vez de la TARDIS.



--¡No, Doctor!. ¡No puedes abandonarme aquí!. ¡No con los Dalek! --la voz de Karlax era chillona y suplicante. Dejó caer el arma, que resonó en el suelo de la TARDIS.



--No es más que lo que merece, Karlax --dijo el Doctor.



Con un shock, Cinder se dio cuenta de lo que estaba sucediendo, la TARDIS se estaba desmaterializando alrededor de Karlax, dejándolo allí en medio de todos esos Dalek enfadados. Abandonándole para que muriese.



Karlax dio un paso hacia delante, con una expresión llena de consternación. La carne de su rostro había adquirido una extraña y pulsante translucidez. Abrió la boca pero no salió ningún sonido. Levantó las manos en alto, mirando a su alrededor con horror al ver a unos Dalek que sólo él podía ver. Unas luces destellaron a su alrededor cuando los Dalek abrieron fuego, su rostro se retorció de dolor, y después, tan repentinamente como si nunca hubiera estado allí, parpadeó hasta desaparecer, abandonado en la estación Dalek mientras la TARDIS se esfumaba en el Vórtice del Tiempo.



Con un gruñido el Doctor volvió tambaleándose hasta donde Cinder yacía y cayó de rodillas a su lado. Apartó a Cinder el pelo de la cara. 



--Aguanta aquí. Encontraré una manera --su voz era un suave susurro, como el viento alborotando las ramas de los arboles en la primavera.



--No --dijo con voz ronca, con esfuerzo. Sus palabras eran débiles murmullos, y el Doctor tuvo que acercarse más para oírla--. Ya es demasiado tarde.



La miró fijamente. 



--No te rindas --dijo. Su bigote se crispaba--. Nunca debes darte por vencida --parecía preocupado y había lágrimas formándose en las comisuras de sus ojos.



--La pared de la cueva. La pintura --dijo entre jadeos cortos y superficiales.



El Doctor negó con la cabeza. 



--No. Eso era sólo una de las muchas posib… --paró de ofrecerle perogrulladas--. Quédate quieta --dijo rozándole la mejilla--. Te dolerá menos.



--Tu dijiste que sólo me entrometía --dijo forzando una sonrisa.



--Oh, Cinder. Pero lo hiciste con tanto estilo --cogió su mano y la apretó--. Gracias --le volvió la espalda un momento, pero después se obligó a volver la vista atrás, encontrándose con su mirada--. ¿Por qué tuviste que hacer eso?. ¿Por qué tuviste que ser tan condenadamente… humana?.



Cinder intentó encogerse de hombros, pero el gesto le dolía demasiado. 



--Mi vida a cambio de las de billones --dijo, haciéndose eco de las anteriores palabras del Doctor--. Aceptaré esa apuesta.



Tosió y la boca se le llenó de sangre. Cerró los ojos. Se sentía tan cansada, y el Doctor acariciando su frente era tan relajante. Tal vez si sólo se permitiese dormir un rato…



Cuando abrió los ojos de nuevo, un momento después, estaba de vuelta en Moldox. Era una niña de seis años retozando en los jardines en el exterior de su granja. Su hermano estaba jugando cerca en los columpios, gritando con deleite mientras se impulsaba más y más alto. A través de la ventana podía ver a su madre y a su padre preparando la comida.



El sol le resbalaba por la cara, cálido y tranquilizador. Por primera vez en años, se sentía feliz.



















