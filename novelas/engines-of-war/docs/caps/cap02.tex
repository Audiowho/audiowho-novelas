\chapter*{Capítulo Dos}
\addcontentsline{toc}{chapter}{Capítulo Dos}

Muy por encima de Moldox, una caja azul doblaba la realidad, deslizándose sin esfuerzo fuera del vórtice del tiempo. Parecía incongruente, en los bordes exteriores de la Espiral Tantalus, una reliquia de la antigua Tierra que había caído a través del Tiempo y del Espacio, sólo para aparecer aquí, con su luz superior parpadeando violentamente mientras volvía a su forma corpórea. Si el sonido se hubiera transmitido en el espacio, su aspecto habría sido acompañado por una respiración trabajosa y áspera, pero en su lugar, sólo había silencio.
La llegada de este objeto anacrónico, sin embargo, no pasó desapercibida, y la aparición de la TARDIS encendió todos los pilotos de advertencia en un millar de paneles de control Dalek. Platillos Dalek se pusieron en movimiento, deslizándose a través del vacío para adoptar formaciones de combate, con las luces parpadeando mientras ponían todos los sistemas a pleno rendimiento.
Dentro de la TARDIS, el Doctor, o más bien el Señor del Tiempo que antes de ahora había vivido muchas vidas bajo ese nombre, giró un dial y dió un paso atrás desde la consola. Cruzó las manos detrás de la espalda, y esperó.
A su alrededor, los círculos de las paredes brillaban con una luminiscencia débil, haciendo que las escarpadas líneas de su cara se marcaran con sombras: el mapa de cien o más años, desgastado por el conflicto y el cansancio.
La columna central sonaba suavemente, mientras subía y bajaba, como si la máquina estuviera de alguna manera respirando: inspirar y espirar, inspirar y espirar. El pensamiento era reconfortante. Eso significaba que no estaba solo. Suspiró y levantó la vista hacia el campo de estrellas que se proyectaba a través del techo transparente de la sala de la consola. Por encima de él se situaba la forma etérea del Ojo Tantalus.
El Ojo era una anomalía, una vasta región en el Espacio-Tiempo, una estructura imposible que no tenía derecho a existir, y sin embargo, lo hacía. Cómo se había formado o si era natural o construida, nadie había sido capaz de discernirlo. Todo lo que el Doctor sabía era que era anterior a los Señores del Tiempo, y que Omega, el Gran Ingeniero, en esos primeros días felices de la Diáspora, había escrito sobre el Ojo y sus muchos secretos. Secretos que permanecían indescifrados hasta este día.
Desde tan lejos, en el borde de la Espiral, tenía el aspecto de un inmenso cuerpo gaseoso, un ojo humano en remolino, rodeado por una hélice de mundos habitados. Se erguía con la luz mortecina de los gigantes que mueren y los granos de las nuevas y hambrientas estrellas, recién renacido en un ciclo interminable de muerte y resurrección de los cuerpos celestes atrapados dentro de su horizonte de sucesos y la influencia de sus temporales sacudidas.
Para el Doctor, era absolutamente impresionante. Había venido aquí a menudo en sus otras vidas, en particular su cuarta y su octava, aquellas con un comportamiento más romántico,  aunque ahora esos días eran como recuerdos lejanos, sueños que le hubieran sucedido a otra persona. Ahora, no había nada más que la Guerra. Le había consumido, convirtiéndolo en algo nuevo. Un guerrero.
Al igual que al Doctor, la Guerra había cambiado también la Espiral Tantalus. Una vez fue un remanso de paz, ahora estaba arruinada por la ocupación Dalek. Se había convertido en una zona de guerra, al igual que gran parte del Universo, un punto de parada desde el que los Dalek podían continuar su cruzada para llenar la eternidad con sus progenitores y emprender su incesante campaña contra los Señores del Tiempo. Por eso el Doctor había venido a la Espiral, los Dalek se estaban concentrando aquí, y él necesitaba medir sus fuerzas. Había una manera simple y efectiva de hacer precisamente eso.

--Muy bien --gruñó--. Venid a por mí.

Por encima de la TARDIS, los platillos Dalek comenzaron a converger. No estaban todavía en el rango de sus armas de energía, pero el Doctor sabía que en cualquier momento podía esperar un aluvión de disparos. Dio un paso hacia adelante y tomó el control de nuevo.

--Espera --murmuró para sí mismo--. Espera el momento exacto...

Movió un interruptor y se abrió el canal de comunicación. Cien o más voces Dalek estaban gritando en una cacofonía desordenada. Sus palabras eran apenas perceptibles, pero sabía muy bien lo que estaban diciendo: ¡Exterminar!. ¡Exterminar!. Incluso ahora, ese sonido le puso la piel de gallina.
Se estaban acercando. Aún así, el Doctor esperó. El primer platillo se colocó finalmente a tiro, moviéndose sobre su cabeza.

--¡Ahora! --bramó el Doctor con todo el aire de sus pulmones, moviendo una palanca hacia adelante y agarrándose de los bordes de la consola hasta que los nudillos se le pusieron blancos de la fuerza.
La TARDIS se disparó hacia arriba como un cohete. Cogió al platillo completamente desprevenido, chocando con su vientre en forma de cúpula con incrustaciones y rasgándolo a una velocidad inmensa, saliendo a través de la parte superior de la nave y escindiéndola, girando sobre su eje.
Todo el sistema eléctrico del platillo silbaba y explotaba, visible a través del irregular agujero. Escoró, fuera de control, con sus armas disparando indiscriminadamente. Un haz de energía alcanzó a un platillo vecino, mientras que la nave averiada fue girando hacia otro, que resultó ser demasiado lento como para tomar una acción evasiva.
En sus monitores, el Doctor observó los escudos dañados de los Dalek flotando lejos sin moverse en el vacío mientras las naves ardían.

--Eso ya está hecho, vieja amiga --dijo, manipulando los controles una vez más para hacer pivotar a la TARDIS fuera de la trayectoria de otro arma de energía. Los platillos Dalek se desplazaban como una bandada de pájaros, en picado detrás de él, con sus cañones escupiendo muerte a su alrededor. 

--Así, así --dijo--. Seguidme...

Al igual que el piloto de un avión de acrobacias, que él había podido observar con el Brigadier, en sus días con UNIT en la Tierra, el Doctor sumergía y ondeaba la TARDIS, izquierda, derecha, arriba, abajo, serpenteando a través del vacío, llevando a los Dalek en una divertida persecución, pero siempre estando un paso por delante de sus armas.
Al mismo tiempo, el brillo maléfico del Ojo los miraba impasible.

--De acuerdo, no es hora de ... --el Doctor se interrumpió, sonriendo, mientras un centenar o más de TARDIS de batalla salían del Vortex detrás de la flota Dalek--. Ahora os tenemos --alardeó, girando una palanca e inclinando la TARDIS sobre sí misma para que pudiera pasar como un rayo bajo la oleada de platillos Dalek que se aproximaba y así unirse a sus compañeros.

Las armas se transmutaron en la piel exterior de las TARDIS de batalla, blancas y planas pastillas de color blanco con una capa exterior de metal vivo que podrían transformarse en escudos, o en cualquier número de armas predeterminadas. Las TARDIS se dispersaron, disparando en cien direcciones diferentes mientras los Dalek intentaban revertir su curso, para enfrentarse al enemigo que los había desbordado con tanta facilidad.
Torpedos temporales fueron lanzados en oleadas, una veintena de ellos encontraron su objetivo y lo congelaron, atrapándolo en un patrón de espera temporal, un segundo de bloqueo del que los platillos no podían escapar. Las naves Dalek explotaron en silenciosas bolas ardientes mientras los Señores del Tiempo seguían con su andanada de explosivos.
Los Dalek, sien embargo, no se retiraban, y mientras la TARDIS del Doctor entró por la superficie de otro platillo, para enviarlo hacia uno de los planetas cercanos, se las arreglaron para liberar su primera descarga, detonando TARDIS con cada ataque.
El Doctor observó cómo las máquinas del tiempo morían floreciendo, con sus dimensiones interiores desdoblándose en la realidad, desplegándose como violentas flores hasta alcanzar su verdadero tamaño antes de estallar en el vacío. Sus dedos bailaban a través de los controles y la TARDIS se alejó, justo cuando las naves Dalek escupían una segunda descarga.

--¡Fase! --bramó sobre la plataforma de comunicaciones, y los Señores del Tiempo hicieron lo que les había mandado, sus TARDIS parpadearon repentinamente fuera de la existencia. Aparecieron de nuevo un momento después, tras haber saltado dos segundos en el futuro para evitar los crepitantes rayos de las armas Dalek, que se desvanecieron en el espacio sin causar daño.
Su respuesta fue mucho más eficaz, detonando un sinnúmero de platillos Dalek.

--¡Retirada!. ¡Retirada! --el coro de voces Dalek, ahora disminuído pero todavía audible, había cambiado. Estaban tratando de reagruparse, retrocediendo hacia el Ojo y usando los restos de sus hermanos caídos como cobertura.

--¡Los tenemos en la trayectoria, Doctor! --gritó una satisfecha voz femenina por el comunicador.

--¡Quédese con ellos! --respondió él --. ¡Presione la vanguardia!.

Los Señores del Tiempo, que ahora superaban en número a las naves Dalek en proporción de dos a uno, hicieron precisamente eso, surgiendo hacia adelante, algunos hacia arriba, otros hacia abajo, atrapando entre ellos a los Dalek que se batían en retirada.
Los torpedos temporales hicieron su trabajo, interrumpiendo la retirada Dalek, y en cuestión de segundos, el espacio por encima del Ojo Tantalus estaba lleno de los restos de la flota Dalek. 

--Bien hecho, Doctor --dijo la voz de mujer en el comunicador. Sonaba jubilosa. Era la Capitana Preda, Comandante de la Quinta Flota de Batalla de los Señores del Tiempo--. Los llevamos a un baile verdaderamente agradable.

--No cuentes tus victorias antes de tiempo, Preda --respondió el Doctor con tono sombrío--. No estoy seguro de que haya terminado todavía. Podría haber más de ellos, al acecho en la sombra de esos planetas.

--Entonces vamos a echar un vistazo --dijo Preda. El intercomunicador zumbó al apagarse, y las TARDIS de batalla, se acoplaron en una formación de punta de lanza, que se deslizó hacia el Ojo Tantalus. Con cautela, el Doctor fue detrás de ellos, manteniendo un ojo en sus monitores.

La emboscada llegó sin previo aviso. No había sonado ninguna alarma, ni hubo indicios de que algo anduviera mal, ni de que hubieran desencadenado una especie de trampa. En un segundo no había nada, y al segundo siguiente una armada de sigilosas naves Dalek había parpadeado fuera del Vórtice.
El Doctor había visto estas naves sólo un puñado de veces antes, lisas, ovoides y del negro más puro, desprovistas de las habituales luces parpadeantes que normalmente marcan un platillo Dalek, y dos veces más peligrosas. Eran un desarrollo reciente y desagradable. Estaban colocadas en el Vórtice del Tiempo como arañas en el corazón de una tela, detectando las vibraciones de las TARDIS que pasaran. Sólo entonces se daban a conocer, apareciendo para atrapar a los desprevenidos Señores del Tiempo.
Eran elegantes y mortales y, el Doctor se dio cuenta, Preda y su flota acababan de ser atrapados en su tela.
Los Señores del Tiempo no tuvieron tiempo de reaccionar. Ni uno solo fue capaz de desmaterializarse antes de que las armas Dalek los resquebrajaran como a latas, derramando sus entrañas en el frío vacío del espacio.
El Doctor rugió, golpeando sus puños en los controles y enviando la TARDIS hacia un lado en una acción evasiva que le salvó la vida. Sin embargo, la TARDIS recibió un golpe de refilón en su flanco derecho y fue enviada en un giro salvaje. Con los estabilizadores sin poderlo compensar, el Doctor golpeó contra el suelo de la TARDIS, rodando fuera de la tarima central mientras la nave se sacudía.
La TARDIS, fuera de control, se precipitó de cabeza hacia uno de los planetas cercanos.

