\chapter*{Capítulo Veinte}

\addcontentsline{toc}{chapter}{Capítulo Veinte}

Cinder miró el motor de posibilidad con incomodidad. El Doctor había dirigido a los Intersticiales a apoyar la estructura de forma vertical entre dos pilares de piedra al otro lado de la consola, y en lo que consideró un movimiento sorpresa, había arrancado dos de los cables que colgaban de sus tomas de corriente en el techo y utilizó sus extremos para asegurar la estructura en su lugar. Esperó que no hicieran nada importante.

Ahora Borusa estaba en posición vertical, con sus muñecas y tobillos todavía atados a la estructura. Su barbilla se había hundido para descansar en su pecho, pero aún podía ver la intermitente luz azul bailar tras sus ojos.

Los Intersticiales se habían retirado tras depositar a Borusa en la TARDIS, sin hacer ninguna tentativa de unirseles en su viaje. Quizás ya sabían cómo esto iba a acabar, consideró, y no querían tener nada más que ver. La idea no era especialmente reconfortante.

Habían aparcado en una órbita temporal y el Doctor estaba ocupado manipulando cables de debajo de la consola, conectándolos a tomas de corriente vacías en el chasis del motor de posibilidad. Parecía estar leyendo el diagnóstico en el monitor, con su destornillador sónico entre los dientes.

Él la vio mirar, y cogió el destornillador de la boca. 
--Casi listo --dijo.

--¿Tu o Borusa? --preguntó.

El Doctor le dio una débil sonrisa. 

--Ambos.
Parecía distinto tras dejar la Zona Muerta. Parte de ello, supuso, podría deberse al hecho de que se dirigían hacia la batalla, y se estaba preparando para el venidero conflicto, adoptando un estado de ánimo más contemplativo. Sin embargo había algo más, algo que le molestaba. Le había visto observarla cuando pensaba que no estaba mirando, y había tenido la misma expresión atormentada que le había observado cuando se conocieron por primera vez en Moldox. De alguna forma, era como si le tuviera miedo, y no podía muy bien entender por qué.

Se preguntó si había hecho algo para preocuparle. Era evidente como había evitado decirle sus planes mientras trabajaba conectando a Borusa a la consola, fingiendo concentración.

Le observó rodear la consola, volviendo a comprobar los enlaces. Murmuró en voz baja algo a Borusa, quien no pareció responder, y entonces se dirigió hacia la palanca de desmaterialización, y la accionó. La TARDIS se estremeció, y se deslizó ruidosamente hacia el Vórtice.

Aferrándose a la barandilla, Cinder observó. El Doctor golpeó una secuencia de botones, y el techo, que hasta entonces había sido del mismo material gris apagado que las paredes, pareció aclararse repentinamente, revelando una panorámica del espacio.

Ante ellos se encontraba la familiar vista de la Espiral de Tantalus, los planetas entretejiéndose en uno, una hélice serpenteante alrededor de la agitante anomalía del Ojo.

A lo lejos, las pequeñas motas de platillos Dalek revoloteaban por el espacio como nubes de insectos, pululando alrededor de una colmena.
--¿Qué vamos a hacer? --dijo. El Ojo parecía muy distante desde aquí, en los extremos más lejanos de la Espiral, con la amenaza de miles de platillos Dalek entre ellos y su objetivo. 
¿Cómo iban a acercarse al Ojo, y no digamos ya entrar?.

El Doctor empezó a apretar botones e interruptores en el panel de control. Las luces parpadearon en la sala de la consola. Los motores suspiraron. El rotor se atenuó y paró. 
--Vamos a entregarnos --dijo con cansancio.
Por un momento se quedaron en cercana oscuridad, con solo las fantasmales crepitantes luces del motor de posibilidad como iluminación.
--¿Entregarnos? --dijo Cinder incrédula--. ¿Después de todo esto?. ¿Tan solo vas a apagar la nave y dejar que los Dalek vengan a por nosotros? --levantó la vista. Por encima, el Ojo de Tantalus les miraba a través de la cubierta de la TARDIS, una siniestra, vigilante presencia. Incontables platillos Dalek se colaron a través del campo estrellado, revoloteando alrededor de la gran espiral de mundos ocupados. ¿Seguramente ya habían notado la llegada de la TARDIS?. No pasaría mucho tiempo antes de que las naves Dalek empezaran a converger en su posición.

--Eso es exactamente lo que voy a hacer --dijo el Doctor. Estaba de espaldas a ella, actuando como si no estuviera allí.

--¡No puedes!. No puedes parar ahora. Hemos llegado demasiado lejos. Si no encontramos una manera de detenerlos, los Dalek seguirán para siempre. Lo destruirán todo. Asesinarán a cada uno de los tuyos.

--¡Los mios! --rugió el Doctor interrumpiéndola--. Mi gente os hubiera hecho lo mismo. Ni siquiera estoy seguro de que merezcan nada más.

--Sí, lo hacen --dijo Cinder en voz baja--. A pesar de todo, merecen ser salvados. Sé que crees eso. De lo contrario, ¿por qué estamos aquí?.
La TARDIS dio una sacudida repentina y sus motores chisporrotearon. Las luces parpadearon y disminuyeron, y la columna central empezó firmemente a subir y a bajar con su familiar suspiro.

El Doctor negó con la cabeza. 
--No, vieja chica. No tenemos otra opción. Tenemos que hacer esto --maniobró una palanca de la consola, con una expresión de dolor.
La TARDIS se estremeció de nuevo, con los motores zumbando. En algún lugar en las entrañas de la nave, una campana empezó a tañer repetidamente, haciéndose eco a través del laberinto de los cambiantes pasillos y habitaciones. Llenó la cabeza de Cinder como un incesante metrónomo, un grito de ayuda. Luchó contra el impulso de presionar sus manos sobre los oídos.
--Ella lo sabe, ¿verdad?. La nave sabe lo que planeas hacer, ¿y está tratando de detenerte? --dijo Cinder--. Está tratando de llevarte lejos de aquí. Quizás deberías escucharla...
El Doctor se giró, agitando su mano a Cinder con desdén. 
--¡Oh, solo vete! --ladró furiosamente--. ¡Vete!.
Cinder retrocedió, sorprendida por la ferocidad de la respuesta del Doctor. Su espalda se encontró con la barandilla de metal que rodeaba el estrado central. Se sujetó a ella para apoyarse, armándose de valor. 
--¿Por qué me permitiste venir?. Si tienes tantas ganas de estar solo, ¿por qué me animaste?. ¿Por qué me mantuviste?.

--Para recordarme quien no soy --dijo el Doctor.

--Estás enfadado, por lo que vimos en la Zona Muerta, sobre lo que los Señores del Tiempo han hecho a su propia gente --dijo--. Y tienes miedo de lo que signifique salvarlos. Lo entiendo.
El Doctor negó con la cabeza. Cuando habló, su voz estaba teñida de tristeza. 
--No es eso --dijo. Sus hombros se desplomaron. Se dio la vuelta y vio el peso de siglos descansando pesadamente sobre sus hombros.

--Soy yo ¿cierto? --dijo--. Estás preocupado de lo que pueda hacer, de que voy a ir y voy a hacer que nos maten a ambos.
El Doctor suspiró. 
--No, Cinder. Me preocupa que no seré capaz de protegerte. He perdido a tanta gente, tantos amigos. No... --vaciló, y entonces se irguió--. No sé si podría soportar perder a otro.
Se acercó a donde estaba. 
--Recuerda lo que dije en Moldox. Estoy dentro. Tomé la decisión de venir contigo. Estamos en esto juntos, de una u otra forma. Quiero joder a esos Dalek tanto como tu. No trates de detenerme ahora.

--Muy bien --dijo.

--Así que, ¿esta rendición?.

--Si me muestro, no serán capaces de evitar regodearse. Nos tomaran prisioneros, nos acercarán al Ojo. Me conocen desde hace tiempo --dijo.

Cinder frunció el ceño. 

--No suena como el lugar más se...

--Depredador.
El roce metálico de la voz de un Dalek resonó a través de la habitación de la consola. Cinder se congeló, se enfureció. El sonido de esas cosas, esos frios demonios de metal, parecía arañar su alma.

Se dio la vuelta en el sitio, con los escudos bajos, uno o más Dalek se habían transportado abordo de la nave. Estaba convencida de que, en cualquier momento, sentiría la atroz quemadura de un arma de energía, hirviendo su carne hasta los huesos.

Pero no había nada allí. La voz había sido emitida desde el interior de la Espiral, recogida por los sistemas de comunicación de TARDIS.

El Doctor se alejó de la consola, quitando cautelosamente las manos de los controles. Incluso la TARDIS pareció entender que el momento había pasado, que ahora era el momento de dejar de luchar. La columna central suspiró, y quedó inmóvil. La estridente campana cesó su ensordecedor repique.
--Aquí estoy --dijo el Doctor. Su voz era baja y áspera, con el peso de seriedad y de siglos.

--Doc-tor --dijo el Dalek--. Exterminador de Dalek. El Gran Azote. La Muerte viviente. El Ejecutor --el Dalek hizo una pausa. Cinder observó al Doctor, midiendo su reacción. Su rostro permaneció impasible, con la mandíbula apretada--. Estos son los nombres otorgados a ti por los Dalek, Doctor. Me pregunto si te sientes orgulloso. Me pregunto si te deleitas con la muerte de tus enemigos?.
Este no era como ningún Dalek que Cinder nunca hubiera escuchado. La voz era la misma, pero tenía una calidad diferente, un inteligencia desconocida, quizás incluso un indicio de reverencia.
--Nunca me he deleitado con la muerte --dijo el Doctor--. Valoro la vida por encima de todo lo demás. No soy como vosotros. No soy un Dalek.

--Sin embargo, nos exterminas con impunidad. Permiteme que te asegure, Doc-tor, que los Dalek también valoran la vida.
El Doctor se rió, pero su risa estaba llena de remordimiento. 
--Solo valoráis la vida Dalek. Solo existís para destruir, para consumir. Sois parásitos, viviendo fuera de la carcasa de la creación.

--Los Dalek son la forma de vida superior en el universo --dijo el Dalek--. El resto de vida es irrelevante.

--Ah --dijo el Doctor con un suspiro--. Ahora estás empezando a sonar más como los Dalek que conozco. Ahora me estáis diciendo lo que realmente pensáis.

--Sin embargo, Doc-tor. Eres admirado por los Dalek. Eres reverenciado. Tu mera presencia invoca terror entre los nuestros. No hay mayor honor. Me gustaría conocer a la criatura que puede aterrorizar a una especie entera. Aprenderíamos de ti.
El Doctor hizo muecas ante los más horribles y desagradables cumplidos.

Cinder tragó saliva. Quería decirle al Doctor que lo apagara. Silenciar el monstruo, encender los motores de la TARDIS y llevarla lo más lejos posible de la Espiral de Tantalus. Usar su máquina del tiempo para llevarla a algún lugar donde no hubiera Dalek, ni Señores del Tiempo, ni Guerra.

Pero sabía que no podía. El Doctor tenía razón. No tenían otra opción. Si había siquiera la más mínima oportunidad de que su plan funcionara, de que pudieran encontrar los medios de acercarse lo suficiente al Ojo como para desplegar el motor de posibilidad, para de alguna forma derrotar a estos monstruos, tenían que tomarla.

--Estoy aquí para parlamentar --dijo el Doctor.

--Los Dalek no parlamentan --dijo el Dalek--. No acordamos. No negociamos.

--No --dijo el Doctor--. Realmente no pensé que lo hicierais.
Cinder se tensó de preocupación. ¿Había juzgado mal?. Ahí eran presa fácil. Los Dalek podrían destruirlos fácilmente antes de que la TARDIS pudiera restaurar la energía. Era una increíble apuesta la que estaba haciendo.

--Sin embargo, te admiramos, Depredador, antes de que seas ex-ter-mi-na-do --dijo el Dalek--,.te será concedido una audiencia con el Círculo de la Eternidad. Hablarás con nosotros antes de morir.

--Que amable --el Doctor miró a Cinder, y no pudo evitar pillar la pícara expresión de “te lo dije”. Infantilmente, cedió a la urgencia de sacarles la lengua.

La TARDIS dio una una repentina sacudida inesperada, y Cinder se vio obligada a cogerse fuertemente de la barandilla. Sintió a la nave moverse a sacudidas, balanceándola sobre los talones. El Doctor se estaba sujetando a la consola, aferrándose a una palanca cercana.
--¿Qué...? -- comenzó, pero se detuvo cuando vio que el Doctor estaba mirando atentamente al techo. Siguió su mirada. A través del techo transparente vio una gran destacamento de escolta de alrededor de diez platillos Dalek que se habían reunido entorno a la inmóvil TARDIS. Aros de parpadeante luz azul manaban de la base del platillo más cercano, cercando la TARDIS en lo que asumió que era un rayo tractor. Estaban siendo lentamente arrastrados hacia el Ojo, y al palpitante corazón de la operación Dalek.

--¿Dónde crees que nos llevan? --preguntó, su voz apenas era un susurro.
El Doctor se estiró y apretó un botón en la consola antes de responder, presumiblemente para asegurarse que los Dalek no escucharan esa conversación, aunque Cinder no podía ver ningún micrófono en la sala de la consola. 
--Hay una estación orbitando alrededor del Ojo --dijo--. La sede de este Círculo de la Eternidad. Ahí es donde nos llevan --señaló al techo, a una pequeña mota negra, planeando cerca del corazón del Ojo.

--¿El Círculo de la Eternidad? --dijo--. ¿Pensaba que los Dalek solo aceptaban ordenes de su Emperador?.

--El Emperador de los Dalek está experimentando --respondió el Doctor--.Probando la creación de nuevos tipos de Dalek, esperando que sean capaces de proveer una ventaja, un medio para ganar la Guerra. Por lo que puedo decir, el Círculo de la Eternidad es un selecto grupo de Dalek encargados de desarrollar nuevas armas, y coordinar el esfuerzo de guerra de los Dalek a través del tiempo.

Cinder se encogió de hombros. 

--Al igual que el Consejo de los Señores del Tiempo --dijo.

El doctor hizo una mueca, como si se hubiera tragado algo repugnante. 

--Sí --dijo--. Supongo que estás en lo cierto.
Miró a Borusa, todavía atado en la estructura de la máquina del Señor del Tiempo. Apenas podía soportar ver a la cosa, o siquiera considerarlo un ser. Se sintió desorientada solo de tratar de fijarse en la constante cambiante cara de Borusa, atrapada en ese terrible bucle regenerativo. Sus ojos, azul eléctricos y que todo lo ven, parecían hurgar en ella desde el otro lado de la sala de control. Se preguntó si podía ver su futuro, si sabía si iban a sobrevivir a esto. Se le ocurrió un repentino pensamiento. 
--Espera un minuto --dijo--. ¿Sabías que los Dalek iban a caer en tu plan?. ¿Que iban a ser incapaces de resistir la tentación de verte morir en persona?.

El Doctor sonrió. 

--No necesito que Borusa me diga eso --dijo.
La estación de comando de los Dalek parecía más bien una gran ciudad flotante que la especie de satélite habitable modular que Cinder había imaginado. Estructuras abovedadas enclavadas entre grupos de erizadas espinas y transmisores, y la que fuera la sustancia de la que estaba construida, Dalekanium supuso, brillaba como bronce bruñido en la reflejada luz del Ojo. Cientos, si no miles, de platillos Dalek y sigilosas naves zumbaban alrededor como abejas asistiendo obedientemente a la reina.

Sin embargo, tras ella, empequeñeciendo la gran estación estaba el Ojo de Tantalus. Así de cerca, a través de del opaco techo de la TARDIS, Cinder podía ver claramente la fisura en el corazón de la anomalía: una inmensa, crepitante fisura de pura energía, una herida abierta en la fábrica del tiempo y del espacio, burbujeante con luz de color rubí.

Ahora podía ver más que nunca, porque la anomalía se había conocido como el Ojo, este núcleo era su pupila, rodeada por la espiral de brillante gas que formaba su iris caleidoscópico. Dentro de esa extensa región, el tiempo se volvia loco, acelerándose y desacelerándose, revirtiéndose y generalmente comportándose tan impredeciblemente que las leyes de la física no tenían forma de explicar su existencia.

Cinder observó estrellas estallar con una súbita e intensa ferocidad, solo para convertirse en hinchados agonizantes gigantes en segundos. Mientras tanto, otros colapsaron y recomenzaron, resucitadas del borde de la muerte, brotando de nuevo con vibrante vida. Se preguntó que había sido de los muchos exploradores que habían volado al Ojo, y si aún lo exploraban, atrapados en un espacio intermedio en algún lugar entre la vida y la muerte. ¿Así acabarían ella y el Doctor si su plan tenía éxito?. ¿O, de alguna forma, les protegería la TARDIS?. 

Sin embargo todos esos pensamientos fueron olvidados ante la vista de una estructura Dalek incluso más grande, colgando sobre el Ojo, el inmenso barril de su exterminador de planetas, el arma que ella y el Doctor debían evitar que fuera disparada.

Cinder tragó saliva. Ni siquiera sabía por dónde empezar. El cañón era del tamaño de tres lunas, o más bien estaba compuesto de tres lunas, que habían sido juntadas en una enorme celosía de puntales y nódulos, y segmentadas por unos enormes brillantes discos de metal. En la parte frontal, tres radios se unían para formar una enorme punta, desde donde supuso que la explosión de energía sería disparada.

Enjambres de Dalek se ocupaban de él, miles de ellos, cientos de miles de ellos, como hormigas obreras, reptando por la superficie.

Un chorro de luz de color rubí estaba siendo desviado desde el Ojo, canalizado en dos restallantes antenas hacia la parte posterior del arma. Ella pudo ver que en el final de la pistola, si podía ser llamada un arma, la luz rubí estaba empezando a parpadear, como si estuviera acumulando energía para una descarga, justo como las armas temporales más pequeñas que había visto a los Dalek usar en Moldox.

Entonces esa era el arma que utilizarían para destruir Gallifrey, y cualquier otro planeta que se pusiera en el camino de su terrible ambición. Cinder podía ver ahora, por primera vez, la verdadera magnitud de la amenaza. Podía ver lo que había aterrorizado a Rassilon y a su consejo.

Esto era ingeniería en una escala épica. Con ella, no había ninguna duda de que los Dalek serían capaces de acabar con la Guerra y reclamar el dominio sobre todo el universo. Era en partes iguales una de las cosas más impresionantes y terroríficas que nunca había visto, y lo peor, le pareció que estaban a punto de disparar.

Retrocedió de la consola, desviando la mirada.

Los Dalek habían permanecido en silencio desde que el Doctor había cortado la conexión, pero ahora los sistemas de comunicación chisporrotearon con vida, recogiendo una transmisión de la estación de mando.
--Informe --exigió una abrupta voz de Dalek. Cinder saltó ante la repentina intrusión.

--Objetivo adquirido --respondió otra idéntica voz cercana, presumiblemente abordo de una de las naves de escolta.

--Procede --fue la económica respuesta.
Cinder y el Doctor observaron en silencio mientras eran arrastrados hacia la estación. No podía evitar maravillarse ante el auténtico tamaña de la estructura, conforme eran atraídos y se dio cuenta de la verdadera escala de la operación Dalek. Debía de haber miles de millones de ellos en la Espiral, cuando consideró cuantos había habido en Moldox, y debían haberse esparcido por todos los otros mundos inhabitados. Y esto solo era ahora, en este particular periodo de tiempo, en esta especifica región del espacio.

La idea de un universo a rebosar de Dalek le llenaba de pavor. ¿Se había equivocado?. ¿Lo había hecho el Doctor?. Quizás deberían de haber permitido a los Señores del Tiempo seguir con su plan de desplegar su dispositivo Armageddon. Quizás la perdida de vida humana hubiera valido la pena.

Como si entendiera su creciente sentido de terror y la oscuridad que estaba empezando a emerger de sus pensamientos, el Doctor se puso a su lado, sosteniéndose en la barandilla. 
--No pasará nada --dijo tranquilamente--. Quédate a mi lado y estarás bien.
Quería preguntarle cómo podía estar tan seguro, tan confiado, pero el momento había pasado. Él ya había vuelto a ver su acercamiento. La estación llenaba ahora toda su vista, y las naves escolta estaban empezando a retirarse, yendo hacia el vacío. Los Dalek claramente parecían tener la impresión de que se las habían apañado para encerrar al Doctor, que él y su TARDIS estaban completamente a su merced. Cinder no podía quitarse la sensación de que, de alguna manera, la realidad de la situación era la inversa.

La TARDIS, anclada al platillo Dalek como el péndulo de un reloj, se deslizó a través de la boca cavernosa de la bahía de acoplamiento de la estación, y se adentró en sus fauces.

Apenas tuvo tiempo para hacerse una idea de como era el interior de la estación antes de que, con una repentina sacudida, la TARDIS fuera liberada del rayo tractor y chocara con el suelo de la zona de carga. Fue lanzada hacia delante, y solo evitó tropezar con la consola por el Doctor, que extendió un brazo para atraparla por la cintura, manteniendo al mismo tiempo su agarre a la barandilla. Sin aliento, le dio las gracias, encontrando el equilibrio de nuevo un momento más tarde. Se apartó el pelo de los ojos.
--Entonces --dijo--. ¿Solo vamos a deslizarnos como si fuéramos los dueños del lugar, en los expectantes brazos de un billón de Dalek?.

--Algo así --dijo el Doctor distraído. Volvió a la consola y jugueteo de nuevo con los diales y los interruptores.

Cinder le miró boquiabierta. 

--Quiero decir... yo... solo estaba siendo sarcástica --murmuró--. ¿Ese no es realmente el plan?. ¿Verdad?.

El Doctor la miró sobre su hombro, moviéndose fluidamente alrededor de la consola para accionar otra palanca. 

--Freno de mano --dijo, como si respondiera a todo--. Mejor no dejarlo puesto si pensamos que podemos necesitar una salida rápida.

--Oh, no creo que haya necesidad de eso --dijo Cinder incrédula--. Creo que estaremos demasiado ocupados siendo inequívocamente exterminados para que importe.

El Doctor negó con la cabeza. 

--Tienes talento para el melodrama --dijo--. Ahora vamos. Coge tu abrigo.

Cinder emitió un balbuceó exasperado, pero hizo lo que le dijeron, fue a buscar su vieja y chamuscada chaqueta de donde la había lanzado al suelo tras huir de Gallifrey. Se la enfundó. --No puedo creer que vayamos a hacer esto --dijo. Se volvió para ver al Doctor que ya no estaba ahí. Girándose, le vio, bajando el corto tramo de escaleras del nivel superior. Cinder frunció el ceño. Ni siquiera le había visto salir. Debió de haber salido mientras estaba cogiendo su abrigo.

--Un poco tarde para arrepentirse ahora --dijo caminando con pasos largos hacia la puerta. La abrió y salió a la molesta luz eléctrica de la estación de mando de los Dalek.

--Hola --le oyó decir--. ¿Qué se supone que uno debe decir en tales situaciones?. ¡Ah, sí!, cierto. ¡Llevadme ante vuestro líder!.
Con un profundo suspiro, Cinder se apresuró tras él.