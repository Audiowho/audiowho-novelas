\part*{Parte 2 \\ \vspace{2 mm} {\large Gallifrey}}
\addcontentsline{toc}{part}{Parte 2}
\chapter*{Capítulo Nueve}
\addcontentsline{toc}{chapter}{Capítulo Nueve}

Karlax se encorvó sobre su mesa mientras golpeaba sin ganas una pantalla de datos con su dedo índice. Los glifos que iban apareciendo indicaban innumerables informes provenientes del frente, o más bien, de los numerosos frentes donde los Señores del Tiempo estaban peleando contra los Dalek en ese momento.

Seleccionó uno al azar y lo trajo a la pantalla, luego escaneó las líneas que se desplegaron, sin ni siquiera molestarse en comprobar de a qué época se referían. Daba lo mismo, le estaban empezando a parecer todas igual: cada período de la historia de Gallifrey estaba ahora bajo asalto de los viles mutantes Kaled.

La situación no había cambiado. Cada uno de los informes seguían siendo los mismos. No importaba lo bien que lucharan, cuántos platillos Dalek o naves furtivas consiguieran destruir los Señores del Tiempo, siempre los reemplazaban. Esas cosas eran implacables, y peor, capaces de replicarse a toda pastilla. También eran astutos. Se iban a eras no disputadas para sembrar a sus progenitores, para clonarse y manufacturar legiones enteras que luego permanecerían durmientes durante años hasta el momento apropiado para atacar. Inevitablemente, desplegarían una fuerza de ataque Dalek para asediar una fortaleza Gallifreyan insospechada durante un período caótico de la historia de Gallifrey. Una vez incluso intentaron purgar al Gallifrey prehistórico de sus primitivas formas de vida en un esfuerzo por evitar que los Señores del Tiempo evolucionaran.

Para los Dalek, la vida era barata y fácil de reemplazar. Eso les daba ventaja. Un Señor del Tiempo podría tener trece vidas, pero, reflexionó Karlax, la regeneración no servía de mucho si te atomizaban dentro de una TARDIS de Batalla a punto de explotar o te erradicaban antes siquiera de ser tejido.

Karlax suspiró. Su cuello y su túnica parecían hoy más pesados de lo normal. Tenía la sensación de que estaban al borde del apocalipsis, que todos sus esfuerzos, todas sus supuestas victorias contra los Dalek serían, finalmente, en vano. Estaban bloqueados en un punto muerto, y sólo era cuestión de tiempo antes de que los Dalek encontraran una forma de romperlo y hacer que la cuenta atrás continuara. Tan sólo estaban alargando lo inevitable.

Su pantalla de datos pitó. Se actualizaba cada segundo con más informes, y tenía que leerlos y resumírselos al Lord Presidente. El problema era que Karlax no daba abasto con ellos, no mientras tuviera otros deberes de los que preocuparse. Pero, supuso, no se iban a leer solos. Presionó el icono de otro informe, pero cuando se recostó en su silla comenzó a sonar el chillido de una alarma. Sus hombros se hundieron. ¿Y ahora qué?.

Karlax reaccionó al sonido de la puerta al abrirse, que apenas se podía oír por encima del estruendo de la alarma. Un soldado de la Guardia de Cancillería llegó corriendo desde su despacho. Se detuvo ante la mesa de Karlax para recuperar el aliento.

--¿Y bien?. ¿Qué pasa? --soltó Karlax al guardia--. ¿Qué es todo este alboroto infernal?.

--Es una emergencia de nivel nueve, señor --dijo el guardia, todavía sin respiración. Sonaba preocupado.

--¿Nivel nueve? --inquirió Karlax.  No se acordaba muy bien de lo que eso significaba.

--Una cápsula temporal no autorizada está intentando materializarse en el Panóptico --dijo el guardia.

--¿Qué? --dijo Karlax. El timbre de su voz cambió cuando oyó las palabras del hombre. El Panóptico--. ¿Cómo han conseguido sobrepasar las trincheras aéreas y las barreras de transducción?. 

El guardia se lo quedó mirando con el rostro en blanco.

--No tengo ni idea, señor. Es... debería ser... bueno, es imposible.

--Claro que no --dijo Karlax, sarcásticamente. Se levantó, y apartó su pantalla de datos--. Busque al Gobernador. Dígale que reúna sus tropas inmediatamente. Si este intruso consigue entrar, sólo Omega sabrá lo que pasará.

--Ya está en ello, señor --dijo el guardia--. Fue el Gobernador el que me envió a informarle.

--Bien --murmuró Karlax.

El guardia lo miró como esperando a algo.

--¿Sí? --dijo Karlax--, ¿hay algo más?.

--El Gobernador requiere de su presencia, señor --dijo el guardia, claramente incómodo de ser el encargado de entregar el mensaje.

Karlax suspiró.

--Muy bien.

Siguió al guardia fuera de la sala y por el pasillo. El hombre parecía tener prisa e intentaba ir a paso rápido, corriendo incluso, pero a Karlax no le daba la gana. No le gustaba que lo convocara el Gobernador.

Atravesaron un amplio pasillo, el cual terminaba en una enorme puerta. Cuando se aproximaron, se deslizó automáticamente hacia arriba, dejando vía libre para pasar. Las vistas al otro lado eran inmensas, imponentes: el ojo del Capitolio, la ciudadela era el corazón palpitante de la civilización de los Señores del Tiempo. Su base refulgente se elevaba por encima de sus cabezas, estrechándose con cada distancia que se alzaba en el cielo y rodeado de espinas de torres y matrices de comunicación. La energía iridiscente de la cúpula y su curvatura, que tintaba la luz de color naranja, sólo era contemplable desde abajo.  

Karlax y el guardia atravesaron el gran portal que les llevó del complejo de los aposentos de los Cardenales, a la entrada de la ciudadela, pasando por el foso. Delante de ellos, podía ver como se congregaban otros soldados uniformados. El Gobernador no se estaba andando con chiquitas. 

¿Es que esto era todo?. ¿Había llegado la Guerra finalmente al Capitolio?. ¿Habían conseguido descubrir al fin los Dalek una forma de entrar, un medio de saltarse su seguridad?. Parecía improbable, pero, ¿quién más tendría la caradura de violar las barreras?. ¿Quién estaría tan loco para siquiera intentarlo?.

Tal vez el guardia tenía razón después de todo, y aceleró el paso, pasando de largo a los guardias mientras les gritaba que se apartaran de su camino.

Todo el mundo se estaba reuniendo en el Panóptico, la extensa cámara que servía como parlamento de los Señores del Tiempo y el hogar del Estado. Era un caos.

--¡Apártense! --gritó el guardia que lo había estado escoltando--. Ábranle paso al Cardenal.

Karlax miró al hombre con un poco más de respeto. La multitud de espectadores, algunos guardias, otros simples subordinados que estaban ansiosos de descubrir lo que estaba pasando, se hicieron a un lado para dejarlo pasar.

Alisándose la túnica, se precipitó por el pasillo central hacia el Panóptico en sí. El Gobernador estaba despejando el espacio del centro de la sala, rodeado de un ejército de guardias, todos armados con armas de energía. Vio a Karlax caminando hacia él.

--¿Ha informado al Lord Presidente? --dijo, como forma de saludo.

--Y buenos días a usted también --dijo Karlax.

--Maldita sea, Karlax. Esto es serio. ¿Sabe el Lord Presidente acerca de esto? --El rostro del Gobernador se estaba poniendo como un tomate.

--Todavía no --dijo Karlax--. No hasta que tenga algo que decirle. ¿Qué está pasando aquí?. El guardia dijo algo acerca de una emergencia de nivel nueve, sobre una cápsula temporal no autorizada intentando materializarse aquí, en el Panóptico. Estoy seguro de que eso no es posible. Sé que usted es demasiado bueno en su trabajo como para permitir que pase algo así.

Karlax sonrió para sus adentros. Era bueno establecer de quién sería la culpa si el enemigo conseguía violar la seguridad de la Ciudadela. A Karlax siempre le pareció útil inculpar al principio de todo, sobre todo si con esto podría demostrar que nada de esto descansaba sobre sus hombros. El Gobernador parecía exasperado. 

--Hemos intentado contenerlo, pero ha atravesado todas nuestras defensas, una por una. Sea quien sea, o sea lo que sea parece conocer todos nuestros protocolos. Tenemos que informar al Lord Presidente para que evacúe. Tiene que marcharse ahora mismo del Capitolio por si acaso el enemigo porta un arma.

Karlax miró al Gobernador. Esto no era una simple exageración. El hombre estaba preocupado de verdad.

--Muy bien --dijo. Señaló hacia uno de los guardias--. Usted. ¿Sabe dónde están los aposentos del Lord Presidente?.

--Sí, señor --dijo el guardia con los ojos abiertos de par en par.

Evidentemente, la idea de visitarlos lo aterrorizaba más que la posibilidad de que apareciese un enemigo desconocido a su alrededor.

--Bien. Entonces necesito que vaya a... --Karlax se detuvo en medio de la frase cuando un profundo gemido inundó el aire que había a su alrededor. 

El murmullo de la multitud se redujo inmediatamente a un susurro.

--Demasiado tarde --dijo el Gobernador, redundantemente--. Están aquí.

Karlax se volvió para mirar cómo la silueta de la nave entrante comenzaba a solidificarse en el aire justo a la izquierda de donde estaba él. Los guardias de la sala levantaron sus armas, listos para disparar. Una espantosa sospecha rechinaba en el fondo de la mente de Karlax. Reconocía ese sonido...

El ruido se incrementó hasta convertirse en un balido elefantino, y luego, con un último soplo, la nave se materializó, saltándose todas las medidas de seguridad de los Señores del Tiempo para salirse del vórtice y meterse en el Panóptico.

Durante un instante, todo el mundo que había en la sala se quedó en silencio, como si estuviesen demasiado asustados como para respirar. La nave era una magullada cabina azul con las palabras ``CABINA DE POLICIA'' escritas en gruesas letras blancas.

--Oh --dijo Karlax, con un movimiento asqueado de cabeza--. Es él. 

\mbox{}

--Hemos llegado --dijo el Doctor.

--¿Llegado a Gallifrey? --preguntó Cinder.

La idea de visitar el mundo natal de los Señores del Tiempo la emocionaba a la vez que la aterrorizaba. No se imaginaba que se fueran a mostrar particularmente hospitalarios con una refugiada humana. Ni siquiera estaba segura de que fueran a recibir al Doctor con las manos abiertas, a juzgar por cómo hablaba de ellos. Pero al menos no era Moldox. Y ya que estábamos… El Doctor sonrió.

--Sí, Gallifrey --dijo--. Aunque creo que les he dado un susto de muerte. 

Recogió el cañón Dalek que había dejado sobre una silla durante su corto vuelo. Si es que podía llamarlo vuelo. Cinder no estaba del todo segura.

--Vamos --dijo--. Será mejor que no te separes de mí. --Se dirigió a paso firme hacia la puerta.

Cinder vio el arma Dalek desechada que había sobre la rejilla de metal, y consideró durante un momento si cogerla o no. Decidió que no. No sabía con qué facilidad disparaban los Señores del Tiempo, y no quería darles la oportunidad de mostrárselo.

Encogiéndose de hombros, siguió al Doctor mientras éste se aventuraba fuera de la TARDIS.

Salió a la luz del Panóptico, e inmediatamente levantó las manos. Un mar de guardias los estaba rodeando, todos vestidos de uniformes rojos y blancos y armas relucientes que no parecían diseñadas para incapacitar o aturdir.

--Menudo comité de bienvenida --dijo, acercándose al Doctor--. Ya veo que eres muy popular entre tus amigos.

El Doctor no parecía estar prestando atención, ni a ella ni a los guardias.

--Karlax --gruñó con la vista clavada en alguien concreto de la multitud, una figura vestida en el ostentoso atuendo de los Señores del Tiempo: un casquete, túnicas y el exuberante cuello de color rosado. Tenía una pinta de lo más extravagante--. ¿Dónde está Rassilon?.

--Doctor, no puede seguir apareciendo así. Hay protocolos --respondió el hombre que Cinder asumió era Karlax.

--Y todavía se preocupa por los protocolos --dijo el Doctor, con un tono despreciativo--. No me extraña que estemos perdiendo la dichosa guerra.

Karlax frunció el ceño, ignorando el cruel comentario.

--Podría usted usar la puerta como todo el mundo --dijo.

--Intentaba atraer su atención --dijo el Doctor--. Y tiene que admitir, Karlax --echó un vistazo a su alrededor señalando con la mirada a la aglomerada masa de guardias que todavía estaban blandiendo sus armas--, que funcionó.

Karlax esbozó una fina y calculadora sonrisa.

--No se lo niego, Doctor. Está claro que atrajo nuestra atención.

El hombre que había al lado de Karlax, vestido con túnicas y casquete similares, solo que de color naranja y rojo y sin cuello, mandó a los guardias que bajaran las armas. Una palpable sensación de alivio recorrió la sala. Cinder dejó caer los brazos al empezar a sentirse un poco ridícula.

--Ahora, dígame --dijo el Doctor--, ¿dónde está Rassilon?.

--El Lord Presidente está actualmente envuelto en importantes asuntos de Estado --dijo Karlax con pomposidad.

--Querrá saber acerca de esto, Karlax --dijo el Doctor. Levantó el cañón Dalek y Cinder vio que el hombre de al lado de Karlax se llevaba la mano al cinturón, como si se preparara para desenvainar una pistola.

--Doctor --interrumpió ella, dando paso al frente y poniendo la mano en el cañón del arma Dalek--. Con tantas pistolas en la sala, creo que sería buena idea no meter esa en la mezcla.

Karlax se echó a reír.

--Veo que ha encontrado una nueva... compañera --dijo. Soltó la palabra como si le hubiera dejado un sabor de boca particularmente malo--. ¿Otro animal extraviado?.

Cinder se enfureció. Esto era exactamente cómo se imaginaba que iban a ser los Señores del Tiempo: Maliciosos, presuntuosos y una panda de narcisistas.

--Tendrá que dejarla aquí --continuó Karlax--. No puede traerla a la Cámara del Consejo.

--Yo hago lo que me da la gana --dijo el Doctor--. Está conmigo. Está bajo mi protección. Y ha visto lo que los Dalek traman en Moldox. Su perspectiva nos será útil.

--También está aquí, en la sala --recalcó Cinder. Ambos la miraron durante un momento antes de continuar su discusión.

--A Lord Rassilon no le gustará --advirtió Karlax.

--No --dijo el Doctor--. Pero a mí tampoco me cae usted bien, y me aguanto --añadió.

Las mejillas de Karlax se pusieron de un color escarlata, y Cinder tuvo que reprimir sus ganas de reír.

--Allá usted, entonces, Doctor --dijo--. Será mejor que venga conmigo.

El Doctor miró a Cinder. Había un brillo de algo que no había visto antes, un titileo en su ojo. Se estaba divirtiendo.

--Usted primero, Macduff --dijo, con una sonrisa.

