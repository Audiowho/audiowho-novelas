\chapter*{Capítulo Cinco}
\addcontentsline{toc}{chapter}{Capítulo Cinco}

--Cuidado. Podría seguir allí fuera --dijo Cinder, agachándose junto a la TARDIS y escaneando las ruinas en busca de cualquier signo de Degradación--. Ese estaba armado con cuatro armas de energía.

--Seguro que se ha marchado pitando a avisar a sus amigos --dijo el Señor del Tiempo--. No les gustará nada de nada el hecho de que esté aquí.

Cinder se lo quedó mirando. Había oído que los Señores del Tiempo eran famosos por ser arrogantes, pero este era diferente. No parecía comportarse de forma jactanciosa. De hecho,  había soltado ese último comentario con una agotada inevitabilidad que sugería que no quería en realidad estar aquí. Sin embargo, prefirió, por ahora, seguir recelando de él. Era difícil de descifrar, y no tenía ni idea de si podía confiar en él o no. Simplemente esperaba que no fuera a causar ningún problema si conseguía llevarlo hasta Andor. Un vistazo rápido, y luego de vuelta a la nave. Ese era su plan. Si eran rápidos, podrían estar de vuelta al amanecer.
Tenía el destornillador en la mano otra vez. Observó cómo lo alzaba sobre su cabeza y presionaba el botón. Movió el brazo de un lado a otro, escuchando el sonido que producía, antes de encogerse de hombros y luego devolverlo al hueco de su cinturón de munición.
Cinder avanzó hasta ponerse a su lado. Miró a su alrededor, todavía con esa sensación de estar demasiado expuesta.

--Soy Cinder, por cierto --dijo. No le ofreció la mano.

El Señor del Tiempo asintió. Cinder suspiró.

--Normalmente cuando uno te dice su nombre, lo educado es responder diciéndole el tuyo.

--¿Sí? --dijo el Señor del Tiempo con algo de brusquedad. Se quedaron en silencio durante un momento.

--¿Y bien? --soltó Cinder.

--¿Qué clase de nombre es ``Cinder’’? --dijo, cambiando hábilmente de tema.

--Es el único nombre que tengo, hoy en día --dijo--. Tuve otro, hace mucho tiempo, antes de que los Dalek vinieran. Pero después mataron a mi familia y me dejaron morir dentro de un cubo de porquería oxidado, así que dejé mi vida atrás. La gente que me encontró me llamó “Cinder’’, por mi pelo --levantó el brazo y acarició su revoltijo de nudos naranjas.

El Señor del Tiempo la miró pensativamente.

--Entiendo --dijo--. Yo también tenía un nombre, pero apenas puedo recordar la última vez que lo usé.

--¿Por qué? --dijo--. ¿Tan embarazoso era?.
El Señor del Tiempo le lanzó una mirada de reojo.

--Era un nombre con un significado. Ya no soy digno de él.

--¿Eso no lo deberían juzgar los demás? --dijo Cinder.

--Puede--respondió él.

--Dime --dijo--. Dime cuál era.

Se lo estuvo pensando durante unos segundos.

--El Doctor --dijo--. Me llamaban el Doctor --se dio la vuelta y se puso a recorrer la carretera con la cabeza gacha.

--Bueno, Señor del Tiempo al que llamaban el Doctor --gritó ella--. Estás yendo en la dirección equivocada.

\mbox{}

La temperatura había descendido con la atenuación de la luz de la tarde convirtiéndose lentamente en crepúsculo. Por suerte, la mochila compacta de Cinder no se había dañado con el despeñamiento, y pudo envolverse en el cálido jersey tejido a mano que llevaba siempre por si acaso.
La noche nunca caía del todo en Moldox. La luz del Ojo de Tantalus mantenía al planeta envuelto en un inquietante ocaso. Cinder nunca había conocido nada distinto, por supuesto, y la idea de una oscuridad completa, de un negro impenetrable, la llenaba de terror. A juzgar por su experiencia, la oscuridad albergaba a los monstruos. Al menos en Moldox, podías verlos venir.
Habían optado por desviarse hacia las ruinas en vez de seguir por la carretera. Eso significaba pasar por encima de dinteles rotos y paredes, y dar más rodeo, pero a los Dalek les era más difícil moverse entre los restos, y si se elevaban en el aire serían más fáciles de detectar.
Sólo vieron una patrulla en todos los tres kilómetros de cúpulas, habitación rotas y edificios cívicos: dos Dalek y dos Planeadores vigilando desde encima de los tejados, buscando signos de vida en el suelo. El Doctor la había metido en un refugio temporal bajo el arco de una puerta desmenuzada mientras éstos pasaban por encima. Esperaron durante más de diez minutos, sólo para asegurarse de que la patrulla no daba la vuelta.
Le había dicho al Doctor que tendrían que hacer una parada rápida durante la ruta, y que ya casi estaban allí: la última localización conocida de un campo rebelde. Era un montón surtido de tiendas, improvisaciones y estructuras temporales construidas a partir de la chatarra de los edificios caídos. Desde arriba, estaba diseñado para asemejarse a otro campo de escombros, pero desde aquí abajo se parecía al campamento de un ejército en procesión, anidado entre las estructuras derrumbadas de lo que una vez habían formado una plaza o un parque recreativo.
Alrededor de treinta hombres, mujeres y niños, todos vestidos con harapos encontrados en la basura, deambulaban limpiando armas, cocinando y atendiendo las heridas ajenas. Esta era la única familia que Cinder había conocido desde los 7 años. Esta era la suma total del movimiento rebelde humano, y, hasta donde sabía, las últimas personas libres en Moldox, los que habían elegido luchar contra los Dalek y los que habían sido lo suficientemente fuertes como para sobrevivir.

--¿Qué este lugar? --dijo el Doctor--. Pensaba que ibas a llevarme a Andoc.

--Andor --corrigió Cinder--. Y estoy en ello. Esta es la parada de la que te hablé. Necesito recoger unas cosas.

--¿Aquí es dónde vives? --dijo el Doctor.

Cinder sacudió la cabeza.

--No durante más que un par de días. Tenemos que seguir moviéndonos si queremos que los Dalek no nos cojan. Pero sí, aquí es. Esta es mi vida. Esta es mi gente.

El Doctor no dijo nada, simplemente asintió y observó el lugar con sus viejos y lagrimosos ojos.

--Vamos --dijo Cinder --. No quiero estar aquí más de lo necesario. Sólo necesito echar un par de cosas en mi mochila.

Lo llevó a través de la aldea improvisada, y empezaron a atraer las miradas de la gente mientras pasaban.

--Ignóralos --dijo Cinder, sin levantar la voz--. Ya es raro que se una otro humano a nuestra pequeña banda. Imagina lo que pensarían si supieran que eres un Señor del Tiempo --sonrió, y decidió no añadir que probablemente lo lincharían a la primera de cambio.

--¡Cinder!.

¡Cachis!. Reconoció la voz. Mantuvo la cabeza gacha. Coyne era la última persona con la que necesitaba encontrarse ahora. Tenía la esperanza de irse sin verlo, sin enfrentarse a la culpa de dejarlo, o dejarlos a todos aquí, mientras ella huía con un extraño en una cabina azul. Lo que estaba haciendo no era valiente. Eso lo sabía de sobra, pero se había cansado de correr sin parar, de propiciarse una existencia entre las ruinas y de tener que vigilar constantemente si había Dalek cerca. Nunca había querido ser guerrera, pero las circunstancias la habían empujado a interpretar el papel, y ahora, por fin, había una oportunidad de escapar, de hacer algo diferente con su vida. Sabía que si veía a Coyne se echaría para atrás.


--¡Cinder!. ¿Quién es tu amigo?.

Con un suspiro, se dio la vuelta para ver cómo Coyne se dirigía directamente hacia ellos desde el otro lado de su tienda.

--Hola, Coyne --dijo ella.

Era esbelto y musculoso, de alrededor de 40 años de edad y era uno de los líderes de su tropilla. También era veterano de numerosos encuentros con los Dalek, como lo testificaba la profunda cicatriz morada que había en el lateral izquierdo de su cara, donde un haz de energía le incineró la oreja y le arrancó la carne de la mejilla. 

Fue Coyne quien la sacó del contenedor en las ruinas incineradas de su casa, y fue Coyne quien le había enseñado a sobrevivir, a luchar.

--¿No vas a presentarnos? --dijo, mirando con recelo al Doctor.

--Este es... --vaciló--. Este es...

--John Smith --dijo el Doctor, alargando la mano.

--Bueno, John Smith --dijo Coyne, mirando al Doctor de arriba a abajo--. ¿Dónde te habías metido?.

--Donde los Dalek no pudieran encontrarme --dijo el Doctor, con una fina sonrisa--. Moviéndome de un sitio a otro, sin quedarme allí demasiado tiempo --miró a Cinder, que podía adivinar que no estaba mintiendo--. Encontré a Cinder intentando derrotar sola a una patrulla Dalek --continuó--, y decidí aparecer y echar una mano.

Coyne se rió afablemente.

--Sí, eso es propio de Cinder --le puso un brazo protector sobre el hombro--. Pero, ¿por qué no fuiste con nadie?. Conoces las reglas. No es seguro ir por ahí sola.

--No estaba sola --respondió ella--. Iba con John Smith, ¿a que sí?.

Coyne puso los ojos en blanco.

--Ya sabes a lo que me refiero, Cinder --dijo--. Mira, apuesto a que vosotros dos os morís de hambre. Venga, el estofado ya está casi listo.

Cinder miró al Doctor como disculpándose.

--Bueno, es que...

--Eso suena maravilloso --dijo el Doctor.

\mbox{}

El estofado era un caldo denso hecho de verduras y hierbas, pero estaba caliente y se agradecía, y Cinder lo deglutió con gusto, disfrutando de la extraña sensación de tener la barriga llena.
Así era cómo se pasaba la noche en Moldox, y la luz extraña y etérea del Ojo se propagaba por el cielo como una aurora de estrías amarillas, rosas y azules. Burbujeaba como la superficie de un lago innsondable, como un colorido cuadro al óleo untado en el cielo.
El Doctor, que había entablado una conversación con Coyle durante la última media hora para enterarse de más detalles sobre la fuerza de ocupación Dalek, se fue a sentar al lado de Cinder en un tambor volcado. Siguió su mirada hasta el cielo.

--Hermoso, ¿verdad? --dijo ella.

--¿Sabes lo que son? --respondió. Ella sacudió la cabeza--. Vientos temporales --pegó un buen trago de una taza metálica de té--. Radiación temporal del Ojo. Lo que estás viendo son mil millones de años de historia, un destello en el cielo nocturno del antiguo pasado y del más lejano futuro. La radiación causa anomalías, errores en el espacio-tiempo. Es una ventana que da directamente a otra época, el mundo al otro lado está cambiando en constante flujo. Y sí, tienes razón, es bastante hermoso.

Cinder volvió a alzar la vista, esta vez con nuevos ojos.

--Todo ese tiempo, todos esos años de paz. Ahora sólo existe la Guerra.

--El universo está lleno de maravillas, Cinder. Las cosas que he visto... las lunas de cristal de Socho, el Velo Rojo de la Parábola Oriental, las playas aéreas de Altros. Allí fuera hay cosas que te harían llorar de felicidad --la estaba mirando atentamente.

--Moldox fue así una vez --dijo--. Antes de tu Guerra. Antes de que vinieran los Dalek. Los cielos solían estar llenos de naves de transporte, trayendo gente nueva y exótica todos los días. Las ciudades estaban plagadas de vida. La gente era feliz. Fuera en las llanuras, erigían palacios de lujo que daban al Mar Bario, a su agua dorada y a sus playas hechas de granos de hielo. Construían torres que parecían casi tocar el mismo Ojo, y máquinas que se parecían y pensaban como los hombres. Era un imperio esplendoroso. Ahora no es más que un montón de ruinas.
Se sacudió el polvo que tenía en la punta de la bota.

--Todos esos otros lugares que mencionaste, esos mundos maravillosos... los vais a destruir todos, ¿verdad?. Hasta la última esquina del Universo. Cuando hayáis terminado no quedará nada.

--No si yo puedo evitarlo --dijo el Doctor--. Por eso estoy aquí, Cinder. Eso es lo que estoy intentando detener, el porqué de la necesidad de ver lo que los Dalek están haciendo aquí en Moldox.

Ella asintió. ¿Podía confiar realmente en este hombre, en este Señor del Tiempo?. Había algo en él, algo distinto. En su compañía, había empezado a creer, por primera vez en años, que podría haber una forma de salir de este lío en el que se encontraban, que podría haber esperanza. Era una emoción extraña, y todavía no estaba lista para aceptarla.

--¿Conseguiste lo que querías? --dijo, después de un rato. 

La pregunta la devolvió al aquí y al ahora.

--Sí --dijo ella, señalando su mochila, la cual había abandonado en su litera a unos cuantos metros de distancia bajo una carpa--. Unos cuantos recuerdos. Cosas que no quería dejar atrás --alzó el brazo para enseñarle la pulsera que rodeaba su muñeca. 

Sólo un aro de cables de cobre retorcidos y oxidados por la edad. Se lo había hecho su hermano, hacía todos esos años, y ella lo había guardado desde entonces. No quería irse de Moldox sin ella. Eso era todo lo que le quedaba de él, aparte de sus recuerdos.

--Entiendo --dijo el Doctor. Frunció el ceño al fijarse en algo--. Dime, ¿de quién es esa litera de allí, la que está al lado de la tuya?.

Cinder dirigió la vista hacia la otra cama improvisada, sólo a un metro o dos de la suya. Parecía extrañamente familiar.

--No lo se.

El Doctor asintió, con una expresión seria.

--Bueno, es igual de momento. Es hora de acabarnos la bebida e ir y averiguar qué están tramando los Dalek en Andor.

Cinder posó el vaso, cogió su mochila y se la llevó al hombro. Todo lo que quería hacer ahora era dormir, pero le había hecho una promesa al Doctor, y él también le había hecho una promesa a cambio. Tenía que hacerlo a toda costa, fuera como fuera.



