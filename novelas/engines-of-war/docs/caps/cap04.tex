\chapter*{Capítulo Cuatro}
\addcontentsline{toc}{chapter}{Capítulo Cuatro}

Antes de que los Dalek llegasen, Jocelyn Harris había sido la gobernadora del planeta Moldox, y de los cuatro asentamientos humanos circundantes en las lunas del planeta. También había sido buena en su trabajo: la colonia había florecido bajo su ojo diligente. Las tasas de natalidad aumentaron, el programa de construcción continuó a un ritmo constante y el proceso de terraformación había resultado relativamente suave, con un solo mal funcionamiento memorable que causó un único y crudo invierno. 

Jocelyn había estado orgullosa de su trabajo. Los habitantes de Moldox, que la habían reelegido tres veces seguidas, la habían aclamado como el heraldo de una nueva era. Y como recompensa a su fe inquebrantable, los había traicionado a los Dalek. 

No lo había hecho por un deseo de poder, o por cualquier tipo de devoción a una causa superior. Esas eran el tipo de cosas que impulsaban a la mayoría de los desertores, por su limitada experiencia. No, lo que había hecho había sido motivado por la cobardía, y según la propia opinión de Jocelyn, esto la convertía en la peor especie de desertor. Lo había hecho para salvar su propio cuello. Cuando los Dalek se habían apoderado de Moldox, desmantelando al planeta entero y sacrificando a la población, había accedido a ser su portavoz humana, su marioneta, su juguete. Todo para asegurar su vida. 

A lo largo de los años, había tratado de decirse a sí misma que no había tenido otra opción, que seguramente era mejor si luchaba contra los Dalek desde dentro, infiltrarse en sus planes, advirtiendo a su pueblo sobre el terreno. Sólo que siempre había tenido demasiado miedo de actuar, de pasar información a la resistencia, preocupada de que los Dalek se enterasen de lo que estaba haciendo. La represalia sería rápida, eficaz y sería el fin de todo. Sabía que, pasara lo que pasara, una cosa era segura: ella era perfectamente reemplazable. 

Se preguntó que tenían los Dalek en mente para ella hoy. Dos de esas terribles latas de color bronce habían venido a su habitación, una celda con otro nombre, y exigieron que les acompañase inmediatamente. Como de costumbre, no hubo ningún intento de sutileza, ninguna explicación, sólo la simple orden de que se la requería en la sala de audiencias. 

Se levantó de su escritorio, dejando su tablet e hizo lo que le dijeron. La gravedad artificial de la estación de mando Dalek era débil, a pesar de su tamaño. Sabía que los Dalek no la necesitaban realmente, podían magnetizarse al suelo de metal para evitar salir flotando, e incluso si lo hacían, tenían propulsores que les permitían volar. La gravedad era una simple concesión a los presos que se encontraban a bordo de la estación, y, por lo tanto, no eran particularmente dados a gastar energía para asegurarse de que estaba regulada a un nivel confortable. 

Por lo tanto, Jocelyn se encontró rebotando detrás de los Dalek, dando pasos exagerados mientras trataba de mantener el ritmo. 

La cámara de audiencias estaba a menos de quinientos metros de su celda, y durante los muchos años que habían estado aguantando en la estación, la había visitado en innumerables ocasiones. 

Hoy, al parecer, el Círculo de la Eternidad estaba en plena sesión. Los cinco estaban aquí, descansando sobre sus pedestales elevados, mirando hacia abajo mientras ella entraba trotando en la gran cámara hexagonal. 

Nunca había podido determinar la función de estos Dalek en particular, o lo que los diferenciaba de sus parientes más humildes. Salvo por su color, por supuesto. Eran idénticos en tamaño y forma a los dos guardias que la habían traído desde su celda, pero mientras que las carcasas estándar Dalek estaban decoradas con bronce bruñido y oro, los cinco miembros del Círculo de la Eternidad eran de un profundo azul metálico, con cabezas abovedadas de plata pulida y globos sensores de plata a juego salpicando la mitad inferior. 

Todo lo que Jocelyn sabía era que el Emperador Dalek les había encargado la elaboración de nuevas armas para utilizarlas contra los Señores del Tiempo, algunas de las cuales habían sido probadas en el pueblo de Moldox y los otros mundos de la Espiral Tantalus. Lo sabía porque había tenido que presentar los informes. 

Para Jocelyn, eran criaturas de pesadilla. Demonios encerrados en conchas azules. Eran los monstruos responsables de lo que le había sucedido a su amado planeta, a su hogar, y a sus hijos. 

--Espera --ladró uno de los guardias Dalek. Su voz era como clavos que se le clavaban en la cabeza. Se detuvo. Estaba de pie en el centro de la cámara, mirando desde abajo a los cinco Dalek azules. Parecían mirarla amenazadores, pero ninguno de ellos habló. 

Los guardias se retiraron, deslizandose en silencio de vuelta a dos huecos en la puerta. Decidió permanecer en silencio hasta que fuese conminada a hablar. 

Muy por encima de ella, una pantalla holográfica se encendió parpadeando, tiñendo una zona del aire de un brillante azul brumoso. Su aparición fue acompañada por un olor que le recordó a ozono puro. 

--Informad --retumbó la grave y chillona voz del Emperador Dalek. Jocelyn le miró sorprendida. La ominosa imagen de su enorme ojo impasible fue proyectada en la pantalla, pero su voz parecía venir de todas partes a su alrededor, llenando la cámara. Sintió el retumbo en sus tripas y que el pelo del cuello se le erizaba. 

--El arma está casi completa --dijo el Dalek en el pedestal del extremo izquierda, arrastrando las palabras en su monótono aspero--. Pronto el Erradicador estará listo. 

--Excelente --respondió el Emperador--. Estamos en vísperas de la destrucción de Gallifrey --una pausa--. ¿Qué hay de los progenitores?. 

--Doce de las diecisiete épocas identificadas ya han sido sembradas con progenitores Dalek --replicó otro miembro del Círculo de la Eternidad--. Las fuerzas de los Señores del Tiempo están dispersas. La Guerra se libra en múltiples frentes. 

--Como estaba previsto --dijo el Emperador--. ¿Qué avances se han logrado en el desarrollo del nuevo paradigma?. 

--Las pruebas en el planeta Moldox están casi completas --respondió el Dalek del pedestal central, sus válvulas de radiación destelleaban mientras hablaba--. Los datos sugieren que el nuevo paradigma Arma Temporal está casi lista para su distribución a través del continuo espacio-tiempo. 

--Mostradmelo --ronroneó el Emperador. 

--Obedezco --respondió el Dalek. Su cabeza giró en dirección a Jocelyn--. Jocelyn Harris. Has servido bien a los Dalek --dijo. 

--Lo he intentado --tartamudeó, insegura de a dónde iba todo esto exactamente. 

--La traición a tu propia especie sólo demuestra que no se puede confiar en ti --continuó el Dalek--. Serás ex-ter-mi-na-da. 

--¡No! --gritó--. ¡No!. Haré lo que sea. Dime lo que tengo que hacer para probarme ante ti --comenzó a retroceder hacia la puerta, pero sabía que no había ningún lugar a donde escapar. Estaba en una estación de mando Dalek, que orbitaba alrededor de una gran anomalía espacio-temporal. Cualquier aplazamiento sería provisional. Sin embargo eso no le impediría intentarlo. 

Se dio la vuelta, con la intención de cerrar la puerta, pero lanzó un grito de frustración ante la visión de una silueta Dalek en la puerta, bloqueando su camino. Mientras miraba, tratando desesperadamente de averiguar qué hacer, el nuevo Dalek se deslizó lentamente quedando a la vista. 

Era diferente de los otros. El mismo patrón de bronce y oro, la misma altura y aspecto general, pero la sección media de su cuerpo había sido sustituida, de modo que en lugar del brazo y el arma de costumbre, había un enorme cañón negro montado en una articulación en forma de bola. 

Ella retrocedió, tambaleándose por la baja gravedad. 

El Dalek se fue acercando a ella, nivelando su cañón. 

--¡Erradicar!. ¡Erradicar! --jirones de energía de color rubí comenzaron a acumularse alrededor de la boquilla del arma. 

--¡No!. ¡Por favor! --gritó Jocelyn, levantando sus manos para cubrirse la cara mientras el cañón escupía un chorro de luz sobre ella. 

Lo último que vio fue el ojo del Emperador Dalek mirándola con maléficas intenciones desde la pantalla..




