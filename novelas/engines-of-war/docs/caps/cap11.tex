\chapter*{Capítulo Once}
\addcontentsline{toc}{chapter}{Capítulo Once}

Desde el panel de observación en la antesala tenían una vista de todo el Capitolio. Cinder estaba de pie ante él, hombro con hombro con el Doctor, ambos sumidos en un silencio sobrecogedor. No podía dejar de maravillarse ante el mar de agujas erizadas, las cúpulas de cristal en forma de orbes, los extrañamente complejos angulares de los edificios y las plataformas de transporte. Esta era la aglomeración urbana de una raza antigua y divina, esta era la cumbre de la civilización de los Señores del Tiempo. Era hermoso y terrorífico a partes iguales, muy distinto del asolado desierto de Moldox. 



—No veía Gallifrey así desde hace mucho tiempo —dijo el Doctor al cabo de un rato—. Me recuerda lo que amo del lugar, y también lo que odio. 



—¿Te recuerda por lo que estás luchando? —dijo Cinder. 



El doctor se echó a reír. 



—Sí, quiero suponer que sí. 



Con los Señores del Tiempo, Cinder compartía un enemigo común en los Dalek y observando toda la extensión de esto, su ciudad principal, sintió una especie de empatía amarga con ellos. Esto era lo que estaban tratando de proteger: su hogar. Era lógico que, acorralados en una esquina, atacasen e hiciesen todo cuanto estuviese en su mano para defenderlo. 

Muchos de su pueblo afirmaban que los Señores del Tiempo no eran más que tontos que se autoengañaban, un antiguo pueblo que había asumido la responsabilidad de intentar custodiar el universo, de inmiscuirse en la evolución y el desarrollo de otras razas. Argumentaban que el poder de los Señores del Tiempo se les había subido a la cabeza y los había corrompido y que habían comenzado la guerra con los Dalek en primer lugar, todo hacía muchos siglos. Quizá lo peor era la idea de que fue su interferencia, o incluso el mero hecho de su propia existencia, lo que condujo a los Dalek a evolucionar en las despiadadas máquinas de matar que eran hoy. 

Cinder no sabía si algo de esto era cierto. Sin embargo no alteraba el hecho de que si se enfrentase al mismo problema que los Señores del Tiempo, un ejército merodeador de Dalek amenazando con borrar a su pueblo de la historia, se habría comportado exactamente de la misma manera. Habría luchado con cada gramo de su ser, utilizado toda arma disponible en su arsenal. Eso podía entenderlo. 

Sin embargo, no podía olvidar las cosas que había visto durante este largo y terrible conflicto, la pura arrogancia de los Señores del Tiempo, su desprecio por la vida humana y la horripilante inventiva de sus armas. La empatía era una cosa, la confianza era otra muy distinta. 

Tampoco podía simplemente ignorar el hecho de que, si los Señores del Tiempo tuvieran éxito, su pueblo aún podría estar en peligro de extinción, solo que esta vez a manos de un enemigo completamente diferente. 

Se acercaba el crepúsculo y, mientras Cinder miraba, pequeñas luces comenzaron a parpadear en el cielo alrededor de las cúpulas habitables. Al principio solo había un puñado, pero mientras miraba parecían multiplicarse hasta que hubo decenas de ellas, incluso cientos, yendo lentamente a la deriva en el cielo desde la ciudad de debajo. Parecían luciérnagas, zumbando caóticamente en la brisa. 



—¿Qué son? —dijo Cinder—. ¿Faroles de papel? 

El Doctor negó con la cabeza. 



—No, aunque el principio es el mismo. Son faroles de memoria. 



—¿Faroles de memoria? —repitió Cinder. ?



El doctor la miró. 



—Todos ellos piensan que van a morir —dijo—. Toda esa gente de allá abajo piensa que los Dalek están viniendo a por ellos y que van a ser exterminados —suspiró y el hastío en su expresión hablaba por sí solo. Quizá pensaba que tenían razón—. Así que están grabando todos sus pensamientos en esos faroles y esparciéndolos a través del espacio y el tiempo. Es el último acto de un pueblo desesperado. Están aterrorizados de que vayan a ser olvidados, así que se están sembrando a sí mismos en todos los rincones distantes del universo para ser recordados. 



—Es hermoso —dijo Cinder en voz baja. Se acercó más a la pantalla de observación, mirando como más de esos diminutos puntos de luz subían hacia el cielo antes de parpadear y desaparecer de la existencia, transmitidos a algún lugar profundo en el Vórtice. Se preguntó donde emergerían todos, en el distante y olvidado pasado o quizá en el futuro con cicatrices de la batalla, mucho después del final de la Guerra. 



—Es en vano —replicó el Doctor—, e inapropiado. Una pérdida de tiempo. La mayoría de los faroles no sobrevivirán al viaje a través del Vórtice. Se harán pedazos en los vientos temporales y todos esos preciados recuerdos serán arrojados a la brisa. 



—Esa no es la cuestión —dijo Cinder—. Para esas personas los faroles representan la esperanza. Esperanza de que una pequeña parte de ellos podría sobrevivir a todo esto. No les quites eso —de repente sintió frío y cruzo los brazos sobre su pecho, abrazándose a sí misma. 



El Doctor sonrió por el más fugaz momento. 



—Eres maravillosamente humana, Cinder —dijo en voz baja. Sus ojos se encontraron por el más breve instante, antes de que él apartase la mirada y la expresión cansada y demacrada regresara. 



—¿Qué hacemos ahora? —dijo Cinder. 



El Doctor se encogió de hombros. 



—Aguardaremos su respuesta —dijo.

 

Pasó media hora, tal vez más. Cinder se paseaba impaciente por la habitación, mientras el Doctor permanecía de pie en la ventana, mirando la ciudad que una vez había sido su hogar. 

Se preguntó por cuánto tiempo había estado huyendo. Rassilon le había llamado “Hijo díscolo de Gallifrey”. Esto sugería una historia más profunda e interesante de lo que había logrado averiguar hasta el momento. ¿Qué había hecho para ganarse una reputación como esa? Estaba claro que era un inconformista, por supuesto, el simple hecho de su aspecto, la deslustrada chaqueta de cuero, el pañuelo rojo y blanco, por no mencionar la extraña apariencia externa de su TARDIS, todo eso le distinguía como diferente a los otros Señores del Tiempo. Sin embargo, Cinder tenía la sensación de que había algo más. 

Supuso que podría ser simplemente debido a su antagónico enfoque de la autoridad y su flagrante desprecio por la obsesión de los Señores del Tiempo por las ceremonias y los rituales. Sin duda no se había hecho a sí mismo ningún favor hablando a Rassilon de la forma en que lo había hecho. Aunque, viendo como Karlax adulaba al Lord Presidente, probablemente era una saludable actitud a adoptar. Alguien tenía que hablar. Por supuesto Cinder misma probablemente no había ayudado mucho con su arrebato. De todas formas, al menos había sido capaz de exponer su argumento. 

Se volvió al oír el sonido de la puerta corredera al abrirse. Un guardia entró en la habitación. Sus ojos parecieron pasar por encima de ella a pesar de que estaba mirándolo directamente. Esperó a que el Doctor se volviese y le saludase antes de hablar. 



—Lord Rassilon hablará con usted, Doctor —dijo—. Puede traer a su… compañera. 



Cinder se puso rígida. El hombre había pronunciado la palabra de tal manera que la implicación era clara: no la consideraba de ninguna manera la ``compañera'' del Doctor, sino más bien su ``mascota''. El Doctor también lo sabía, mientras cruzaba la habitación, intencionadamente le puso la mano en el brazo para tranquilizarla. 



—Vamos —dijo entre dientes—. Vamos a ver lo que se les ha ocurrido a los viejos tontos mohosos

. 

Siguieron al guardia a lo largo del pasillo hasta la sala del consejo. Los hizo pasar pero no entró. 

Rassilon estaba sentado sólo en la cabecera de la mesa, todavía agarrando su bastón. Levantó la vista cuando entraron en la habitación. Cinder notó que, afortunadamente, a Karlax no se le veía por ningún lado. 



—¿Ha llegado a una decisión? —dijo el Doctor. 



Rassilon entrecerró los ojos. 



—Olvida cuál es su sitio, Doctor. Asume un derecho aquí, en la sala del Alto Consejo, donde no tiene ninguno. Es un renegado, un fugitivo. Un desertor. 



—Soy un antiguo Presidente de Gallifrey —dijo el Doctor airadamente. 



Rassilon se burló. 

—En nombre, tal vez, pero nunca más. Nunca apreciarías la importancia de semejante cargo. 



—Al contrario, Rassilon. Fui el único que lo hizo —el Doctor empujó una silla, arrastrándola por el suelo y dejándose caer en ella pesadamente, de cara a Rassilon—. La Lágrima de Isha. ¿Cuál es su decisión? 



—Que se utilizará en el Ojo —respondió Rassilon. 



Cinder sintió a su corazón sacudirse en el pecho. De pronto sintió nauseas. Iban a hacerlo. Iban realmente a asesinar a todo ser vivo en una docena de mundos. 



—Rassilon —dijo el Doctor claramente exasperado—. Está condenando a billones de almas a una muerte terrible. Más. ¿Cómo puede si quiera considerarlo? 



—¿Qué son billones de vidas humanas para nosotros, Doctor? —dijo Rassilon—. No son más que motas de arena en la brisa. Se reproducen como virus, infestando cada rincón del universo. Donde algunos mueren, otros ocuparán su lugar. 

Hizo una pausa, sus agudos ojos verdes fijos en el Doctor, como si se clavaran en él. 



—Ahora estamos hablando de vidas de Señores del Tiempo, Doctor. Lo que nos ha descrito es un dispositivo del fin del mundo, un arma con el poder de aniquilarnos, para poner fin a la Guerra a favor de los Dalek. Peor, si está en lo cierto, si los Dalek logran utilizar ese arma, Gallifrey y todos sus muchos niños serán completamente erradicados de la historia. Será como si nunca hubiesen existido. ¿Y dónde estarán sus preciosos humanos entonces? A merced de los Dalek, sin que nadie vele por ellos para mantener a los monstruos a raya. El destino del tiempo mismo está en la balanza. La muerte de billones no es nada para nosotros, Doctor, si eso ayuda a derrotar a los Dalek. 



—Rassilon, no puede sinceramente creer eso. Estamos hablando de una docena de mundos habitados —dijo el Doctor poniéndose de pie. La exasperación, y la incredulidad, eran evidentes en su voz—. Está hablando de genocidio. 



—Mundos habitados ahora por Dalek, Doctor, no lo olvide —dijo Rassilon.

 

—A pesar de eso no tenemos el derecho a decidir quién vive y muere. No a ese tipo de escala. No somos dioses, no importa cuanta pose os guste hacer con vuestras sofisticadas capas y graciosos sombreros —el Doctor dejó que sus palabras quedasen colgadas por un momento—. No puede decidir eso —continuó, razonable, en voz baja—. Si utiliza ese arma, seremos tan malos como los Dalek, aquello contra lo que luchamos. ¿No recuerda por qué estamos en guerra, en primer lugar? 



—¡Basta! —bramó Rassilon, golpeando el puño enguantado en la mesa. La saliva salpicó la superficie de la mesa ante él. Se puso de pie, mirando ceñudo al Doctor—. Estamos luchando, Doctor, porque debemos. Porque estamos siendo atacados y no tenemos otra opción. Estamos luchando para salvarnos de la extinción. 

—Eso no es suficientemente bueno —dijo el Doctor—.No es una justificación lo suficientemente buena para lo que estáis sugiriendo. Sellando el destino de doce planetas para salvar uno. Esa es la elección que estáis haciendo. Estáis poniendo vuestras propias vidas por encima de todas las demás. 



—¿Y qué si lo hago, Doctor? ¿No es esa la carga de los Señores del Tiempo? Si sobrevivimos a esta guerra, como debemos, continuaremos garantizando la inviolabilidad de las líneas temporales. Restauraremos la historia; desharemos el daño causado por los Dalek. Solo los Señores del Tiempo tienen la capacidad, el ingenio, para lograr esto. Es nuestro deber sobrevivir. 



El Doctor se echó a reír. 



—Oh, qué grande debe ser la vista desde allá arriba en su pedestal, Rassilon. Su deber. Cada día suena más como un Dalek. 



Cinder pudo ver que a Rassilon le estaban chirriando los dientes, flexionando sus dedos dentro del guante. 



—Esta conversación ha terminado —dijo poniéndose en pie—. Una vez más, Doctor, ha abusado de nuestra hospitalidad. He tomado mi decisión. La Lágrima se utilizará. Pero ten por seguro, a pesar de sus acusaciones de lo contrario, que no somos monstruos. Si hay una manera de neutralizar el Ojo y usar la Lágrima de Isha sin ningún… daño colateral, la encontraré. Consultaré la máquina de las posibilidades. 



—¿La qué? —dijo el Doctor. 



—El medio para vuestra salvación —dijo Rassilon crípticamente—, y nada que le concierna. Ha llegado la hora de que se vaya. Vaya y vuelva a revolotear por el universo, entrometiéndose en los asuntos de especies inferiores. 



—He visto algunas “especies inferiores” en mi vida, Rassilon, pero ninguna tan baja como se han rebajado los Señores del Tiempo. 



Rassilon levantó su mano enguantada, golpeando fuertemente al Doctor en el pecho con un dedo índice extendido. El hombro del Doctor se sacudió hacia atrás, pero conservó el equilibrio. 



—Fuera. ¡Ahora! 



—Vamos, Cinder —dijo el Doctor sin quitar los ojos de encima a Rassilon—. Está claro que no hay nada más que hacer aquí —dio un paso atrás y después se volvió y la agarró por el brazo, llevándosela a toda prisa hacia la salida. 



Cinder miró por encima del hombro para ver a Rassilon aun de pie con el brazo levantado y el dedo señalando la puerta. 

Una vez en el pasillo, el Doctor empujó a Cinder a un lado y la detuvo en seco. Miró por el pasillo, comprobando si había guardias. No había nadie allí. 



—¿Qué…? —empezó. 



Puso un dedo en sus labios para hacerla callar. Cinder frunció el ceño, echándole una mirada de duda burlona. 



—Quiero ver qué es lo siguiente que hace —susurró—. Tengo la sensación de que debería saber de qué trata todo esto de la “máquina de las posibilidades”. 



Cinder asintió. Vio como el Doctor se arrastraba de nuevo hacia la puerta abierta que conducía a la Sala del Consejo. Se detuvo justo antes, mirando alrededor. Con un encogimiento de hombros, Cinder decidió unirse a él. Se escabulló detrás de él y poniendo una mano en su espalda para hacer palanca, se asomó por encima de su hombro. 

Rassilon aún estaba dentro de la sala, de pie de espaldas a ellos. Mientras miraba, se volvió y subió a la pequeña plataforma que había notado anteriormente, situada entre los dos espolones negros. Hizo un ajuste a algo en el panel de control, pulsó un botón y después, con un destello de luz chispeante, parpadeó y desapareció. 



Sintió que el Doctor exhalaba un suspiro de alivio. Caminó hacia la puerta. 



—¿Qué le ha pasado? —susurró confusa—. ¿Dónde ha ido? 



—Dos preguntas muy diferentes con dos respuestas muy diferentes —dijo el Doctor—. Este dispositivo es un transmat, diseñado para transferir materia. 



—¿Teletransportación? —preguntó Cinder. 



—En cierto sentido —dijo el Doctor—.Eso es lo que le pasó. En cuanto a dónde se fue… —miró por encima del hombro y después entró en la sala, cruzando el dispositivo transmat. Saltó a la plataforma y golpeó los controles, frunciendo el ceño ante la pantalla—. Ah, sí. Justo lo que pensaba. 



—¿Y bien? —dijo Cinder—. No me tengas en la inopia. 



—Se ha ido a la Torre —dijo el Doctor. Estaba claramente distraído, tratando de desentrañar su próximo movimiento.

 

—Bien —dijo Cinder—. Ahora me queda más claro —cruzó los brazos en el pecho. 



El Doctor levantó la vista de los controles del transmat. 



—Escucha, Cinder. Vuelve a la Sala de Observación. Espérame allí. No tardaré mucho. Intenta mantenerte alejada de los problemas.

 

—¡Espera! Un momento... ¿Adónde vas? —dijo. 



—Voy tras Rassilon —respondió antes de parpadear hacia la nada. 



Cinder se quedó mirando fijamente a una plataforma vacía en una sala vacía. 



—Genial —dijo.