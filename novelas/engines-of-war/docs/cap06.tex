\chapter*{Capítulo Seis}
\addcontentsline{toc}{chapter}{Capítulo Seis}

—¡Shhh! 

—¡No he dicho nada! —dijo el Doctor.

—No, tus pies —dijo Cinder entre dientes—Por la gravilla no. Camina por el barro.

El Doctor la miró como si estuviese loca.

—Pero se me van a llenar las botas de porquería —dijo—. Van a poner perdida la TARDIS. ¿Y quién lo limpia después? ¿Tú?

Cinder puso los ojos en blanco.

—Pues sí, si es necesario. Hazlo y calla. Es mejor ensuciarse las botas que acabar en una cuneta con un agujero en el pecho. Ya casi hemos llegado. El lugar estará plagado de Dalek.

El Doctor chasqueó la lengua dramáticamente, pero le hizo caso y se apartó del camino.

Habían llegado a la periferia de Andor, justo a la frontera de los muros de la ciudad. Las paredes se habían ido derrumbando durante la ocupación Dalek, y ahora no eran más que obstáculos hechos de escombros y bloques rotos. Le recordaban a una imagen que había visto de pequeña en uno de sus álbumes de fotos, los de una ciudadela de la antigua Tierra asentada en un afloramiento escarpado al pie del mar.

Tenían la similitud de que, si te acercabas a la ciudad correrías peligro y, lo más preocupante, estarías expuesto.

Saltaba a la vista que Andor había sido una vez espectacular, una joya en el corazón de una colonia. Lo que había comenzado como una colección heterogénea de arquitectura práctica: bloques de convivencia, escuelas básicas y centros cívicos portátiles... se había convertido, a lo largo de los años, en una pintoresca metrópolis.

Aquí, los edificios de una miríada de culturas de la Tierra original se erigían apoyados unos contra otros: iglesias, rascacielos, teatros y santuarios... y numerosas líneas delgadas de pasarelas aéreas atravesaban el cielo. Muchos de ellos estaban ahora en ruinas, derruidos durante el bombardeo. Los edificios se habían abandonado hacía mucho también, ya que los supervivientes, como Cinder, se las arreglaban para sobrevivir en las ruinas periféricas mientras los Dalek se instalaban en la ciudad.

Cinder le hizo señas al Doctor para que se acercara a la carcasa de una vivienda en donde estaba agachada. Plantas trepadoras ascendían por la fachada, la única cosa que quedaba viva en este lugar olvidado de Dios.

Agachándose para ocultarse, el Doctor se arrastró hasta ella de cuclillas.



—Allí —dijo ella, señalando hacia la enorme brecha en las paredes de la ciudad —¿ves esas cúpulas?—el Doctor asintió—. Son edificios Dalek. Ocuparon una antigua escuela y la adaptaron para asentarse en ella. Creemos que es su base de operaciones.

—¿Y la gente? —dijo el Doctor—. Los que traen aquí a la ciudad. ¿Dónde están?

Cinder se encogió de hombros.

—Nadie lo sabe. Se los llevan a esas cúpulas para “procesarlos’’ y nunca los volvemos a ver. Al principio, se especulaba con lo que estaba pasando allí dentro, pero después de un tiempo todos dejamos de hablar de ello. Creo que dimos por sentado que estaban muertos. Nunca he oído de nadie que saliera de allí vivo.

—Entonces tenemos que ir allí —dijo el Doctor.

Cinder sacudió la cabeza de lado a lado.

—Oh no, eso no es lo que acordamos. Dijiste que tenías que echar un vistazo. Ya lo has echado. Ahora volvemos a tu TARDIS y salimos volando tan lejos de aquí como podamos.

—Cinder, tengo que ver lo que están haciendo con esa gente. Si los Dalek los están matando sin más, ¿por qué se esfuerzan en desplazarlos y meterlos aquí? ¿Por qué no los exterminan a la primera de cambio y ya está? Ése es el modus operandi de los Dalek, ¿no? No se los conoce precisamente por su piedad, que digamos —se atusó la barba pensativamente—. Están tramando algo, y quiero llegar al fondo de todo esto.

Cinder, frustrada, lanzó una piedra. Rebotó sobre el camino de gravilla hasta golpear la pared de enfrente. En el fondo, sin embargo, siempre había asumido que esto iba a pasar.

—Puedes esperar aquí, si quieres —dijo el Doctor—. No tardaré.

—No pienso dejar que te metas ahí solo —dijo—. Especialmente si vas desarmado —lo que ella estaba pensando, sin embargo, era: si los Dalek te encuentran fisgoneando por ahí, no tendré forma de averiguar cómo se conduce tu nave. Y además, le estaba empezando a caer bien.

Oyó un leve chirrido metálico que venía de alrededor de diez metros de distancia, y rápidamente se agachó detrás de la pared. El Doctor también lo oyó, e hizo lo mismo. Se asomó por encima de la pared, sus ojos relucían.

—¿Qué ha sido eso? —susurró ella—. ¿Ves algo?

—Por allí —dijo el Doctor, inclinando la cabeza—. Vienen hacia nosotros.

Cinder se giró y echó un vistazo por un agujero que había en la pared. A través de la espesa hiedra, pudo distinguir una larga fila de humanos, de entre quince y veinte personas, avanzando hacia las puertas de la ciudad. Parecían exhaustos, pálidos y moribundos. Iban rodeados por al menos cinco Dalek, dos de los cuales estaban flotando, uno a cada lado de la fila, escaneando las ruinas en busca de signos de rebeldía.

Bajó la cabeza cuando un ojo apuntó en su dirección. Contuvo la respiración, esperando el ladrido de una voz, o el disparo de un arma de energía. Por suerte, no ocurrió nada. Parecía que los Dalek estaban más preocupados por transportar a sus prisioneros.

Pasaron cuatro, cinco minutos, tiempo durante el cual ninguno de los dos se atrevió a moverse o a hablar. Oyeron los sonidos de las puertas de la ciudad al abrirse, el distante graznido de dos Dalek intercambiando órdenes y el gemido de un humano sucumbiendo finalmente al miedo o a la fatiga. Cinder quiso pegarse los dedos a las orejas y no escuchar nada.

Los Dalek dieron más órdenes a sus prisioneros, y un minuto o dos después las puertas se volvieron a cerrar tras ellos. Cinder soltó aire lentamente, después de lo que habían parecido horas.



—Se han ido —dijo el Doctor echando una vista rápida—. Deberíamos movernos rápido, ver si encontramos una forma de colarnos por detrás.

Se levantó, le ofreció la mano, y al aceptarla, se paralizó cuando atisbó el punto brillante del ojo de un Dalek que los estaba mirando.

—¡Intruso! ¡Alerta! ¡Alerta!

Sólo se le podían ver la cabeza y el ojo, el brazo manipulador y el arma permanecieron ocultos bajo el muro en ruinas.

—¡Elevar! ¡Elevar!

—¡Vamos! —el Doctor la levantó de un salto—. ¡Corre!

—¡No! —gritó ella, soltándose de la mano. Levantó el arma por encima del hombro y una tira de cuero improvisada, y se la llevó a las manos en busca del gatillo.

El Dalek se había comenzado a elevar progresivamente en el aire.

—¡Extermina...

Hubo una tremenda explosión cuando una descarga de energía salió del extremo del arma de Cinder y se llevó la cabeza del Dalek por delante en el proceso. Atravesó el lateral de un edificio cercano y rebotó por el suelo hasta que se detuvo a unos cuantos metros de ellos. Del cráter donde estaba antes su cabeza empezó a emanar un chorro de líquido.

El Doctor se la quedó mirando.

—Pensaba que esa cosa se había quedado sin munición —dijo sorprendido pero claramente aliviado.

—La volví a recargar en el campamento —dijo con una sonrisa—. Me pareció que sería útil.

El Doctor sonrió.

—Bueno, así tendrán algo de lo que hablar. Nos van a alcanzar de un momento a otro. Vamos, mientras tengamos una distracción. Es hora de entrar.

—¿En serio? —dijo Cinder—. ¿De verdad quieres entrar ahí?

—Pensaba que ya lo habíamos hablado —dijo el Doctor.

—Sólo lo comprobaba —dijo Cinder—. Porque es el peor plan que he oído en mi vida.

Un coro de voces de Dalek resonó en la distancia, desde detrás de los muros de la ciudad.

—Me parece que no tenemos mucha elección —respondió el Doctor. Comenzó a caminar y sus botas, a hacer crujir la gravilla—. Vamos. Por aquí.



Mientras los Dalek llegaban al lugar donde se habían estado escondiendo hacía unos instantes, el Doctor y Cinder cogieron precipitadamente carrerilla hasta los muros de la ciudad.

El Doctor fue en cabeza, pegado a la orilla embarrada (irónicamente, notó Cinder) y a las paredes de las casas abandonadas, oculto en las sombras.

Tras ellos, oyó a un Dalek expidiendo un flujo de instrucciones a su vil raza.



—¡Buscar! ¡Ubicar! ¡Exterminar!



Esto era una completa y absoluta locura. Nunca había hecho nada tan temerario en toda su 

vida. Estaba segura de todo esto sólo podía acabar de una única forma... y aun así, era estimulante. Por primera vez hasta donde podía recordar, tenía un propósito aparte de destruir tantos Dalek como pudiera antes de morir. Tenía algo por lo que vivir. Lo que, supuso, era también irónico, dado que estaba caminando directa a territorio enemigo, donde la cosa más probable que se encontraría sería una descarga de energía entre los omoplatos.

El Doctor había alcanzado el pie de la pared y ahora estaba trepando por un montón de escombros, en dirección a una estrecha grieta a través de la cual podría acceder a la ciudad en sí. Era inesperadamente atlético para ser un viejo cascarrabias. Saltó el muro sin preocuparse en mirar a atrás para ver si los Dalek lo habían divisado.



—¡Espérame! —dijo ella entre dientes mientras lo intentaba alcanzar por detrás. Era una escalada abrumadora, pero apenas tenía elección. Era esto o los Dalek.

Los Dalek acababan de encontrar a su camarada muerto y se estaban dispersando para rastrear las ruinas en busca del perpetrador. Cinder se percató de que no tardarían en localizarlos.

Se impulsó, agarrándose de un saliente, pero sus dedos resbalaron sobre el suave granito y quedó colgando de una sola mano. Reprimió un grito de alarma, que salió como un gruñido impropio.

El frío y afilado borde le pinchaba en la mano que la sujetaba, y sentía que perdía agarre. Se impulsó otra vez, pero no llegó a poder sujetarse. Estaba a punto de caerse a las rocas del fondo donde, sin duda, la encontrarían los Dalek, si no se mataba contra las rocas primero. Miró hacia abajo, intentando calcular la caída. Se le fue la vista.

De pronto, una mano la agarró del brazo. Alzó la vista y vio que el Doctor la estaba mirando, sujetándola por la muñeca.



—Deprisa —susurró—. Lugares que visitar, gente que ver.

La arrastró hasta el saliente.

—Estás disfrutando de esto, ¿a que sí? —dijo con un tono de acusación en su voz.

El Doctor sonrió.

—¿Tú no?

Cinder se encogió de hombros, pero esbozó una pícara sonrisa.

—Tal vez —respondió sin sentirse comprometida.

La grieta de la pared parecía mucho más grande desde aquí arriba que desde abajo. Creía que tendría que escurrirse por ella de lado, pero de hecho era lo suficientemente grande como para que pudieran caminar perfectamente de frente. Cuando lo hicieron, Cinder se dio cuenta de que el Doctor todavía seguía aferrado a su mano. No sabía si esto era más para él que para ella, pero no le importó.

Al otro lado había una caída de alrededor de seis metros hasta lo que parecía ser una alfombra de barro pegajoso. Más allá había un descampado que terminaba en una fila de estructuras humanas abandonadas. Lo que sí podía decir era que no había Dalek observándolos. Evidentemente, sus reflejos allí abajo en las ruinas habían demostrado ser una distracción eficiente.



—Tú primero —dijo Cinder, mirando al Doctor—. Fue idea tuya.

—Oh, mejor juntos —dijo.

Cinder suspiró con resignación.

—Muy bien. —Se volvió a asomar por el borde para considerar la prudencia del siguiente movimiento, pero decidió no empezar a ponerse sensible ahora. Era demasiado tarde para eso—. A la de una, a la de dos...

El Doctor saltó, sin soltarla de la mano, y se vio obligada a saltar detrás de él. Aterrizaron de pie y, con un movimiento sincronizado que habría sido gracioso de no ser por las circunstancias, cayeron de rodillas en el barro mojado.

—Puaj —dijo Cinder, soltándose de la mano del Doctor y poniéndose de pie—. Mis leggings están empapados. —Ayudó al Doctor a levantarse.

—No te preocupes —dijo—. Estoy seguro de que habrá algo similar en uno de los armarios de la TARDIS.

Levantó una ceja.

—¿Con que nos gusta la ropa de mujer?

—Sí —dijo, señalando sus pantalones sucios—. Claramente tengo una predilección.

Ella se rió por lo bajo.

—Bueno —dijo él, señalando a los edificios derruidos de delante. 

Estaban muy abandonados, envueltos en oscuridad, con las ventanas rotas y las plantas sobresaliendo inquisitivamente de los agujeros de los tejados.

—Creo que es por allí.

—No —dijo Cinder—. He estudiado los mapas de este sitio. Si quieres acercarte a las cúpulas Dalek tendremos que seguir la pared por aquí durante un rato. Luego podremos atajar, siempre por las sombras. No deberíamos encontrarnos a nadie por el camino.

El Doctor sonrió.

—¿No te alegras de haber venido? Porque yo sí.



Recorrieron despreocupados las calles vacías de la ciudad, pasando viviendas y fachadas de tiendas abandonadas en las que los productos seguían todavía años después en el escaparate, aunque en estado de putrefacción.  

La amenaza de los Dalek era una continua presencia que le deprimió el buen humor que tenía al principio. Podía oír sus voces rasposas gritándose los unos a los otros órdenes sin ton ni son mientras barrían las ruinas en busca del que había destruido a una de sus patrullas.

Cinder no tenía ni idea de cómo iban a salir de ésta. Volver a escalar la pared no era una opción, estaba demasiado alta. Tendrían que encontrar una ruta alternativa para salir de la ciudad, preferiblemente una que no estuviera vigilada por Dalek.

Eso, sin embargo, era para otro momento. Ahora mismo, tenía que concentrarse en llegar hasta la base Dalek sin saltar ningún sistema de alarma o enfadar a ninguna patrulla.

Se detuvo en la esquina de una intersección, poniéndole al Doctor una mano en el pecho para que no siguiera, y miró a su alrededor. Al final de un largo callejón vio la curva de una de las bóvedas Dalek, una superficie plagada de bulbos idénticos. Antes de eso, sin embargo, había un único Dalek de espaldas a ellos, cuyo ojo se meneaba de lado a lado como si estuviese vigilando.

Retrocedió.

—Dalek —susurró.

—La verdad es que no me esperaba encontrar uno de esos aquí —susurró el Doctor.

Cinder le dio un suave puñetazo en el hombro.

—En serio, ¿qué hacemos? Si le disparo desde tan cerca, lo oirán. Nos rodearan antes de lo que tarda en cantar un gallo.

El Doctor asomó la cabeza para evaluar la situación por sí mismo.

—¿Y si se lo pedimos con educación? —dijo—. ¿Si le decimos que estamos perdidos y que queremos volver a nuestras celdas? Es una forma tan buena como otra de entrar.

Cinder lo miró como si estuviese pirado.

—Mi libertad es mucho más importante para mí que entrar dentro de esa cúpula —dijo—. Y mi vida. Tengo mis límites.

El Doctor sonrió.

—En ese caso, rodeemos.

Retrocedieron hasta que encontraron una brecha entre dos filas de casas que formaban un estrecho callejón. Silenciosamente, la atravesaron chapoteando sobre los residuos tóxicos que emanaban de forma constante de los desagües.

—Vamos, aquí —dijo el Doctor atrayéndola a la puerta de una casa vacía. 

Parecía relativamente intacta, un bloque normalito de habitaciones prefabricado para una familia. Intentó abrirla, pero estaba cerrada.

Cinder vio cómo sacaba su destornillador del hueco de su cinturón de munición que llevaba atado diagonalmente en el pecho, y como jugó durante un minuto con los ajustes. Apuntó el extremo hacia la cerradura y presionó el botón. La punta se iluminó y emitió un chirrido electrónico. Segundos después, oyó cómo el mecanismo de bloqueo se abría.

—¿Qué has hecho? —preguntó ella.

—Agitar unas cuantas moléculas —susurró, dándose toquecitos en la punta de la nariz—. Entremos —la llevó dentro del edificio.

Sin el brillo parpadeante del Ojo de Tantalus y las tormentas de radiación tronando encima, el interior estaba oscuro. La única luz que había se estaba filtrando por las grietas entre el liquen que crecía por encima de los cristales del piso de abajo, pero sólo consiguió ver cuando sus ojos se ajustaron a la penumbra.

Tragó saliva. Se sentía como si tuviera el corazón en la boca. La sala en la que habían entrado estaba distribuida como si la familia que la había ocupado hubiera sencillamente recogido y marchado, como si se hubieran levantado y salido con toda la intención de volver más tarde para coger lo que se habían dejado. Los juguetes de los niños estaban desperdigados por la alfombra. Un vaso vacío descansaba en una mesita de noche. El marco de un cuadro en la pared proyectaba todavía la imagen holográfica de un hombre y una mujer, envueltos en un feliz abrazo.

Cinder sintió el peso de la culpa, de la inmensa tristeza sobre sus hombros. ¿Cómo es que había sobrevivido todo este tiempo cuando los Dalek habían tomado la vida de esta gente y sus familias? ¿Qué derecho tenía a seguir viva? ¿Cómo podía seguir viviendo cuando su madre, padre y hermano habían sido exterminados?

Toda su vida hasta ahora se había esforzado en erradicar esos recuerdos, esos insidiosos pensamientos de culpabilidad; en enterrarlos bajo violencia y venganza, en convertirlos en el odio ardiente hacia los Dalek que ahora recorría el mismísimo núcleo de su ser.

Nunca se había planteado en intentar rescatar a nadie, en intentar cambiar las cosas. Siempre le había parecido muy inútil, inalcanzable para ella. Y por eso se había conformado con tirotear las patrullas Dalek que pasaban casualmente, o con emboscarlos en las ruinas de su antiguo hogar. Cada muerte se convertía en una victoria para ella.

Entonces el Doctor apareció, cayendo del cielo en su caja mágica, y en unas cuantas horas la había obligado a abrir los ojos, a reconocer que tal vez había algo que hacer, que nada era tan imposible como parecía. Había distintas formas de contraatacar. No estaba muy segura de lo que pretendía hacer con la información de la que se enterara aquí en Moldox, pero sabía que no era simplemente por su propia satisfacción. Se estaba involucrando, porque quería ayudar, porque quería detener todo esto.

Había descubierto que todo lo que había estado haciendo había caído en saco roto. Esas victorias que había ido acumulando en el cañón de su arma no significaban nada, ninguna de ellas. No había cambiado nada, no había marcado nada de diferencia. Había gastado muchísimo tiempo.

Aun así algo en su interior sabía que todavía había tiempo de cambiar eso. Había seguido al Doctor hasta aquí, un Señor del Tiempo que apenas conocía, y ahora, de pie en los restos de Andor, acababa de darse cuenta de que podía representar su salvación. Esto no se trataba sólo de ayudarla a huir de su antigua vida. Se trataba de mostrarle cómo cambiarla por sí misma. Y él lo sabía.

Miró a su alrededor para buscarlo y se dio cuenta de que ya se había adentrado en la casa. Oyó sus pasos subiendo por las escaleras y lo siguió.

Cinder lo encontró en el cuarto de los niños en el segundo piso, delante de la ventana, cuyas coloridas cortinas había corrido para poder ver la base Dalek. Se acercó a su lado.

A esta distancia, las estructuras Dalek no parecían tan sofisticadas como había imaginado... De hecho, parecían enfrentarse entre sí, con las estrechas pasarelas de metal conectando las cúpulas entre sí. Había cinco en total, formando un gran círculo alrededor de un patio central. Eran extensas y casi idénticas, en forma de disco con una torreta central en alto, y decoradas con los mismos patrones de bronce y oro de los propios Dalek.

La base tenía un diseño práctico y económico que tenía muy poco o nada que ver con la estética y todo que ver con la función. Todo el lugar ofrecía una sensación de temporalidad y transitoriedad, pese al hecho de que había estado en el mismo sitio desde hacía mucho más de una década.



—¿Qué son? —dijo Cinder.

—Naves espaciales —dijo el Doctor—. Embarcaciones Dalek. No se han apropiado de la antigua escuela, como mucho la han nivelado y han aterrizado sus platillos encima. Han erigido pasarelas entre las naves, pero son sólo estructuras temporales. Podrían desmantelar la base entera en cualquier momento. Está claro que no tienen intención de instalarse en Moldox.

—¿Entonces qué están haciendo aquí? —preguntó Cinder. Siempre había supuesto que la ocupación era porque los Dalek querían controlar el planeta. Nunca se le había pasado por la cabeza que podría haber otro propósito menos permanente.

—Esa es una pregunta de la que estoy ansioso por conocer la respuesta —dijo el Doctor.

Cinder creyó haber visto movimiento en el patio y se asomó hasta que su nariz estuvo a punto de tocar el cristal sucio de la ventana. Entornó los ojos, intentando ver lo que estaba pasando. Había en efecto movimiento, gente, de hecho, un grupo de humanos a quienes estaban llevando hasta el espacio pavimentado que una vez había sido un patio infantil.

Los focos se encendieron de repente, la aturdieron, mientras todo quedaba expuesto a la luz. Tres Dalek estaban empujando a los prisioneros humanos (alrededor de diez, tanto hombres como mujeres) para que se pusieran en una larga fila, hombro contra hombro. Cinder no podía oír nada a esta distancia, pero se podía imaginar las amenazas que estaban lanzando los monstruos de metal para que éstos cooperaran.

El Doctor puso una mano en el alféizar y se asomó con interés para observar.

¿Por qué estaban haciendo una fila?

—¡Oh, no! —dijo Cinder al caer en la cuenta—. ¡Van a ejecutarlos!

—Tal vez —dijo el Doctor con un grave gruñido—. Pero lo vuelvo a decir, ¿por qué hacerlo así? ¿Por qué tomarse la preocupación de capturarlos, de llevarlos hasta aquí medio muertos de hambre, sólo para ponerlos en fila en un patio y ejecutarlos? Tiene que haber algo más.

Cinder no quería mirar, estaba aterrorizada de lo que pudiera ver, pero estaba paralizada, fue incapaz de apartar la vista. Mientras observaba, los tres Dalek se alejaron, perdió a dos de ellos de vista, pero otro apareció por otro lado.

Este tenía una silueta ligeramente diferente, pero familiar.

—Ése es como el que vi durante la emboscada —dijo Cinder—. El que decapitaste cuando te estrellaste. Es uno de los mutantes, una Degradación.

Era igualito a la cosa monstruosa con la que se había encontrado antes ese día, con el tamaño y la forma de un Dalek normal, salvo por el hecho de que la mitad de su cuerpo había sido reemplazada por un enorme cañón negro.

—Eso no es una Degradación —dijo el Doctor—. Es diferente. Es algo nuevo.

El Dalek rotó para mirar hacia la fila miserable de prisioneros humanos. Uno de los otros Dalek apareció, y Cinder supo que estaba hablando por las luces parpadeantes de su cabeza esférica.

En respuesta, el Dalek del cañón cargó su arma. Un aura de intensa luz de color rubí salió del final del cañón. Hubo una repentina y masiva descarga y el arma disparó un rayo de luz rosa que engulló a cuatro personas mientras éstas gritaban e intentaban huir.

Los prisioneros que quedaban se apartaron de ellos, claramente aterrorizados ante su probable destino.

Las cuatro víctimas se retorcieron en agonía mientras la luz rosa calaba sus cuerpos, borbotando de sus bocas, de sus ojos, y filtrándose a través de su piel. Entonces, como si su carne fuera simplemente incapaz de contener tal cantidad de energía, se abrieron, sus formas se disolvieron y la luz rosa se intensificó antes de dispersarse y desaparecer como hilillos de humo.

Cinder se apartó de la ventana con ganas de vomitar. Se puso la mano en la frente. Sabía que algo iba muy mal, pero no podía concretar el qué. Miró al Doctor y lo agarró del brazo para estabilizarse.



—¿Qué acaba de pasar? —dijo—. Estoy segura de que algo horrible acaba de pasar, pero, ¿qué era?

Volvió a mirar hacia el patio, donde los Dalek estaban sondeando a los seis prisioneros que habían traído hasta el lugar hacía unos minutos.

El Doctor se alejó de la ventana y, agarrando a Cinder del antebrazo, la apartó de allí también.

—Es un arma temporal —dijo—. Un arma de desmaterialización. Los Dalek han desarrollado una nueva arma, un nuevo paradigma, que tiene el poder de erradicar a una persona de la historia.

—¿Cómo lo sabes? —dijo Cinder—. ¿Cómo puedes saberlo con tan sólo mirar?

El Doctor entrecerró los ojos.

—¿No lo acabas de ver? ¿No acabas de ver lo que acaban de hacer con esas cuatro personas?

Cinder se soltó del Doctor. Volvió a la ventana. No, allí había seis personas, igual que antes.

—¿Cuatro personas? —dijo—. Allí abajo sólo hay seis.

Pero cuando lo dijo, supo que había algo erróneo. Podía sentirlo, desvaneciéndose en su interior. Se estaba olvidando de algo. ¿Es que no podía ya ni fiarse de su propia mente?

—Es el arma, Cinder. Eso es lo que te está haciendo esto —dijo el Doctor—. Ese cañón: puede borrar la línea temporal de una persona de la historia, hasta el mínimo rastro de ella, como si nunca hubiese existido. Eso es lo que le ha pasado a tu amigo, allí en las ruinas, la persona cuyo catre estaba al lado del tuyo en el campamento, aquel a quien no podías recordar. Tu mente se está esforzando en comprenderlo. Sabes que hay algo mal, que se te escapa algo. Los recuerdos siguen allí, enterrados en tu cabeza, pero ya no son congruentes, ya no están relacionados con la persona que conocías o viste, porque la realidad se ha enrollado a tu alrededor.

Cinder sacudió la cabeza, como si intentara aclararse. No lo entendía. ¿Un arma que no sólo te mataba, sino que también reescribía la historia como si nunca hubieras nacido? Era la cosa más atroz que había oído en su vida. La pura violencia, de no sólo quitar una vida, sino de deshacer cada acción, cada pensamiento, cada emoción que mostrara o experimentara esa persona... tenía que ser el aparato más malvado nunca antes concebido. Se secó las lágrimas de los ojos, recordando el dolor, si no las personas.



—Lo siento —dijo el Doctor—. De veras. Pero ese viaje en la TARDIS va a tener que esperar un poco más. Si los Dalek son capaces de diseminar este arma, la Guerra estará perdida —se acercó a ella, la envolvió con los brazos y la acercó hacia su pecho para abrazarla—. Voy a evitar que le hagan esto a más gente.

Sorbiéndose las lágrimas, Cinder se apartó del Doctor. Lo miró con una mirada desafiante. Su decisión se solidificó.

—Cuenta conmigo —dijo—. No importa lo que pase, te ayudaré a detenerlos.

El Doctor esbozó una triste sonrisa.

—Esa es mi chica —dijo.

