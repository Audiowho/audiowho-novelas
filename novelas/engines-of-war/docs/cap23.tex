\chapter*{Capítulo Veintitres}
\addcontentsline{toc}{chapter}{Capítulo Veintitres}

El Doctor emitió un lamento que estaba a medio camino entre un aullido de angustia y un furioso grito de guerra.

Tiernamente, bajó la cabeza sin vida de Cinder al suelo y cruzó los brazos sobre el cuerpo en reposo reparador. Tenía los ojos cerrados y parecía en paz, como si simplemente estuviera durmiendo.

Se levantó de un salto y corrió hacia la consola, con la vista borrosa por las lágrimas que se negaban a rodar por sus mejillas.

--¿Doctor? --dijo Borusa. Su voz era seca y áspera.

El Doctor no levantó la vista. Estaba ocupado estableciendo una nueva ruta de vuelo.

--¿Doctor? --repitió--. Es la hora.

--¿Cree que no lo sé? --le espetó el Doctor en respuesta--. ¿No cree que soy consciente de lo que hay que hacer? --sus dedos bailaban sobre los controles de la nave.

En el monitor la vista de la cámara de audiencias Dalek había sido sustituida por el torbellino tormentoso del Vórtice del Tiempo.

El Doctor miró el cuerpo tendido de su compañera. Quizá aún había tiempo. Tal vez, con ayuda de Borusa, aún podría encontrar una manera de salvarla. Todavía no estaba listo para darse por vencido con ella.

Se agarró a la consola cuando la TARDIS se precipitó a través de la envoltura exterior del Ojo de Tantalus, zarandeada por las tormentas temporales que se derramaban de su cúspide.

En las profundas entrañas de la nave la Campana del Claustro comenzó a sonar de nuevo. Los motores aullaban, balbuceando y chillando en protesta cuando el Doctor los forzó en la anomalía. La integridad de la nave estaba siendo probada hasta el límite mientras trataba de mantener el rumbo, hundiéndose cada vez más profundamente en el Ojo.

Algo en la consola explotó, enviando una ducha de chispas al aire. Las diminutas luciérnagas encendidas cayeron sobre la chaqueta del Doctor, quemándole el cuero allí donde cayeron. Las ignoró, agarrándose al borde de la consola cuando la sala comenzó a sacudirse violentamente. Cables que colgaban se desprendieron de sus conectores en el techo, oscilando sueltos como lianas en una selva. Un redondel de la pared de la izquierda detonó, las llamas lamieron con avidez el interior de la carcasa rota.

El Doctor sabía que la TARDIS no podría aguantar mucho más, que se arriesgaba a que se hiciese pedazos en las tormentas, marchitándose hasta la nada. Sin embargo no tenía elección. Era la única manera que se le ocurría para conseguir que Cinder volviese.

Había muerto bajo su cuidado, protegiéndolo. Era su responsabilidad y no podía permitir que sucediese de nuevo. No ahora. No Cinder.

Miró a Borusa que todavía estaba anclado a la estructura de metal de la máquina de posibilidades, encajado entre dos pilares de piedra al otro lado de la consola. Justo como el Doctor había previsto, Borusa estaba comenzando a absorber la radiación temporal de la anomalía. Sus ojos ardían como brasas mientras observaba la curvatura y la trama del tiempo mismo. El ciclo de sus regeneraciones se aceleraba, su rostro era un parpadeante borrón, cambiando en su ciclo de constante cambio. Su carne ahora brillaba vivamente con una luz interior. Su proximidad al centro de la tormenta le estaba alterando.

Aun así, la TARDIS se sumergió, buceando cada vez más profundamente en la grieta espacial.

El Doctor recorrió tambaleándose la consola, apenas capaz de mantener el equilibrio mientras la nave se sacudía. Se puso de pie ante Borusa, colgándose de los pilares de piedra como apoyo. 

--¿Puede verlo? --gritó, por encima de los ruidos desgarradores que provenían del exterior de la nave--. ¿Puede ver un hilo de posibilidad en la que aún viva?.

Borusa emitió un gemido sordo, como si el acto mismo de hablar fuese demasiado difícil de soportar. 

--Puedo verlo --dijo. Su voz había cambiado, tanto que ahora, con cada palabra, cada una de sus regeneraciones hablaba a coro. Era hermoso y discordante--. Puedo verlo todo.

El Doctor pudo ver, sin embargo, que Borusa empezaba a consumirse. La falta de armonía temporal del Ojo le estaba sobrecargando mientras trataba de canalizar toda su energía para abrir su mente y que todo fluyese a través de él.

El Doctor sólo tenía una oportunidad. Podría salvar a Cinder. Podría decirle a Borusa que seleccionase ese hilo en que ella sobrevivía, decirle que tirase de él para desenredarlo y que se hiciese realidad, para que la historia se reescribiese hasta un punto en el que Cinder viviese y Karlax muriese. Ese era el verdadero potencial de la máquina de posibilidades. Aprovechar toda esta gran energía, todo ese enorme potencial, Borusa no podría hacer nada. Podía elegir una línea de tiempo y forjar la realidad a su alrededor. Podía tejer el universo de una manera distinta, dar vida a los que habían muerto, o tomarla de los que no eran dignos. Era la encarnación de la vida y la muerte, el portador del apocalipsis, el heraldo del olvido.

Esta era su oportunidad de traerla de vuelta, de resucitarla. Siempre había hecho lo que era necesario, lo que nadie más haría. Nunca se había permitido un solo momento egoísta, un solo segundo de debilidad, en el que elegía el futuro que quería, y no lo que otros necesitaban de él. Tal vez era justo que consiguiese hacerlo ahora, en el corazón de la tormenta. Le diría a Borusa que lo hiciese, usar la máquina de posibilidades para traerla de vuelta.

--Entonces… --el Doctor se detuvo. ¿Qué era lo que le había dicho cuando yacía en sus brazos?. Mi vida a cambio de las de billones. Eso era para lo que había dado su vida, para asegurarlo. Habían venido aquí con un solo propósito: impedir que los Dalek usasen el poder del Ojo de Tantalus, que moldeasen su energía a su voluntad, para liberar a su pueblo y acabar con la amenaza sobre los Señores del Tiempo y el cosmos en general.

Si él lo usaba para esto, ¿era él mejor?. ¿Tenía derecho?. Sabía, en el fondo de sus corazones, que no lo tenía. Si ella estuviese aquí, ahora, nunca permitiría que lo hiciese, sacrificar sus posibilidades en aras de una vida.

Se tambaleó cuando la sala de la consola dio un bandazo y otro redondel explotó, duchándoles con fragmentos rotos de vidrio. Borusa rugió de dolor cuando la energía recorría su mente y crecía en intensidad. Se estaba convirtiendo en un conducto para la energía, un centro de coordinación para toda la energía temporal que bullía en el corazón del Ojo.

No había elección. Tenía que terminar el trabajo y liberar a Borusa de su carga. 

--Borusa, es la hora. Un trabajo final para la máquina de posibilidades y después serás libre. Tenemos que destruir a los Dalek. Debes encontrar un hilo de posibilidad en el que ellos nunca tengan control sobre la Espiral Tantalus y que sus Armas Temporales nunca se dispersen. Destruye el Circulo de la Eternidad y su arma con la energía del Ojo. Elimínalos del universo como si nunca hubieran existido. Hazlo ahora.

Borusa echó la cabeza hacia atrás y gritó en todas sus muchas voces. Fue  un sonido que el Doctor sabía que resonaría a través de todo el tiempo y el espacio, y que lo perseguiría por el resto de su vida.

A su alrededor la tormenta parecía amainar momentáneamente y después una onda de luz se escapó del Ojo, un pulso temporal, una sola y enorme explosión de energía que se extendió hasta circundar toda la Espiral. En su estela fueron reescritos los hilos de posibilidad.

Flotas enteras de platillos Dalek y naves sigilosas se disolvieron cuando una luz de color rubí se estrelló sobre ellos, abrasándolos hasta que no quedó nada.

A bordo de la estación de mando Dalek, los Dalek del Circulo de la Eternidad, aun descansando en sus pedestales, dejaron de existir en un suspiro, brillando en la no vida como fragmentos de un sueño que se desvanece, solo recordados a medias.

En Moldox y una docena de otros mundos, la última de las patrullas Dalek y sus hordas de Degradaciones fueron vencidas, borrados en fragmentos de luz ante los ojos de sus prisioneros humanos.

Todo terminó en cuestión de segundos.

La TARDIS, devastada por la tormenta, salió renqueando del Ojo por su propia voluntad, a la deriva en una órbita holgada alrededor de Moldox. En su interior los restos de la máquina de posibilidades burbujeaba y estallaba, donde una vez Borusa había estado anclado, ahora no había nada.

El Doctor yacía inconsciente en el suelo debajo de la consola con un brazo alcanzando el cuerpo de su compañera muerta.