\chapter*{Capítulo Quince}
\addcontentsline{toc}{chapter}{Capítulo Quince}

Cinder se retorció. Un lado de su cara estaba presionando contra algo frío y duro. Su cabeza reverberaba como si alguien estuviera usando su cráneo como batería, ¡bang!, ¡bang!, ¡bang!, y por un momento no tuvo ni idea de dónde estaba, o qué había hecho para llegar hasta aquí. ¿Estaba de resaca otra vez?. Estaba segura de que no había habido fiesta anoche. Había salido de emboscada, pero luego algo ocurrió, y...
Se sentó de un salto e, instantes después, cuando el mundo se le echó encima y desfalleció, deseó no haberlo hecho. Unas luces bailaron ante sus ojos como manchas que le oscurecieron la vista. Respiró hondo, lo cual le dolió un poco, y rompió a toser. Parpadeó para despejarse.
Estaba en una celda, sobre una losa de piedra rugosa. Al otro lado de la sala, yacía el Doctor sentado contra una pared con los pies proyectados hacia adelante. La miró miopemente.

--Hola --dijo.

--¿Dónde estamos? --dijo ella. Su boca estaba seca. Se frotó la nuca.

--En una celda --dijo redundantemente.

--¿Una celda?.

--Sí, bajo el Capitolio de Gallifrey. ¿Recuerdas...?.

--La sonda mental --dijo--. ¿Cómo podría olvidarla?.

El Doctor suspiró.

--Lo siento --dijo--. No debería haberte dejado sola. No debería ni haberte traído aquí, a Gallifrey, y meterte en todo este embrollo.

Cinder se masajeó las sienes.

--Como te dije, allá en la base Dalek en Moldox, estamos juntos en esto --dijo--. Aunque admito que no me imaginaba que acabaríamos metidos en una celda --meditó durante un momento--. ¿Por qué estamos en una celda?.

--Ah --dijo el Doctor--. Es una larga historia.

--Le dijiste a Rassilon que se fuera a cagar, ¿a que sí? --dijo. Sonrió--. Por dónde se podía ir a meter su Lágrima de Isha.

El Doctor se echó a reír.

--Más o menos --admitió--. Tal vez con menos vulgaridad.

Cinder se encogió de hombros.

--Tal vez un poco de vulgaridad sea lo que necesite. Bueno, tal vez mucha.

--No te lo niego --dijo el Doctor.

Cinder estudió la celda. Vaya con la celda. Ni fontanería, ni calefacción, ni pantallas, libros o pizarras de datos... solo cuatro paredes de piedra, una losa de piedra y una puerta. El suelo estaba revestido de baldosas desniveladas y cubierto de una gruesa capa de polvo. El Doctor yacía sobre él. La única luz provenía de un pequeño panel en el techo, tenue y diluida. 

--Bonito lugar que te has agenciado --dijo ella--. Me gusta lo que has hecho con ella.

El Doctor hizo un gesto de dolor.

--Es positivamente medieval --dijo.

--¿Mediqué? --dijo Cinder.

--Barbárico --respondió el Doctor--. Poco imaginativo. Primitivo.

La cabeza de Cinder todavía seguía girando. A ella, toda la situación le parecía de algún modo surrealista.

--¿Cuánto llevo inconsciente? --dijo.

--Dos o tres horas --respondió el Doctor--. Aguantaste bien los efectos de la sonda mental. Mejor que bien. He visto cómo las mentes de intelectos mucho más superiores se deshacían ante ella.

--Oh, gracias --dijo.

--¡Era un cumplido! --dijo el Doctor.

--Fijo --dijo Cinder.

El Doctor volvió a reír.

--Sabes, eres bastante extraordinaria, Cinder --dijo--. Conoces tu mente. Eres consciente de lo que quieres, y vas y lo consigues. Es una cualidad envidiable.

--Eso sí que es un cumplido --dijo Cinder--. ¿Ves la diferencia? --Se estiró, y bostezando, arqueando la espalda. Se puso de pie--. Vale, quiero una respuesta honesta y directa, ¿vamos a detonar el arma?.

El Doctor asintió.
--Me temo que sí --dijo. Su voz se hizo más grave--. Intenté detenerlos, pero Rassilon ya tenía algo en mente.

La forma en la que el Doctor soltó su nombre dejó claro que había perdido todo el respeto para el Señor del Tiempo Presidente, si es que le quedaba algo en un primer momento.

--Bueno, todavía no es demasiado tarde --dijo Cinder--. ¿Cuánto tiempo tenemos?.

--¿Hasta que estén listos para detonar? --el Doctor hizo un rápido cálculo en su cabeza--. No más de un par de horas --dijo.

Cinder se puso ante él, ofreciéndole ambas manos.

--Entonces, ¿qué haces ahí de brazos cruzados? --dijo--. No vas a salvar a nadie revolcándote en el polvo y la mugre.

El Doctor la agarró de las manos y dejó que la ayudara a enderezarlo, pero su expresión decía otra cosa.

--Ojalá fuera tan fácil --dijo--. Estamos en una celda de los Señores del Tiempo. A pesar de su aspecto primitivo, no hay forma de salir. Han incautado la TARDIS y no van a dejarnos salir de aquí hasta que la Lágrima se haya detonado y el Ojo de Tantalus haya sido neutralizado.

Cinder lo miró con su mirada más incrédula.

--Cuántas excusas me estás poniendo --dijo.

Era una locura total. Lo sabía. Su corazón dio un vuelco ante la sinceridad de la respuesta del Doctor. Su pecho estaba tenso y podía sentir cómo crecía el pánico, amenazando con dominarla. Simplemente no quería creer que tenía razón, que este extraño en el que había aprendido a confiar había sido derrotado, y que todo el mundo que conocía, todo el mundo remotamente similar a ella, en doce mundos habitados, iba a morir. El Doctor parecía dolido. Todavía la estaba sujetando de las manos.

--No pasa nada --dijo--. Entiendo.

--¡No! --dijo--. No, no entiendes. No puedes ser amable. No puedes sujetar mi mano mientras todo lo que conozco es destruido. Así no es cómo funciona --inspiró aire--. Vas a encontrar una forma de salir de aquí y vas a detenerlos --soltó las manos y le dio un fuerte golpe en el pecho con los dos puños. Sus lágrimas se le derramaban de los ojos--. ¿Entiendes?.

El Doctor la miró con unos ojos tristes y angustiados.

--Si hubiera una forma... --susurró.

Ella sacudió la cabeza.

--Cuando nos conocimos, me dijiste que solías tener un nombre, que ya no te lo merecías. Hoy es tu oportunidad de volvértelo a ganar.

Cinder caminó hasta la puerta de la celda. Estaba hecha de pesados tablones de madera, contenidos en un cinturón de hierro. Había una enorme cerradura mecánica.

--A ver, mira --dijo--. Puedes usar tu destornillador para abrirlo.

El Doctor se puso a su lado. Le puso una mano en el hombro.

--Lo siento, Cinder --dijo--. Ya lo he intentado. Recuerda lo que dije. Esta es una celda de los Señores del Tiempo. La cerradura es inmune a los efectos de los dispositivos sónicos. Por eso ni siquiera se molestaron en quitármelo cuando nos echaron aquí. Estuve buscando una forma de salir durante una hora y media. No pude encontrar ninguna. Si pudiéramos salir de esta celda, tal vez tendríamos una oportunidad. Pero estamos atrapados.

Cinder le dio una patada a la puerta. Ni siquiera se movió un milímetro. El pie rezumaba de dolor. Miserablemente, se tiró en el suelo, frotándose los resentidos dedos que tenía bajo la bota. El Doctor, decidiendo que lo mejor era dejarla sola con sus cosas durante unos minutos, volvió a donde estaba contra la pared y se acomodó.
Cinder echó un vistazo a la cerradura. No parecía tan sofisticada. De hecho, era como las cerraduras humanas de Moldox, un sencillo interruptor de palanca que se abría con una llave. ¿Tan arrogantes eran los Señores del Tiempo de pensar que adaptar una simple cerradura metálica para hacerla inmune a manipulaciones sónicas iba a ser suficiente para mantener cautivos a sus prisioneros?.
Vio un rayo de esperanza. Miró al Doctor, que había cogido su destornillador del hueco de su cinturón de munición y ahora estaba jugueteando con los ajustes, supuestamente en un intento de invalidar los protocolos del cierre.
Se remangó el jersey hasta el codo. A penas se atrevía a mirar. Tal vez...
Seguía allí. Suspiró de alivio. El brazalete que había traído con ella de Moldox, el que su hermano le había hecho cuando era niña, enroscando un bucle de hilos de cobre. De aquella le quedaba grande, pero lo conservó, y cuando Coyne y su equipo la encontraron en las ruinas chamuscadas de su casa, esa fue la única cosa que había podido salvar.
Lo punteó con los dedos, reflexionando. Si lo desenroscaba, puede que el alambre fuera lo suficientemente fuerte para hacer dos ganzúas. Había una parte de ella que no quería hacerlo, que quería desenrollarse la manga de su jersey y acurrucarse, hacer como si nunca se le hubiera ocurrido la idea, pero sabía que no podía. Había demasiadas vidas en riesgo. Su hermano lo habría entendido.

--Lo siento, Sammy --susurró, mientras se quitaba la pulsera y lentamente comenzaba a deshilachar los hilos de metal.

Estaban oxidados por la edad, y por un momento pensó iban a desmenuzarse en sus manos, pero los maleó hasta que gradualmente comenzaron a desenredarse.
Pocos segundos después, la pulsera se separó hasta convertirse en dos alambres distintos. Los enderezó lo mejor que pudo y los puso ante ella en el suelo. El Doctor estaba todavía andando con su destornillador, con una mirada de profunda concentración en su ceño fruncido.
Cinder se puso de rodillas, se acercó a la cerradura y cerró un ojo para poder ver por el agujero de ésta. Al otro lado podía ver un pasillo y una que otra puerta al otro lado de la sala. No había rastro de guardias. Recogió sus herramientas improvisadas del suelo. Cuidadosamente, las insertó en la cerradura, con la esperanza de no recibir una violenta sacudida eléctrica, o al menos de que no saltara ninguna alarma, pero no ocurrió nada. Lentamente, fue avanzando con la ayuda de las barras de metal hasta forzar el mecanismo, tirar de interruptores y abrir un agujero en la pared. Oyó al mecanismo clicar. Apenas había estado con ello unos segundos. ¿Tan fácil era?.
Se dio cuenta de que había estado manteniendo la respiración todo el tiempo y la soltó. Luego, poniéndose de pie, metió las ganzúas en el bolsillo y, con la mano temblando, intentó abrir la puerta. El manillar giró, y la puerta se abrió un par de centímetros. El pulso le iba a reventar los oídos. Suavemente, la volvió a cerrar, y se giró para ver si el Doctor estaba mirando. Seguía toqueteando su destornillador.

--¿Doctor? --dijo, con una voz ligeramente titubeante.

--Mmmm --respondió, apenas escuchando. 

--Lo dijiste tú, si pudiéramos salir de esta celda, tendríamos una oportunidad de evitar que los Señores del Tiempo detonaran la Lágrima.

El Doctor dirigió su vista hacia ella, entornando los ojos.

--Sí --dijo--. Pero tengo que...

Cinder le hizo un gesto en silencio. Se dio la vuelta, viró el mango y dejó que la puerta se abriera de par en par.

--No lo desaproveches --dijo.

El Doctor miró la cerradura, y luego a Cinder.

--Estoy impresionado --dijo.

Ella se encogió de hombros.

--Es obvio que no se esperaban a una mísera humana con una ganzúa.

--No --risoteó el Doctor, poniéndose de pie--. Creo que ninguno de nosotros lo esperaba.

Su chaleco estaba arrugado bajo su chaqueta y sus botas estaban caladas de barro mojado. Parecía estar sucio. Pero claro, supuso, ambos habían estado en guerra durante los últimos días, literalmente. Sin más dilación, salieron de la celda.

--¿Por dónde? --dijo Cinder.

--Por la izquierda, creo --dijo el Doctor, reduciendo la voz a un susurro--. Por suerte, no me he pasado tanto tiempo aquí abajo, pero creo que tenemos que bajar. Debería haber un pasillo delante, a la izquierda.

--¿Bajar? --dijo Cinder--. Pensaba que estábamos en las mazmorras. Porque parecen mazmorras.

El Doctor asintió.

--Hay un sótano subterráneo que se erige justo debajo de la ciudadela principal. Es donde mandan las TARDISes a morir --su voz se rompió al hablar--. Allí es donde ella estará.

El Doctor se dirigió al pasillo, y Cinder lo siguió. Estaba poco iluminado e invadido por la humedad y el frío, y las cuatro o cinco otras celdas que pasaron estaban todas vacías y con las puertas abiertas. Las paredes apenas estaban labradas y no tenían casi adornos, salvo por algún que otro candelabro de lumen. O esta ala de la prisión estaba reservada especialmente para ellos, o los Señores del Tiempo no tenían la costumbre de tomar prisioneros. Cinder le mencionó esto al Doctor mientras caminaban.

--Por lo que sé, últimamente Rassilon prefiere la ejecución antes que el castigo --dijo oscuramente.

Cinder frunció el ceño.

--¿Entonces por qué meternos en una sucia celda? --dijo--. No es que me queje, ni nada.

--Sabe que soy útil --dijo el Doctor--, y de ti te puede sacar ventaja, así de despreciable es.

A Cinder no le gustaba mucho la idea de que la usaran como ventaja, pero al menos la consoló algo saber que el Doctor la defendía, y que no la abandonaría sin más para salvarse el culo, ni la dejaría morir en alguna parte.
Llegaron al final del túnel y giraron a la izquierda, directos al ojo de una guardia que estaba sentada en un taburete contra la pared, y casualmente leyendo detenidamente una tabla de datos.  Era una mujer alta y musculosa, vestida con el famoso uniforme rojo y blanco de la Guardia del Gobernador. Cinder no pudo evitar fijarse en la pistola que llevaba en el cinturón.
Lentamente, la mujer se levantó del taburete, poniendo la tableta de datos en el asiento.

--¡Quietos ahí! --dijo.

Echó a correr hacia ellos, con sus pasos resonando en el confinado pasillo. El Doctor se adelantó para saludarla, extendiendo la mano.

--Hola --dijo.

--A ver, ¿qué están haciendo ustedes aquí abajo? --dijo la mujer--. La prisión está terminantemente prohibida a civiles.

--Ah --dijo el Doctor--. Lo siento. Debimos de habernos metido mal allí atrás. Esto no ha sido más que un malentendido. No se preocupe. Proseguiremos nuestro camino --se dio la vuelta, haciendo como que se iba.

--Espere un momento --dijo la mujer--. Me es usted familiar. ¿No es...? --Sus ojos se abrieron con sorpresa--. ¡Es el Doctor! --dijo--. Se supone que tiene que estar en esa celda. ¿Cómo ha salido? --buscó su pistola.

--Escuche --dijo el Doctor, alargando una mano en un esfuerzo de calmarla--. Como he dicho, es todo un simple malenten...

Cinder dio un paso adelante, impulsó el puño y propinó un buen gancho en la mandíbula de la mujer. Ésta dio un salto, y el arma cayó al suelo.

--¡En serio! --dijo el Doctor--. ¿Era eso necesario?.

Cinder puso los ojos en blanco, masajeándose su mano dolorida.

--Algo me dice que no has pillado toda la dinámica de esta situación --dijo--. Esto es una huida. Ella es una guardia. Se supone que tenemos que escapar.

El Doctor pensó esto durante un instante, y luego se encogió de hombros.

--Bueno, si te pones así... --dijo. Miró a la forma inconsciente de la guardia--. Pongámosla cómoda.

Cinder suspiró, mientras el Doctor arrastraba a la mujer hasta la pared del túnel y la enderezaba, poniéndole las manos en el vientre.

--Así --dijo, desempolvándose las manos--. Nos lo agradecerá cuando se despierte.

--A mí me da que no --dijo Cinder--. Ahora sigamos.

El final del túnel descendió, como el Doctor había predicho, hasta convertirse en una larga y suave cuesta. 

--Por aquí --dijo, haciéndole un gesto con la mano.
Cinder oyó voces provenientes de atrás: dos hombres, gritándose entre sí, alarmados. Claramente habían descubierto a su compañera inconsciente.

--La han encontrado --dijo--. Vamos, será mejor que corramos.

Abandonando toda esperanza de discreción, el Doctor y Cinder comenzaron a correr, bajando a toda prisa la cuesta hacia el profundo sótano subterráneo. Momentos después, oyeron pasos por detrás.
El pasillo siguió bajando durante lo que parecieron kilómetros, retorciéndose hasta que Cinder estuvo completamente desorientada. Estaba cansadísima, los músculos de sus muslos dolían de tanto correr, su cabeza seguía mareada por los efectos secundarios de la sonda mental. Lo único que la mantuvo en marcha fue el sonido de los pasos lejanos que parecían hacerse cada vez más fuertes, alcanzándolos con cada segundo.

--Ya casi estamos --soltó el Doctor, sin aliento.

Delante, el túnel se abría abruptamente, y el suelo se iba nivelando a medida que se desplegaba la boca de una enorme cueva. El Doctor se paró en seco, y Cinder casi chocó contra su espalda, pero lo cogió de la mano para reducir su velocidad, aunque por poco estuvo de llevarse a ambos por delante en el proceso.
El sótano era inmenso, justo como el Doctor había descrito, se desplegaba bajo toda la ciudad. El techo era alto y abovedado, claramente construido hacía al menos un milenio, y tenía unas hileras de luz que atravesaban toda la fachada, proporcionando una cierta iluminación. Las paredes estaban pobremente labradas en piedra, y desaparecían en el horizonte, absorbidas por las sombras.
El suelo de la cueva estaba plagado de las carcasas de TARDISes muertas o a punto de morir. Había miles de ellas, decenas de miles, incluso. Era imposible estimar la cantidad.
Permaneció en la boca de la cueva, contemplando un mar de TARDISes con una miríada de formas y tamaños distintos. Algunas eran cilindros blancos con negras cicatrices de batalla, otras, cápsulas plateadas y grises, con sus superficies marcadas y agrietadas por la edad.
Una que había cerca estaba abierta como un huevo, su interior se desplegaba para crear un paisaje desordenado de locuras geométricas. Cinder no podía darle sentido: las paredes estaban en el techo, el techo estaba en el suelo. La sala de la consola yacía perpendicular a un fragmento de la carcasa exterior, que poco a poco se transformaba en una estantería de pino vacía. Era como observar un extraño paisaje etéreo hecho de acero y madera.
A lo lejos, pudo ver que algunos habían crecido hasta proporciones desmesuradas, sus formas exteriores presionaban contra el techo de la caverna, como pilares asimétricos, soportando así la ciudad de encima.
A su izquierda había uno que se parecía a un platillo Dalek dañado puesto de lado, otro que se asemejaba a un antiguo roble, alzado sobre un entramado de raíces enmarañadas, otros habían adoptado la forma de una columna neoclásica, una carpa de circo, un galeón agujereado, y ahora yacían apilados contra la pared. Había muchos más, pero tenían forma de objetos extraños e inusuales que ella no reconocía, ya que supuestamente pertenecían a otras civilizaciones alienígenas.
Había algo terriblemente desolador en el lugar, en la forma en la que estas naves se habían abandonado aquí hasta el fin de sus días.
--Es un cementerio --dijo.

El Doctor asintió.

--El último lugar de descanso de las amigas --dijo. Se atusó la barba--. Una vez estuvieron vivas, ¿sabes?.

Cinder frunció el ceño.

--Pero son máquinas.

El Doctor negó con la cabeza.

--No. Son mucho más que eso. Deberías intentar huir con una.

Tras ellos, los pasos de los guardias avanzaban.

--¿Cómo vamos a encontrarla entre todas estas? --preguntó Cinder, con repentina urgencia. Estaban perdiendo el tiempo--. Tu TARDIS. Hay muchísimas --agitó las manos para abarcar la inmensidad de la caverna--. Podríamos tardar semanas en dar con ella.

Su afirmación fue interrumpida por un corto pitido electrónico que pareció venir de debajo de la solapa de la chaqueta del Doctor. Levantó una ceja.

--Oh, es buena --dijo--. Una chica listísima. 

Fue a coger el destornillador sónico. La punta se encendió sin más y sin hacer ruido. Después de un segundo emitió otro pitido.

--Nos está llamando --dijo--. Sabe que estamos aquí. Quiere que la encontremos. ¡Vamos!.

Sujetando el sónico en el aire como una antorcha, el Doctor saltó del bordillo y desapareció entre el bosque de TARDISes rotas. Cinder fue en su busca, guiada por el brillo del sónico.

--¡No tan rápido!.

Tras ellos, los guardias emergieron en la boca de la caverna. Cinder no se detuvo a mirarles la cara, sino que se lanzó a protegerse detrás de la cáscara rota de una TARDIS de Batalla.

--Vuelvan a las celdas ahora --exclamó uno de los guardias--, y no dispararemos.

--¡Y una porra! --respondió el Doctor por lo bajo, haciendo que Cinder explotara a carcajadas.

Pero sólo duró un instante, ya que un rayo de energía de una de las pistolas de los guardias le rozó la oreja y chocó contra el galeón, mandando una lluvia de chispas al aire.
Así que no eran para aturdir.
Podía oír el sónico del Doctor pitar delante de ella, los sonidos eran cada vez más altos a medida que se adentraba en el laberinto, y supuestamente se acercaba a su TARDIS.

--¡Por aquí! --gritó, haciendo gestos con su sónico hacia la parte superior de una TARDIS que parecía una barcaza, y esquivando otro disparo de uno de los guardias.

Manteniendo la cabeza gacha, Cinder fue en su busca. Más disparos le pasaron rozando la cabeza mientras los guardias, claramente asumiendo que su puntería no era su fuerte, comenzaban a disparar indiscriminadamente en su dirección.
Las TARDISes rotas formaban un laberinto aleatorio lleno de esquinas aserradas y geometría desorientadora.
Mientras los guardias se dividían en formación de pinza para alcanzarlos, el Doctor y Cinder seguían el pitido del destornillador sónico, correteando caóticamente de un lado a otro, tropezando con callejones sin salida, retrocediendo y dando la vuelta.
Un disparo certero de uno de los guardias dio contra la pálida piel de una de las TARDIS de Batalla que yacía justo a su lado, y ella se echó para cubrirse, agachándose tras lo que parecía un invernadero. La mitad de sus paneles estaban rotos e infestados de trampas trepadoras.
El Doctor estaba en cabeza.

--Vamos, ¡ya estamos cerca!.

Intentó alcanzarlo, pero tropezó sobre el polvoriento suelo. Estuvo a punto de caerse, pero hizo equilibrio con los brazos, antes de divisar el uniforme rojo y blanco del guardia que los perseguía. Se estaba acercando.
Justo delante, el Doctor viró a la derecha, y ella esprintó hacia él, agarrando lo que parecía ser una hélice de torpedo y usándola para propulsarse y dar la vuelta a la esquina.

--¡Allí está! --exclamó el Doctor con júbilo.

Cinder divisó la ahora conocida cabina azul, alrededor de una pila de fragmentos y paredes interiores rotas con los mismos extraños redondeles del interior de la sala de control del Doctor. Cuando llegaron, la puerta se abrió sola, como en señal de bienvenida. Entraron a toda pastilla, y el Doctor se apresuró a golpear su puño contra un enorme botón rojo de la consola. La puerta se cerró al instante.

--Ya está --dijo, sin aliento--. ¡Ja!. Ahora ya no entrarán --carcajeó como un chaval--. Gracias, chica --extendió los brazos hacia los lados y alzó la vista como si le hablara a la nave--. Tú nunca me fallas.

Miró a Cinder.

--Esta no es la primera vez que he tenido que robarla para huir de Gallifrey --dijo--. A veces me da la sensación de que ella quería irse de aventuras tanto como yo.
En la pantalla del monitor, Cinder vio dos guardias justo delante de la puerta de la TARDIS. Le dieron patadas, intentando abrirla a golpes, y como no podían, retrocedieron, apuntaron, y dispararon simultáneamente a la cerradura.
El sonido dentro de la TARDIS fue tremendo, pero los rayos de energía rebotaron hacia la oscuridad del cementerio mientras las puertas siguieron intactas. Vio cómo uno de los hombres, un tipo moreno con una espesa barba negra, hablaba por el comunicador de su muñeca.  Se imaginó alarmas saltando en complejo.
El Doctor estaba ajustando los controles de vuelo.

--Algo me dice que ya no somos bienvenidos aquí --dijo sarcásticamente. Pulsó una secuencia de botones, giró un dial y accionó una palanca.

La columna de cristal en el corazón de la consola empezó a moverse, emitiendo una luz brillante y etérea. Dentro, el montón de tubos de cristal comenzó a elevarse y a descender lentamente, acompañado del gemido de los motores de la nave. Cinder se agarró a la barandilla, soltando al fin un suspiro de alivio. La nave se estremeció de repente, los motores tosieron como si se ahogaran.

--Oh, no, no, no --dijo el Doctor mientras sus dedos bailaban sobre los controles.

Volvió a ejecutar la misma secuencia, tirando de la palanca. La columna central comenzó a elevarse una vez más, pero luego se detuvo en seco. Los motores hicieron un sonido de asfixia, como si lucharan por arrancar. El Doctor retrocedió, mirando a Cinder.

--Han cambiado los protocolos de seguridad --dijo--. No van a dejar que nos desmaterializemos.

Una voz salió del enlace de comunicación, sobresaltando a Cinder.

--Se le acabó la fiesta, Doctor, venga de inmediato --era el Gobernador, hablando desde algún lugar de la Ciudadela, asumió Cinder--. Su TARDIS ya no tiene acceso a los protocolos correctos para salir de Gallifrey. Ríndanse ante los guardias y volverán a su celda. No habrá más sermones. Esto es todo lo que puedo hacer por ustedes.

--¡No! --exclamó el Doctor, lleno de rabia--. Usted va a dejarnos ir.

El Gobernador se echó a reír.

--Me temo que eso está más allá de mis capacidades --respondió.

--Puede que sí --dijo el Doctor, con un tono serio--, pero por primera vez en su vida va hacer lo correcto. Porque usted sabe que lo que están haciendo está mal, ¿no, Gobernador?. Todavía no está seguro de cómo va a poder vivir con miles de millones de vidas en su conciencia, una vez esto acabe --el Doctor había empezado a pasearse por la consola, perdido en su discurso--. En lo más profundo de sus corazones, sabe que tengo que evitar que detonen la Lágrima, y por eso va a transmitir los protocolos correctos a la consola de mi TARDIS.

--Pero... --el Gobernador titubeó, y Cinder pudo distinguir un tono en su voz de que no podía protestar a lo que el Doctor le había dicho--. ¿Y Karlax?.

--Esto es mayor que usted y que Karlax --dijo el Doctor--. Mayor que yo, mayor que cualquiera de nosotros, Rassilon incluido. Usted oyó lo que dije en la Cámara del Consejo. Si permite que esto pase, no será mejor que un Dalek.

Hubo un chillido proveniente de la consola, y un botón comenzó a parpadear.

--Gracias --dijo el Doctor--. Comprendo lo que ha sacrificado.

--No deje que caiga en vano --dijo el Gobernador, antes de cortar la comunicación.

Una vez más, el Doctor ajustó los controles y la TARDIS dejó de existir con un gemido, dejando a los dos guardias de pie en el sótano, mirándose confusos el uno al otro.



