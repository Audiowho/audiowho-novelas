\chapter*{Capítulo Dieciséis}
\addcontentsline{toc}{chapter}{Capítulo Dieciséis}

Karlax siempre había detestado viajar por un dispositivo transmat, la forma que hacía que la punta de sus dedos hormiguearan durante una hora tras ello, el modo en que sus corazones latían cuando las partículas de su cuerpo comenzaban a desenredarse, agitándolo a nivel molecular. Incluso el concepto de ello le preocupaba, y recordó estar acostado despierto toda la noche como consecuencia de su primer viaje, muchas vidas atrás, preguntándose si aún era la misma persona, o si el proceso le había cambiado irrevocablemente de alguna manera.
Por supuesto, todo esto fue mucho antes de su primera regeneración, cuando los conceptos de cambio e identidad se habían convertido de manera eficaz en irrelevantes para él. Ahora se daba cuenta de que el cambio es inevitable, solo otro tejido en el largo tapiz de la vida. La clave estaba en asegurarse de que dirigías el cambio para conseguir lo que querías, incluso estimularlo en una dirección en particular. Supuso que alguno podían llamar a eso manipulación. Karlax simplemente lo veía pragmático. Hasta ahora, era una estrategia que le había dado grandes dividendos, y no veía razón para parar.
Así que por ello había venido aquí, a la tumba del Lord Presidente, a hablar con él. Estaba ansioso de ser el primero en darle la noticia, de forma que podría influenciar en cualquier decisión que viniera después. Karlax sabía exactamente lo que quería, y haría todo lo posible para conseguirlo. Había llegado el momento de vengarse. El Doctor conseguiría con creces lo que se merecía.
Karlax había visitado la tumba una vez anteriormente, y sabía que Rassilon estaba allí, todos lo sabían. Todos los miembros del Alto Consejo habían autorizado la creación del motor de posibilidad. El trabajo se había llevado a cabo en el Capitolio, a puerta cerrada, y al Consejo se le había mantenido al tanto del progreso. Sin embargo, cuando hubo llegado el momento de revelar el notable nuevo dispositivo ninguno de ellos había sido capaz de contemplarlo. Se alejaron, llamándolo una abominación, y Rassilon fue forzado a moverlo a su vieja tumba para mantenerlo fuera de la circulación.
Ahora, apenas era conocido. El Alto Consejo sabía que Rassilon habló con una autoridad otorgada por la máquina, pero permaneció no expresada, y el nombre de Borusa nunca fue pronunciado.
Esa era una cosa sobre la que el Doctor estaba absolutamente en lo cierto, admitió Karlax de mala gana, todos eran hipócritas, preparados para tomar las decisiones necesarias y obtener las recompensas, siempre y cuando nunca tuvieran que afrontar las consecuencias.
Se había implicado tanto con Rassilon que ese día iba a susurrar en su oído que el Gobernador estaba empezando a desarrollar una conciencia, que no tenía el desapego necesario para ser capaz de cumplir correctamente con sus funciones. El incidente con la chica humana lo había evidenciado, y ahora Karlax sospechaba que había tenido algo que ver con la fuga del Doctor. Habría repercusiones.
Había llegado a la entrada de la Torre. Entró, inclinando la cabeza, consciente de mostrar la debida reverencia.
Rassilon estaba a los pies de la tumba, hablando con el motor de posibilidad, exigiendo que Borusa le enseñara un camino a través del fluctuante caos de lineas temporales, que predijera el momento más efectivo para desplegar su arma. 

--Dime, Borusa, de la Lágrima de Isha. ¿Cuál es el verdadero camino a la victoria?. ¿Cómo puedo asegurarme de su despliegue golpee en el mismo centro de la causa Dalek?.

Karlax se encogió al ver el motor de posibilidad, acodado sobre su cama de metal para que Rassilon pudiera mirarlo. Para Karlax era como un cadáver pálido, revitalizado con vida, esquelético, pero aun así lleno de una vitalidad que no era la suya propia. El Vortice Temporal corría por su cabeza, llenándolo con maravillas y terrores más allá de lo imaginable. En la tenue luz de la tumba, resplandecía con el aura de energía regenerativa oscilante.

--Las lineas temporales ya no me son claras --murmuró Borusa--. No hay un camino verdadero. Cada resultado es un flujo, y florecen posibilidades. Un factor aleatorio ha sido introducido lo cual... desestabiliza las cosas.
--¿Un factor aleatorio? --repitió Rassilon--. ¿Qué es?. ¿Un arma Dalek?.
--No, mi Señor. Algo más. Un nudo de potencial, moviéndose sin control a través de las líneas temporales, tejiendo patrones en el futuro y el pasado.
--Hablas con acertijos --maldijo Rassilon--. Adivinanzas y mensajes en clave. ¡No puedo entenderte!.

Karlax merodeaba junto a la entrada, inseguro de como proceder. No deseaba provocar la ira de su maestro, pero a pesar de todo había noticias que necesitaba contar. 

--¿Lord Presidente? --dijo vacilante.

Rassilon se giró. Parecía molesto al ver que era Karlax quien había venido a molestarle allí, a su refugio.

--¿Qué pasa, Karlax? --espetó--. ¿No puedes ver que estoy ocupado?.
--Mis disculpas, Lord Presidente --dijo Karlax--, pero traigo serias noticias del Capitolio. Nuestros planes podrían estar en peligro. Pensé en informarle de inmediato.

Rassilon le hizo un gesto para que se acercara. Karlax cruzó el mausoleo, vacilante al estar tan cerca de la criatura que una vez había sido Borusa. A pesar de su desdén por la forma en que los miembros del Alto Consejo habían insistido en el cese de Borusa del Capitolio, Karlax tenía serias dudas acerca de estar tan cerca de la cosa. Le miró con sus raros, titilantes ojos, y sonrió complaciente. Karlax se estremeció. Se preguntó qué estaba pensando, que podía estar viendo cuando le miraba.

--¿Y bien? --dijo Rassilon--. Adelante.
--Es el Doctor --dijo Karlax--. Ha escapado.

La mandíbula de Rassilon operó conforme le rechinaban los dientes, evidentemente tratando de preservar su postura. 

--¿Escapado? --había un duro filo en su voz. Karlax tuvo que elegir sus siguientes palabras con mucho cuidado.
--Claramente, fue ayudado de alguna forma --dijo--. Tenemos un traidor entre nosotros. Debería de ser una cuestion sencilla rastrear el registro de datos y ver quien concedió a la TARDIS del Doctor el acceso a los protocolos de seguridad.
--Hazlo --dijo Rassilon--. Tendré la cabeza de alguien por esto.

Karlax tuvo la sensación de que el Alto Consejo estaría buscando una nuevo Gobernador en los próximos días. Reprimió una sonrisa. 

--¿Qué pasa con el Doctor? -- se atrevió--. Aún podría tratar de prevenir que el Comandante Partheus desplegue la Lágrima.
--Sí --dijo Rassilon--. El factor aleatorio --miró a Borusa, momentáneamente perdido en sus pensamientos. Entonces, tras un momento, pareció tomar una decisión--. Lo veo claramente ahora, Karlax. El Doctor es un comodín, un renegado, y ha dejado su posición muy clara. Tiene la intención de actuar contra nosotros. Se le debe impedir influenciar en el resultado de los sucesos. Sólo hay un recurso.
--¿Milord? --dijo Karlax.
--El Doctor debe morir, Karlax --dijo Rassilon--. Solo entonces podremos estar seguros.

Karlax no pudo evitar que su sonrisa se extendiera a los labios.

--Tú, Karlax, eres el único en quien puedo confiar tal importante tarea --continuó Rassilon.
--¿Yo? --dijo Karlax, súbitamente aterrorizado. Esto no era para nada lo que había esperado. Él no era un hombre de acción. Había construido su carrera mediante la manipulación de otros. Apenas se había aventurado fuera en una TARDIS desde sus días en la Academia--. ¿Está seguro de que estoy adecuadamente equipado para tal crucial papel, Milord? --dijo Karlax.
--Exclusivamente --dijo Rassilon--. Tú, más que ninguno de nosotros le quieres muerto.

Así que Rassilon era más perceptivo de lo que parecía. Karlax podía ver que no había manera de hacerle cambiar de opinión. Había caminado justo hacia él, y ahora tendría que llevarlo a cabo. Aún así, reflexionó, al menos de esta forma podría ver la mirada en el rostro del Doctor mientras moría. Eso sería en cierto aspecto un consuelo.

--Muy bien, Milord --dijo.
--Excelente. Cada segundo cuenta. Emplea la ayuda de la Agencia de Intervención Celestial. Si tu has de ser la bala, ellos serán el arma.
--Inmediatamente --dijo Karlax. Se dio la vuelta y caminó hacia la puerta, con la cabeza dándole vueltas.
--¿Karlax? --llamó Rassilon, justo cuando estaba apunto de pasar por la puerta.

Karlax se detuvo y miró hacia atrás. 

--¿Si, Milord?.
--Si fallas, no te molestes en volver.

Karlax tragó saliva. No había palabras para describir lo aterrorizado que estaba en ese momento. 

--Mi vida o la del Doctor --dijo, asintiendo con la cabeza--. Lo entiendo.

Por lo tanto, había llegado a esto. Karlax estaba seguro de una cosa: pasara lo que pasara iba a hacer sufrir al Doctor.

