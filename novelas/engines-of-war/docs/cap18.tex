\chapter*{Capítulo Dieciocho}
\addcontentsline{toc}{chapter}{Capítulo Dieciocho}

Cinder silbó mientras se paraba al lado del Doctor, mirando a la pantalla del monitor. 

--Eso son un montón de TARDIS --dijo.

Habían surgido del Vortex sobre los límites exteriores de la Espiral de Tantalus, y la imagen en la pantalla se agrandó para proporcionarles una vista de la enorme flotilla de los Señores del Tiempo que se arrastraba de manera constante hacia el Ojo.
La escala de eso era simplemente demasiado para que Cinder lo comprendiera. ¿Cuántas naves había allí?. Quinientas, mil, era imposible de decir, pero llenaban el espacio en el monitor como una bandada de gaviotas, siguiendo con decisión a su líder.
En los bordes de la gran formación revoloteaban platillos Dalek en escuadrones de cinco o diez, entrando y saliendo, destruyendo alguna TARDIS ocasional pero sin hacer ninguna mella significativa en la armada. Observó cómo un puñado de TARDIS se liberaba de su formación, moviéndose para atacar a los buques enemigos.

--Parece que han decidido que la fuerza bruta es la respuesta --dijo--. Sólo van a ir ahí dentro, ¿no es así?. Directos hacia el Ojo, con un total desprecio por cuantos de ellos no vuelvan.
--Son soldados --dijo el Doctor, como si eso en sí mismo fuera suficiente explicación.
--No me atrevo a preguntar esto pero, ¿cómo demonios vamos a detenerlos?. Quiero decir, ni siquiera tenemos armas, excepto un viejo neutralizador Dalek y un solo cañón temporal.

El Doctor se inclinó hacia delante, mirando de cerca el monitor, con su nariz casi tocando la pantalla. 

--Esa de allí --dijo él. Tocó la pantalla con la uña, y al hacerlo, tapó completamente el objeto de la vista--. Esa es la TARDIS de Partheus. Apostaría a que es donde vamos a encontrar la Lágrima. Él no confiaría en nadie más.

Se inclinó hacia atrás. La TARDIS que había estado apuntando fue rodeada por un grupo de al menos otras veinte TARDIS de batalla, cada una de ellas fuertemente armada. 

--¡No vamos a conseguir acercarnos! --dijo Cinder.

El Doctor dio un golpecito en los controles y la imagen en la pantalla cambió a una serie de iconos de desplazamiento. Los estudió con atención por un momento. 

--No vamos a acercarnos --dijo el Doctor--. Vamos a entrar en ella.

El comandante Partheus se paró en el puente de la TARDIS, examinando su ruta hacia el Ojo. Había vuelto transparente las paredes y el techo, por lo que tenía la impresión de que estaba de pie en una gran plataforma gris, a la deriva en el vacío.
A su alrededor las demás TARDIS acudieron, manteniendo su formación de batalla, mientras que lejos el halo de un furioso tiroteo mostraba donde los buques de su flanco izquierdo mantenían a raya al enemigo. Más adelante, el propio Ojo de Tantalus parecía mirarlo con enojo, advirtiéndole de que no siguiera.
A esa distancia de la Espiral, la radiación del Ojo estaba afectando a los sistemas de vuelo de sus TARDIS, lo que significaba que no podían simplemente entrar y salir del Vórtice del Tiempo, así que tenían que hacer su aproximación final en el espacio real. Esto hizo que Partheus se sintiera expuesto e inquieto, y vulnerable a los ataques.
Echó un vistazo a cada uno de los tres hombres apostados en las consolas. 

--¿Cual es la situación? --dijo.
--Tenemos el camino despejado, Comandante --respondió uno de los hombres, su lugarteniente--. Unos pocos años luz más y estaremos dentro del rango para desplegar la Lágrima.
--Excelente --dijo Partheus--. Mantened la línea.

Más y más Daleks salían del Vórtice, tanto sigilosas naves como platillos, y alrededor de ellos, la batalla continuaba.
La TARDIS de Partheus, sin embargo, enclavada en el corazón de una formación defensiva, permanecía sin ser molestada. Se acarició la barba.
Estaba a punto de hacer una llamada para recibir un informe de las otras naves, cuando el sonido de un claxon estridente ahogó sus pensamientos. 

--¿Qué diablos? --rugió--. ¡Informe, ahora!

Su teniente se volvió, con expresión de pánico. 

--No sé, Comandante. Parece que tenemos un entrante.
--¿Un entrante qué? --retumbó Partheus--. ¿Un misil?.
--No --respondió el teniente--. Una máquina del tiempo.

A la derecha de Partheus el aire pareció brillar, mientras el intruso intentaba materializarse acompañado por un profundo y silbante gemido. Partheus buscó su pistola.

--Es una TARDIS, Comandante --dijo uno de los otros hombres. Partheus no podía recordar su nombre. Tenía problemas para recordar alguno de sus nombres. Nunca sobrevivían el tiempo suficiente para que valiera la pena.
--¿Una TARDIS?. ¡Pero eso es una locura!. Nos va a desgarrar. Nos va a aniquilar --la otra nave estaba claramente luchando para conseguir anclarse, temblando mientras trataba de salir del Vórtice del Tiempo--. ¿No puedes detenerlo? --ladró.
--Lo estoy intentando, Comandante --dijo el teniente--. Los escudos están ajustados a la máxima potencia.

Hubo un repentino y estruendoso tañido, y una alta cabina azul con la marca "CABINA DE POLICIA" estaba de pie en la sala de la consola de Partheus.

--Demasiado tarde --dijo, con una voz llena de ira--. Ya están aquí.
--Esta es, entonces --dijo Cinder.
--Esta es --respondió el Doctor, dirigiéndose a la puerta. Se detuvo y miró hacia atrás. Su expresión era severa--. Coge tu arma, pero bajo ninguna circunstancia puedes utilizarla.

Cinder la recogió del suelo mientras trotaba tras él. Cuando ella salió de la TARDIS, su primer pensamiento fue que algo había salido terrible e inexplicablemente mal, y que en lugar de aterrizar dentro de la otra TARDIS como el Doctor tenía intención, estaban a la deriva en el vacío del espacio abierto.
Presa del pánico, miró de izquierda a derecha, en busca de una manera de ponerse a cubierto, pero lo único que podía ver era el espacio abierto y el infierno rugiente de la batalla entre los Señores del Tiempo y los Dalek.
Abrió la boca, y luego se dio cuenta que aún respiraba, y que algún tipo de gravedad todavía la sostenía al piso. Estaban dentro de una TARDIS.
Después del shock inicial, su mente comenzó a procesar el resto de lo que estaba pasando. Estaban de pie en la sala de la consola de la otra nave, pero era tan fundamentalmente diferente a la del Doctor que en un principio no se había dado cuenta. Había tres consolas hexagonales bajas, de color gris oscuro y aparentemente idénticas en apariencia. A diferencia de las del Doctor no estaban cubiertas con todo tipo de palancas improvisadas y artilugios, y en vez de orgánicas parecían suaves y prefabricadas. Aburridas, era la palabra para ellas, pensó Cinder.
La sala en sí era cavernosa, tres o cuatro veces más grande que la del Doctor. Las paredes y el techo se habían vuelto transparentes para que el Comandante Partheus pudiera ver mejor cómo su flota estaba yendo en contra de los Dalek.
Había tres Señores del Tiempo, cada uno de ellos atendiendo a cada una de las consolas. Iban vestidos con el mismo uniforme rojo y blanco que los guardias del Gobernador del Capitolio.
El propio Comandante Partheus, subido en una plataforma elevada, la miró ceñudo. Estaba agarrando una pistola en su mano izquierda. Era alto, corpulento y con una barba negra y espesa. Llevaba una túnica negra con una capucha roja. 

--¡Tú! --dijo. Su voz era atronadora--. ¿Qué te crees que estás haciendo?. Ese pequeño truco tuyo podría haber dado lugar a una Embestida de Tiempo, aniquilándonos a ambos.

El Doctor se encogió del hombros. 

--Vamos, Partheus. ¿De verdad crees que yo sería tan imprudente? --sonrió.

Partheus ofreció al Doctor una mirada de pura incredulidad. 

--Creo que eres el único que puede --dijo--. ¿Pensé que estabas en una celda en Gallifrey?.
--Así lo hizo Rassilon --dijo el Doctor--, pero ni siquiera el Señor Presidente puede salirse con la suya todo el tiempo.

Partheus levantó la pistola. 

--Lo siento,Doctor. Yo no quiero hacer esto, pero me estás poniendo en una posición horrible.
--No es una posición tan terrible como la de aquellos que estás a punto de asesinar --replicó el Doctor.
--Si intentas algo ... --continuó Partheus.

Cinder tosió. 

--No lo creo --dijo ella, agitando el arma de su cadera.

Los otros tres hombres parecían claramente incómodos, pero permanecieron donde estaban, de pie junto a sus consolas. El Doctor se acercó a la primera de ellas, echó un vistazo a los controles, y luego negó con la cabeza. Caminó alrededor del hombre hacia la siguiente consola, repitiendo el procedimiento. Evidentemente él vio lo que quería allí,se acercó y empezó a pulsar botones.
El otro hombre, Cinder supuso que era un piloto, lo miró indignado. 

--¿Qué estás haciendo?.
--Estableciendo un nuevo rumbo --dijo el Doctor--. Esto es, después de todo, un secuestro.
--¡Pero no puedes! --el hombre se volvió, tratando de interceptar al Doctor.
--Aléjate --dijo el Doctor--. Esto no es asunto tuyo.

El hombre siguió tratando de empujar el Doctor.

--Créanme cuando digo que estoy verdaderamente arrepentido de esto --dijo el Doctor con resignación.
--¿Qu...

Se volvió y dió un rápido gancho de derecha directo a la mandíbula del hombre haciéndolo caer al suelo inconsciente. El Doctor dobló la mano y movió los dedos con una expresión de dolor. 

--Oh, realmente no me gusta esta parte --dijo--. Me gustaría que la gente simplemente escuchara.

Se agachó de repente, esquivando a la derecha, mientras un rayo de energía pasaba silbando sobre su mano izquierda, provocando una oscura y humeante quemadura en la consola. Cinder se volvió para ver a Partheus sosteniendo su arma. 

--Un disparo de advertencia, Doctor. Para hacerle saber que voy en serio.

El Doctor se acercó a Partheus y le arrebató la pistola de su mano, tirándola para que cayera al suelo cerca de los pies de su propia TARDIS.

--Estoy tratando de ayudarte, Partheus. Créeme, esta es una carga con la que no puedes vivir.
--A la esquina --dijo Cinder a los otros dos, agitando su pistola, pero manteniendo a Partheus cubierto.
--No te preocupes --dijo el Doctor--.Esto no va a durar mucho. Sólo un pequeño vistazo al futuro --sus dedos bailaban sobre las teclas mientras comenzaba a girar diales y a manipular los controles. Cinder escuchó el lejano gemido de los motores que respondían a sus controles. Era un sonido completamente diferente al de la TARDIS del Doctor, más como el de un sutil zumbido que como el de un elefante.
--Bueno, no puedo decir que estime su TARDIS, Partheus --dijo el Doctor--. Le falta... carácter.
--Vas a pagar por esto con tu vida, Doctor --dijo Partheus--. Lo sabes, ¿verdad?.

El Doctor se encogió del hombros. Ni siquiera se volvió para mirar al hombre. 

--Mi vida a cambio de miles de millones de otras. No es un mal intercambio, supongo. He vivido el tiempo suficiente --miró por encima del hombro al Comandante--. ¿Has considerado siquiera qué es lo que vas a hacer?.
--No seas condescendiente conmigo --espetó Partheus--. Por supuesto que soy consciente de la gravedad de la situación. Como yo lo veo, sin embargo, tenemos pocas opciones. Los Dalek deben ser detenidos.
--Hay más de una manera de pelar un gato --dijo el Doctor. Hizo una mueca, mirando a Cinder--. Nunca me ha gustado esa expresión. ¿Por qué alguien querría hacer eso?.

Cinder notó que estaba de nuevo en su elemento. Podía verlo en el brillo de sus ojos. Él se estaba divirtiendo. Se preguntó si sería así todo el tiempo, si no fuera por el peso de la guerra que llevaba sobre él. Estando cerca de él cuando era así...no pudo evitar sonreír.
La vista a su alrededor había cambiado. Ya no eran ellos en cualquier lugar en las proximidades del Ojo de Tantalus, o la flota de TARDIS de batalla en combate contra los platillos enemigos.
Ahora, estaban colgando solos en el vacío. A su alrededor la oscuridad lo era todo, sólo un puñado de estrellas titilaban en la noche del espacio. Todo a excepción del enorme e hinchado cuerpo de una gigante roja, un sol moribundo, que llenaba la pantalla delante de su vista. El núcleo de la estrella se quemaba con suavidad, una llama moribunda en los últimos rescoldos de un fuego que había ardido durante milenios. La envolvente exterior era pálida y delgada, casi transparente.

--¿Dónde nos has traído? --preguntó Cinder.
--Al fin del universo --dijo Partheus--. A los últimos y persistentes momentos antes de que las últimas estrellas se apaguen y el universo se contraiga.
--Veo que al menos tienes algo de poesía en tu alma --dijo el Doctor--. Tal vez no seas irremediable, después de todo.
--No te puedo dejar hacer esto, Doctor --dijo Partheus--. La Lágrima es nuestra última esperanza.
--Encontraremos una manera --dijo el Doctor--. Esto no ha terminado. No puede llegar a esto --se trasladó a otra de las consolas y pulsó una secuencia.--. ¡Ahí!. La Lágrima está preparada.
--¿Vas a disparar al corazón de esa estrella? --dijo Cinder.
El Doctor asintió.
Con un rugido repentino, Partheus se lanzó desde la plataforma hacia el Doctor. Cinder luchó contra el impulso de apretar el gatillo de su arma. El Doctor, después de todo, le había dicho que no disparara bajo ninguna circunstancia, y si lo hacía, corría el riesgo de alcanzar al Doctor. De todos modos, la mantuvo apuntando a Partheus mientras chocaba con el Doctor en la consola.
Ambos hombres cayeron sobre los controles y la TARDIS se inclinó a la derecha en respuesta.
--¡Aléjate de los controles! --bramó Partheus. Agarró al Doctor por la cintura y lo alejó de la consola. Con una inmensa fuerza, tiró al Doctor al suelo, donde aterrizó sobre su trasero, con una expresión indignada.
Se puso de pie y sin perder el ritmo, cargó contra Partheus tomando al otro Señor del Tiempo por el pecho con el hombro y enviando a ambos en la otra dirección. Partheus golpeó en la espalda del Doctor con los puños, y el Doctor se deshizo de él.
Cinder miró a los otros dos Señores del Tiempo, que todavía estaban acobardados en la esquina. Ella les mostró su arma, sólo para recordarles que debían mantenerse alejados de ella.
El Doctor saltó mientras Partheus todavía estaba luchando por mover su considerable volumen del suelo.

--Mira, Partheus. Todo esto es terriblemente inapropiado. ¿Por qué nosotros no sól... --el Doctor se detuvo en seco cuando Partheus le dió una patada en los tobillos, barriendo sus pies. Cayó de nuevo, evitando apenas romperse la cabeza con el borde de la consola. Gimió mientras rodaba para sentarse.

Cinder había tenido suficiente. Gritó con fuerza. 

--¿Qué botón es?.
--El rojo --dijo casi sin aliento el Doctor.

Cinder se encogió de hombros. Supuso que debería haber sido obvio, la verdad. Pulsó con un puñetazo el botón.

--¡Chica estúpida! --rugió Partheus. Ambos se estaban poniendo de pie y quitándose el polvo.

Hubo un ruido mecánico debajo de sus pies, seguido por una serie de sonidos como de abrazaderas al ser liberadas. Partheus sacudió la consola, golpeando los botones. Era demasiado tarde. La secuencia se había iniciado. La Lágrima está siendo desplegada.
Hubo un rumor de encendido y entonces, mientras miraban, un cohete brilló en silencio, aparentemente por debajo de sus pies, hacia el corazón de la gigante roja.
El cohete era pequeño y Cinder no podía dejar de pensar que lo que le habían descrito como un arma devastadora, era, de hecho, un poco decepcionante. Tal vez, como una TARDIS, el arma de los Señores del Tiempo era más grande por dentro.
Todos se quedaron en una aturdido silencio, esperando a ver qué pasaría cuando la Lágrima cayera en la estrella y comenzara a desplegarse.
--Sugeriría que hagamos una retirada estratégica --dijo el Doctor.

Partheus, todavía de pie en los controles, tecleó una serie de comandos y la TARDIS comenzó a alejarse lentamente de la estrella.
El cohete era ahora una pequeña mota, apenas visible frente a la aurora del gigante celeste.

---No está sucediendo nada --dijo Partheus.
--Sigue mirando --dijo el Doctor, y un segundo después Cinder notó el ligero toque de una sombra en el centro de la estrella. Mientras miraba, comenzó a crecer constantemente de tamaño, expandiéndose como un derrame de petróleo. La mancha negra siguió extendiéndose, ganando impulso, tomando y consumiendo la tenue luz roja desde el borde exterior de la estrella.
--La estrella está colapsando --dijo el Doctor--, arrastrando todo hacia una singularidad gravitacional.
--Eres un necio, Doctor --dijo Partheus--. Has tirado la mejor oportunidad que teníamos de derrotar la amenaza Dalek --se había detenido y estaba ayudando a su teniente a levantarse del suelo cuando el hombre había empezado a volver en sí--. Te harán pagar por esto.
--No lo dudo --dijo el Doctor desapasionadamente--. Pero eso no significa que lo que hice estuviera mal.

El último haz de la luz roja había cambiado a negro, y la colapsada estrella estaba empezando a arrastrar hacia sí a la materia circundante. El suelo de la TARDIS comenzó a vibrar mientras los motores luchaban contra el tirón.

--Es hora de irse --dijo el Doctor.

Partheus había recuperado su pistola y estaba apuntando al Doctor.

--Debería matarte ahora --dijo--, por todas las vidas que acabas de condenar.

El Doctor lo miró con ojos desafiantes. 

--Bueno, entonces hazlo si lo vas a hacer. Acabo de impedir que te convirtieras en un genocida, pero si sigues con la idea de que...

Partheus pareció vacilar. El extremo del arma se inclinó hacia abajo. 

---Vete --dijo--. ¡Fuera de mi nave!.

Sin decir palabra, el Doctor le dio la espalda y caminó hacia su propia TARDIS. Cinder lo siguió, manteniendo su arma apuntando a Partheus, aunque podría asegurar que no había peligro.
El Doctor abrió la puerta de la TARDIS y ambos cruzaron el umbral.
Cinder dio un suspiro de alivio, dejando caer su arma, mientras la puerta se cerraba tras ellos. 

--Lo hicimos --dijo--. Realmente lo hicimos.

El Doctor sonrió. 

--Sí, supongo que lo hicimos. Pero me temo que nuestros problemas no han terminado aún. Todavía hay Daleksa los que hacer frente, por no hablar de un montón de furiosos Señores del Tiempo aullando por mi sangre.
--¿Qué vas a hacer al respecto? --dijo Cinder.
--Sólo hay una cosa que pueda hacer --dijo el Doctor--. Vamos a volver a Gallifrey.
--¡Qué! --dijo Cinder--. Estás realmente loco.

El Doctor se echó a reír. 

--Me gusta pensar que sí.

