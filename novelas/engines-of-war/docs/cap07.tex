\chapter*{Capítulo Siete}
\addcontentsline{toc}{chapter}{Capítulo Siete}

--¿Cómo vamos a entrar? --dijo Cinder. 

Habían dejado la casa, emergiendo hacía la todavía vacía calle. Las bóvedas Dalek se cernían amenazadoramente y se presagiaban en el siguiente cruce. Cinder estaba tratando de calcular el mejor plan para entrar.

--Siempre encuentro en momentos como estos --dijo el Doctor--, que el mejor recurso es utilizar la puerta principal.

--¿La puerta principal?. ¿No puedes estar diciendo en serie que simplemente vas a caminar hacia allá e intentar usar el pomo? --dijo Cinder. No podía decir si era ingenuo, confiado, o simplemente peligrosamente temerario. Tampoco sabía si las puertas de las naves de los Dalek tenían pomos.

--Precisamente --respondió el Doctor--. Por lo general el truco funciona --echó a andar en dirección a la bóveda.

Exasperada, Cinder corrió tras él. 

--Te encuentras en este tipo de situaciones a menudo, ¿no? --preguntó. 

--Más de lo que quieres saber --dijo el Doctor con un profundo suspiro. Sus ojos parecían acuosos y cansados.

Se preguntó cuántos años tendría. Parecía viejo, pero no tenía ni idea de cuánto tiempo podría vivir un Señor del Tiempo. Había oído decir que eran inmortales. que no podían morir, pero también que podían cambiar sus caras a voluntad, convertirse en alguien diferente y nuevo. No sabía si algo de ello era cierto. Por lo que sabía, el Doctor era tan mortal como ella, e igual de susceptible al disparo de un arma de energía Dalek.

--¿Que pasa con los Dalek? --dijo--. Has visto lo que pueden hacer. Esa nueva arma, la pistola de desmaterialización, ¿y si aparecen con una de esas?. 

--Los Dalek son tan arrogantes como los Señores del Tiempo --dijo el Doctor--. Quizás incluso más. Esa es la belleza de un plan como este. No esperarán que nadie simplemente llegue y se invite a entrar.

--Difícilmente llamaría a eso un plan --murmuró Cinder. Agarró su arma un poco más fuerte. Cuando dijo que se apuntaba a esta aventura, había esperado que él tuviera un ligera idea de cómo ban a hacerlo.

Al final de la calle miró a la izquierda, preparada para correr, pero el Dalek que habían visto antes se había movido. Comprobó la otra dirección, mirando a lo largo de la calle.
La ciudad estaba organizada en un patrón de rejilla básico, diseñada de acuerdo a un plan que los colonos habían traído con ellos desde la Tierra. Habían llegado con una cierta cantidad de materiales prefabricados, y estos habían formado la base de los primeros edificios, estos y el revestimiento de la nave que los había llevado allí. A medida que la colonia se había desarrollado, habían aprendido a manufacturar, a cosechar la madera local y a extraer minerales y metales. Los edificios se hicieron más sofisticados, pero aún siguieron el plano de la Tierra. Mes tras mes, año tras año, la colonia había crecido, olvidando del todo que era una colonia y convirtiéndose en un hogar.
La gente había florecido aquí, y con el tiempo se habían extendido a lo largo de los planetas de la Espiral. Sin embargo, Moldox había sido el primero, el origen de la vida humana en este sector. Ahora, miles de millones de esas personas habían muerto, posiblemente borradas por completo de la historia, mientras miles de millones era esclavizadas por los Dalek.
El Doctor tenía razón. Evitarían que esto le pasara a nadie más. Tenían que hacerlo. Era el momento de dejar de dudar de él. Si caminar desvergonzadamente hacia al platillo y pasear hacia el punto más cercano iba a ser la mejor forma de entrar en la base Dalek, entonces debería de seguirlo. Había algo en el Doctor, algo que le inspiraba a confiar en él.
Cruzaron la intersección y continuaron por la sucia calle, hasta que estuvieron en la sombra del platillo más cercano. Era inmenso, elevándose sobre ella, y podía ver, desde el nivel del suelo, que estaba asentado sobre bóvedas que brotaban de su base. Debajo estaban los escombros de uno de los antiguos edificios de la escuela. La armadura ablativa que formaba el revestimiento exterior estaba llena de agujeros y cubierta de verdín. Ninguna de las luces parecía ser funcional. Las enredaderas habían comenzado a hacer incursiones, enrollándose de arriba a abajo como esbeltas manos de jardinero, aferrándose a los intrusos alienígenas. Parecía tan abandonada como los edificios humanos que lo rodeaban.
Se inclinaron hacia delante, mirando de un lado a otro. En lo alto, en uno de los pórticos, un Dalek y dos Degradaciones, de la variedad en forma de huevo con piernas de araña, estaban cruzando de un platillo al otro. El Doctor parecía no haberlos visto. Cinder lo cogió del brazo y lo arrastró a las sombras bajo el vientre de la nave. Sacó su arma de forma silenciosa en dirección a los Dalek y él asintió. Esperaron un momento hasta que los Dalek hubieron pasado.

--Debería de haber una rampa en este lado, si no me equivoco --dijo el Doctor jugueteando con el nudo de su bufanda. Siguió adelante, siguiendo el borde del platillo hasta que estuvieron cerca del borde del patio central, pero todavía ocultos en gran parte por las sombras.

Los Dalek parecían haber terminado sus pruebas de armas, y los humanos restantes, contó seis, estaban siendo llevados de vuelta al platillo en el otro lado.
A Cinder le parecía incomprensible que este sitio, un viejo patio de recreos, se pudiera haber convertido en tal sitio de muerte. Las descoloridas marcas de rayuela y los círculos pintados en el suelo parecían incongruentes, erróneos. Estaba llena de un fuerte sentimiento de desasosiego. Era casi como si los Dalek hubieran elegido esta localización para burlarse de sus humanos cautivos, para recordarles tiempos más felices, ahora perdidos para siempre.

--Muévete o serás ex-ter-mi-na-do --dijo uno de los Dalek, empujando a un prisionero en la espalda con su brazo manipulador. El hombre se tambaleó hacia delante, pero no respondió al Dalek, ni siquiera gritó. El pelear se había ido claramente de su cabeza, y arrastró los pies por la rampa, con la cabeza gacha.
Cinder se dio cuenta de que ese era un hombre que esperaba morir. Todos ellos lo esperaban. Cada uno de esos prisioneros, hombres y mujeres, sabían que solo era cuestión de tiempo, y de alguna manera, probablemente habían empezado a anhelarlo. Incluso a ansiarlo. Al menos la muerte sería una liberación del tormento infligido por sus captores. Todo lo demás era una extensión de su agonía.
Observó a los rezagados finales de la pequeña fiesta montar en la rampa y desaparecer en la otra nave.

--Bien --susurró el Doctor, tocando la parte superior de su brazo para captar su atención--. Esta es nuestra oportunidad. Hay una rampa justo a la vuelta --indicó agitando su pulgar--. Despacio y en silencio, y quédate a mi lado.

Cautelosamente, cruzaron el patio y ascendieron la rampa. Cinder mantuvo su arma colgada en la cadera, con su dedo cerca del gatillo. Apenas podía creer lo que estaba haciendo. Si Coyne pudiera verla ahora...
Hombro con hombro, los dos se adentraron en la ancha fauces de la nave Dalek.
En el interior, las paredes estaban comprimidas en una serie de arcadas cristalinas estampadas con pequeños círculos, y a través de los cuales colores chillones, amarillos, verdes, ocres y púrpuras, latían como sangre palpitando por una red de arterias y venas.
Un amplio pasillo apareció para rodear la circunferencia de la nave, ofreciéndoles la opción de ir a la izquierda o a la derecha. El corazón de Cinder martilleaba en su pecho, esperando que un Dalek doblara una de las curvas en cualquier momento. Por ahora, sin embargo, parecían estar solos.

--Bueno, esto ha sido más fácil de lo que pensaba --susurró.

--Entrar es la parte fácil --respondió el Doctor--. Es salir lo que generalmente suele ser el problema.

--Oh, gracias por eso --murmuró. Se dio cuenta de que sus manos estaban temblando mientras trataba de mantener su pistola nivelada--. ¿Y ahora qué?.

El Doctor se encogió de hombros. 

--Echemos un vistazo. Cada una de estas cúpulas será para un propósito específico. Averigüemos en cual de ellos estamos.

Manteniéndose cerca de la pared, siguieron el pasillo conforme serpenteaba hacia la izquierda, mirando hacia adelante ante cualquier signo de Dalek. Cinder estaba segura de que sólo la mera suerte les había permitido llegar tan lejos, y estaba convencida de que se encontrarían rodeados en cualquier momento. Los Dalek debian de tener sistemas de vigilancia en sus naves, ¿verdad?.
Después de un tiempo el pasillo se bifurcó, dividiéndose en un número de angostos túneles que parecían conducir hacia el interior de la nave. El Doctor, quien parecía estar decidiendo arbitrariamente el camino a tomar, la guió por una de esos pequeños pasillos afluente agitando su mano.
Aquí, había una fila de paneles que parecían puertas, grandes láminas de metal incrustadas en arcadas. No parecían tener ninguna clase de controles. O, consideró Cinder, cualquier picaporte. Bueno, al menos eso respondía a esa pregunta.

--¿Son celdas? --preguntó Cinder--. ¿Podría haber prisioneros dentro?. 

--Posiblemente --dijo el Doctor--. Es difícil decirlo desde aquí fuera, aunque imagino que los mantienen a todos junto en el otro platillo, o en alguno de los edificios cercanos.

--Deberíamos de comprobarlo --dijo--. ¿Cómo abro la puerta?.

--Camina hacia ella. Se activan con el movimiento --respondió.

Cinder se movió lentamente hacia la puerta, pero nada sucedió.

--No, así no --dijo el Doctor--. Camina con un objetivo, como un Dalek --avanzó confiado, hinchando su pecho. Hubo un click y un zumbido mecánico, y un segundo más tarde la puerta se abrió, deslizándose hacia el techo.

La sala revelada más allá era un cuarto relativamente grande, lleno de todo tipo de equipamiento extraño y recuerdos tecnológicos. El hedor que emanaba, sin embargo, era casi suficiente como  para que se desplomara y hacerla vomitar. Inmediatamente, deseó haber seguido caminando.
El Doctor entró, y ella le siguió, arrugando su nariz debido al olor. Era nauseabundo, como carne rancia y podrida. Algo en esa habitación estaba muy mal.
Cinco estructuras de vidrio se mantenían contra la pared trasera. Eran transparentes, pero moldeados en la arquetípica forma de un Dalek, completado con un brazo manipulador de cristal y un arma.
Cinder alzó su arma, esperando a que entraran en acción en cualquier momento. Retrocedió, mirando de un lado a otro.
El Doctor agarró su mano, tranquilizándola. 

--No están vivos --dijo--. Al menos aún no. Míralos de nuevo.
 
Todavía un poco insegura, se acercó más. A través de las paredes de cristal del recipiente podía ver la materia orgánica del interior, una pesada, pegajosa masa de carne y tubos, continuamente inflándose y desinflándose como un pulmón enfermo y pegajoso. La habitación era una especie de cámara de incubación.
Esto en sí mismo era suficiente para causar otra arcada involuntaria, pero fue cuando miró a la segunda de las cámaras de incubación cuando se dio cuenta de la verdadera magnitud del horror. En ésta, el componente orgánico aún tenía un rostro humano.
En otros tiempos había sido una mujer, pero ahora, si quedaba algo en los ojos amarillos era solo locura. La cabeza había mutado, convirtiéndose en una sin pelo, deforme. La carne tenía ampollas y burbujeaba, estaba cubierta de tumores torcidos. Las extremidades de la mujer habían sido retiradas, y cables salían de su pecho, cableándola con la carcasa de incubación.
Cinder se tambaleó hacia atrás, mirando a otro lado, incapaz de procesar lo que estaba viendo. Era, al mismo tiempo, la cosa más asquerosa y despreciable que nunca había visto.

--¿Es esto lo que están haciendo aquí? --dijo--. ¿Experimentando con los prisioneros?. 

--Convirtiéndolos en Dalek --dijo el Doctor, con voz triste.

--¿Convirtiendo humanos en Dalek? --repitió Cinder, incapaz de exponer adecuadamente su disgusto. 

--Sí, parece que no están tan preocupados con la pureza racial como lo solían estar --dijo el Doctor--. Es curioso cómo los ideales se van por la ventana cuando tienes la espalda contra la pared.

--¿Pero por qué?. ¿Qué podrían ganar?.

--Están haciendo soldados --dijo--. Carne de cañón. Están empapando a gente con radiación para que sus células muten en formas parecidas a los mutantes Kaleds. Una vez que las han alterado fisiologicamente, extraen toda emoción, lobotomizándolos y los insertan en revestimientos normales de Dalek. Aceptarán ordenes tan bien como cualquier otro Dalek, y si son destruidos, bueno, al menos no eran auténticos Dalek.

--Es obsceno --dijo Cinder.

El Doctor asintió. 

--Es sólo la punta del iceberg --dijo.

Cinder miró alrededor de la habitación. Además de las cinco incubadoras había algo pequeño digno de ser notado: una cuba burbujeante conteniendo algo que parecía carne derretida, y una telaraña de cables conectados a las incubadoras, que desaparecían en el techo y en las paredes. Claramente, eran estos los que llevaban la energía y los nutrientes necesarios para mantener a los mutantes humanos vivos durante su transición.
Echó un vistazo al Doctor, había una pregunta en sus ojos. Él asintió con la cabeza, y se acercó a la puerta para vigilar.
Cinder dejó caer su arma, permitiendo que se balanceara en su correa del hombro, y cogió un manojo de cables con ambas manos. Tiró de ellos con fuerza, utilizando todo su peso para tratar de arrancarlos de sus soportes en el techo. En el tercer intento, al menos la mitad de ellos se desprendieron, rompiéndose, un horrible fluido negro rociando con gotas desde los deshilachados extremos como sangre brotando de una herida fresca. Cinder dejó caer los extremos deshilachados al suelo. Continuó así durante unos momentos, sacando a todos los cables de sus tomas de corriente, permitiendo a su ira arder brillante y violentamente en su pecho.
Cuando terminó, se agachó ante la incubadora de la que una vez fue una mujer, y miró a los ojos de la cosa. Sus pupilas se fijaron en ella, pero la mirada era vacía, inquietante. 

--Encuentra la paz --dijo Cinder. Se levantó y caminó hacia el doctor. Él aún estaba esperando en la puerta, vigilando.

--Veamos que más tienen aquí --dijo. Salió de la habitación e inmediatamente saltó hacia atrás, atrapando a Cinder en el pecho con su brazo casi tirándola al suelo--. Dalek --susurró.

Retrocedieron, uno a cada lado de la puerta abierta, presionando contra la pared mientras los tres Dalek pasaron deslizándose. Se movían casi en silencio, con sus apéndices oculares rotando, sus brazos manipuladores retorciéndose como si sintieran la perturbación del aire.
Cinder contuvo la respiración, esperando a que pasaran, asumiendo que en cualquier minuto uno de ellos notaría que algo estaba mal en la cámara de incubación y volvería para investigar.
Afortunadamente, no parecían estar prestando atención, y siguieron rodando, adentrándose en la nave.
Esperó a que el Doctor le indicara que todo estaba despejado antes de permitirse exhalar. 

--¿Viven a bordo de estas cosas? -- dijo cuando estuvo segura de que no serían escuchados--. Los platillos, quiero decir. ¿Es esta su casa?.

--En tanto un Dalek vive --dijo el Doctor--. No duermen, comen o beben. No tienen un concepto de amistad o compañerismo. Son decididos, implacables en la búsqueda de su meta final: erradicar toda vida en el cosmos excepto la suya.

--Sí, más o menos entendí eso --dijo Cinder con una sonrisa torcida.

Prosiguieron, continuando su vuelta a la nave.

--La cubierta de vuelo está en el corazón de la nave --dijo el Doctor--. Ahí es donde estarán la mayoría de los Dalek. Estoy mucho más interesado en ver qué está sucediendo en otra parte.

Caminó hacia otra puerta, que se abrió en cuento se acercó. Los contenidos de esta habitación eran igual de perturbadores, y Cinder notó un olor igual de nauseabundo. Aquí, los experimentos estaban claramente siendo llevados en las Degradaciones.
La carcasa de un planeador yacía en el suelo en pedazos, mientras que el torso había sido removido del interior de la cámara de cristal y estaba extendido en una losa de metal. Parecía como si los Dalek hubieran estado llevando a cabo una autopsia, y simplemente la hubieran abandonado a la mitad. Órganos extirpados en cuencos de metal, pudriéndose lentamente, y la carcasa no cubierta se estaba secando y estaba empezando a pudrirse.
Componentes de otros Dalek de aspecto inusual estaban esparcidos por la habitación: un alargado apéndice ocular, la mitad inferior de una unidad de desplazamiento en la que todos los globos sensoriales eran transparentes y titilaban con una exótica luz azul, una cabeza dorada con cuatro válvulas de radiación. 

--¿Es cierto? --dijo Cinder cubriendo su boca y negándose a mirar al cadáver--, ¿que estos son los resultados de experimentos de los Señores del Tiempo, intentos de rediseñar la historia y evolución de los Dalek?.

El Doctor se encogió de hombros. 

--Hay algo de verdad en eso --dijo--, por supuesto que la hay, pero solo en cuanto que le dio a los Dalek la idea, los medios de experimentar y de adaptarse a si mismos. Lo han llevado a extremos más lejanos de lo que lo hicieron  los Señores del Tiempo.

--¿Quieres decir que se están haciendo esto a ellos mismos? --dijo Cinder. La misma idea la paralizó. 

El Doctor asintió. 

--Un programa de eugenesia Dalek --dijo--. Zambulléndose en tantas realidades alternativas como puedan encontrar y corrompiendo su ADN, intentando criar la perfecta máquina de matar para utilizarla contra los Señores del Tiempo.

--Debéis de ser un enemigo bastante aterrador--dijo Cinder-- para inspirar eso.

El Doctor miró hacia otro lado, incapaz de mirarla a los ojos. 

--No vamos a encontrar lo que estamos buscando aquí --dijo--. Creo que es hora de que comprobemos uno de los otros platillos. Estos solo son laboratorios experimentales.

Cinder quería preguntarle qué es exactamente lo que estaba buscando, pero antes de que tuviera la ocasión había partido, desapareciendo en el pasillo. Sus preguntas tendrían que esperar, al entablar una conversación mientras reptaban por la nave solo corría el riesgo de atraer a los Dalek. Se apresuró para alcalzarle.
Continuaron con su reconocimiento en torno a los pasillos exteriores de la nave, hasta que se toparon con una rampa de acceso que conducía a un nivel superior. Se apresuraron, estimulados por el sonido amortiguado de Dalek, haciéndose eco a través de una puerta tras ellos.
El nivel superior de la nave era muy parecido al inferior, aunque se encontraron frente a una gran escotilla abierta conforme emergían de lo alto de la rampa. Aquí, la pasarela de metal sobresalía de un agujero, abarcando abiertamente el espacio abierto hacia el platillo opuesto.
Era provisional, parecía como si hubiera sido improvisado al igual que los edificios temporales que le habían servido de hogar durante tanto tiempo. No parecía particularmente seguro, no había pasamanos o barandillas, solo una lisa banda de metal, de unos cuatro metros de largo, estirada entre las dos naves para que los Dalek se desplazaran. A diferencia de un Dalek, sin embargo, si Cinder cayera, no sería capaz de encender sus propulsores para estabilizarse o para volar.
Se movió con lentitud hacia la abertura y miró hacia abajo, mientras que el Doctor comprobaba que no había Dalek viniendo de otras direcciones. Era una buena caída. Bajo ellos, el patio ahora parecía estar vacío, los prisioneros Dalek habían vuelto a uno de los otros platillos.

--No parece particularmente seguro --siseó Cinder, mientras el Doctor se unía a ella a los pies del puente de metal.

--Estarás bien --dijo el Doctor, en lo que parecía ser un intento de calmarla. No lo hizo, sin embargo, sonaba particularmente seguro de si mismo. Golpeó el puente de metal con el borde de la bota, y luego dio tumbos, probándolo con su peso--. Lo ves, bien --dijo--. Echó a andar en dirección a la otra nave.

Sintiéndose demasiado expuesta, y decididamente insegura, Cinder le siguió por el puente, tratando de no mirar hacia abajo. Si se centraba en la espalda del Doctor, y se daba prisa, entonces no era tan malo...
Demasiado tarde, se dieron cuenta de que en la fantasmal calma de la base sus pisadas sonaban como disparos, resonando contra el metal a cada paso. A mitad de camino, el Doctor paró durante un momento, mirando hacia atrás. 

--Mejor nos apresuramos --susurro--. En cualquier momento, uno de ellos estará obligado a venir a investigar. No estamos pasando terriblemente desapercibidos.

Cinder le echo su mejor mirada de 'no me digas' y continuó, simultáneamente intentando caminar más rápido y asegurándose de permanecer de pie en la pulida superficie de metal. Ella casi se cayó de espaldas cuando alcanzó el punto donde el puente se inclinaba, conduciendo a la otra nave Dalek, y giró sus brazos frenéticamente, intentando mantener el equilibrio. Su arma se aflojó en la correa de su hombro y se enredó en el recodo de su codo, amenazando con liberarse y caer por el borde.

--¡Aguanta! --dijo el Doctor. Alargó la mano hacia ella, titubeando por un momento, y entonces finalmente se las arregló para sujetarla por el brazo. Esperó un momento, aún enredado en ella, mientras se recobraba y se remangó su funda del arma de nuevo en su hombro, y entonces la guió hacia lo seguro, atrayéndola hacia la trampilla. Ella se alegró de tener sus pies de nuevo en tierra firme, aunque fuera a bordo de una nave Dalek.

--¿Dónde están todos? --dijo un momento después, una vez que recuperó el aliento. 

--Habrán menos de los que piensas --dijo el Doctor--.Tienen otras bases en otros lugares del planeta, similares a esta, y patrullas en las ruinas como las que encontraste antes. La mayoría de ellas habrán proseguido, sin embargo, desplegándose a otros frentes, o uniendo sus flotas de ataque aqui, cerca del Ojo.

--Asumí que habría un ejército entero de ellos --dijo Cinder--. Miles y miles de ellos. Quiero decir, mira lo que le han hecho a este lugar. Mira la devastación que han provocado. Y sin embargo, aquí estamos, escabulléndonos por su base, y apenas hemos visto rastro de ellos.
--Esto es lo que hacen --dijo el doctor--. Aparecen, destruyen, desaparecen. No tienen ningún interés en el planeta. Creo que estoy empezando a darme cuenta que hay más que hacer con la proximidad de Moldox al Ojo. Creo que eso es por lo qué estoy aqui.

--¿Qué quieres decir? --dijo Cinder. 

--¿Esa radiación temporal de la que hablé? --dijo el Doctor--, ¿la cosa que causa la aurora en el cielo?.

Cinder asintió.

--Se filtra del Ojo, en una descarga constante. El Ojo es una anomalía, una estructura que no debería existir. Una arruga en el espacio-tiempo: un agujero, si lo quieres, entre universos. Esas armas temporales, las pistolas de desmaterialización, creo que están alimentadas por él. Los Dalek han encontrado una manera de emplear la radiación para usarla a su voluntad.
Cinder miró hacia arriba involuntariamente, como si estuviera buscando el Ojo. Sin embargo, todo lo que vio fue el interior del platillo Dalek. 

--¿Que hay de los prisioneros? -- dijo--. ¿Por qué los mantienen aquí?.

El Doctor se encogió de hombros. 

--Materia biológica que pueden utilizar para construir nuevos Dalek mutantes, o material para probar sus armas experimentales. Eso es todo, me temo.

La simple crueldad de ello estaba dejando atónita a Cinder. Tener tal descarada desconsideración por la vida, le parecía repugnante. 

--¿Y esto es lo que harán en todos esos otros planetas en la Espiral?.

--Probablemente --dijo el Doctor--, o habrían creado minas y esclavizado gente para cavar, recogiendo minerales y metales preciosos para su esfuerzo bélico. No tienen muchísima imaginación, a menos que sea para matar gente --se alisó la parte delantera del chaleco--. Vamos, habrá tiempo de sobra para hablar después.

El interior del segundo platillo se parecía mucho al del primero, casi idéntico, excepto por el hecho de que había una serie de puertas frente a ellos casi inmediatamente cuando entraron.

--Quiero encontrar una de sus terminales --dijo el Doctor--. Probemos con... --meneó su dedo de atrás a delante como su estuviera contando-- pito, pito, colorito..., aquí --se dirigió deliberadamente hacia una de las puertas y, al igual que antes, se abrió para dejarle pasar.

Ésta se abría en una gran cámara de la que una serie de puertas derivaban en habitaciones contiguas. Parecía ser un laboratorio, con un banco de nueve monitores en la pared más lejana, todos ellos estaban mostrando complejas secuencias de números, animados para formar dobles hélices sobre un fondo verde borroso.
Había dos mesas desplegadas con un surtido de equipo y herramientas quirúrgicas, aunque afortunadamente, pensó Cinder, esta vez ninguno de ellos llevaba los restos de un experimento Dalek.

--Vigila la puerta --dijo el Doctor. Caminó hacia el banco de pantallas y empezó a tocar en el cristal, almacenando signos y arrastrándolos para crear patrones inusuales. A Cinder le parecía un galimatías, pero supuso que debía de significar algo para le Doctor.

--¿Puedes leerlo? --dijo. 

--Un poco --respondió el Doctor, pero estaba distraído, prestando atención a la información que aparecía ante sus ojos. Ahora los diagramas estaban parpadeando en las pantallas, alambres que parecían describir un edificio u otra construcción masiva. Se preguntó si eran mapas del platillo o de la base.

Cinder esperó en el umbral de la puerta, sosteniendo el arma sobre su pecho de forma que era capaz de cubrir el pasillo, así como la entrada a la nave. Su corazón aún estaba estremecido, y sus palmas estaban resbaladizas por el sudor. A pesar de la fanfarronería, de alguna forma se sentía aterrorizada, y la inicial explosión de adrenalina estaba empezando a desaparecer.

--Es peor de lo que creía --dijo el Doctor. Ella miró por encima del hombro para ver lo que estaba mal, pero él aún estaba de espaldas, leyendo las pantallas--. Están clonando Dalek mutantes --miró sobre su hombro para ver si estaba prestando atención, señalando a una de las puertas--. Eso es una incubadora. Están engendrando Dalek para poner su nuevo paradigma a plena producción.

--¿Los de los cañones? --preguntó. 

--Sí --confirmó el Doctor--.Pero la cosa empeora. También están construyendo algo más.

--¿Qué? --siseó Cinder. 

--Un destructor de planetas --dijo el Doctor--. Una super arma. Están planeando convertir el propio Ojo de Tantalus en un cañón de energía enorme, y dispararlo contra Gallifrey.

--¿Qué significa eso? --dijo Cinder. 

--El fin de todo --gruñó el Doctor--. Borrarán por completo a Gallifrey, erradicarán su existencia, re-escribirán la historia como si los Señores del Tiempo nunca hubieran existido. Condenarán al Universo, infestarán todo --por primera vez desde que llegaron a la base, el Doctor realmente parecía preocupado. 

--¿Qué podemos hacer?.

--Podemos empezar por darles algo más de lo que preocuparse --dijo el Doctor, Volvió a jugar con los iconos en las pantallas de los monitores, y uno de las puertas de su lado se abrió para revelar una cámara más pequeña.

Se dio la vuelta, medio esperando ver un Dalek emerger por la puerta, pero no había nada allí. Retrocedió, manteniendo su arma apuntando a la puerta principal, hasta que pudo echar un vistazo a la recién revelada habitación.
Tanques claros llenos de un fluido azul pálido estaban alineados a ambos lados de la sala. En el interior, verdes y feas criaturas del tamaño de una cabeza humana, con un único y tenue ojo, tentáculos gordos como de gusano y curvadas garras afiladas, estaban suspendidas en el burbujeante fluido. Había cientos de ellos. Tan solo la visión de ello le provocó arcadas a Cinder, dejando de lado el tufo ácido que asaltaba sus fosas nasales cada vez que respiraba.

--¿Qué son? --dijo. La aversión en su voz era evidente. 

--Mutantes Kaleds --respondió el Doctor, yendo a ponerse tras ella--. Clones, en una fase bastante tardía de desarrollo. Estos estarán pronto listos para ser colocados en carcasas Dalek.

--Eso es lo que hay dentro de un Dalek --dijo Cinder, mirando un poco más cerca--. No es de extrañar que tengan problemas de imagen.

El Doctor entró en la habitación y empezó a trastear con la rejilla de metal en el suelo.

--¿Qué estás haciendo? --dijo. 

--Algo que debería de haber hecho hace mucho tiempo --respondió. Pulsó una palanca y quitó un pequeño panel de acceso a sus pies, revelando un nido de coloridos cables. Agarró un puñado y los arrancó, ordenándolos hasta que encontró el que buscaba.

--Ah, este es --dijo.

--¿Qué es?.

--Tubo de refrigeración --respondió, estirando violentamente de él hasta que se rompió, cortándolo en sus manos. Un vapor gris pálido empezó a salir de sus extremos, condensándose en el cálido aire y salpicando el suelo. Lanzó la arruinada tubería. En algún lugar del platillo una alarma empezó a sonar con estridencia, un insistente trinar, haciéndose eco a través de los pasillos vacíos.

--Los estás matando --dijo. No había ninguna acusación en su voz. Fue una declaración de los hechos. Ya podía ver a los mutantes empezar a retorcerse en sus tanques, cada vez menos cómodos conforme la temperatura del agua empezaba a incrementarse--. Van a recalentarse, hervir vivos en esos tanques.

El Doctor la miró con una dura mirada. 

--He estado aquí antes --dijo. Por primera vez, su voz sonaba vieja, agotada con el peso de los siglos--. Me he enfrentado a esto en el pasado, y no actúe a tiempo. Si tan solo hubiera tenido el valor de hacer lo que era necesario entonces, las cosas podrían haber sido muy diferentes ahora. Pero ahora soy un hombre diferente. No me rijo por los mismos ideales. Tengo un trabajo que hacer, y esta vez, no tengo tales recelos.

A pesar de la frialdad de sus palabras, Cinder podría decir que realmente él no creía esto. Estaba intentando convencerse así mismo tanto como convencerla a ella.

--¿Estás seguro? --dijo. 
Asintió con la cabeza. 

--Déjalos. Dejó la habitación, yendo derecho hacia una de las otras puertas. Se abrió, revelando otra pequeña antecámara. Ésta parecía ser una despensa, guardando una variedad de componentes Dalek: brazos manipuladores, globos sensoriales, armas de energía. El Doctor se acercó a un estante exhibiendo una extensa fila de cañones negros, al igual que los utilizados en los nuevos Dalek.
Cogió uno, alzándolo, probando su peso. Le dio la vuelto en sus manos, comprobando el cargador, y luego asintió a Cinder, claramente satisfecho. 

--Es hora de irse --dijo.

La alarma seguía sonando cuando entraron en el pasillo, Cinder se aterrorizó ante la visión de un Dalek a no más de unos metros de distancia, dirigiéndose directamente hacia ellos.

--¡Alto!. ¡Intruso!. ¡Serás exterminado!.

--No si puedo evitarlo --dijo el Doctor tras ella. Se echó contra la pared de la nave cuando el Doctor apretó el gatillo del cañón, volando al Dalek con una dosis de radiación temporal. El Dalek chilló furioso, retrocediendo, pero la chispeante luz rosa del arma pareció formar un capullo a su alrededor, deformándose y entrelazándose conforme intentaba encontrar una forma de entrar.

--¡Eliminar!. ¡Eliminar! --gritó el Dalek. 

Algo iba mal. El arma no se estaba comportando de la forma que él esperaba, habiendo observado como era usada contra los prisioneros humanos. En lugar de filtrarse en el interior del Dalek, comiéndolo desde el interior, la luz empezó a dispersarse en brillantes volutas de humo, disolviéndose en el aire, hasta un momento más tarde, que se había desvanecido completamente. Y el Dalek permaneció ante ellos. El Doctor bajó el cañón. 

--Se han hecho inmunes a ella --dijo.

El Dalek disparó su arma de energía y el Doctor se tiró al suelo, golpeándose el hombro con la pared y rebotando, aterrizando con sus rodillas. El haz de energía falló por poco, quemando una larga y negra línea en la pared.

--No son inmunes a esto --dijo Cinder alzando su arma y apretando el gatillo. Un rayo de energía blanca atravesó la red de sensores del Dalek, agrietando el enchapado de su armadura y estallando por la parte trasera de su cabeza, empapando el pasillo de fragmentos de Dalekenium y materia biológica. Su apéndice ocular se atenuó y se quedó quieto.

--Ahora hemos acabado --dijo el Doctor. 

Cinder podía escuchar canto de voces Dalek viniendo del interior de la nave. Estaban mezcladas, reunidos por la alarma y las explosiones de las armas de energía. 

--Un simple 'gracias' hubiera sido suficiente --dijo ella ayudándolo a levantarse. 

El Doctor sonrió. El sonido de las voces de Dalek era cada vez más cercano. 

--Vamos --dijo-- ¡Ahora! .

Salieron de la nave por el mismo pórtico por el que habían subido a bordo. Estaban a diez metros de altura, al menos, y bajo ellos otros Dalek estaban saliendo en manada de los otros platillos. No había forma de poder saltar sin hacerse alguna herida, y si lo hicieran, nunca serían capaces de escapar a tiempo.

--De vuelta a la otra nave --gritó el Doctor, cogiéndola por el brazo.

--¡Exterminar! --Un rayo de energía les pasó volando, lo suficientemente cerca como para chamuscar la parte trasera de la chaqueta del Doctor. Cargaron por la pasarela hacia la otra nave.
Delante de ellos un Dalek emergió de la escotilla abierta, pero Cinder no dudó. Apretó de nuevo el gatillo y vio al Dalek explotar, el impulso llevó su carcasa de vuelta a la boca de la nave.
Los Dalek alzaron el vuelo, chillando un coro de amenazas mientras desencadenaban disparo tras disparo, pero el Doctor y Cinder corrieron, deslizándose en el otro platillo. De alguna manera, se las habían apañado para cruzar el puente sin ser disparados. No parecía un gran consuelo.
Con un empujón, el Doctor envió los restos del Dalek muerto a rodar en el camino de otros dos que venían por el pasillo hacia ellos, y condujo la carga en la dirección contraria, rodeando la nave de vuelta por donde habían venido, bajaron al nivel inferior y pasaron la incubadora humana.
Irrumpieron en el patio para ver al menos a diez Dalek dirigirse directamente hacia ellos. Cinder sabía que no podía acabar con todos antes de que la derribaran, y el cañón aún agarrado por el Doctor había probado ser completamente inefectivo.
Todos estaban allí, todas las variedades de Dalek y Degradación que había visto antes: los normales de bronce y oro, Planeadores, Arañas, Armas temporales. Había otras nuevas, también, versiones que nunca se había encontrado en las patrullas: negros, plateados con cúpulas azules, otro de un blanco impoluto, como un fantasma pálido, cada uno de ellos igual de mortífero que los otros.
Niveló su arma hacia la marea que se aproximaba, determinada a llevarse por delante al menos la mitad de ellos.

--¡La pasarela! --vociferó el Doctor. Cinder echó un vistazo. Había tres Dalek saliendo del pórtico. 

Giró el cañón de su arma y disparó tres tiros consecutivos, no a los Dalek, sino al metal del pórtico en el que estaban. El metal se dobló y se desplomó, causando que los Dalek bambolearan incontrolablemente, y un cuarto disparo dividió la pasarela en dos, enviandolos a estrellarse sobre los Dalek que estaban bajo ellos. Al menos cinco de ellos fueron enviados a rodar sobre el patio, con sus armas disparando indiscriminadamente, mientras que otros dos estaban incapacitados, tumbados, con su cabezas cedidas por el impacto superior.
No los detendría durante mucho tiempo, pero era suficiente distracción como para que el Doctor y Cinder salieran de la linea de fuego.

--Dirígete a la muralla --dijo el Doctor--. Los distraeré -- empezó a dirigirse en la dirección opuesta. 

--¿Qué pasa con los prisioneros? --dijo Cinder. ¿Ciertamente no los podían abandonar ahora?.
 
El Doctor vaciló, deteniéndose. Parecía dolorido, como si estuviera tratando de decidir si arriesgarse o no. 

--¡Maldita sea! --ladró-- Frénalos.
 
Se dio la vuelta, cargando en diagonal por el patio hacia la nave, donde antes habían visto a los Dalek pastoreando a los prisioneros humanos.
Cinder corrió tras él, conforme subía la rampa, pero paró pronto, dándose la vuelta para afrontar a tres Dalek restantes que se estaban dando la vuelta. 

--¡Vamos! --chilló--. ¡Venid a cogerme estúpidas latas de metal!.
 
Justo cuando estaba a punto de vaciar los restos de su paquete de energía en los Dalek, hubo una terrible explosión dentro de uno de los platillos. Sintió la vibración como un retumbe bajo sus pies, agitando sus huesos.
Una columna de llamas y un oscuro y aceitoso humo salió de lo alto de la bóveda, y los Dalek rodaron en la dirección opuesta ladrando órdenes. Se dio cuenta de que debía de haber sido la incubadora que el Doctor había amañado para sobrecargarse, finalmente alcanzando la masa crítica.
Aprovechó la oportunidad y sacó su arma, hizo tres disparos, volando las bóvedas del resto de los Dalek mientras intentaban decidir qué hacer.
Sin embargo aún más Dalek estaban llegando a escena y sabía que solo tenía unos pocos segundos antes de que se viera desbordada.

--¡Doctor --gritó, justo cuando un enjambre de gente se precipitaba por la rampa tras ella, derramándose sobre el viejo patio. Había decenas de ellos, y se dio cuenta de que debían de haber estado apiñados abordo de la nave.
Escuchó al Doctor llamándolos desde la parte alta de la rampa. 

--¡Vamos!. ¡Corred por vuestras vidas!. ¡Defendeos!. Ahora es vuestra oportunidad --los prisioneros libres respondieron con ganas, rodeando a los Dalek. Incluso sin armas estaban encontrando formas de desactivarlos, volcándolos por pura superioridad numérica. 
Los Dalek se movían en manada desde arriba, disparando indiscriminadamente a la multitud, pero los prisioneros estaban en plena sublevación y no estaban siendo disuadidos.

--Es bastante impresionante, lo que un puñado de humanos pueden conseguir cuando ponen sus mentes a ello --Cinder se dio la vuelta para encontrarse al Doctor a su lado. En una mano aún sostenía el cañón, mientras que en la otra sostenía su destornillador sónico, que claramente había usado para abrir las cerraduras de las puertas. 

--Ahora, Cinder. Es hora de correr --dijo--. Llegar a las puertas. Encontraremos una forma de pasar.

Entusiasmada por ver a la gente arrollar a los Dalek, Cinder hizo lo que el Doctor dijo y retrocedió. Corrieron, hombro con hombro, zambulléndose en una de las calles y enfrentándose a una carrera hacia las puertas.
Con la insurgencia arrasando en la base, las puertas, claramente como el Doctor había anticipado, estaban desatendidas. Cinder no se molestó en tratar de encontrar una forma de desbloquearlas, y disparó a las viejas puertas de madera con el arma Dalek, perforando un agujero suficientemente grande para pasar.
En cuestión de segundos, el Doctor y Cinder habían desaparecido una vez más en las ruinas. Tras ellos, el resplandor naranja de la base Dalek en llamas iluminaba el cielo de Andor.
No pararon hasta que no alcanzaron la TARDIS. Exhaustos, sin aliento, tropezaron con el borde del cráter causado por el Doctor en el poco ortodoxo aterrizaje. Los restos del Dalek muerto todavía estaban en la carretera, fríos e inertes.
Quizás la TARDIS era la vista más bienvenida que Cinder podía haber imaginado, esa cabina azul, tumbada de lado en el barro. Para ella, representaba la seguridad, la oportunidad de escapar, de dejar la Guerra atrás. Pero ahora también representaba algo más: liberación. El día que el Doctor cayó del cielo y la hizo mirar al mundo de forma diferente, a lo que era posible. Y ahora, aunque sabía que solo era una mota en el ojo de la Guerra, la más pequeña de las victorias, había ayudado a liberar a alguna de su gente.
Agradecidamente, Cinder saltó al interior de la TARDIS, esta vez preparada para el extraño cambio en la alineación entre la dimensión exterior y la interior. Sin embargo, desorientada, todavía se tambaleó a un lado como un borracho, forzada a sostenerse en la barandilla de metal para sostenerse.
El Doctor cerró la puerta tras ellos, y cayó de rodillas, arrojando su arma al suelo y envolviéndose con sus brazos. Sintió lágrimas brotar, lagrimas de alivio, pero luchó contra ellas, sorbiendo y limpiándose los ojos.
Notó que el Doctor aún tenía el cañón Dalek en sus manos. 

--¿Por qué trajiste eso? --dijo--. Sabes que no funcionará contra los Dalek. 

El Doctor miró al arma, y luego la arrojó al suelo, donde retumbó antes de descansar. 

--Necesito llevarla a Gallifrey --dijo--. Necesito enseñar a los Señores del Tiempo a que nos enfrentamos. 

Cinder le miró boquiabierta. 

--Pero teníamos un trato --dijo--. ¿Creía que me ibas a sacar de todo esto, de la Guerra?.
 
El Doctor asintió. 

--Lo haré. Lo prometo. Te llevaré a algún lugar seguro. Pero primero tengo que visitar Gallifrey. Lo que los Dalek están haciendo aquí, podría significar el fin de la Guerra. Peor aún, el fin del Universo. Si son capaces de desplegar esa arma entonces no habrá ningún lugar seguro, en ninguna parte de la realidad.

--Entonces voy contigo --dijo Cinder atrevidamente--. No me vas a dejar aquí. 

El Doctor negó con la cabeza. 

--Viajo solo. No tengo tiempo para niños abandonados. Solo te interpondrías --empezó a darse la vuelta, pero Cinder se puso de pie y lo cogió del brazo. No había más que eso. Podía verlo en sus ojos. Tenía miedo de ella de una forma en la que no había tenido miedo ni siquiera de los Dalek. 

--Oh, no --dijo--. No te librarás tan fácilmente. Dije que estaba dentro, y eso significa que no me dejas atrás.

Sus ojos se encontraron. Se miraron el uno al otro en silencio durante un momento, ninguno de los dos estaba dispuesto a ceder. Finalmente, el Doctor cedió. 

--Está bien --dijo, alzando sus manos en un gesto de resignación--. De acuerdo, puedes venir. Pero esto es un acuerdo temporal. No tengo tiempo para preocuparme por nadie más. 

Cinder sonrió. 

--Creo más bien, Doctor, que yo sí tengo tiempo para preocuparme por ti.

--Te lo he dicho, ya no utilizo ese nombre --dijo con el ceño fruncido.

--Oh, creo que te lo has ganado hoy --dijo Cinder. Caminó para ponerse tras él en la consola, examinando el extraño surtido de palancas, diales y botones parpadeantes--. Veamos --dijo--. ¿Vas a enseñarme cómo funciona esta cosa?.
 
--No tientes a la suerte --dijo el Doctor mientras presionaba el interruptor de desmaterialización.

