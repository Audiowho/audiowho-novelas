\chapter*{Capítulo Diez}
\addcontentsline{toc}{chapter}{Capítulo Diez}

La Sala de Guerra no era en absoluto como Cinder se la había imaginado cuando había escuchado a Karlax decirle al Doctor a donde se dirigían. Si este era el centro neurálgico de todas las operaciones de los Señores del Tiempo contra los Dalek, entonces tal vez estuvieran realmente en más problemas de los que pensaba. Era más bien… bueno, supuso que la palabra adecuada era sobria.

La sala ni siquiera parecía imponente y no estaba particularmente bien equipada, al menos hasta donde pudo ver, la mayor parte de la tecnología Gallifreyan estaba más allá de su comprensión, más parecida a la magia que a algo manejado por una persona.

De todas maneras, la Sala de Guerra venía a ser poco más que una gran cámara ovalada, flanqueada por pilares de piedra derruidos y dominada por una enorme mesa de ébano. La superficie de la mesa brillaba con la brumosa luz azul de pictogramas holográficos y runas. Parecían deslizarse justo por debajo de la laca, como peces en un estanque, floreciendo en nuevas y elaboradas formas cada vez que las tocaban o interactuaban entre sí. Era una danza extraña e hipnótica y no pudo descifrar lo que significaba.

Esto, supuso, era el lenguaje de los Señores del Tiempo. Sin duda parecía complejo, lo suficientemente lógico para ser un lenguaje y lo suficientemente preciso para pertenecer a la única raza en el universo que parecía hacer de la pedantería un arte.

Había poco más digno de mención en la sala, aparte de una serie de pantallas colgadas como marcos en las paredes, difundiendo imágenes retransmitidas desde lo que asumía eran naves o TARDISes de guerra de los Señores del Tiempo. Las silenciosas imágenes pasaban en una confusión de explosiones y luces intermitentes, ventanas hacia encuentros de Señores del Tiempo con los Dalek. Mientras miraba vio estallar en llamas naves de ambas facciones y después extinguirse casi al instante, sus moles muertas iban a la deriva en el vacio frio y sin aire. 

No podía decir si era una retransmisión en directo o si el hombre sentado en la silla estaba revisando imágenes de batallas que ya habían pasado. Supuso que en una guerra del tiempo la cuestión era probablemente discutible.

Sin embargo, lo más extraño de toda la instalación era el hecho de que la Sala de Guerra estaba escondida en una tranquila esquina de la Ciudadela, bien lejos de los salones principales y del Panóptico. A Cinder le daba la impresión que los Señores del Tiempo estaban tratando de esconderla, barrer toda evidencia de la guerra en un polvoriento rincón abandonado del edificio y así simplemente podrían ignorarlo. ¿Pensaban que si decidían no reconocerla, de alguna manera no sería real y la vida podría continuar en el Capitolio como siempre? Tuvo la impresión clara de que para muchos de los Señores del Tiempo la Guerra era un asunto de otros, una perturbación que se resolvería a su debido tiempo. Nada que consiguiese preocuparles.

Se preguntó por la otra gente de Gallifrey, los hombres movilizados para ser soldados y si ellos pensaban lo mismo sobre proteger un modo de vida que probablemente se había vuelto rancio y arcaico antes incluso de que la superficie de Moldox se hubiera enfriado de su ardiente creación.

Por supuesto, Cinder sabía que el Doctor pensaba de forma diferente, a juzgar por su reacción a lo que había visto en Moldox. Esa era la razón de su visita. Había venido a advertirles. Este era su pueblo y planeaba protegerles.

El hombre de la silla no se levantó ni se volvió a mirarlos cuando entraron en la sala. Una muestra de poder tal vez, un recordatorio de quién estaba al mando.

Entonces, ese era el Lord Presidente de Gallifrey, el hombre al que el Doctor había llamado Rassilon. Muy a su pesar, Cinder sintió un nudo en el estomago. Ayer había estado luchando por su vida contra una patrulla Dalek en Moldox. Ahora estaba aquí en presencia de uno de los seres más poderosos del universo.

Solo podía verle de perfil. Era un hombre mayor, delgado y recio. Tenía el pelo rapado y moreno, encanecido. La luz de los monitores proyectaba sus rasgos en un marcado relieve: el ceño afilado y poco profundo, la nariz aguileña, la mandíbula cuadrada y apretada. He aquí un hombre que no veía mucho humor en el universo, que había sido templado por la carga del deber. El peso de esa carga era casi palpable en la sala.

—Ah, Doctor —dijo Rassilon, todavía negándose a apartar los ojos del monitor que tenía encima—. Tengo entendido que su llegada ha causado bastante revuelo. Se aplaude su inventiva —hablaba educadamente, y su voz era suave y profunda. Se echó a reír—. Solo me complace que trabaje para nosotros, en vez de en contra nuestra —se volvió y le dio al Doctor una sonrisa torcida. Sin embargo sus ojos eran fríos y duros—. ¿No está de acuerdo, Karlax?

—Efectivamente, señor —dijo Karlax con un servilismo repugnante.

—Las noticias vuelan —dijo el Doctor, mirando a Karlax—. Apenas acabo de llegar.

Rassilon se echo a reír. 

—Vamos, únase a mí —dijo, haciendo señas al Doctor para que se adelantara. Sus dedos brillaban bajo la luz reflejada, y Cinder se dio cuenta que llevaba un guante de metal en la mano izquierda.

El Doctor hizo lo que le pedía. Cinder se mantuvo justo en la puerta, tratando de permanecer invisible, mientras Karlax tomaba asiento en la mesa, desde donde podía observar el proceso.

—Hijo díscolo de Gallifrey, mire aquí —señaló los monitores con un movimiento de la mano—. Nuestras naves arden, nuestras TARDISes explotan, nuestros niños mueren a manos de los Dalek. Luchamos por nuestra propia existencia —suspiró—. Pero, usted lo sabe, por supuesto. Ha estado allí fuera, en medio de todo. Dígame, Doctor, ¿qué nos trae a casa del frente?

—Traigo una advertencia —dijo el Doctor secamente.

—Una advertencia —dijo Rassilon, evidentemente divertido—. Ciertamente somos unos privilegiados, Doctor —se echó a reír—. Sin embargo, primero me gustaría oír noticias de su búsqueda. ¿Le ha encontrado ya? ¿Ha localizado al Amo?

El Doctor negó con la cabeza. Cinder no tenía ni idea de sobre qué o quién estaban hablando. 

—Le ha abandonado, Rassilon. Nos ha abandonado  a todos. Ha huido y dudo que lo veamos de nuevo. Al menos no hasta que la Guerra haya terminado.

—Miró al ojo de la tormenta, y lo que vio fue demasiado para soportarlo —dijo Rassilon—. Es débil y solo piensa en su propia supervivencia. Aun así no puedo culparlo. Estamos todos de pie en un precipicio, mirando hacia abajo —estudió al Doctor por un instante—. Y ahora usted, Doctor, trayendo más noticias desagradables.

—Me temo que sí —dijo el Doctor—. He venido directamente de la Espiral Tantalus, donde he visto la flota de Preda destruida por una emboscada de naves Dalek invisibles.

Rassilon señaló las pantallas con un gesto expansivo. 

—He visto los últimos momentos de su TARDIS con el corazón afligido —Cinder pensó que ni siquiera parecía preocupado.

El Doctor asintió. 

—Mi TARDIS fue dañada en el ataque. Sobreviví a un aterrizaje forzoso en el planeta Moldox, donde descubrí una instalación de pruebas Dalek. Han desarrollado una nueva arma, alojada en un nuevo paradigma. Emplea la radiación temporal que se escapa del Ojo de Tantalus.

—¿La anomalía? —dijo Rassilon.

—Precisamente —respondió el Doctor—. Han creado un arma demat. Los vi probarlo en sus prisioneros humanos. Esta listo para ser distribuido a sus fuerzas en el frente.

—Esto es… preocupante —dijo Rassilon.

—Hay algo peor —continuó el Doctor—. Me las arreglé para entrar en uno de sus platillos y consultar sus bancos de datos. Están usando la tecnología para construir un asesino de planetas. Tienen la intención de dispararlo contra Gallifrey.

—Para desmaterializar un mundo entero —dijo Rassilon—. Lo admito, Doctor, estoy impresionado por su ingenio.

—Tenemos que actuar —exhortó el Doctor—, y pronto. Ese precipicio que mencionó, acabamos de movernos incómodamente más cerca del borde.

Rassilon parecía divertido. 

—¿Karlax?

—Sí, mi señor.

—Organice una sesión de emergencia del Alto Consejo. Nos reuniremos en una hora. El Doctor y su asistente presentarán sus descubrimientos —Rassilon miró al Doctor y sonrió.

—Muy bien, señor —respondió Karlax.

Cinder no podía evitar tener la sensación de que el Doctor acababa de hacerse él mismo el objeto de una especie diferente de emboscada.

\mbox{}

Karlax les hizo entrar en la sala del consejo. El Doctor entró primero, arrastrando el cañón Dalek que había recuperado de su TARDIS, y cuando Cinder le siguió, se detuvo al notar una mano sobre su hombro, presionando demasiado fuerte como para ser cómodo. Se volvió y vio a Karlax cernirse sobre ella.

—Espera conmigo por aquí —dijo, mientras la empujaba por la fuerza hacia la esquina de la sala, justo a la izquierda junto a la puerta. De mala gana cedió, permitiendo que él la quitara de en medio.

—Oh, ya veo —dijo secamente, cruzando los brazos sobre el pecho y poniendo la mayor distancia posible entre ella y el pequeño y odioso hombre—. Solo a los Señores del Tiempo importantes se les permite un asiento en la mesa.

Karlax la miró frunciendo el ceño, pero por lo demás no respondió. Observó como el proceso comenzaba a desarrollarse, consciente de que Karlax estaba a su lado.

Dado que se trataba de la sala de reuniones del Alto Consejo de Gallifrey, era mucho menos ostentosa de lo que esperaba en su breve estancia en el Capitolio: las paredes eran totalmente blancas, el suelo estaba cubierto con un suave mármol color crema y el mobiliario era escaso.

Una gran mesa ovalada ocupaba la mayor parte del espacio. Era de forma y tamaño similar a la de la Sala de Guerra, pero con una superficie brillante de madera lacada con incrustaciones de finas filigranas de oro. No parecía tener ninguna tecnología incorporada, pero era difícil asegurarlo.

Aparte de esto, Los únicos otros objetos en la habitación eran una gran arpa de oro, una pintura de un Señor del Tiempo de aspecto decrépito tocando el arpa y una plataforma que contenía dos espolones y una interfaz de ordenador.

Alrededor de la mesa, algunos Señores del Tiempo ya habían tomado asiento. Rassilon se sentó en la cabecera, vestido con el traje de gala de su cargo y sosteniendo un bastón de oro en la mano izquierda. Era delgado y de metal, coronado con un remate elaboradamente forjado. Cinder no tenía ni idea de su finalidad, pero supuso que su origen era ceremonial.

Una de las sillas estaba vacía y Cinder notó que el emblema del Señor del Tiempo se había tallado en lo alto del respaldo con detalle excepcional. Se pregunto si esto denotaba el rango de la persona que debería haber estado sentada ahí.

A la izquierda de Rassilon estaba sentada una Señora del Tiempo. Parecía joven, con el pelo moreno cortado al estilo paje, enmarcando un delicado y bonito rostro. También estaba vestida con ropas elaboradas, esta vez en color morado oscuro con un adorno de platino y un ancho collar dorado descansando sobre sus hombros.

Frente a la mujer estaba el Gobernador, el jefe del servicio de seguridad, que Cinder se había encontrado brevemente a su llegada al Panóptico, y otros dos hombres, uno de ellos mayor, de piel oscura y arrugada y el pelo muy corto, el otro más joven pero aun así encaneciendo, con una barba bien recortada y penetrantes ojos azules. Ambos llevaban guantes hasta el codo con los nudillos cubiertos de anillos.

Parecía que todo el que se suponía que tenía que estar presente estaba allí. Karlax cerró la puerta y después volvió a su lugar a su lado. Lo miró por un instante. Sus ojos estaban fijos en el Presidente, atento a cada movimiento del hombre.

Karlax debió de sentir su mirada porque se volvió y le dio una sonrisa burlona. 

—Es una privilegiada —susurró—.  No sé de ningún humano al que le haya sido permitido jamás asistir a una sesión del Alto Consejo.

Cinder se encogió de hombros. 

—Circunstancias desesperadas requieren medidas desesperadas —dijo.

—Totalmente —dijo Karlax amargamente.

Rassilon se levantó de su asiento, golpeando con firmeza el suelo con su bastón.

El metal resonó contra el mármol y todos los ojos se volvieron hacia el Presidente. 

—Esta sesión queda iniciada —dijo—. El Doctor se dirigirá a nosotros ahora.

El Gobernador sonrió y se recostó en su silla. Observo al Doctor con una mirada divertida en los ojos. 

—Tengo entendido que hay un pequeño asunto al que desea atraer nuestra atención, Doctor —dijo. Su tono era condescendiente. Cinder se sintió indignada en nombre del Doctor.

Por supuesto, el Doctor podía cuidar de sí mismo. 

—¿Pequeño, dice? ¿Pequeño? —miro al Gobernador—. La única cosa pequeña en esta sala, Gobernador, es su mente.

El Doctor arrojó el cañón Dalek en la mesa, que resonó con fuerza, haciendo estremecer a los Señores del Tiempo como si el Doctor hubiese sido lo suficientemente grosero como para tirar un animal muerto sobre la mesa de la cena.

El Doctor comenzó a pasearse por el lugar, con las manos cruzadas a la espalda. Cinder podía ver que estaba lleno de rabia a punto de explotar. 

—¡Es hora de despertar! —dijo—. Estamos en guerra y, por lo que dicen, estamos perdiendo en todos los frentes. Nos superan y nos superan en número y estamos enterrando las cabezas en la arena, negándonos a reconocer lo que es evidente para el resto del universo. Mientras nos miramos el ombligo, los Dalek se han establecido en la Espiral Tantalus y están construyendo sus fuerzas allí.

El Cardenal de la barba se encogió de hombros exasperado. 

—Entonces debemos enviar una flotilla para ocuparse de ello.

El Doctor golpeó la mesa con los puños, inclinándose hacia delante para elevarse sobre el hombre. Sacó la barbilla, acercando su cara a la del otro hombre. 

—Ya lo hemos intentado, Grayvas —dijo—. Lo hemos aplazado demasiado tiempo. Son demasiados y ya estamos escasamente desplegados. Vi como toda la flota de Preda ardía, las TARDISes de batalla explotaban bajo el fuego de cientos o más naves Dalek invisibles. Y eso es sólo la punta del iceberg. Se han establecido en una docena de planetas desde hace años, dando forma a sus planes, construyendo sus flotas.

—Hemos visto esto antes —dijo el Gobernador con tono desdeñoso—. Los Dalek están construyendo ejércitos dondequiera que miremos. Es lo mismo. Siembran sus progenies infernales a través de la historia y recolectan materia biológica de la población local para crear nuevos mutantes. No es nada nuevo, Doctor. La Guerra sigue.

—Oh, pero sí lo es, Gobernador —el Doctor usó el titulo como una maldición—. Están minando la radiación temporal que se filtra en el Ojo de Tantalus, usándola para crear armas de desmaterialización como esta —señaló el arma sobre la mesa—. Esto se cogió de uno de sus nuevos paradigmas. He visto lo que hace, cómo ha reescrito el tiempo y erradicado totalmente a cuatro seres humanos de la historia.

Rassilon se inclinó hacia delante, mirando concentrado el arma. El Gobernador extendió su mano enguantada como para tocarla y luego la retiró, cambiando de opinión. Su expresión era de desolación.

—Así es —dijo el Doctor—. ¿Recuerda lo que un arma demat puede hacer a un Señor del Tiempo? No hay posibilidad de regeneración, solo el simple olvido. Encerramos los nuestros, enterrándolos en una cámara de seguridad a causa de los horrores que eran capaces de infringir en otros —se pasó una mano por el pelo—. Ahora los Dalek los tienen y están produciendo sus nuevos paradigmas mientras hablamos.

Grayvas se aclaró la garganta. 

—Este Arma Temporal Dalek, Doctor, ¿lo ha visto más de una vez? —dijo.

El Doctor asintió. 

—Es viable. Destruí uno de sus criaderos en Moldox, pero habrá cientos más, incluso miles. La Espiral Tantalus se ha convertido en un caldo de cultivo. Si no los detenemos pronto, van a empezar la siembra a través del tiempo, a todas las diferentes épocas en las que estamos luchando y en otras en las que no. El genio estará fuera de la botella y nunca se podrá volver a meter.

Rassilon se echó hacia atrás en su asiento, pensativo. Golpeo los dedos enguantados en la mesa, rat-ta-tattat, rat-ta-tat-tat 

—Dígales el resto, Doctor. Dígales la amenaza real.

—Esto es solo el comienzo —dijo el Doctor—. Obtuve acceso a sus sistemas informáticos mientras estaba a bordo de uno de sus platillos. Están construyendo un asesino de planetas. Están usando la misma tecnología para convertir el Ojo de Tantalus en un arma de energía masiva. Un arma temporal —hizo una pausa para tomar aire—. Están planeando borrar Gallifrey de la historia, de todas y cada una de las permutaciones de la realidad. Los Señores del Tiempo dejarán de existir, la historia será reescrita como si nunca hubieran existido y el universo caerá en manos de los Dalek —el Doctor se apartó de la mesa, mirando ceñudo a Rassilon—. Tenemos que actuar ahora.

Rassilon frunció el ceño. 

—Si se equivoca, Doctor, y mostramos nuestras cartas, podríamos quedar totalmente expuestos a los Dalek. Como tan acertadamente ha apuntado, nuestras fuerzas están ya demasiado dispersas tal como están. Si nos comprometemos a una ofensiva en la Espiral Tantalus, corremos el riesgo de dar a los Dalek una oportunidad de establecer un lugar de desembarco en otro sitio. 

—No me equivoco —dijo el Doctor. Su tono era contundente—. Aquí está la evidencia, justo delante de vuestros ojos —Dio un empujón al cañón Dalek, que se deslizó por la mesa hacia Rassilon—. Haga que los técnicos la examinen, si duda de mí.

Rassilon sonrió. 

—Entonces, ¿qué se debe hacer? Díganos, Doctor, ¿qué sugiere?

El Doctor suspiró, dejando caer sus hombros. 

—No lo sé —dijo—. Parecen tener un puesto de mando en el corazón de la Espiral, justo encima del Ojo. Supongo que también sería la ubicación del arma. El problema es llegar a ella. Hay una armada de platillos estacionados allí, sin mencionar un sin número de naves invisibles, escondidas en el vacío. Se requeriría todo lo que tenemos.

—Imposible —dijo la Señora del Tiempo—. Simplemente no tenemos los recursos.

—Hay una manera —dijo el Gobernador—. Hay un arma en el Arsenal Omega.

El otro Cardenal, que hasta ese momento había permanecido en silencio, se volvió hacia el Gobernador. 

—No puede estar refiriéndose en serio al Momento. Sin duda aún no se ha llegado a eso.

—No —dijo el Gobernador—. No es eso. La Lágrima de Isha.

El Doctor frunció el ceño. 

—Pero la Lágrima fue diseñada para colapsar agujeros negros —dijo—. Es una herramienta de ingeniería estelar. ¿Cómo podría…? Oh —se detuvo, su mente alcanzó a su boca—. Sí, ya veo.

—Veo que lo ha entendido, Doctor —dijo el Gobernador—. Si usáramos  la Lágrima en el corazón de la anomalía, podríamos cerrar el Ojo. Se nos permitiría rediseñar el pliegue en el espacio-tiempo y neutralizar la fuente de la energía temporal de los Dalek para siempre.

—No puede hacerlo —dijo el Doctor—. Billones de vidas se perderían. Hay una docena de mundos habitados en la Espiral, colonias que se han establecido allí hace siglos. La Lágrima causaría que el Ojo implosionara y la tormenta subsiguiente asolaría los planetas, envejeciéndolos hasta convertirlos en polvo. No puedo permitirlo.

—¿No puede permitirlo? —dijo Rassilon—. Realmente, Doctor, creo que tiene una idea exagerada de su importancia. ¿Quién es para decir lo que podemos hacer o no?

—Rassilon, estaría justificando el genocidio a una escala masiva —contraatacó el Doctor—. Las apuestas son demasiado altas. Tiene que haber otra manera.

—Entonces díganos, Doctor. Ilumínenos. ¿Qué otra manera ve? —Rassilon se puso en pie, cerrando su enguantado puño—. ¿Viene a nosotros con la palabra fatalidad inminente y todavía espera que nos sentemos y nos neguemos a actuar a causa de su pequeña simpatía por un puñado de seres humanos?

Cinder no pudo aguantar por más tiempo. 

—¿Un puñado? —dijo, meneándose hacia delante. Estaba harta de escuchar toda esa charla informal de genocidio—. Es de mi casa de lo que estáis hablando. Billones de vidas. Hay más gente en esos mundos que la suma total en Gallifrey. No son peones en vuestro juego para ser sacrificados a voluntad.

Rassilon miró al Doctor. 

—Silencie amablemente a su asistente, Doctor. No tiene voz en esta sala.

Cinder sintió el agarre de Karlax en su hombro una vez más y esta vez presionó hasta que fue doloroso. 

El Doctor la miró y ella pudo ver la frustración en el conjunto de su mandíbula. Claramente quería agarrar a Rassilon por los hombros y sacudirlo hasta que escuchase.

—Por favor, Lord Presidente —dijo, con un esfuerzo que debía de haber sido claro para todos en la sala. Estaba conteniendo una diatriba—. Si le da tiempo, si le da la debida consideración, habrá otras maneras. Solo que no podemos verlas aún.

Rassilon agitó su brazo a los Señores del Tiempo reunidos. 

—Iros —dijo—. Todos vosotros. Esta sesión ha terminado. Tomaré mi decisión y seréis informados —miró hacia arriba, fijando en el Doctor una mirada amenazante—. Doctor, puede esperar en la sala de observación. Hablaré con usted en breve.

—Muy bien —dijo el Doctor. Los otros consejeros se levantaron y salieron de la sala, cada uno negándose a encontrarse con la mirada de Cinder. Si esto era por pura arrogancia o por su incapacidad de hacer frente a un ser humano después de su complicidad en lo que equivaldría al genocidio de su pueblo, no lo sabía. Tampoco le importaba particularmente. Quería que sufrieran.

Karlax se movió de su lado para ir a hablar con el Presidente y ella corrió hacia el Doctor, que estaba apoyado pesadamente en la mesa, con la frente arrugada. 

—Vamos —dijo—. Ven y muéstrame esa sala de observación.

El Doctor miro a su alrededor y ella le sonrió esperanzada. Sabía que tenía que sacarlo de la sala. Aún había tiempo. Él encontraría una manera. Sin embargo, la expresión de su cara sugería que si le dejaba aquí con Rassilon y Karlax, las cosas no iban a terminar bien. Además, ella no creía que pudiese soportar verlos tampoco por más tiempo.

El Doctor se enderezó, recogiendo el cañón Dalek de la mesa. 

—Por aquí —dijo, saliendo furiosa y bruscamente de la habitación.









