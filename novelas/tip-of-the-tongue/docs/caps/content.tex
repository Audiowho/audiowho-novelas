\part*{En la Punta de la Lengua}
\addcontentsline{toc}{part}{En la Punta de la Lengua}

---¿Está roto? ---preguntó Jonny frunciendo el ceño.

La cara de Nettie parecía que estaba preparada para esto.

---No se puede romper ---dijo ella---. Está vivo. O algo por el estilo.

---Bueno, pues enfermo.

---Está bien. Mira, ¿lo quieres o no?

Jonny sintió de nuevo el dinero que había envuelto en un apretado rollo
en su bolsillo. Dos dólares. Una fortuna, casi seis meses de ahorrar las
propinas de níquel que sacaba de su trabajo como ayudante de camarero en
la cafetería del señor Finnegan. Por supuesto, él le daba la mayor parte
de su sueldo a su madre, necesitaban cada penique ahora que su papá
estaba luchando en el extranjero la guerra contra Hitler.

Pero la familia de Nettie también necesitaba el dinero (fue el argumento
Jonny había ensayado una docena de veces en su cabeza para el inevitable
momento en que su madre estallara cuando se enterara de cuánto dinero
estaba a punto de gastar). El papá de Nettie había muerto poco después
de que ella naciera, pero mientras que la madre de Jonny pudo encontrar
un buen trabajo de guerra en la fábrica local de Temperance haciendo
revestimientos para tanques, la madre de Nettie no pudo. La madre de
Nettie era negra, y aunque se trataba de Maine y no el sur profundo, el
señor Acklin, el dueño de la fábrica, había encontrado formas de
mantener a la madre de Nettie sin trabajo. Se estaba manteniendo a ella
y a Nettie con el sueldo de una limpiadora que apenas podía pagar la
ropa del colegio de Nettie.

La madre de Jonny lo sabía. Ella y la madre de Nettie habían sido amigas
desde que Jonny y Nettie se habían conocido cuando tenían cinco años en
el único colegio público de la ciudad, sus madres aunaron esfuerzos
rápidamente de la forma en que a veces lo hacen los marginados. La madre
de la única niña de raza mixta de la ciudad y la madre judía del niño
con apellido alemán.

``Ella lo entenderá'', pensó Jonny, rodando el dinero entre sus dedos.
``¿Verdad?''

---¿Estás segura de que funcionará? ---dijo Jonny.

Nettie suspiró, sonando veinte años mayor que sus catorce.

---Funcionó con el tío Paul ---dijo.

---Se enfadará si lo vendes.

---Está en una isla tropical de Australia en algún lugar disparando a
gente en los pantanos.

El joven tío de Nettie había alquilado una habitación a su madre antes
de que se largara. Había una severa ansiedad en la voz de Nettie que no
tuvo mucho éxito en disimular.---Tiene otras cosas de qué preocuparse.

Jonny todavía estaba indeciso. Nettie chasqueo la lengua exasperada.
---¡Pruébalo! ---prácticamente gritó---. Si no funciona, no tienes que
comprarlo. Podemos olvidar todo el asunto.

---Está bien--- dijo Jonny finalmente, tendiendole la mano. Nettie
colocó el Contador de Verdades en su palma. Este le miró apenadamente,
con sus ojos amarillos llenos de tristeza cansada---.Parece deprimido.

---Si era una máquina, lo arreglé ---dijo Nettie impaciente. Lo cual era
cierto. Podía arreglar casi cualquier cosa. La bicicleta de Jonny varias
veces, una puerta en su casa que nunca había colgado correctamente. Su
tío Paul había sido genial con sus manos y Nettie lo adoraba y había
dado vueltas a su alrededor durante años como la más devota hermana
pequeña del mundo, casi accidentalmente aprendiendo cómo reparar
tostadoras y cambiar el aceite de un Studebaker---.Todo lo que te puedo
decir es que Paul dijo que funcionaba.

Jonny empujó suavemente el pequeño objeto en su mano. No dijo nada, pero
le echó una mirada de sentimientos dolorosos poco sorprendentes.

Dejó escapar un largo suspiro. El Contador de Verdades era real. Era lo
más cerca que nunca iba a estar de poder comprar uno. Podía ser suyo por
dos dólares.

Y puede ser que finalmente, finalmente, haga que Marisa Channing se fije
en él.

Le dio la vuelta, abrió las dos puntas en la parte posterior y desplegó
el largo cuerpo. Nunca había llevado uno y no estaba muy seguro de cómo
se suponía que debía quedar, pero había tantas maneras en que podía ir.
Abrió la boca, puso las dos puntas a cada lado de la lengua, y enrolló
el cuerpo por debajo de su barbilla y bajo su cuello, adornándolo como
si su cara estuviera llevando una corbata.

---¿Cómo me queda? ---preguntó, tropezando su lengua un poco por las
puntas.

Nettie se cruzó de brazos.

---A mi no me preguntes. Odio esas cosas. No apruebo para nada esta
transacción, ¿recuerdas?

---Recuerdo ---dijo---. ¿Cómo haces que diga algo?

---Debería ---dijo acercándose.

---Me gustas sólo como amigo ---dijo el Contador de Verdades, con sus
ojos mirando directamente desde la barbilla de Jonny a la cara de
Nettie.

Ella frunció el ceño.

---Funciona.

Después de que los dos dólares fueran entregados y un desconcertado pero
nerviosamente feliz Jonny, fue por un camino de regreso a su trabajo en
la cafetería y una más rica, pero de alguna forma todavía irritada
Nettie, fue por otro camino de regreso a su trabajo en la gasolinera del
Señor Bacon, un hombre rubio vestido en lo que parecía ser un uniforme
de cricket con lo que parecía ser un tallo de apio en la solapa, salido
de la nada con una mirada pensativa en su rostro.

Pero seguramente nada de eso podía ser posible.

---Seguramente nada de esto puede ser posible ---dijo casi alegremente.

---¿Nada de qué? ---dijo una mujer con largo y rizado pelo, saliendo de
la nada detrás de él. ---Parecían unos niños bastante agradables.

---Sí, Nyssa ---dijo el hombre---. Pero esta es la Tierra de 1945
---tomó una respiración profunda. Había sal marina en el aire---. Maine,
si no me equivoco.

---¿Y?

---Y los Dipthodat se supone que no deben de llegar hasta al menos cien
años. Todavía deberían de estar a mitad camino atravesando la galaxia en
este punto de la línea temporal.

---Entonces, ¿qué están haciendo aquí ahora Doctor?

---¿Qué, en efecto, Nyssa? ---dijo el Doctor, poniendo sus manos en los
bolsillos---. Suena como una pregunta que necesita respuesta.

La moda había empezado justo antes del verano, lo que la hizo complicada
al principio. No había colegio para ir todos los días para comprobar qué
tipos de Contadores de Verdad todo el mundo estaba usando o qué
actitudes que estaban tomando hacia ellos o, lo más importante, cómo de
moda estaban. Además, eran tiempos de guerra: si no tenías más de doce
años (a veces si siquiera eso), tenías un trabajo de verano al que ir,
trabajando en granjas, trabajando en tiendas locales, incluso unos
cuantos niños más mayores que Jonny o Nettie cogieron turnos en verano
en la fábrica de armas del señor Acklin.

De hecho, fue la hija del señor Acklin, Annabelle, quien había aparecido
tres días antes del fin del año escolar con un Contador de Verdades azul
oscuro montado en su barbilla, como una de esas señoras tetonas de los
barcos piratas. Su padre no sólo poseía la fábrica, sino que también
poseía la tienda de Temperance, que su madre llevaba, y Annabelle
también era la primera en cualquier moda que emigrara de las grandes
ciudades del sur. De ningún lado, por cierto, toda la familia Acklin se
había mudado cinco años antes y visitado regularmente, así que ella se
convirtió en una autoridad inherente, una chica con la que ninguna otra
en Temperance se podía igualar. Había llevado la primera falda hasta la
rodilla (en lugar de hasta la pantorrilla) al colegio que nunca se había
visto y se le mandó pronto a casa por ello, aunque durante la noche
todas las otras chicas del colegio habían hecho el dobladillo de sus
faldas hasta la rodilla y el director Marshall, se saturó con lo que
llamó `un estallido similar a la Revolución Bolchevique'. Annabelle
también fue la primera en llevar un cuello de auténtica piel tanto en su
abrigo de verano como de invierno, de conejo, pero teñido para que
pareciera un visón, porque:

---Hay una guerra en marcha, sabes--- dijo.

Además, ~fue la primera en utilizar maquillaje en sus piernas en lugar
de medias después de que los japoneses cortaran el suministro de seda a
América. De todas formas, no es que cualquier otra chica de la ciudad
podría haberse permitido medias de seda, pero las empolvadas piernas de
Annabelle llevó a una semana de todo el colegio lleno de manchas al
nivel del tobillo hasta que el director finalmente se puso firme.

---¿Qué diablos es esa cosa en su cara, señorita Acklin? ---dijo el
primer día de los Contadores de Verdades, como siempre pareciendo como
si estuviera luchando contra un ataque al corazón.

Annabelle alzó su cabeza con orgullo. Ni siquiera había llegado a la
puerta principal de la escuela, y el resto de la población escolar -
incluyendo, Jonny vio, Marisa Channing - tenían sus ojos puestos en
ella.

---¿Qué? ---dijo Annabelle, aparentando un aire de
cúal-es-el-gran-problema insinuando que cualquier shock era enteramente
el problema otra persona---. ¿Esta vieja cosa?

La boca del director Marshall se abrió y se cerró varias veces como si
la incredulidad luchara contra la ira, y perdió. ---¡Quitátelo en este
instante!

Y entonces habló.

La cosa en su mandíbula abrió su dentada boca - se alejó del punto en el
que la barbilla de Annabelle caía- y una ligera y triste voz, que
inmediatamente puso a Jonny en la mente de una vaca triste, sonó,
tranquila pero de una extraña e imposible forma que llevó a todos los
oídos a escuchar con impaciencia.

---Estás calvo--- dijo la voz--- y tu esposa no está visitando a su
hermana en Boca Ratón. Se ha ido con el señor Edmundsen de la panadería.
Todo el mundo lo sabe, sí.

---Podías haber escuchado a un mosquito eructar--- dijo Nettie después
del silencio que siguió, y ciertamente se hubiera podido. La cara del
director Marshall cada vez estaba más y más roja, hasta que finalmente
gritó ---¡Cómo te atreves!

--- No es más que la verdad ---dijo Annabelle, con los ojos brillantes.
---¿Cómo se puede castigar a alguien por la verdad?

Sin palabras, los ojos del director Marshall se abrieron ampliamente y
desgraciadamente eso les dio un momento a todos para reflexionar que,
sí, a decir verdad, el director Marshall estaba calvo y era por todos
sabido que la señora Marshall había sido la clienta más entusiasta de la
panadería antes de que cerrara misteriosamente y ella se fuera poco
después a visitar a su hermana.

Por supuesto, Annabelle fue expulsada, fue enviada a casa de inmediato a
la espera de otras medidas disciplinarias, pero no antes de decir a
todos lo que había un stock de Contadores de Verdades en la tienda de su
madre para cualquiera pudiera estar interesado en comprar uno.

Otras medidas disciplinarias nunca se materializaron. Los padres de
Annabelle eran los más ricos, más poderosos, y francamente la gente más
aterradora de la ciudad, y aunque en realidad a nadie le gustara el
señor Acklin (¿Cómo podrías? La zalamería, la arrogancia de la gran
ciudad, la forma en que malgastaba su dinero, incluyendo derribar la
gran mansión de la ciudad para construir una nueva en su lugar. ¡En
tiempo de guerra!), tampoco nadie podía permitirse el lujo de oponerse a
él. Aunque los Contadores de Verdad nunca se permitieron oficialmente en
los colegios, eso no impidió que una multitud de personas, jóvenes y
mayores, hombres y mujeres, cualquiera que pudiera permitírselos, y con
frecuencia incluso los que no podían, los compraran en tropel.

---Su trasero es manifiestamente demasiado ancho para ese vestido
---Jonny oiría desde la acera mientras vaciaba la basura de la
cafetería. La mujer cuyo trasero estaba en duda daría la vuelta en
horrorizada ira, pero el hombre detrás de ella se ya estaría encogiendo
de hombros con una sonrisa no enmascarada por el Contador de Verdades
que adornaba su barbilla. ---Estas inoportunas cosas ---diría ---No se
puede discutir con la verdad, ¿eh?

En este punto el Contador de Verdades de la mujer hubiera dicho ---Tu
aliento siempre ha olido como una persona muerta hace mucho tiempo. Y
nadie cree que tu excusa de los pies planos para no alistarse con el
resto de nuestros valientes soldados.

Ambas partes se irían enfadadas y frustradas y de algún modo incluso más
preparados para infligir verdad en alguien más.

Dos semanas después de la venta del primer Contador de Verdades, la
pequeña cárcel de Temperance estaba llena repleta de casos de agresión
menor, entre esa gente algunos que hasta ese momento habían sido amigos
de toda la vida. Su pequeño juzgado estaba viendo archivo tras archivo
de nuevos divorcios de parejas que habían sido felices desde antes de la
Gran Depresión. El incompetente ayuntamiento suplicó al señor Acklin que
dejara de vender las `condenadas cosas' pero dijo que no había razón tal
y como es ahora, después de todo:

---Solo es un capricho pasajero. Pasaremos a la siguiente cosa pronto.
Por ejemplo, ¿han oído hablar de esa cosa llamada hula hoop? Sólo un
dólar noventa y nueve en la tienda\ldots{}

Pero al igual que muchas modas pasajeras incomprendidas por los mayores,
es decir cualquier persona mayor de veinte años, los Contadores de
Verdades encontraron un secundario, pero próspero, mercado en la
población en edad escolar de Temperance. Conforme madres y padres
devolvían los Contadores de Verdades a la tienda a gran velocidad (no
por reembolso, por supuesto) se desarrolló una subcultura no oficial. En
el Horizon, el único cine de Temperance, multitud de jóvenes se
agolpaban en sus asientos, esperando a que las luces se apagaran, y
luego discretamente ponerse los Contadores de Verdades abandonados por
sus parientes más ancianos o, en casos más extraños, comprar uno de
segunda mano con su propio dinero bajo el mostrador de la tienda de la
señora Acklin, una mujer que nunca hacía la vista gorda cuando se podía
hacer dinero.

Difícilmente alguien estaría viendo los noticiarios cinematográficos
como una Sesión de Verdades, como empezaron a ser conocidas, empezó en
las primeras cinco filas del anfiteatro.

---A Troy Davis le gustas pero no tanto como tú crees, y no de la forma
en que desearías.

---Crees que tu pelo se parece al de la estrella de cine Veronica Lake
mientras todos comentan a tu espalda de su parecido con el de la melena
de un caballo. Esto no se dice en ninguna forma que pudiera tomarse como
un cumplido.

---Eres demasiado egoísta y llamativo y los espíritus de tus amigos caen
cuando entras en una habitación.

---La historia de Troy Davis de meterse en el camino de la familia de
Debbie Madison por lo que ella tuvo que mudarse es una mentira.

---Tienes un ligero olor del que la gente habla a tus espaldas.

---Todo el mundo sabe que tu padre es un borracho.

---Tus muslos son tan gordos como piensas que son.

---Tú, Troy Davis, estás guardando tus afectos por otro miembro de tu
equipo de baloncesto en una forma que no sería ampliamente aceptada por
las actitudes de la sociedad de este período de tiempo y ubicación.

Las Sesiones de Verdades fueron iniciadas por Annabelle Acklin que, por
supuesto, elegiría la que pudiera ser la víctima del día. Bueno, no
ella, precisamente, estaba completamente fuera de sus manos, ¿no? Fue el
Contador de Verdades quien tenía todas las verdades que decir, y la
verdad, después de todo, no podía ser negada o defenderse contra ella. Y
si a veces después de la Sesión, las personas se retiraban de su círculo
social o decidían sentarse para siempre en otra parte de la clase o
incluso - como cuando Troy Davis mintió sobre su edad para alistarse y
fue enviado inmediatamente al extranjero - personas que se fueron por
completo, entonces quizás Temperance podría funcionar sin ellos, ¿no?

Jonny nunca había estado en una Sesión de Verdades. No lo
suficientemente alto en la cadena alimenticia de la escuela. Además,
demasiado bajito, demasiada apariencia extranjera (con ese apellido que
le colgaba como un saco de trigo) él era, francamente, demasiado
invisible para haber sido siquiera registrado en el radar de alguien
como Annabelle Acklin.

Sin embargo, eso cambiaría. Oh, sí, eso estaba apunto de cambiar.

Debido a que Jonny sabía que Marisa iba a la Sesión de la Verdades,
aunque no creía que alguna vez ella hubiera sido el foco de una, sólo
contribuyó con el Contador de Verdades color rosa nacarado que su padre
le había comprado para su cumpleaños.

Hermosa, alta, elegante Marisa. Con su solitaria peca y su fluido pelo
rubio y su misticidad que la hacían parecer un ángel.

Todas las palabras Jonny tendría preferirían morir que ser dichas en voz
alta.

Aún así, ¿no se sorprendería cuando Jonny apareciera en el anfiteatro
con su propio Contador de Verdades?

---¿Esta vieja cosa? ---diría.

Él no diría que lo hubiera comprado usado a Nettie.

(Por su parte, Nettie nunca había estado dentro de una milla de una
Sesión de Verdades y él sabía que no habría ido incluso si fuera
invitada por la propia Annabelle Acklin.

A veces no entendía para nada a Nettie.)

---Hoy ha ocurrido una cosa rarísima --- dijo su madre, colgando su
abrigo en el gancho en la cocina. Jonny estaba pelando las dos patatas
que tendrían para una cena temprana antes de que se fuera a su turno de
tarde en la cafetería del señor Finnegan. Su madre siempre volvía de la
fábrica oliendo fuertemente a metal y aceite, y para cuando se hubiera
lavado (con el agua que había calentado para ella en el fogón) la cena
ya estaría preparada. Ellos comerían, Jonny volvería a trabajar y su
madre empezaría la costura que hacía por dinero extra. A veces la madre
de Nettie también vendría y más de una vez que había regresado a casa
para encontrarlas en sus sillas separadas, vasos de `café de guerra' en
la mesa entre ellas, ambas completamente dormidas, con costuras a medio
hacer en sus regazos.

---¿Qué cosa extraña? ---preguntó Jonny, cuando ella no continuó. Cortó
la segunda patata y la echó en el agua hirviendo.

---¿Qué? ---dijo su madre, distraída. Echó un vistazo. Su madre estaba
mirando a la imagen de su padre que había colgado de la puerta, tocando
sus dedos a ella, sin decir nada. A pesar de su apellido, y a pesar de
que ambos abuelos de Jonny eran tan alemanes como podían ser, su padre
no había tenido ningún problema para encontrar un lugar en el ejército
de los Estados Unidos y ahora estaba en Europa, luchando contra gente
con los que probablemente estuviera lejanamente emparentado. Es curioso
lo que la guerra hacía, ¿no?

---¿Lo extraño? ---dijo Jonny, suavemente.

---¡Oh! ---dijo, sentándose con cansancio en la mesa de la cocina---.Sí.
Hemos tenido una inspección.

Jonny añadió unas cuantas onzas de guisantes a las patatas hirviendo.
Eso y un poco de mantequilla y sal sería toda su cena---¿Por qué es eso
raro?

---Fue este hombre y la mujer. Británicos, creo, que era bastante
extraño, pero ella llevaba pantalones, si te lo puedes creer, ¡y él iba
todo de blanco para inspeccionar una fábrica! ---se recostó en su silla
como si la incredulidad le hubiera empujado allí---.Sin embargo, él no
parecía haberse ensuciado nunca.

Jonny siguió cocinando, intentando ignorar el Contador de Verdades en su
bolsillo, que en su mente sonaba con estridencia como una sirena,
chillando en la parte superior de sus pulmones (¿tenían pulmones?), y se
había estado preguntando desde que su madre entró por qué ella no había
sido capaz de escucharlo.

Pero, por supuesto, no hizo ningún sonido real. No a menos que
estuvieran siendo usados. No hacían nada a menos que estuvieran siendo
usados. Nunca necesitabas alimentarlos, nunca necesitaban ser limpiados,
ellos solo te observaban, tristes, hasta que te los ponías sobre tu
barbilla y decían las verdades que no eras lo sufucientemente valiente
de decir tú mismo.

Como un ejemplo puramente aleatorio, decirle a tu madre que te habías
gastado dos dólares en un Contador de Verdades cuando solo te podías
permitir carne cada dos semanas.

---Mamá ---empezó.

---Sin embargo deberías de haber visto al señor Acklin ---su madre se
rió para sus adentros.

---Creo que puedo ser feliz con cualquier cosa que haga a ese hombre
retorcerse.

---Mamá ---dijo Jonny de nuevo.

---Continuaba preguntando a todo el mundo por esos horribles Contadores
de Verdades ---dijo su madre.

Jonny se congeló. Su madre no lo hizo.

---Oh, odiaba esas cosas ---frunció el ceño---.Gente siendo grosera todo
el tiempo y actuando como si fuera valiente, actuando como si fuera tu
culpa si te molesta que hayan sido horribles contigo--- ella lo miró
---.Sé que a vosotros los niños os gustan, pero me alegro tanto de que
tú nunca consiguieras uno Jon.

Jonny no dijo nada. Se apartó un mechón de pelo de su ojo.

---¿Qué era lo que ibas a decir?--- preguntó, inocente como cualquier
cosa.

---Nada ---dijo Jonny---. Sólo que \ldots{} el agua para lavarte está
lista. Si la quieres.

---Gracias, Liebchen ---dijo. Se puso de pie y le dio un beso en la
frente. Él la dejó---.Eres un buen chico.

Tomó la olla de agua para lavarse y Jonny se quedó en la cocina, dejando
el vapor de su cena hirviendo volar sobre él.

---Así que ¿funcionó?--- le preguntó Nettie mientras caminaba. Ella se
había encontrado con él de camino a su turno de tarde. Volvía a casa de
su día en la única gasolinera de la ciudad. Se suponía que sólo tenía
que llevar la caja en la pequeña tienda de allí, pero con la inspiración
de su tío Paul para arreglar cosas, y en ausencia de de todos los
mecánicos profesionales que se había llevado la guerra, terminó pasando
más tiempo cambiando correas y remplazando bujías. Olía más a productos
químicos que que su madre. ---¿Se dijeron tus verdades?

---Todavía no he tenido tiempo de probarlo adecuadamente--- dijo Jonny.

Fueron en silencio durante un momento, Jonny pedaleando, Nettie
deslizándose hábilmente en las curvas fáciles como siempre hacía.

---No seas\ldots{}--- Nettie comenzó, pero no terminó.

--- No seas ¿qué?

Nettie detuvo su bicicleta. Jonny también detuvo la suya. Estaban a una
manzana de la cafetería, a una manzana de donde Nettie volvería a la
parte más pobre de la ciudad que ella y su mama llamaban casa.

---Con Marisa--- dijo Nettie, sin mirar a Jonny a los ojos---.No te
sorprendas si no\ldots{} consigues lo que quieres de ella.

Jonny sintió que cada pulgada de su piel comenzaba a sonrojarse. El sol
se estaba poniendo y él sólo podía esperar que estuviera tan oscuro que
Nettie no pudiera verlo sonrojarse.

---No sé de qué estás hablando--- dijo, su voz sólo chilló un poco.

---Las chicas como\ldots{} ---empezó Nettie, pero no dijo como eran las
chicas así. Todo lo Jonny escuchó en su silencio fue: nunca miran a
chicos como tu.

---No sé de qué estás hablando ---dijo de nuevo, más enfadado esta vez y
pedaleó lejos de ella, dejando Nettie para verle irse.

---Tengo una nariz muy grande--- dijo el Contador de Verdades en el
espejo. ---Y mi frente es grasosa y sin duda se llenará de acné a medida
que envejezca.

---Bueno, no le digas eso a la gente--- dijo Jonny, su lengua todavía
tropezaba un poco con las puntas del Contador de Verdades.

Dejó de hablar y le miró con tristeza. Jonny no estaba en un descanso, y
sólo podía estar en el baño tal vez un minuto más antes de que el señor
Finnegan viniera llamando a gritos, gritando que había mesas que
despejar.

Pero Jonny había levantado la vista de su jaula de platos sucios para
ver Annabelle Acklin y tres de sus amigas llegar, sentarse en una cabina
y beber batidos. Dos de esas amigas eran Virginia Watson y Edith Magee.

La tercera era Marisa Channing.

Jonny se había escapado al baño. En el pequeño espejo, intentó dar forma
a su rebelde pelo a algo que se acercara a lo normal, se limpió la
mostaza de sus labios de donde había estado robando patatas, e intentó
calmar su respiración a algo más bajo que la hiperventilación.

Entonces cogió el Contador de Verdades de su bolsillo y se lo puso en su
boca.

No estaba yendo particularmente bien.

---Tengo miedo de que la guerra no termine a tiempo ---dijo ---y seré
enviado a luchar y moriré.

Jonny cambió de un pie a otro.

~---Bueno, todo el mundo tiene miedo de eso. ¿verdad?

---El señor Finnegan está a punto de llamar---dijo.

---¿Qué estás haciendo ahí?--- dijo el señor Finnegan a través de la
puerta, golpeándola a martillazos. ---Esas mesas no se limpiarán solas.

Jonny se quitó el Contador de Verdades y lo metió de nuevo en el
bolsillo.

---Ya salgo--- dijo.

Jonny se acercó a la cabina donde estaban las chicas sentadas. Estaban
bebiendo sus batidos con pajitas sobre sus Contadores de Verdades.

--- Realmente deberías esconder más tus orejas con tu corte de pelo---
estaba diciendo Annabelle a Edith Magee---.Sobresalen

---Te tengo miedo ---dijo el Contador de Verdades de Edith---.Haré todo
lo que digas.

---¿Acaso no lo sé?--- dijo Annabelle con una sonrisa. Entonces vio a
Jonny. ---¿Qué estás mirando?

---Nada--- tartamudeó, solo para luego darse cuenta de que estaba
mirando fijamente. Sostuvo la caja de platos sucios bajo un brazo, pero
su mano libre fue a su bolsillo, buscando su propio Contador de
Verdades.

--- Tienes una nariz muy grande--- dijo el Contador de Verdades de
Annabelle antes de que él pudiera alcanzar el suyo.

---Y la piel muy grasa--- dijo el de Virginia Watson.

---¿No eres judío?--- dijo el de Edith.

---Y definitivamente alemán--- dijo el de Annabelle---.Con tu engorroso
apellido, podrías ser un traidor a este país.

---Yo no ---empezó Jonny.

---¿Y que bajito eres--- dijo el de Virginia.

---Y feo.

---¿Y crees que esos tres pelos bajo tu nariz son un bigote?

---Y nadie sabe quién eres.

---Y nadie lo hará.

Jonny se dió la vuelta y huyó, los platos en la caja tintineaban bajo su
brazo. Él los dejó sobre el mostrador de la cocina.

---¡Eh!--- gritó el señor Finnegan desde la parrilla ---.Si rompes
cualquiera de esos, los reemplazarás de tus propinas.

Jonny respiró mucho un momento. Su único consuelo era que el Contador de
Verdades de Marisa Channing no había dicho ni una palabra.

¿Pero era eso algo bueno? ¿O era la peor cosa de todas?

---Dos batidos de chocolate--- dijo Jonny, instantes después, colocando
las bebidas en la esquina de la cabina sin establecer contacto visual
con la gente que se sentaba ahí.

---Parece que él es el único empleado en todo el establecimiento--- dijo
Nyssa conforme veía a Jonny irse--- a parte del propietario.

---Hay una guerra en marcha--- dijo el Doctor olisqueando su bebida
---.Sólo chocolate y leche y helado--- tomó un sorbo ---.Delicioso.

---Doctor--- dijo Nyssa, algo impaciente. Estaba observando la cabina de
cuatro chicas dos mesas más abajo ---.Dijiste que estaban en grave
peligro.

El Doctor se giró y siguió su mirada.

---Y lo están. Los Dipthodat son despiadados. Se esconden a plena vista,
sosegando un planeta para aceptarlos. Y entonces\ldots{}

No terminó.

---¿No deberíamos de hacer algo?--- preguntó Nyssa.

---Eventualmente--- dijo el Doctor---.Tenemos que averiguar su origen.
Ah--- las chicas de la cabina se habían levantado a la vez. El Doctor no
dejó de beber su batido mientras las observaba ponerse sus abrigos de
verano e irse riendo a la puerta principal ---.Hora de irse--- dijo
tomando un último sorbo y moviéndose para salir.

Hizo una pausa, y luego puso dos dólares de propina.

---¿Funciona?--- dijo Nettie, desde debajo del nuevo y brillante coche,
a pesar de ser tan raro.

---No lo sé---dijo Jonny, mirando al Contador de Verdades en la mano.
Más o menos. Supongo.

Hubo una pausa en el sonido metálico que Nettie estaba haciendo. ---Tú
no lo hiciste, ¿verdad?

Jonny no respondió. Lo cual era una respuesta en sí misma.

---De todos modos, ¿por qué te gusta?---preguntó Nettie. ---Marisa
Channing apenas sabe que estás vivo.

---Ahora lo sabe un poco más.

Nettie se escabulló de debajo del coche. Tenía la mejilla manchada con
aceite. ---¿Conseguiste tu propia Sesión de la Verdad, no?

Una vez más, la respuesta de Jonny consistió en no contestar nada.
Nettie negó con la cabeza y volvió abajo del coche. Él estaba visitando
a Nettie en su pausa del almuerzo el día después de lo desagradable en
el restaurante. Él tenía que volver en un par de minutos, pero aun así
se sentó en el garaje de la estación de servicio, con el sol brillando a
través de la puerta abierta. El Contador de Verdades en su mano sostuvo
la mirada hacia él, como siempre lo hacía. Tristemente.

---¿De dónde provienen éstos, de todos modos?--- preguntó Jonny.

---Europa---dijo Nettie. ---O Sudamérica o algo. Paul dijo que pensaba
que estaban haciendo una cosa química para la guerra\ldots{}

---Bueno, bueno, bueno\ldots{}---una voz retumbó desde la puerta---.
¿Qué tenemos aquí? Era el Sr. Acklin. Entró, con su caro abrigo ondeando
detrás de él en la ligera brisa.

Llevaba un Contador de Verdades sobre su barba rala.

---La niña de raza mixta de la ciudad está trabajando en su
automóvil---dijo el aparato, con tristeza---junto con el único judío de
la ciudad.

---El único judío alemán de la ciudad---se burló Mr Acklin, mirando a
Jonny. ---¿No es así, señor Heftklammern?

Y allí estaba. El peso de dos toneladas de su apellido. Había sido su
perdición mientras crecía Así que sin lugar a dudas era alemán,
especialmente con el alzamiento del Sr. Hitler, como si el pequeño,
judío Jonny Heftklammern no se sentía lo suficientemente excluido entre
los protestantes de Temperance, Maine.

Ni siquiera era un nombre alemán. El padre de su padre había emigrado
justo después de la Primera Guerra Mundial y, teniendo en cuenta de que
esa era otra guerra librada contra Alemania, le habían advertido
anglificar el nombre de la familia al llegar a los Estados Unidos.
Originalmente se había pensado en cambiar Mueller a Miller, pero el
abuelo Dietrich, muerto de hambre y sin dormir después de una largo y
horrible viaje por el mar, se había acobardado frente al rostro severo
del secretario de inmigración. Incapaz de recordar su discurso
largamente practicado sobre su nombre, había echado un vistazo al
escritorio del secretario y dijo lo primero que vio.

Heftklammern era la palabra alemana para ``grapas''.

El Sr. Acklin negó con la cabeza, una sonrisa que se asomaba
desagradablemente por encima de su Contador de Verdades. ---¿Cómo está
tu madre, señorita Washington?--- preguntó a Nettie, que había se
escabulló de vuelta abajo del coche.

---¿Este es tu coche?--- fue la única respuesta de Nettie.

---Sí, en efecto---dijo el señor Acklin. ---Recién comprado. E imagina
mi sorpresa al encontrar a la hija de una doncella de color trabajando
en él.

Nettie se puso cuidadosamente sobre sus pies y tragó saliva, nerviosa,
con la llave inglesa todavía en la mano. Jonny sintió un repentino
destello en su estómago, sorprendido de lo mucho que odiaba ver a Nettie
nerviosa.

---Se necesita un poco de trabajo en el embrague\ldots{}---empezó a
decir ella.

---Me pregunto qué pensaría la ciudad si supieran a quién tiene el Sr.
Bacon trabajando en sus automóviles.

---Ellos han estado muy bien hasta ahora---dijo Nettie.

El señor Acklin sonrió. ---¿Lo han estado? ---dijo él. ---¿Vamos a
averiguarlo con certeza?

Dio un paso brusco hacia ella. Ella, enojada, desafiante, pero de todas
maneras treinta centímetros mas chica, dio un paso atrás. ---¡Eh!---
dijo Jonny, poniéndose de pie, y cuando se volvió a mirar, vio tropezar
a Nettie. Ella cayó de espaldas contra la pintura negra del coche del
señor Acklin.

La llave inglesa en su mano raspó por todo el costado del auto mientras
caía.

Por un momento, nadie se movía excepto el polvo en la luz del sol.

Entonces el señor Acklin sonrió. Una vez más.

---Van a pagar por eso---dijo el señor Acklin.

Su Contador de Verdades añadió: ---Me aseguraré de que nunca trabajas
aquí de nuevo.

---Y esa es la verdad---terminó el señor Acklin. ---No se puede discutir
con la verdad, ¿verdad?

Se fue, de nuevo bloqueando el sol que entraba por la puerta abierta,
diciendo en voz alta el nombre del Sr. Bacon.

---Él no puede hacer eso---dijo Jonny. ---Fue un accidente.

---Puede intentarlo---dijo Nettie, enfadada. ---Y sabes que va a salirse
con la suya--- agarró la llave inglesa y la movió con fuerza hacia la
ventana del auto del señor Acklin.

---¡No! ---dijo Jonny. Nettie se detuvo a mediados del movimiento. ---Él
te hará pagar cristal, también.

Ella dejó caer la llave inglesa a su lado, suspirando. ---¿Qué voy a
hacer ahora?

Jonny recordó de pronto los dos dólares que alguien había dejado,
sorprendentemente, incomprensiblemente, como propina la noche anterior.
Los sacó de su bolsillo y se los entregó a ella. ---Toma esto.

Nettie frunció el ceño. ---Yo no necesito tu caridad.

---Tómalo como un segundo pago---dijo, levantando su Contador de
Verdades.

---Ese no es el precio que acordamos\ldots{}

---Un préstamo, entonces--- todavía tenía el dinero en la mano. ---Para
pagar la abolladura.

Nettie gruñó y luego se lo llevó. ---Un préstamo--- ella miró a través
de la puerta. ---Pero esto no ha acabado. Sr. Bacon no puede\ldots{}

Ella se detuvo, porque el señor Bacon, como si el sonido de su nombre lo
había invocado, entró, con el señor Acklin caminando detrás de él.

---Nettie---dijo el señor Bacon, y la expresión de su rostro era
suficiente.

~

Una campana sonó en la puerta de la tienda de la señora Acklin. Ella se
levantó de detrás del mostrador, con la sonrisa en su rostro congelada
al ver a un hombre y una mujer con ropas raras entrando. Entonces ella
los miró perspicazmente. Tal vez a otras mujeres les gusta usar
pantalones. Tal vez era algo que pudiera vender a esto\ldots{}

---¿Está la propietaria? ---preguntó el hombre.

Por el amor de Dios, ¿era apio lo que llevaba en la solapa de su ropa?

---Sí---dijo la señora Acklin.

---Entonces me pregunto si usted podría decirme dónde encontró éstos,
exactamente?

Extendió la mano. En ella había cuatro Contadores de Verdades, entre
ellos uno que reconoció como el de su hija Annabelle. Lo miró con miedo.

Entonces comenzó a gritar.

~

Jonny siguió esperando a que ella llorara. El llanto al que él estaba
acostumbrado. Su madre lloraba mucho, sobre todo cuando recibió una
carta de su padre, aunque eso fue en su mayoría un alivio, que
significaba que su padre podía escribir cartas y no había sido devorado
por la guerra en Europa.

Él también había visto un montón de llantos por la ciudad, cuando esa
misma guerra se llevaba esposos e hijos, hermanos y novios. Había visto
llorar a la madre de Nettie cuando su hermano, el tío Paul, había sido
enviado fuera para un segundo viaje a principios del verano, y la había
visto llorar de nuevo cuando ella compartió noticias de él con su propia
madre durante sus sesiones de costura.

Nunca había visto llorar a Nettie, sin embargo. Ni siquiera cuando Paul
se fue.

Y ella no lo estaba haciendo ahora tampoco.

Bueno, no realmente.

---Voy a matarlo---dijo por centésima vez. ---Voy a tomar esa llave
inglesa y voy a golpear ese cerebro racista por la parte trasera de su
cabeza.

Tomó un sorbo de la botella de soda que Jonny había sacado para
compartir después que su turno de noche hubiera terminado. Tenía que
llegar a casa; había trabajado todo el día y estaba muy cansado, y ya
era de noche, y el día siguiente era otro día en el que iba a hacer lo
mismo de nuevo. Pero en cambio, estaba sentado en el banco de la parada
de autobús en la calle donde se encontraba la casa más grande de la
ciudad. Propiedad, por supuesto, de los Acklin. La que habían construido
justo en el medio de una guerra, que parecía casi criminal.

---La gente probablemente lo apreciaría---dijo Jonny. ---Pero entonces
irías a la cárcel.

---Por lo menos no tendría que pagar por la comida---dijo ella, con
amargura.

---No puede hacerlo---dijo Jonny. ---No fue culpa tuya. Le diré al Sr.
Bacon, y estoy seguro que en un par de días\ldots{}

---Si lo haces, Acklin podría decidir que no le gustan los judios---dijo
Nettie, casi con calma. ---Tu mamá necesita ese trabajo.

---Y tú necesitas el tuyo---dijo Jonny. ---Ya se nos ocurrirá algo. Te
lo prometo.

---Oh, lo prometes, ¿verdad?--- dijo ella, pero el sarcasmo era luz y
cuando la miró, sus ojos brillaban mojados bajo la luna. Cuando ella vio
que él se daba cuenta, se los secó apresuradamente con el dorso de la
manga. ---¿Qué le voy a decir a mi mamá? ¿Qué vamos a hacer?

---Ya se nos ocurrirá algo---repitió Jonny. ---Lo digo en serio.

Nettie lo miró. ---Sé que lo dices en serio---dijo, en voz baja.
---Siempre lo dices en serio.

Y la manera en que lo dijo hizo que sonara como una cosa muy buena. Él
la miró, y ella la corrió hacia el otro lado, tomando otro trago de la
botella de soda. ``Solo me caes como amiga'' había dicho su Contador de
Verdades, y ni él ni Nettie lo había cuestionado en absoluto.

Pero viéndola ahora, al verla enjugar más lágrimas que ella esperaba que
él no se diera cuenta, de pronto sintió que le gustaría mucho agarrar la
llave y destrozarle los sesos al Sr. Acklin, sólo por hacer sentir de
esa forma a Nettie.

---Nettie---empezó Jonny.

---Alguien viene---dijo ella, sentándose.

Miró por la calle. Era un grupo de chicas, el grupo de las niñas,
dirigido por Annabelle Acklin. Y también estaban Virginia Watson y Edith
Magee. Y, por supuesto, Marisa Channing, luciendo como si la luz de la
luna existiera sólo para iluminar su piel fresca y clara. Estaban
discutiendo profundamente. Annabelle sonreía con esa sonrisa peligrosa a
Virginia y Edith, que parecían estar discutiendo y llorando al mismo
tiempo.

Ninguna de ellas llevaba sus Contadores de Verdades.

---Ahora es tu oportunidad---dijo Nettie.

Por un segundo, se había olvidado de que Nettie estaba allí. Él la miró.
Sus ojos parecían un poco desafiantes, y tal vez no de una manera
agradable. ---¿Oportunidad para qué? ---dijo él.

---Oportunidad para impresionar a tu Marisa---dijo ella. Se levantó del
banco, con la botella de soda todavía en la mano.

Él la miró por un momento más, luego a las niñas que aún se acercaban
por la calle, obviamente en dirección a la casa de Annabelle. ---¿Eso
crees? ---preguntó.

Los hombros de Nettie se cayeron y le sacudió la cabeza. A Jonny no le
gustaba la expresión de su cara. ---Idiota---escuchó que ella susurraba.

---¿Por qué?

---Sólo póntelo---dijo ella. ---Vamos a ver lo que sucede cuando Marisa
Channing oye cuando le dices la verdad. Vamos, vamos a ver qué pasa, ¿de
acuerdo?

---Nettie, yo no\ldots{}

---Porque tú piensas que es sólo una cuestión de verdad, ¿no? Ella verá
que te alcanza para comprar una de esas cosas horribles y que no tienes
miedo de decir cosas groseras y horribles a todos los demás, pero cuando
la miras a ella la verdad se trata de lo bonita que es su piel y qué tan
ondulado es su cabello y de cómo el mundo deja de oler pedos cuando ella
pasa.

Jonny parpadeó, ruborizándose. La mayor parte de eso era correcto,
aunque no se le había ocurrido lo de los pedos\ldots{}

---Honestamente---dijo Nettie, sacudiendo la cabeza. ---Las mentiras que
la gente se dice a sí misa y llaman verdad. Bueno, adelante. ¿Qué estás
esperando?

Las chicas estaban casi directamente al frente. En un segundo, entrarían
a la casa de Annabelle y el habría perdido su oportunidad. Su estómago
estaba rugiendo por lo enojada que Nettie parecía, y fue sólo en ese
momento cuando se dio cuenta de lo que podría estar pasando y lo que
podría él pensar sobre eso. Pero también había otra voz en él, la que
había mirado a Marisa Channing durante tanto tiempo con un anhelo que no
podía describir, y ella estaba a punto de entrar a la casa
Acklin\ldots{}

Metió la mano en el bolsillo y sacó el Contador de Verdades. Ignoró la
mirada en el rostro de Nettie mientras ponía los dientes del aparato
debajo de su lengua y lo cubría por debajo de la barbilla Se levantó,
con su altura no tan impresionante, respiró hondo y dio un paso hacia
adelante para cruzar la calle.

En el mismo momento en el que la casa Acklin explotó.

~

No eran sólo las ventanas que explotaban hacia fuera o una habitación o
una puerta o algo por el estilo. Todo el edificio explotó, los tres
pisos enteros, incluyendo el techo, hacia afuera y hacia arriba en un
destello de fuego y luz. Las cuatro niñas de pie delante de la casa y
Jonny y Nettie todos gritaban mientras la onda expansiva los lanzó hacia
atrás.

Jonny ni siquiera pensó. Se arrojó delante de Nettie mientras el silbido
de las llamas y el humo los envolvían y cayeron juntos en la parada de
autobús. El fuego se apoderó de ellos, pero no los quemó y desapareció
en un instante. Protegió a Nettie tanto como pudo; ella era varios
centímetros más alta que él, así que se preguntó, ¿de qué servía lo que
estaba haciendo? Jonny seguía esperando quedar cubierto bajo una lluvia
de escombros y ladrillos\ldots{}

Pero no había nada.

---Suéltame---dijo finalmente Nettie, empujándolo. Al otro lado de la
calle, las cuatro chicas se juntaron en la acera, pero parecían tan
ilesas y confundidas como Jonny y Nettie.

La casa Acklin no existía más. Ni siquiera quedaban escombros envueltos
en llamas.

Sólo\ldots{} no existía más.

---¿Qué ha pasado?--- dijo Jonny.

---Ellos construyen sus casas con una especie de polímero
secretado---dijo una voz femenina con acento detrás de ellos. ---Es muy
similar a su azúcar, en realidad, lo que es curioso---. Una mujer con el
pelo castaño y rizado y un pantalón extendió su mano para ayudarlos.
---Podría ser el origen de todas esas casas de caramelo que hay en sus
cuentos populares.

Aturdido, Jonny tomó la mano de la mujer y dejó que lo levantara. Nettie
hizo lo mismo.

---De cualquier manera----continuó la mujer---sólo parecía una casa.
Pero cuando el polímero es disparado contra con sus propias armas, más o
menos se vaporiza, quemando rápidamente a una temperatura muy baja y
dejando pocos residuos.

Jonny y Nettie la miraron.

---¿Qué?--- dijo Jonny finalmente.

---¿Es usted británica?--- preguntó Nettie, como si esa fuera la parte
más sorprendente de todo el asunto.

La mujer simplemente les sonrió y luego extendió la mano para tocar con
un dedo suave el Contador de Verdades que Jonny todavía llevaba puesto.

---Estoy terriblemente confundido---dijo el aparato.

---Apuesto a que lo estás---dijo la mujer. ---Pero hemos pasado el día
recogiendo a sus hermanos y hermanas.

---¡Nyssa! ---gritó una voz desde el otro lado de la calle. Un hombre
estaba saliendo del lugar en donde la casa Acklin solía estar.

Y no estaba solo.

Se escuchó un pequeño jadeo cuando Virginia Watson se desmayó. Todos los
demás simplemente se quedaron allí mirando, boquiabiertos.

El hombre estaba llevando lo que parecían unas ovejas muy grandes, pero
ovejas que tenían la cara de una especie de pez gigante. Pero mezclado
con una ardilla. Y una calabaza.

La oveja-pez-ardilla-calabaza no parecía muy feliz. Ambas tenían sus
pies delanteros-aletas atados, mirando hacia todas partes como si
estuvieran bajo arresto.

---¡Yo los he puesto bajo arresto!--- el hombre gritó, guiándolos por
los restos de la vía principal. ---Yo puedo hacerlo, ¿o no?---

Edith Magee gritó mientras se acercaban y huyó por la calle, gritando
por todo el camino. Annabelle y Marisa se quedaron mirando, congeladas,
con los ojos muy abiertos.

La mujer británica empezó a cruzar la calle hacia el hombre. Le hizo una
seña a Jonny y Nettie. ---Vamos---dijo ella. ---Todo está bien.

Aturdido, Jonny se dio cuenta que la seguía. Miró hacia atrás. Nettie
estaba siguiéndola también; su rostro parecía tan sorprendido como el
suyo. Todavía estaba agarrando la botella de soda.

---Y ahí está el último---dijo el hombre, mirando la barbilla de Jonny
mientras se acercaban. La oveja-pez lanzó chillidos enojados, pero se
callaron cuando el hombre las miró con severidad.

---¿Qué está pasando?--- preguntó Nettie, mirando a Marisa y a
Annabelle, que seguía con los ojos abiertos. ---¿Quiénes son?

Pero antes de que el hombre pudiera responder, el Contador de Verdades
de Jonny habló.

---Usted es El Doctor---dijo.

---¿Quién?--- dijo Jonny.

---¿Qué? ---preguntó Nettie.

Pero el doctor estaba hablando con el Contador de Verdades. ---Lo
soy---dijo. ---Y tú, mi pequeño amigo, estás a salvo.

---Estoy seguro---dijo el Contador de Verdades, y Jonny se dio cuenta de
que, por primera vez, no sonaba triste.

---¿Esos son tus padres?--- dijo una voz. Todos se volvieron cuando se
dieron cuenta que era de Marisa. Annabelle estaba todavía de pie a su
lado, mirando tanto furiosa como aterrorizada.

---¿Mamá?--- dijo Annabelle. ---¿Papá?

Hubo una pausa mientras todos se daban cuenta que hablaba con las
ovejas-pez.

---Me temo que tendrás que venir con nosotros, también, Annabelle---dijo
El Doctor.

La oveja-pez hizo un sonido extraño.

He ignored the look on Nettie's face as he fitted the prongs under his
tongue and draped it down his chin.~Pero me gusta estar aquí---dijo
Annabelle. ---Todo el mundo me obedece.

Las dos ovejas-pez hicieron más sonidos que parecían indicar que también
les había gustado esa parte.

---Me enteré del accidente de la fractura temporal---dijo el hombre a
Annabelle. ---Tus padres me lo explicaron---. Volvió a mirar hacia el
agujero donde la casa había estado. ---Después de un rato. Cayeron aquí
por accidente, cien años antes de la línea de tiempo natural. Y eso de
alguna manera reducirá las sentencias que recibirán.

---¡¿Las sentencias?! dijo Annabelle.

---La esclavitud es ilegal,---dijo el hombre, de pronto más severo y,
admitámoslo, daba un poco de miedo también ---en este sistema solar y
cada sistema vecino entre aquí y su planeta natal Dipthodat. Y lo que es
peor para ti---se inclinó hacia Annabelle---No me gusta la esclavitud.

Annabelle parecía que iba a discutir con él, pero la oveja-pez hizo esos
extraños sonidos y Annabelle, aunque sin dejar de parecer molesta,
finalmente dijo: ---Oh, está bien---. Suspiró y se sacó la chaqueta que
llevaba puesta, y delante de todos ella \ldots{}

Se transformó en una oveja-pez. Hizo unos sonidos molestos, dirigidos al
hombre.

---No, no te pondré detrás de las rejas---dijo el hombre---, siempre y
cuando se comporten.

La oveja-pez Annabelle se acercó a sus padres ovejas-pez y comenzaron a
hablar en esos sonidos extraños, que parecían los sonidos que hacen los
peces.

---¿Alguien podría explicar qué está pasando?--- dijo Nettie, y Jonny
vio que estaba sosteniendo la botella de soda como un arma potencial
contra las ovejas-pez, el misterioso Doctor con un suéter blanco y la
mujer británica con pantalones.

---Qué valiente eres, Nettie---dijo El Doctor. ---El peligro ha pasado.
Una desgracia que haya tenido que destruir sus viviendas, pero los
Dipthodats pueden ser una especie bastante tediosa.

---¿Diptho-qué? ---dijo Jonny.

El Doctor hizo un gesto hacia la oveja-pez. ---Dipthodat. Una raza
xenófoba.

---Xeno-qué? ---dijo Jonny.

---No les gusta nadie que no sea como ellos---dijo El Doctor.

---¡Ya lo creo!---dijo Nettie, mirando a la oveja-pez, que le devolvió
la mirada.

---Ellos van a planetas---dijo El Doctor---,avivan el malestar entre los
lugareños, y se alimentan de la energía negativa. Eso es lo que comen.
Toda su dieta. Toda la materia de la lucha y la ira y el odio---. El
Doctor se inclinó a la altura de Jonny y miró al Contador de Verdades de
nuevo. ---Estas pequeñas criaturas maravillosas ayudan a completar esa
tarea enormemente--- miró a los ojos de Jonny. ---Convences a la gente
de que lanzar las ``verdades'' más dolorosos es una buena idea, y
obtienes suficiente energía negativa para dirigir el mundo.

---Funcionan mejor en los jóvenes---dijo Nyssa. ---En buena gente como
ustedes. Tanto parece verdad que se siente doloroso. El suministro es
casi interminable. Rompe el corazón.

~

---Pensamos que eran juguetes---dijo Jonny.

---No juguetes---dijo El Doctor. ---Esclavos. Se llaman Veritans.
Pequeños psíquicos que pueden buscar lo que todos piensan que es la
verdad. Conquistada hace siglos por los Dipthodats y forzados a
trabajar. Llorando por su mundo natal, sin paga, sin consuelo, y sin
esperanza de escape--- frunció el ceño una vez más en la oveja-pez.
---Lo cual es algo que me hace muy triste.

La oveja-pez parecía avergonzada.

---No se supone que lleguen a su planeta por otros cien años, cuando,
por suerte, estarán equipados para encargarse de ellos ustedes mismos.
Estos tres se quedaron atrapados aquí por accidente. Les tomó un par de
años para conseguir disfrazarse y establecerse, y luego se aprovecharon
de un mercado negro traficante de esclavos para encontrar algunos
Veritans.

---Ahora estoy a salvo---dijo el Contador de Verdades de nuevo.

---Sí, lo estás---dijo El Doctor. ---Y te iras a casa.

---Me voy a casa---dijo el Contador de Verdades, y la felicidad en su
voz le rompió el corazón a Jonny.

---De acuerdo---dijo Nettie. ---Perdóname si no estoy entendiendo nada.
Los Acklins eran del espacio exterior desde el principio?

---¿Es eso un alivio? ---preguntó El Doctor.

---No---dijo Nettie, enfadada. ---Parecían bastante humano. De todas las
formas equivocadas---. La oveja-pez la miró. Ella levantó la botella de
refresco, pero en su rostro no había miedo.

El Doctor asintió. ---Supongo que los Contadores de Verdad no fueron la
única travesura que causaron.

---¿Travesura? ---dijo Nettie, molesta por la palabra.

---Bueno, se irán pronto---dijo El Doctor. ---¿Ayudará eso a las cosas
por aquí?

Nettie asintió, desafiante. ---Eso ayudará, si.

Hubo un chillido agitado proveniente de la oveja-pez Annabelle. Ella se
contorsionó hasta la mitad, de nuevo en una forma humana, y gritó:
~---¡Baja eso! ¡Es mío!

El Doctor la detuvo mientras intentaba arremeter contra Marisa que,
ahora podían ver todos, había recogido el abrigo de Annabelle y fue
envolviéndolo alrededor de sus hombros.

---¿De verdad? ---dijo Marisa. ---Creo que probablemente se ve mejor en
mí ahora.

~

Se meció de lado a lado, haciendo girar la capa detrás de ella, frotando
su cara contra la piel. Ella rompió el cuello del abrigo bajo la
barbilla y enderezó la tela. Miró a Jonny. ---¿Qué piensas?

---Tú \ldots{}---dijo Jonny, mirándola de cerca ahora, viendo cómo
parecía estar enamorada del abrigo, viendo cómo se levantaban las cejas
con expresión de alabanza, una mirada que nunca había visto antes en
ella.

O, por lo menos, una mirada que nunca había notado.

---Te pareces mucho a Annabelle---dijo.

---Yo no me parezco en nada a eso---dijo Marisa, asintiendo con la
cabeza hacia Annabelle.

---No, pero\ldots{}---dijo Jonny, pero vaciló.

---De repente no siento ningún tipo de interés hacia ti---dijo su
Contador de Verdades---aunque exactamente no se por qué.

---Ya se te ocurrirá la razón---dijo El Doctor.

---¿Tú estabas interesado en mí? ---dijo Marisa, con una expresión en su
cara que solo podía describirse como horror.

Jonny escuchó un sonido de Nettie, pero cuando se volvió hacia ella,
estaba silbando inocentemente con la botella de refresco vacía.

El sonido de las sirenas se escuchó débilmente en la distancia. Los
camiones de bomberos estaban en camino.

---Muy tarde, después de una explosión tan grande---dijo Nettie.

---Oh, tenemos formas de organizar un retraso---dijo El Doctor. luego
una mirada triste apareció en su rostro. ---El señor Heftklammern
(fantástica nombre, por cierto, no se te ocurra cambiarlo) y señorita
Washington. Las personas decentes de este pueblo hablan muy bien de
ustedes, y los indecentes---dijo, mirando a las ovejas-pez---hablan muy
mal de ustedes--- suspiró---. ---He perdido a dos amigos recientemente.
Uno murió valientemente, y la otra volvió a su propia vida. Nyssa es una
compañera maravillosa, pero me gusta estar rodeado de un montón de
gente--- hizo una pausa. ---Algunos de los dos estaría interesado en
viajar?

---Doctor---dijo Nyssa, con un tono de advertencia en su voz.

---Estamos en medio de una guerra---dijo Jonny. ---Viajar es peligroso.

---No puedo---dijo Nettie, casi al mismo tiempo. ---Mi madre depende de
mí.

---La mía también---dijo Jonny.

---Pero, por supuesto---dijo El Doctor. ---Por supuesto, es verdad---.
Las sirenas se acercaban. ---Lleva a los Dipthodats a la TARDIS,
Nyssa---. Se volvió de nuevo hacia Jonny y Nettie. ---Díganle a la
policía que escucharon una explosión y que eso es todo lo que saben.

---No creo que ellos creerían cualquier cosa del resto de todos
modos---dijo Jonny.

---No estoy segura de que yo lo crea---dijo Nettie---y lo vi con mis
propios ojos.

---¿Están todos los Veritans a bordo?--- preguntó El Doctor a Nyssa.

---Todos menos uno---dijo Nyssa, llevándose a una oveja-pez; Annabelle
le chillaba a Marisa, que no le prestó atención en absoluto.

---Todos menos uno---El Doctor repitió y se acercó a Jonny. ---Es hora
de irse---le dijo al Contador de Verdades.

---Libertad---dijo.

---Libertad, ciertamente.

---Tengo una verdad más que decir---dijo.

---Apuesto a que si---dijo El Doctor, pero lo tapó suavemente con la
mano antes de que el Contador de Verdades pudiera hablar. Se estiró de
la barbilla de Jonny, sacó los dos dientes de su boca, y fue felizmente
hacia la palma abierta del Doctor. El Doctor miró a Jonny y luego a
Nettie. ---Pero sospecho que esa es una verdad que nuestro joven amigo
aquí debería guardársela para si mismo.

~

Una especie de chirrido embrujado llenó el interior de la TARDIS
mientras aceleraba sus motores para salir.

---¿Primero dejamos a los Dipthodats con las autoridades apropiadas,
Doctor? ---preguntó Nyssa.

---En efecto---dijo El Doctor, mirando al niño y a la niña en la
pantalla delante de él.

---Y luego, ¿adónde vamos?

---'Oh, ya sabes---dijo, mientras el chico y la chica (no de la mano, no
besándose, nada ridículo) se alejaban juntos después de dar cualquier
explicación que fuera suficiente a los bomberos. Pero caminaban juntos
de una manera sociable, de una manera que dio a entender, tal vez, sólo
tal vez, futuros por venir. ---Tenemos todo el tiempo y el espacio por
delante de nosotros. Y detrás de nosotros, para el caso. Y por debajo y
por encima de\ldots{}

---Doctor\ldots{}---dijo Nyssa, sonriendo sobre la canasta que los
Contadores de Verdades usaban para dormir. Estaban todos acurrucados,
murmurándose uno a otro que eran libres.

---Todo lo que estoy diciendo, Nyssa---dijo El Doctor, apagando la
pantalla---es que, como siempre, todo es posible.

Él apretó un botón, y envió a la TARDIS volando hacia la gran eternidad
del espacio y el tiempo.

---Lo cual, al final---dijo---es toda la verdad que realmente se
necesita.