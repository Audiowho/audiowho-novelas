\chapter*{CAPÍTULO SIETE}
\addcontentsline{toc}{chapter}{CAPÍTULO SIETE}

{El Doctor continuó subiendo más aun en la penumbra fría y agobiante. El
 agua congelada goteaba sobre él desde arriba, implacablemente. Los
 peldaños de metal atravesaban las finas suelas de sus deportivas y le
 dolían los brazos y las piernas del esfuerzo. Las pequeñas luces blancas
 hacían muy poco para disipar la oscuridad, por lo que se detenía cada
 pocos metros para intentar escuchar el ruido de alguien encima de él.
 Quien fuera, no se secaba, sus dedos seguían encontrándose en los
peldaños húmedos.}

{Contuvo el aliento y escuchó: nada excepto el rápido compás de sus dos
corazones resonando en sus orejas.}

{Finalmente se topó con una barrera de metal sólido, bloqueándole el
 camino hacia arriba. Ya que encajaba en unas muescas a cada lado del
 pozo, aquella tenía que ser la plataforma del carguero, esperando para
 llevar hacia abajo fuera lo que fuera que cargaran. Y ya que no había
 controles en el fondo del pozo, debían de estar controlados desde el
 remolque de arriba. ¿Pero cómo iba a poder atravesarlo? Manoseó con una
 mano por la parte baja de la plataforma. Tenía que haber una\ldots{}}

{--Escotilla de inspección --silbó el Doctor, con sus manos cerrándose
 en un tirador de metal. Estaba templado y pegajoso entre sus dedos
 cuando lo giró, abriendo la escotilla. La cubierta circular de ésta se
abrió. Una pálida luz brillaba desde allí, iluminando la escotilla.}

{El Doctor vio cómo su mano ahora estaba pegajosa de sangre.}

{--Curioso, muy curioso --dijo--. Y también desagradable y asqueroso.}

{Tendría que inclinarse hacia arriba y empujarse a través del agujero
 bajo la parte inferior de la plataforma. Eso le dejaría colgando sobre
 una caída de al menos unos noventa metros, mientras quien fuera que
 hubiera allí arriba podría estar esperándole a aparecer, sangrando y
muy, muy húmedo.}

{Comenzó al oír una repentina lluvia de sonidos muy por debajo: el
 amortiguado grito de una orden, pisadas cronometradas, el crujido de
algo encendiéndose. Entonces hubo un silencio incómodo.}

{El Doctor decidió inclinarse hacia la escotilla abierta, justo cuando
comenzó el tiroteo.}

\mbox{}

{--Bueno, esto es divertido --anunció Mickey, y Rose le lanzó una
mirada.}

{No podía culparle, a pesar de todo. Habían estado sentados en un tenso
 silencio durante una hora, esperando a que pasara algo espantoso. Por el
 sonido de los ronquidos suaves, Anne había optado por abandonar el
maratón de nervios por el momento.}

{A través de la ventana, Rose era ligeramente consciente de lo que
 pasaba en el mundo a su alrededor. Los chicos se gritaban los unos a los
 otros en la calle, pateando latas. Música de baile sonaba a todo volumen
 desde algún coche. Un grupo de chicas paseaba por la acera haciendo
 sonar sus tacones, emocionadas y riéndose. Era viernes por la noche.
 Hacía mucho tiempo, sólo eso habría sido motivo de celebración. Habría
sido impensable quedarse en casa un viernes por la noche.}

{Keisha comenzaba a enviarle mensajes a ella y a Shareen los martes y
 planeaban su gran noche con todo detalle: cómo peinarse, qué ponerse,
 qué bar recorrer. Qué clubs, por supuesto, y eso era un problema muy
 serio. Tenías que valorar a los DJ por lo buenos que eran y lo monos que
eran y las discusiones y las risas se alargaban toda la noche.}

{Juntas, Rose y sus amigas hacían que los viernes llegaran antes y
 fueran lo que más duraba en toda la semana. Ese era el tipo de viaje en
 el tiempo que el Doctor nunca entendería ni en un millón de años. Y aun
 así, todo le parecía ya muy lejano. Sólo tenía diecinueve años, pero
estar con el Doctor le había hecho crecer mucho más.}

{O quizá crecía más rápido.}

{--Voy a dar una vuelta, a que me dé el aire --anunció Mickey--. Quiero
 mirar los diarios, a ver si hay algo más sobre ese barco fantasma, gente
desaparecida y todo eso.}

{--Es una buena idea --dijo Rose.}

{--¿Está abierto hasta tarde el quiosco de la esquina?}

{Kesha asintió:}

{--Quizá hoy no. Algo ha pasado ahí, ¿no es verdad?}

{Rose asintió.}

{--El Doctor dijo que unos clientes se desmayaron o algo. Había
 ambulancias y\ldots{} --se detuvo en seco--. Oh, cielos. ¿Cómo he podido
estar tan ciega?}

{Mickey frunció el ceño.}

{--¿Qué?}

{--Justo antes de que Jay apareciera, esa gente en el quiosco de la
 esquina se desmayaron o algo así. Y cuando Anne vio a su hijo, los
soldados también se desmayaron.}

{--Y yo me sentía muy mareado --asintió Mickey--. ¿Qué quieres decir?
¿Que está relacionado?}

{--Jay nunca haría daño a nadie --dijo Kesha, llanamente.}

{Rose la miró con compasión.}

{--Puede que no lo hiciera a propósito pero\ldots{}}

{--¡Nunca lo haría! --su cara se oscureció-- ¿Qué te hace pensar que
 sabes tanto sobre esto, Rose? Sólo porque hayas estado viajando\ldots{}}

{--Oh, ¿qué pasa? ¿Que el Doctor me ha dado aires de grandeza? ¡Suenas
como mi madre!}

{--¡Ey! --silbó Mickey, señalando a Anne-- La vais a despertar.}

{Rose y Keisha se callaron. A través de las sucias ventanas llegaba el
amortiguado sonido de la música y de la gente pasándoselo bien.}

{--¿Por qué no pueden ser las cosas como solían ser? --susurró Keisha.}

{Rose no tenía respuesta.}

{Lentamente, se levantó.}

{--Creo que intentaré ir a ver qué ha pasado en el quiosco.}

{Mickey frunció el ceño.}

{--Voy contigo.}

{--No vale la pena, serán solo cinco minutos --señaló a Anne,
articulando: ``Cuida de ella''--. Te traeré ese diario, ¿vale?}

{--Pero yo soy el que quería aire fresco --parecía herido--. ¿Crees que
no puedo manejarlo?}

{--Te mareaste la última vez, ¿recuerdas? Si algo raro pasara ahora, te
podrías poner peor.}

{--Bueno, ¿y qué pasa contigo? ¿Qué pasa si tú te mareas esta vez?}

{--Son solo cinco minutos --se giró y cruzó la puerta principal--.
Cuidaos.}

{--Sólo porque seas su acompañante eso no te hace indestructible --gritó
 Mickey detrás de ella, pero había cerrado la puerta y sus pisadas ya
casi no se oían.}

{Mickey miró a Keisha, aun de rodillas en el suelo delante de la silla.
Le estaba mirando.}

{--Aquí estamos de nuevo, entonces --dijo ella con frialdad--. Solos al
fin.}

\mbox{}

{Mientras el estridente repiqueteo letal de una docena de armas
 comenzaba, el Doctor se lanzó contra el borde de la escotilla abierta.
 El sonido del tiroteo, distorsionado y aumentado por la acústica
 cavernosa de la piedra, retronó a su alrededor mientras se intentaba
 agarrar. Estuvo colgado en la oscuridad, agarrado con las puntas de sus
 dedos mientras las balas le pasaban rozando. Metralla de hormigón y
pedazos de metal explotaban en el aire, arañándole la piel.}

{De repente, el fuego cesó. Con sus orejas aun pitando, el Doctor se
 arrastró a sí mismo a través de la escotilla de inspección y se puso
 boca abajo, con el frío suelo y mojado bajo su piel. Las puertas del
 montacargas del carguero estaban abiertas y vio que la plataforma estaba
 alineada con el suelo de otro enorme túnel, esperando para la siguiente
carga para ser colocada a bordo.}

{Pero no había nadie esperándole. En la ligera luz del túnel de acceso
 el Docto podía ver salpicaduras de agua sangrienta en la marcada
superficie de la plataforma.}

{Otro fuerte repiqueteo de tiroteo comenzó y el Doctor sintió los
 impactos de las balas mientras chocaban con la parte baja de la
 plataforma. Entonces la voz de un hombre se oyó, iracunda, e incluso los
ecos sonaban de alguna manera secos y oxidados con la edad.}

{--Subid ahí arriba y alertad a las patrullas del perímetro --dijo--.
 Quiero ese montacargas bien cubierto, pero nadie subirá a bordo hasta
que yo lo diga.}

{--Enseguida, señor.}

{El Doctor asintió para sí. «El Vicealmirante Crayshaw, supongo».}

{--Los demás, dispersaos y buscad por donde hemos venido --hubo una
 pausa y entonces, antes de que los espeluznantes ecos cesaran--. Yo
comprobaré la piscina por mí mismo.}

{El Doctor arrugó la nariz.}

{--Buena suerte.}

{Se puso en pie y estaba a punto de marcharse cuando oyó el eco de otra
 voz. Era más suave, más silbante, una voz femenina quizás, alzándose
como una niebla repentina por las aguas oscuras.}

{--Permaneced selectivos con quienes escogéis. Por ahora, necesitamos a
esos soldados con vida.}

{«¿Sabías que puedes hacerte con ellos en cualquier otro lado? ¿Pero
quién eres?».}

{--Capturaremos al que se ha escapado. Atadle. Traedle de vuelta. Debe
de hacer su parte para atraer a los suyos al festín.}

{--¿Yo? --murmuró el Doctor-- ¿O el que ha venido por delante de mí?}

{Ardiendo de curiosidad, durante un lunático momento el Doctor pensó en
 asomarse por el pozo para ver qué estaba pasando. Quien fuera que
 hubiera dejado la sangre y el agua estaba como poco herido y seguramente
 bastante asustado. Un compañero fugitivo, parecía, ¿pero de qué había
 escapado? El Doctor sabía que debía encontrarle rápido y alcanzar la
 cubierta del remolque antes de que los soldados le cortaran por completo
su única posibilidad de escape.}

{Así que se escabulló por el túnel de acceso, chapoteando una vez más a
 través del agua mugrienta y congelada. Se iban volviendo más profundos
 cuanto más se alejaba, cada vez más apestaba a sal y el agua le llegaba
 a las pantorrillas. Pero tenía que ser transitable, pensó, girando una
 esquina. No había aun señal del misterioso fugitivo\ldots{} Vale, como
si no hubiera dicho nada.}

{El Doctor frenó con un derrape. Una oscura silueta estaba girada frente
 a él, a medias en la sombra, encorvado contra una puerta en la pared. Se
agitaba y temblaba, tosiendo en busca de una respiración.}

{--¡Hola! --le llamó el Doctor-- ¿Estás bien? ¿Te has herido?}

{La silueta no reaccionó.}

{--Está bien, estamos en el mismo barco. Literalmente, al parecer. Pero
 no hablo del remolque. Me refiero a que estamos los dos en las brasas,
 eh, metafóricamente esta vez. De hecho el agua está helada, ¿no? ¿O soy
sólo yo?}

{La silueta le ignoró. El agua estaba casi por su cintura en ese
momento.}

{--Así que, de cualquier manera, probablemente puedo abrir esa puerta.
¿Quién eres? Yo soy el Doctor.}

{La figura se detuvo.}

{--¿Doctor?}

{--Sí, bueno, no literalmente un doctor, pero yo\ldots{}}

{--¿Puedes\ldots{} puedes ayudarme? --era la voz de un hombre, seca y
con un acento del sur de Londres-- ¿Puedes?}

{El Doctor vadeó a través del agua hacia él.}

{--Está bien. Enséñame dónde estás herido.}

{El hombre se giró para encararle. El Doctor se detuvo.}

{Era un joven negro, alto y fornido. La manga de su americana harapienta
 estaba cosida con una única línea de puesto naval. Su cara estaba en las
 tinieblas, pero gruesos reguerones de sangre le corrían por las mejillas
 y el cuello. Los coágulos chorreaban hacia abajo para mancharle de
carmesí el blanco cuello de la camisa.}

{--¿De dónde has venido? --murmuró el Doctor-- ¿Qué te han hecho?}

{--Tengo que salir de aquí --dijo el joven, comenzando a temblar de
 nuevo--. No puedo evitar que vengan. Eres un doctor, haz que no vengan.
Haz que no vengan.}

{--Tendremos que averiguar cómo --le aseguró el Doctor--. Pero ahora
 mismo tenemos que salir de aquí --le tendió la mano--. Vamos, apártate
 de la puerta, déjame ver\ldots{}}

{--¿Está bien mi hermana pequeña? Y mi madre --el joven comenzó a
 apartarse, aun temblando, y el Doctor vio su cara por primera vez. Había
 tres marcas profundas en la piel de cada mejilla, retorciéndose y
arrugándose como bocas de bebés, escupiendo y absorbiendo aire.}

{Y sus ojos eran enormes, blancos y salidos de las órbitas, de un tono
blanco grisáceo suave como unas perlas enormes.}

{El Doctor observó al chico con tristeza, con la mano aun tendida hacia
él.}

{--Vamos --dijo con calma--. Está bien.}

{--Mamá, mi pequeña Keisha --el marinero respiraba con dificultad y una
 gota de sangre le salió de uno de los ojos perlados--. No quiero
 herirles, pero no puedo\ldots{} no puedo evitarlo\ldots{}}

{--Oh, cielos --dijo el Doctor cuando su mente hizo click--. Eres el
hermano de Keisha. Jay, ¿verdad?}

{--¿Keish\ldots{}?}

{--Soy amigo de Rose. La conoces, ¿verdad? ¿Rose Tyler? Sí, por supuesto
 que sí --mientras Jay tropezaba hacia adelante, asfixiándose, una furia
 tan fría como el agua arremolinándose se extendió por dentro del
Doctor--¿Quién te ha hecho esto?}

{Espera un momento, ¿arremolinándose?}

{--Ya vienen --silbó Jay, abrazándose a sí mismo--. Nunca podré evitar
que vengan. No me puedo alejar.}

{--Podemos alejarnos --insistió el Doctor. El agua se espesaba como una
 pesada y salada sopa mientras se abría camino hasta la puerta y
 presionaba la punta del destornillador sónico contra las bisagras. El
 agua la abrió con la presión y la cruzó para engullir los primeros
 escalones de una escalera de metal, una forma de que los encargados del
carguero fueran desde éste hasta el montacargas.}

{--No hay mucho trozo que recorrer --el Doctor subió las escaleras,
saliendo del agua--. Vamos, Jay. ¡Ahora!}

{Pero Jay se convulsionaba, gritando locamente al agua arremolinada. El
 Doctor estuvo a punto de volver atrás a por él cuando una figura pirata
 emergió del agua que le llegaba por la cintura. Era un hombre, con los
 ojos perlados, vestido en un material que alguna vez estuvo bordado
 bellamente. Su pelo era negro, pero su cara era del color blanco de la
 tiza y parecía estar hinchada. Parecía como si perteneciera a un tiempo
 pasado mucho tiempo atrás. Entonces una figura emergió en silencio del
 frío líquido: un delgado hombre pálido con una barba muy corta. Sus ojos
 salidos brillaban como las cruces de hierro que colgaban de su oscuro
 abrigo, y su gorro picudo llevaba la insignia del águila de la
Kriegsmarine del Tercer Reich.}

{¿De dónde habían salido? El Doctor comenzó a bajar los escalones.}

{--¡No voy a hacerlo! --gritó Jay, pero un segundo más tarde el pirata y
 el capitán de la marina alemana le habían arrastrado por debajo del
agua.}

{El Doctor chapoteó ruidosamente detrás de él, pero atisbó un movimiento
 oscuro en la esquina del túnel de acceso. Era como si las tres siluetas
 hubieran sido arrastradas por alguna corriente repentina e imposible,
aunque el agua se había quedado quieta y en calma.}

{No había nada que pudiera hacer por Jay.}

{--Volveré --le prometió el Doctor, entonces frunció el ceño--. ¿Alguien
 ha dicho eso antes? --se aclaró la garganta--. ¡Yo volveré! No, esto no
 son las Filipinas\ldots{} Mejor que\ldots{} ¡Sólo me voy un tiempo, pero
 volveré! Oh, cielos, no\ldots{}}

{Se giró y subió los escalones corriendo. Los soldados seguramente le
 estarían esperando. Al menos no les habían ordenado que estuvieran en el
 remolque. ¿Por qué? ¿Porque Jay estaba allí y se suponía que los
 soldados no debían saber de él? Bueno, ahora que el pobre chico había
 sido llevado por aquellas criaturas, no había moros en la costa, o
 bueno, en el remolque. Salió al pasillo, atento por si se cruzaba con
soldados.}

{El Doctor corrió hasta que llegó a una mampara cerrada. Con un brillo
 azul sónico la abrió y se agachó en su interior, en la cabina del
 remolque. Las ventanas estaban cubiertas por un pesado toldo, a saber
 cuántos soldados estarían allí fuera a los lados del río con sus armas
 apuntándole. Pero al menos el toldo le tapaba de su vista, y quizá
 pudiera encontrar algo que usar allí\ldots{}}

{Miró los controles del barco. Parecía que una vez fue un remolcador
 normal, conectado a un carguero especialmente diseñado a través de un
 acoplamiento en popa, formando un solo barco. Pero alguno de los
 controles habían sido personalizados: una pantalla táctil mostraba un
 esquema verde brillante de la base del remolque, que parecía estar
encajada en la escotilla de carga del carguero.}

{Y ahora que se ponía a escuchar, pudo oír el repiqueteo de unas suelas
en el metal. Alguien estaba a bordo. Y se acercaba.}

\mbox{}

{El pequeño quiosco era una brillante ventana en la oscura parada.
 Cuando Rose se metió en su interior, se vio rodeada de bolsas de
 patatas, revistas, tabletas de chocolate y rollos de papel higiénico. No
 había muchos diarios y ninguno de los que quedaban anunciaba nada
 impactante sobre el Ascendant. Fue a lo seguro y cogió una revista pija
y el Star.}

{Una mujer asiática estaba sentada en un taburete detrás de la caja
registradora, con una sonrisa cansada en su cara.}

{--¿Eso es todo, cielo?}

{--Eso creo --Rose pasó cerca de dos neveras mugrientas llenas de pizza,
falafel y helado--. He oído que hoy ha tenido ambulancias por aquí.}

{La mujer sonrió.}

{--De hecho las hemos tenido todo el día. Ha habido una gran venta de
botellas de agua.}

{--¿A qué se refiere?}

{--Los médicos decían que era deshidratación avanzada. ``Una peligrosa
 falta de agua en el cuerpo''. Eso es lo que les hizo desmayarse. Y fue
 algo muy raro porque era como si no se conocieran de nada pero todos
 estaban igual\ldots{} --la mujer frunció el ceño-- Nunca había visto eso
 pasar en ningún lugar, ni siquiera en la Casualty. De cualquier manera,
 debió asustar a la gente que lo vio, porque al rato se vendieron todas
las botellas de agua.}

{Rose se mordió el labio. Le recorrió un escalofrío cuando recordó el
 fantasma de Jay, con el agua saliéndole por la boca, y la forma en la
que los soldados se habían desmayado.}

{--¿Cuándo los clientes se cayeron, tenían convulsiones?}

{La mujer asintió.}

{--Y algunos de ellos estaban mareados y todo eso. Muy triste. Pero los
 médicos dijeron que son los síntomas normales. Pero a largo plazo, no en
dos segundos. Todos estaban perfectamente después, ya ves.}

{--Sí --dijo Rose--. Creo que sí --había algo conectado, tenía que
 haberlo. ¿Pero por qué el agua? ¿Por qué el barco se había hundido y
había ahogado a todos los de a bordo?}

{--Así que vas a comprar esas revistas, ¿verdad? --le irrumpió la
mujer.}

{--Oh, sí, lo siento --Rose rebuscó en su bolsillo y le pasó un par de
pavos.}

{La mujer cogió las monedas. Y entonces se cayó de espaldas del
 taburete, llevándose por delante una estantería y provocando una
avalancha de paquetes de cigarrillos.}

{--¡Dios! ¿Está bien? --Rose se agachó a su lado, buscando a alguien
para ayudar.}

{Pero sólo estaba el fantasma de Jay, de pie entre las empanadillas y
las tarjetas de cumpleaños con orejas de perro.}

{--Ayúdame, Rose. Ven a mí --le rogó. Un torrente de agua le salió de su
 nariz y él le dedicó una pequeña sonrisa llena de esperanza--. Por
favor. Antes del festín.}
