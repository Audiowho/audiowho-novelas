\chapter*{Capítulo Cinco}
\addcontentsline{toc}{chapter}{Capítulo Cinco}

Aldridge estaba ligeramente sorprendido.

--¿El Doctor ha vencido a toda una tripulación de piratas con una sola mano, si me perdonan la expresión?

Susan golpeaba con la uña en el banco de trabajo de Aldridge, algo que se parecía mucho a una TARDIS en miniatura.

--Sí, mi abuelo se hizo cargo de ellos. Codificó el haz antigravedad de la nave con su propio ADN para que la explosión del rayo no le afectara ni a él ni a nosotros. Es un genio.

Aldridge movió la pequeña TARDIS lejos de los dedos de Susan. 

--Hay un octo-tiburón gigante aquí dentro y no creo le guste que toques su caja.

--Un octo-tiburón, ¿en serio?

--Tal vez. Por favor, deja de tocar las cosas.

Susan estaba contando a Aldridge su aventura mientras esperaban a que el Doctor despertara después de su operación.

--Así que devolvimos a los niños a su casa y pusimos al soldado de guardia en la puerta. Con un poco de suerte, pensarán que todo el episodio fue un sueño.

--La maldición se ha roto --dijo Aldridge--. No sé por qué esa familia no se mudó. No es que haya precisamente una escasez de viviendas en Londres, sobre todo para la gente rica.

Susan comenzó a ponerse los anillos de una bandeja en cada dedo, consiguiendo tener al final treinta anillos en las manos. 

--Dígame, Sr. Aldridge. ¿Cómo hace ese truco con la barba?

Aldridge se erizó, como solía hacer cuando escuchaba esos comentarios.

--Es una disciplina. Todo lo que se tiene que hacer es practicar y beber un vaso muy diluido de veneno cada noche. Ahora haz el favor de poner los anillos de nuevo en la bandeja. Esto es un negocio, sabes, no una juguetería.

Los gemidos llegaron desde la habitación de atrás, seguidos por un largo episodio de tos.

--¿Dónde está? --dijo la voz del Doctor --. ¡Susan!

Susan se quitó rápidamente los anillos y los arrojó en la bandeja.

--Es el abuelo. Está despierto.

Corrió detrás de la pantalla para encontrar al Doctor ya sentado en el catre de un soldado, rodeado de una gran variedad de equipos altamente sofisticados, que habían sido disfrazados de objetos victorianos cotidianos.

Una vez, alguien trató de usar lo que él pensaba que era un inodoro, Aldridge se lo había contado a Susan en un intento de que dejara de tocar cosas. Y tenía los dos lados de su parte inferior cosidos juntos.

--Aquí estoy, abuelo --dijo Susan--. Todo está bien.

El pánico del Doctor desapareció como barrido por una rafaga de viento.

--Bien, querida. Bien. Tuve sueños por la anestesia. Pesadillas. Ahora me despierto para encontrarte a mi lado y apenas puedo recordarlas.

Aldridge apareció alrededor de la pantalla. 

--Tanta poesía, tanta efusividad. Es suficiente para hacer que un viejo cirujano derrame una lágrima.

El Doctor frunció el ceño. 

--Supongo que el trasplante fue un éxito, ¿Aldridge?

--Esa mano durará más tiempo que usted, siempre y cuando no deje que se la corte algún pirata, --dijo Aldridge.

El Doctor levantó la mano izquierda, examinándola de cerca. La única señal de cirugía era una línea gruesa de color rosa alrededor de la muñeca.

--Fue una situación delicada en algunos momentos --dijo Aldridge --. Casi se regenera dos veces.

--Hmmm --masculló el Doctor--, Hmmmmmm.

Aldridge le dio un codazo a Susan. 

--Hace eso de hmmmm cuando está buscando defectos, pero no puede encontrar ninguno.

El Doctor se sentó y luego se puso de pie, tendiendole la mano a Susan para su inspección.

--Dime, nieta. ¿Qué piensas?

Susan apretó la palma y le estiró los dedos uno por uno.

--Honestamente, abuelo --dijo --. Me parece un poco grande.