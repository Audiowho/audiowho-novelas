\chapter*{Capítulo Dos}
\addcontentsline{toc}{chapter}{Capítulo Dos}

The Strand estaba llena de una multitud de vendedores ambulantes, y niños salvajes que llegaban diariamente desde las colonias de Londres para seguir a caballeros adinerados como las limaduras de hierro siguen a un imán, y juerguistas de mejillas coloradas invadiendo la calle en el exterior del infame pub El Perro y El Pato. Si alguien se hubiera fijado en el anciano cascarrabias caminando hacia Charing Cross, no habrían notado nada extraño en este caballero, aparte del hecho de que estaba mirando a su mano izquierda, con cierta sorpresa, como si le estuviera hablando.
 
Un hombre retirado del ejército podría haber adivinado, señalando su abrigo y su andar mesurado.
 
Un viajero del mundo quizá, podría haber conjeturado, debido a su sombrero ruso.
 
O un excéntrico científico, deducible por los rayos de pelo blanco rebosando a su paso, por no mencionar el mango de marfil de una lupa asomando de su bolsillo.
 
Nadie podría haber sabido que había un Señor del Tiempo entre ellos esa noche. Nadie, excepto su nieta, Susan, que era posiblemente la única persona en el universo que podía hacer sonreír al Doctor sólo por el mero hecho de pensar en ella.
 
Había muchas cosas que no hacían sonreír al Doctor: la cháchara, respondiendo preguntas en momentos de emergencia, respondiendo preguntas en momentos de calma total, los caracteres del subjuntivo Gallifreyan (embaucadores muchos de ellos), el alimento de la Tierra conocido como Marmite, el programa de televisión humano ``Los Siete de Blake'', que era evidentemente absurdo, y el húmedo, mordaz apretujón de una multitud del Londres victoriano. Los londinenses soportaban una receta de aromas compuesta por dos partes de alcantarillado crudo, una parte de humo de carbón y una parte de olor a cuerpo sin lavar. El gran hedor no tenía amo y lo olían desde la reina hasta la lavandera. Este hedor podía exacerbarse por el calor del verano o por los vientos predominantes y el Doctor pensaba que no había olor que despreciase más en todo el universo.
 
Para cuando llegó a Charing Cross, el Doctor no podía soportar más el hedor por lo que llamó a un carruaje. Rechazó la oferta del cochero de la mitad de un sándwich, se apretó en la cara una máscara de filtro de aire oculta en un pañuelo, y se encorvó agachado en el banco para desanimar al cochero de hacer más preguntas. El Doctor hizo caso omiso del viaje, incluyendo el rodeo alrededor de Piccadilly, donde un camión de leche había volcado, derramando su carga a través de la avenida, y se puso a pensar en el problema que le había costado muchas noches de sueño y, más recientemente, su mano izquierda.
 
Los Piratas del Alma eran unas abominables criaturas: una escoria de las especies humanoides del universo con sólo dos cosas en común. Una, ya mencionada, eran aproximadamente de apariencia humana, y, dos, no les importaba un ápice las vidas de otros. Los Piratas del Alma tenían un modus operandi muy concreto: elegían un planeta en donde los habitantes todavía no tenían capacidad hiperespacial, entonces rondaban por encima de las nubes y enviaban un jinete, montando un haz tractor antigravedad cargado con un agente soporífero a las habitaciones de los niños dormidos. El haz antigravedad era inteligente, pero el agente soporífero en el haz era genial, ya que, incluso si las víctimas se despertaban, el sedante hacía que sus cerebros confeccionasen algún fantástico cuento de hadas y así podrían hacerles desaparecer gustosos. Se creían capaces de volar, o veían al jinete del haz como un aventurero glamuroso que necesitaba desesperadamente su ayuda. En cualquier caso, no había lucha ni alboroto, y, lo más importante, la mercancía no sufría daños. Cuando los niños secuestrados eran arrastrados a la nave de los piratas, o eran enviados a la sala de máquinas y enganchados a los cascos drena-cerebros, o troceados por órganos y partes del cuerpo, que los piratas se trasplantaban en o dentro de sí mismos. Nada se desperdiciaba, ni una uña del pie, ni un electrón, de ahí el apodo de los bandidos: los Piratas del Alma.
 
El Doctor había cazado implacablemente a los piratas a través del tiempo y el espacio. Se habían convertido en su misión, su obsesión. Según su red galáctica, la tripulación que se había apoderado de su mano eran los únicos que aún operaban en la Tierra. Tuvo su último enredo con ellos en esta misma ciudad y ahora la TARDIS había detectado su firma anti-gravedad aquí de nuevo. Para los piratas habían pasado veinte años desde que su capitán cortó la mano izquierda al Doctor, pero para el Señor del Tiempo, habiendo saltado años hacia delante en la TARDIS, era de hecho una herida muy reciente.
 
Esto era lo que Susan llamaría una oportunidad. Las naves de los Piratas del Alma a menudo eludían a las autoridades durante siglos porque tenían escudos impenetrables, haciendo difícil localizarlos.
 
--Deben haber perdido una de sus placas de protección --conjeturó el Doctor--. Y eso había hecho visibles a los piratas durante unos pocos minutos, antes de que efectuasen las reparaciones. Tiempo más que suficiente para que la TARDIS los encontrase. Bien hecho, vieja amiga.
 
Desgraciadamente, cualquier agujero que había permitido tener una filtración de la firma de la nave pirata ya se había tapado y el Doctor no podía saber si los piratas seguían rondando por encima de Hyde Park, escondidos en los bancos de nubes, o fuera en su próximo puerto de escala. Una tripulación típica pirata tenía más de un centenar de calles que revisitar en orden aleatorio. Pero los piratas tenían la tendencia a volver a los buenos sitios de recolección. Así que si alguien realmente quería seguirles la pista, todo lo que necesitaba era determinación y un montón de tiempo.
 
``Y tengo ambas cosas'', pensó el Doctor. ``Además de una nieta ingeniosa''.
 
A veces demasiado ingeniosa. Después de todo quizá sería conveniente controlar a Susan. A veces parecía como si ignorara deliberadamente instrucciones específicas porque, según sus propias palabras, parecía que era lo correcto.
 
Y, mientras que a menudo era lo correcto moralmente, raramente era correcto desde un punto de vista táctico.
 
Justo cuando el Doctor pensaba en llamar a Susan, ella debió haber pensado en ponerse en contacto con él puesto que su comunicador de muñeca vibró para indicar la llegada de un mensaje. Sorprendentemente, zumbó una segunda vez, y luego una tercera. Le siguieron varios zumbidos más urgentes.
 
El Doctor comprobó la pequeña pantalla para ver una docena de mensajes todos de Susan llegando al mismo tiempo. ``¿Cómo puede ser? Había diseñado y construido él mismo estos comunicadores. Podían transmitir a través del tiempo si era necesario.''
 
Entonces se dio cuenta.
 
``Estúpido. Estúpido''. ¿Cómo no había previsto esto?
 
Aldridge estaba fuera del radar en esta ciudad. Obviamente, había establecido una serie de fuentes de interferencia. Cualquiera que escanease el planeta no encontraría ningún rastro del cirujano ni sus artilugios.
 
Susan había estado intentando comunicarse toda la noche, pero él había estado dentro de la zona de interferencia.
 
El Doctor se desplazó hasta el último mensaje, que había llegado sólo unos segundos antes, y lo abrió.
 
--Abuelo --dijo la voz entrecortada de Susan, y podía oír sus pies golpeando mientras corría--, no puedo esperar más. El rayo ha alcanzado el número catorce como predijiste. Repito, el número catorce. Tengo que ayudar a esos niños, Abuelo. No hay nadie más. Por favor, ven rápido. Date prisa, Abuelo, date prisa.
 
El Doctor se maldijo a sí mismo por ser un tonto, arrojó algunas monedas en dirección al cochero y le gritó al hombre que se diese prisa hacia los jardines de Kensington al final de Hyde Park.
 
Se suponía que ella tenía que esperar.

--Le dije que esperara. ¿Por qué tiene que ser tan temeraria?
 
Mientras se acercaban a la hilera de casas adosadas en el extremo del parque, el Doctor oyó el resto de mensajes de Susan esperando alguna información que le pudiese ayudar a rescatarla a ella y a los niños.
 
Por lo que pudo entender, Susan había hecho amistad con tres niños en el parque y había averiguado que sus padres se habían ido a Suiza a un nuevo y revolucionario spa debido a problemas nerviosos del padre. Temerosos de la maldición, habían dejado al Capitán Douglas, un soldado de la guardia personal de la Reina, a cargo de los niños para protegerlos.
 
La maldición. La familia, como muchas otros, creía que los niños desaparecían a causa de una maldición.
 
El Doctor pudo ver entrar en la casa la firma leonada anaranjada clara del haz de los piratas. Saltó del carruaje, y entonces corrió por un sendero iluminado por el resplandor como de luciérnaga de lámparas de gas, y subió los escalones del número catorce. La puerta era típicamente victoriana: sólida e inquebrantable para derribarla con el hombro.
 
``¿Qué hay de mi mano bio-hibrida?'' pensó el Doctor, decidiendo poner a prueba la tecnología de Aldridge.
 
Sin apenas una pausa, dio un puñetazo a la puerta con la mano izquierda, golpeando con el nudillo medio el ojo de la cerradura de bronce con aros, y a pesar de las circunstancias sintió un momento de satisfacción cuando la cerradura de metal se arrugó bajo el golpe y el panel de madera que lo rodeaba literalmente explotó en pedazos. Uno de sus falsos dedos del guante se dividió como una salchicha cocida, pero el Doctor sabía que Aldridge lo entendería. Después de todo la vida de Susan estaba en juego.
 
Irrumpió en el vestíbulo y subió las escaleras, sin mirar ni a derecha ni a izquierda. Los piratas entrarían por la planta superior, directamente en el dormitorio. El Doctor sabía qué dormitorio porque un resplandor emanaba de debajo de la puerta, y oyó un zumbido sordo como una lejana, colmena agitada.
 
El haz anti-gravedad.
 
``Llego demasiado tarde. Susan, querida''.
 
Con un grito que era casi animal, el Doctor abrió el falso pulgar rompiendo la puerta del dormitorio, y lo que vio allí casi paró sus dos corazones de Señor del Tiempo.
 
Era el tipo de dormitorio que uno esperaría en una casa adosada normal de clase alta de Kensington: papel tapiz estampado de terciopelo, grabados enmarcados en la pared - y un haz naranja retirándose a través de las ventanas como una serpiente asustada. Quizá en el mundo del Doctor, el haz naranja era menos que normal.
 
Susan estaba suspendida en el aire, flotando hacia la ventana, con una sonrisa soñadora en su encantadora y joven cara.
 
--Abuelo --lo llamó. Sus movimientos eran lentos, como si estuviera bajo el agua--. He encontrado a Mamá. Voy a verla ahora, ven conmigo. Coge mi mano, Abuelo.
 
El Doctor casi cogió la mano que le ofrecía, pero habría significado entrar en el haz, al igual que la compasión de Susan por los niños había provocado que lo hiciera, y era demasiado pronto porque en el momento en que respirase el agente soporífero en el haz le afectaría también, e incluso los Señores del Tiempo sólo puede contener la respiración durante un tiempo.
 
Más allá de Susan, el Doctor vio a un grupo de figuras suspendidas por el haz.
 
Los niños y el guardia habían sido apresados. ``Debo salvarlos a todos y de alguna manera poner fin a esto esta noche''.
 
Y así el Doctor rodeó el haz, ignorando las súplicas de Susan, a pesar de que le estaban rompiendo su corazón, y subió a través de una ventana lateral al tejado donde había un Pirata del Alma con una gran espada esperando para volver a su nave en el otro extremo del haz anti-gravedad. El pirata era enorme y estaba desnudo hasta la cintura, su piel era un mosaico de injertos y cicatrices. Su enorme cabeza estaba completamente rapada, aparte de un mechón trenzado, que se erguía en su coronilla como un signo de exclamación.
 
``Mano a mano'', pensó el Doctor, con determinación. ``Y ese pirata es una mano mucho más grande que yo''.