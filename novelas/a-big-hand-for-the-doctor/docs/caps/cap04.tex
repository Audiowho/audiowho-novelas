\chapter*{Capítulo Cuatro}
\addcontentsline{toc}{chapter}{Capítulo Cuatro}

El rayo antigravedad absorbió al Doctor hasta su vientre e imaginó que esa debía ser la sensación de ser comido. De hecho, era más que una mera suposición. Había sido comido dos veces antes, en esas mismas vacaciones, por ballenas ``blarph'' en el Lago Rhonda. Las ballenas pensaban que era divertido tragar bañistas para expulsarlos luego a través de sus orificios nasales. Después, todas las ballenas salían a la superficie, ``chocaban los cinco'' y se reían a carcajadas frente a los bañistas. Estos, por lo general, se tomaban todo el asunto con buen humor. Después de todo ¿quién querría tener problemas con una ballena ``blarph'' de veinte toneladas?

El Doctor dejó de pensar en eso porque no era el momento apropiado, no cuando estaba suspendido en el rayo de antigravedad de una fragata de los Piratas de Almas.

El Doctor sabía que tenía pocos minutos de plena consciencia antes de que el agente soporífero del rayo le indujera un sueño tranquilo, cuando parecería que todos sus sueños estaban a punto de pasar. El Doctor se sacudió con fuerza para mantenerse despierto, conteniendo la respiración al mismo tiempo.

De repente estaba de vuelta en Gallifrey, con su familia, a salvo.

--Eso es --dijo su madre y le sonrió, con su largo cabello rozándole la frente--. Quédate aquí, mi pequeño Doctor. Quédate aquí conmigo y cuéntame historias de los mundos que has visitado. Me muero por escuchar tus historias.

``Es tan bonita'', pensó. ``Justo como la recordaba''.

--¡D'Arvit! --gritó el Doctor en voz alta--. Me están drogando. 

Comenzó a describir lo que estaba sucediendo a su alrededor para mantenerse alerta.

--Hay media docena de almas atrapadas en el rayo. Tres niños y tres adultos, contando a Susan como una adulta, pero no estoy tan seguro de que lo sea considerando el hecho de que desobedeció mis instrucciones deliberadamente. Todos sanos. Los piratas necesitan juventud y fuerza para manejar su nave. No puedo ver la cara de Susan, aunque puedo sentir su alegría. Me pregunto qué ve en sus sueños

El rayo era más que luz. Ofrecía resistencia cuando se tocaba y era muy fuerte para permitir la suspensión de materia densa.

--Sé que nos estamos moviendo, --continuó el Doctor, narrando su viaje --. Sin embargo, no hay sensación de movimiento. No hay ningún tipo de fricción. Puedo decir que, a pesar de las circunstancias siniestras, nunca he estado tan a gusto.

Una forma esbelta pasó revoloteando y el Doctor supo, incluso con esa visión fugaz, que era Susan. La reconoció con tanta seguridad como un bebé reconoce la voz de su madre.

--¡Susan, querida! --exclamó, soltando más aliento valioso, pero la sonrisa de Susan no varió y no respondió.

El Doctor vio en su expresión lo optimista que era Susan sobre el universo y se dio cuenta de que se derrumbaría completamente en manos de los Piratas de Almas. No podía permitirlo.

Pasaron a través de las capas de un cúmulo hinchado y emergieron hacia las estrellas. La segunda estrella a la izquierda parpadeó y crepitó de repente cuando su escudo invisible se apagó, y donde había estado el cielo ahora se cernía la enorme nave-fábrica pirata.

El haz los atrajo hacia el hangar especialmente modificado de la fragata interplanetaria de tamaño medio. La parte inferior estaba dañada por los impactos de muchos asteroides y disparos. El Doctor podía ver claramente los puntos de soldadura, donde una nueva placa había sido unida recientemente.

Las puertas espaciales se abrieron y el Doctor vio que el rayo antigravedad había sido modificado para disparar desde el interior de la nave en sí, algo increíblemente peligroso si no se calibraba correctamente, pero permitía a los Piratas de Almas llevar a sus víctimas directamente a la bodega para su procesamiento.

--Los cañones antigravedad disparan desde el interior de la bodega --dijo el Doctor, pero pudo sentir que perdía la batalla para mantenerse alerta--. Los sujetos son metidos dentro y a menudo, de forma espontánea y en perfecta sincronización, cantaban palabra por palabra la ópera monzorian ``Gruñido del Naysayer''.

``¡Basta!'' El Doctor se reprendió a sí mismo. ``Saca tu ingenio a relucir. Di lo que ves.''

--La nave de los Piratas de Almas funciona bajo el mismo principio que esas despreciables fábricas balleneras Orthonian, --dijo, sintiendo un zumbido adormecedor a lo largo de los brazos--. Una vez que los sujetos han sido depositados en el interior de la nave, son analizados por ordenador para decidir cuál será su mejor uso. La mayoría son conectados a las plataformas de la batería y se les drena su electricidad, pero otros son enviados directamente a que se les diseccione. Los Piratas de Almas son humanoides, sobre todo pero no exclusivamente del planeta Ryger. Sus sistemas son muy robustos y pueden aceptar todo tipo de trasplantes, aunque sean de diferentes especies, tales como los terrícolas. Con los trasplantes oportunos un pirata puede vivir trescientos o cuatrocientos años de la Tierra.

Las puertas gigantes se abrieron lentamente y aspiraron a los sujetos a un vasto matadero. Las filas de ganchos de carne colgaban del techo de metal y un par de piratas vestidos con delantales de goma estaban listos para rociar a los recién llegados con cañones de agua. Llevaban placas de calor curvadas unidas a los paquetes de baterías en los cinturones en caso de que el ordenador recomendara una amputación inmediata.

El rayo se apagó y su carga cayó con un golpe en un agujero de la cubierta. El Doctor confirmó que había otros cuatro, además de Susan y él mismo.

``Seis a salvar'', pensó. ``Y los piratas tienen ventaja sobre nosotros''.

Tan pronto como los últimos resquicios del rayo antigravedad se desvanecieron, los Piratas de Almas abrieron las mangueras y las usaron en sus últimas víctimas, lanzando a Susan, al Doctor y a los otros cuatro a una mezcla de extremidades y torsos en una esquina de la fosa.

Los piratas se rieron. 

--Que estúpidos, --dijo uno--. Voy a rociarles de nuevo.

Golpeado con agua por dos lados, el Doctor apenas podía respirar. Estaba ciego y no podría haber luchado aunque hubiese querido. Pero no quería. Cuando tu enemigo cree que estás inconsciente, hay que hacer que sigan creyéndolo hasta que adquieras una ventaja táctica.

O dicho claramente: hazte el muerto hasta que se acerquen.

El segundo pirata dejó caer la manguera y comprobó una consola de ordenador con grandes botones de colores.

--La nave dice pitido, Gomb, --dijo perplejo--. ¿Qué significa pitido?

¿Qué significa pitido? Era evidente que los piratas mantenían a los miembros menos inteligentes de la tripulación en las cubiertas inferiores. En el caso de Igby, probablemente en las cubiertas del fondo.

Gomb colgó la boquilla de la manguera en un gancho especial del cinturón y se apresuró a comprobar la pantalla.

--¡Pitido especial! --exclamó--. Tenemos Señores del Tiempo. El ordenador dice Señores del Tiempo. Cerebros por valor de mucho dinero. Cerebros muy grandes.

Incluso enterrado bajo una montaña de cadáveres en el matadero, el Doctor encontró un momento para la ofensiva.

``Gran cerebro, en efecto''.

Gomb miró el montón de cuerpos dormidos. 

--¿Cuál?

--Túmbalos ahi --ordenó su compañero--. Yo se lo diré personalmente al capitán y tal vez me dé una botella de grog para nosotros dos. Tu busca a los Señores del Tiempo.

El Doctor trató de desenredar sus miembros para tener así alguna oportunidad en una lucha física, pero estaba atascado por la parte inferior de un cuerpo. Su rostro estaba a un metro del de Susan. Tenía los ojos abiertos, y pudo ver cómo recobraba la consciencia.

``Está asustada'', pensó. ``No puedo permitir que muera aquí''.

Pero Susan no estaba muerta todavía ni tampoco el Doctor.

--Abuelo --le susurró--. ¿Qué podemos hacer?

- Shhhh --dijo el Doctor deseando poder darle un poco de ánimo, pero lo peor estaba por venir antes de que las cosas pudieran mejorar, lo que era probable que no pasara--. Sueña un poco

El Pirata Gomb saltó al foso, sus botas golpeando la cubierta con un sonido metálico. Se paseó por las puertas espaciales cerradas donde valiosos cerebros de Señores del Tiempo estaban esperando a ser estudiados. Gomb cantó en un tono sorprendentemente puro mientras caminaba, lo que era casi tan inesperado como escuchar una conferencia de física cuántica de la boca de un lemming.

--Grog, grog,

Tragaló,

Ella se cura del estreñimiento

Ella ya no frunce el ceño.
El Doctor pensó que tal vez Gomb había compuesto este clásico él mismo.

``¿Fruncir?''

Gomb alcanzó la pila de cuerpos y tiró de dos niños dormidos, poniéndolos uno al lado del otro y estirando sus ropas.

--Váis a ver al Capitán --dijo--. Portaos bien y tal vez sólo os drene el alma en lugar de cortaros en pedazos.

El pirata regresó a la pila y se inclinó hacia Susan.

Eso fue lo más lejos que llegó porque el Doctor se estiró hacia él y tiró del interruptor de liberación de la manguera del cinturón del Gomb. No era precisamente el plan que al Doctor le hubiera gustado realizar, pero si había calculado correctamente la presión de la manguera e intuyendo que el cinturón del pirata no se rompería, el resultado debería ser ventajoso para los prisioneros.

Ventajoso era una manera de decirlo. Gomb apenas tuvo un momento para registrar lo que estaba ocurriendo, cuando la manguera se agitó por la presión del agua que la recorría para luego lanzarle por los aires, hasta enviarle por el pasillo, fuera de la vista.

El Doctor sabía que tenía meros segundos antes de que todo el mundo en la nave se enterase de su intento de fuga. Probablemente estuvieran bajo videovigilancia en estos momentos.

Se arrastró desde debajo de los humanos que dormían y se volvió hacia Susan.

--Querida --dijo, secándose los ojos-- ¿estás herida?

--No --respondió ella, pero estaba aterrorizada. El Doctor pudo ver cómo empezaba a darse cuenta de lo que estaba pasando mientras miraba absorta los ganchos de carne que se sacudían en el techo.

--Susan, escúchame --dijo el Doctor, tomando su cara entre las manos, bueno, una mano y una garra--, Voy a sacarnos de aquí, pero necesito que me ayudes. ¿Lo entiendes?

Susan asintió. 

--Por supuesto, abuelo. Puedo ayudar.

--Esa es mi chica. Lleva a los otros al centro de las puertas espaciales. Dentro del círculo.

--Dentro del círculo.

--Tan rápido como puedas, Susan. Tenemos poco tiempo antes de que lleguen los refuerzos.

Susan comenzó a poner a los otros cautivos en el interior del círculo. Se deslizaron por la cubierta resbaladiza con bastante facilidad, incluso el adulto, que estaba vestido con un uniforme de soldado.

El abrigo empapado del Doctor le hacía sentir como si llevara un oso, así que se lo quitó y se apresuró a subir los escalones hasta la consola. Los controles estaban en rygerio, que el Doctor podía entender bastante bien, pero los cambió al inglés y bloqueó las preferencias, lo que podría darles un segundo o dos cuando más lo necesitaran.

El Doctor siempre había mecanografiado con un dedo y el pulgar por lo que trabajar con una garra no le retrasó demasiado. Buscó más cautivos en la nave pero no encontró a nadie más que a su propio grupo. Los secuestrados del día anterior ya habían sido eliminados, lo que hizo que el Doctor se sintiera mucho mejor acerca de las medidas que había decidido tomar.

Eludió códigos básicos de seguridad de la nave pirata y rápidamente reestableció los parámetros del haz antigravedad y los controles de la puerta. Una vez que el ordenador había aceptado sus imperiosos comandos, el Doctor estableció una contraseña tan complicada que se tardarían cerca de diez años o un milagro en conseguir que el ordenador realizara cualquier tarea más complicada que jugar al solitario.

Los piratas no tenían diez años, y el universo, sin duda no les debía un milagro.

Susan había conseguido reunir a los presos en el círculo del centro de las puertas de la bodega. El soldado estaba tratando de ponerse de pie y el niño más pequeño, un chiquillo, estaba vomitando sobre sus propios zapatos. El Doctor le cogió en brazos, haciendo caso omiso de los gritos de protesta.

--Rápido --dijo--. Ahora, todos juntos. Debeis poner las manos sobre mí.

Podría muy bien haber estado hablando con monos. Esos humanos estaban en mitad de la transición del paraíso al infierno. Si tenían suerte, era posible que sus mentes se curaran, pero por el momento lo único que podían hacer era respirar.

Sólo Susan estaba capacitada. Abrazó al Doctor con un brazo, al soldado con el otro y reunió a un chico y una chica, que podrían ser gemelos, entre sus rodillas.

-- Buena chica --dijo el Doctor, izando al niño enfermo sobre sus hombros--. Esa es mi chica. 

Todos estaban conectados ahora: un circuito.

--¡Pase lo que pase, no rompamos el circuito!

Susan asintió, abrazando a su abuelo con fuerza. 

--No lo haré.

--Sé que no lo harás --dijo el Doctor.

Los segundos pasaron y al Doctor le entró la preocupación de haber puesto demasiado tiempo en el temporizador. Los piratas estarían sobre ellos en cualquier momento. De hecho, el alboroto que se acercaba haciendo eco por el corredor sugería que ese momento había llegado.

Una docena o más de los piratas cayeron unos sobre otros para acceder a la bodega de carga, apuntando con sus armas al Doctor y a sus compañeros de cautiverio. Pero no dispararon. ¿Por qué hacerlo? Estos prisioneros representaban una noche de trabajo. Al parecer, habían logrado sorprender a Gomb, pero hasta una caja con un resorte podría sorprender a Gomb. Era tan estúpido. ¿Y qué podrían hacer los presos ahora, superados en número, rodeados y sin armas? No había nada que pudieran hacer salvo aceptar su destino.

El Capitán se abrió paso a la parte delantera del grupo. Era un espécimen temible. Tres metros de altura, con una cara plana de color gris, ojos profundamente brillantes y una larga cicatriz vertical en la cara.

--El Señor del Tiempo --bramó. Y sonó como si alguien hubiera enseñado a hablar a un rinoceronte --. ¿Dónde está el Señor del Tiempo?

--Estoy aquí --dijo el Doctor, verificando a través del tacto y de la vista que el grupo de terrícolas aún estaba conectado.

La risa del Capitán era inusualmente aguda para una persona tan grande.

--Eres tu, Doctor --dijo, tocándose la cicatriz de la cara--. No deberías haber vuelto.

El Doctor se dio cuenta de que el Capitán llevaba una mano encogida en una cuerda alrededor de su cuello.

``Esa es mi mano, ¡desalmado!''

--Tenía un asunto pendiente --dijo el Doctor, contando mentalmente hacia atrás desde cinco.

--Los dos tenemos asuntos pendientes --dijo el Capitán.

En general, el Doctor no estaba a favor de las réplicas, pero este Capitán era un espécimen realmente vil, así que se propuso tener la última palabra.

--Nuestro asunto ha terminado --dijo el Doctor y las puertas se abrieron debajo de ellos, dejando caer al Doctor y a su grupo a la oscuridad de la noche, a tres mil metros por encima de las brillantes luces londinenses.

El Capitán estaba decepcionado por no poder disfrutar personalmente de la sustracción de órganos del Doctor, pero el hecho de que el Señor del Tiempo estuviera muerto en cuestión de segundos lo animó un poco. Sin embargo, había un pequeño detalle que le molestaba: si el Doctor  había dejado las puertas abiertas, ¿qué otros ajustes habría estado manipulando en el ordenador?

Corrió a la pantalla más cercana y fue recibido por un complicado texto desconocido moviéndose en círculos cada vez menores.

--¡Doctor! --bramó--. ¿Qué has hecho?

Para responder a su pregunta, el cañón antigravedad disparó a través de las puertas espaciales que se estaban cerrando. Sólo una ráfaga, que rozó mientras salía las puertas antes de que se cerraran del todo.

``Suerte para mí'', pensó el Capitán. No pensó  ``suerte para nosotros'', ya que era un egoísta y un tirano que vendería a toda su tripulación a una granja de cuerpos para comprarse un minuto más de vida.

``Porque si el cañón anti-gravedad se disparaba cuando las puertas estaban cerradas, sería el fin para toda la nave''.

Una vez más, parecía como si el ordenador pudiera leer su mente cuando desvió cada chispa de energía hacia el cañón y la descargó directamente a las selladas puertas espaciales.
El Doctor y su séquito se precipitaban hacia la Tierra, aunque parecía como si Londres se precipitara hacia arriba para encontrarse con ellos. No había cabida en sus vidas para el acto de pensar. La vida se había reducido a la más básica de las urgencias: la supervivencia. Y si sobrevivían a esa noche, su vida nunca sería la misma. Habrían estado en el borde, asomados al abismo, y vivido para hablar de ello. Sólo el Doctor mantenía alguna de sus facultades ya que  las experiencias cercanas a la muerte eran más o menos su especialidad.

Caían en un montón irregular, unidos en fuertes apretones y miembros enredados. De alguna manera, en medio del caos, el Doctor y Susan se encontraban cara a cara. El Doctor trató de sonreír, pero el aire se precipitó entre sus labios e infló sus mejillas.

``Ni siquiera puedo sonreír a mi preciosa nieta''.

Lo vio venir por el rabillo del ojo, una flor anaranjada en el cielo por encima de ellos.

``Física, no me falles ahora'', pensó. Después pensó: ``La física no puede fallar, pero mis cálculos podrían ser erróneos''.

La flor se expandió y se convirtió en un haz, que se dirigió hacia ellos con una precisión infalible, dejando una estela de chispas trás él.

El Doctor sujetó fuerte a todos, abrazándolos.

``Vivir o morir. En este momento se decide''.

El pulso antigravedad envolvió al pequeño grupo, y aminoró su descenso en una serie de saltos y chisporroteos bruscos. El Doctor se encontró flotando de espaldas viendo la nave pirata desde un gran banco de nubes. Ocho plantas de metal dañado.

``Se lo merecen'', se dijo a sí mismo. ``Estoy salvando vidas de niños y vengando muchas otras más''.

Pero aún así se dio la vuelta cuando el rayo antigravedad que había programado para disparar empezó a destruir la nave desde dentro, cambiando toda su estructura atómica hasta que sus moléculas se disolvieron y se convirtieron en aire.

Susan lo abrazó con fuerza y lloró en su hombro.

Sobrevivirían.

Estarían bien.