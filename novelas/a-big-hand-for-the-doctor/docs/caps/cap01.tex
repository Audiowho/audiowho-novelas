\chapter*{Capítulo Uno}
\addcontentsline{toc}{chapter}{Capítulo Uno}

\textbf{\large{The Strand, Londres, 1900}}

\mbox{}

El Doctor no estaba contento con su nueva mano bio-híbrida.

--Ridículo. Ni siquiera es una verdadera mano --se quejó a Aldridge--. Sólo hay dos dedos, lo que es bastante menos de la cuota humanoide tradicional.

Aldridge no era alguien que aguantase tonterías, incluso de un Señor del Tiempo.

--Devuélvela entonces. Nadie te obliga a tenerla.

El Doctor frunció el ceño. Conocía el estilo de trueque de Aldridge, y a estas alturas el cirujano Xing solía usar una maniobra de distracción con su cliente. 

--¿Quieres saber por qué cerré mi consulta en Gallifrey? --preguntó Aldridge.

La maniobra funcionó como se esperaba. Cada vez que acudía a Aldridge para que le ayudara, sacaba a relucir esta historia.

--¿Fue por nuestro título, tal vez? --preguntó el Doctor inocentemente.

--Exactamente --dijo Aldridge--. ¿Os llamáis a vosotros mismos Señores del Tiempo? Qué pomposo es eso ¿Alguien previamente se registró como Emperadores del Tiempo, tal vez? Una pena, lo podrías haber abreviado a Temperadores.

``Temperadores'', pensó el Doctor. ``Era casi divertido''.

Divertido porque un Señor del Tiempo conocido como ``El Diseñador de Interiores'' una vez lo sugirió en una conferencia y fue apodado ``Temperador malvado'' para el resto de sus cuánticos días.

Pero el Doctor no podía ni tan siquiera permitirse mostrar un atisbo de sonrisa nostálgica en sus labios, en primer lugar porque las sonrisas tendían a parecer el rictus de un cadáver en su alargado  rostro, y en segundo lugar porque Aldridge aprovecharía el momento para subir el precio.

--Cinco dedos, Aldridge --insistió--. Necesito una mano entera sólo para ponerme mi camisa por las mañanas. Los humanos ponen botones en los lugares más difíciles, incluso siendo conscientes de que existe el velcro. 

Consultó su reloj de bolsillo. 

--O existirá dentro de medio siglo, más o menos.

Aldridge hizo sonar uno de los bordes de la cerámica con el bisturí. 

--El exoesqueleto tiene dos dedos, eso te lo garantizo, Doctor, pero el guante tiene cinco, incluyendo el pulgar, controlados por señales del exoesqueleto. Una maldita genialidad bio-híbrida.

El Doctor estaba impresionado, pero no podía permitirse mostrarlo. 

--Hubiera preferido una bio-bio genialidad si no te importa, estoy en un apuro terrible.

--Vuelve en cinco días --dijo Aldridge--. La carne y los huesos de la mano estarán listos para entonces. Todo lo que necesito es una muestra. 

Puso un frasco de pruebas debajo de la nariz del Doctor. 

--Escupe si no te importa.

El Doctor se lo obsequió, sintiéndose un poco aliviado de que la saliva fuese todo lo que Aldridge necesitara de él. Hacía algún tiempo, después de todo el Inescrutable fiasco Doppelgänger, se había visto obligado a partir con dos litros de sangre de un raro TL-positivo con el que trabajaba hasta el plasma.

--¿Cinco días? ¿No podrías hacer el trabajo un poco más de prisa?

\mbox{}

Aldridge se encogió de hombros.

--Lo siento. Tengo un grupo de Hombres Anfibio, todos silbando por sus extensiones de cola. Me está costando una fortuna el contratar un camión de bomberos para mantenerlos hidratados.

El Doctor miró Aldridge abajo hasta que el corpulento cirujano Xing cedió.

--Muy bien. Dos días. Pero te va a costar caro.

``Ah, sí'', pensó El Doctor, preparándose para las malas noticias.

--¿Cuánto me va a costar exactamente?

Aunque el cuánto era quizás un término poco adecuado para usar con Aldridge, ya que suele tratarse de productos básicos en lugar de monedas.

\mbox{}

El cirujano se rascó las cerdas que salpicaban su barbilla como las púas de un puerco espín. Si alguno de los granujas, sinvergüenzas o montañas de estiércol londinenses victorianos entró en el Taller de Reparación y Restauración de Aldridge con malas intenciones, habrían tenido una desagradable sorpresa. Porque Aldridge podía hinchar sus mejillas y expulsar una de esas cerdas cargadas con veneno a una velocidad y una precisión comparable a la de los nómadas de la selva tropical de Borneo blandiendo sus cerbatanas. El villano se despertaba seis horas más tarde, encadenado a las rejas de la prisión de Newgate, con recuerdos muy difusos de los días anteriores. Los guardianes de la prisión habían llegado a llamar estas entregas ocasionales ``Bebés de cigüeña''.

El Doctor señaló fijamente a la barbilla de Aldridge. 

--¿Estás tratando de intimidarme, Aldridge? ¿Es una amenaza?

Aldridge se rió y su barba se rizó. 

--Oh, vamos, Doctor. Estar aquí me da el derecho a hacerlo. El trueque y tal. Nuestro pequeño juego.

El rostro del Doctor era ilegible. 

--Incluso si no hubiese perdido una de mis manos, yo no estaría sonriendo como un idiota. No me hagas reír. Yo no juego a juegos. Tengo una misión seria.

--Tú solías reírte --refutaba Aldridge--¿Recuerdas aquella cosa con las lombrices de tierra homicidas? Hilarante, ¿no?

--Esas lombrices excretan óxido nitroso --dijo El Doctor--conocido en la Tierra como gas de la risa, así que me estaba riendo en contra de mi voluntad. Normalmente no me complace la alegría. El universo es un lugar serio y además dejé a mi nieta mirando una casa.

Aldridge extendió los dedos sobre el escritorio. 

--Muy bien, y sólo te hago esta oferta debido a la maravillosa Susan. ¿Qué necesito para el alquiler de la bio-híbrida y el crecimiento de una nueva mano en mi tina de magia...? --hizo una pausa, pues incluso Aldridge sabía que lo que estaba a punto de pedir no sería asumido fácilmente por un Señor del Tiempo que no poseyese sentido del humor--Una semana de tu tiempo.

El Doctor no lo entendió al principio.

--¿Una semana de mi tiempo? --Entonces cayó--. Quieres que yo sea tu asistente.

--Solo por una semana.

--¿Siete días? ¿Me quieres como tu asistente durante siete días enteros?

--Tú me entregas tu tiempo y yo te entrego...una mano. Tengo un cliente muy importante que no deja de repetir que necesita que le preste mis servicios. Tener a un compañero tan inteligente como tú a mi lado sería de gran ayuda.

El Doctor se apretó la frente con la mano que le quedaba. 

--No es posible. Mi tiempo es muy valioso.

--Siempre puedes regenerarte --sugirió Aldridge inocentemente--. Tal vez el próximo tendrá un mejor sentido del humor, por no hablar del sentido de la moda.

El Doctor se encrespó, aunque no tan drásticamente como Aldridge lo  hiciera en alguna ocasión.

--Este traje ha sido elegido por el ordenador para que pueda mezclarme con los lugareños. La moda no tiene nada que ver con esto. De hecho, la obsesión por la moda es el tipo de distracción frívola que lleva a que la gente sea...

El Doctor no completó su sentencia y el cirujano optó por no completarla por él, aunque ambos sabían que ``asesinada'' era la palabra que faltaba. El Doctor no quiso decirlo por si el hecho de pronunciar la palabra atraía a la muerte misma, y ya había habido demasiada muerte en la vida del Doctor. Aldridge lo sabía y se compadeció.

--Muy bien, Doctor. A cambio de cuatro días de tu tiempo, haré crecer una mano para ti. No puedo ofrecerte un trato más justo que este.

El Doctor se apaciguó a regañadientes. 

--¿Cuatro días, dices? ¿Tengo tu palabra, como becario visitante de este planeta?

--Te doy mi palabra como cirujano Xing. Puedo entregarte la mano en tu TARDIS, si quieres-- . ¿Dónde estás aparcado? 

--Aquí en Hyde Park.

--¿Mantienes tu nariz fuera de la niebla? De hecho, creo que tengo un par de narices por aquí si te apetece algo menos...pronunciado.

Esto hizo que la conversación derivase hace una charla y al Doctor nunca le había molestado demasiado el participar en una pequeña charla. En cambio, odiaba el chismorreo y el parloteo.

--Cuatro días --repitió. El Doctor levantó el muñón de su muñeca izquierda sobre el que se sentaba y sin decir una palabra presionó los dedos como garras bio-híbridas en el pecho del cirujano Xing.

Aldridge consideró la acción en silencio y levantó las altas y tupidas cejas hasta que El Doctor se vio obligado a preguntar: 

--¿Podrías, por favor, implantarme la mano bio-híbrida temporal?

Aldridge cogió un bisturí sónico de su cinturón.

--Cuidado con eso --dijo El Doctor--. No hay necesidad de dejarse llevar.

Aldridge hizo girar el bisturí como un bastón. 

--Si, Señor. Cuidadoso es mi segundo nombre. En realidad, Torpe es mi segundo nombre, pero eso no anima a los clientes y hace que me parezca a uno de esos enanos que van a ser tan populares cuando las imágenes en movimiento se pongan de moda.

El Doctor no respondió, ni se movió, mientras Aldridge ya estaba trabajando en su brazo, fijando la parte híbrida temporal a su muñeca y cortando lejos del muñón de la carne quemada en búsqueda de las terminaciones nerviosas.

Increíble, pensó El Doctor. Parece que apenas está prestando atención a lo que hace y no siento nada de dolor.

Por supuesto, esa era la marca de fábrica de los cirujandos entrenados en el  Monasterio Xing: su velocidad y precisión increíbles. El Doctor había oído una vez una historia sobre cómo se despertaron los acólitos en medio de una noche oscura por el dolor causado por su dedo gordo del pie, que  había sido amputados por uno de los profesores. Luego se cronometro el tiempo que tardaban en volver a colocarse el dedo utilizando sólo un paquete de hilo dental, tres clips de lagarto y un frasco de luciérnagas.

Parece Hogwarts, pensó El Doctor, dándose cuenta de que nadie apreciaría esta referencia durante casi un siglo.

En cuestión de minutos el cirujano estaba tirando el guante de plasti-piel sensible y dando un paso atrás para admirar su obra.

--Bien, dale un meneo.

El Doctor lo hizo y descubrió, para su vergüenza, que las uñas estaban pintadas.

--¿Podría, por un casual, ser la mano de una dama?

--Sí --confesó Aldridge--. Pero ella era una gran dama. Tan varonil como tú. Odiaba reírse y tal, así que vosotros dos os deberíais llevar muy bien.

--Dos días --dijo El Doctor, señalando con un dedo cuya uña estaba pintada de color rubí.

Aldridge se esforzó tanto para contener un ataque de risa que una de sus cerdas salió disparada a la pared. 

--Lo siento, Don Señor del Tiempo, Señor. Pero es muy difícil el tomarte en serio cuando estás llevando  esmalte de uñas.

El Doctor cerró sus dedos falsos en un puño, se enderezó su sombrero de astracán y decidió adquirir un par de guantes tan pronto como fuera posible.

Aldridge le dió al Doctor su bastón.

--¿Nunca me dirás cómo perdiste la mano?

--No --dijo El Doctor--No lo haré. Si necesitas saberlo, estaba peleando con un Pirata de Almas, quien me hirió con una cuchilla caliente. Si la hoja no hubiese cauterizado la herida, creo que estarías mirando a un Doctor diferente ahora. Por supuesto, me las arreglé para compartimentar el dolor a través de pura concentración.

--Piratas de Almas --dijo Aldridge--. Nunca voy a trabajar con esos animales. Están excluidos por principios.

--Hmmmph --dijo El Doctor, tirando de su abrigo del ejército cerca de la garganta. Podría haber dicho ``paparruchas'', pero ese eslogan ya pertenecía a otra persona.