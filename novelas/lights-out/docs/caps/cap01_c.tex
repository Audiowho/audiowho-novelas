\chapter*{Capítulo Uno}
\addcontentsline{toc}{chapter}{Capítulo Uno}


El espacio es tan oscuro que mirarlo desconcierta al cerebro. Cuanto más
miras su inmensidad, a las estrellas suspendidas y las galaxias
arremolinadas, más empiezas a notar cuan imprecisas son las palabras
``oscuro'',``negro'' y ``sin fin''. Hay muchas gamas de sombras, todas
ellas aterradoras para mí.

Por eso mantengo las luces encendidas.

Sé que es una tontería. Me encuentro a muchos años de los espacios
estrechos de mi infancia en la guardería de la Colonia de Investigación
de Collabria. Pero en la oscuridad siempre existe la posibilidad de que
las cosas se acerquen: al acecho, cosas que llevan máscaras, con las
manos frías y agujas afiladas. Cosas horribles escondidas que no puedo
ver. Últimamente, incluso mis sueños están llenos de monstruos.

Por eso me gusta viajar solo. Aquí, en mi propia nave, puedo realizar
seguimientos de señales de calor para asegurarme de que soy el único a
bordo. Puedo mantener todo tan superbrillante como quiera. Puedo ver mi
bodega de carga, actualmente vacía a la espera de un nuevo envío de los
granos de café recién tostados de la Estación Intergaláctica de Café
Tostado (o ICRS, pronunciado ``ICARUS'' por los antiguos transportes)
aquí mismo desde la consola de la sala de control. Pero, no importan las
precauciones, aún estoy mirando siempre por encima de mi hombro. El
corazón me martillea en el pecho, siento mi piel fría y húmeda, tensa y
con una punzante piel de gallina. Gracias a Dios que rara vez tengo que
dejar mi nave.

Después de dejar Collabria, viví precariamente en un montón de
sitios. Incluso firmé un contrato con la Corporación de Minas
Galatron. Siempre he sido grande para mi edad, así que no me la
preguntaron al firmar el contrato, aunque no estoy seguro de que fuese
legal. Trabajé para ellos tres años enteros, tamizando la arena roja e
inhalando veneno, pero al final ahorré lo suficiente para comprar una
nave de segunda mano.

Fue hacia el final de mi periodo en las minas cuando comenzó el
miedo. Hubo un derrumbamiento y quedé atrapado profundamente bajo
tierra. Sin aire. Con un calor increíble. El terror se disparó dentro de
mí en la oscuridad.

Al principio, me preocupaba que estar a bordo de una nave se pareciera a
estar atrapado bajo tierra o en una de las cunas-jaula de la guardería,
pero en cambio se parecía mucho más a estar cerca de la libertad. La
reparé yo mismo, aprendiendo de un gran libro viejo. Ahora mi nave va lo
suficientemente bien como para que tenga un trabajo estable, el primero
que tengo, trasportando café.

Mi ruta entre la ICRS y el Planeta de las Tiendas de Café es un largo
trecho. Paso mi tiempo viendo holovideos, rociandome con niebla de
ficción, intercambiando mensajes con 78342 y 78346 de la guardería
(últimamente, 78342 se pasa horas quejándose de que su antena secundaria
finalmente llegó y, que desde entonces, los chicos le han estado
prestando más atención) y evitando el sueño. Debería haber sido
aburrido, nadando a través del mismo viejo mar de estrellas en la misma
vieja nave, pero no me sentía así en absoluto. Me siento perseguido,
como si no quisiera ir más despacio por temor a algo sin nombre y
asfixiante pisándome los talones.

Es difícil dejar atrás el pasado. Tienes que ir más rápido de lo que mi
destartalada nave puede soportar.

Jugueteando con los controles, veo el reflejo de mi cara en la
superficie de cristal pulido, mi piel gris moteada y mi lengua bífida
que prueba el aire a mi alrededor sin ni siquiera darme cuenta de que lo
estoy haciendo la mitad del tiempo. Mis ojos son veteados y de color
rojo. Parece que me vendrían bien más horas de sueño.

Lo que realmente necesitamos es más café.

La Estación Intergaláctica de Café Tostado órbita el planeta Cloris, que
tiene un clima ideal para el cultivo de los granos de café con
súper-cafeína que la hacen famosa. La gente dice que la ICRS es el lugar
con más cafeína del universo. Se supone que sólo el respirar su aire
hace despertar a los muertos recientes, crezca pelo en los globos
oculares y recargue las baterías agotadas.

Me gusta el lugar. Por extraño que parezca, en realidad me siento menos
nervioso cuando estoy en la estación, aunque podría ser el único. La
cafeína me calma. Me preocupa que me esté convirtiendo en un adicto,
pero tal vez ese es el riesgo de un trabajo como el mío.

Para cuando atraco, mi cronómetro dice que estoy llegando un poco antes
de lo previsto. Salgo y firmo con mi huella digital un montón de
formularios oficiales para un irritable Vinvocci, mientras unos robots
cargan mi bodega con bolsas de granos que brillan como el caoba y están
pegajosos de aceite. Incluso a través de la esclusa de aire, seré capaz
de olerlos durante el viaje de vuelta. Eso siempre es un plus.

El Vinvocci me dice algo y cuando me doy la vuelta para responderle, da
un paso hacia atrás. Supongo que soy un poco intimidante cuando no sabes
lo joven que soy: grande, ligeramente encorvado y demasiado tímido para
sonreír cuando debería. Al crecer de la manera en que lo hice, no sé
cómo hablar con la gente o hacer que se sientan cómodos. Firmo su tablet
de datos, vuelvo a mi nave, me como la cena de un envase y me preparo
para irme a la cama. Mientras miro al techo encima de mi cama, tengo el
agradable pensamiento de que mañana, antes de irme, voy a ir a tomarme
una taza de café, café de verdad, ardiente y elaborado en la estación
por camareros que saben lo que hacen, y no en la vieja cacerola manchada
de herrumbre de mi nave.

La boca se me hace agua sólo de pensarlo.

No me gusta dormir porque, no importa lo brillante que sea la
habitación, cuando tengo los ojos cerrados, estoy en la oscuridad.

Tampoco me gustan mis sueños. Allí tumbado, me encuentro a mí mismo
deseando poder salir a hurtadillas y tomar un café rápido, pero la
tienda estará cerrada. Quizá podría encontrar algunos granos
sueltos. No, me digo a mí mismo, obligando a mis pensamientos a
centrarse en lo impresionada que estará 78342 cuando haya ahorrado lo
suficiente para que consigamos un hogar propio en uno de los sistemas
estelares más bonitos. Y 78346 también. Vamos a vivir todos juntos y ya
no tendre que estar nervioso nunca más, porque siempre estarán mirando
por encima del hombro por mí.

Cuando me despierto, las luces están apagadas y mi corazón late con
fuerza. Durante un largo momento, creo que todavía estoy en la misma
vieja pesadilla, sumido en la oscuridad, con escalofríos en la
piel. Busco a tientas en la pantalla del ordenador de la pared. Poco a
poco, la sala se ilumina. Parpadeo rápido. Debo haber golpeado el
interruptor de la luz por la noche.

Atontado, me dirijo a trompicones hacia la electroducha, pongo la
temperatura lo mas alta que se puede y dejo que la suciedad y el sudor
de la noche se quemen fuera de mi piel.

Cuenta una leyenda que beber suficiente café de la Estación
Intergaláctica de Café Tostado hace que una persona permanezca despierta
durante una semana por lo menos. Ahora mismo me gusta como suena eso.

Me visto y camino a través de los pasillos hacia la pequeña tienda de
café. Las mantienen pequeñas, sucias y sólo abiertas durante horas
limitadas, ya que no quieren que el personal de transporte, como yo,
consuma demasiado de las cosas buenas. Está lleno de trabajadores del
interior de la estación, junto con unos pocos viajeros que sentían la
necesidad de un café lo suficientemente fuerte como para levantar un
sistema solar. Detrás de la barra, en las humeantes máquinas, hay una
camarera de piel púrpura con seis brazos. Está haciendo expresos y
espumando la leche más rápido de lo que puedo seguirla.

Mirando alrededor de la sala, veo a una pequeña Graske pedir dinero
prestado a un Terileptil con una capa. Creo que los he visto antes en la
estación. Un Blowfish con un parche en el ojo se sienta atrás en un
rincón oscuro, escaneando la tienda con una mirada siniestra en su
cara. En una mesa, un grupo de trabajadores en ropa de faena están
relajándose, riendo juntos. En otra, un soldado mira sombríamente el
fondo de su taza.

La fila es corta y estoy al final, en el mostrador hay una mujer que
lleva un uniforme militar. Entre nosotros se encuentra un hombre delgado
como un galgo vestido con un abrigo azul marino y un forro de color
escarlata. Se vuelve a mirarme con unos penetrantes y hundidos ojos
grises, surcados por unas impresionantes cejas plateadas.

---Voy a comprar un café ---dice---. Para una chica.

---Ah ---digo, preguntándome si su antena secundaria llegó. Sí, eso
parece ser---. Estupendo.

---Cree que estoy comprándole un café de la Tierra del siglo XXI
---continúa, balanceándose sobre los talones---. ¿No le
sorprenderá?. Resulta que soy de los llevan cafés después de
todo. Quiero decir, este no es tan bueno como el increible café
preparado por Elisabeth Pepsis, por supuesto, o ese extraordinario que
solía hacer Benton, ¿cual era su nombre de pila?. Ah, sí,
Sargento. Sargento Benton. Dijo que tenia que ver con la temperatura del
agua. Pero este aún está bueno. Clara está un poco molesta conmigo, pero
una vez que pruebe esto, su estado de ánimo mejorará enormemente. O
posiblemente no, pero al menos tendrá el café.

---¿Clara? ---repito. Ha dicho un montón de nombres, pero ese parece ser
el importante.

Asiente con la cabeza. 

---Ella es imposible.

---Me resultas familiar ---digo antes de pensarlo dos veces. El hombre
al que me recuerda parecía muy diferente, pero hablaba con el mismo
vertiginoso y alegre afán. Entendí menos de la mitad de lo que ese
hombre dijo, pero había salvado mi vida, así que estaba decidido a
prestar atención a cualquiera que fuese incluso un poco como él.

El ceño fruncido del hombre se profundiza. Sus cejas hacen cosas que ni
siquiera sabía que las cejas pudieran hacer. 

---Me ocurre a menudo.

---Se hacía llamar el Doctor y salvó\ldots{}

---Ahhhhh, bien ---dice, interrumpiéndome---. Te recuerdo a mi. Oh,
bueno, eso tiene mucho más sentido.

---¿Qué?.

En ese momento, un anodino ser humano vestido de gris se une a la cola
detrás de mí. Usa una mascarilla en la mitad inferior de la cara, del
tipo blanco de papel, que los científicos siempre llevaban en la
guardería. A pesar de que sólo puedo ver sus ojos, me resulta horrible e
incómodamente familiar. ¿Podría realmente ser uno de los científicos?.

---Yo soy el Doctor ---dice el hombre de las cejas, bastante orgulloso
de ello---. Apuesto a que pasamos un buen rato, ¿verdad?.

No sé cómo responder a eso porque no tiene ningún sentido. Si realmente
es el Doctor, entonces seguramente debe recordar Collabria, debe notar
el parecido de la figura enmascarada detrás de mí con los científicos de
allí. Abro la boca para preguntar, cuando los molinillos de café se
estremecen y se paran, las hojas crujen metálicamente una contra la otra
en una ausencia repentina de granos.

Palabras de asombro se escapan de la boca de la camarera. Incluso parece
sorprendida al tener que decir.

---¡No hay más café!.

Los granos han parado automáticamente de alimentar las máquinas desde
una rampa en el techo. En la Estación Intergaláctica de Café Tostado,
eso es lo peor que podría pasar.

Entonces se apagan las luces. Para mí, eso es incluso peor.

Todo el mundo a mi alrededor está gritando. Siento ese terror familiar,
tan intenso que soy incapaz de pensar más allá de él. Quiero correr,
pero me siento frío y caliente a la vez y me parece que no puedo
controlar mis pies. Justo cuando pienso que podría ser capaz de moverme,
las luces se encienden de nuevo parpadeando. Doy un grito ahogado.

El hombre de la mascarilla se ha ido, pero tirado en el suelo de metal
está el cuerpo de la mujer soldado que estaba delante del Doctor en la
fila. Todavía tiene una taza en la mano, derramando su precioso
contenido sobre las baldosas grises. Los vapores del café recién hecho
no parecen, como cuentan las leyendas, que vayan a devolverla a la vida.
Está muerta.

No es el primer cadáver que he visto, pero es el primero desde las
minas. Esperaba no volver a ver ninguno de nuevo.