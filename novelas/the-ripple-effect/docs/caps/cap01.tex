\chapter*{Capítulo Uno}
\addcontentsline{toc}{chapter}{Capítulo Uno}

El Doctor estaba boca arriba con la cabeza dentro de la consola
de la TARDIS. Ace estaba de pie junto a él, sosteniendo un puñado de
herramientas que no reconocía. No eran pesadas, pero eran complicadas.

---Pásame el desentrelazador magnético---dijo el Doctor, con el
brazo extendido, expectante.

---¿El qué? ---preguntó Ace.

---El tubo de metal con la bola roja en el extremo.

Ace tenía que hacer malabarismos con las herramientas en sus
brazos para no tirar todo antes de extraer con cuidado la correcta y
ponerla en la mano del Doctor.

---Profesor, ¿qué estás haciendo exactamente? ---preguntó con el
ceño fruncido.

---Estoy reconfigurando el tensor crono-dinámico para tener un
ángulo de fase no octogonal ---dijo como si eso lo aclarase todo---.
Ahora necesito un filtro de taquiones. Por favor.

Una vez más, surgió una mano, con los dedos retorciéndose. Ace
examinó las herramientas que todavía sostenía. Esperaba que estuviera
convenientemente etiquetada como `filtro de taquiones'. No hubo suerte.

La mano vacía se agitó con impaciencia.

---Vale, ¿cuál es? ---preguntó frustrada. Dale algo que volar
con explosivos Nitro-9 y era buena, pero para los delicados retoques de
una máquina del tiempo, era una perfecta inútil.

---El tubo brillante con el resplandor azul en su interior
---fue la respuesta amortiguada.

Los dedos del Doctor se movían ahora tan rápido que eran un
borrón. Ace encontró la herramienta y la puso en la mano expectante del
Doctor. Sus ojos se dirigieron a la pantalla. Desde hacía más de una
semana había estado mostrando lo mismo. ¡Niebla! No una niebla
auténtica, por supuesto, estaban en el espacio, sino una borrosa
nebulosa multicolor que cambiaba constantemente. De vez en cuando
durante unos segundos aparecía a la vista algún trozo de basura espacial
a la deriva antes de desaparecer de nuevo en las tinieblas. Y la TARDIS
no era la única nave varada aquí. Ocasionalmente todo tipo de naves
espaciales aparecían entre los escombros. Algunas eran pequeñas
lanzaderas, otras eran enormes cruceros estelares, pero todas ellas
compartían una cosa, incluida la TARDIS: estaban realmente atrapadas. El
Doctor ya había señalado unos cuantos navíos ahora obsoletos que habrían
estado aquí desde hace siglos, tal vez incluso milenios. Ace estaba que
se subía por las paredes. Ocho días atrapados en este lugar eran más que
suficientes para ella.

---¿Cuándo consigas sacarnos de aquí, las otras naves se
liberarán también? ---preguntó.

El Doctor asomó la cabeza y la miró.

---No ---respondió---. Me temo que no. El Plexo Temporal es como
unas arenas movedizas cósmicas. Puedes tirar de ti mismo, quizá, pero
los demás se quedarán varados a menos que encuentren la manera de
escapar por sí mismos.

---¿No podemos hacer algo para ayudarlos?

La idea de abandonar a todos los demás no le sentó bien a Ace.

---No hay garantías de que nosotros podamos salir de esto, no te
preocupes por nadie más.

La cabeza del Doctor volvió a desaparecer en la consola. La
niebla se aclaró brevemente en la pantalla, y Ace creyó distinguir una
forma familiar. Se la quedó mirando intensamente.

---Necesito un estabilizador cuántico ---dijo el Doctor sin
cabeza.

---Sí, un minuto.

Ace miró fijamente la pantalla. Como si lo hubiera deseado, la
niebla se aclaró de nuevo y lo vio. Una cabina de policía\ldots{}

---¿Todos los Señores del Tiempo tienen TARDISes? ---preguntó
mientras los zarcillos de niebla se cerraban de nuevo.

---Sí ---respondió el Doctor---. Aunque la mayoría no las usen.
Prefieren sentarse en Gallifrey pareciendo importantes. ¿Por qué?

---Acabo de ver otra.

---¿Dónde?

---Ahí fuera.

El Doctor se levantó, se sacudió las manos y se acercó a ella,
mirando la pantalla. A parte de la niebla espacial en constante cambio
no había nada más que ver.

---¿Qué te hace pensar que era una TARDIS?

---¡Duh! Porque parecía una cabina de policía.

---Mi querida Ace, no todas las TARDISes parecen cabinas de
policía. Sólo esta lo hace,desde que el circuito camaleónico de la
vieja amigase atascó.

---Pero definitivamente vi\ldots{}

---Entonces era un eco temporal. El espacio y el tiempo están
tan mezclados por aquí que lo que viste podría haber sido hace diez
minutos o tal vez una década en el futuro.

---Realmente no vamos a estar varados aquí tanto tiempo,
¿verdad?---preguntó Ace, horrorizada.

---Bueno, no, si hago algo al respecto ---dijo el Doctor con una
sonrisa confiada, antes de regresar a la consola.

Ace suspiró profundamente. Todo estaba bien para el Profesor.
Probablemente para él un año en la TARDIS pasaría como una hora para la
gente de la Tierra. La TARDIS era enorme, y había suficientes maravillas
en su interior para mantener entretenido a alguien durante toda su vida.
Había una biblioteca, ordenadores, tesoros alienígenas de un millar de
mundos y una piscina, pero Ace odiaba estar encerrada, no importaba lo
interesante que fuese la jaula. Necesitaba salir. ¡AHORA!

Ace apagó la pantalla.

---¿Qué pasará si no puedes sacarnos? ---preguntó en cuclillas.

Hubo una pausa que fue algo demasiado larga para que fuese
cómoda. El Doctor levantó la cabeza.

---No llegaremos a eso. 

Le guiñó un ojo. Ace no estaba convencida. Habían pasado tres
días desde que el Doctor había jugado con sus miserables cucharas. Eso
en sí mismo le decía que tenían un GRAN problema.

Diez minutos más tarde, se levantó y pulsó el estabilizador
cuántico en el bolsillo superior de su arrugada chaqueta beige. Puso
suavemente sus manos sobre la consola e inclinó la cabeza. Podría haber
estado sólo revisando uno de los instrumentos, pero sospechosamente a
Ace le pareció como si estuviera tranquilizando a la TARDIS. Entonces
pulsó dos interruptores, cerró los ojos y tiró de una gran palanca. El
Rotor de Tiempo, la brillante columna de cristal en el centro de la
Consola, comenzó a subir y bajar y Ace oyó el familiar zumbido,
resoplido y chirrido de la TARDIS cuando empezaba a desmaterializarse.

---¡Sí!

El Doctor golpeó el aire de alegría y su rostro se iluminó con
una gran sonrisa. Ace estaba a punto de unirse a la celebración cuando
un estremecimiento recorrió toda la TARDIS, el Rotor de Tiempo se fue
debilitando y se detuvo. El silencio era absoluto.

---¿Qué ha pasado? ---dijo Ace---. ¿Somos libres?

El Doctor corrió alrededor de la consola comprobando
instrumentos, un ceño profundo recortándose en su cara. Después de unos
pocos segundos se detuvo. 

---No ---dijo en voz baja---. Me temo que somos cualquier cosa
menos libres.

A Ace se le cayó el alma a los pies. No quería terminar como las
otras naves que había visto, aprisionadas como un mosquito en ámbar para
siempre.

---Entonces, ¿cuál es el siguiente plan astuto? ---preguntó
Ace---. ¿Más reconfiguraciones? 

Le tendió el desentrelazadormagnético, con sus cejas levantadas
esperanzada.

---No ---dijo el Doctor---. Más bien me temo que me he quedado
sin planes astutos.

El Doctor dejó de juguetear con la TARDIS y encendió la
pantalla.La miró como buscando inspiración, entonces se volvió de nuevo
hacia la consola, sus manos jugaban con los extremos de la bufanda
estampada de cachemira que le gustaba llevar.

---Vamos Profesor ---dijo Ace---. Nunca te has quedado sin
planes astutos. ¿Me estás diciendo que no tienes otro as en la manga?

El rostro del Doctor era la viva imagen de la frustración. Ace
no se lo podía creer. ¿Había el Doctor realmente llegado a una situación
de la que no podía salir? Pero entonces su rostro se despejó, sus ojos
se abrieron y brillaron esperanzados.

---Sabía que no me defraudarías ---sonrió Ace---. ¡Deslúmbrame
entonces!

---Podría tener un vago plan ---dijo en voz baja---. Pero
realmente es más desesperado que astuto.

En ese momento, Ace se hubiera apuntado a todo.

---¿Cuál es\ldots{}?

---Podría ser posible conseguir salir de aquí. El Plexo está
gravitatoriamente anclado a una estrella\ldots{} así que si puedo hacer
que esa estrella se convierta en una nova\ldots{}

---¿Qué estrella?

---La más cercana, esa que está justo\ldots{} ---hizo un gesto
vago--- allí fuera\ldots{} distante.

El Doctor se puso manos a la obra de nuevo, murmurando mientras
lo hacía.

---Eh, ¿la explosión de una estrella no es un pelín peligroso?
---preguntó Ace.

---No, si lo haces con cuidado.

``¿Con cuidado? ¿Cómo se hacía eso exactamente?'' Ace tenía
experienciacon los explosivos e incluso llevaba a mano un suministro de
bombas de Nitro-9 en su mochila, pero hacer explotar toda una estrella
estaba mucho más allá del alcance de su imaginación. El Doctor pasó unos
minutos más ajustando, retocando y comprobando. Ace se sentía
absolutamente impotente. Lo único que podía hacer era mirar.

---Doctor, háblame ---suplicó Ace.

---Voy a enviar un pulso de fase a la estrella más cercana, pero
tengo que calcularlo correctamente o haré que las cosas vayan a peor, no
a mejor---dijo el Doctor.

``¿Podrían las cosas ir a peor?''---Entonces asegúrate de
hacerlo bien ---aconsejó Ace.

---¡Gracias! No había pensado en eso ---dijo el Doctor
secamente---. Bueno, ¡allá vamos! ¡Agárrate fuerte!

Ace se agarró a la consola, preparándose lo mejor que pudo. El
Doctor pulsó un interruptor.

Una luz cegadora fue seguida por una repentina serie de
sacudidas que llegaban hasta los huesos. Ambos, el Doctor y su
compañera, fueron arrojados al suelo. La TARDIS se sacudió tan
violentamente que Ace se estrelló contra el borde de la consola,y
dolía. ¡Mucho! El zumbido en los oídos de Ace tardó unos pocos segundos
en ceder. A medida que lentamente volvía a levantarse, advirtió el
sonido de un tañido,un profundo, lento y discordante retumbo, como un
reloj de pie estropeado dando la hora.

---¿Qué es eso? ---preguntó alarmada.

---Algo que nunca quieres oír ---respondió el Doctor,
comprobando frenéticamente los instrumentos---. Es la Campana del
Claustro. ¡Es lo que la TARDIS hace en vez de gritar cuando algo malo
sucede!

---¿Como qué? Dime por favor que salimos del Plexo esta vez
---dijo Ace.

---¡Oh sí! Estamos fuera del Plexo ---dijo el Doctor ---.
Pero\ldots{}

---¿Dónde estamos?

El Doctor estudio el panel de navegación. Su boca cerrada se
movía adelante y atrás mientras masticaba su dilema.

---No tengo ni la más remota idea ---admitió finalmente.