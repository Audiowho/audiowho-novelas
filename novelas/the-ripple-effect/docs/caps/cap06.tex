\chapter*{Capítulo Seis}
\addcontentsline{toc}{chapter}{Capítulo Seis}

A la mañana siguiente, Ace se despertó con un sobresalto a causa
de que su cama temblaba violentamente. Ruidos inquietantes llenaban la
TARDIS y se tambaleaba y sacudía como si estuviera atrapada en un
tornado. Ace corrió a buscar al Doctor. Cuando llegó a la Sala de
Control, patinó hasta detenerse y miró boquiabierta.

La consola estaba casi en el techo, levantada sobre un pedestal
blanco brillante. Faltaba medio suelo, el otro medio estaba cubierto de
herramientas y Ace apenas podía ver la cabeza del Doctor mientras se
arrastraba por alrededor, haciendo afanosamente Dios
sabía qué.

---¿Haciendo algo de redecoración? ---preguntó mientras iba de
isla en isla cruzando la habitación.

---Estoy convirtiendo la TARDIS en un Vortexcopio.

---¿Un qué-copio?

---Un Vortexcopio. Es una forma de examinar el Vórtice del
Tiempo y\ldots{} ---se detuvo---. El caso es que me permitirá determinar
las coordenadas de la dislocación inicial del espacio-tiempo.

Ace podía ver que estaba emocionado. Por primera vez en mucho
tiempo, se mostraba entusiasmado de nuevo. Por supuesto no tenía ni idea
de lo que estaba hablando.

---¡Una dislocación del espacio-tiempo suena doloroso!

El Doctor sonrió.

---Piensa en el espacio y el tiempo como un lago. Sabemos que alguien ha
cambiado su forma, puestos nuevos peces, nuevas plantas, cambiado su
profundidad.

---Vale\ldots{}

---Pero tuvieron que empezar en alguna parte. Tuvo que haber un
primer cambio: el primer nuevo pez que cayó en el agua.

---¿Y?

---Eso provocaría ondas.

Ace finalmente lo entendió.

---Así que se puede saber\ldots{}

---\ldots{} donde comenzaron las ondas, lo que confirmará de una
vez por todas si fueron los Daleks los responsables de este universo
alterado. ¡Ah! ¡A veces me sorprendo incluso a mi mismo! ---terminó el
Doctor felizmente.

---¿Y cuando descubras que no fueron ellos? ---dijo Ace.

El Doctor agitó el desentrelazador magnético en su dirección.

---¡No vendamos la piel del oso antes de cazarlo!

---Sé que los Daleks no causaron esto, Profesor. Y cuando
confirmes eso, entonces podemos quedarnos\ldots{} ¿verdad?

Después de pensarlo un momento, el Doctor dijo con cautela.

---Si me equivoco, si no hay nada fundamentalmente erróneo en
este universo, entonces nos quedaremos\ldots{} por un tiempo.

Tardó casi todo el día. Ace se ofreció a ayudarle, pero cuando
el Doctor dijo que no por tercera vez, en su lugar se fue a ver la
habitación de Tulana en la Academia. Después de ver la partida de los
Señores del Tiempo, pasaron el día intercambiando experiencias y pasando
un buen rato. Tenían tanto en común que Ace supo que había hecho una
buena amiga. No tenía demasiados.

---Ace, ¿cuáles son tus planes para el futuro? ---preguntó
Tulana mientras estaban sentadas dando sorbos a sus lodos, que parecían
un espeso gel gris, pero que sabía como mango y fruta de la pasión de la
Tierra.

---Ni idea ---Ace se encogió de hombros, antes de lamer sus
labios---. Viajar y tener aventuras, supongo. ¿Y tú?

Tulana lo dijo sin vacilar.

---Quiero ser una fuerza para el bien, una voz por la paz,el
camino que los Daleks me han enseñado.

Ace sólo pudo mirarla con admiración.

---Bueno, Tulana, si alguien puede hacerlo, esa eres tú.

Compartieron una sonrisa y regresaron a sus lodos.

Cuando Ace volvió a la TARDIS, la Sala de Control aún parecía
haber sido alcanzada por una bomba. La consola había sido bajada de
nuevo, pero no del todo. El Doctor estaba de puntillas para ver qué
controles estaban operando. Dejando las puertas abiertas, se dirigió
hacia él.

---¿Y bien? ---dijo---. ¿Has acabado de construir tu detector de
ondas?

El Doctor se quedó mirando fijamente los instrumentos.
Lentamente, se volvió hacia Ace. 

---Sí.

---¿Y?

---Sé lo que causó el problema.

Ace le miró fijamente, luego le fulminó con la mirada.

---¿Y bien? No me tengas en ascuas.

---Yo lo hice.

---¿Hiciste qué?

---Yo causé todo esto ---dijo el Doctor.