\chapter*{Capítulo Siete}
\addcontentsline{toc}{chapter}{Capítulo Siete}

Los ojos de Ace se abrieron completamente por la sorpresa.

---¿Cómo?

---Tenías razón sobre haber visto una segunda TARDIS en el
Plexo, pero era más que un eco temporal. Creo que éramos nosotros, pero
en dos momentos diferentes en el tiempo existiendo simultáneamente
dentro del Plexo. Cuando creé la supernova para sacarnos, la
contra-descarga se enrolló a través del Plexo y enredó las líneas
temporales de ambas TARDISes, haciéndolas girar juntas. Y luego, cuando
toda esa energía de ambas naves se liberó, creó esta línea de tiempo
alternativa.

Ace parpadeo rápidamente mientras trataba de asimilarlo.

---Espera. ¿Entonces eso significa que no habrá más invasiones
Daleks? Eso es algo bueno,¿verdad?

En todo caso, el Doctor parecía menos feliz ahora que antes.
Lentamente negó con la cabeza.

---He cambiado todo.

---Sí, pero tú cambias las cosas todo el tiempo ---señaló
Ace---. Vas hacia atrás y adelante en el tiempo, entrometiéndote en
cosas, derrocando tiranos y saboteando invasiones alienígenas.

---Esto es diferente ---dijo el Doctor---. Esto no es un cambio
menor en algún rincón tranquilo. Esto es una reescritura total de la
historia de todo,y tiene que ser corregido.

---¿Quieres cambiar todo esto? ---preguntó Ace, horrorizada.

---Tengo que hacerlo.

---Pero,¿por qué? Este universo está bien. Puede que no sea el
que tú y yo recordamos, pero, ¿qué te da el derecho a decir que este
cambio en particular está mal?

---Porque soy un Señor del Tiempo.

---¡Oh, perdona!

---Esa es una razón más buena de lo que piensas.

---Y sin embargo a los demás Señores del Tiempo no parecía
importarles. Entonces, ¿no eres más listo que los Daleks y Tulana, eres
más listo que los demás Señores del Tiempo también?

---¿Crees que no he pensado en esto? ---expuso el Doctor---. Sé
lo que parece, pero este universo es defectuoso.

---No lo es. ¡Funciona! ¡Los Daleks son fantásticos! Todas esas
personas son felices y productivas. No puedes simplemente pulsar algún
interruptor de Señor del Tiempo y enviar todo de vuelta a como era. No
es justo.

---Ace, este universo no debería existir.

---¡Pero lo hace! Estas haciendo esto porque odias a los Daleks.
Siempre los has odiado. Crees que no se merecen prosperar en este
universo o en cualquier otro. No eres más que un arrogante Señor del
Tiempo con un complejo de Dios mezquino, castigándoles para siempre.

El Doctor se paso una mano nerviosa por su pelo.

---Esto no es arrogancia ni elitismo, lo prometo. No creo que mi
opinión sobre esta situación sea mejor porque sea mucho mayor que tu o
porque tenga un ego monstruoso, es porque soy un Señor del Tiempo. Para
ti el tiempo es como las olas en una playa en la que metes dentro un
dedo del pie. Para mi es todo un mar, todo el camino de costa a costa y
desde la superficie hasta el fondo del océano. Siento el tiempo en el
corazón mismo de mi ser de una manera que tu nunca podrás. Algunas cosas
no están hechas para ser. Algunos cambios son demasiado fundamentales;
amenazan la propia realidad. Esos Daleks filosóficos no son un problema
en sí mismos, sino que son un síntoma de un universo que ha ido
terriblemente mal, por mi culpa.

---Vale, así que las cosas solían ser diferentes. ¿Y qué? ¿Por
qué no puedes mantener este universo? En muchos sentidos, este nuevo
universo es mejor que el viejo.

---No, Ace, este universo está mal. Hay un fallo en el diseño
básico de su corazón. En este momento soy el único que puede sentirlo,
pero las grietas ya están ahí, e irán a peor. Para cuando el resto de
los Señores del Tiempo se den cuenta, ya será demasiado tarde para
reparar el daño.

---¿Quién lo dice? ¿Tu?

---Ace, sólo tienes que confiar en mí en esto.

---Pero, ¿y sí no hubiéramos estado en el Plexo? Entonces nadie
sabría que había un problema.

---¿No lo entiendes? Nuestra escapada del Plexo causó el
problema en primer lugar ---dijo el Doctor---. He creado este lío. Todo
depende de mí para solucionarlo.

---Y, ¿cómo vas a hacer eso?

---Tenemos que volver al Plexo ---dijo el Doctor.

Ace parpadeo como un búho aturdido.

---Nos las arreglamos para escapar de allí por los pelos. ¿Y
ahora quieres que volvamos?

---No tenemos otra opción.

---Siempre hay otra opción. Tú me enseñaste eso ---argumentó
Ace.

---Pero las opciones en este caso es o no hacer nada o poner las
cosas en su sitio ---dijo el Doctor---. Y créeme\ldots{}

Una extraña vibración onduló bajo los pies de Ace, seguidade
cerca por otra, y otra. Cada ondulación era progresivamente más fuerte.

---Profesor, ¿has notado eso? ---Ace frunció el ceño.

---Por supuesto que sí ---replicó el Doctor.

---¿Qué es?

El Doctor revisó la consola. Una atónita incredulidad se
extendió por todo su rostro.

---No tenía que pasar todavía ---murmuró, corriendo alrededor de
la consola para comprobar aún más lecturas.

---¿Qué no tenía qué pasar?

---Te dije que este universo era intrínsecamente inestable
---dijo el Doctor---. Es sólo que no me esperaba que la decadencia del
espacio-tiempo sucediese tan rápido.

---En cristiano, por favor ---suplicó Ace.

---Este universo ya se está haciendo pedazos ---dijo el
Doctor---. Pensé que podría tardar décadas, posiblemente incluso un
siglo o dos antes de que se pusiera tan mal, pero la tasa de decadencia
es obviamente exponencial.

Ante la mirada en blanco de Ace, explicó.

---Creciendo rápidamente más grande y más rápido a un ritmo alarmante.

Como para subrayar sus palabras, la tierra bajo ellos comenzó a
dar bandazos. Ondas de choque sacudieron a la TARDIS violentamente hacia
atrás y adelante como si estuviesen siendo lanzados a un mar tormentoso.
Y luego las ondas se calmaron en una quietud sobrecogedora.

---¿Ha acabado?

---No ---dijo el Doctor sombríamente---. No ha hecho más que
empezar.

Sin previo aviso, Pytha apareció en la puerta de la TARDIS, pero
no estaba solo. Más Daleks estaban a su lado y detrás hasta donde Ace
alcanzaba a ver.