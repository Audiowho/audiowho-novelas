\chapter*{Capítulo Seis}
\addcontentsline{toc}{chapter}{Capítulo Seis}


Hammurabi miró al prisionero con curiosidad. Afirmaba ser un doctor no
un espía, pero no pudo explicar cómo había aparecido del interior de la
extraña cabina azul que había aparecido en medio del templo. Hammurabi
temía a la magia, y este hombre era un brujo, de eso no había duda. Un
hechicero y un espía. Éste era un gusano que podría destruirlo todo.
¿Acaso sus sacerdotes no le advirtieron de tal infiltrado?. Estaba
asustado. Tenía que salvar a su ciudad y proteger a su pueblo, de modo
que tenía que deshacerse del brujo antes de que extendiera su veneno.

Hammurabi había reunido a tres jueces, que habían estado disfrutando
juntos de la comida en una de las tabernas cercanas al río, y les había
orientado acerca de cuál debía ser el castigo. De esta manera, la muerte
del hechicero funcionaría también como una ofrenda a Marduk.

El juez principal levantó la mano para entregar la sentencia.

---Arrancadle el corazón al espia ---gritó.

---¿Cuál? ---dijo el hechicero, con una sonrisa loca en su rostro
pálido.

---Permaneceréis en silencio ---chilló el juez principal.

---La verdad es que no ---dijo el hechicero---. Siempre entendí que el
gran y sabio Hammurabi era un rey justo. Un rey que estaba orgulloso de
las leyes que escribía. Un rey que siempre dejaba que un acusado se
defendiera y que no lo declarasen culpable sin antes celebrar un juicio.

---No hay nada que podáis decir que me haga cambiar de opinión ---dijo
Hammurabi, temeroso de que el hechicero usara palabras inteligentes y
magia para ofuscar su mente. No iba a escuchar. No debía escuchar---. La
ley es la ley. Y el castigo para los espías y los hechiceros es que les
arranquen el corazón.

---¿No puedo decir nada? ---gritó el hechicero---. ¿Y si os dijera que
estáis en grave peligro?. ¿Y si os dijera que vuestra bonita ciudad,
vuestro reino, vuestro imperio, todo vuestro mundo estuviera a punto de
ser atacado por un ser de tan inmenso poder que hace parecer a vuestro
ejército un montón de soldaditos de plomo, y que cuando haya acabado con
vos no quedara nada de la poderosa Babilonia excepto sus escombros y sus
cenizas?.

---Mis sacerdotes ya me han advertido ---dijo Hammurabi, agitando
despectivamente una mano.

---¿De verdad? ---El hechicero alzó las cejas---. Son más listos de lo
que parecen.

---Y está claro que vos sois la amenaza de la que me advirtieron.

---Esperad, escuchadme\ldots{}

---Basta ---dijo el juez principal---. Es hora de ejecutar la sentencia.

---No. ¡Tenéis que escucharme! ---El hechicero parecía preocupado por
primera vez. Forcejeó mientras cuatro guardias lo agarraban y lo
arrastraban sin cuidado hacia la piedra de ejecución.

---¡El Hombre de las Estrellas está llegando! ---gritó---. ¡La Gran
Bestia!. ¡Y yo soy el único que puede detenerlo!.

Desde su posición en el balcón real, Gurgurum podía ver forcejear al
hombre mientras lo obligaban a ponerse sobre la roca de ejecución. Había
escuchado sus palabras y éstas lo habían hecho sentirse incómodo.
Gurgurum era un hombre sin una pizca de miedo a la hora de la batalla,
pero ahora ya no estaba tan seguro.

La gran bestia de la leyenda estaba a punto de llegar\ldots{} eso era lo
que el prisionero había dicho. ¿Qué forma iba a tomar la Bestia?. ¿De
dónde venía?. ¿Cómo podrían defenderse de ella?.

Las leyendas contaban las historias de los dioses antiguos. Todas las
constelaciones del cielo eran dioses. El zodíaco babilónico les enseñaba
sus nombres a los niños, y sus padres los usaban para asustarlos y para
que les obedeciesen. Habían pasado cientos de años desde la última vez
que los dioses habían caminado sobre la Tierra. ¿Cómo descendería la
Bestia de los cielos: con la forma de un toro monstruoso, de un
escorpión, o de un león enorme?. ¿O sería un monstruo nuevo?.

Se rió para sí. Se estaba comportando como un niño otra vez, temeroso de
las sombras. No iba a venir ningún monstruo.

Captó un sonido y se dio la vuelta de inmediato, al mismo tiempo que
ponía todos sus sentidos alerta e intentaba distinguir algo en la
oscuridad de los aposentos del rey.

Y el corazón se le salió del pecho.

Estaba aquí.

La Gran Bestia.

La tenían encima.

Y no se parecía a nada que hubiese visto antes.

De pie era más alta que cualquier otro hombre, y se parecía a un
escarabajo y a un cangrejo o a una langosta de río. Tenía seis patas,
pero sólo se erguía sobre cuatro. Su cuerpo segmentado se curvaba hacia
arriba de forma que sus dos patas delanteras quedaban libres como si
fuesen brazos, y una de ellas acababa en una enorme y abrupta pinza.
Tenía como una cabeza con cuatro ojos negros y redondos, pero lo peor de
todo era su boca: un agujero alargado en el centro de su rostro, lleno
de filas y filas de dientecitos aserrados y afilados y rodeada de
antenas que al sacudirse partían el aire en dos.

Gurgurum rezó a Marduk y cargó, sacando su espada de la vaina de su
cinturón.

Sabía que no podía derrotar a este monstruo, pero valía la pena
intentarlo.

Era babilónico de pura cepa.