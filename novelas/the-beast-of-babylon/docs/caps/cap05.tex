\chapter*{Capítulo Cinco}
\addcontentsline{toc}{chapter}{Capítulo Cinco}


Gurgurum estaba mirando hacia el lugar de la ejecución desde el balcón
de las dependencias reales. Hubo un murmullo de voces mientras los
soldados formaban filas sobre el barrizal rojo de la plaza. Otros
soldados se alinearon en las almenas de las paredes circundantes. Y más
allá de las murallas de la ciudad, en la llanura, el ejército de
Hammurabi permanecía inmóvil, como si estuviera esperando.

Gurgurum vio cómo el rey salía de las puertas del palacio justo debajo
de él. Estaba rodeado de sus esclavos, consejeros y sacerdotes. Su
guardia real marchaba a su lado. Pero no Gurgurum. Un rayo cayó al este
y a Gurgurum le dio un escalofrío. Había nubes de tormenta reuniéndose
sobre las colinas, absorbiendo arena del desierto y oscureciendo el
cielo de mediodía. El calor se había esfumado.

Quería estar allí con su rey. Debía estar a su lado, pero le habían
ordenado que se quedase aquí para vigilar a la familia real.

Observó cómo el rey se acomodaba en su tarima y esperaba mientras la
Puerta del Prisionero se abría y tres jueces de la ciudad obligaban a
entrar al espía, flanqueado por una unidad de soldados fuertemente
armados. Gurgurum se sentía inútil desde aquí y eso alimentaba su furia
y odio. Este hombre era un extranjero, un extraño, no tenía sangre
babilónica pura corriendo por sus venas, y mucho menos se merecía un
juicio.

Gurgurum agarró fuerte la empuñadura de su espada. Quería saltar del
balcón, atravesar corriendo la plaza y meterle la espada en todo el
vientre.

Ali se estaba subiendo por las paredes, repiqueteando la punta de sus
dedos contra sus dientes como siempre hacía cuando estaba nerviosa y
pensando. Había visto con impotencia cómo se llevaban al Doctor a
rastras mientras éste gritaba cosas sobre el Hombre de las Estrellas y
el peligro en el que estaban todos. No se atrevía a imaginar lo que le
estaban haciendo allí fuera. El tiempo estaba pasando, marcado por el
taconeo de sus pies sobre el suelo de metal de la nave del Doctor. Le
había dicho que se quedara quieta, pero si no hacía algo lo matarían. Y
si lo mataban, no habría esperanza de derrotar al Hombre de las
Estrellas.

Lo peor era que el Doctor se había dejado el orbe en su cápsula sobre la
consola, de modo que, si por algún milagro no lo mataban los babilonios,
cuando el Hombre de las Estrellas atacara, no tendría forma de
derrotarlo.

Y entonces oyó un golpe.

Miró el escaner. Varios guardias se habían reunido alrededor de la
TARDIS y uno de ellos estaba apaleando la puerta con su lanza. Ali
dudaba que pudiera causar daño alguno, pero eso la puso furiosa y,
aunque intentó mantener la ira a raya, creció en su interior hasta
fulminar la pantalla con la mirada.

Otro golpe.

---Me estáis poniendo de los nervios ---dijo entre dientes---. Y no os
gustaría verme cuando me pongo de los nervios. Creedme.

Otro golpe.

El idiota gracioso todavía estaba cargando su estúpida lanza contra la
puerta.

Otro golpe.

Todo el cuerpo de Ali estaba ardiendo. Su furia era como un ser físico
en su interior, a punto de estallar. La imagen de la pantalla se estaba
encapotando de una niebla roja de batalla.

Otro golpe.

Uno de los otros guardias se echó a reír y dijo algo obsceno sobre la
TARDIS, y eso fue la gota que colmó el vaso para Ali.

Ni en broma iba a callarse por eso.

Ni en broma iba a quedarse allí pasmada.

Ni en broma iba a pasar de largo y dejar que mataran al Doctor.

Así que escuchó y midió los golpes y se sincronizó con el ritmo del
hombre, mientras se arrastraba lentamente hasta la puerta. Se puso a
contar mientras esperaba, y luego abrió la puerta justo cuando el
guardia volvió a cargar. Éste perdió el equilibrio y atravesó a tumbos
el repentino espacio vacío. Estaba preparada para enfrentarse a él y le
dio una patada en el pecho. Éste retrocedió, dándose un porrazo con dos
de sus conmocionados amigos, y Ali echó a correr por la puerta. Los
guardias no se esperaban esto y vacilaron al intentar darle sentido a
esta nueva amenaza. Cuando dos de ellos se separaron y huyeron
despavoridos hacia el portal del templo, Ali agitó sus antenodes y ellos
cayeron al suelo, aturdidos. Iban a tardar en despertarse.

Eso hacían tres más, además de los tres que seguían todavía en el suelo,
que salían huyendo de ella gritando de terror y pánico.

No había tiempo para pensar. Ali tenía que salir de aquí y encontrar al
Doctor. Y no debía dejar que nadie hiciese sonar la alarma. Sus
antenodes eran armas precisas y efectivas, pero las había usado en el
ataque y tardarían un tiempo valiosísimo en retractarse para prepararlas
otra vez.

Mientras estaba procesando toda esta información, uno de los guardias
lanzó una lanza. Se quebró sin apenas penetrar en su armadura - pero eso
fue suficiente. La rabia la inundó. No había vuelta atrás. A la porra lo
de intentar ser maja y aturdir a estos estúpidos. Usaría armas
letales\ldots{}

Avanzó.

Y el guardia que le había arrojado la lanza gritó\ldots{}

Y murió.