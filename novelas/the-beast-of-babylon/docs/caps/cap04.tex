\chapter*{Capítulo Cuatro}
\addcontentsline{toc}{chapter}{Capítulo Cuatro}

En ese momento, Zabaia, sumo sacerdote de Marduk, estaba de rodillas, su
cara presionando la fria piedra del suelo del templo, sus huesudas
piernas temblaban. Un viento de la nada estaba pateando la arena y el
polvo en el aire, tirándolos bajo sus ropas, y el rugido de algún
terrible dragón estaba chillando en sus oídos. No se atrevió a mirar
hacia arriba, pero podía sentir que había una nueva presencia. La forma
del espacio había cambiado. Algo había aparecido. ¿Tal vez su dios había
llegado? El mismo Marduk.

Reunió el valor suficiente para abrir un ojo y mirar a través del suelo
para ver lo que los guardias estaban haciendo. Se alegró de que, como
él, se habían postrado ante esta aparición. Ahora nadie se reiría de él
por cobarde.

---Es un huevo de un dragón ---oyó a uno de ellos susurrar.

---¡Está eclosionando! ---dijo otra voz, y una tercera voz ofreció un
rezo a Marduk para protegerlos a todos.

Zabaia esperó, su corazón latiendo, su aliento en su boca. Esperó a que
la ira de Marduk descendiera sobre su cabeza.

---Hola.

Lentamente Zabaia alzó la cabeza. Un hombre estaba de pie, vestido
extrañamente con ropa de color negro. Y detrás de él, ¿qué loco podría
imaginar que era un huevo?, había un gran cofre azul. El hombre estaba
mirando a Zabaia con la misma expresión de desconcierto que el propio
Zabaia sin duda sentía.

---¿Está todo bien? ---dijo el hombre---. Realmente no hay necesidad de
arrodillarse, ya sabes. Un simple apretón de manos es suficiente.

Gurgurum, capitán de la guardia real, estaba de pie junto a su rey,
Hammurabi, mirando el mundo desde el balcón de fuera de los aposentos
del rey. Desde aquí, en lo alto de los muros del palacio, podían ver
toda Babilonia. Pero Hammurabi no estaba contento. Estaba tirando de su
barba, jugueteando con los abalorios y los anillos preciosos anudadas en
ella.

---Mi familia gobernó Babilonia cuando era poco más que un pueblo
polvoriento del desierto y la convirtieron en la ciudad más grande que
existe ---estaba diciendo.

Gurgurum estaba orgulloso de servir al poderoso Hammurabi. Fue Hammurabi
quien había ampliado los templos, levantó las murallas de la ciudad y
fortaleció los diques que paró al gran y embarrado Eufrates cuando quiso
inundar las calles. Fue Hammurabi quien había hecho Babilonia segura. Y
una ciudad segura puede crecer rica y poderosa. Cada día nuevas casas
eran construidas, cada una más grande que la anterior, y bajo Gurgurum
se podía ver a la gente ocupándose de sus cosas, regateando en las
plazas, corriendo por las abarrotadas calles y a través de los muchos
puentes.

Fuera de las murallas de la ciudad, extendiéndose a través de la
exuberante y verde llanura regada por el Éufrates y el Tigris, había
árboles frutales y palmeras datileras y campos de trigo repletos de
esclavos trabajando duro, haciendo crecer la comida que alimentaba el
imperio de Hammurabi. Y más allá de la llanura fértil estaba el
desierto, sus colinas y tierra cocida del mismo color amarillo rojizo
que los edificios de la ciudad.

---Es el gobernador más grande que el mundo haya visto jamás ---dijo
Gurgurum---. Y ha construido el mayor imperio que el mundo ha conocido.
En unos pocos años ha derrotado a los reinos de Eshnunna, Elam, Larsa y
Mari. Ha pisoteado a su gente y los ha convertido en esclavos; ha
masacrado a sus jóvenes. Ha traído gloria a Babilonia.

---Pero, ¿no lo sientes, Gurgurum? ---dijo Hammurabi, golpeando con sus
puños sobre la balaustrada de piedra del balcón.

---¿Sentir qué, mi rey?.

---Como si una sombra hubiera caído sobre nuestro mundo. Temo que todo
esto podría desmoronarse. Que nuestros enemigos nos lo arrebatarán.

---Estamos preparados ---dijo Gurgurum---. Su ejército está preparado.
Sus carros viajan alrededor de las murallas de la ciudad para ahuyentar
a las tribus enemigas que pudieran ser tan insensatas como para lanzar
un ataque.

---Pero los sacerdotes me han advertido de que nuestro gran dios, Marduk
de los Cincuenta Nombres, podría abandonar la ciudad ---dijo
Hammurabi---. Le realizamos sacrificios, lavamos la boca de su estatua,
pero a los dioses no les importa mucho las débiles preocupaciones del
hombre. Nuestras ciudades son polvo bajo sus pies.

---No debería escuchar a los sacerdotes ---dijo Gurgurum amargamente---.
Son como viejas asustadas. Su fuerza está en su ejército. Debe gobernar
con la espada.

Gurgurum sabía, sin embargo, que había algo de verdad en las palabras de
Hammurabi. Los temblores se habían sentido bajo la tierra últimamente, y
la pared de un templo se había derrumbado, matando a un sacerdote y a
tres de sus sirvientes.

---Debo escuchar a los sacerdotes ---dijo Hammurabi---. Ellos son los
únicos que me pueden decir lo que están pensando los dioses. No puedo
dormir por temor a que Babilonia esté siendo atacada por fuerzas
misteriosas, que mis enemigos conspiran contra mí, que enviarán espías y
hechiceros para debilitarme.

---Entonces, mate a sus enemigos. Mate a todos en Babilonia que no sean
babilónicos ---instó Gurgurum---. Deje que el Éufrates corra rojo con su
sangre. Sofoque el Tigris con sus cuerpos. No se han ganado el derecho a
la justicia. Confíe en sus fuerzas y en los afilados bordes de las armas
de sus soldados.

---¿Y si nuestros enemigos son los mismos dioses? ---dijo Hammurabi---,
¿entonces qué?.

---Entonces oraremos, mi señor ---dijo Gurgurum y rió sombríamente.

Ocultándose a un lado de la puerta abierta, Ali estaba ansiosa de mirar
fuera de la TARDIS y tener su primera ojeada de un planeta alien. Pero
estaba obedeciendo al Doctor. Lo último que le había dicho antes de que
hubieran aterrizado era que se mantuviera oculta y cuidara de la TARDIS
por él hasta que estuviera seguro de que no había peligro.

---No te quiero meter en ningún problema.

---Puedo cuidar de mí misma ---había protestado Ali.

---Estoy seguro de que puedes ---había dicho el Doctor---. Pero no
quiero sorpresas. Sorpresas que no estoy esperando.

---Si lo estás esperando, entonces no es una sorpresa.

---Esos son mis tipos favoritos de sorpresas, Ali, las predecibles.
Tenemos que ser discretos, ¿vale? Sólo necesito neutralizar al Starman y
salir rápido.

Así que, ¿esto cuenta como neutralizar el Starman?. No estaba segura.
Sin embargo, cuando el Doctor había abierto la puerta de la TARDIS hizo
un pequeño sonido de decepción, como si hubiera cometido un error en sus
cálculos y no había estado esperando encontrar lo que había ahí afuera.

Podía oír su voz a través de la puerta.

---Tengo que hablar con quien esté al mando. Es bastante urgente.

---¿Quién eres?. ¿Eres un mensajero de los dioses?.

---Er\ldots{} Podrías decir eso\ldots{} Sí, digamos que soy un mensajero
de los dioses.

Ali deseó poder ver lo que estaba sucediendo. Se oyó el ruido de pies
rozando, de voces en conversación apresurada y luego un grito de pánico
del Doctor.

---¡No! No vayas ahí!.

Un hombre apareció en la puerta, más bajo que los otros hombres que
había visto, llevando un casco de reluciente bronce. Tenía el torso
desnudo y llevaba una lanza y un escudo. Cuando vio a Ali jadeó con
sorpresa y antes de que pudiera hacer nada más, instintivamente, Ali
arremetió con una de sus antenodes. Giró en el aire y golpeó al hombre
en un lado del cuello. Su cuerpo se convulsionó y cayó fuera de la
puerta, muerto para el mundo.

El Doctor le había dicho que protegiera la TARDIS, ¿cierto?.

No creía que nadie más tratara de subir, pero esperó junto a la puerta
por si acaso.

Hubo gritos desde fuera y el sonido de una pelea. Cómo deseaba tener una
mejor vista. Y entonces oyó la voz del Doctor, cansada y ahogada.

---¡No te muevas Ali! ---gritó---. Cierra la puerta y espérame. ¡Estaré
bien!.

Ella se acercó y cerró la puerta. No había sonado bien. El Doctor estaba
en problemas, estaba segura de ello. Si tan solo\ldots{}

Bueno, ¿por qué no? Él no le había dicho nada de no tocar los controles.
Estaba segura de que habría algo aquí. Alguna pieza de equipamiento que
la ayudara a ver qué le estaba sucediendo al Doctor. Incluso una vieja
reliquia cascarrabias de tecnología antigua como la TARDIS tendría
escáneres de algún tipo. Seguramente.

Se apresuró sobre la consola, localizó el acceso principal a la
pantalla, se inclinó y manipuló los controles con los dedos, al igual
que había pasado horas haciendo en el colegio y en la universidad y en
su habitación en Karkinos. Era buena con la tecnología, y aunque esto
era ridículamente retro pensó que tendría una idea bastante buena de
cómo encontrar lo que necesitaba.

Ahí.

Unos ajustes rápidos y ella tenía una visión clara del exterior. Otra
modificación y tendría sonido. Afortunadamente, el campo telepático del
circuito de traducción de la TARDIS le permitía entender cada palabra de
lo que se decía.

Desafortunadamente, no sonaba bien.

Un hombre vestido con elaboradas túnicas bordadas y llevando algún tipo
de subordinados le gritaba al Doctor, que estaba rodeado por más hombres
armados con lanzas.

---¡Mentiroso!. No eres un emisario de los dioses, eres un hombre, como
yo. ¡Eres un espía y la ley de Hammurabi establece claramente lo que se
debe hacer a los espías!