\chapter*{Capítulo Dos}
\addcontentsline{toc}{chapter}{Capítulo Dos}

Tengo como norma no hurgar en el pasado de mis prisioneros. Todos
tenemos algún esqueleto en el armario, ¿no?, y me esfuerzo para cumplir
mi palabra. Cuando le dije al Prisionero 428 que quería ser su amigo y
que los detalles de sus crímenes no me conciernen, lo dije en serio.

De todos modos, estaba comportándose de manera extraña. Suele ser
habitual en los recién llegados, ya que la Prisión es un lugar inusual y
cuesta un poco acostumbrarse. Recuerdo que cuando la ví por primera vez
desde el transbordador, mi ánimo, ya bajo, se deslizó hacia mis botas y
se escondió debajo de los calcetines. Sabía que aspecto tenía la Prisión
(después de todo, en mi antiguo trabajo, había participado en las
primeras etapas de su planificación), pero cuando todos los mundos de
nuestro sistema, incluso ahora, son tan imaginativos, ayudar a crear
algo tan gris y frío era horrible. Los centellantes cinturones de
anti-gravedad y la red de defensa externa emitían pequeñas luces en la
oscuridad, creando bolsas de casi-color con las que auto engañarte y ver
como el gris se transformaba en un púrpura brillante, incluso podías ver
como se teñía en tonos azules.

Pero en realidad, el asteroide era sólo un formidable puño de roca,
enorme e imponente, oscura y muy final. Nos habíamos quedado con un
peñasco que no quería nadie y en él habíamos metido a las personas menos
deseadas. Y las olvidamos por completo.

Mientras mi lanzadera se aproximaba me atraparon ensoñaciones típicas de
colegial, y me puse a pensar en formas de escapar. ¿Qué haría si fuera
un prisionero aquí?. ¿Cómo saldría de mi celda?. ¿Cómo saldría del
asteroide?. No pude evitarlo y me dejé llevar por mi entusiasmo, pero al
acercarnos a la roca, esos sueños de niño murieron, y no creo que haya
sido el mismo desde entonces.

Honestamente, la mayoría de los muchos sistemas de seguridad son
inútiles. No hay manera de salir de esta prisión. Las lanzaderas ni
aterrizan, en su lugar un sistema unidireccional de transporte cruza la
Matriz de Defensa, transportando suministros y prisioneros de forma
directa a la zona de recepción. No estoy diciendo que la gente no haya
tratado de escapar, pero nunca termina bien. La única manera de escapar
de aquí es morirse y, con el tiempo, todo el mundo se da cuenta. Una vez
se dan cuenta, no tengo más problemas con ellos.

Pero, ¿qué pasaba con el Prisionero 428, que preferiría ser llamado
Doctor?. Bueno, ¿qué pasaría con él?. Había visto gente como él antes,
muchas veces. Hablaría sin parar y gritaría, trataría de organizar
grupos de protesta furtiva y, a continuación, uno más obvio. Todos actos
de rebelión, fatigosas campañas, tal vez un boletín samizdat, tal vez
unos pocos intentos de fuga en masa. Inevitablemente, se producírían
lesiones (en su lado), y el apoyo se retiraría silenciosamente hasta que
el Prisionero 428 se quedaría solo y más apenado que cuando llegó.

Quería evitárselo. Por supuesto que sí, era mi deber humanitario. Él era
mi amigo, tanto si quería como si no, por eso había incumplido la
promesa que me hice a mí mismo y me había interesado por él. Simplemente
por su propio bien, por supuesto.

Llamé a Bentley que llegó tan rígida e inmaculada como siempre.

---Eso ha sido divertido, ¿no, Bentley? ---dije.

---Si usted lo dice, Alcaide ---el tono de Bentley era frío, pero los
bordes de su boca se movieron. Siempre me tentaba con la promesa de una
sonrisa. Sólo la había visto sonreír una vez, cuando un intento de fuga
salió terriblemente mal. Pobre Marianne. Para ser sincero, me alegraba
de no haber visto sonreír de nuevo a Bentley.

---¿Tomarás el té conmigo?.

Bentley inclinó la cabeza como señal de asentimiento.

---Si así lo ordena.

---Apenas es una orden, simplemente una costumbre entre amigos.

No éramos amigos y era una estupidez pretender que sí lo éramos pero,
aún así, no pude dejar de intentarlo. Ella trabajaba para mí y, sin
embargo, sólo me trataba ligeramente mejor de lo que lo hacían sus
superiores. No importaba lo que hiciera, lo correcto, severo y profundo
que fuera, ella siempre me miraba como si mi uniforme estuviera manchado
de mermelada. Ni siquiera sé por qué le ofrecía té. Todo aquello era
estúpido, pero había hecho la oferta, por lo que debería seguir adelante
con ella. Le sonreí, aunque quizás fuera demasiado forzado. Aún así, una
bebida entre colegas. Un Custodio nos trajo té y ambos pretendimos
disfrutar del mismo. La bebida estaba bien, siempre y cuando no dudaras
de su procedencia, de la del agua.

Bentley se instaló en la silla de metal situada frente a mí, era la
única persona a la que nunca parecía importarle aquella incomodidad de
hierro. Estaba esperando a que hablara.

---Creo que vamos a tener problemas con este ``Doctor'', ¿no crees?.

Asintió.

---¿Vas a llamar a 428 por su nombre?.

Estaba comunicativo. ~

---Podemos permitirnos ser generosos, dudo que vaya a estar con nosotros
mucho tiempo.

Bentley casi me llamó la atención durante un momento.

---¿Quiere que organice \ldots{}?.

---¡No, no! ---le aseguré a toda prisa---. Sólo quería decir que ya
hemos visto a gente de este tipo y nunca termina bien, ¿verdad?.

Bentley consideró la airada declaración con seriedad.

---Todavía tenemos a 112 en el Nivel 6.

Me llevó un momento recordar el número.

---Oh ---se refería a Marianne Globus. Pobre Marianne. Pobre 112. Una
querida amiga---. Ah, sí ---ninguno de los dos dijo nada durante un
momento---. Qué extraordinario que la recuerdes, Bentley. Me había
olvidado por completo de ella, la verdad. Prácticamente lo he olvidado
todo sobre ella, bueno, de lo que queda de ella ---pretendía sonar
idealista cuando, en realidad, la sola idea de lo que había sido de la
pobre 112 me hizo sentir mal---. ¿Y cómo está?.

Bentley casi vaciló un instante.

---No la he supervisado personalmente durante algún tiempo, pero los
Custodios en el Nivel 6 no han dicho nada negativo sobre la condición de
112 o su tolerancia al dolor .

Pobre Marianne. Habíamos dejado de pensar en ella. El Nivel 6 estaba
bastante vacío y ni siquiera había visto un tutor humano desde hace
bastante tiempo. ¡Oh, Dios!.

---Debería organizar una visita personal en algún momento ---no me
apetecía.

---En efecto ---Bentley inclinó la cabeza, insatisfecha de que la
reprendieran a ella.

---No te preocupes por eso ---le aseguré---. Tu haces un trabajo
espléndido supervisando el funcionamiento de toda la prisión. No puedes
preocuparte por cada detalle, ése es mi trabajo. Mi mujer solía citarme
un dicho de la Antigua Nueva Tierra: ``Cuida los centavos y las libras
se cuidarán solas.''

Bentley inclinó la barbilla, interesada.

---¿Qué significa, Alcaide?.

---No estoy del todo seguro, aunque, por otra parte, también me decía:
``No te preocupes por las cosas pequeñas''. Ese es el problema de los
refranes arcaicos, a nuestros oídos parecen contradictorios y difíciles
de cumplir.

---¿Un poco como el Prisionero 428? ---para Bentley era una broma.

---Sí ---sonreí tratando de demostrar que estaba contento con lo que
Bentley había dicho, ya que encajaba con el punto al que quería llevar
la conversación---. ¡Se parece mucho al Doctor!. Un hombre
extraordinario. Si ---me recliné en la silla, sintiendo como los treinta
y seis bolsillos de confort de la silla hacían su trabajo de forma
espléndida---. Ya sabe que me interesa especialmente no acabar con otra
situación como la del Prisionero 112 en nuestras manos\ldots{} bueno,
esparcido sobre nuestras manos.

---¿Qué quiere que haga? ---Bentley esperó a que yo hablara primero.

---Me preguntaba si, en este caso, ser prevenido vale por dos. Estaba
pensando en echarle el más pequeño de los vistazos a los informes de
428. ¿Crees que sería prudente?.

---Lo que usted piense que es mejor, Alcaide ---Bentley mantuvo un tono
neutral---. Se puede arreglar, puedo pedir los archivos por la TransNet.
Puede tardar un poco.

Las comunicaciones eran terriblemente lentas en este lugar. La
transmisión de los satélites TransNet de vuelta al Sistema HomeWorld era
errática. Al principio la idea había sido usar la TransNet para
transmisiones casi en directo de la programación de entretenimiento,
noticias y comunicaciones con los seres queridos pero, lamentablemente,
una vez que la Prisión se había finalizado descubrimos que el proveedor
de la TransNet había hecho un trabajo lamentable con las comunicaciones.
Incluso las comunicaciones más simples eran dolorosamente lentas. Los
presos simplemente llegaban aquí, a menudo sin que nosotros supiéramos
quiénes eran. El entretenimiento se enviaba desde los transbordadores en
el antiguo soporte de papel (¿quién dijo que el cristal de datos estaba
muerto?), y las pocas noticias que recibíamos lo hacíamos a través de
anuncios de texto extremadamente breves o por sumarios traspuestos a
copia en papel. En un principio te sentías aislado, pero ahora nos
habíamos acostumbrado y casi llegábamos a disfrutarlo. Presos y
guardianes, éramos todos ermitaños.

Viendo que la despedía, Bentley hizo ademán de levantarse, su taza de té
estaba a medio terminar. Le indiqué con un gesto que permaneciera
sentada.

---Tranquila ---le aseguré---. Puedo hacerlo desde mi terminal.

A veces pienso que ella cree que soy un viejo sin esperanza, pero
toquiteé el teclado para despertar el ordenador, éste respondió
lentamente. Los terminales con los que nos equiparon fueron
suministrados por el mismo contratista que puso el lamentable sistema
TransNet. Son horribles. Los iconos volvieron lentamente a la pantalla y
golpeé el de «Registros». Volví a hacerlo. Finalmente acepté que la cosa
se había congelado.

En casa me había acostumbrado a pedírselo todo a mi tableta,
constantemente, y ahora apenas me molestaba en tocarla una vez al día.
Me vi obligado a confiar en mi propio ingenio. Del cual estaba bastante
orgulloso por la libertad que me proporcionaba. De todas maneras,
estaría bien que el sistema funcionara ni que fuera una sola vez.

Bentley estaba de pie y se dirigía a la puerta.

---Tal vez sería mejor que buscara yo los informes ---se ofreció
gentilmente.

Realmente cree que estoy acabado. En fin. Había otra taza de té en la
tetera, así que la serví. No había terminado cuando Bentley regresó con
los informes de 428 pasados a papel y metidos en una carpeta. Me senté
con el resto del té a leer a fondo. Tras unas cuantas páginas dejé de
leer a fondo y simplemente pasaba por encima hasta que aparté la carpeta
asqueado.

Cogí el vaso, pero el té se había enfriado. Tampoco podía enfrentarme a
esto.

Me di cuenta de que Bentley todavía estaba en la habitación, mirándome,
y evaluando mi reacción con curiosidad. Es como uno de los Custodios en
muchos sentidos, silenciosa, sólida y sombría. Nunca se lo diría, por
supuesto, tiene sentimientos, estoy seguro. En alguna parte. Se sentiría
terriblemente herida.

---¿Ha leído lo relativo a los crímenes del Doctor? --- preguntó.

---Prisionero 428 ---dije con firmeza. Ya no se merecía un nombre.
Asqueado, empujé la carpeta hacia ella---. Llevatela.

Mi tableta se había reiniciado y la usé para entrar a la cámara de la
celda de 428. La suya era tan espartana como todas las de nuestros
prisioneros. Cada caja contenía un estante para sentarse y dormir, y una
puerta. No había ventanas porque no había vistas y sólo a los Guardianes
se les permitía ver las estrellas y el espacio. Los presos simplemente
veían las paredes y a otros presos. Cada celda era del tamaño
reglamentario, aunque las de Nivel 6 quizás fueran un poco más pequeñas.
Y sin embargo, la celda de 428 parecía estrecha, como si el hombre
llenara la habitación.

Se paseaba por el área, tirando del uniforme naranja como si tratara de
transformarlo en algo distinto de la prenda sin forma que era. El
naranja era el único color que los prisioneros veían y, al estar en
todas partes, ya ni lo notaban.

Lo miré con fascinación. Así que este era el hombre, el hombre que había
\ldots{}negué con la cabeza. Sus crímenes apenas podían pensarse. Lo
odiaba. Fue poco profesional de mi parte hacerlo, pero lo odiaba.

Me preguntaba cuando 428 se cansaría de pasear. Todos acababan
cansándose. Cuando yo era un niño, todavía teníamos zoológicos, y mis
prisioneros eran como los animales del zoológico, caminando por los
límites de su confinamiento, como si de alguna manera pudieran desgastar
el suelo y las barras, antes de aceptar la derrota.

Prisionero 428 aún no había cedido, aún no se había dado cuenta de que
nunca saldría de la Prisión.

Me centré en su rostro, tratando de leer sus crímenes en él. Éramos, más
o menos, de la misma edad, pero sus rasgos parecían demasiado estirados
por el esfuerzo de contener su culpabilidad, como si tratara de contener
varias vidas de cansancio y enojo. Era la cara de alguien acostumbrado a
mandar. No era exactamente guapo, pero sin duda era inolvidable. Me
entró un escalofrío al pensar que esa era la última cosa que muchas de
sus víctimas habían visto. No una puesta de sol, ni sonrientes rostros
de sus seres queridos tras un triste adiós, sino aquella cara enojada
hirviendo lejos como una estrella moribunda. Me estremecí.

Me prometí a mí mismo que, costara lo que costara, le haría pagar por lo
que había hecho.

Las alarmas me despertaron de mi ensoñación. Me había adentrado en mis
pensamientos, cosa que era siempre un error. Hay mucho que hacer en la
Prisión, y no es apropiado que el Alcaide se ponga a soñar despierto.
Incluso cuando las cosas funcionan sin problemas.

Volví a mirar a la cámara de la celda y me espanté, era casi como si 428
me estuviera mirando a través de la lente. Aquellos ojos. Las cosas tan
terribles que habían visto.

Corté la imagen rápidamente. Fue entonces cuando saltaron las alarmas.

En la prisión tenemos una gran cantidad de alarmas, y ninguna de ellas
suponen una buena noticia, además suenan como almas en pena gritando.
Esta no producía la particular agonía propia del ``Escape del
Prisionero'', pero aún así era bastante estridente. Últimamente la
habíamos oído bastante.

Bentley llamó bruscamente a la puerta de mi oficina y entró. ``FALLO DEL
SISTEMA'' anunció, en mayúsculas. Los dos lo sabíamos, pero el
Procedimiento de la Prisión establecía que el Alcaide debía ser
informado. Asentí con la cabeza y me levanté.

Los dos fuimos rápidamente hacia la Sala de Control, donde los Custodios
se deslizaban silenciosamente entre los terminales. Las pantallas
mostraban cada celda, cada pasillo, cada área de la Prisión. Un mapa
gigante de todo el asteroide brillaba en una de ellas. En teoría,
debería estar mostrando dónde estaba el fallo del sistema, pero en su
lugar, la pantalla estaba parcialmente oscurecida por un icono que decía
``ACTUALIZANDO\ldots{} ACTUALIZANDO\ldots{}''. Era de lo más inútil.

El sistema de diagnóstico de la Prisión había sido instalado por un
contratista diferente del que había proporcionado las tabletas y la
TransNet y, definitvamente, no se habían llevado bien, si bien ambos
habían hecho un trabajo de mala calidad.

Miré a Bentley, moviéndose rápidamente entre los Custodios, accediendo a
las actualizaciones verbales de sus compañeros humanos. Si todo en la
vida pudiera ser tan eficiente como Bentley, pensé. Tal vez con un poco
más de cariño, sólo un poco. Pero ella era todo lo que uno podía desear
en una crisis.

La verdad era que había muy poco que pudiéramos hacer. Estos cortes se
producían cada vez con más frecuencia y no había explicación. Si este
último seguía el patrón, pasarían entre tres y cinco minutos y
volveríamos a la normalidad. Pero mientras las alarmas sonaban, dependía
de Bentley y su equipo el asegurarse de que no hubiera sistemas
centrales afectados. Había encargado a algunos custodios que trataran de
averiguar la raíz del problema, pero hasta ahora no habían informado de
nada. En su lugar, se habían convertido en expertos en lidiar con estas
emergencias, reasignando recursos sobre la marcha para que los cierres
no fallaran, manteniendo la red de contención y el sistema
medioambiental estabilizados. A veces, esto significaba que la cena
quedaba cruda, la gravedad un poco alta o el aire un poco rancio. Hasta
ahora no habíamos tenido que hacer grandes sacrificios.

Una noche, Bentley y yo esbozamos algunos protocolos de emergencia.
Mejor dicho, yo había hecho algunas sugerencias y ella había escuchado,
para luego decir: ``Si me lo permite \ldots{}'' y que me tocara
corregirlos luego. Pero estábamos preparados por si la situación
empeoraba y las fuentes de energía no podían ser cambiadas a tiempo. No
había sido una conversación fácil. Habíamos acordado ponerlas en
práctica si el fallo de sistema se alargaba hasta los siete minutos. Eso
sería el fin. Un reloj rojo y parpadeante contaba los minutos que
llevábamos con la corriente cortada.

El mapa de la Prisión brillaba con las palabras
``ACTUALIZANDO\ldots{} ACTUALIZANDO\ldots{}'', y el reloj indicaba que
habían pasado cuatro minutos. Bentley continuó moviéndose con tranquila
eficiencia y los Custodios seguían desplazando la antena a través de los
paneles, informado de otros fallos y de los éxitos en la relocalización
de recursos.

El reloj pasaba ya de cinco minutos. Me di cuenta de que los Guardianes
humanos se buscaban con la mirada, nerviosos. El pánico se estaba
extendiendo. La mayoría de las veces, nos podemos olvidar de que estamos
en una roca en el espacio profundo construida artificialmente para
contener vida. Cuando los sistemas funcionan, la fragilidad de nuestra
existencia desaparece de nuestras mentes, pero vuelve de repente durante
una alerta y nos recuerda que, si la alimentación falla por completo,
esto es todo. Tenemos un suministro limitado de oxígeno. Aunque
pidiéramos ayuda, incluso si esa ayuda llegara del Planeta Nata o de la
colonia más cercana, hay muy pocas posibilidades de que llegara antes de
que nos quedáramos sin aire. Todos nosotros, prisioneros y guardias,
estaríamos enterrados en nuestra tumba.

El reloj pasaba ya de cinco minutos y quince segundos. Una horrible
primera vez. Me pregunté si debía decir algo para tranquilizar el
ambiente o para dar ánimos, o hacer algo que oliera a la normalidad de
un lunático, como hacerme una taza de té. No me la bebería, sería
simplemente por las apariencias. Su Alcaide no es presa del pánico, está
bebiendo té, ésta en calma así que usted también puede estarlo.

El reloj marcaba cinco minutos y veintinueve segundos. Un nuevo y
bastante formidable récord. Vi a Bentley mirándome, tratando de llamar
mi atención pero seguí con la mirada en el frente. Teníamos noventa y un
segundos antes de empezar a tomar terribles decisiones. Disfrutemos de
los noventa y un segundos lo mejor que podamos. Si sobrevivíamos, las
decisiones que tomáramos quedarían en nuestra conciencia para siempre.

Cinco minutos cuarenta y un segundos, de repente el mapa de la Prisión
se aclaró. ``SISTEMAS NORMALES'', informó. La alarma paró. El rojo
desapareció de la luz.

De repente había mucho silencio, salvo por un colectivo suspiro de
alivio y el olor a sudor y pánico que impregnaba el aire.

---Bien hecho, Bentley ---dije---. Bien manejado.

Como si hubiera evitado de alguna manera una terrible crisis. La verdad,
la terrible y aterradora verdad, era que no teníamos ni idea de lo que
estaba pasando.

Mi comunicador pitó. Era una llamada desde el Nivel 7. Descolgué de mala
gana, sabiendo que sería el Oráculo.

La grasa de la cara del Oráculo llenó la pantalla, la papada temblaba
cuando movía la cabeza hacia los lados.

---Dios mío ---ronroneó---. Ha estado cerca, ¿no?.

Una de las pocas cosas en las que Bentley y coincidíamos era en nuestro
odio hacia el Oráculo. Todo en él nos irritaba. Ninguno de nosotros
había tenido contacto físico con él, pero aún así nos resultaba
repelente. Sus manos estaban por todas partes, siempre llenando la
pantalla, tocando un teclado invisible cada vez que hablaba,
revoloteando, subiendo y bajando.

Al Oráculo le gustaba hacer sólo dos cosas en la vida: predecir el
futuro y decir ``te lo dije''. Sus predicciones rara vez parecían
cumplirse, aunque eran de una naturaleza tan relativa que no podías
reclamar nada tras el evento.

Así lo hizo en esta ocasión.

---¿No le dije que se producirían vibraciones púrpuras? ---dijo, alzando
los dedos por encima del pelo y dejándolos caer hasta su barbilla---.
Bueno\ldots{}yo diría que un fallo del sistema de casi seis minutos es
definitivamente púrpura. ¿No cree?.

Frunció los labios y esperó una respuesta. Lo que más nos molestaba del
Oráculo era que lo necesitábamos. Sin él, no habría nadie que se ocupara
del Nivel 7.

El Oráculo dejó de esperar una respuesta y se reclinó en la silla,
formando con los dedos primero un campanario y luego una catedral.

---Os diré una cosa, amigos míos, hay tiempos púrpura por delante.
Recordad mis palabras ---cortó la comunicación.

Volví a mi habitación para calmarme, relajarme, reflexionar, para tratar
de trazar el futuro, para tratar de pensar en algo. Mi tableta volvía a
mostrar la vista de la celda del Prisionero 428. Estaba de pie, mirando
hacia mí de nuevo, impasible. Levantó una ceja con curiosidad, como si
estuviera esperando algo. ¿Podría estar detrás de todo,, pensé con un
estremecimiento.

Apagué la tableta, pero el fantasma de esa mirada se mantuvo en la
pantalla por un momento. ¿Qué sabía?, pensé. ¿Qué sabía realmente el
Prisionero 428?.