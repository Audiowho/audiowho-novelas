\chapter*{Capítulo Cinco}
\addcontentsline{toc}{chapter}{Capítulo Cinco}

Los dos primeros días y noches del acuerdo con 428 pasaron sin
incidentes. Se lo comenté a Bentley, pero ella cerró los labios y se fue
a presentar un informe TransNet. Y luego, en la tercera noche\ldots{}

Nadie sabía exactamente cómo comenzó el incendio. Pero fue en algún
lugar de la pila de libros, entre biografías y ficción. Papel empezando
a arder y primeros humos que deberían haber hecho saltar una señal de
alarma. Pero no hubo ninguna alarma. No fue sino hasta que el primer
libro estalló perfectamente en llamas y entonces el fuego se extendió.
En orden de sistema de clasificación Decimal Dewey.

La noticia se extendió alrededor de la prisión con la misma rapidez. A
pesar de que era mitad de la noche y todo el mundo estaba encerrado en
sus celdas, aún así se difundió la noticia. Los cotilleos viajan más
rápido que los incendios. Las alarmas ayudaron, moviendo a la gente de
sus problemáticos sueños a los pasillos llenos de humo.

El prisionero 428 se asomó a la ventana con rejas de su celda, hablando
directamente al Custodio que estaba custodiando el pasillo. Era uno de
los modelos más nuevos. Uno equipado con un sistema vocal básico.

---La situación está bajo control.

---¿Qué situación?.

---Hay un incendio. La situación está bajo control.

---¿Dónde está el incendio?. ¡Dónde! ---428 estaba alerta, sospechando
ya.

---La biblioteca de la Prisión.

428 se mantuvo en la ventana de su celda, como un león enjaulado que
gruñía. Vi las imágenes de nuevo más tarde y, en realidad, no gruñió ni
gritó. Y, sin embargo, si me hubieran preguntado, habría dicho que lo
había hecho. Él se quedó allí, con mucha calma.

---¿Dónde está Lafcardio? ---preguntó finalmente.

El Custodio no respondió. 428 repitió su pregunta. Y luego, con un
cansado suspiro de resignación, desaparecido de la ventana de su celda.
Un momento después, la puerta se abrió y la cámara del Custodio se quedó
en blanco. Mientras lo hacía, 428 murmuró algo amargamente.

---Mesas y sillas.

Cuando 428 llegó fuera de la biblioteca, una falange de Custodios se
había reunido, formando una barrera delante de la humeante puerta
abierta.

---¿No estáis haciendo nada? ---preguntó.

Bentley apareció, sonriéndole con dulzura.

---¿No estás en tu celda 428?.

---¿Qué estáis haciendo con el incendio?.

---Es una lástima ---Bentley estaba tranquila mientras sonaba una nueva
alarma---. Alarma de Inflamación. No se preocupe. Los protocolos de
vacío se están activando automáticamente. En menos de treinta segundos,
la biblioteca estará sellada y el aire será expulslado al espacio.

428 escuchó lo que dijo, asintiendo con atención.

---¿Y qué hay Lafcardio?.

---Él estaba en su celda cuando sonó la alarma ---Bentley se encogió de
hombros.

---Sí. En su celda. La celda que no está cerrada por la noche debido a
que es de fiar ---428 ya estaba pasando a empujones sobre ella---. En su
celda. Cuando sonó la alarma diciéndole que su preciosa biblioteca
estaba ardiendo. En su celda. Alarma de Inflamación. Menos de treinta
segundos, ¿no?.

El Doctor desapareció en el humo. Veintiséis segundos después, los
Protocolos de vacío se activaron. El espacio se llenó de una breve
ráfaga de fuego que brillaba hasta que fue más allá de la burbuja
atmosférica que rodeaba el asteroide. Entonces la llama se apagó.
Flotando lejos de la prisión estaban las primeras cosas que habían
escapado alguna vez de ella. Paquetes de libros quemados caían, algunos
en grupos, algunos sólo flotantes páginas individuales, y se movían
hacia el espacio para siempre. ``La mujer de blanco''~chocó contra ``El
Código Da Vinci''~y, como de común acuerdo, se apartaron de ``¿Se lo
decimos al Presidente?''.

428 estaba de pie en el otro lado de la puerta blindada, y exhalaba
rápidamente. Su cara estaba ennegrecida por el hollín, pero parecía
ileso por su heroísmo. Lafcardio era un pequeño montón arrugado en sus
brazos. 428 lo depositó suavemente en el suelo, y comenzó rápidamente a
administrarle respiración artificial.

Esto no había podido suceder. Estoy seguro que esto no había podido
suceder. El shock de Bentley parecía auténtico. Se acercó a 428.

---Deja que un Guardián cuide de él ---pero él la apartó airadamente.

Trabajó en el cuerpo como un experto y durante mucho tiempo pareció
inútil. Entonces Lafcardio jadeó, balbuceó y miró aturdido a la cara de
428.

---Bueno, esto es, sin duda inesperado ---graznó, luego tosió. Su tos no
paraba.

428 lo levantó para que estuviera en una posición sentada y luego esperó
hasta que la sequedad disminuyera ligeramente. Se volvió a Bentley.

---Le conseguirás un vaso de agua ¿verdad?. Oxígeno sería encantador,
pero agua estaría bien, querida.

Sobresaltada, Bentley comenzó a obedecer antes de detenerse a sí misma
mientras retransmitía la orden a un Custodio. Se volvió hacia 428, cruzó
los brazos y lo observó.

---Lo siento mucho ---susurró 428 a Lafcardio.

---¿Mis libros? ---gruñó el hombre.

---Se han ido todos ---dijo 428---. Lo siento. Todo es culpa mía.

---¿Cómo\ldots{}?. ¿Cómo\ldots{}? --- Lafcardio estaba llorando, las
lágrimas corrían sobre el hollín de su cara. No parecía haber escuchado
a 428 en absoluto---. Traté de detenerlo\ldots{} pero las llamas\ldots{}
tan difícil.

428 lo abrazó y miró directamente hacia mi cámara.

---Alguien ---el tono de 428 era sombrío--- estaba tratando de darme una
lección.

---Teníamos un acuerdo ---dije a 428 cuando llegué más tarde---. No
dejabas tu celda.

---Vas a ser infantil, ¿no? ---428, en todo caso, parecía ser el
infantil. Estaba apoyado contra la pared, habiendo dejado de mala gana a
Lafcardio con un Custodio Médico. Hizo un gesto hacia la biblioteca
sellada---. Todo esto fue idea tuya, ¿no?.

---¿Me estás acusando de vandalizar propiedades de la prisión?
---grité---. Si lo deseas, una investigación de este accidente se
llevará a cabo\ldots{}

---Oh, ahórrate la molestia ---428 estaba brutalmente aburrido. Bostezó
y miró a donde ir, pero luego se dio la vuelta, con sus dedos golpeando
en mi cara---. Escúchame, pequeño hombre estúpido\ldots{}

---Yo\ldots{}yo soy el Alcaide aquí. ¡Se me debe respeto! ---me
rehice---. Soy más alto que tú.

---Eres muy diminuto. Pequeño. Minúsculo. No vale la pena el esfuerzo.
El esfuerzo de todo esto ---agitó los brazos hacia la habitación y luego
me señaló---. Escúchame. Nunca quemes libros. Incluso los peores. Sobre
todo cuando son el deleite de un hombre inofensivo. ¿Y por qué?. Tu plan
salió horriblemente mal. Podrías haberlo matado, sólo para darme una
lección. Podrías haber tenido eso en tu conciencia.

---Te aseguro que mi conciencia está tranquila.

---¿Lo está?---toda la furia de 428 era como estar atrapado en un
congelador. Aparté la vista. ---Creía que no. Si algo de esto va a valer
la pena\ldots{} oh, da igual --- se encogió de hombros, con los hombros
subiendo y bajando como si fueran montañas---. Ya es negra noche. Me voy
a mi celda. A menos que\ldots{}

Bentley apareció al lado de 428 sin ningún problema.

---Si me permites señalarte, 428\ldots{}

---Ah ---428 parecía poco impresionado ---. Es la Pequeña Cerillera.

La sonrisa de seda de Bentley no vaciló.

---¿Me permites recordarte, Preso 428, que has infringido el toque de
queda, además de un acuerdo personal con el Alcaide?.

428 arqueó una ceja lentamente y comenzó a aplaudir. Fue lento y
sarcástico.

---Por favor, dime que es confinamiento solitario.

---En efecto ---Bentley se negó a parecer inquieta.

---Espléndido ---428 se frotó las manos---. Porque no soporto la vista
de ninguno de vosotros en este momento. Me apetece un descanso de todas
vuestras caras. Llévame a mi nueva caravana. Oh\ldots{} ---se volvió a
Bentley---. Y pídele perdón a Lafcardio de mi parte. ¿Lo harás?.

Bentley asintió.

---Bien--- dijo 428, paseando hacia los Custodios que esperaban---.
Porque no creo que vaya a conseguir una disculpa vuestra.

Habíamos colocado una cámara en su nueva celda, pero no mostraba nada
más. Simplemente 428 se sentó, inmóvil. De espaldas a la cámara. Durante
horas.

La chica regresó. Estaba en la pista de aterrizaje, con el pelo
cuidadosamente peinado hacia atrás. Su ropa era idéntica, solo que
cubierta de manchas de pintura. Estaba sosteniendo una pancarta.

---Oh ---dijo---. ¡Hola! ---extendió su mano--- No la agites ---dijo---,
la pintura sigue húmeda. Y además, valla electrificada.

---Sí ---dije.

---Sabía que vendrías ---Clara parecía contenta.

---Bueno, solo estoy obligado a hacer una visita. Por supuesto, puedo
salir una segunda vez.

---Saldrás muchas veces más ---me aseguró.

---Bueno ---reí--- eso es estrictamente criterio mio.

---Realmente eres un culo pomposo ---sonrió Clara---. Te gustaría mi
director.

---Ah ---dije---.Así es como vas a ganar clemencia para el Doctor.

---Ajá ---dijo---. Así y con los carteles. ¿Te gustan?.

Agitó la pancarta. ---LIBERAD AL DOCTOR---decía, decorada con varias
huellas de manos de colores.

---La clase 2B los hizo ---dijo. Le dio la vuelta a la pancarta. En la
parte posterior, decía SALVAD A DOT COT. La miré.

---Oh sí, la 2A se confundió un poco. Dot Cotton. La famosa chimenea
cockney. No importa. El palo de escoba me lo dio Danny. Es otro maestro
de la escuela. No, espera, él no importa Claramente era un novio,
¿estaba celoso de esto?. Extrañamente, no.

---¿Estás bien? ---pregunté.

---Sí. Er. ¿Por qué?.

---Pareces nerviosa.

---Ah, sí. Bueno, en este momento estoy en dos lugares a la vez ---el
rostro de Clara cayó---. No importa. ¿Cómo lo lleva el Doctor?.

---El Preso 428 está siendo atendido de acuerdo a los protocolos
acordados.

---¿Protocolos acordados?. Apuesto a que le encanta ---Clara hizo un
mueca.

---No, no lo hace.

---Lo imaginaba ---trató muy bien de parecer casual---. Escucha
---dijo---. Puedo\ldots{} ¿puedo darte algo?.

---¿Es una petición? ---suspiré---. Puedo aceptarlos, pero no regalos o
cartas para los presos ---Clara me había decepcionado. Aburrido, señalé
un hueco en la valla metálica. Había una caja plateada en ella---. Si
así lo deseas ---realmente me esperaba más de ella---. Coloca lo que
tengas en la caja.

Clara dudó.

---Es de vital importancia que leas esto. Lo entenderás.

---¿Pintado para mi por la clase 2B?.

---Bueno, no. Bueno, de acuerdo, un montón de cosas importante y solo un
dibujo. Pero es bastante bonito. Y la 2B me hizo prometerlo. Después de
que el Doctor viniera una tarde e hiciera animales con globos. De todos
modos, esa no es la cosa ---sacó un bulto de su bolso y lo colocó en la
caja---. Tienes que leer esto. Oye, ¿a dónde vas?.

---Bueno ---dije ---te aseguro que leeré tu material. Una vez que los
siete sistemas necesarios para transferir estos objetos hayan sido
desactivados. Lo cual requiere supervisión y revisión. Si los artículos
han pasado la revisión, entonces el contenido de la Caja Honesta será
pasado.

---¿Caja Honesta?.

---Sí. Se llama así.

---¿Incluso el correo se censura?.

---Se revisa.

Clara arrugó la cara, furiosa.

---Apuesto a que con lo que acabas es con un cuadro de unos perros
jugando al fútbol. Es bastante bueno, aunque se pegaron por el fútbol.
Está hecho de papel de aluminio. Así que si se te cae quizás encontrarás
la finalidad del cuadro un poco desconcertante. Muchos perros perdidos.
Desconcertante e inútil ---subrayó las últimas palabras.

Hice una reverencia formal.

---Entonces ansío verlo ---le dije.

---¿Cómo lo lleva el Doctor? ---gritó conforme me giraba.

---Oh, sobreviviendo ---dije y la dejé en la pista de aterrizaje,
agitando su pancarta.

La Guardiana Donaldson acabó con la pelea en la cafetería. 203\ldots{}
siempre había sido un poco matona. Abesse, ese había sido su nombre.
Había sido una mercenaria en el bando perdedor durante la Revolución de
HomeWorld y siempre había parecido un poco resentida por ello. Debido a
que el nombre erróneo estaba en su cheque ese mes, el nuevo Presidente
la había enviado aquí. No estoy seguro de echarle la culpa por estar
resentida. No realmente.

Se había instalado en la vida de la Prisión al igual que alguna gente se
instalaba en una misión que no les importaba mucho. Tenía un ojo en el
final de ella, lo sabía. Lo cual parece irónico porque, como estoy
seguro de que sabéis, nadie sale de aquí.

En cierto modo, era sorprendente que el Preso 428 no se diera cuenta de
su presión sobre él. 203 era alta y bastante llamativa. Creo que la
habría llamado hermosa si su cara no fuera tan dura. Su pelo seguía
cortado a la moda, como si estuviera lista para ponerse un vestido e ir
a una fiesta. Solo esperando la llamada de un lugar mucho más
interesante.

Personalmente, pensaba que era un gran desperdicio tener a un mercenario
de sus habilidades aquí. Por definición, estaría igual de feliz de
trabajar para el nuevo régimen. Pero el gobierno había sido muy tajante
sobre estas cosas.

En la Prisión, 203 en ocasiones actuaba como un intermediario informal
del sistema judicial. Esta es una forma educada de decir que Abesse
atacaba a la gente que le molestaba.

De ahí la razón por la 428 salió de pronto volando por los aires. Se
levantó de los pies de los presos que habían retrocedido a toda prisa.
Se sacudió y palmeó arrugas imaginarias de su uniforme.

---Si querías mis gachas solo necesitabas pedirlas ---dijo.

Abesse se puso sobre él. Sobrepasándolo.

---Así que\ldots{} ¿no son las gachas? ---aventuró 428.

No tuvo tiempo de protegerse antes de que fuera lanzado contra la pared
con un chasquido y un golpe seco. Me di cuenta de que los Guardianes no
estaban reaccionando, y me pregunté si debía de enviar algunos
Guardianes. Pero, de nuevo, esto era el Preso 428. Esto le haría algún
bien. A veces todos hacíamos la vista gorda con 203. Era un activo útil.
El montón arrugado que era 428 se sentó y gesticuló a 203.

---Estás enfadada conmigo. Entiendo eso. Pero lo que no entiendo
completamente es por qué. Si tuviera que adivinarlo sería aburrido,
créeme, y soy muy consciente de que mis modales en estas situaciones
ocasionalmente pueden ser poco afortuna\ldots{}

428 rodó un buen trozo por el suelo antes de detenerse. Entonces se
levantó, sonriendo. 203 cargó contra él. 428 no se movió. Bueno, apenas
se movió. Sólo un dedo, moviéndose cerca del hombro de 203 mientras se
acercaba hacia él. El Preso 203 de repente cayó al suelo con un gemido.

---Tienes suerte ---428 estaba sobre ella---. Últimamente no utilizo
mucho el Aikido Venusiano. Y rara vez en damas. No es una regla. Pero,
claro ---un guiño---. No eres una dama.

203 abrió los ojos, mirándolo. Le costó levantarse. 428 la consideró.

---Estás enfadada conmigo. Lo entiendo. ¿Es por la biblioteca, por el
pobre Lafcardio, o porque no hay ni sillas ni mesas?. ¿O\ldots{}
realmente es por los tres?. Porque, si lo juntamos, he de admitirlo, sí
que parece malo.

203 se levantó, gruñiendo. Y entonces, extrañamente, cayó al suelo de
nuevo. Esta vez no se levantó.

---Eso fue más bien una palmadita ---murmuró 428 avergonzado. Entonces
se agachó junto a 203 y le susurró algo.

Para entonces, la Guardiana Donaldson había llegado para acabar con la
pelea. Contempló los restos con su típica sequedad. Encontró a 428 tan
poco divertido como yo. Pero 428 parecía contento de verla. 428 ayudó a
203 a levantarse.

---Su nombre es Abesse ---informó a Donaldson--- y me temo que está un
poco sacudida ---Abesse estaba apoyada y atontada contra la pared---.
Ten cuidado con ella. Su oído interno esta temporalmente apagado por lo
que encontrarás que su sentido del equilibrio no responderá preguntas
durante unos días.

A posteriori, colocar al Preso 203 junto a Lafcardio en el centro médico
fue un error. La siguiente vez que vi a 203, unos días más tarde, ella y
el\ldots{} Preso 428 se estaban dirigiendo al taller de artesanía, un
lugar donde los presos se dedicaban a la artesanía. Los dos parecían,
bueno si me permitís la expresión, como uña y carne. Cómo traspasó
exactamente el sello de la biblioteca, no lo sé. Solo lo averiguamos más
tarde. Cuando le pregunté a 428 de dónde había sacado la madera.

---¿No te gustan las mesas? ---preguntó dolido.

Estábamos en la cafetería. Que ahora estaba llena de mesas. Y bancos.

---Los hemos repintado ---explicó 428---. El amarillo era un poco
provocativo. El rosa tiene un efecto calmante. Lo descubrieron en
Alcatraz ---se inclinó indicando un banco.

---Toma asiento ---insistió---. Bueno, no en este banco. Se tambalea. Un
poco.

---Dónde\ldots{} ---mi boca estaba seca--- ¿De dónde habéis sacado la
madera?.

---Oh, bueno ---428 parecía casual---. Había todas esas estanterías en
la biblioteca. Estanterías vacías. Verás, ya no hay libros. Y Abesse y
yo pensamos que\ldots{} bueno\ldots{} es una maravilla con una sierra
---sonrió complacido.

Me quedé allí. Me quede allí mientras los presos empezaron a aparecer
para la comida matutina, recogiendo su avena y sentándose en las mesas.
Todos estaban sonriendo. No hablando, simplemente sonrientes. Abesse se
situó detrás de todos. Con los brazos cruzados. Mirándome. A la espera
de un desafío.

---A veces ---sonrió 428--- no es el destino final lo que cuenta, sino
como llegar ahí. Y a veces, tanto el viaje como el destino son más que
gratificantes.

Acercó algo de la mesa y me lo dio. Era un tiesto, con una planta.

---Es un rosal ---dijo---. Sé que te gustan las flores. La hice crecer
en el centro hidropónico.

---Pero a los presos no se les permite\ldots{}

El Doctor se encogió de hombros.

---¿Y no es eso vergonzante?. Ponla en tu escritorio. Riégala. Disfruta.

Me quedé mirándolo. Pequeños bulbos rojos ya estaban emergiendo. A Helen
siempre le habían gustado las rosas. Quise decir algo. Pero no podía
agradecerselo al Doctor, no podía. Eso hubiera hecho su victoria
completa. En su lugar cogí la planta. Ya podía olerla. El Doctor se giró
y comenzó a alejarse.

---No te quedas ---me detuve. Mi tono había sido abrupto, mostrando lo
mucho que me había afectado. Así que me detuve. No dije ``para disfrutar
de tu victoria''. En su lugar me las apañé con un:--- ¿para disfrutar de
tu avena?.

428 se giró y se encogió de hombros molodramáticamente.

---No. No soporto la comida de aquí. Pero no te preocupes. Está en mi
lista.

Me sonrió y se fue. Bentley y yo nos miramos el uno al otro. Casi me
estaba mirando a los ojos.

---Está ganando, ¿verdad? ---dije.

Bentley no respondió. Apartó la mirada.

Hubo una fluctuación de energía esa noche. Seis minutos y doce segundos.
Casi me había olvidado. Bentley y yo discutimos sobre ello, con bastante
firmeza.

---Nivel 7 ---decía---. Tiene que aislar el Nivel 7 ahora y recuperar
los recursos.

---No puedo hacer eso ---protesté.

---Los Protocolos dicen que es la única manera.

Sabía, más o menos, que tenía razón. El Nivel 7 era autónomo. Estarían
bien. Un poco. Pero simplemente no me gustaba la idea. Fue entonces
cuando las luces rojas se desvanecieron y la alarma se detuvo. Habíamos
escapado de nuevo. Después de que la alarma se detuviera, dos personas
fueron encontradas donde no deberían de estar. Uno era 428. Estaba sobre
el cuerpo de la Guardiana Donaldson.

---Sé lo que parece ---dijo 428 cuando Bentley le golpeó---. Estaba
tratando de encontrar la fuente de la alarma\ldots{} al igual que ella.

Pero Bentley continuó golpeándolo. Estaba llorando. Curiosamente, 428
puso sus brazos alrededor de ella y la abrazó.

---A mi también me gustaba bastante Donaldson ---dijo---. Siento tu
pérdida.

Visité a 428 en la celda de aislamiento. Más bien era un armario, pero
él parecía indiferente por su entorno.

---¿Cómo está Bentley? ---preguntó.

Negué con la cabeza.

---Eso no importa. 428, ¿por qué mataste a Donaldson?.

---No lo hice ---parecía molesto por la pregunta---. Ambos estábamos
buscando la fuente de la alarma. Me temo que ella tuvo más éxito que yo.
La encontró y la mató.

---¿El qué?.

428 se encogió de hombros. Se inclinó y me dio un golpecito en la nariz,
pero el muro de fuerza crujió amenazadoramente en torno a su mano.

---Hay algo en esta prisión que ni siquiera tú conoces, Alcaide.

---¿Esperas que crea eso?.

428 extendió sus brazos de nuevo.

---¿Tenía yo un arma?. Llegados a ese punto, y dado el estado de
Donaldson, ¿tenía yo todas las armas?.

Le fulmine con la mirada. Había momentos en los que no había lugar para
la falta de seriedad.

Le dejé en solitario. En lo que a mi respectaba podía pudrirse ahí.
Donaldson había sido religiosa. O, al menos, su familia lo era. Así que
sus creencias se habían manifestado en sus formularios de personal.
Según la costumbre, su cuerpo sería colocado en su féretro y descansaría
en la capilla durante la noche antes de ser dispuesto la siguiente
madrugada (Hora relativa de HomeWorld).

Alguien velaría a Donaldson durante la noche. Decidí que sería yo. Me
alegré de que el ataúd estuviera cerrado. Eso significaba que podía
sentarme y sentir lástima de mí mismo.

Realmente no conocía a Donaldson muy bien. Me había gustado. Actualmente
me sentía terriblemente responsable de ella. Pero eso era todo. Eso y un
terrible sentimiento de culpa que se extendería durante ocho horas. Hubo
una tos suave y 428 se deslizó en la silla junto a mí.

Por un momento le ignoré. Realmente no pude pensar en nada más que
hacer. 428 tampoco parecía tener prisa en hablar.

Nos sentamos allí, mirando a su ataúd en silencio. Me di cuenta, por la
angustia en su rostro, de que se sentía tan culpable como yo. No
culpable como si la hubiera asesinado, pero culpable como si fuera
terriblemente responsable por su muerte. Me pregunté si la culpa, su
culpa, mi culpa, mejoraría. Miré a 428, tratando de averiguar si aun se
sentía aplastado bajo la culpa de sus victimas. Si pudiera venir y
sentarse a mi lado\ldots{} bueno, debe de hacerlo, seguro.

Nos sentamos allí un rato más. Dejé de mirar mi reloj cada cinco minutos
y gradualmente el sosiego se apoderó de nosotros. El féretro de
Donaldson, Preso 428 y yo. Iluminado sólo por las velas, el aire
suavemente perfumado por las flores artificiales que había pedido a
partir de síntesis, y mucho más aún por las flores que 428 había traído.
Había encontrado lirios reales (no tengo ni idea de cómo), pero pronto
el aire apestaba a ellos.

De repente, me di cuenta que casi había amanecido. La noche se había
desplazado en su suave melancolía. 428 se levantó y se inclinó ante el
féretro.

---Doctor\ldots{} ---dije y entonces me detuve. No me atreví a darle las
gracias.

---Dile a Bentley que lo siento ---me dijo 428, su voz era suave---. Y,
para que lo sepas, es igual de fácil salir de aislamiento ---sonrió con
tristeza, me dio una palmada en el hombro, y se fue.

El Oráculo me llamó desde el Nivel 7. Su cara estaba recogida en una
comedia grasosa de pena.

---He oído que ha sufrido una tragedia ---entonó, meneando sus dedos---.
Muy lamentable. Muy triste. Muy\ldots{} púrpura.

---Ve al grano ---normalmente hacía un trabajo mejor en ocultar lo que
me irritaba el Oráculo.

---Oh, ¿estoy interrumpiendo? ---el Oráculo se golpeó una mano con la
otra---. Que tonto, debería de haber previsto eso\ldots{} no se me
ocurriría interrumpirle en una ocasión tan seria, solo\ldots{} que con
los tristes eventos\ldots{} el aura llega incluso hasta aquí. Tales
pensamientos tristes afectan la claridad de mis visiones.

No podía creerlo, el idiota nos estaba echando la culpa de trastear con
su falsa clarividencia. Ni siquiera era como si fuera realmente un
místico, simplemente era un cotilla bien informado. Pero tampoco era
inofensivo. Pensé que ponerlo a cargo del Nivel 7 fue una decisión de
bastante mal gusto.

Como si contestara a algún comentario de arrepentimiento, el Oráculo
alzó sus manos. ---¡Realmente no hay necesidad de pedir perdón por la
interferencia en mis visiones!. Es muy lamentable, pero estas cosas
pasan. Sin embargo, puede estar seguro, mi querido Alcaide, que no es
culpa suya ---por la forma en que lo dijo, claramente me echaba la
culpa---. Es simplemente que\ldots{} ---presionó sus dedos contra su
frente, hundiéndolos en la carne alrededor de las sienes---, ah, sí, que
lástima. Debería de ser capaz de ver claramente por delante, pero tal y
como es, con tanta interferencia hay una nube\ldots{} a través de un
velo de pensamientos de color burdeos\ldots{} ---se amasó las cejas, y
entonces asumió un semblante de piedad benevolente---. Yo puedo decirle
esto. Pronto tendrá que decidir el destino de Nivel 7. Lo lamentables
que somos ---agitó un dedo y finalizó el pitido.

Clara estaba de nuevo en la pista de aterrizaje. Y estaba sosteniendo
una nueva pancarta.

---Esta vez, la he hecho yo misma ---dijo con un encogimiento de
hombros---. Creo que si hubiera hecho a los niños hacerla hubiera
acabado en una lista. Y no en una buena lista.

---Ya veo ---dije mirando a la pancarta---. ¿Y de que manera esperas que
no interprete esto como una amenaza terrorista?.

---Oh ---ella miró la pancarta---. ¿Recibiste los documentos de la
última vez?.

Me rasqué la cabeza. Para ser honesto, muchas piezas de papel me
llegaron.

---No estoy seguro. No creo.

---Bueno, honestamente ---suspiró---. Es una conspiración.

---Quizás ---dije tratando de ser útil---. Tal vez solo estoy realmente
atrasado con el papeleo.

---¿En serio esperas que me crea eso?.

---Hemos tenido mucho en marcha ---dije. Podía oír lo cansada que sonaba
mi voz.

Ella puso los ojos en blanco.

---El Doctor es muy importante.

---Puede ser. Pero él es Preso 428. Tenemos otros 427 reclusos, los
cuales todos son igual de importantes para nosotros. Está siendo cuidado
y atendido de acuerdo con los Protocolos. Por favor créeme. Si tienes
alguna pregunta sobre el proceso judicial que lo trajo aquí, te animo a
que lo busques en HomeWorld.

---No puedo ---dijo Clara y miró sobre su hombro---. Realmente sólo
puedo conseguir venir aquí. A mi transporte no le gusto mucho.

---¿Perdón?. ¿Tu transporte?.

Clara hizo una mueca.

---Es temperamental. Solía odiarme. Ahora solo me tolera. A casa o aquí.
Eso es todo. Y no necesariamente en el orden correcto.

---Te entiendo ---no la entendía. Parecía estar hablando un galimatías.

---Honestamente ---puso los ojos en blanco---. Un día se averiará y
acabaré tirada en algún lugar. ¿Te imaginas que\ldots{}?. ¡Oh! ---se
tapó la boca avergonzada---. Disculpa.

---Para nada ---dije rígidamente---. Sigue hablándome sobre tus viajes.

---No, no ---me aseguró Clara---. Ya he acabado, me he puesto firme.
Podemos hablar de política si lo prefieres. El nuevo Presidente de
HomeWorld está resultando ser completamente impopular. Puedo hablar de
eso si quieres.

---Preferiría no hacerlo ---dije firmemente.

---Entonces, siempre está el tiempo\ldots{} ---Clara miró arriba al
cielo estrellado, como si esperara lluvia.

---Si no tienes nada de relevancia que decir ---estaba complacido de lo
agrio que sonaba--- quizás será mejor que te deje para que hables con
tu\ldots{} nave sobre hacer otros arreglos para llevarte a casa. Si me
disculpas, tengo una prisión que dirigir.

---Ya veo.

---Clara. ¿Te gusto? ---no sé por qué dije eso.

---¿Gustarme? ---parecía sobresaltada. Como si un tarro de mermelada le
hubiera preguntado cómo votar.

---No importa. Tengo que irme ---me levanté rápidamente, avergonzado.

---Antes de que lo hagas\ldots{} ---Clara tosió y agito su cartel
importantemente---. Mi cartel. ¿Te lo leo?.

---No, por favor. Puedo leer.

---Liberad. Al. Doctor. O. La. Matanza. Comenzará ---hizo una pausa---.
¡Ahí!.

Por alguna razón absurda parecía satisfecha. Yo solo gemí.

---Llegas demasiado tarde. La matanza ya ha comenzado ---dije y la dejé
ahí.

Su rostro se ensombreció. No creo que volviera a salir a verla de nuevo.
Si soy sincero, en el ascensor de vuelta a mi oficina, me sentí un poco
preocupado de cómo había dejado las cosas con Clara. Una pequeña
irritante voz me decía que sería mucho más fácil ser amable con ella.
Pero ya era demasiado tarde para eso, y ahora me esperaba mi oficina con
la mirada juzgante de Bentley. Entré y me desplomé sobre el escritorio.
Un Guardián me trajo té. No lo bebí. Un Guardián, posiblemente el mismo,
me trajo más té. Apenas toqué ese. Simplemente me quedé sentado viendo a
Clara en la cámara, de pie en la pista de aterrizaje, sosteniendo
pacientemente su pancarta. Con el tiempo, sus brazos se cansaron y se
frotó uno y después el otro. Entonces desfiló de arriba a abajo un rato.
Por último, puso la pancarta en el suelo, se enfadó, y se fue triste.