\chapter*{Capítulo Tres}
\addcontentsline{toc}{chapter}{Capítulo Tres}

La chica. Las visitas a la Prisión son raras, pero las tenemos de vez en
cuando. Contratan lanzaderas privadas, ocasionalmente desde HomeWorld,
más a menudo desde una de las colonias más tétricas, y vuelan todo el
camino hasta aquí. Hay una plataforma de aterrizaje. Nosotros nunca la
usamos. La plataforma fue diseñada específicamente para estar aislada
del resto de la prisión. Sabíamos que las visitas vendrían.

Algunas veces se presentará toda una familia. Una madre y un padre, un
marido, algunos niños. Algunas veces se quedarán de pie llorando en la
plataforma de aterrizaje. Algunas veces se quedarán de pie en silencio.
Esperando.

No existe regulación que se ocupe de las visitas. Los Protocolos se
limitan a indicar que, lamentablemente, a los presos no se les permite
tener visitas. Como cortesía, la primera vez que vienen, siempre saldré
a la valla que separa la plataforma de aterrizaje del resto de la
prisión. La valla es poco más que un símbolo de los setenta y tres
sistemas que no se pueden romper entre la plataforma de aterrizaje y la
prisión. Como todo lo demás, si así lo decido, bajo medidas ordinarias
puedo desactivar siete de esos sistemas para permitir que me pasen
objetos. Como por ejemplo, digamos, una petición. Normalmente es una
petición. No se permiten cartas ni regalos para los presos. Además,
tampoco puedo dar nada a las visitas. Ni siquiera yo puedo acceder a
todos los setenta y tres sistemas. Ni siquiera yo puedo permitir que
alguien inocente pase desde el interior de la Prisión al exterior.

Como ya he dicho, la primera vez que alguien viene de visita, siempre
saldré a su encuentro. Me parece humano. Algunas veces se quedarán allí
de pie clamando y gritando a través de la valla. Algunas veces hay
pancartas. Algunas veces, sólo uno de ellos dará un paso adelante y en
voz baja me hablará.

``¿Sabes quién soy?'' Les preguntaré. ``Nos gustaría hablar con
\textless{}insertar nombre del preso aquí\textgreater{}''. Educadamente
le responderé que, por desgracia, no es posible. ``Pero nos prometieron
comunicaciones TransNet con ellos'', insistirán. ``No hemos sabido nada
desde que llegaron. Simplemente queremos saber si \textless{}insertar
nombre del preso\textgreater{} está bien. Los queremos. Eso es todo.''

Asentiré gravemente y luego responderé: ``Les puedo asegurar que
\textless{}insertar nombre del preso\textgreater{} está absolutamente
bien y recibe el tratamiento aprobado. La red TransNet no está
funcionando actualmente con un ancho de banda suficiente para permitir
la comunicación entre los presos y los de HomeWorld. Les puedo asegurar
que el problema no está en nuestro lado. Les recomendaría que lo
planteasen ante las autoridades de HomeWorld. Me han dicho que los
problemas actuales están causados por el viento solar''.

Siempre me miran de un modo extraño cuando digo eso. Pero es lo que
Bentley me ha dicho y tengo que confiar en ella.

Entonces preguntan si pueden pasarme cartas para sus seres queridos. Me
disculparé y explicaré que lo único que se permite es la comunicación a
través de TransNet. Les diré que estoy atado por los Protocolos. Me
mirarán de una manera divertida otra vez. Y luego les pediré que me
pasen una petición llena de esperanza y de indescifrables firmas.

Nunca he entendido las peticiones. Gente de la que nunca has oído hablar
quieren que hagas algo. No hay nada que pueda hacer al respecto. Me
ocupo de los presos, de acuerdo con mi propia conciencia y con los
Protocolos. De todos modos, envío sus peticiones al Gobierno de
HomeWorld. Quizá nos sorprenderán a todos liberando a alguien, u
ordenándome que conceda a alguien privilegios extra. Pero nunca lo
hacen.

Se lo explico a las visitas pacientemente, y espero, amablemente, a que
me entregen una petición que, simplemente escanearé y enviaré por el
espantosamente lento TransNet retransmitido de nuevo al Gobierno. La
conexión que tienen en sus lanzaderas es sin duda mucho más rápida. Pero
insisten en que lo haga. Quizá les hace sentir mejor, así el viaje largo
y costoso ha merecido la pena. Si es así, lo menos que puedo hacer es
cogerlo y mirarlos seria y gravemente. Nunca me devuelven la mirada.

Así es como va una buena interacción. A veces incluso me dan las gracias
por mi tiempo antes de irse. He sido entrenado para hacer frente a
escenarios menos ideales. A veces me gritan. ``¿Cómo pudiste?. ¿Cómo
puedes vivir contigo mismo?'' gritan. Pero entonces, son preguntas que
nadie puede responder. Hacemos lo que hacemos y podemos vivir con
nosotros mismos. De alguna manera. Esa es la única respuesta que alguien
puede dar.

De todos modos, eso es un amplio resumen de cómo va una típica primera
visita. En la que tengo la cortesía de saludarlos.

Las demás veces que vienen, por lo general les dejo ahí fuera. Los
Procedimientos indican que sólo tengo que reunirme con ellos una vez.

Se quedan allí. Agitan sus pancartas. Se asoman con esperanza a través
de la valla. Nadie sale a su encuentro. Y, finalmente, se marchan.

Raramente acuden una tercera vez.

La chica, sin embargo. La chica sería diferente.

Llega sin ninguna fanfarria, está justo de pie allí en la plataforma de
lanzamiento. Curiosamente, la Matriz de Defensa no ha captado a ninguna
lanzadera acercándose. Ni siquiera hemos tenido el tiempo de encender
las luces de aterrizaje debidamente. Pero está bien, porque no parece
necesitar ayuda para aterrizar su lanzadera. Acaba de llegar, como si se
hubiera lanzado un hechizo.

Tampoco está vestida apropiadamente. No lleva un traje espacial, ni
siquiera un traje de vuelo. Sólo un jersey pasado de moda y una pulcra y
anticuada falda en su pequeño y resuelto cuerpo Ni siquiera lleva una
banda en el pelo. Recuerdo gente como ella. Hacía mucho tiempo que no
veía un Vintager. Pensé que habían muerto con la Vieja Nueva Tierra.
Curiosamente me recuerda al Preso 428. La misma sensación de que está
aquí pero que desentona.

Por supuesto, puedo imaginar que ha venido a ver a 428.

Con arreglo al Protocolo, salí obedientemente a la pista de aterrizaje.
Me estaba esperando. No estaba sosteniendo una pancarta, ni un montón de
aburridos pedazos de peticiones formales. Estaba sentada en un pedazo de
roca, leyendo un libro de papel de verdad. Cuando llegué a la valla,
fingió no notarlo por un momento, sólo permanecía leyendo, su nariz
arrugándose ligeramente mientras volvía una página. Luego marcó la
página (¡con un artefacto que no tiene precio!. La odié sólo un poco),
lo deslizó en el bolsillo y me miró con una sonrisa.

---Lo siento ---dijo---. Llegaba a la mejor parte. Así que \ldots{} Hola
---sonrió, educadamente---. ¿Puedo ayudarte?.

---Soy el Alcaide ---le dije, ya un poco descompuesto---. Es más cómo
puedo yo ayudarte, ¿no?.

---Bueno, si tú lo dices ---se encogió de hombros. Su paciente sonrisa
hacia su cara incluso más bonita.

---Estás con 428, ¿verdad?. ¿El Doctor?.

Asintió.

---¿Te gustaría verle?.

Asintió de nuevo.

---Bueno, me temo que eso no es posible.

---Ah ---parecía seria, su mano manoseaba el libro en el bolsillo---. He
recorrido un largo camino. Y realmente sería una buena idea si me
permitieses verle.

Y ahí estábamos. De vuelta a un terreno familiar.

---¿Es un pariente. ¿Su hija tal vez?.

Se rió ante eso, una completa risa gutural y horrorizada.

---Nunca le cuentes que has dicho eso. Te mataría.

Fruncí el ceño. Mencionó matando a 428, pero casi de manera casual. Como
si no estuviera al tanto de todo el horror que había hecho. O lo
ignorase intencionalmente. Traté de no permitir que me afectase.

---¿Entonces, tal vez eres su\ldots{} esposa?.

Frunció el ceño y entonces su rostro estaba claramente haciendo: ``Oh,
vamos''. Tristemente, conocía el tipo.

---Querida, lo siento por ti. Desafortunadamente, no eres la primera en
aparecer aquí en tu situación. Quizá has visto la cara del Doctor en los
TransCasts, o has leído sobre su juicio, y te has enamorado de él
---ignoré los chillidos de protesta que hizo---. Estás aquí porque estás
enamorada de él, y crees que, si te reunes con él, podrías reformarle.
Sé lo que tratas de hacer aquí ---negué con la cabeza tristemente---.
Estás aquí para salvarlo de sí mismo.

La chica lo consideró.

---Bueno, ahora mismo, creo que es un poco idiota. ¿Eso cuenta?.

Una vez más me quedé perplejo. No se estaba comportando como una fan
enamorada. Tendió una mano, los dedos no rozaban demasiado la valla,
sólo golpeaban el Sistema de Protección 3, el campo eléctrico. No los
quitó. Ni se inmutó.

---Empecemos de nuevo ---dijo---. Hola, soy Clara. Soy amiga de Doctor.
¿Qué te trae por aquí?.

---Soy el Alcaide ---le dije, haciendo una reverencia en señal formal de
saludo---. Estoy autorizado a saludar a cualquier persona en su primera
visita a la prisión con arreglo al Protocolo.

---¿Sólo en su primera visita? ---Clara levantó una ceja.

Asentí. Había habido algunos rumores de que yo tenía que salir a todas
las visitas, pero después de un tiempo parecía enfatizar la inutilidad
del ejercicio. Bentley me había asegurado que no tenía que hacerlo.
Estaba agradecido por esta inusual amabilidad hacia mí.

---Estoy obligado a salir y hablar contigo una vez. Después de eso,
bueno, eres bienvenida a volver tantas veces como desees. Pero esta es
tu única oportunidad de hablar directamente con el Alcaide.

El ceño fruncido de Clara se profundizó.

---Muy Bien. Así que, para que yo lo entienda, ¿no vas a liberar al
Doctor a pesar de que me he puesto mi falda más elegante?.

Negué con la cabeza.

---¿Y no hay absolutamente ninguna posibilidad de tener una breve charla
con él?.

Negué con la cabeza de nuevo.

---Bien, entonces ---Clara se encogió de hombros---. ¿Así que sólo somos
tú y yo? ---no parecía que le molestase---. Muy bien, aunque me siento
como la chica que encontró la lámpara que le concedió tres deseos. O
algo así. Sabes cómo es. Para mi siguiente deseo querría infinitos
deseos.

Sonrió. Le devolví la sonrisa, a mi pesar.

---Me temo que no conozco las fábulas de tu tribu.

---Oh, ¿no? ---la sonrisa de Clara se ensanchó. Había algo en esa chica
que estaba ganando bastante. No estaba tratando esto exactamente como
una broma, era más que me estaba tratando como a un ser humano. De
repente me di cuenta, hacía mucho tiempo que nadie lo había hecho.
Normalmente las visitas sólo me gritaban. Nunca parecieron darse cuenta
de que estoy obligado tanto por la ley como por mis responsabilidades,
hacia aquellos a los que consideraba mis amigos.

Caminó de un lado a otro por la plataforma de aterrizaje durante un rato
y después levantó la mano. Supuse que estaba acostumbrada a hablar en
público. Algo en sus maneras, un educador. Eso era. Tendían a ser unos
locos y fanáticos, aunque describir así a esta Clara como parecía algo
injusto.

---En resumen, ¿la única cosa que puedes hacer es escucharme y sólo
tienes que hacerlo la primera vez que visito?.

---Correcto.

---Pero tienes que escuchar ---sonrió, como golpeada por un pensamiento.

---¡Por supuesto!. Me parece lo menos que puedo hacer por los amigos de
mis amigos.

---Bien. Entonces tengo una fábula para ti ---dijo, agitando su pequeño
dedo hacia mi---. Trata de una mujer que era, ah, digamos, una especie
de Reina de Jordania. Y esta Jordán estaba decidida a conseguir
exactamente lo que quería en la vida. Y así se casó con un montón de
reyes. Y no importó con cuántos reyes se casó, nadie le dio exactamente
lo que quería. Uno era un, um, cantante. O algo así. Uno era un
guerrero. Uno huyó de vuelta a casa confuso. Y uno parecía muy majo con
un tanga. Creo que hubo otros reyes temporales, pero esos fueron los
principales. De todos modos, la cuestión es que ninguno de estos reyes
dio a la Reina de Jordania exactamente lo que quería, pero ella siguió
casándose con ellos mientras estaba decidida a conseguir lo que quería.
No iba a conformarse con nada menos que la perfección e iba a seguir
adelante, incluso si se le terminaban los reyes. Que parecía lo más
probable.

Consideré atentamente esta parábola.

---¿Y estás diciendo que eres como Jordán, Reina de Jordania?.

Clara asintió, mordiéndose el labio superior con determinación.

---De maneras que ni siquiera puedes imaginar ---juró. Se acercó a la
valla, la electricidad se desataba alrededor de su cara. Relampagueaba y
brillaba en sus ojos y parecía sombríamente seria---. Escúchame,
Alcaide. Continuaré viniendo hasta que hagas lo que digo. Suelta al
Doctor. O un montón de gente morirá ---y entonces sonrió con esa dulce
sonrisa, y se alejó.

Fanáticos.
