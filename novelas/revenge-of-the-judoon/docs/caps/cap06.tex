\chapter*{El desierto arábigo}
\addcontentsline{toc}{chapter}{El desierto arábigo}

Por un breve instante, el silencio en el desierto se vio interrumpido
por un sonido de caballos jadeando y gruñendo. Una cabina azul de
policía se materializó en lo alto de una duna de arena.

El Doctor salió, tambaleándose un poco por el terrible calor. Se dio
cuenta de que aquellos tweeds gruesos no servirían para el desierto,
volvió para cambiarse a uno de sus típicos trajes de rayas. Pero incluso
con aquello hacía demasiado calor.

Consideró volver de nuevo para ponerse unos pantalones cortos y decidió
no molestarse. Miró la escena a su alrededor.

A su alrededor había arena, milla tras milla de ella. Dunas con formas
extrañas con lados de formas extrañas, como sobre la que estaba de pie.
Zonas arremolinándose con complejos diseños recorriendo la arena. No
había nada excepto arena, en cada dirección.

Y allí, justo delante de él, al fondo de un gran vacío poco profundo en
el desierto, había un edificio enorme. O mejor dicho, varios edificios,
conectados por alas inferiores. Había una grandiosa torre cuadrada en
una esquina y una pequeña redonda más cercana.

~---El castillo de Balmoral ---murmuró el Doctor---. Oh, ¡eres un
excelente ejemplo de un exagerado gótico victoriano! ¿Así que, qué
demonios estás haciendo aquí?

En la lejanía, justo detrás del castillo, se alzaba un grupo de tres
columnas gigantescas de metal. Eran altas como columnas de piedra, y se
alzaban en unas pequeñas patas que se extendían hacia afuera de la base
de la columna. Eran las naves de los Judoon.

---¿Y por qué aquí? ---se preguntó.

Quizá los Judoon, con su corriente rectitud, habían escogido el espacio
vacío más grande del planeta y habían hecho aterrizar sus naves
espaciales y su castillo robado en el medio de ello. Con un chasquido
repentino, una bala silbó cerca de su cabeza. El Doctor se tiró contra
la arena. Lentamente, levantó la cabeza. ~Unos jinetes con ropas blancas
estaban merodeando por los lados del hueco. Gritaban, chillaban y
disparaban con sus rifles en el aire: eran jinetes del desierto,
llevando a cabo su típica carga ruidosa para aterrorizar a los enemigos.
Asaltar el castillo en el desierto debía de haberles parecido una gran
oportunidad para atisbar un botín.

Aquella vez, sospechó el Doctor, no tendrían suerte.

El jinete que les guiaba cargó hacia el castillo y salió rebotado,
repelido por una pared invisible. El golpe del impacto le sacó volando
del caballo. El caballo mismo se bamboleó hacia atrás y se cayó en la
arena.

Más y más atacantes se acercaron cabalgando, sólo para sufrir el mismo
destino. En nada, en la zona de delante del castillo había un enredo de
hombres caídos y caballos encabritados. Gritos, chillidos y maldiciones
resonaron mientras los furiosos jinetes se esforzaban en volver a
subirse a sus monturas y a sus lugares en las sillas de montar.

Se las arreglaron al final, y se reagruparon. Se pasearon por allí
durante un rato, aún gritando, chillando y soltando maldiciones. Pero
sus balas no servían para nada contra el campo de fuerza alrededor del
castillo. Finalmente, frustrados y furiosos, desaparecieron en el
desierto, en búsqueda, sin ninguna duda, de una presa más fácil.

El Doctor esperó hasta que las últimas galopadas dejaron de oírse, se
levantó y se abrió camino hacia el castillo. Tenía que encontrar alguna
forma de entrar dentro. Algo mejor que atacar en caballo, gritando y
disparando un rifle.

~«Aún así, tendría buena pinta», pensó mientras atravesaba la arena, «Un
ataque de la caballería a la antigua. Salido justo de Lawrence de
Arabia, o más bien de Carry on up the Khyber- O posiblemente de\ldots{}
Spamalot\ldots{}».

~

Martha y Carruthers esperaban al Pacificador Cósmico que les había
invitado a entrar. Se había presentado a sí mismo como Profesor
Challoner, y se había ido en cuanto lo preguntaron por el apartaro.

Para Martha casi parecía como si se hubiera movido del pasado, los
cocheros de cabriolé y los bares del Londres eduardiano, al futuro. La
sala en la que estaban sentados estaba amueblada toda de blanco. El
mobiliario era simple y liso, y las paredes de cristal brillaban con una
luz interna.

El profesor Challoner volvió a la sala, con su delgada cara llena de
preocupación. Tres siluetas vestidas con ropas blancas le seguían, todas
altas y delgadas y con la misma expresión.

---Estos son mis colegas. Están ansiosos por escuchar lo que tenéis que
decir. Perdonad el retraso. Tenía algo muy importante que atender.
Ahora, el aparato provocó un desastre, ¿decís? Me aseguraron que era
inofensivo. De otra manera, por supuesto, nunca se lo habría dado a su
amigo Doyle. Por favor, dígame qué ha pasado.

Carruthers había evitado cautelosamente cualquier mención del
desaparecido castillo de Balmoral, y no estaba muy seguro de qué decir.
Miró con esperanzas a Martha.

---No estamos muy seguros ---dijo Martha---. Los informes de Escocia
siguen siendo un poco confusos y nadie parece saber exactamente qué está
pasando. Lo que necesitamos saber, profesor, es lo que es el aparato.
¿Para qué sirve? Y lo que es más importante ¿dónde lo conseguisteis?

Challoner suspiró.

---Sólo deseo ser de ayuda. El propósito, como yo lo entiendo, es actuar
como algún tipo de fuente de energía. Un gran poder almacenado en un
espacio muy pequeño.

---¿Y de dónde viene? ¿Lo han desarrollado aquí, en estos laboratorios?
---preguntó Carruthers.

---No. Ha venido de fuera. Lo ha inventado, lo ha descubierto, uno de
nuestros colegas del extranjero. El aparato vino de un monasterio en el
Tíbet, una gran fuente de una sabiduría antigua.

---¿Pueden contactar con esos colegas? ---preguntó Martha---. Hacer que
nos digan lo que necesitamos saber.

---Naturalmente, lo haré inmediatamente ---dijo el profesor
Challoner---. De cualquier manera, puede llevarnos un tiempo. Es difícil
llegar a tan remota parte del mundo\ldots{}

Martha asintió.

---Lo entiendo. Por favor hagan lo que puedan. No queremos quitarles
demasiado de su tiempo.

Se levantaron y el profesor Challoner les mostró la muerta. Martha se
detuvo.

---Me pregunto si podemos dar una vuelta por sus laboratorios antes de
partir.

Challoner negó su cabeza con tristeza.

---Me temo que no será posible. Nuestro trabajo aquí es extremadamente
avanzado. Sería desconcertante y posiblemente peligroso. ¿Me acompañan
por aquí?

Minutos más tarde, estaban de nuevo subiendo los sucios escalones que
llevaban de nuevo a Black Dog Lane.

---¿Qué sacamos de esto? ---preguntó Carruthers mientras llegaban al
callejón.

---Estaba entreteniéndonos ---dijo Martha---. Monasterio en el Tíbet,
las ganas.

---Bueno, no hemos aprendido mucho más, ¿verdad?

---Oh, no lo sé ---dijo Martha---. Como dice el Doctor: conoce a tu
enemigo. Y al menos ya le hemos visto. Y a sus cuarteles generales. La
sala en la que estábamos sentados\ldots{}

---¿Qué le pasa?

---¿Has visto esa iluminación antes?

---Bueno, no. Pero no soy un experto en las nuevas formas de iluminación
eléctrica.

---Esa sala no pertenece a este tiempo. Y había algo más\ldots{}

---¿Qué?

---¿Recuerdas la cara que apareció tras la escotilla e intentó hacernos
retroceder?

Carruthers se estremeció.

---El rostro de la muerte, como mínimo.

---Entonces la puerta se abrió y estaba el bueno de Challoner.

---Bueno, no puede haber sido el mismo hombre ---dijo Carruthers---.
Deben de tener un horrendo amo de llaves que fue a por él. El amo de
llaves se fue y Challoner apareció.

Martha negó con la cabeza.

---No hubo tiempo.

---¿Pero eso qué significa?

Martha se encogió de hombros.

---No lo sé. Pero puede significar que Challoner pueda cambiar de forma.
¡De horrible a buenazo en un segundo!

Dejaron Black Dog Lane y caminaron hacia el lugar de dónde habían bajado
del carro.

---¿Y ahora qué? ---preguntó Carruthers---. ¿Te vas a poner en contacto
con el Doctor?

---Aún no. Volvamos y veamos a tu amigo sir Arthur. Nos dijo que estuvo
pasando tiempo con este grupo de los Pacificadores Cósmicos. Debió de
haber aprendido algo sobre ellos, y no creo que nos dijera todo lo que
sabe.

Tuvieron que caminar durante un tiempo antes de que llegaran a una zona
donde pudieron encontrar otro carruaje. Cuando uno apareció al final,
Carruthers lo paró.

---Conductor, al Gran Hotel.

~

Salieron del carro, y Carruthers pagó al conductor. Mientras se giraban
para entrar en el hotel, vieron al mismo ~sir Arthur Conan Doyle
saliendo.

---¡Arthur! ---le llamó Carruthers---. ¡Justo el hombre que
necesitamos! Me alegro de haberte cogido.

Conan Doyle le miró asombrado.

---¿Harry? ¡Creía que seguías en Escocia! ¿Y quién es tu encantadora
joven amiga?

Carruthers le observó asombrado.

---Ya sabes que no estoy en Escocia. ~Vinimos a verte hace tan solo una
hora.

---No es cierto, querido joven---dijo Conan Doyle---. No seas tonto. Tú
sabes que no nos hemos visto en casi un mes.
