\chapter*{El tipo de Sherlock Holmes}
\addcontentsline{toc}{chapter}{El tipo de Sherlock Holmes}

El Capitán Carruthers estaba recuperándose de su shock. Aún confuso, se
centró en tomar la delantera en aquella situación.

---¿No sería mejor dar la alarma sobre todo esto? ---señalando al cráter
abierto.

---De hecho, no ---dijo el Doctor---. Más que nada porque no hace falta.
Incluso en las remotas Highlands de Escocia, alguien se acabará dando
cuenta de que falta un castillo tras un tiempo. Pero con suerte, tardará
bastante tiempo en pasar. Por el momento, cuanto menos revuelo mejor.

---Si tú lo dices, Doctor ---dijo Carruthers---. Bueno, si vamos a
volver a Londres, será mejor ponernos en marcha. Hay un tren que\ldots{}

---Eso tarda demasiado ---dijo el Doctor.

---¿Entonces qué?

---Yo os llevo ---dijo el Doctor---. Os puedo llevar mucho más
rápidamente.

---¿Más rápido que un tren?

---Mucho más rápido ---dijo Martha.

---Ya sé, ¿no tendrás una de esas nuevas máquinas voladoras? ¿Cómo los
hermanos Wright en Estados Unidos? ¿No hay un tipo francés llamado
Bleriot que jura que va a sobrevolar el canal? ¿Eres piloto, Doctor?

---Lo fui, pero por moda ---dijo el Doctor---. Soy pionero, aún así.
Vamos, el tiempo corre.

Mientras bajaban por la colina, Martha sonrió para sí. Al capitán Harry
Carruthers esperaba otro shock.

De hecho, Carruthers aceptó el viaje de vuelta a Londres en una cabina
de policía que aún no se había inventado, con bastante calma. Era como
si su batería de sorpresas se hubiera gastado del todo. Incluso el hecho
de que la TARDIS fuera más grande por dentro que por fuera no le había
sorprendido demasiado. Miraba alrededor de la compleja sala de mandos y
decía secamente:

---Es impresionante.

El Doctor se apremió en toquetear los controles de la consola de control
durante un tiempo.

---Perdón por tardar tanto ---les dijo después de un rato---. Lo creáis
o no, estos viajes cortos son más complicados que los largos. Plutón
sería más fácil que Picadilly. Luego está el factor del tiempo, no
queremos llegar antes de que salgamos ---miró hacia el vacío. Entonces
negó con la cabeza---. No, definitivamente no.

Al fin, el Doctor rió y la columna central comenzó a subir y a bajar.
Unos cuantos minutos más tarde, la TARDIS estaba en una esquina de una
calle silenciosa justo al lado de Picadilly Circus.

El Doctor abrió las puertas, y Martha se asomó. Siempre sentía aquella
ola de emoción descubriendo dónde estaban.

Estaba claro, estaban en las familiares calles londinenses, pero no
pobladas de coches sino de buses arrastrados por caballos y carruajes de
cabriolé. No había estación de metro (no se abriría hasta 1906, en unos
cuantos años) y no había carteles de neón. Pero estaba el Eros. Era
obviamente Picadilly Circus.

---Aquí os dejo, pues ---dijo el Doctor---. Buena suerte. No lo olvides,
Martha, llámame en cuanto puedas.

Dejaron la TARDIS y la observaron desaparecer. Había venido y se había
ido tan rápido que nadie pareció darse cuenta. O si lo habían hecho, no
parecieron creer lo que habían visto.

Carruthers levantó la mano y llamó algún carruaje de cabriolé.

---Mount Street por favor, ¡conductor!

---¿Dónde estamos yendo? ---preguntó Martha.

---A mis habitaciones.

---¿Por qué?

---Mi querida joven, no puedo ir dando vueltas por la ciudad en tweed de
campo.

A pesar de las protestas de Martha, Carruthers no dio su brazo a torcer.
Entonces, cuando llegaron a sus habitaciones, intentó hacer que Martha
esperara fuera.

---Mi mayordomo está de vacaciones, y no te puedes quedar sola conmigo
en mis habitaciones. ¿Qué hay de tu reputación?

---Sí bueno, tampoco no es que vayamos a encontrar a nadie que me
conozca.

Martha esperó en la pequeña salita de estar de Carruthers hasta que
apareció con un traje oscuro, completado con un sombrero, un bastón y
unos guantes.

---Muy bonito ---dijo Martha---. ¿Ahora podemos ir a buscar a tu amigo
Doyle? Es urgente, ya lo sabes.

---Podemos intentarlo en el Gran Hotel en Northumberland Avenue. Casi
siempre se hospeda allí.

~

El carruaje de cabriolé arrastrado por caballos les dejó en el exterior
de la entrada de un gran hotel.

---Ahí tiene, caballero ---dijo el conductor---. El Gran Hotel.

Harry Carruthers salió y ayudó a Martha a bajar. Pagó al conductor, que
se tocó el sombrero, hizo chasquear el látigo y se alejó. Carruthers se
giró hacia Martha.

---Como he dicho, normalmente se queda aquí cuando está en Londres. Pero
si no, hay un par de otros lugares donde podemos intentarlo.

La acompañó al interior. Aún así, tuvieron suerte: Doyle se hospedaba en
el hotel. Carruthers les dijo a los de recepción que él era un viejo
amigo de Doyle, y que había ido por un asunto urgente. Insistió en que
le acompañaran a su habitación.

Les acompañaron hasta lo alto de una pequeña escalera, hasta una puerta
de una suite del primer piso. Entraron en una salita de estar amueblada
elegantemente. En el mismo momento, un hombretón con un gran bigote
salió a través de las puertas de la habitación. Éste saludó con la mano.

---¡Harry! Creía que seguía en Balmoral.

---Lo estaba, hasta hace poco. Permítame presentarle a mi amiga, la
señorita Jones, una visitante de\ldots{} ---se interrumpió, dándose
cuenta de que no tenía ni idea de dónde era Martha---. Eh, la señorita
Jones, Sir Arthur Conan Doyle, el famoso creador de Sherlock Holmes.

Sir Arthur frunció el ceño.

---Y lo que es más importante, eminente doctor y un famoso novelista
histórico ---añadió Carruthers rápidamente.

Sir Arthur sonrió y se inclinó.

---Es un honor, señorita Jones. Por favor, siéntense.

Doyle se sentó en un gran sillón de cuero, Martha y Carruthers en un
sofá de terciopelo.

---Felicidades por el título de caballero, Sir Arthur ---dijo
Carruthers---. Un guiño real, como diría usted.

---¿A qué se refiere?

---Dicen que Su Majestad es un gran fan de Sherlock Holmes, que quiere
más historias de usted.

---El honor no ha tenido nada que ver con Holmes ---protestó Doyle---.
Ya le he matado en una de mis novelas, como sabrá. Distraía la atención
de mis trabajos como novelista histórico. El título de caballero fue por
escribir La guerra en Sudáfrica: sus causas y conductas. Y por supuesto
por mi trabajo médico ---se giró hacia Martha---. ¡Intenté que me
metieran en la lucha pero no quisieron admitirme! Así que fui como
médico.

Martha sintió que era hora de ponerse con el trabajo.

---Hemos venido para verle porque nos encontramos en un asunto muy
urgente, sir Arthur. Necesitamos su ayuda.

---Por supuesto, querida. Cualquier cosa que pueda hacer.

---¿Recuerda visitarme en mis habitaciones, Arthur, justo antes de que
partiera a Balmoral? ---preguntó Carruthers---. Me dio un dispositivo,
¿una esfera de cristal con una luz brillante en su interior?

Doyle miró a Martha.

---Ese es un asunto privado, capitán Carruthers.

---La señorita Jones tiene toda mi confianza ---dijo Carruthers. Tuvo
una idea repentina---. La señorita Jones es la ayudante de un científico
renombrado, un tal doctor Smith, que está muy interesado en su
dispositivo.

Carruthers se giró a Martha en busca de ayuda.

---El Doctor cree que el aparato tiene un gran potencial como fuente de
energía ---dijo ella solemnemente---. Pero quiere saber más sobre ello,
como el lugar de dónde lo consiguió.

---Es cierto ---dijo Carruthers---. ¿Quién se lo dio? ¿De dónde vino?

Doyle parecía preocupado, y su gran silueta se removió, incómoda.

---No tengo la libertad de decírselo. Me dijeron\ldots{}

---A no ser que confíe en nosotros, el asunto no puede continuar ---dijo
Carruthers, inflexible---. Los beneficios del aparato para la humanidad
se perderán.

Doyle parecía dudoso.

---Solo díganos quién se lo dio ---apremió Martha---. Eso es todo.
Entonces podemos ir y preguntarles sobre ello. Dependerá de ellos qué
nos dirán entonces. Por favor, sir Arthur. ¡Debe decírnoslo!