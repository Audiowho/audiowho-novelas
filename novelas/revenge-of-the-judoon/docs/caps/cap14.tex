\chapter*{Retailación}
\addcontentsline{toc}{chapter}{Retailación}

Dentro de la TARDIS, el Doctor estudiaba la pantalla de nuevo. Esta vez
mostraba un mapa de las Docklands de Londres, la Black Dog Lane.

La mano del Doctor se movió rápidamente por encima de los controles y un
pulso de luz apareció en la pantalla.

---¡Ahí estás! ---murmuró el Doctor---. Un generador de reversión
temporal, si alguna vez he visto uno. Me pregunto de dónde lo habrá
cogido. Muy avanzado para Challoner ---se movió rápidamente alrededor de
la consola de la TARDIS, ajustando los controles y comprobando las
lecturas---. Si pudiera cerrar el pulso\ldots{} difícil, vaya que sí,
muy difícil ---el Doctor se detuvo y sonrió---. Bueno, pero no en vano
soy un genio\ldots{}

Sus dedos corrieron por los controles.

~

En su laboratorio, Challoner estaba muy ocupado, trabajando en la
consola tras la gigantesca esfera de cristal. La esfera misma pulsaba
con luz, brillando cada vez más.

El líquido del bol por debajo burbujeaba y se removía.

Justo detrás de él, apoyada contra la pared, Martha le miraba con
impotencia.

En el instante que llegaron, Challoner le había agarrado su cuello con
sus largos y huesudos dedos. ~Apretó hasta que Martha estuvo
semiinconsciente. Entonces cogió un rollo de cinta de plástico de un
cajón y ató sus muñecas y sus tobillos. Cuando ya estuvo de nuevo
consciente, estaba atada tan firmemente que no se podía mover.

Se pregunto por qué Challoner no la había matado. Quizá necesitara
público. O quizá tenía miedo de que el Doctor les pudiera encontrar.

---La energía está creciendo de maravilla ---dijo Challoner---. Cuando
esté en su punto álgido, activaré los aparatos temporales a control
remoto. París, Berlín, Tokio, Moscú, todas las mayores ciudades del
mundo dejarán de existir.

---¿De qué te serviría eso? Los Judoon se han ido y el Doc\ldots{}

---Reclutaré otros soldados, guerreros Schlangi, o quizá Ogrons. Cuando
sea el amo, perseguiré al Doctor y le mataré ---se detuvo---. No, te
mataré a ti delante de él ---decidió Challoner---. Y entonces le mataré.

Martha decidió que la impresión de ser vencido le había trastocado el
cerebro.

---No tienes que preocuparte por encontrar al Doctor ---dijo ella, con
valentía---. Yo me preocuparía de que él te encontrara a ti.

Como para reafirmar sus palabras, la TARDIS se materializó, y el Doctor
salió de ella. Miró a la esfera de cristal palpitante.

---¡En el blanco! A ver, si te soy sincero, que lo soy, todos estos
controles son impresionantes. ¿No crees? ---miró a Martha---. ¡Es algo
maravilloso! Oh, ¿estás un poco liada? ---se giró hacia Challoner---.
Ahora creo que es hora de que desenchufes ese pedazo de chatarra antes
de que algo vaya terriblemente mal.

---¿Eso es lo que crees, Doctor?

---No lo creo, lo sé.

---Te equivocas ---dijo Challoner---. La energía casi se ha construido,
y en unos momentos activaré los controles remotos.

---No si se sobrecarga.

---¿Y qué va a hacer que se sobrecargue, Doctor? ---preguntó Challoner.
Cogió una pistola alienígena y la apuntó al Doctor.

---¡Yo mismo! ---dijo el Doctor.

Martha se dio cuenta de que el Doctor tenía su destornillador sónico en
la mano. Lo señaló hacia la consola y su punta brilló con una luz azul.

La consola comenzó a arder con humo y llamas. La gigantesca esfera de
cristal comenzó a palpitar con violencia.

Con un grito de furia, Challoner apuntó su pistola hacia el Doctor.

Aunque sus muñecas y sus tobillos estaban atados, Martha estaba a
escasos metros de Challoner. Se apoyó contra la pared, y entonces se
lanzó contra él, golpeando a Challoner con todo su cuerpo.

Challoner se tambaleó hacia atrás. El borde del bol bajo la enorme
esfera le golpeó por detrás de las rodillas. Se tambaleó por encima de
ello, cayendo hacia el líquido burbujeante.

El Doctor observó el bol. Vio un atisbo de una forma de lagarto
retorciéndose. Entonces se disolvió.

---¡Cuidado! ---gritó Martha, mirando hacia arriba.

La esfera de cristal brilló con fuerza. Entonces explotó, haciendo
llover fragmentos de cristal por el laboratorio.

El Doctor agarró a Martha y la puso a salvo. Sacó un cortaplumas de su
bolsillo, cortó sus ataduras y la ayudó a levantarse.

---Ya dije que todo podría ir mal ---le dijo.

Martha se quitó el resto de la cinta de sus muñecas y tobillos. Miró
hacia la maquinaria destrozada.

---¿Es seguro ahora?

Antes de que el Doctor pudiera responder, la consola brilló brevemente.
Entonces la consola, el líquido y los fragmentos de cristal simplemente
desaparecieron.

---¿Qué ha pasado?

---Retaliaciones temporales ---dijo el Doctor---. Se ha prohibido a sí
mismo. No sólo no existe ahora, sino que nunca lo ha hecho.

---¿Y Challoner?

---Él tampoco. ¡Ahora vámonos, Martha, parece que tenemos una agenda muy
apretada por delante!

Martha estaba indignada.

---¿Qué? ¡Creía que ya había acabado todo! ¡Estaba buscando un poco de
la vida grácil que me habías prometido!

---Podrás tenerlo ---dijo el Doctor---. Pero no en Londres.

---¿Cuándo entonces?

---¡Siempre!

---¿De qué hablas?

El Doctor extendió su brazo por el laboratorio en ruinas.

---Este puede ser ahora inofensivo. ¿Pero qué hay de esos aparatos de
reversión temporal que iba a activar? No queremos que los humanos
pongan las manos en la ingeniería temporal demasiado pronto. Y supón que
comienzan a crearla ellos mismos. No, tenemos que rastrearlos.

---Ya veo.

---¿Te apetece una visita relámpago a las capitales del mundo, Martha?
París, Roma, Berlín, Tokio, Moscú, Pekín\ldots{}

Abrió la puerta de la TARDIS.

---Pero ahora aclaremos esto, Doctor. Me estás ofreciendo una visita a
toda pastilla por el mundo, del palo ``si es viernes, debemos estar en
China''. ¿Con aparatos alienígenas e inestables, como atracción añadida,
por ser descubiertos y destruidos en cada ciudad?

---Bueno, si lo dices así\ldots{} pero de cinco estrellas. De cinco
estrellas definitivas.

---¿Un tipo de vacaciones de un equipo de bombarderos?

---De alguna manera\ldots{}

Martha no respondió. Se giró para que el Doctor no pudiera verla sonreír
y entró en la TARDIS.

El Doctor la siguió:

---Vamos, Martha. Te encantará Pekín y en cuanto a Moscú\ldots{}

La puerta se cerró por detrás de ellos.

La TARDIS crujió y gruñó y comenzó su viaje\ldots{}