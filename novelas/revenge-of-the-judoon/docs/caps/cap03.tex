\chapter*{A la caza de un castillo}
\addcontentsline{toc}{chapter}{A la caza de un castillo}

Harry Carruthers les lanzó una mirada de asombro.

---¿Qué?

Martha miró al Doctor.

---Judoon ---dijo de nuevo---. Tiene que ser, ¡verdad? ¿Doctor?

---La técnica parece familiar ---coincidió el Doctor---. Los Judoon, o
una banda de trabajadores muy eficientes con una grúa muy, muy, muy
grande.

El Doctor y Martha se habían cruzado con los Judoon con anterioridad. De
hecho, fue así cómo se conocieron. El hospital donde Martha había sido
estudiante de medicina había sido llevado a la Luna por los Judoon. Se
habían llevado a Martha y al Doctor con ellos.

En los eventos peligrosos que siguieron, Martha había aprendido que los
Judoon eran algún tipo de policía de alquiler. ~Se habían llevado el
hospital en su caza de un cruel multiforme Plasmavore, un tipo de
vampiro espacial.

Al final, como la caballería canadiense, los Judoon habían encontrado a
su hombre, o más bien a su anciana. El Plasmavore se había disfrazado de
una paciente anciana.

En el proceso, habían arriesgado las vidas del Doctor, de Martha y de
todo un hospital lleno de doctores, enfermeras y pacientes. Los Judoon
eran despiadados, decididos en su caza de lo que veían como justicia.

El Doctor estaba sacando un aparato parecido a una linterna de su
bolsillo. Ajustó los controles y entonces lo zarandeó hacia el cráter.
El aparato vibró y sonó.

---Trazos de bucles de plasma ---dijo el Doctor---. Son los Judoon sin
ninguna duda.

Apagó el aparato y lo guardó. Harry Carruthers, lentamente saliendo del
shock, miró a ambos una y otra vez.

---Mirad, ¿quiénes sois? ¿Qué está pasando? ¿Y qué demonios son esos
Judoon?

El Doctor suspiró. Explicar siempre era la parte difícil.

---Soy el Doctor y esta es Martha. Por el momento, no tengo ni idea de
qué está pasando. Y los Judoon\ldots{} ---se detuvo---. Digamos que son
una raza alienígena con la habilidad de causar lo que está pasando aquí.

Carruthers negó con la cabeza.

---Pero es imposible\ldots{}

El Doctor señaló hacia el cráter debajo de ellos.

---Pero está pasando ---dijo amablemente---. No tienes que creerme, pero
debes creer a tus propios ojos.

Carruthers se esforzaba en entenderlo.

---Has dicho una raza alienígena. ¿Quieres decir del espacio exterior?
¿Como en los libros del tal Wells? ¿``La guerra de los mundos'' y todo
eso?

---Muy parecidos ---coincidió el Doctor---. Excepto que los Judoon son
como unos rinocerontes gigantescos más que como unos pulpos gigantescos.

---Pero eso sólo fantasía ---dijo Carruthers---. Sólo imaginación.

---Oh, vamos ---le espetó el Doctor---. Hay cientos de planetas en la
galaxia y millones de galaxias. ¿De verdad te imaginas que tu pequeña
Tierra es la única que tiene formas de vida inteligente? ---girándose,
estudió el cráter---. Para llevarse algo de ese tamaño, deben de haber
necesitado algún tipo de baliza de plasma para concentrar la energía. De
alguna manera deben de haber colado alguna dentro del castillo ---se
giró de vuelta a Carruthers---. Piensa detenidamente. ¿Has visto algún
tipo de aparato de alta tecnología alienígena dentro del castillo?

---No lo creo. ¿A qué se parece? ¿Algo grande?

---No tiene por qué. Una esfera de cristal probablemente, del tamaño de
una pelota de criquet. Debe de haber algún tipo de de vórtice de energía
giratoria, como un tipo de torbellino.

Carruthers parecía asombrado.

---Puedo decirte exactamente dónde estaba, Doctor. Estaba en mi
equipaje. Yo lo metí dentro del castillo ---miró dentro del cráter donde
el castillo estuvo una vez. Agarró el brazo del Doctor y dijo
incontroladamente:

---¿No lo veis? Yo soy el responsable. ¡Yo soy el responsable de esto!

El Doctor se liberó de su brazo amablemente.

---Capitán Carruthers, usted no es responsable de esto, sino los Judoon.
Los Judoon y quien quiera que esté detrás de ellos.

---¿Qué quieres decir con detrás de ellos? ---preguntó Martha.

---Los Judoon no actúan en solitario ---dijo el Doctor---. Alguien les
ha contratado, les ha alquilado para hacer esto.

---¿Pero por qué?

---Es una de las cosas que tenemos que descubrir ---dijo el Doctor.

---¿Vosotros? ---dijo Carruthers, incrédulo.

---Oh sí ---le dijo el Doctor.

---Pues claro ---añadió Martha.

---¿Quién más está cualificado? ---preguntó el Doctor---. De cualquier
manera, nunca me gusta que la gente toquetee edificios antiguos. Y
tenemos una razón mucho más egoísta.

---¿Cuál? ---dijo Carruthers.

---Lo que ha pasado ha puesto todo el futuro de Martha en peligro, a sus
amigos y familia también.

---¿Y eso?

---Una gran interferencia en el espacio tiempo ---dijo el Doctor---. Sin
el rey Eduardo VII no hay Jorge V, sin el cual no hay Jorge VI.

Carruthers comenzó a farfullar, pero el Doctor siguió de todas formas.

---Sin Isabel II, dios la bendiga, no habrá el rey Carlos III y la reina
Camila, ni tampoco el rey Guillermo V, bueno aún no habéis llegado ahí,
¿verdad? Bueno, ya sabéis qué quiero decir.

---No, exactamente no ---dijo Martha---. ¿Y la familia real me afecta a
mí y a mi familia?

---La vida de todo el mundo será afectada ---dijo el Doctor---. Toda la
historia de la raza humana cambiada. Y este puede ser sólo el principio
de ello. Imaginaos castillos desapareciendo por toda Bretaña. Cualquier
cosa podría suceder o no suceder. Tu madre puede que nunca hubiera
conocido a tu padre y que tú nunca hubieras nacido. Tenemos que
arreglarlo.

---No hay nada que hacer con una persona completamente incapaz de
entender nada de lo que está pasando, ¿supongo?

---Ciertamente no ---dijo el Doctor---. Un Señor del Tiempo tiene que
hacer lo que un Señor del Tiempo tiene que hacer ---el Doctor le apretó
el hombro con amabilidad---. Y será divertido. Quiero decir, ¿no tienes
curiosidad, aunque sea sólo un poco?

Harry Carruthers les observaba con una mirada alocada a ambos.

---¿Qué demonios estáis hablando vosotros dos? Tenemos que hacer algo.

---Oh sí ---dijo el Doctor---. Y tú nos puedes ayudar.

---¿Puedo? ¿Cómo?

---Has dicho que trajiste la baliza de plasma al castillo. La esfera de
cristal con el vórtice de remolino. Y no la encontraste tirada en el
barro, ¿verdad? Así que, dinos, ¿dónde la conseguiste? ¿Quién te la
dio?

---Mi doctor ---dijo Carruthers---. Cielo santo, otro doctor. Este
negocio está lleno de doctores. Bueno, de hecho ya no es mi doctor. Era
mi doctor de cabecera, pero dejó la medicina hace unos años. Ahora es
escritor.

---¿Nombre? ---preguntó el Doctor, lleno de paciencia.

---Arthur. Arthur Conan Doyle.

---¿El tipo de Sherlock Holmes? ---dijo Martha.

---Así es.

---¿Arthur Conan Doyle? ---dijo el Doctor---. ¿Cómo es posible que
Arthur Conan Doyle te diera el dispositivo? ¿Por qué te dio el
dispositivo? Quiero decir, ¿de dónde habrá sacado el dispositivo?

Carruthers se encogió de hombros.

---Se pasó por mis habitaciones la noche antes de que viniera hacia
Balmoral. Parecía muy preocupado, ~y me sorprendió verle. Nos habíamos
visto una vez o dos antes de que dejara la medicina, pero no teníamos
una relación cercana.

---¿Y bien? ---el Doctor le apremió con la mano.

---Había hecho esta esfera de cristal con un tipo de luz arremolinada en
su interior. Dijo que era un descubrimiento científico importantísimo,
una posible fuente de energía de gran poder. Me rogó que la llevara a
Balmoral, para enseñársela al Rey. Quería que le persuadiera para
enseñársela a sus colegas científicos de la Academia de las Ciencias.
Dijo que estaba escribiendo un informe explicándolo todo, pero que
estaba desesperado por poner el dispositivo en manos seguras.

---¿Y accediste?

---Bueno, al principio no. Pero insistió y parecía frenético. Y, bueno,
la cosa parecía bastante inofensiva. Parecía un tipo de bola de cristal
rellena de nieve falsa. Creí que podría divertir al Rey. Así que acepté
y me la llevé.

---¿Qué pasó después?

---Me hizo jurar que no se lo contaría a nadie antes de irse.

---¿Dijo de dónde había conseguido la bola de cristal? ---preguntó
Martha.

---No me dijo ni media palabra. Para ser sincero, no estaba bastante
interesado como para preguntar. Parecía un truco de espectáculos, como
he dicho, una diversión. La metí en la maleta y me olvidé de ella.

---¿No te sorprendió todo esto? ---preguntó Martha.

---Bastante.

---Bueno, a mí no ---dijo el Doctor---. Le conocí una vez. Por fuera, es
un tio serio con un gran bigote. El típico caballero victoriano. Pero
hay una parte de él que adora a los fantasmas, los elfos y las hadas.

---¿Y los alienígenas? ---sugirió Martha.

---Podría ser ---dijo el Doctor---. Podría ser el títere de algún
alienígena con una excusa perfecta.

---¿Y qué hacemos ahora? ---preguntó Martha.

---Oh, creo que es bastante obvio.

---¿Lo es?

---Capitán Carruthers, necesitaré su ayuda.

---Por supuesto, cualquier cosa que pueda hacer ---dijo Carruthers.

---Lleve a Martha de vuelta a Londres. Encuentre a Conan Doyle y hacedle
deciros todo lo que sepa sobre la esfera de cristal. Luego volved y
decidme lo que os haya dicho.

---¿Y qué harás tú mientras? ---preguntó Martha.

---Tengo un castillo perdido que rastrear ---le dijo el Doctor.

---¿Y luego?

---Me meteré dentro del castillo, y haré el recorrido turístico. Quizá
me pase por la tienda de regalos y me compre la camiseta. Y luego
descubriré qué está pasando.

---Espera un minuto ---dijo Martha---. Voy contigo. El capitán
Carruthers puede encontrar a Conan Doyle él solo.

---Martha ---dijo el Doctor---. También necesito tu ayuda. No puedo
estar en dos lugares al mismo tiempo. La información que vais a
conseguir es crucial. Y eres la única que me la puede dar.

Martha sabía qué quería decir aquello. El Doctor había pasado el
destornillador sónico por su teléfono móvil para que tuviera una batería
casi infinita y una cobertura increíble.

---Al menos dime que no me estás mandando fuera de peligro ---dijo,
llena de sospechas.

---Más bien, me preocupa estar enviándote donde está el peligro ---dijo
el Doctor---. Al menos yo sé a lo que me enfrento, tú no.