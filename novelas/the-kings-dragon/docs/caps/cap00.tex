\chapter*{Prólogo}
\addcontentsline{toc}{chapter}{Prólogo}

{Solo venían durante la noche. Se deslizaban alrededor de los lugares
oscuros, los sitios escondidos, los sitios pobres y solitarios.}

{Se decía en la ciudad que podías saber cuándo se acercaban. Primero tu
	piel se ponía de punta y luego un enfermizo miedo frío se te metía en el
	estómago y se alzaba y alzaba, hacia arriba, bien arriba, hasta que no
	podías hablar y no podías respirar, y el farol que llevabas no podía
	soportarlo más y salía: ¡¡fuuumf!! Y luego las sombras se espesaban y se
	oscurecían, y ya no podías ver más allá de la esquina del callejón o de
	la curva de la carretera. No podías ver el peligro que se escondía por
delante, pero estaba allí. Y allí permanecía.}

{Eso es lo que se decía. Pero cuando le preguntabas al que te contaba la
	historia si él o ella había visto aquellas cosas por sí mismos, ``¡Yo
	no!'' era la respuesta. Más bien un primo, o el amigo de un primo,
	habían oído la historia de otra persona: ``¡Una fuente reputada, por
	cierto! ¡Mi primo no es de los que cuentan historias!'' y tú asentirías
	educadamente y responderías: ``¡No, no! ¡Por supuesto''. Pero en privado
	desecharías la historia (de nuevo) y volverías a tus asuntos. Pues otros
	asuntos estaban a la orden del día en aquellos días en Geath. Ya que la
ciudad ahora tenía un nuevo y joven rey.}

{Aun así, pensarías, mientras cerrabas las puertas, adelante y atrás, y
	sellabas las ventanas, era extraño cómo de vacías se ponían las calles
	tras el anochecer. Era extraño también, cómo cerrábamos todos las
puertas y las ventanas en estos días.}

{Y cada noche, alguien escurriéndose por un estrecho callejón o cruzando
	una plaza desierta, en algún negocio que tristemente no podía esperar
	hasta la mañana, se imaginaría que podía ver sombras moviéndose por
	delante de él o ella, moviéndose sin ningún viento tras ellos, siendo
aquel verano de los calurosos.}

{Y algunas personas, las más imaginativas, claramente, y de las que
	menos se podía fiar uno, añadirían un poco de color a su historia (pues
	los mejores de nosotros no nos podemos resistir a un poco de color).
	Había un extraño ruido, decían, como el gruñido de una bestia salvaje, y
	algunos jurarían que en la pared del pasaje, curvándose por delante de
	ellos, habían visto la larga sombra de una mano, o de una garra,
alargándose.}

{Y lo divertido era, dirían, que aquella mano tenía demasiados
	dedos\ldots{}}
