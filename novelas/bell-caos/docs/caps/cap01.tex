\chapter*{UN DIA}
\addcontentsline{toc}{chapter}{UN DIA}

Plovia allà al turó, amb el continu tamborineig colpejant el gran
ombrel·la de golf com bales contra una llauna. Per a ser sincers, plovia
arreu, però dalt el turó, allà a l'hort, era l'únic lloc on en Wilfred
Mott es preocupava de veritat si plovia o no.

Sempre que plovia, no podia evitar recordar-se'n. Aquell terrible,
terrible dia quan ell va arribar a casa, portant la Donna amb ell.
Inconscient, sense poder recordar-se'n de res. Per la seva pròpia
seguretat.

En Wilf va recordar aquell últim cop que va veure el Doctor, xop per la
pluja, amb la cara regalimant d'aigua, el cabell amarant aigua i la roba
humida i enganxada al seu llarg cos. I els seus ulls, els seus ulls
semblaven tan turmentats, tan tristos, tan perduts. I tan, tan vells.
Semblaven els ulls d'un home gran, atrapat a un ridícul cos jove. Tan
miserable, tan sol i solitari.

Llavors aquella meravellosa cabina blava es va esvair quan en Wilf el
saludava. I mai no va tornar a veure al Doctor.

Però això no li privava de mirar allà dalt. Dalt el cel nocturn, als
estels que només continuaven allà gràcies a la Donna. Cap als estels que
escalfaven i il·luminaven incomptables planetes, amb incomptables vides
que li devien la seva existència a la Donna Noble.

No volia pensar en això. No ho entenia del tot, ell només confiava en el
Doctor amb la seva vida. I el Doctor mereixia aquesta confiança perquè
els havia salvat a tots. De les naus espacials del cel, de l'estel de
Nadal, del gegantí Titanic, dels Adipose, dels Sontaran i dels Daleks.

I aquells eren els únics que ell coneixia. Ho coneixia per la Donna, la
Donna que havia estat abans, que n'hi havia hagut incomptables més.

Va negar amb el cap quan va pensar en tota la grandesa d'allò. I com
n'era de petit i insignificant que era ell en comparació. Però no
l'importava. Perquè l'honor era en haver conegut al Doctor. Va recercar
a la seva butxaca xopa i va treure una cartera de cuir. I del seu
interior, va treure unes fotografies.

Una mostrava a la Donna el dia de la seva boda la boda que mai va tenir
lloc. La Donna creia que mai havia arribat a l'altar perquè el pobre
Lance s'havia trobat enmig de tot el que va passar amb l'estel de Nadal
atacant els carrers de Londres. El Lance va morir llavors, i aquella era
la història que li havien dit. De fet, és la història que en Wilf va
explicar a tothom. I també li va dir a la Donna que havia estat tan
traumatitzada per la mort d'en Lance, que s'havia anat a Egipte per a
superar el mal tràngol.

Fos el que fos el que el Doctor hagués fet als seus records va fer que
el seu cervell acceptés la història i trobés una manera d'encaixar-ho
pel que estava convençuda ~de que això era el que realment havia passat.
Potser era això l'única cosa bona que havia sortit de l'''accident'',
que el seu cervell feia allò per a enfrontar-s'hi, més que un any en
blanc, si li deixaves caure una idea, ella podia racionalitzar-ho sense
preguntar. Com un trencaclosques on no encaixessin les peces sinó que es
transformaven per a encaixar, i formaven una imatge lleugerament
diferent que la Donna mai s'atreviria a preguntar.

Una altra fotografia era de la Donna amb la seva mare i el seu pare al
sopar de l'aniversari del seu pare a la ciutat. Va resultar ser el seu
últim aniversari.

L'última fotografia era d'una dona gran a una cadira de vímet, amb un
got de cervesa negra a la mà, fent un brindis cap al fotògraf.

En Wilf va sospirar. Massa tristor a la casa dels Noble durant l'últim
parell d'anys.

Va apartar les fotografies i va tornar a fer-hi un cop d'ull. Res a
veure.

--Com està el cel aquesta nit? --va preguntar una veu per darrere.

--Hola, reina--va dir en Wilf, indicant a la nouvinguda que s'unís amb
ell sota l'ombrel·la--. Què fas aquí? Et refredaràs.

--Oh, estic bé, pare--dic la Sylvia Noble, passant-li un termos--. T'he
portat una mica de tè i una tauleta de xocolata.

En Wilf va agafar agraït el termos, i es van asseure en silenci durant
una estona, deixant que la pluja fes una simfonia per sobre d'ells.
Llavors va destapar el termos i se'l va oferir a la seva filla. Ella fer
que no amb el cap.

--La Donna feia això--va xiuxiuejar.

La Sylvia va fer que sí amb el cap.

--Ho sé. Potser faci falta que li estimuli la seva memòria i torni a fer
el mateix.

En Wilf es va encongir d'espatlles amb tristor.

--Millor que no, eh? Només per si de cas.

La Sylvia va canviar de tema.

--Llavors no hi ha cabines blaves al cel aquesta nit?

--Avui no. Però un dia el veuré.

Va haver-hi una pausa.

--I això t'importa de veritat? Després de tot el que ha fet per
nosaltres?

--Si, reina, m'importa--va dir en Wilf--. Necessito saber si continua
allà fora, vigilant encara per nosaltres. Vigilant l'univers. Perquè
llavors sabré que la Donna va sofrir per quelcom que mereixia la pena.
Però sense ell no estem segurs.

--Això és massa fe per a posar a un sol home, pare--va dir la Sylvia amb
tranquil·litat--. I molta responsabilitat.

En Wilf sabia que a la Sylvia no li agradava el Doctor, i no només pel
que li havia passat a la Donna. Ella pensava que si el Doctor mai hagués
vingut a la Terra potser aquells Daleks tampoc ho haguessin fet. Era una
discussió molt llarga, i ells mai no es posarien d'acord. Per això no
intentaven parlar-hi gaire del Doctor.

--És allà fora, reina. Protegint-nos. I els marcians i els venusians. I
Déu sap què més--va prendre un glop de te--. Hauries de tornar a
escalfar-te.

La Sylvia va fer que sí amb el cap i es va aixecar.

--T'hi estaràs molta estona?

--No, només vull ser-hi fins a les onze, llavors hi aniré.

--La Donna ha suggerit anar a visitar a la Netty demà.

En WIlf va abaixar el te.

--No, gràcies--va dir ràpidament.

--Pare, hauràs de veure-la en algun moment--la Sylvia es va estirar i va
estrènyer la seva mà--. Si no és pel teu bé, fes-ho pel seu.

--No hauries de deixar que la Donna hi anés--va dir en Wilf--. No és
segur. Què passa si diu res del Doctor?

--No ho crec. I encara que ho fes, la Donna no ho entendria i la Netty
no podrà explicar-ho--la Sylvia va fer que sí amb el cap i va caminar
per la pluja. Llavors va mirar enrere al seu vell pare--. Hem passat
coses que ningú hauria de passar, pare-- va dir amb calma--. No
provoquem més nosaltres. Vine, si us plau.

--M'ho pensaré. Ara, ves-te'n, abans de que et refredis.

La Sylvia va assenyalar cap al cel.

--El Doctor ho voldria--va dir.

En Wilf es va girar, arrufant les celles.

--Això no és cosa teva, reina. Si us plau, no ho diguis.

La Sylvia va fer que sí amb el cap.

--Ho sento, pare--i va baixar per l'hort i el turó.

En Wilf va veure la seva silueta fent-se més petita fins que es va
perdre de vista, llavors va treure la tauleta de xocolata i va
mossegar-ne un tros. Es va girar per tornar a mirar un altre cop pel
telescopi, preocupat perquè la Sylvia havia parlat del Doctor. Preocupat
perquè era un cop molt baix. I preocupat perquè la Sylvia tenia raó.

Una llàgrima va caure per la galta d'en Wilf.

Per moltes raons.

Un mes després que els cels s'omplissin de foc.