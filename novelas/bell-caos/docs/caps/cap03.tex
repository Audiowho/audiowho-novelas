\chapter*{DISSABTE}
\addcontentsline{toc}{chapter}{DISSABTE}

La Caitlin era dempeus esperant a la terminal 5 de l'aeroport de
Heathrow, esperant als vols que haurien d'haver anat a la terminal 2.
Aquella estava sent demolida en aquell moment, llesta la Terminal de
l'Est que substituiria a la que els vols europeus anaven. Ella es
trobaria amb diferents vols, inclòs un que venia del JFK al migdia, que
arribaria abans que els europeus. Va veure com un jet 787-9 aterrava,
amb la llum del sol brillant per sobre del seu nou material mentre
frenava lentament, abans de donar una volta per la porta d'entrada i ser
col·locat en posició pel pilot automàtic.

La Caitlin estava vestida amb un uniforme polit del personal de la
Terminal, amb el seu passi a totes les àrees, amb l'accés de màxima
seguretat que la Madame Delphi havia pogut aconseguir, penjant d'una
corda pel seu coll. Va somriure amb dolçor a una parella d'assistents de
vol, que cap dels dos li va dir res perquè potser mai l'havien vist
abans en la seva vida. Era el passi què feia tot allò, tot i que el
cabell llarg i els ulls blaus de la Caitlin haurien valgut per a
distreure qualsevol qui la mirés de prop. Un somriure ràpid era el que
li costava a la Caitlin per a obtenir el que volia.

Va ser un guàrdia de seguretat qui la va veure mentre caminava a través
de les portes d'Accés Restringit cap a la porta de les Arribades, anant
exactament on ella no hauria de ser-hi.

--Perdoni? --li va cridar.

--Hi ha cap problema\ldots{} --la Caitlin va mirar el seu cartell amb el
nom--. Hi ha cap problema, Keith?

En Keith Brownlow la va mirar, amb el seu cap lleugerament girat cap a
un costat, com si la estigués mesurant.

--No t'he vist abans--li va dir, tranquil·lament.

--No hi sóc des fa molt--va respondre ràpidament.

--És divertit--va respondre--. Conec a tothom a aquesta Terminal. I
aquells a qui no conec, no hi entren. Em temo que hauràs de sortir, fins
que pugui verificar qui ets qui dius ser.

La Caitlin es va aturar per un moment, intentant recordar un nom.

--Estic segura de que si ho comproves amb el senyor Golding, li dirà qui
soc. Em vaig unir a la plantilla aquest dijous.


En Keith es va encongir d'espatlles.

--I així ho faré, però només quan hagis tornat pel camí que has vingut i
esperis a l'Àrea Groga.

--Ets molt bo en això, Keith, veritat?

--Ens prenem la seguretat molt seriosament--va dir.

--Per suposat que sí--va filar la Caitlin--. Amb molta raó. Però mira,
només som tu i jo, has vist el meu passi, pots veure que hi tinc accés.
Estic segura que és només l'administració que no ha pogut informar-te de
que sóc aquí. Necessito de veritat ser a l'altre costat d'aquesta porta
per a anar a buscar a alguns VIPs.

En Keith la va ignorar, amb la seva mà dreta descansant sobre el seu
revòlver penjant de la cintura, i la seva mà esquerra agafant el seu
walkie-talkie.

--Oh, estimat i jo que creia que podríem ser amics--va dir la Caitlin,
aixecant el seu braç dret i enviant una bola d'energia morada cap al pit
d'en Keith, reduint-lo a una pila de cendres abans que pogués registrar
qualsevol altre moviment.

Va caminar cap endavant, amb els seus talons trepitjant les poques
cendres que quedaven a l'alfombra. Amb un ràpid i profund sospir va
empènyer les portes de la zona VIP i hi va entrar per a saludar als seus
hostes.

Allà hi eren dempeus, esperant-la, cap d'ells sense semblar que
s'adonessin d'on eren o què hi feien allà. Els dos vells americans,
vinguts des de Nova York.

--El senyor i la senyora DiCotta? Felicitats per la seva boda i
benvinguts a la seva lluna de mel.

En Donnie i la Portia DiCotta no van dir res, només van fer que sí amb
el cap mentre la Caitlin els va guiar cap un dels cotxets de servei que
els passatgers ancians i discapacitats feien servir per a moure's per
les terminals de l'aeroport.

Un confós encarregat d'equipatge va mirar el seu passi quan li va
ensenyar, llavors es va apartar, permetent que agafés el seu cotxet. Els
DiCotta s'hi van pujar mentre l'encarregat es girava.

--Tenen equipatge? --els va dir.

La Caitlin va fer el seu somriure més dolç, que esperava que no suggerís
el que pensava (Oh, apartat, estúpid ximplet), i llavors va dir en veu
alta:

--Ha estat endarrerit al JKF, així que haurem de trucar demà al matí.

Va engegar el cotxet i va conduir cap endavant.

--Sou els primers--va dir amb calma, sense dolçor ni somriures.

--On són la resta?

--El grec arribarà aviat, el grup italià no trigarà més d'una hora. Hi
esperarem, els recollirem i anirem a veure a la Madam Delphi junts. Té
una missió per a vosaltres aquesta nit--la Caitlin va riure una mica--.
Us preguntaria pel jet lag, però suposo que la Madam Delphi s'ha
assegurat de que no sentiu res d'això.

--Ens sentim bé--va dir la Portia DiCotta.

--Bé--va dir la Caitlin--. Perfecte.

El cotxet va anar cap endavant, evitant les rutes principals i a través
d'una altra porta on el vol de l'Atenes Internacional arribaria.

--I llavors anirem a trobar el fastigós 787-3 de l'Aeroporto Leonardo da
Vinci di Fiumicino--va dir--. Espero que gaudeixin de la seva estada.
Serà breu però dramàtica.

El diminut cotxet va travessar els llargs passadissos llest per a
recollir més gent del petit exèrcit que la Madam Delphi faria servir per
a portar a la raça humana a una completa destrucció.

La Donna estava arreglant la corbata del Doctor. Li havia fet centenars
de cops allò al seu pare, especialment vora el final, que li havia donat
una excusa de ser rondinaire i sentir-se inútil. El Doctor no tenia
aquella excusa, però allò no li va estar de no deixar de queixar-se.

--Donna, de veritat, puc cordar-me la corbata.

--De veritat? Perquè no ho he vist pas, jo això--era la seva resposta,
seguida d'un massa entusiasta cenyida de nus massa propera a la nou del
Doctor--. Ups! --li va somriure--. Ho sento molt.

Ell va esmunyir un dit sota el coll de la camisa i d'alguna manera amb
un sol gest no només va afluixar el nus, sinó que a més va descordar el
botó superior i fer més bonica la camisa.

Era un art, va dir. Molt xic, va dir. El Doctor descamisat, va dir ella,
rendint-se en una causa perduda.

--Com es porta la Netty? --va preguntar ell.

--Avui està bé--va dir la Donna--. L'Avi s'encarrega d'ella. La mare ha
començat a queixar-se, així que s'ha anat abans d'hora i ens ha dit que
ens trobarem amb ella allà. Espero que s'hagi anat a comprar un altre
barret estrafalari!

--La teva mare només està preocupada per en Wilf, això és tot. La Netty
és molt divertida però comporta molta responsabilitat--va dir el Doctor.

--Ho sé, però la mare no ha de ser tan negativa amb ella, oi?

El Doctor es va encongir d'espatlles.

--Ella és una mare, Donna. És el seu treball fer culpable a tothom de la
seva família. És al manual.

--La teva mare també et criticava tot? La forma en la que et pentinaves,
la roba que vesties, els amics amb els que sorties, la música\ldots{}

--Sí, bé, va ser una mica diferent per a mi--va dir amb calma.

La Donna el va mirar i li va donar un somriure recomfortant.

--Per suposat que ho va ser. Ho sento, no hi he pensat.

--Oh, no passa res. Només dic que li donis a la teva mare una mica de
crèdit. És molt amb el que assimilar, ha estat cuidant del teu avi ella
sola durant molt de temps. Ara ell té algú a la seva vida i ella ho ha
vist com una intromissió.

--Oh, no et posis Spock amb mi.

--Spock?

--Sí. el psicòleg infantil aquell, o qui sigui que fos. El que parlava
sobre relacions entre els pares i els fills.

--Ah, el doctor Spock. És clar.

--Per què? Que hi ha un altre Spock que coneguis? --va riure la Donna
mentre anava cap a l'habitació de convidats. Tot i que sospitava que el
Doctor no havia dormit gens allà, mai semblava que necessités dormir com
una persona normal.

El Doctor es va mirar a si mateix al mirall. Sempre havia cregut que li
quedaria bé una jaqueta de vestir i una corbata negra, odiava els
corbatins, li feien semblar un cambrer, i pel que havia passat altres
cops, millor no tornar-ho a intentar. Així que aquella nit una decent
corbata negra. Per suposat, ara semblava que anés a un funeral, però ei.
I després estava el que tenia amb les jaquetes: no importava com es
cordava els botons, sempre semblaven o massa petites o massa cenyides, i
els pantalons mai li arribaven fins als talons.

Però bé, era la nit d'en Wilf, no la seva.

Van trucar a la porta.

--Sí, ara vaig, Donna, dóna'ns un minut.

La porta es va obrir. Era la mare de la Donna.

--Ah, hola--va dir ell. Era una dona intimidant i, com la majoria de les
mares, no semblava que li agradés gaire. Era la seva imaginació o la
seva galta començava a coure de nou?

A algunes mares les havia guanyat per encant propi (ah, Jackie Tyler,
què estaràs fent avui dia?), o per provar que la fe de la seva filla en
ell era justificada (sempre vas tenir un bon ganxo, Francine Jones,
beneïda siguis).

Però ara era la Sylvia Noble. Tota plena d'orgull, temperada de tanta
fúria, tanta frustració. Era com si mai hagués sentit el control de la
seva vida que tant s'esforçava en dir que tenia, i que allò li enfadés
molt.

Per suposat, tampoc havia estat fàcil perdre a en Geoff. El Doctor només
l'havia conegut un cop, a la boda de la Donna, on semblava ser
el\ldots{} més calmat dels pares Noble. Ara la pobra Sylvia intentava
tot el que podia per comportar-se amb una filla capritxosa que ja no era
tan capritxosa com la Sylvia s'imaginava (seguia sense tenir cap idea
d'on la portava el Doctor en realitat) i un pare gran, que estava tan
obsessionat en no ser una càrrega per a la seva filla que s'havia
convertit en una més gran per defecte. En Wilf Mott volia provar que era
independent, fort i vint anys més jove del que era, creient que trauria
pressió de la Sylvia, però no s'adonava de que el que la Sylvia tenia
encara del doble de preocupació de la que hauria tingut si senzillament
s'hagués assegut a la butaca a veure tot el dia Countdown.

Quant feia que el Doctor no s'asseia a veure Countdown? Li acostumava a
agradar Countdown.

--Vosaltres us encarregueu d'ell.

--No vindràs?

La Sylvia va fer com si estigués a punt de dir alguna cosa, però només
va fer que no amb el cap.

--No és cosa meva.

--Però és el teu pare\ldots{}

--Hem tingut una\ldots{} discussió sobre això.

--Ah--el Doctor només es podia imaginar com havia anat--. Estàs segura?
Perquè estic segur de que ell, la Donna i la Netty s'estimarien\ldots{}

Es va aturar. La Sylvia va a estar a punt d'encongir-se amb només la
menció de l'Henrietta Goodhart.

--Cuida d'ell, Doctor, si us plau--va dir amb tranquil·litat--. Ja no és
un jove, tot i el que ell pensi.

El Doctor va somriure encisadorament.

--Per suposat que ho faré.

--No hi ha cap ``per suposat que ho faré'' amb tu, Doctor, així que no
em facis totes aquestes coses encisadores que fas. No t'ho estava
demanant, t'ho estava dient.

El Doctor va intentar no somriure, era com si li estigués ordenant la
mestressa. Llavors els ulls de la Sylvia li van recordar que allò no era
ni remotament divertit.

--Promès--va dir, afegint mentalment: ``I sinó copiaré tres-centes
línies: No perdré cap membre de la família Noble a Londres''.

--Ells són tot el que tinc--va dir i va sortir de l'habitació.

El Doctor es va llepar el dit índex i el va aixecar.

Sí, la temperatura ambient de l'habitació havia baixat de fet uns quants
graus.

Mitja hora després, s'havien ajuntat a un taxi. En Wilf i la Donna
s'asseien al seient, i el Doctor en un d'aquells seients afegits a la
part del darrere. Es va remoure incòmode durant tot el viatge.

En Wilf anava vestit de vint-i-un botons (una corbata de seda, una bona
camisa, una jaqueta una mica ajustada que probablement havia estat
comprada durant els setanta) però el que ressaltava eren les esportives
que portava als peus.

Com si notés que el Doctor el mirava, en Wilf va aixecar una bossa de
plàstic. Dins hi havia un parell de sabates elegants negres.

~--Em tallen la circulació als peus i és per això que les porto el menys
possible--va explicar.

El Doctor va fer que sí amb el cap.

--Et veus molt bé--li va dir a la Donna.

--Gràcies. Ho he hagut de demanar a la Veena, que li havia deixat a la
Mooky, així que ha hagut de passar-se aquesta tarda mentre tu eres fora
posant-te el vestit i\ldots{} què?

El Doctor va somriure.

--Les teves amigues tenen uns noms impressionants--va riure
amablement--. Mooky.

La Donna va aixecar una cella.

--No parlis. Senyor Ood. Senyor Matrona Cofèlia. Senyor Ventraxian
Gol-Zeeglar. No pensis que les meves amigues tenen noms graciosos,
perquè els teus\ldots{}

--Ben vist, tot i que Ood no és ben bé un nom, sinó una designació
d'espècie, i jo, eh\ldots{}

La Donna li estava llençant una d'aquelles mirades com dient ``de
veritat creus que m'importa?'', així que es va girar cap en Wilf.

--Emocionat?

--Massa, Doctor. He trobat una nova estrella. I li han posat el meu nom.
És genial!

--És més que genial, Avi--la Dona li va estrènyer la mà--. És fotudament
meravellós. Aquí l'home de l'espai, pregunta-li si algú li ha posat el
seu nom a alguna estrella.

--Ho han fet, Doctor?

Si ho havien fet o no, no era allà ni llavors.

--És clar que no--li va dir a en Wilf--. I estic molt gelós.

--Sí--en Wilf es va inclinar, per tal que el conductor no l'escoltés--.
Sí, però i tu? Tu vas a veure-les, veritat? Tu te'n vas allà dalt--es va
girar cap a la Donna i va somriure--. I tu te n'aprofites, reina meva

--Oh, i tant! No et preocupis per aixó! --va dir ella.

El Doctor va fer que sí amb el cap.

--Oh, i tant que ho fa, no t'has de preocupar per això.

--Vull dir, n'he vist i fet de moltes coses a la meva època, Doctor,
però no hi ha rest comparat amb el que li ensenyes a la meva petitona,
oi?

--Ei, no sóc només una passatgera, eh--va somriure la Donna--. Jo prenc
moltes decisions sobre on anem, a qui veiem, com de ràpid ens hem de
marxar perquè ell no hi és, i em deixa a algú altre al càrrec. Amb un
exèrcit. I una destral. I dotze potes.

--Deu potes--la va corregir automàticament el Doctor.

--Oh, d'acord, llavors deu. Deu potes, i dos braços que penjaven fins el
terra, si ets pedant. Que és quelcom que ets clarament. Aquesta nit.

El Doctor els va somriure a ambdós.

--Potser et podríem donar una volta algun cop.

--No!

Ambdós la Donna i el Wilf ho van dir junts, i llavors es van mirar l'un
a l'altre.

--És perillós, Avi.

--Tu hi vas!

--Jo sóc\ldots{} puc valdre'm per mi mateixa. Si qualsevol cosa et
passés a tu, què faria la mare?

--Matar-te?

--Bé, el mataria a ell primer--la Donna va assenyalar al Doctor amb el
cap--. Jo aniria després sent espellada viva. Probablement.

--Tu també has dit que no, Wilf--va dir el Doctor.

En Wilf va mirar cap a les estrelles mentre anaven cap al sud de
Londres, creuant el pont Bauxhall.

--M'agrada mirar-les, Doctor. M'agrada mirar-les, imaginar-me-les i
somiar-hi. Però la realitat? Tots són monstres i pistoles i aquestes
coses. No. Prefereixo les meves idees.

El Doctor va fer que sí amb el cap.

--Molt savi. Si et serveix per a alguna cosa, series una influència
tranquil·litzadora per a ella.

--Oh, ho sé. Sempre és una fera, veritat?

--Estic aquí asseguda. Just aquí--va dir la Donna.

El Doctor va seguir parlant amb en Wilf.

--I òbviament hi ha alguna cosa a la seva infància amb els centpeus,
però mai m'ho dirà. Perquè el dia que vam anar a aquell lloc\ldots{}

--Ei!

En Wilf va riure.

--Oh, t'hauria d'haver dit això. Quan tenia uns vuit anys, el seu pare i
jo la vam portar cap a Norfolk. Vora de Broads? Bé, sigui com sigui,
estava nedant com un gosset quan\ldots{}

--EI!

Ambdós van mirar a la Donna. S'assenyalava a ella mateixa. Un dit a cada
mà. Cap al seu cap.

--Com he dit: estic aquí asseguda. I ho escolto. I no m'agrada el que
escolto--els dos homes li van somriure.

--Després--va dir el Doctor.

--Després--va coincidir en Wilf.

La Donna els va interrompre.

--Ja hem arribat.

El taxi els va deixar davant de la Societat, un gran edifici de totxana
vermella construït a començaments de l'època victoriana, igual que el
vestíbul principal de l'estació de Vauxhall.

La Donna va pagar al conductor (--Tu pagaràs la tornada--li va dir al
Doctor). I mentre el taxi sortia corrents, es va planxar el vestit, es
va comprovar els talons i li va fer una petita empenta al Doctor, que
mirava cap a dalt al cel nocturn i les estrelles.

A la nova estrella que en Wilf li havia ensenyat la nit abans. Que ara
semblava més brillant que abans. I també semblava que hi havia un parell
d'estrelles que no creia que no haurien de ser-hi allà\ldots{}

La Donna el va tornar a fer una empenta.

--Què?

Ella va indicar-li amb el cap al seu avi, que s'estava agafat a un
fanal, intentant treure's les esportives i agafant la bossa de plàstic
al mateix temps.

--No em puc ajupir amb aquesta cosa--va xiular ella--. Si el trenco, la
Veena em matarà la setmana que ve. Va a això de les arts marcials.

El Doctor va agafar-li la bossa a en Wilf i va deixar que l'home gran es
recolcés en ell mentre es treia les esportives i les canviava per les
sabates de vestir.

--La Sylvia me les va comprar--va dir-li al Doctor--. Aquestes maleïdes
coses són tres talles més petites.

--No ho són--va dir la Donna automàticament--. El que passa és que no ho
intentes prou.

--Caram, quan t'has tornat la teva mare? --va dir en Wilf. La Donna va
obrir la boca per respondre-li, però el Doctor, percebent que una
retirada era millor que un atac, els va agafar a ambdós i es va posar en
mig.

--El sopar ens espera--va dir i van entrar a l'edifici.

La gran porta de fusta es va obrir quan van arribar al graó de dalt i un
net i polit cavaller, d'uns trenta anys, amb la pell olivàcia, el cabell
negre i els ulls brillants, els va saludar.

--Bona tarda, senyor Mott--va dir amb un lleuger accent europeu--.
Senyoreta Noble, Doctor Smith. Sóc en Gianni, cap de Relacions
Públiques.

La Donna va fer cara de sorpresa.

--Ho té molt ben assajat.

--Sou convidats d'honor--va dir en Wilf--. Els havia de dir a qui
portava.

En Gianni els va portar cap a una petita àrea on una parella de gent
d'uns seixanta o dos-cents onze anys estaven recolzats contra un bar. O
possiblement el bar els aguantava a ells. Sigui com sigui, semblaven
formar part del mobiliari.

La Donna va arrufar el nas amb el fort olor a Scotch i va mirar rere
seu, on una altra porta conduïa cap a una zona d'àpats, i un tumult de
sorolls.

--Vagi'n--va dir el Relacions Públiques, així que la Donna va anar la
primera.

Mentre els tres entraven al menjador, el tumult va parar i va ser
substituït per una ronda d'aplaudiments iniciada per, la Donna es va
alegrar de veure-la, l'Henrietta Goodhart, resplendent amb un altra
estrany però preciós barret de pamela.

Ella va caminar cap a ells, amb els braços oberts, va fer petons a la
Donna, després al Doctor i finalment a en Wilf, a tots un petó a cada
galta, a l'estil del continent. Llavors li va plantar un de ràpid als
llavis d'en Wilf i va parpallejar.

--Estic bé aquesta nit--va respondre a la seva pregunta sense fer.

Un home amb cinquanta molts es va apropar, i va estrènyer la mà d'en
Wilf.

--Crossland. Sóc en Cedric Crossland. El Doctor Cedric Crossland. El
Doctor Sir Cedric Crossland, però em pot dir Rick, senyor Mott.

--Oh, només Wilf estarà bé--va dir en Wilf, llençant una mirada buscant
ajuda del Doctor, de la Donna o de la Netty.

La Donna es va avançar, però la Netty va anar primer.

--No, no, deixa'l anar aquesta nit. És la seva nit a les verdes i a les
madures i és capaç de tot i més. A més, el pudding de xocolata fa que
valgui la pena haver vingut--van mirar com el Wilf es veia en mig de
celebracions--. Sembla tan feliç--va acabar.

--Suposo que tens gran part de la culpa--va dir el Doctor, afegint--. No
és que pretengui entendre coses com aquesta.

La Netty li va somriure.

--Per suposat que no, Doctor. Al ser de l'espai exterior.

El Doctor la va mirar bocabadat, i llavors va somriure.

--De fet, sóc de Nottingham\ldots{}

--Ell és de Walthamstow--va dir la Donna al mateix temps.

--Nascut a Nottingham--va dir el Doctor.

--Però criat a Walthamstow--va afegir la Donna, una mica més tard.

--En Wilf m'ho ha explicat tot, Doctor. Sobre tu. Sobre allò de l'ATMOS.
Sobre on és la Donna quan és amb tu. No tenim secrets entre nosaltres.

El Doctor va sospirar.

--Bé, no estic segur del que t'hagi dit en Wilf, però soc\ldots{}
eh\ldots{} bé.

La Netty li va estrènyer la mà.

--Està bé. La major parts dels dies gairebé no puc recordar-me de qui
sóc jo, imagina't des quin planeta tu ~i la Donna esteu enviant postals.
El teu secret és segur amb mi.

--Crec que el mataré aquest cop--va dir la Donna, mirant cap al seu avi
que es bevia un got extraordinàriament llarg de brandy en mig d'un grup
gran d'homes i dones.

La Netty va fer que no amb el cap.

--Està molt orgullós d'ambdós vosaltres, no sigueu dolents amb ell. A
més, em dona l'oportunitat de parlar-vos del Cos del Caos. Ja sabeu, al
menys mentre pugui.

La Donna va arrufar les celles.

--Em sap greu, Donna--va dir la Netty--. Parlar de la meva condició et
molesta? No hi ha cap necessitat, no hi ha res que hagi d'amagar a
ningú. Al menys de mi mateixa.

--No és això--va dir la Donna--. És només que\ldots{} és una mica trist.

--Ho és. Molt trist, creu-me. Però m'he acostumat a viure-hi i aprofito
la major part dels dies lúcids que tinc perque els que no ho són es
tornen més i més freqüents.

--Com de freqûents?

--Doctor! No ens ho explicarà tot!

Però la va fer callar.

--Com de freqüents, Henrietta?

--Si puc arribar a divendres recordant què vaig fer el dimarts, és una
victòria.

En Gianni era al seu costat, just on un bon Relacions Públiques hauria
de ser, amb les begudes a una safata de plata per a ells i ells van
agafar els gots ràpidament, com si intentessin omplir el buit a la
conversació.

--Així que--va dir la Netty--. Els cossos del caos.

--Quan van començar a aparèixer? El primer, vull dir.

--Ah--va dir la Netty--, has vist els altres. Només els he vist per mi
mateixa aquesta nit i ningú aquí aquesta nit sembla haver-los esmentat.

--Això és perquè no hi eren l'última nit. O quan van deixar Chiswick, al
menys.

La Netty va riure.

--Sé que saps més sobre l'espai exterior que aquesta pila d'aquí junts
però, científicament, les estrelles no es mouen tan ràpidament. I si en
poguessin, la devastació seria espectacular.

El Doctor va fer un brindis amb ella.

--Ah, doncs llavors no són estrelles. No són estrelles de veritat. Un
tros del caos, tot i així, és el que es pot veure.

--Què són llavors? --va preguntar la Donna.

El Doctor es va encongir d'espatlles.

--Tinc una sospita. La primera, l'original, que sembla com una estrella
de veritat, i és certament una bola d'energia combustible súper-calenta
que comparteix propietats menors amb una estrella, però la resta, són
com satèl·lits. Però no dels estel·lars.

--Fets per l'home?

--Bé, fets per algú, sí. I en algun lloc en un racó del meu cap hi ha
una veueta intentant dir-me on ho he vist abans.

Hi va haver un sorollet d'algú colpejant un vidre amb una cullereta.

--Dames i cavallers, abans que sopem, hauria de presentar-vos al nostre
hoste d'honor--deia el Doctor Crossland--. En honor de ser el primer en
veure la nova estrella M7432·6, oficialment coneguda com 7432MOTT,
aquest és\ldots{} en Wilfred Mott!

Hi va haver un estrepitós aplaudiment i a més un ``visca, visca'' quan
un dels cambrers va conduir al Doctor, a la Donna i a la Netty a la
taula a unir-se a la resta de comensals.

El Wilf va enganxar al Doctor i el va posicionar entre ell i en
Crossland, mentre la Netty estava a l'altre costat d'en Wilf i la Donna
al seu costat.

El Doctor va mirar expectant al lloc buit a la seva dreta, preguntant-se
qui s'asseuria allà. Era, pensà, una mica com estar assegut a un tren i
desitjar que el seient buit al teu costat no vagi a estar ocupat per un
home boig amb uns auriculars molts alts o un nen hiperactiu o, el pitjor
de tots, algun home de negocis desendreçat que es passaria el viatge
sencer xerrant en veu alta pel telèfon mòbil. I cada cop que pengés,
algú altre el trucaria i ell deixaria que el molest to de trucada sonés
tota l'estona abans de contestar.

El Doctor sempre es preguntava pel dia en que totes aquelles petites
coses trivials van començar a molestar-lo de tal manera. Devia de ser
que s'apropava als 900.

El seient va ser empès cap enrere per una dona, roba de la qual podria
ser millor descrita com excèntrica i boja com a molt, un horrible xoc de
colors, estils i bé, de tot. El crim més gran contra la moda era la
brusa que portava posada, que semblava tenir-hi el mapa celestial de
Galileu brodat. I a mà. Eren el tipus de robes que et posaries si et
vestissis amb els dos ulls tancats després de que algú hagués agafat el
teu armari i l'hagués ficat una bona sacsejada.

Tampoc no és que la qui el portava semblés remotament conscient de la
seva\ldots{} única i espectacular peça de roba d'alta costura. I el que
era més alarmant, cap dels altres membres semblava immutar-se\ldots{}
només en Wilf i sobretot la Donna van reaccionar, en Wilf amb
incredulitat i la Donna per aguantar-se un riure i trobant el got
d'aigua davant seu com la cosa més fascinant de la Terra.

--Ariadne Holt--va dir amb un to que suggeria al Doctor que allò no era
només una introducció sinó que era de fet una completa explicació per a
la raó per la que semblava el que era.

--Què hi ha? --va dir ell, oferint-li la mà. Ella li va oferir la seva
com si li suggerís de fer-li un petó, o com a mínim, inclinar-se
lleugerament. No va fer cap de les dues coses, intentant en canvi, de
canviar-ho en l'estrenyiment de mà ~que ell havia començat.

Ella li va fer una mirada que semblava dir: ``Oh, és clar, et
comportaràs així, veritat?'' i va apropar la cadira a la taula i molt
lleugerament més lluny d'ell.

--Així que--va dir l'Ariadne--. Quin és el seu camp?

--Els del deu acres--va somriure.

Li va fer una mirada gèlida.

--És una broma interna--va balbucejar.

--Una de molt dolenta.

Una mirada gèlida de nou.

--Sóc el Doctor, per cert.

--Ho sé--va dir l'Ariadne Holt.

--Ho saps?

--En Crossland m'ho va dir. Em va suggerir que m'hi assegués. Aquí. Al
seu costat. Per sopar.

--Ho va fer?

--Sí.

--Cert. Bé, ho sento. No conec de fet al senyor Crossland.

--Doctor.

--Sí?

--Què?

--Què?

--Ell és el Doctor Crossland. No el ``senyor''.

--Ja veig.

--Vostè és en Smith. En John Smith. Vostè escriu llibres. Amb
fotografies. Sobre constel·lacions, oi?

--Ah, no. No sóc jo. Ho sento. Encara que quan era un nen pintava
quadres amb els dits del cel nocturn. Jo acostumava a afegir-hi trossos
de\ldots{} bé, vostè en diria pasta, per fer els planetes en 3-D. I
purpurina. Molta purpurina. M'agradava molt la purpurina.

Una mirada gèlida de nou.

--No li importen gaire els meus dibuixos amb els dits, veritat, Ariadne?
No et culpo. No eren gaire bons. Una D baixa, baixa. Quatre de deu. Eren
durs, però sempre he cregut que probablement estaven justificats. Així
que, què estudia vostè?

L'Ariadne Holt el va ignorar i el va mirar a través directament cap al
Doctor Crossland.

--No és el Smith correcte, vell boig--va dir ella--. Aquest no és
l'autor.

En Crossland va mirar al Doctor.

--Què és ell llavors?

--Estic aquí assegut, per cert--va dir el Doctor.

--Ja! --va riure la Donna, recordant el viatge en taxi.

--Un boig--va respondre l'Ariadne Holt--. Parlant monòtonament sobre
pintures amb els dits i menjar italià.

La Donna li va aixecar al Doctor els dos polzes i una picada d'ullet
exagerada.

Ell li va fer una mirada que hauria d'haver-la convertit en gel.

Ella li va somriure àmpliament.

En Crossland va mirar al Doctor i llavors a en Wilf.

--Pensava que va dir que el xaval era un expert, senyor Mott.

Ara era el torn d'en Wilf de semblar un conill atrapat en un focus de
llum, perquè estava fora de la seva àrea de confort. El millor que va
poder inventar-se va ser un:

--Ell ho és.

En Crossland va mostrar indignació tossint.

--De fet--va dir el Doctor--. Crec que trobarà que el ``xaval'' és una
mica un expert en els teus tan coneguts Cossos del Caos.

Allò va cridar-li l'atenció.

El Doctor va respirar profundament.

--No és una estrella, ja sap.

--Sabem que no ho són--va dir l'Ariadne.

--Només no sabem què són de veritat--va dir en Crossland.

--Jo ho sé.

Tothom es va girar per mirar-lo.

--És l'encantadora Ariadne la que em va donar la peça final del
trencaclosques, amb aquesta\ldots{} úica brusa que porta a sobre.

--Endavant--va dir.

Però el Doctor es va detenir momentàniament, perquè ell mirava
l'Henrietta Goodhart.

La Netty no estava formant part de la conversació, encara que ningú no
se n'havia adonat. En canvi, ella estava mirant cap endavant, com si
s'hagués quedat en blanc per un moment.

La Donna va captar la mirada del Doctor i va fer que sí amb el cap
lleugerament, i el Doctor li va fer un somriure trist. Llavors va tornar
a captar l'atenció cap a ell mateix de nou, donant a la Donna temps per
a aixecar a la Netty, breument reposant una mà sobre l'espatlla d'en
Wilf per a evitar que els acompanyés. Mentre s'allunyaven, el Doctor va
captar la mirada de la Donna i li va fer amb la boca un ``gràcies''.
Llavors va resumir la seva explicació.

--No és una estrella igual que no ho és una bola de súper-calor de
energia psiònica que actua com un camp de protecció contenidor al
voltant d'una intel·ligència maligna. Una intel·ligència que vol dominar
i expandir-se i sobreviure. Alguna cosa que està completament
autoritzada per fer la seva al seu petit cantó de l'espai on normalment
no pot ferir a ningú. Però quan s'arrossega fora de la seva petita
dimensió i entra a la nostra, quan s'arrossega a si mateixa a meitat de
camí a través de l'univers a aquest planeta, a aquest temps, llavors és
quan m'interesso. M'interesso i m'intrigo. Bé, quan dic intrigat vull
dir emprenyat. I una mica espantat. Poden veure que el mapa estel·lar de
Galileu m'ha recordat de quan m'hi vaig trobar per primer cop, al
voltant del segle XV i estúpidament, m'hi vaig portar un resquill d'allà
cap aquí, a la Terra. A Itàlia, de fet. Vora Florència. Bé, a aquella
regió. De qualsevol manera, uns quants resquills han obert el seu camí
cap aquí i de nou des de llavors perquè estan bastant fascinats per la
Terra i aquest sistema solar.

--Vostè és boig--va dir l'Ariadne amb tranquilitat.

--Ha estat això del segle XV, oi?

En Crossland feia que sí amb el cap.

--Això i tota la resta.

En Wilf va tocar el braç d'en Crossland.

--Hauria d'escoltar-lo, senyor Doctor Crossland de l'Orde de l'Imperi
Britànic, Sir, perquè el Doctor normalment té raó.

--Oh, gràcies, Wilf--va dir-li el Doctor.

--El plaer és meu, Doctor--va respondre.

--Així que, Doctor--va dir en Crossland--, com es diuen aquestes
divertides boles d'energia, llavors?

--Oh, això és senzill--va dir el Doctor--. Una paraula. Mandràgora.

Mentre el Doctor deia ``Mandràgora'', el Cos del Caos va començar a
emetre polsos a l'espai. Les altres llums properes van començar a polsar
en una resposta rítmica. I es van apropar. Amb una missió.

Mentre el Doctor deia ``Mandràgora'', set persones seien tensament en un
quatre per quatre aparcat a un aparcament a l'aeroport Heathrow, com si
els haguessin acabat de connectar: els acabats de casar DiCotta; els
tres estudiants, el seu professor i el seu assistent, acabats d'arribar
de Roma; un granger grec, acabat d'arribar d'Atenes.

El granger grec estava al seient del conductor. Va girar la clau del
motor i va encendre els llums, i el quatre per quatre va començar a
avançar. Amb una missió.

Mentre el Doctor deia la paraula ``Mandràgora'', la Donna Noble estava
asseguda al bar de fora de la Reial Societat Planetària amb la Netty,
agafant-la de la mà, quan va veure a en Gianni, el cap d'Hospitalitat,
va aturar-se mentre servia una beguda. Va obrir la seva boca com si fos
a parlar. Les paraules estaven formades però no emetia cap so, fins que
va tossir i finalment va parlar. Però va parlar tan suaument que la
Donna no estava gaire segura de que hagués del tot parlat.

Mentre el Doctor deia la paraula ``Mandràgora'', el senyor Murakami
estava assegut en Primera Classe al seu avió de les aerolínies japoneses
dirigint-se cap a Narita, amb la beguda a la mà, amb els ulls tancats,
escoltant la recopilació de cançons dels 60 de Hibari Misora que s'havia
descarregat al seu prototip del M-TEK.

Quan es va adonar de que sota la música hi havia missatges subliminals
sobre la MorganTech, era massa tard. Alguna cosa a la seva ment cridava
i xisclava, adonant-se exactament de quin era l'increïble ingredient el
que hi havia al M-TEK, però també sabia que mai seria capaç
d'alliberar-se i avisar al món.

--¿Una beguda, Murakami-San? --la hostessa de vol li va oferir una
selecció de vins i begudes espirituoses.

Amb un somriure va optar per un vas de vi vermell. Va fer-se copets als
auriculars.

--Un cantant meravellós, però va morir jove. Jo sempre he dit que és una
tragèdia quan tanta gent ha de morir amb tant potencial sense completar.

Amb un gest del cap, l'hostessa es va moure cap al següent passatger.

El senyor Murakami va continuar escoltant la música, amb la seva ment
gradualment tornant-se corrupta per tots els missatges subliminals que
el donaven, i no hi havia res que pogués fer.

Mentre el Doctor deia la paraula ``Mandràgora'', la Madam Delphi va
crear ones plenes de vida i emoció a la suite de l'àtic del Hotel Oracle
darrere de la milla daurada de Brentford.

--Dara Morgan! --va exclamar--. Ell sap que existeixo!

En Dara Morgan va creure que l'ordinador s'hauria remogut emocionat si
hagués estat possible.

--Qui?

--El Doctor. Després de tots aquests eons, després de tota aquesta
distància, les estrelles estaven alineades tal i com vaig predir. Els
horòscops de demà seran molt diferents ara.

I a la pàgina web de la Madam Delphi, llegida per gent d'arreu del món,
les prediccions del diumenge per cada signe del zodíac es van
reescriure.

Tot el que ara deien era: ``Benvinguts de nou! La teva vida canviarà
durant les properes 48 hores de maneres que mai podries imaginar.
Aprofita aquest canvi i prepara't per a la següent i millor fase de la
teva vida mentre Mandràgora s'empassa els cels, i et somriu des de les
altures.''

En quinze minuts, un home a Ciutat del Cap havia posat aquestes paraules
a una samarreta. Una dona de Paris estava creant un grup de Facebook per
Mandràgora. I a Milwaukee tres joves estaven fent un graffiti amb la
paraula ``Mandràgora'' a les parets de la seva escola.

El Doctor s'estava exasperant amb en Cedric Crossland, i l'altra dona
guillada, l'Ariadne no-sé-què, s'encarregava de donar maldecaps a en
Wilf, així que els va deixar tots per al Doctor i va anar a fer un tomb
per la zona del bar exterior.

Va veure la Donna i la Netty, i va saber d'immediat que la Netty se
n'havia anat, de nou al seu món propi. I el cor d'en Wilf bategava amb
una mica més de dificultat perquè, encara que ho havia vist centenars de
cops, cada cop que ho feia, es feia a si mateix una pregunta: què
passava si aquell era l'últim cop? Què passava si ella se n'anava i mai
no tornava?

La Donna va fer un somriure mentre ell s'apropava, amb el seu braç
agafat fermament al voltant d'aquella vella dona que amb dificultat
coneixia però que havia adoptat sota la seva aprovació només perquè el
seu boig i ancià avi l'agradava.

En Wilf va arrossegar-hi un tamboret i es va asseure mirant-les a
ambdues.

-- Ets molt bona, Donna--va dir-li--. No has de fer això. Ella no és
família.

--Sí, potser no ho és, però és important. Per a tu. I ho haig de
reconèixer, en el fons, per a la mare.

--Oh, ja coneixes a la teva mare, sempre ploriquejant, sempre
queixant-se però al fons de tot hi ha\ldots{}

--Més ploriquejos i més queixes? --la Donna va començar a riure fort--.
Déu, l'estimo, avi, però hi ha vegades que també em treu de polleguera.

--Aquí, no en parlaré d'això jo pas, Donna. La teva mare és molt bona.
Parla amb sinceritat, però això no és res dolent. I ella també t'estima.
Ella senzillament no sap com lluitar amb moltes coses des que
Geoff\ldots{} ja saps\ldots{}

--Va morir? Ho pots dir.

--Des que el teu pare se'n va anar, sí.

La Donna va aixecar el braç de la Netty i el va allargar per agafar-li
la mà al seu avi.

--Així que, com vas conèixer la Netty?

En Wilf va somriure.

--Ella em va escriure a mi després que se'm publiqués una carta a la
Revista.

--Ah, la Revista. Encara rutlla?

--Faran seixanta anys l'any vinent. Aquest és l'Any Internacional de
l'Astronomia, és tot el doble d'important. Els vaig escriure una carta
sobre la Triple Conjunció! Júpiter i Neptú! Netty la va veure, no va
estar d'acord amb els meus pensaments sobre com de difícil seria
veure-ho amb un telescopi com el meu i BANG, vam tenir una petita guerra
per mitjà de tres assumptes. Llavors un dia em va telefonar del no-res,
vam prendre un cafè a Londres per arreglar les nostres desavinences i
una setmana després va fer servir les seves influències per fer-me
membre de la Reial Societat Planetària. I aquí estem.

La Donna el va somriure.

--Així que, quan vas descobrir el del seu Alzhèimer?

--Oh, m'ho va dir ella a la nostra segona\ldots{} trobada.

--Anaves a dir ``segona cita'', veritat? Oh, vella guineu astuta! --la
Donna se'l va quedar mirant--. Estic feliç per tu, avi. De trobar una
amiga, algú amb qui t'agradi estar. I segur que la mare també ho està.

--Oh, ho sé. Només està preocupada per la seva malaltia, i sobre tota la
pressió que em pot provocar a mi. Ella i la Netty parlaven de
residències, però jo no n'acceptaré cap.

Ell va mirar l'Henrietta Goodhart.

--Ella és una gran dama, Donna. M'agradaria que la veiessis com jo la
veig.

--Ho faig. Ahir a casa i aquest matí. És encantadora, i crec que no
l'hauries de deixar escapar.

I en Wilf es va sentir tant i tant trist que va dir:

--Però un dia, la perdré. És inevitable. Ho vaig buscar a Internet.

--Oh, bé. Ha de ser veritat, llavors.

--De veritat, no és bo. No vull dir que es vagi a morir, però la perdré
perquè un dia es quedarà atrapada on sigui que vagi i no tornarà. No
tenim medicines, el coneixement per guarir-ho. No és just-- llavors un
pensament se li va aparèixer--. Segur que allà fora, a les estrelles,
segur que la poden tractar. Segur que hi ha alguna cosa\ldots{}

I la Donna li va estrènyer la mà.

--No funciona així, avi. Déu ho sap que tot el que he vist, tota la gent
que he conegut, hi ha hagut vegades que he cregut que deu haver
solucions a les malalties, les fams i tots els tipus de coses molestes.
Creia que si cridava al Doctor prou alt trobaria una forma. Però no
importa si ets de Chiswick o de Cestus Minor, no hi ha respostes fàcils.
Senzillament hem de lluitar amb el que el destí ens dóna\ldots{}

--No és just\ldots{} --va repetir.

--No. No ho és. I ho sento molt, moltíssim per tu, avi. Perquè t'estimo
i perquè de veritat m'agrada la Netty i si pogués trobar una forma de
fer-ho tot senzill per a vosaltres dos, ho faria de veritat, de veritat.
I saps què? El Doctor no us coneix tan bé, però segur que ho intentaria
deu cops més que jo. I probablement no faria cap diferència, així que no
hi ha cap raó de donar-hi voltes sobre quelcom que no en tenim el
control.

En Wilf mirava a la Donna, i es preguntava què li havia passat a aquella
estúpida i violenta noia que havia estimat però per la que s'havia
preocupat tots aquells anys. Ara era una fina, valenta, brillant i jove
dona. I l'estimava encara més.

I llavors hi havia la Netty.

Va apartar la mà de la de la Donna i va agafar les dues de la Netty
entre les seves.

--Ei, tu--va dir amb calma--. Henrietta Goodhart, crec que és hora d'una
cançó, com ens agradava cantar als vells temps, oi?

La Donna va arrufar les celles en confusió, però ell li va picar
l'ullet.

--Sé el que em faig.

Suaument va començar a taral·lejar una melodia. Un antic himne de
gospel.

--Quan les estrelles comencen a caure--va començar a cantar
tranquilament--. Oh, Senyor! Quin matí, oh Senyor! Quin matí\ldots{}!

Ell va mirar a la Donna.

--Ella em va dir que el seu marit cantava això quan era amb ella, durant
la guerra.

--Creia que no s'havia casat.

En Wilf va somriure amargament.

--No li diguis mai a ningú el que et diré ara, reina. Es va casar.
Durant tres dies. I ell va ser assassinat a Singapur, quan va començar
el bombardeig. Em va dir que van cantar això el dia de la seva boda, pel
camí a un gran Rolls platejat. Té una foto d'això i me la va ensenyar:
era preciós. I llavors, quan van intentar fugir de Singapur, ell va
morir agafant-la de la mà i ella li va cantar això mentre ell mori a la
seva falta--va tornar a mirar a la Netty--. No li diguis mai a ningú que
t'ho he explicat, i molt menys a ella. Promet-m'ho. Ella l'estimava
moltíssim i es va jurar que mai no tornaria a casar-se.

--Per suposat que ho prometo, avi. Per suposat que sí.

Ell va començar de nou.

--Oh, pecador, què faràs quan les estrelles van començar a caure\ldots{}
Oh, senyor, quin matí!

Els ulls de la Netty van semblar centrar-se, i va respirar profundament,
com si es despertés.

--Me n'he anat, veritat? Oh, no, no em digueu que he estat pels carrers
sense portar res és que la meva roba interior-- va mirar a la Donna i va
parpallejar--. Un altre cop!

En Wilf va somriure, i una llàgrima gairebé li queia del seu ull, així
que se la va apartar abans que cap de les dues el poguessin veure.

--Crec que hem de tornar a la festa, per rescatar al Doctor, d'acord?

La Netty es va aixecar i va deixar que en Wilf fos primer. Es va aturar
una mica i es va recolzar en la Donna.

--Em canso cada cop més--va dir--. Ah, i gràcies.

--Per què?

--El seu nom era Richard Philip Goodhart. I el teu avi és l'única
persona que he conegut que s'hagi apropat gaire.

Quan van arribar al vestíbul principal, el sopar ja havia acabat.

En Wilf i el Doctor s'apropaven a la paret més allunyada i en Wilf es
disculpava perquè l'Ariadne Holt i en Cedric Crossland havien negat de
prendre's seriosament al Doctor.

--M'avergonyeixo de conèixer-los.

El Doctor el va mirar amb tristor.

--No ho estiguis. Són bona gent. Alguns són una mica estranys, però de
cor són meravellosament normal. Per què haurien de creure'm?

--Bé, hem tingut naus espacials i Sontarans i aquestes coses els últims
anys.

--No hi ha justificació per a l'habilitat de la humanitat de
racionalitzar les coses, Wilf. Del que un grup de gent s'espanta, un
altre grup no hi veu perill perquè és dintre de la seva zona de confort.
Aquesta gent són meravellosos pioners, amants de les estrelles, les
constel·lacions i observen i anoten i cataloguen els cels. Igual que tu!
Res d'això hauria d'acabar mai, és massa important, encara si les coses
no poden ser recognoscibles durant un parell de segles. Ningú no es va
prendre Galileu, Copèrnic o l'Organon d'Aristòtil seriosament en els
seus propis temps.

--Et prens la seva grolleria massa bé, Doctor.

El Doctor es va encongir d'espatlles.

--No és personal. La gent com el doctor Crossland senzillament no volen
contemplar coses que cauen fora de la seva esfera de referència. En els
pitjors dels casos, imprudents; en els millors ignorants--llavors va
mirar al got de llimonada a la seva mà--. Normalment.

--Però això de la Mandràgora, no és normal, oi?

El Doctor va negar amb el cap.

--És una entitat maligna, Wilf. L'últim cop que va ser aquí amb força,
va morir molta gent. Però intentava deturar el Renaixement italià, per
detenir que la ciència arribés a l'estat al que està avui dia. No puc
saber què intentar fer avui. Ves-te'n quaranta anys enrere i detura el
transistor o el microxip i sí, hauràs fet fracassar la propera generació
del progrés humà. Però aquí? Res particularment especial passa aquest
any, ni tan sols aquesta dècada, que realment pugui afectar al futur de
la Terra tant. Espereu un parell de segles o així. Arribareu a Mart, un
parell de vols espacials més llargs i\ldots{}

--Mart? Arribem a Mart? Trobem marcians?

--Spoilers--va picar-li l'ullet--. Els meus llavis estan segellats--va
empassar-se la llimonada--. Així que no estic segur de si deixar sola
aquí a Mandràgora Hèlix i assumir que només està vigilant les coses, o
preparar-me per a la gran batalla.

--Potser té alguna cosa a veure amb aquests nois Carnes, Doctor. Has dit
que creies que tenien alienígenes a la família.

--Oh, sí! I com és que en Joe Carnes sabia el meu nom? --el Doctor va
fer un sospir--. Oh, Wilf, Wilf, Wilf! Acabes d'arruïnar una nit
perfectament agradable.

--Jo? Com? I, eh, ho sento.

--Perquè acabes de veure un petit i diminut, però que molt diminut
defecte a la meva lògica. Mandràgoda està enllaçada, en una forma
estranya, amb l'astrologia, no només amb l'astronomia.

--L'astrologia no té cap sentit.

--Bé, la major part d'això es crea als diaris. Però té el seu origen a
l'Època Fosca, així que probablement hi ha quelcom en això. Torna al
naixement de l'univers i veuràs cada societat, cada civilització té
alguna forma de zodíac, una creença en un poder de llums antigues
enllaçades a algun tipus de sistema de creences basades en el moviment
dels planetes i les estrelles i els canvis de les constel·lacions. Els
astròlegs del planeta Hynass juren cegament que no hi ha res com la
coincidència i tenen absoluta fe en el coneixement de que des que el Big
Bang va estar divinitzat, és qüestió de destí preestablert de que ningú
pot escapar-ne. Ara pots pensar que no té sentit i jo puc pensar que
tampoc té gaire sentit, però Mandràgora s'aferra en aquesta creença, en
aquest sistema sense provar, i el fa servir. Causa i efecte.

En Wilf va arrufar les celles.

--Però segueix sense tenir sentit.

--Oh, sí! Sí, per suposat que no en té. Sense sentit! Bé, probablement.
Això no detura de que Mandràgora sigui capaç ~d'aprofitar-se en aquestes
energies.

En Wilf es va arronsar d'espatlles.

--El que diguis, Doctor.

La Donna es va apropar.

--Avi, crec que la Netty necessitaria un recolzament en contra de les
opinions d'aquella vella boja bruixa sobre el lloc d'una dona en la
societat moderna.

En Wilf va fer que sí am ble cap.

--Brindo per tu, Doctor. Espero que t'equivoquis, per cert--i es va
allunyar.

--De què anava tot això, maco? Estaves preocupant al meu avi?

El Doctor va fer que no amb el cap.

--No, Donna, no del tot. M'ha fet pensar sobre la coincidència i la
casualitat--va mirar a la Netty--. Com està?

--No estic segura. Ella se n'ha anat durant una estona però llavors ha
semblat que ha tornat, somrient i m'ha endarrerit un moment.

--Passa, em temo--va dir, sense deixar d'observar-la mentre posava un
braç al voltant de la cintura d'en Wilf--. I no te n'oblidis, ella ja
està acostumada.

La Donna el va agafar de la ma.

--I hi ha una cosa més. Al bar. Saps el paio ben plantat que va
aparèixer abans?

-- En Gianni?

--Sí, ell. Anava a dir alguna cosa sobre algú.

--Qui?

--No ho sé, no estic segura de que estigués parlant però ha semblat de
ser alguna cosa sobre un home llepant un dofí boig.

El Doctor es va encongir d'espatlles.

--Podria ser qualsevol cosa. Probablement ha begut massa.

--O treballant amb aquesta gent l'ha embogit--va somriure la Donna--.
Oh, bé. No estic segura de perquè un home lleparia un dofí boig, per
cert.

El Doctor va riure.

--Ni jo tampoc. Hauríem de pensar en anar-nos-en d'hora, per això.

--Per què?

--Té alguna cosa a veure amb una entitat alienígena antiga i perillosa
suspesa no massa lluny del teu planeta que no sembla que es vulgui
amagar massa.

--Com de perillosa?

--Bé, estarà esperant per alguna cosa com un eclipsi lunar el qual,
mirant a la lluna aquesta nit, no sembla ser especialment imminent.

--Sempre hi haurà la Triple Conjunció.

--La què?

--L'avi me n'ha parlat, és per això pel que tots estan tan emocionats
pel descobriment de la nova estrella. És l'Any Internacional de
l'Astronomia, i estant tots esperant de veure la primera triple
conjunció entre Júpiter i Neptú.

Donna estava prou orgullosa d'haver retingut tota aquella informació,
però el Doctor es dirigia amb gambades cap al Doctor Crossland.

--La Triple Conjunció--li va cridar--, quan és?

--Perdó?

--Aquest any, oi? Però quan aquest any?

En Crossland va fer un sospir.

--Creia que se suposava que eres intel·ligent.

--Ho sóc, però com totes les persones intel·ligents, només puc aprendre
les coses quan la gent em dóna respostes directes a les preguntes
directes i no sarcasme.

El Doctor Crossland semblava triomfant. Havia superat en intel·ligència
al Doctor.

--Bé, si sabessis tant sobre astronomia com dius saber, sabries que està
tenint lloc ara mateix. Va començar fa una mica.

--Quan va ser vist el primer Cos del Caos? ~

--Ho suposo, sí.

--I quan té el seu punt més àlgid?

--És per això pel que estem tenint aquest sopar, Doctor. Els
esdeveniments principals tenen lloc el dilluns, sobre les tres en punt
de la tarda, hora local.

--Sembles estar molt estufat, Doctor Crossland--va dir el Doctor--, per
a ser un home que pot estar mort en quaranta-vuit hores. Calculo que en
unes cinc o sis hores.

El Doctor Crossland va arrufar les celles.

--És una amenaça?

--Sí--va dir el Doctor--. No meva, meu és el consell. L'amenaça és de
Mandràgora. Del teu Cos del Caos. És aquí per matar-nos a tots.

Va girar-se a tota velocitat cap a en Wilf, la Netty i l'Ariadne Holt.

--Sento trencar la festa, Wilf, però haig de marxar. Pots portar a la
Netty a casa sana i estàlvia, Donna?

--Oh, Doctor, ves-te'n, no et preocupis per nosaltres. Tinc un taxi
llogat per portar-me a casa a les onze de qualsevol manera--va dir la
Netty. Va tocar el braç d'en Wilf--. I ni intentis discutir amb mi,
Wilfred Mott. Aquesta és la teva nit, així que no vull que et sentis
responsable de mi aquesta nit.

En Wilf va mirar-la a ella i al Doctor.

--En Wilf no pot marxar--va dir l'Ariadne Holt--. Encara no hem fet les
presentacions.

--Només ens anem nosaltres--va dir la Donna--. L'avi es queda.

--I una merda que ho faré--va dir en Wilf--. Vaig amb vosaltres.

--Ningú vindrà amb mi--va dir el Doctor, però ningú l'escoltava.

Donna el va apropar.

--Avi, la Netty ja ha tingut un\ldots{} encanteri aquesta nit. Hauries
d'estar amb ella. ~Assegura de que torna a casa. Ves amb ella al taxi,
llavors queda't el taxi i torna a casa, d'acord? --va buscar una cosa a
la seva bossa i va treure tres bitllets de deu--. No sé si és prou, però
hauria d'ajudar.

En Wilf va rebutjar els diners.

--Puc pagar-ho jo mateix, gràcies, reina.

--Sí, estic segura de que pots--va dir la Donna--. Però agafa'l, només
perquè no tingui a la meva consciència de que t'has quedat en algun lloc
i has d'arrossegar a la mare fora del llit per anar a buscar-te,
d'acord?

En Wilf va mirar a la seva neta, llavors al Doctor, que intentava fingir
que havia trobat quelcom interessant al sostre. Va agafar els diners.

--Truca'm--va dir--. Tinc el meu mòbil.

--Sé que el tens. I estarà apagat o tindrà poca bateria. Igual que
sempre. Estarem bé, et veuré demà al matí--el va fer un petó, i llavors
a la Netty i va agafar la mà del Doctor--. Vinga va, és hora
d'anar-nos-en.

El Doctor va dir adéu a la Netty, a en Wilf i a l'Ariadne Holt mentre la
Donna l'arrossegava a través de la porta i de nou a l'entrada, passat el
porter i de nou a l'exterior, al fred aire octurn.

--Anava a anar-hi jo sol--va protestar, però la Donna ja havia aturat un
taxi (bé, es va posar dempeus al mig de la carretera de South Lambeth i
va xiular-ne un que va prendre la bona decisió entre aturar-se,
ignorar-la o atropellar-la).

La Donna es va ficar, arrossegant al Doctor amb ella.

--Cap a on? --va preguntar el taxista--. Acabo el meu torn d'aquí a no
res, així que millor que no sigui gaire lluny.

--Al Chiswick High Road--va dir-li el Doctor, i afegint per a la Dona--.
Necessito el meu TARDIS.

El conductor va arrancar i va conduir fins el pont de les vies del tren
i va anar cap Nine Elms i a l'oest de Londres.

El primer informe va arribar a les 23.04. era de l'Observatori
Clemenstry a l'Austràlia occidental. Va informar que la nova estrella,
la que havia aparegut als cels feia una setmana més o menys, semblava
estar-se movent en conjunció amb una altra estrella, la M84628·7.

El qual era bastant inusual, va declarar el professor Melville, apuntant
a la pantalla del seu ordinador amb la seva ploma. Ell estava a la seva
oficina al Copernicus Array d'Essex, però probablement volia estar a la
sala de control del telescopi ell mateix. Hi solia estar.

--Aquest és el problema amb aquestes estrelles noves, aquests Cossos del
Caos--va dir a la seva jova ``adjuvant'', la senyoreta Oladini--. Són
caòtics i no tenen cap sentit, científicament parlant. No estàs d'acord?

--Just a la fusta, professor--va dir ella, sense tenir ni idea del què
parlava. Ella només era allà per un contracte a curt termini de
l'Agència Lovelace a Brentwood, trobant llocs de treball temporals per
aprendre noves habilitats. ``Noves habilitats''\ldots{} en tenia 25 i
encara necessitava ``noves habilitats''. Era vergonyós d'alguna manera
però no estava ni remotament interessada en astronomia però no tenia la
força suficient per dir-li-ho a en Melville. En lloc, l'alimentava i el
donava de beure amb barretes de xocolata i te i l'escoltava parlar sobre
la seva gata i la seva mare. (Ell vivia amb una i parlava d'haver de
posar a l'altra a dormir perquè tenia malament els ronyons, però la
senyoreta Oladini encara no estava totalment segura de quan en parlava
de quina. Tenia la lleugera sospita, de totes maneres, de que la gata
era la que tenia bona salut).

El professor Melville era un home vell molt simpàtic. Emfatitzant en
``simpàtic'' i en ``vell''. Deia que havia estat una estrella del pop
als seixanta, però no estava segura de creure'l. A la senyoreta Oladini
li agradava, encara que sospitava que ell estava només contractat al
Copernicus Array (perquè estava clar que havia passat de la seva edat de
jubilació) només per compassió. Probablement per que s'encarregava de
les guàrdies nocturnes, era millor tenir-lo al lloc.

El Copernicus Array mateix era un radiotelescopi construït als jardins
d'una antiga mansió d'estil georgià que havia estat convertida en les
oficines, sales de reunió i tot el de l'Array. Una llàstima, creia la
senyoreta Oladini. Era una encisadora casa antiga, sovint li agradava
visitar les cases antigues i encara que molt havia estat invertit en
preservar els elements fixos i el mobiliari originals, aquell lloc
semblava estèril i faltat de caràcter natural. Sovint es preguntava qui
havia viscut allà feia centenars d'anys, què els havia passat i com
s'haurien sentit al saber que les seves calaixeres, les seves cuines,
les seves habitacions i les seves sales de ball havien estat convertides
en sales plenes de científics avorrits i administradors.

A les 23.09, l'informe havia arribat de l'Observatori Griffin a
Maryland. També mencionava un Cos del Caos movent-se cap a l'alineació
amb una altra estrella. Però era una diferent a la M84628·7 de
Clemenstry. Aquella era la M97658·3. El que era evidentment absurd.

--És que han estat bevent tots aquests? -- es va preguntar en Melville.

Allò li va semblar una molt bona idea a la senyoreta Oladini, tot i que
ella només volia anar-se'n cap a casa al seu llit. No és que estigués
massa disposada a pedalejar a la fosca i, a aquelles hores de la nit, hi
havia molta gent a les carreteres que eren pitjors que la por.

A les 23.17, l'informe va arribar del Projecte Tycho, vora Beaconsfield.
El Cos del Caos s'aprovava cap a la M29034·1.

Per suposat, tot allò no va ser simultani, després de tot, no era fosc a
Califòrnia o a Perth en aquell moment, era només que en Melville feia
una guàrdia nocturna i només havia encès el seu ordinador.

--Tot el que necessitem ara, senyoreta Oladini, és que Minsk ens
ofereixi alguna cosa ridícula i\ldots{}

I just a les 23.19, el Colossus de Minsk va donar detalls de com el Cos
del Caos estava en alineació amb la M23116·3.

Ara en Melville estava alerta i encuriosit. La senyoreta Oladini també,
malgrat la seva falta d'interès en l'astronomia, perquè ella havia estat
una estudiant de matemàtiques (per això estava ella allà) i va calcular
totes les probabilitats de que un Cos del Caos de sobte formi una nova
constel·lació amb quatre estrelles que abans ja existien tot a la
mateixa nit i era\ldots{} bé, eren més probabilitats que espais numèrics
a la seva calculadora.

En Melville mirava les últimes fotografies a la gran pantalla que
dominava una paret de la seva oficina. Era de fet una gran pantalla, una
d'aquestes tecnologies d'alta definició per la que molts altres
observatoris i radiotelescopis al voltant del Regne Unit donarien molts
ronyons. Tot el que en Melville havia de er era seguir una línia
invisible de la pantalla del seu portàtil cap a la gran pantalla a la
paret i les imatges i les paraules fluïen de una a l'altra com si fos
alguna cosa d'una pel·lícula de ciència ficció. En Melville estava
orgullós del software, però no en tenia ni idea de com funcionava. Ell
només sabia que ho feia i això significava poder moure imatges al
voltant d'una pantalla tan gran com la paret sense deixar la cadira. I
era el que feia en aquell moment.

Primer es va centrar en la seva pròpia fotografia de l'estrella del
Caos. Llavors va sobreposar-hi la de Clemenstry. Llavors la de Griffin,
la de Tycho i finalment la del Colossus.

--Professor\ldots{}?

--Ho sé.

--Però això és.

--Ho sé.

--Vull dir, com\ldots{}?

--No ho sé.

--No havia vist mai res com això.

--Ho sé.

En Melville va agafar un telèfon del seu escriptori. Era vermell. I
mentre teclejava els números va aixecar la vista cap a la senyoreta
Oladini.

--Ha signat l'ASO, senyoreta Oladini?

Ella va arrufar les celles.

--Si he fet què?

--L'Acta de Secrets Oficials. Te l'han fet signar quan vas signar els
papers per a acceptar el lloc de treball quan vas arribar a l'agència?

La senyoreta Oladini va pensar-ho per un moment. En Melville s'espantava
de la seva pròpia pregunta. Normalment, ella un vell molt simpàtic, una
mica grillat, una mica dispers, fumava de la pipa massa sovint. Però ara
estava en alerta i treballador, rígid i tot el sentit de l'excentricitat
se n'havia anat.

I la senyoreta Oladini es va adonar de que aquell vell estúpid, exigent
i grillat era una actuació. A sota d'allò, el professor Melville era
afilat com un xiulet. Potser de veritat havia estat un cantant de pop.

--I bé?

Ella va fer que sí amb el cap. Havia signat alguna cosa que hi tenia la
paraula ``oficial'', d'allò se'n recordava. Francament ella no n'hi
havia prestat gaire atenció quan la senyora Lovelace a l'agència l'havia
trucada per signar els papers. Tots els treballadors temporals signaven
trossos de paper per salut i seguretat, assegurances i tot aquell tipus
de coses, quan els cridaven per a treballar. Una extra no significava
gaire en aquell temps.

Ara semblava quelcom gran i esgarrifós.

--Per què?

--Perquè sense la teva signatura al final d'aquest imprès, el que estic
a punt de fer i el que estàs apunt d'escoltar ens ficaria a tots dos a
la presó per la resta de les nostres vides si no ho has fet.

I la senyoreta Oladini ho va pensar de debò. El seu cervell era bo amb
els números.

--L'imprès KD62344-- va dir de cop--. La vaig firmar dos cops i hi vaig
ficar les inicials a un requadre, al marge a l'esquerra.

En Melville va parpellejar.

--Gràcies--va pitjar un últim número al telèfon--. Amb l'Aubrey
Fairchild, si us plau. Sóc del Copernicus Array, codi 18. Em dic
Melville.

La senyoreta Oladini va mirar cap enrere cap al conjunt d'imatges. El
Cos del Caos a més de les altres estrelles separades no significaven
res. Però ara que en Melville les havia posat juntes, formaven una
imatge. I no qualsevol sense sentit abstracte que la gent veia com una
parella de peixos, o una arada o un coet.

Allò era clara, distintiva i agudament una cara. Una cara amb una boca
girada en un riure.

La senyoreta Oladini va tenir un calfred perquè el riure no era un riure
feliç. Era de pura malevolència.

--És el primer ministre? Sóc en Melville, professor administratiu del
Copernicus Array. L'envio una imatge de codi 18--hi va haver una
pausa--. Sí, senyor. No, senyor, les imatges només les he combinades
aquí, segueix sent una amenaça a Regne Unit però doni-li un parell
d'hores\ldots{}

El Professor Melville va mirar cap a la senyoreta Oladini.

--No, senyor, només jo mateix i la meva adjuvant. Ens hi quedarem fins
que escoltem sobre la seva gent, primer ministre. No, absolutament,
tancament total, sense comunicacions entrants o sortints del Copernicus
Array sota cap circumstancia. Bona nit, senyor.

El professor Melville va tornar al lloc el telèfon.

--Era de veritat el primer ministre? --va preguntar la senyoreta
Oladini.

En Melville va fer que sí amb el cap.

--Ho sento, reina, però crec que ens espera una llarga nit. Podries anar
a comprovar si tenim bastant te i llet?

La senyoreta Oladini va començar a abandonar la sala, però llavors es va
girar i va veure a en Melville agafar el seu telèfon mòbil de la butxaca
de la seva jaqueta. Ni tan sols havia sabut que en tenia un.

--Professor? No li acaba de prometre al senyor Fairchild que no tindríem
cap comunicació?

Ell va somriure.

--És això pel que necessito que comprovis el te. Si no ets a la sala, no
podràs ser responsable quan trenqui la promesa, cometi traïció i
segurament un suïcidi professional. Ara, per la teva seguretat,
ves-te'n.

Confosa, la senyoreta Oladini va deixar l'oficina. Però va esperar fora
de la porta, per veure si podia escoltar amb qui parlava.

Va escoltar la típica música de les tecles pitjades i llavors, després
del que van haver de ser un parell de rings, va parlar.

--Bona nit, el meu nom és el Professor Melville. Puc parlar amb el
Doctor, per favor?

El taxista estava a meitat de la carretera Prince Albert, entre Vauxhall
i Chiswick, anant gairebé a 20 pels turons de velocitat.

Al final del taxi, la Donna va treure el seu mòbil mentre sonava i va
mirar el número, tot i que no el reconeixia. Amb un encongiment
d'espatlles, va pitjar la tecla d'acceptar.

--Hola?

Ella va escoltar-ho i li va passar amb el Doctor.

Ell va somriure.

--Tinc el TARDIS connectat al teu número. T'ho hauria d'haver dit. Ho
sento.

--Com t'ho feies abans dels mòbils? --va sospirar ella--. El Professor
Melville, suposo.

El Doctor va somriure.

--I hi ha una altra coincidència, és una d'aquestes nits, oi? --va dir
abans de parlar pel telèfon--. Correcte! Què puc fer pel Copernicus
Array aquesta nit? Deixi-m'ho endevinar. Té a veure amb les paraules Cos
del Caos?

La senyoreta Oladini estava tenint prou sorpreses aquella nit. Primer
les estrelles fent imatges. Llavors el grillat vell professor Melville
parlant per la línia calenta amb el primer ministre. Llavors descobrir
que la llet de la cuina s'havia acabat per primer cop. Oh, i llavors el
seu xicot trucant-la al bell mig de mitjanit.

--Spencer? Què vols?

--No em diguis que no heu fet servir aquest maleït gran telescopi per
mirar al cel?

La senyoreta Oladini va creuar la porta de la cuina, la que portava cap
a l'exterior i no al passadís i s'hi va abocar cap al fred aire de la
nit, sense pensar que veuria res amb l'ull nu.

Així que es va sorprendre bastant quan ho va veure.

--És algun tipus de focs artificials? --va preguntar l'Spencer.

Es va preguntar què dir-li. Deu minuts abans, els cels havien estat
clars i la cara rient només es podia veure a les imatges. Ara era
visible a través dels cels, i si el seu estúpid xicot l'havia vista, i,
sense donar-ho per perdut com un borratxo trinxeraire, llavors alguna
cosa havia d'estar feta. Va recordar l'Acta de Secrets Oficials ~i el va
dir a l'Spencer que sí, que probablement era alguna cosa projectada a la
teulada d'algun dels cinemes de la carretera de Dagenham.

--Alguna cosa promocional d'una pel·lícula--va dir, afegint--, i ens van
trucar ahir avisant-nos-ho--que era una mentida terrible, però va
semblar que feia feliç a l'Spencer.

Després depengés, se li va ocórrer que la resta de Gran Bretanya no
podria estar llesta de creure's la seva història, al menys perquè no
podria telefonar a cada persona individualment i explicar-los-ho.

Potser ho podria fer el primer ministre. O en Patrick Moore.

--Com va el te?

La senyoreta Oladini es va girar a veure en Melville a la porta, amb un
gran somriure a la cara. Ella li mencionà que la cara de la imatge de la
fotografia ara era visible al cel, i en Melville es va passejar cap a la
finestra i s'hi va abocar.

--No et preocupis--va dir ell--. Un expert ho està mirant.

--De Downing Street?

--Oh, no. Algú millor equipat que ningú aquí--va tornar a mirar fora de
la finestra--. És una cosa d'aparença terrible, no creus? --va dir,
mentre la tetera començava a bullir de nou--. Era llet sense sucre, oi,
senyoreta Oladini?

Ella va fer que si amb el cap sense dir res.

--Ara ja no estic preocupat--va afegir--. El Doctor ho arreglarà.

Hi va haver uns flashos de llum del exterior, i van escoltar un parell
de cotxes frenar a l'aparcament.

--Quina velocitat--va murmurar en Melville, arrufant les celles
lleugerament.

--Downing Street o el teu amic Doctor? --va preguntar ella, intentant no
sonar nerviós--. O potser algú més d'aquí al voltant, preocupat per la
cosa al cel?

Però en Melville no escoltava de veritat, estava mirant qui era dins
dels cotxes.

Els fars davanters van ser deixats encesos, il·luminant a través de la
finestra de la cuina, així que no podien veure qui s'havia baixat dels
cotxes, només escoltaven els sons de les portes obrint-se i tancant-se.

La senyoreta Oladini va tremolar. Allò estava\ldots{} malament.

En Melville presumiblement coincidia amb ell, perquè de sobte va córrer
cap a la porta exterior, com per a tancar el pany. Va arribar massa
tard. Algú va obrir la porta.

En Melville es va posar davant de la senyoreta Oladini, protegint-la
cavallerosament de qui fossin els nouvinguts.

Eren\ldots{} bé, molts d'ells, de diferents edats, i no semblaven gaire
especialment\ldots{} governamentals. O terriblement amenaçants.

Però hi havia alguna cosa sobre ells, va pensar la senyoreta Oladini,
alguna cosa sobre la forma amb la que miraven la cuina, els seus caps
movent-se de forma antinatural com si veiessin l'interior de la cuina
per primer cop.

I el vell home gras era al front, amb un home de mitjana edat darrere
seu. Darrere d'ells hi havia una dona gran i un trio de nois estudiants,
dos homes i una dona.

--On són la resta? --va preguntar el primer home vell amb un fort accent
americà.

El Professor Melville es va aclarir la gola.

--Això és propietat privada. Som al Copernicus Array. No poden aparèixer
de cop a meitat de la mitjanit i\ldots{}

La jove estudiant es va obrir pas entre la resta, mirant en Melville com
si no pogués entendre perquè estava allà dempeus.

--Tu\ldots{} treballes aquí? --va dir.

Era anglesa, aquella.

Ell va fer que sí amb el cap.

--Estic a càrrec del radiotelescopi, i aquesta nit estic a càrrec del
projecte sencer. Hauria d'advertir-vos que hem fet sonar les alarmes i
la policia estarà aquí en uns minuts.

Un altre estudiant va parlar. La seva veu sonava europeu, espanyol o
potser italià.

--No puc detectar cap sistema d'alarma. L'humà està mentint.

Humà? Quina expressió més estranya.

Però el professor Melville semblava impertèrrit per l'elecció de les
paraules, de fet, semblava gairebé que ho gaudís.

--Quin planeta representeu? Esteu connectats amb la constel·lació del
Caos?

El gran americà va parlar de nou.

--Ens pots servir.

Sense evitar un batec, en Melville va assenyalar a la senyoreta Oladini.

--I la meva adjuvant. Si necessiten la meva ajuda, la necessitaran també
a ella. Tot al Copernicus requereix dos operadors experimentats.

Potser estava boig, va pensar la senyoreta Oladini.

--Ella és una física altament qualificada i una experta en el camp de
les ciències còsmiques.

Totalment boig.

--Hi ha quelcom estrany aquí--va dir la dona gran, també americana,
caminant cap endavant. Sense cap advertència, es va apropar i va tocar
l'espatlla del professor Melville. A primera instància, la senyoreta
Oladini va suposar que ambdós havien estat electrocutats amb una forta
guspira morada que havia estat llençada entre ells, i en Melville va
trontrollar lleugerament, tossint amb dolor.

--Professor? --la senyorieta Oladini va cercar la seva veu, però
immediatament es va penedir.

En Melville es va girar per mirar-la i, durant un segon, ella encara
podia veure-hi vestigis d'aquella llum morada als seus ulls.

--D'acord, he mentit. Es la meva adjuvant temporal--va dir en Melville
sense respiració--. No entén res del Copernicus Array i és inofensiva.

--La podem absorbir.

--Deixeu-la anar, si us plau\ldots{}

En Melville es va ensorrar i els tres estudiants ~van fer una passa
endavant mentre s'aprovaven cap a ella.

La senyoreta Oladini va fer una cosa que cap dels munts de xerrades amb
la senyora Lovelace a Brentwood, o els impresos sobre Salut i Seguretat
o les 100 paraules per minuts l'havien preparat. Va girar en rodó i va
córrer per la seva vida, estrepitosament per la cuina i de nou al cor
dels passadissos i les oficines del Copernicus Array. D'alguna manera
sabia que estava corrent per la seva vida.

El taxi conduïa a través de les carreteres i va passar el centre
d'exposicions Earls Court quan el Doctor va colpejar amb els dits contra
el vidre i va preguntar-li al conductor:

--Pot parar només un minut a la cuneta, si us plau?

--I bé? --va preguntar la Donna al Doctor, mentre el conductor frenava.

--Eh, bé\ldots{} De fet tu has d'anar cap a casa perquè jo vaig a
demanar a l'home amable de portar-me a un altre lloc.

--A la carretera major de Chiswick?

El Doctor va arrufar les celles, llavors va recordar on tenia el TARDIS
aparcat.

--No, no, de fet una mica més lluny que això.

La Donna se'n va adonar.

--Vas a veure al paio aquest del telèfon. En Copper Knickers o com fos
que es digués.

--Llavors a on, amic? --va preguntar l'irritat taxista.

--A Essex. Al costat de la A127.

El conductor va bufar.

--A aquestes hores de la nit? Visc a Bounds Green. T'he dit que me
n'anava a casa.

--Llavors no trigaràs tant en arribar a casa des d'allà, oi ``amic''?
--li va engegar la Donna--. Quan costarà? Perquè segur que el Doctor no
té gaire canvi a sobre. I com ja he pagat a l'avi aquesta nit, també
hauria de fer-ho amb ell.

--Cent.

--D'acord--va dir el Doctor.

--D'acord? --va grallar la Donna--. Com poden estar cent lliures
``d'acord''?

El Doctor la va ignorar i es va inclinar cap al conductor.

--Però necessito arribar-hi ràpidament.

--Cent vint, llavors.

--El que sigui--va sospirar el Doctor--. Donna, et veuré demà. Pots
agafar un altre taxi per arribar a casa des d'aquí, oi? Oh, i Donna?

--Sí?

--Portes cent vint lliures a sobre?

La Donna va mirar-lo com si l'anés a piconar, llavors va dir-li al
conductor:

--Essex--va engegar--. I passant per un caixer automàtic, si us plau.

El Doctor va aixecar una cella.

--Donna?

--Solet--va somriure la Donna--, si et dono cent tacos per anar cap a un
elegant radiotelescopi, jo vaig amb tu.

El Doctor va somriure.

--Esperava que diguessis això.

I el taxista va conduir cap a Kensington mentre el Doctor i la Donna
feien plans.

En Wilf i la Netty deien els seus comiats, disculpant-se per fer que la
nit acabés prematurament.

--Ho tornarem a fer ben d'hora--va dir-li l'Ariadne Holt--. I fer la
presentació de veritat.

--Ho sento molt per haver-nos d'anar així, de pressa--va mentir en
Wilf--, però ha passat quelcom. I haig de veure la casa de la senyora
Goodhart. Espero que em disculpin.

--No hi ha res que disculpar-- va dir en Crossland, palmejant-li
l'esquena--. Qualsevol excusa per a una festa de sopar, oi? La trobada
de la setmana que ve?

A la seva cara, en Wilf tenia un somriure fix, però els seus ulls
explicaven una història diferent.

--M'agradaria molt--va dir.

La Netty va inclinar-se i li va xiuxiuejar:

--Estaran bé. El Doctor sap el que es fa.

En Wilf li va fer un somriure, intentant semblar més segur del que en
realitat es sentia.

Es preguntava si hauria d'haver anat amb ells.

I es va preguntar encara més què anava a dir la Sylvia quan aparegués
després de mitjanit sense ells.

El Copernicus Array estava a la fosca quan el taxista va aparcar a
l'aparcament. La Donna va pagar al taxista malhumorat, que se'n va anar
cap a Londres el més ràpid possible.

--Puc només preguntar com ens ho farem per tornar a casa des d'aquí?
--va xiuxiuejar al Doctor la Donna quan es van enfonsar a la foscor--.
Es un camí molt llarg.

--Caminant?

La Donna va estrènyer la seva espatlla, i mentre ell la mirava, ella va
fer gestos cap avall, com dient que ella hauria de mirar cap a la seva
roba.

--Estàs molt guapa--va somriure ell, col·locant-se les seves ulleres de
set-ciències.

--Bé, gràcies, Casanova, però això no era el que volia dir. Estic
vestida fins a les celles amb un vestit de festa que no està dissenyat
per a un camí de quaranta milles a través d'Anglaterra--va fer un
sospir--. He hagut de prendre-ho prestat, ja saps. La Venna no em
perdonarà si se l'espatllo. I si no em perdona, qui sap el què farà quan
li digui per qui he fet aquesta passejada.

--No li ho diguis, llavors.

La Donna es va rendir i va intentar-ho d'una altra forma.

--D'acord, així que per què estem aquí i què fan els llums apagats?
Certament, si és un observatori i és de nit, aquest lloc hauria d'estar
en ple funcionament?

--Ben vist, ben vist. I és així pel que nosaltres estem aquí gatejant
per un aparcament i arbustos i no anant cap a la porta davantera.

--Però creia que el teu amic treballava aquí.

--Ho fa.

--Però per què no anem cap a la porta del davant i els diem ``Ei,
col·lega del Doctor, ens has trucat i hem arribat''?

El Doctor mirava a un cotxe aparcat a través d'una plaça de
l'aparcament.

--Què li passa al cotxe?

La Donna va mirar a través de la fosco.

--Que no saben aparcar bé?

--I?

--Que espatllaran l'herba.

--I?

--Que\ldots{} els llums se'ls fondran?

--Això sí.

--I bé?

El Doctor va somriure.

--Pensa-t'ho. Sembla molt antic el cotxe?

--Nou. Així que hauríem de picar-los per a que no es deixin els llums
encesos. ~Però ho han ignorat i si és tan nou, el que fos que fa que
funcionin els llums no hauria d'esgotar-se tant d'hora. El meu pare es
va deixar els llums tota la nit encesos fa un parell d'anys i seguien
enceses el següent dematí i el cotxe estava bé. I el cotxe era
antiquíssim.

El Doctor va caminar cap al cotxe, amb el tornavís sònic a la mà, i va
escanejar el vehicle.

--Un gran drenatge d'energia, localment--va murmurar.

La Donna va tocar el seu braç.

--Ei, has vist això?

El Doctor va seguir la seva vista. Arriba al cel hi havia una cara
somrient feta d'estrelles.

--La claror de la ciutat deu haver amagat això mentre conduíem cap
aquí--va dir ell amb tranquil·litat.

--Focs artificials?

--Esperem que la majoria de la gent ho pensi--va respondre.

--No són focs artificials, llavors.

--No són focs artificials--va confirmar--. Un Cos del Caos en
funcionament.

--Què és un Cos del Caos quan és a casa?

El Doctor es va encongir d'espatlles.

--No en tinc ni idea. El teu avi em va presenta el concepte anit. Però
una nova estrella que pot dibuixar altres estrelles a través del cel en
una alineació com aquella? Bastant caòtic, jo diria.

--Jo també--la Donna es va passejar cap a al porta davantera de
l'edifici del Copernicus Array--. Bé, no n'aprendrem gaire aquí fora,
oi? --va trobar uns timbres i va pitjar una que deia NOMÉS EMERGÈNCIES.

Res no va passar.

--Un drenatge d'energia local, suposo--va dir la Donna, somrient-li.

El Doctor va fer un sospir i la va atrapar, passant el tornavís sònic
per la porta i obrint-la amb una empenta.

Total foscor.

--Així que, qui és aquest professor col·lega teu, llavors?

El Doctor feia servir la llum blava del sònic com una llanterna,
intentant identificar els noms de les plaques.

--L'hauries d'haver conegut anys enrere. Un del grup de persones
d'aquest planeta amb qui puc confiar de tant en tant quan necessito
ajuda d'un especialista. I ells em poden trucar si em necessiten,
viceversa--va senyalar cap a dalt--. L'oficina del professor Melville és
un pis per sobre, la sala de control és allà fora, a través dels jardins
i girant cap a l'esquerra.

--L'avorrida oficina o l'emocionant telescopi?

El Doctor va somriure, semblant lleugerament esgarrifós amb la llum
blava del sònic.

--Intentes influenciar les meves eleccions, Donna Noble?

--Per telescopi que no, Doctor. Jo no telescopi podria imaginar intentar
telescopiar les teves eleccions d'on anar al telescopi o què fer al
telescopi.

El Doctor la va mirar fent una ganyota amb el seu últim comentari i ella
es va riure.

--Bé, d'acord--va dir ella--, però no en aquest sentit.

--Per alguna raó, crec que hauríem d'investigar el telescopi.

--És com si en Derren Brown estigués a cada habitació--va riure la
Donna.

Van seguir pel passadís de la porta davantera fins que van arribar a un
conjunt de finestres franceses que miraven cap al pati. El Doctor les va
apuntar amb el sònic i les van creuar cap al fred aire de la nit.

--Ens observen--va dir la Donna després d'haver estat caminant un praell
de minuts.

--Com ho saps?

--Se'm eriça el cabell--va respondre--. Això i el fet de que els puc
veure per davant de nosaltres.

El Doctor va mirar cap a la foscor.

Hi havia un grup d'homes i dones dempeus a l'entrada del Array mateix.

La Donna va mirar cap a la freda estructura de metall d'esglaons i
passarel·les que portaven cap a una cabina que estava dominada per un
radar amb forma de plat ovalat que formava l'Array.

--Estic impressionada--va dir.

--Per ells?

--No, per l'Array. Mai havia vist un radiotelescopi abans, només a la
tel·le. És bonic--de cop va parlar amb el grup--. Teniu un bon plat allà
a dalt, gràcies per deixar-nos entrar i veure'l. Hi ha una botiga de
records? M'agradaria comprar-li al meu avi una tassa amb una foto
d'això. O un imant de nevera.

--O--s'hi va unir el Doctor--, unes tovalles per al te. Feu tovalles per
al te? Tothom fa tovalles per al te avui dia. Mai estic segur perquè
hauria de comprar unes tovalles per al te enlloc, però avui ho faria.

No hi va haver resposta del grup.

--De fet estic buscant al professor Melville--va continuar el Doctor--.
Està ell allà? Està bé?

Un home petit amb una cicatriu a la seva galta va donar un pas endavant.

--És ell? --va xiuxiuejar la Donna.

El Doctor va fer que no amb el cap.

--No, ho sento.

--Excel·lent--va entonar l'home, amb un lleuger accent ~amb el seu
precís i diferent anglès--. Tu ets el Doctor.

--Què et fa dir-ho?

--Podem percebre que no ets\ldots{} sencerament humà.

--``Sencerament''? --el Doctor va semblar afligit--. No soc ni
remotament humà, gràcies. Unes espècies terribles. Sempre lluitat i
barallant.

--Ei! --va dir la Donna.

--Veus el que vull dir? Amb prou feines van evolucionar ells mateixos de
l'ús dels sorolls guturals i, ostres, ja tenen això que anomenen música
pop. Vull dir, sé que m'agrada però arriba un moment on no pots
diferenciar els nois de les noies o entendre les lletres, oi? Llavors,
hi ha per suposat el menjar. Heu menjat algun cop les seves
hamburgueses?

El vell home va aixecar una mà per aturar el discurs del Doctor, i ho va
fer però va somriure.

--Podria seguir tota la nit, així que suposo que us estic molestant,
així què em dieu si senzillament xerrem sobre per què us heu apoderat de
les pobres ments d'aquesta gent?

--És el llinatge--va dir l'home menut.

El Doctor va fer-li una mirada a la Donna, i podia veure que no era la
resposta que ell havia estat esperant.

--Com a la genealogia? --va preguntar.

--Quatre dels humans aquí presents poden rastrejar les seves famílies
cap a un específic lloc en el temps. Això em dóna poder per sobre ells.

--I la resta?

--Esclaus. Ases de càrrega. Servents desitjosos.

--Desitjosos? De veritat? És encantador. No és gaire cert, però bé, ho
és? Però si ho voleu creure, és just--el Doctor va començar a caminar
cap a ells, així que la Donna el va haver de seguir--. D'acord llavors,
d'on sou originalment? --va senyalar cap a la cara formada als cels--.
Alguna cosa a veure amb això, suposo.

El home menut amb la cicatriu es va encongir d'espatlles.

--Tot al seu temps, Doctor. La Madam Delphi necessita parlar amb tu.

Ell va aixecar els seus braços en direcció del Doctor i, abans que la
Donna pogués tossir, un gruix i espurnejant torrent de llum morada
vermellosa va sortir disparada dels seus dits, colpejant al Doctor
directament al pit i llençant el seu cos un parell de metres enrere.

Estava semiinconscient quan la Donna el va agafar.

--Ves-te'n\ldots{} adverteix la gent\ldots{} --va ser tot el que va
poder dir quan els seus ulls es van posar en blanc i el rítmic ascendir
i davallar del seu pit cremat es va afluixar fins a un moviment
inconscient.

La Donna sabia que no hi havia rest que pogués fer més que el que
l'havia demanat. Havia de trobar ajuda.

--I l'humana? --va preguntar una de la resta del grup.

--Mateu-la--va respondre l'home, i la Donna va veure que la resta
alçaven els braços en moviments similars.

--Avui no, gràcies--va cridar la Donna i va córrer cap a les finestres
franceses, corrent en ziga-zaga, conscient dels trets d'energia morada
xocant-se contra els arbres a ambdós costats seus.

Era gairebé a les finestres franceses quan van esclatar en metall i
vidres al seu voltant.

Llençant els seus braços cap a la seva cara la Donna va córrer
directament a través de les restes i cap a la mansió enfosquida. Es fa
obrir pas pel vestíbul, on una grandiosa escala de fusta anava cap a
pisos superiors.

--He vist les pel·lícules--va xiuxiuejar. Va anar cap a la part darrera
dels esglaons on, segurament, hi havia una porta que portava a un
celler.

La va obrir grinyolant, va comptar tres i la va obrir de cop, fent molt
soroll.

Llavors va anar de puntetes cap a una de les oficines i s'hi va ficar
dins.

Segons després, alertats pel soroll, un petit grup de gent de fora va
arribar a la porta del celler. Va veure com dos d'ells anaven escales
avall, deixant a un de guàrdia. Maleït sigui, havia esperat que anessin
tots junts i poder-los deixar tancats allà a baix.

Llavors el que era de guàrdia va semblar estar mirant al seu voltant i
per un segon, la Donna va creure que havia sabut on s'amagava. Llavors
una altra persona que no havia vist abans va xocar-se de cop contra
l'home, enviant-lo volant cap a una paret.

La nouvinguda va apropar una cadira contra el pom de la porta del
celler, atrapant els dos de dins com la Donna havia planejat i llavors
li va cridar:

--Vinga!

La Donna va córrer directament cap a la seva nova aliada i es va deixar
guiar a través del passadís.

--La cuina\ldots{} --va respirar la nouvinguda--. Per aquí\ldots{}

Hi va haver una petita explosió i la Donna va suposar que el duo
~s'havia obert camí fora del celler.

--Era bona idea mentre durava--va somriure--. Sóc la Donna Noble.

--Sóc la senyoreta Oladini--va dir la senyoreta Oladini a tota pressa--.
Encantada de conéixer-te. Hi ha cotxe?

--He vingut en taxi.

--No, vull dir el seu cotxe.

--Val la pena intentar-ho.

Van xocar contra la porta de la cuina i van sortir cap a l'aparcament,
directament cap al cotxe abandonat. La Donna va escollir el seient del
conductor.

--I les claus? --va preguntar la senyoreta Oladini.

--No hi ha claus--va dir la Donna--, perquè la vida mai no és tan
convenient.

La senyoreta Oladini es va ficar sota el taulell de control i va ajuntar
un parell de cables.

--He passat una dolenta joventut--va dir.

El motor es va encendre peresosament, i llavors es va apagar.

--Un cop més--va dir la Donna a tota pressa--. Però si no funciona,
rendeix-te, perquè sabran que som aquí ara per ara!

La senyoreta Oladini va intentar fer un pont de nou, sense sort.

Dos dels seus perseguidors van apareixer a la porta de la cuina, amb els
braços aixecats.

--Corre! --va cridar la Donna i es va llençar fora i allunyant-se del
cotxe mentre l'energia morada xocava contra ell i el cotxe explotava.

La Donna estava tombada contra uns arbustos, amb el vestit de la Veena
fet esquinçalls.

--Oh, estic molt morta--va xiuxiuejar.

No hi va haver cap senyal de la senyoreta Oladini, però era difícil
veure res amb el cotxe cremant bloquejant la seva vista. La Donna va
mirar al seu voltant i va veure una bicicleta recolzada contra una paret
llunyana.

--Has d'estar bromejant--va xiuxiuejar, llavors va mirar la seva roba--.
Qui no arrisca, no pisca.

Esperant que les mateixes flames que l'amagaven dels seus atacants, la
mantindrien allunyada del seu punt de vista, es va estirar fins a
l'altre costat de l'aparcament.

Amb una última mirada al voltant buscant a la senyoreta Oladini, i una
trista realització de que el cotxe havia estat segurament la seva pira
funerària, la Donna va agafar la bicicleta i s'hi va pujar. Va pedalejar
lleugerament, lentament acostumant-se a muntar en bicicleta de nou, i
llavors va sortir disparada cap a la vorera i llavors cap a la carretera
principal.

No hi havia cap forma en la que tornaria a Chiswick amb bicicleta, doncs
no tenia tres dies lliures per fer-ho, però allò la portaria cap a la
ciutat més propera.

Mentre la Donna pedalejava furiosament, el car cremant va encendre la
façana del Copernicus Array a la seva esquena i les flames reflectien
les grans finestres de la mansió. Però de gent no en podia veure cap
senyal.