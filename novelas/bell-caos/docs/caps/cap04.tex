\chapter*{DIUMENGE}
\addcontentsline{toc}{chapter}{DIUMENGE}

Quan era petita, la Donna havia escoltat la frase de ``el tret que va
ser escoltat per tot el món'' que s'acostumava a dir per descriure
l'efecte que l'assassinat del president estatunidenc John Kennedy havia
tingut sobre la civilització occidental sencera. La gent sempre deia que
podien recordar on eren quan va succeir.

Com a filla dels anys 70, va créixer escoltant sobre les coses com
l'aterrament a la Lluna, els assassinats tant dels dos Kennedy i Martin
Luther King i el funeral d'estat de Winston Churchill, però mai no els
va entendre del tot. En una infantesa de pilotes de salts, Donny Osmond,
bicicletes Chopper i segells Green Shield, paraules com Blitz,
racionament i el puré de patates barrejat amb col només significava que
la gent gran estaven recordant vint anys enrere i probablement planyent
que la joventut d'avui dia mai no sabria com ho tenien de bé.

La primera vegada que la Donna s'havia trobat a si mateixa dient allò, a
un dels fills dels veïns que havia fet una rascada a l'últim cotxe del
seu pare, es va quedar bocabadada. Havia acabat convertint-se exactament
del que es mofava dels seus pares i els seus avis quan tenia la seva
edat. Avui dia no hi havia res que l'agradés més que escoltar l'Avi Wilf
parlar de la guerra, la seva vida al regiment de paracaigudistes o els
dies de la iaia Eileen com a membre de l'exèrcit terrestre femení.

Aquell dia era com el 22 de novembre de 1963, un dia quan un altre tret
s'escoltaria al voltant del món.

El diumenge havia, honestament, començat bastant malament. La Donna
s'havia llevat al seu llit (allò era quelcom bo), encara que només havia
tingut dues hores de son (allò era quelcom dolent).

El vestit esquinçat de la Veena estava fet una pilota al terra (allò era
quelcom dolent); al seu costat, un tiquet que li havia donat el taxista
(dolent, doncs a veure qui li reclamaria tot allò?) que l'havia portat
des d'un lloc anomenat South Woodham Ferrers cap a Chiswick. El tiquet
era de 225\pounds (molt, molt dolent, segurament el seu compte hauria d'estar
buit en aquell moment). En Wilf estava despert quan va tornar (molt bo
allo) i havia escoltat tot el que havia succeït al Copernicus Array. Ell
l'havia agafat de la mà, prometent que trobarien una forma de rescatar
el Doctor i la va enviar a dormir per fi.

La Donna havia estat furiosa amb si mateixa, havia abandonat al Doctor
allà, l'havia abandonat d'una forma que mai no havia fet abans (molt
dolent). Però també havia estat pràctica. Ell li havia dit que marxés, i
allò havia estat bé perquè sinó l'haurien matat (definitivament una cosa
dolenta).

La Donna se sentia com morta però necessitava aconseguir ajuda i tornar
a Array.

Potser podria trobar a la Martha Jones i els seus amics al UNIT,
l'agradava la Martha i sabia que faria tot el possible per ajudar.

Mentre arribava al final de les escales, va agafar les Pàgines Grogues i
ja estava a la meitat de la U quan es va adonar de que potser UNIT no hi
seria allà, amb la resta d'Organitzacions Militars Dedicades a Fotre
Fora els Marcians (que no eren ni bons ni dolents, només una mica
lleugers de trets).

--Bé, bon dia, senyoreta--va engegar la Sylvia Noble al bell mig del
vestíbul--. Estava preocupada per tu ahir a la nit.

--Per què?

--Oh, no ho sé. Tornant de puntetes a poc del trenc d'alba després d'una
nits de clubs amb el Doctor? A la teva edat? Vull dir, anar de clubs és
genial quan tens 21, però quan la mitjana edat és a uns pocs aniversaris
lluny\ldots{}

--Ei!

--És igual. La cosa és, que és hora de créixer, joveneta.

--Gràcies, mare. Massa gran per anar-se'n de clubs, però no massa gran
per viure amb la meva mare. Genial.

--Ningú et força a viure aquí--va dir la Sylvia, posant una tassa de te
a les mans de la Donna--. No és que ho facis gaire aquests dies. Te'n
vas amb el Paio Camesllargues setmana sí i setmana també--. Estirant de
la camisola de dormir de la Donna, estrenyent el cinturó, afluixant el
coll--. I on és ell avui, llavors? Segur que s'ha emportat el teu avi a
l'hortet, sense dubtar-ho. I no és un matí massa càlid, i s'ha deixat el
termos aquí--. La Sylvia es va llepar el seu dit índex i va esborrar una
marca de la galta esquerra de la Donna--. Encara que, imagino que ambdós
tornaran per dinar. La carn rostida de cada diumenge i tota la pesca?
Ja! Què més voldrien. Et diré què: un viatge al Jolly Lock Keeper i al
tot-el-que-puguis-menjar per deu lliures és el que farem avui, noieta.
Aquells dies de tenir-me esclavitzada amb la vedella i el pudding de
Yorkshire se'n van anar amb el teu pare, si em permets dir-t'ho-- la
Sylvia va deixar anar el cabell vermell de la Donna de darrere de les
seves orelles i li va arreglar el serrell--. I hi havia una nota per a
tu ahir a la nit, la vaig trobar quan el teu avi em va despertar,
trontollant quan el vau deixar. No la vaig llegir, però és d'aquests
nois Carnes dels que parlava el Doctor.

La Donna li volia preguntar com sabia de qui eren si no l'havia llegida,
però allò els portava a xerrades indesitjades sobre les notes dels
directors quan la Donna en tenia dotze fins a les cartes del Martyn Hart
quan tenia quinze, les quals havien estat obertes quan estaven dirigides
a una altra, així que es va callar.

La Donna també tenia una punxada per la vedella rostida i el pudding
Yorkshire, evocats per la seva mare, coberts amb un fantàstic suc de
carn. Els talls del pare. L'avi i la iaia visitant-los els diumenges,
parlant d'horaris de trens i de clubs de cotxes i llargues passejades
pel parc Windsor Great. De cop volia tornar a tenir deu.

I va voler plorar.

--Mare?

--Què?

--Trobo a faltar al pare.

I la Sylvia Noble va abraçar la seva filla d'una manera que no havia fet
en molt de temps. Llavors la va apartar, com si acabés de recordar que
l'estat normal de la Sylvia Noble era malhumorada i inflexible i no
tocable i càlida. Especialment amb la seva filla capritxosa.

Però va ser prou amb un moment per er a la Donna feliç. Perquè havia
estat un vertader gest instintiu.

--Pots trucar al teu avi, si us plau, i esbrinar quan tornarà i si bé el
teu Doctor ens trobarà al pub?

--I la Netty?

(Oh, quelcom molt dolent).

--Si devem\ldots{}

Desitjosa de canviar l'ànim, la Donna va obrir la porta davantera per
comprovar el temps.

--Mare, per què dius que l'avi s'ha anat a l'hortet?

--Què sinó està fent? És o amb els vegetals o amb la Netty. I quan no és
un, és l'altra.

La Donna li va llençar una mirada i la Sylvia va tenir la decència de
disculpar-se.

--Ho sento. M'acostumo a estar tant jo sola, que m'oblido de que no
hauria de dir les coses que penso per a mi en alt per a la resta de la
gent.

--Ja ens preocuparem de tu i de la Netty més endavant. L'avi s'ha
emportat el cotxe, el qual no necessita per anar a l'hortet.

La Sylvia va arrufar les celles.

--No em va dir que se l'emportava. Va dir que anava a buscar al Doctor.

--Així que has suposat que era a l'hortet?

--Com l'altra nit, sí.

La Donna va trucar al seu avi amb el seu mòbil, però estava apagat. El
molt tonto anava cap al Copernicus, oi? I havia deixat a la Donna
enrere.

I en aquella situació, el moment ``el tret que es va escoltar per tot el
món'' va tenir lloc.

La Donna sempre recordaria que portava la seva camisola, dempeus amb la
seva porta lleugerament oberta, una nota dels nois Carnes a la seva mà
(sense llegir, almenys per ella) mirant l'espai on un cotxe hauria
d'haver estat, amb la seva mare just al darrere. Hi havia una tassa de
te recent vora les Pàgines Grogues. Vora la carretera, el vell senyor
Lyttle passejava el seu gos. Una petita cosa negra d'una raça
indeterminada que sempre feia olor a pell mullada. A l'esquerra, amb una
d'aquestes coses de visió perifèrica de veritat, hi havia una caravana
blava aparcada.

I dominant el cel blau hi havia una gegantesca i feroçment terrible
pilar de pura llum morada, amb els extrems coberts amb un lleuger
espurneig morat.

La Donna va escoltar a la seva mare dir:

--Oh, déu meu, no un altre cop, no el cel en flames de nou.

Però no era sencerament en flames. Només aquella columna única,
acompanyada pel so com d'una flama de gas amb el fogó, però reforçada
per deu mil decibelis.

La Donna només sabia que alguna cosa terrible estava tenint lloc on la
columna de llum tocava el terra, en algun lloc a l'oest de Chiswick.
Encara que no ho sabia llavors, per tot el món, no importava la franja
horària, columnes semblants d'energia calorífica estaven repetint-se.

I al cel, aquella maliciosa cara d'estrelles encara seguia allà, aquell
lleig somriure semblava més ample i més balder que abans.

--Mare, a dins, ara. Tanca les portes, no deixis entrar a ningú excepte
a mi, o a l'avi, o al Doctor. Especialment al Doctor.

--I per què és tan especial?

--Oh, només m'has de dir que ho faràs, oi que ho faràs?

--D'acord--va murmurar la Sylvia. Llavors: --. I on aniràs?

--Haig d'intentar trobar-lo. A ambdós, de fet.

--Que no hi estan junts?

--És això el que em preocupa.

--Bé, no pots anar fora així.

La Donna es va adonar que encara no estava vestida i va córrer escales
amunt, llençant la seva camisola al terra quan no era encara a la porta
de la seva habitació.

--No deixis allò tirat al terra--va cridar la Sylvia.

La Donna el recollí, el penjà del ganxo de la porta i va fer un sospir.

--Tinc prioritats, mare--va murmurar.

Vestida amb una samarreta i un vell xandall que no podia creure que
hagués portat algun cop però que sabia que escalfaria, la Donna va
tornar a baixar les escales, agafant un abric pesat.

Fora, als carrers, podia escoltar la gent i els cotxes accelerant.
Tothom havia vist el pilar de llum i ara estaven veient aquella terrible
cara al cel.

Dóna-li mitja hora i entraran en pànic als carrers, fent disturbis,
agafant botins i amb la policia per tot arreu. Havia de sortir de
Londres l'abans possible.

--No tinc cotxe--es va maleir a si mateixa.

Va obrir la porta davantera i va veure aquella furgoneta blava.

Estava molt malament. Molt malament, Donna Noble. Molt malament.

Llavors estava a la finestra del conductor, mirant al seient. El taulell
de control, la falta de claus i la porta tancada.

La pobra senyoreta Oladini havia fet que tot semblés massa fàcil quan li
havia fet un pont al cotxe, però la Donna no tenia ni idea de com
començar. Genial. Va intentar la porta, només per si de cas. Estava
oberta.

Va mirar cap al carrer però ningú a la multitud de gent estava
cridant-la o reclamant la furgoneta com a seva. Es va ficar dins del
seient del conductor i a posar la mà a sota per a ajustar-lo. No aniria
enlloc, perquè ningú aquell dia i en aquell temps era prou estúpid com
per posar les claus sota el seient d'una furgoneta sense tancar. Però
Déu sap per què, quan va treure les mans de sota, portava les claus de
la furgoneta amb elles.

--I vull un tricicle i un pony, i un subministrament per a tota la vida
de xocolata amb llet--va dir en veu alta, posant la seva mà sota el
seient només per si de cas els seus desitjos de Nadal de quan en tenia
vuit també es tornaven realitat.

Però ni ponis, ni bicicletes, ni tan sols una fosa barreta de xocolata.
Però sí les claus, allò era bo.

Va arrancar la furgoneta cap enrere, i segons més tard estava de camí
cap a la carretera principal de Chiswick, planejant el seu segon viatge
cap a Essex en dotze hores.

Va fer un últim cop d'ull al retrovisor de casa seva mentre redreçava la
furgoneta i llavors va accelerar, desitjant que la seva mare no l'hagués
vista fent allò. Perquè llavors haurien de pagar-ne tres ronyons i mig.
I més encara!

Abans que hagués arribat tan sols a la carretera principal, les
multituds eren als carrers, mirant i assenyalant, i ella podia escoltar
les sirenes de les ambulàncies, policies i bombers per tots els costats,
totes anant cap a l'oest, cap a la M4. Cap on el pilar de llum havia
xocat amb el terra. Ella anava cap a Londres i aquell costat del carrer
estava relativament buit, sobretot per a un diumenge.

L'atenció de la Donna es va anar per un grup de gent fora de diferents
botigues d'electrodomèstics que hi havia a ambdós costats del carrer. La
carretera principal de Chiswick tenia un gran nombre de cafeteries i
botigues de roba quan era petita, però aquella invasió de botigues
d'aparells era estrany. Recordava el Doctor dient que havia conegut als
nois Carnes en una.

Tot allò va passar per la seva ment en un breu segons, probablement
perquè allà als carrers davant d'ella estaven en Lukas i en Joe Carnes.

Com si l'haguessin estat esperant.

Literalment.

Dempeus al carrer. Un minut abans, la carretera havia estat buida. Al
següent, dos nois estaven just davant seu.

La Donna va pitjar els frens, i just va evitar derrapar en sec, de fet
realitzant una parada bastant gràcil, encara que un home darrere d'ella
va fer sonar el clàxon.

--Sí? ~Què coi vas rebre per Nadal, rei? --va cridar-li de nou--. Per
què no t'ho fiques pel\ldots{}?!

La porta del passatger es va obrir, i en Joe i en Lukas s'hi van ficar.

--Joe diu que ha d'estar a un lloc que es diu Copernicus--va dir en
Lukas, amb calma--. També sabia que series aquí. En aquest moment.

--Per suposat que sí--va respondre la Donna, conduint cap endavant
mentre el conductor enfurismat la passava per davant, amb una mà sense
agafar el volant i fent-li gestos. Encongint-se d'espatlles, la Donna va
continuar conduint cap Hammersmith.

--Bon dia, Joe--va dir-li al noi, que ara estava al seient del darrere.

Joe no va respondre però va treure una cosa de la seva butxaca.

--Què és això llavors? Alguna cosa semblant a un nou MP3?

--És un M-TEK--va respondre en Lukas, avançant-se a en Joe.

--El què? --la Donna va intentar sonar interessada, però no ho estava.
Estava més centrada en com havien sabut que hi seria allà.

--Li va dir on hauries de ser--va seguir el Lukas--. Li parla.

Allò responia d'alguna manera la seva pregunta, va decidir la Donna,
però inquietantment li plantejava un parell de dotzenes més.

--Era així com va saber el nom del Doctor l'altre dia, llavors?

En Lukas es va encongir d'espatlles.

--No ho sé. L'home de la botiga se'l va donar. Va dir que era una versió
de prova. Va donar uns deu parells. Va dir que en Joe era la persona
perfecta per tenir-ne un. No m'ho va dir fins que no vam arribar a casa
i me'l vaig trobar descarregant-se música.

Allò tenia sentit per a la Donna, encara que no tenia cap sentit del
tot. Quan viatjaves amb el Doctor, començaves a acceptar que les coses
que no tenien gaire sentit el tenien en una forma de
no-tenir-ne-per-a-la-resta-de-la-gent.

Així que aquell aparell del M-TEK feia que en Joe Carnes sabés coses. O
li deia coses. Coses que atreien l'atenció del Doctor.

--No us va dir el vostre pare que no s'accepten regals de la gent
estranya?

--El meu pare ho va fer--va dir el Lukas, mirant al Joe--. Però el pare
del Joe no s'hi va quedar gaire.

Bé, va pensar la Donna, això mata una conversació. Llavors va fer un gir
de sobte cap a la rotonda de Hammersmith que va provocar que algú fes
sonar el clàxon. Potser era el mateix conductor que abans, però ni ho
sabia, ni l'importava. Va girar cap a la carretera de Talgarth.

Estava buida. Bastant buida. Allò era una carretera de sis carrils que
anava cap al centre de Londres, per Earls Court i llavors per
Knightsbridge, llavors passava pel parc Hyde i eventualment cap a
Piccadilly. Els hauria haver portat vint minuts, potser trenta d'arribar
a Piccadilly un diumenge a l'hora de dinar, i allò era sense cap tipus
d'obra a la carretera. La Donna ho va fer en deu i tampoc no es que anés
gaire ràpida.

Era com si tota la gent de Londres fugís, no, anant cap a algun lloc.
Aquella llum. Tots anaven cap allò.

Tafaners, desitjos de prendre fotografies amb els seus mòbils i dir
``Ei, mireu, vam veure la matança!'' o alguna cosa encara més sinistra?
I llavors per què no estava ella afectada?

--Perdoneu-me, nois, per trencar més lleis\ldots{} --la Donna va treure
el seu telèfon mòbil mentre conduïa i va trucar a la seva mare. No va
tenir resposta. Allò no eren bones notícies.

I allà era ella, a una furgoneta robada, conduint a través d'un Londres
desert, cap a l'Essex més fosc per rescatar el seu avi i al seu amic
d'uns assassins, incapaç de contactar amb casa seva, completa amb els
Nois dels Condemnats al seu costat.

--Genial, Doctor--va dir, sense dir-ho a ningú en particular.

A unes vint milles davant de la Donna i els nois, hi havia una gran
presència policial i d'ambulàncies al voltant de l'àrea de Ruislip
Woods, amb encara més serveis d'emergències arribant del proper RAF
Hillingdon.

El gegantesc raig d'energia blanca que havia colpejat l'arbreda, un dels
boscos més protegits de Gran Bretanya, encara que no hi havia gaire que
protegir llavors. Havia creat un gran cràter amb forma de bol gegantí
d'un quart de milla d'amplada, delmant els arbres, les herbes i els
arbustos. Una petita brisa de fum vagava per l'aire del matí i les
multituds de tafaners sorpresos s'ajuntaven molt propres, en part
bocabadats, en part en shock, però en gran part per por.

Havia estat un accident d'avió? Una bomba d'al-Qaeda? Alguna cosa de la
base de la Reial Força Aèria havia anat malament? Hi havia víctimes? Oh
déu meu, els meus nens jugaven aquí? Algú ha vist el meu gos, una
barreja de Labrador? Perdoneu, heu vist al meu marit, estava fent un
tomb? Algú més ha vist aquesta terrible cara al cel? És alguna cosa
d'alguna pel·lícula? Mai m'he confiat de que l'IRA abandonés les
armes\ldots{}

La sergent de policia Alison Pearce estava intentant controlar les
multituds i els seus propis oficials i deixa passar la gent
d'emergències. La guàrdia del diumenge al matí havia semblat gaire bona
idea. Tres nens significaven que fer les guàrdies nocturnes no eren
possibles, però la seva mare els podria cuidar el diumenge mentre ella
era de guàrdia. Normalment, arribaria a casa a les deu, veure'ls
adormits i preparar-los per l'escola al matí. Ja havia trucat a casa i
avisat a la seva estimada mare que l'atenció de l'àvia haurà de durar un
parell de dies més. La paperassa només hauria de mantenir-la molt
ocupada. I allò assumint que s'anés en algun moment d'aquell lloc.

--Ei, tu, perdona? --va cridar a un paio jove que intentava passar sota
la cinta vermella i blanca--. Senyor? No es pot creuar\ldots{}

L'home el va ignorar. La sergent Pearce va agafar la seva ràdio i va
cridar a un parell de col·legues mentre ella feia una passa sota la
cinta ella mateixa i corria rere seu.

--Benvinguda de nou--va dir a\ldots{} bé, al no-res. Només a les fines
cendres blanques que havien estat temps ha arbres i el cel sabia què.

--Senyor, haig de demanar-li que torni rere la cinta. Això és l'escena
d'un crim.

L'home va seguir ignorant-la, i la sergent Parce va adonar-se de que
cinc persones més havien fet el mateix al voltant del perímetre.

--Nois--va dir a la ràdio--, què està passant?

Un dels seus agents va respondre-li.

--No els hem pogut aturar, sergent, s'han desempallegat de nosaltres.

La sergent Pearce va fer un sospir i va arribar on era l'home, però ara
estava de genolls, estirat cap al terra cendrós.

--Benvinguda de nou--va dir de nou.

I tenia els dits connectats amb el terra mentre la Sergent Pearce li
allargava la mà per tocar-lo pel braç.

Ella va sentir una mena de xoc, petit i elèctric, però poderós, i es va
trobar a si mateixa un parell de metres lluny, tombada panxa amunt,
movent el cap per aclarir-lo.

L'home era ara dempeus, d'esquena al cràter de cendra. La sergent es va
adonar que la resta de gent, ara eren set en total, havien fet el
mateix. Era com si estiguessin vigilant el lloc.

El jove agent que li havia parlat per la ràdio estava al seu costat.

--Està bé, sergent? --va dir, ajudant-la a aixecar-se.

El va apartar d'una empenta.

--Estic bé, Steve. Què dimonis és això? --l'agent de policia Steve
Douglas es va encongir d'espatlles. La sergent Pearce va intentar parlar
per la seva ràdio però tot el que escoltava era un soroll estàtic.

L'agent de policia Douglas va intentar el seu. El mateix.

--D'acord, això està mort i és estrany--va dir.

La sergent Pearce es va allunyar i va tornar a creuar la cinta, dient-li
a en Douglas que es quedés allà i mantingués un ull en ells:

--Però no t'hi apropis.

Ella es va donar pressa en anar cap a un grup de bombers i policies, que
ara incloïen el seu superintendent.

--Senyor, tenim un problema--va informar, i va explicar que set persones
estaven vigilant el cràter.

El superintendent Shakiri va arrufar les celles i va començar a moure's
cap endavant, cap al perímetre.

--Faci que el públic s'allunyi més, sergent. Mogui la cinta uns altres
sis metres.

Ella va fer que si amb el cap, però la seva ràdio seguia sense
funcionar. En Shakiri va intentar la seva. Res.

--Funcionava fa deu minuts--va murmurar.

--I la meva igual--va dir la sergent--. Ha de ser alguna cosa elèctrica.

--Per què diu això?

Ella li va dir que havia tocar l'home i el xoc que havia rebut.

--Vagi a que la vegi un dels metges, sergent.

--Estic bé, senyor\ldots{} --va començar, però ell la va fer anar--.
Reacció desaprovada, sergent. Vostè li diria a qualsevol de fer el
mateix. I si diuen que es troba bé, la veuré en cinc minuts--li va fer
un somriure--. Si us plau?

La sergent Pearce es va encongir d'espatlles i va caminar cap a una de
les ambulàncies, mentre escoltava com en Shakiri cridava ordres per a
que la cinta fos endarrerida manualment.

Mentre arribava al metge que l'esperava, alguna cosa\ldots{} alguna cosa
instintiva va fer que mirés enrere. Era com un d'aquells moments a
càmera lenta a una pel·lícula, tantes coses van passar en un moment, que
no podia dir si ho havia vist tot o només havia tingut lloc al seu
cervell que ho havia posat tot junt després.

Un flaix de llum morada, com un raig d'un tret elèctric a través de la
multitud de tafaners, abatent-los a tots i cadascun d'ells.

L'agent de policia Steve Douglas es va esvair, tot i així, per un petit
segon, la sergent estava convençuda de que el va veure llençar els seus
braços a dalt per protegir-se del flaix morat i ella el va poder veure'l
(no a ell, sinó al seu esquelet), només per un segon, quan ell es va
anar.

Els set ``guàrdies'', no més amagats a la multitud, havien estirat els
braços apuntant els uns cap els altres, i l'electricitat morada els
connectava a tots, com una corda.

El superintendent Shakiri es va llençar a si mateix al terra,
emportant-se a un parell d'altres oficials amb ell amb una tàctica de
rugby, probablement salvant les seves vides.

Hi va haver un flaix al cel, com un raig solar, només per un segon, i la
sergent va jurar que el cel sencer brillava de color morat.

I llavors va acabar. Més o menys.

La gent s'estava aixecant i fugia. Ningú volia estar a prop del que fos
elèctric. Allò era bo en el sentit de que el públic se n'anava, però
estava desorganitzat, i allò era perillós. Si una sola persona
queia\ldots{} Recordava la història del desastre a un túnel del metro de
l'East Londres durant la guerra quan es feia servir com un refugi per a
amagar-se dels bombardejos aeris. Mentre la gent espantada corria
escales avall, una dona va caure, portant a la multitud sencera al
terra, matant gairebé dues-centes persones de cop.

El pànic fugint en aquell moment, no estava tan atapeït, però podia ser
igual de letal. Va veure en Shakiri, arrossegant-se a si mateix, cridant
als oficials al seu voltant intentar ajudar al públic. Va llençar una
mirada cap on en Steve Douglas hauria de ser, era obvi que també ho
havia testimoniat tot, i llavors cap a ella.

Apartant al metge, va córrer per a unir-se'l a l'escena.

--Què dimonis era allò? --va respirar.

Ell va assenyalar cap als set ``guàrdies'' al voltant del cràter.

--Imagino que volen que ens allunyem--va mirar cap a les multituds
fugisseres.

--Alguna baixa?

La sergent només va mirar cap on el seu jove agent de policia havia
estat.

--Ho sabrem algun cop? --va dir--. No queda res d'en Steve Douglas.

En Shakiri va veure la seva mirada.

--I és per això pel que necessitem saber si hi ha hagut altres. Si els
fem responsables d'una mort, necessitem fer-los responsables de
qualssevol altra.

Ambdues ràdios van fer un soroll i es van engegar.

--Bon dia a tothom, enlloc al voltant del món--era una veu femenina,
parlant un clar i precís anglès--. Em dic Madam Delphi i sóc l'única veu
que mai necessitareu escoltar. Us estic parlant a tots per cada senyal,
per cada ràdio, televisió, ordinador i PDA al voltant del món. Ara heu
vist el que puc fer i seguiré fent-ho. Aquest planeta és meu. Podeu
tornat tots a les vostres tristes i petites vides i espereu a que jo us
digui què fer a continuació. Ara us tornaré les vostres programacions
establertes. Oh, ho sento, excepte per aquells països on estan emetent
qualsevol versió de Germà Gran. Ho sento, tots els concursants i
presentadors d'aquest programa, on sigui que estiguin, estan morts. Ja
m'ho agraireu després.

Els dos oficials de policia van mirar cap a les set persones que
vigilaven el cràter, amb aquella electricitat morada que els seguia
unint.

--Li diré una cosa, senyor--va dir la sergent Alison Pearce, mentre
mirava cap amunt on tot havia començat.

--Què passa, sergent?

--Aquella cara esgarrifosa al cel ha desaparegut.

La senyoreta Oladini estava pensant seriosament en presentar la seva
dimissió. Aquell no era un treball prou bo com per que meresqués totes
aquelles molèsties.

La última nit havia estat perseguida, havia estat electrocutada, gairebé
volada del mapa en un cotxe i, el pitjor de tot, algú li havia robat la
bicicleta. Esperava que hagués estat aquella dona pèl-roja que havia
estat al cotxe amb ella, perquè això significaria que havia escapat de
l'explosió.

La senyoreta Oladini no estava del tot segura de com s'ho havia fet ella
mateixa, però sabia que havia involucrat molt de fer la croqueta pel
terra, sense fer cas a la calor i córrer cap a un arbust i aguantant-se
la respiració pel que va semblar gairebé una hora però que podrien haver
estat només un o dos minuts fins que els seus perseguidors van assumir
que les dues dones estaven mortes.

No tenia i dea del què passava al Copernicus, però el seu cos havia
estat en xoc i havia caigut inconscient al terra de la casa de la vella
mansió, eventualment llevant-se de nou, freda ihumida amb molta gana. I
sense una bicicleta.

Va esperar fins que va veure si algú la veia, llavors es va fer camí de
nou dintre de la casa cercant calor. Després d'un parell de minuts va
trobar un parell de abrics abandonats. Sabia que estava en xoc. El seu
cos necessitava protecció i escalfor.

Es va posar els abrics, un a sobre de l'altra, llavors va anar cap al
petit armari. Es podria amagar allà, i les seves condicions atapeïdes
l'ajudarien a retenir el calor que necessitava. Va trobar una ampolla
d'aigua beguda a la meitat a una taula i també la va agafar.

Després d'un parell d'hores, es va sentir prou forta com per a
aventurar-se fora de l'armari i veure si la gent seguia allà, al menys
per veure si el professor Melville seguia amb ells.

Es va estar arrossegant en silenci per un passadís quan va fer un salt
perquè una pila de ràdios i televisions i un parell de portàtils van
cobrar vida, i va escoltar l'aportació de la Madam Delphi d'amenaces,
sentint-se freda de nou.

I en quant es va recuperar prou de l'ensurt de l'última nit, es va fer
de dia i era hora d'allunyar-se del Copernicus. Oblidar-se'n del
professor Melville i d'aquella gent, allò era massa amb el que haver de
tractar, i sospitava que l'emissió de la ràdio hi tenia a veure. La
policia, potser l'exèrcit, necessitaven saber que alguna cosa era allà.
Va començar a arrossegar-se lentament cap a la gran escala quan una mà
va arribar sortida del no-res i li va tapar la boca, silenciant cap
soroll que pogués fer.

La senyoreta Oladini ho va pensar, allà acabava tot.

--No faci soroll, si us plau, reina--va dir una veu a la seva oïda--. Em
dic Wilfred Mot, i no vull fer-te cap mal.

Va apartar la seva mà de la cara, i la senyoreta Oladini es va apartar.
Va mirar a l'home, vell, però no dèbil, òbviament. Amb els seus ulls
cremant d'intel·ligència, però no hi havia res d'amenaçador.

--Per què ets aquí? --va preguntar amb valor.

--La meva neta va ser aquí ahir a la nit. Em va dir el què passava.
Estic buscant al Doctor.

--Neta? Pèl-roja?

--Aquesta és la Donna. Tu has de ser la senyoreta Oladini? Ella pensava
que potser estaries morta, així que s'alegrarà de que estiguis bé.

La senyoreta Oladini no estava segura de tot allò. Aquella gent que
l'havia atacat podia saber tot allò. Però en sabrien sobre\ldots{}?

--Com se'n va anar la Donna?

--Amb una bicicleta. Era teva? La va deixar vora una comissaria a algun
lloc. Em sembla que era a South Woodham Ferrers--li va somriure--. Va
arribar molt tard ahir a la nit i jo l'esperava. Em va dir tot el que
havia passat i després la vaig manar a dormir. He decidit comprovar el
lloc per mi mateix.

La senyoreta Oladini va arrufar les celles.

--No la creia vostè?

--Per suposat que sí! La Donna no s'inventa les coses. Però volia trobar
al Doctor i mantenir a la Donna segura al mateix temps. Ja ha passat per
prou coses. Així que la he deixat dormint i m'hi he escapat de la casa
aquest matí--va mirar el seu rellotge--. Ara mateix deu haver esbrinat
això i estarà traient fum pels queixals, suposo.

La senyoreta Oladini encara no estava convençuda però no semblava tenir
el caràcter de zombi que tenia la resta de gent allà.

--Està buscant un doctor? Qui de tots? Són tots doctors i professors
aquí.

--Ell va venir amb la Donna. És un paio alt, amb el cabell ridícul.
Parla de moltes coses sense sentit.

--Francament, això defineix la majoria de les persones que treballen al
Copernicus, senyor Mott.

--Digue'm Wilf. I no, ell no treballa aquí. Li van demanar que vingués
aquí un tal professor Melville. És això pel què ell i la Donna van venir
tan tard.

--He passat gran part de l'última nit amagant-me i sent perseguida, no
vaig veure a la Donna fins que vam escapar. No tinc ni idea de si hi
havia alguna altra persona amb ell, ho sento.

Va semblar que en Wilf es desinflava.

--Oh, estava tant segur de que seria aquí. Crec que és l'únic que ens
pot salvar d'aquesta Madam Delphi que hem escoltat fa poc.

--Per què pensa això?

--És el tipus de coses que fa. Salvar-nos.

--És algun tipus de capellà?

En Wilf va riure.

--No, no del tot. Així que, on és tothom?

La senyoreta Oladini es va encongir d'espatlles i li va explicar que
pensava en anar-se'n.

--Dóna'm quinze minuts--va dir en Wilf--. Si no trobem al meu amic, et
portaré en cotxe a casa, d'acord?

La senyoreta Oladini va valorar les seves opcions i llavors va
accedir-hi. Malgrat tot, no hi havia una altra forma de tornar a casa. I
en Wilf Mott no semblava gaire amenaçador.

El va portar escales avall, a través de les finestres franceses fetes
miques i fora cap al jardí del darrere, senyalant cap al radiotelescopi,
explicant que allò era tot el que importava allà.

En Wilf va fer que sí amb el cap.

--Aquella cara al cel, estava feta d'estrelles, veritat? Suposo que
aquest observatori és on el Doctor hauria de ser-hi.

La senyoreta Oladini va tremolar i es va arrapar més fort als abrics.

--No estic segura--va dir amb tranquil·litat--. No vull tornar-hi.

--Per què no?

Però la senyoreta Oladini no ho podia explicar. Hi havia alguna cosa
sobre allò, alguna cosa sobre com el telescopi li havia semblat sempre
un lloc segur per a treballa però ara\ldots{}

En Wilf li va agafar de l'espatlla.

--D'acord, espera't aquí i jo m'hi passaré a veure si el Doctor és allà.
No trigaré.

La senyoreta Oladini el va veure mentre s'hi passejava. Va tornar a
tremolar. Per un breu moment s'havia sentit segura amb aquell vell home
estrany, i ara tornava a estar sola, ella\ldots{}

Ella el va agafar en pocs segons.

--L'entrada és per aquí--va dir.

Ell li va fer un somriure.

--Bé per tu, nena--va dir--. T'haig de reconèixer que tampoc és que
m'agradés haver d'anar-hi sol.

Es van somriure l'un a l'altre.

--Així que la Donna és la teva néta, oi? --va dir la senyoreta
Oladini--. M'alegro de que se n'anés.

--Jo també. Estaria perdut sense ella. La família és quelcom que s'ha de
mantenir.

La senyoreta Oladini ho va considerar.

--No sé on és la meva família ara mateix--va dir--. Probablement de
volta a Nigèria.

--Com és que heu perdut el contacte?

Ella va somriure.

--Oh, ja saps com és, vaig arribar al Regne Unit per a la universitat,
vaig perdre el meu estatus, vaig quedar-m'hi aquí amagada, vaig entrar a
una agència per trobar treball sota un fals nom, coses normals.

--Això és molt valent per la teva part--va dir en Wilf--. I arriscat
també, treballar aquí.

--Just a sota del nas del govern--va respondre--. És la forma més fàcil
de desaparèixer del radar:amagar-se a simple vista. És això el que el
meu pare em va dir l'últim cop que vaig parlar-li.

En Wilf va estar-hi d'acord.

--Jo acostumava a dir això dels espies, durant la guerra--va dir--. La
millor forma d'infiltrar-se és ser vist, així ningú no sospita de tu.
Converteix-te en un membre de la societat.

--És això el que vaig fer. Mira on m'he ficat. Espantada de per vida.

En Wilf li va picar l'ullet.

--Estaràs bé.

Estaven a la porta del radiotelescopi. Estava oberta a mitges i s'hi van
ficar.

El professor Melville era mort. No hi havia cap dubte, el seu coll
estava en un angle tan estrany i encara que la senyoreta Oladini no
havia vist mai a ningú mort, senzillament ho sabia. La seva mà cobria la
seva boca, silenciant el crit que volia sortir de la seva gola. En Wilf
va comprovar el pols del pobre home, però va deixar caure amb amabilitat
la seva mà.

Semblava que havia estat treballant en els sistemes de guia del Array
quan va morir. Quan va estar assassinat, va pensar la senyoreta Oladini.
Després de tot, la gent no es trencava els seus propis colls.

En Wilf estava pujant per la petita escala de mà que portava cap al pis
superior, on era el mateix telescopi. No era un antiquat telescopi
tubular, sinó un conjunt d'ordinadors ajuntats a través de la sala,
connectats a l'antena de ràdio al cim de l'edifici.

Allò sempre havia decebut a la senyoreta Oladini quan va arribar a
treballar per primer cop amb el pobre professor. D'alguna manera un
gegantí telescopi hauria semblat més romàntic que un conjunt
d'ordinadors.

Va tornar a mirar cap al seu cos, amb els ulls oberts mirant cap al
sostre, i va pensar en la seva vella mare. I en la seva gata. I com
havia estat d'espantada, l'últim cop que l'havia vist. I ara tot en el
que podia pensar era en la seva gata.

I va començar a plorar per primer cop des que tot havia anat malament.

La Donna colpejava els botons de la ràdio fins que va trobar un canal.
No volia música, volia notícies. No va ser gaire difícil de trobar-ho.
Arreu del globus, massius rajos de llum havien caigut, i estaven
rodejats de persones. Alguns observadors deien que eren terroristes
vigilant un lloc bomba, alguns creien que eren fanàtics religiosos
vigilant quelcom sagrat i especial. Altres els creien alienígenes,
venint per reclamar aquells que havien estat abduïts i llavors havien
tornat amb microxip als seus caps durant els últims cinquanta anys.
L'estrany missatge de la ``Madam Delphi'', que tothom assumia que era
només un hacker d'Internet intentant ser graciós, es deia que ho havia
començat tot\ldots{}

--La ironia és--li va dir a la ràdio--, que aquesta és la que pot ser de
veritat.

No semblava que hi havia gaires víctimes però ningú no es podia apropar
a la gent vigilant els cràters. Anglaterra, Amèrica, Rússia, l'Orient
Mitjà, Àsia, Nova Zelanda, Àfrica, Grenlàndia, no hi havia lloc sense
tocar. No semblava haver-hi cap connexió entre la gent reunint-se al
voltant dels cràters: diferents edats, sexes, ideologies i passats.

--Em pregunto si els Pols hauran estat atacats--va murmurar després de
d'escoltar els informes una mica més mentre conduïen vora la Torre de
Londres.

--Hi va haver una menció d'un a Alemanya--va dir en Lukas.

La Donna va somriure.

--No parlava dels pols de Polònia, parlava dels Pols Nord i Sud.

--És que importa?

--Sí, probablement sí. Semblen ser zones bastant poblades, més que
desolades. Així que hi ha d'haver alguna cosa significant en això.

--Què?

--No en tinc ni idea. Però estic pensant--va veure un cartell que deia
A13 Tilbury--. Essex cap allà--llavors es va encongir d'espatlles--. No
és que sàpiga exactament on estic anant. Era negra nit i pensava massa
en el Doctor com per a fixar-me en els cartells.

--Has d'agafar l'A127--va afegir el Joe--. Tres milles per endavant
després de l'encreuament amb la M25, llavors cap a l'esquerra cap a
Meadow Lane, mitja milla més enllà i a la dreta cap a Gorsten Road.
Segueix per allà sis milles i mitja, llavors cap a l'esquerra, cap a
South Woodham Ferrers.

--Oh sí, recordo aquest nom--va dir la Donna--. Com ho saps?

--Després de passar el pont de la via, has d'anar vuit milles per la
carretera Tributary i mentre vas cap a la B8932, gira a l'esquerra cap
al carreró Allcomb. El Copernicus és a dues milles d'allà.

La Donna va mirar el Lukas, qui es va encongir d'espatlles.

--Sabia on trobar-te--li va dir a ella--. I a quina furgoneta hi
aniries.

--Això espanta--va dir la Donna, amb calma, mirant a en Joe pel
retrovisor.

--Així és el Joe--va dir el Lukas--. Gràcies a Déu que només som
germanastres.

--No diguis --li va castigar la Donna--. Tot i així és el teu germà.

--Sí--va coincidir el Lukas--. Però si fóssim germans de veritat, potser
ambdós seriem estranys. D'aquesta manera, puc traduir-ho si comença a
parlar en italià.

--Per què faria això?

--Perquè és d'allà d'on és el seu pare, segons la mare.

I alguna cosa es va encendre dins el cap de la Donna. Alguna cosa que el
Doctor havia dit al sopar la nit abans, quan ajudava a la Netty i el
vell Crossland creia que només bordava. Quan parlava d'aquella cosa
Mandràgora.

Primer se l'havia trobada al segle XV a Itàlia.

Alguna cosa dins del cap de la Donna es va connectar. Dofins bojos! Per
suposat! No podia ser així de senzill, per suposat\ldots{} però llavors,
va dir que havia estat cinc cents anys enrere. Prou temps per a la gent
d'Itàlia de viatjar pel món, tenir generacions de fills\ldots{} Aquell
home, l'última nit al telescopi que havia electrocutat al Doctor, el seu
accent podria haver estat italià. I havia parlat de genealogia\ldots{}

--I saps exactament d'on a Itàlia?

--No--va dir el Lukas.

--De San Martino--va afegir en Joe.

--He pensat que potser ho sabries--va dir la Donna.

--Per què? --va preguntar el Lukas.

--Perquè no crec que tingui poders de GPS, perque al darrere hi ha una
coincidència. Crec que alguna cosa l'està fent servir per fer-nos
arribar al telescopi per una raó.

--Vols dir que el meu germà petit és un alienígena?

--No sones gaire emocionat per la idea.

--No, és molt guai--en Lukas s'hi va apropar--. Sempre li he dit a la
mare que era estrany.

--No és un alienígena. Però pot haver alguna cosa al seu passat que
podrà ajudar al Doctor a arreglar-ho.

En Lukas va mirar enrere cap al seu germà, que ara escoltava el seu
M-TEK de nou.

--No vull que res li passi.

La Donna li va somriure.

--No li passarà res, el Doctor s'assegurarà de que està bé--però dins
seu, no estava segura de que pogués garantir-ho.

El s ulls del Doctor es van obrir i en Wilfred Mott va aparèixer a la
seva visió.

Ell va somriure.

--Hola, Wilf.

--Hola, Doctor--va dir el vell, ajudant-lo a aixecar-se--. Què feies al
terra?

--Estava llençat, dipositat, abandonat. Què desagradable! --va murmurar
el Doctor. Llavors va agafar a en Wilf pels dos braços--. On és la
Donna?

--Està bé. És a casa amb la Sylvia, pensant que ambdós som al hort.

--Bé. Excel·lent. Brillant fins i tot. I ara, perquè em van deixar aqui?

--Aquell munt de gent estranya amb l'electricitat morada?

--Sí, aquest són ells. Ostres, la Donna no es va deixar gairebé res, oi?

--La vaig fer dir-m'ho tot. Doctor?

--Sí?

--Hi ha un home mort allà fora, a l'oficina.

El Doctor va obrir la seva boca per parlar, però es va aturar.

--M'ho temia--va seguir en Wilf cap a la sala de control, fent un últim
cop d'ull al seu voltant, i comprovant alguns nombres a la pantalla. Un
moment després, estava deixant el cadàver d'en Melville fora al terra,
comprovant-lo.

--Tu has de ser la senyoreta Oladini? --va dir.

La senyoreta Oladini va fer que sí amb el cap.

--Com\ldots{}?

--Sabia que t'estaven buscant. El professor Melville em va demanar
d'intentar trobar-te. Mantenir-te segura. I alguna cosa d'una
gata\ldots{}?

--El professor Melville era viu llavors?

--Oh sí. L'obligaven a moure el radiotelescopi en alineació amb el Cos
del Caos allà a dalt.

En Wilf el va informar sobre els esdeveniments al voltant del món.

--Madam Delphi?

--Escriu columnes d'astrologia als diaris--va afegir la senyoreta
Oladini.

El Doctor li va fer una mirada.

--Ho sento--va dir ella--. Sé que és informació innecessària.

--Oh, no ho és, senyoreta Oladini. És informació brillant, de fet. Ens
explica molt. Si ets un súper-ésser alienígena el poder hèlix del qual
està governat per les estrelles, llavors qui hi haurà millor que una
astròloga paraules de la qual són llegides i devorades per milions que
l'empren com a mèdium? Necessitem trobar aquesta Madam Delphi i
preguntar-li d'on obté la informació.

--Per què assassinarien al professor Melville? --va preguntar
tranquil·lament la senyoreta Oladini.

--La Mandràgora és bastant bona per a les eines. Totes aquestes persones
que vas veure anit? Eren eines. Eines per desfer-se un cop no són útils.
M'imagino que el pobre professor Melville va fer el que necessitaven que
fes i senzillament se'l van treure de sobre--va posar una mà sobre
l'espatlla de la senyoreta Oladini--. Un estúpid i inútil gast d'un bon
home i un amic. Ho sento molt.

Ella li va fer un somriure.

--Hi ha alguna cosa que pugui fer per aturar-los?

El Doctor va tornar a mirar a la sala de control.

--Wilf, hi ha algun senyal d'algú més?

--Ningú.

--Crec que van haver d'anar-se'n sobre les nou de la matinada--va afegir
la senyoreta Oladini--. No els vaig poder veure gaire, però els vaig
escoltar parlar.

--Tens cotxe, Wilf?

--A l'aparcament.

--Bé, dóna-se'l a la senyoreta Oladini.

--Per què?

--Sí, per què?

--Perquè estàs viva, senyoreta Oladini, i li vaig fer una promesa a un
home de mantenir-te així. Ves-te'n a casa. Wilf, tens telèfon?

--Està mort, mai el recarrego.

--Telèfon?

En Wilf se'l va treure de la butxaca de la jaqueta i el Doctor el va
apuntar amb el sònic llavors va anar corrents cap a la sala de control.
Un moment després, va tornar i li va donar el telèfon a la senyoreta
Oladini.

--Recarregat, hauria de durar un parell de setmanes. Wilf, ho sento, he
hagut d'esborrar la targeta sim.

--La targeta què?

--Si no ho saps, és igual. Senyoreta Oladini, emmagatzemades a aquesta
targeta són les coordenades en que està posicionat el telescopi. També
he fet algunes coses i joguinejat amb ell, és a dir, que quan et truqui
a aquest telèfon, no respondràs.

--Com sabré que ets tu?

--Perquè ningú trucarà perquè que l'oblidadís amo sempre el té apagat i
deixa que s'acabi la bateria.

En Wilf va tossir.

--Així que--va seguir el Doctor--, quan et truqui, no responguis, sinó
que en lloc d'això, pressiona la tecla del coixinet. I passi el que
passi, no el pitgis accidentalment abans que et truqui jo.

En Wilf volia saber per què, però el Doctor va fer que no amb el cap.

--És més segur si no fas preguntes. Senyoreta Oladini, fes tot això, i
podràs ser responsable de salvar el món. Possiblement l'univers sencer.
Wilf, les claus.

En Wilf se les va passar entre dents i la senyoreta Oladini va posar el
telèfon amb amabilitat dins d'una de les butxaques dels seus abrics.

--Bona sort, senyoreta Oladini. I gràcies--va dir el Doctor--. Que vagi
bé.

Va llençar una última mirada trista a en Melville.

--Era encantador--va dir només.

--Ho sé--va dir el Doctor--. L'hauries d'haver conegut el 1958. Tenia
una banda de jazz del carrer, es deien els Geeks, i jo vaig tocar el
teclat per ells. En Joe Meek anava a produir l'àlbum. Nosaltres el dèiem
Ahab perquè el seu cognom era Melville. Avui dia però, no sé quin era el
seu nom de pila.

--Brian--va dir la senyoreta Oladini--. És el que deia la seva fitxa de
personal--va fer un somriure trist mentre el Doctor s'avançava.

--Brian--va dir el Doctor al cadàver--. Adéu, Brian.

En Wilf va mirar a la senyoreta Oladini.

--Estarà bé? I si aquesta gent encara ens vigila?

--No, se n'han anat. Ella estarà bé, viu a la ciutat. Podràs anar a
recollir el cotxe la setmana que ve. Si és que seguim vius.

--Encisador.

--Sempre és un risc--va mirar el rellotge--. Suposo que tenim una hora
per esbrinar perquè em van deixar amb vida.

--I després què passarà?

--La cavalleria.

En Dara Morgan estava dempeus a l'àtic de l'Hotel Oracle, mirant cap
avall a l'elevada autopista per sota.

--Semblen formigues--va dir la Caitlin a la seva espatlla--. Que és el
que més aviat són.

En Dara Morgan va obrir la boca, com si anés a parlar, i llavors la va
tancar.

--Estàs bé? --va preguntar la Caitlin.

Ell es va encongir d'espatlles.

--Em\ldots{} em sembla recordar alguna cosa. Cotxes de joguina. Puc
veure munts de petits cotxes de metall i un nen que juga amb ells. El
nen sembla\ldots{}

--Familiar?

--Anava a dir ``feliç'', de fet--en Dara Morgan es va apartar de la
finestra--. Madam Delphi--va dir a les pantalles de l'ordinador posades
per l'escriptori alineats amb una llarga paret--. Com va?

--És fantàstic--va respondre l'ordinador, amb unes ones brillant de
forma positiva--. Arreu del món, els fills del Mandràgora s'estan unint,
protegint els llocs d'arribada. I MorganTech ara controla el
vuitanta-set per cent de les franquícies mundials d'informàtica--va
riure la Madam Delphi--. Empassa't això, Wiliam Henry Gates III.

La Caitlin va començar a llegir alguns informes d'Internet.

--Ara hi ha un nou culte a la Mandràgora a Sud Amèrica--ella es va
riure--. Com de lluny va anar el Mandràgora l'altre cop?

Les pantalles de la Madam Delphi van parpellejar.

--Vam anar molt lluny. Oh, mira, una línia de llinatge sencera a
Noruega. És que hi ha algun lloc on no haguem arribat?

La Caitlin va teclejar un parell de tecles més.

--I una altra al Zaire!

--Això és veritablement el trenc d'alba de l'Era dels Aquaris! --va
cridar la Madam Delphi.

En Dara Morgan mirava els cotxes de nou. I a la finestra, dibuixa
inconscientment lletres amb el seu dit.

La Caitlin va mirar cap a dalt. I va arrufar les celles. En Dara Morgan
estava dibuixant una C. I llavors una F.

Es va aixecar i es va posar darrere seu d'immediat.

--Ei tu--va dir, portant-lo de nou--. La Madam Delphi té una cosa que
ensenyar-nos. Una desena part de la població mundial se'ns unirà. I quan
els M-TEKs surt a la venta aquesta semana, i tindrem sis cops més! Ja
hem repartit els prototips gratuïts ~que havien d'activar els nostres
fills adormits.

En Dara Morgan va fer un últim cop d'ull cap a la finestra, cap a
l'autopista M4 i llavors cap als potencials venedors del M-TEK.

--En quant en Murakami hagi arreglat les coses a Tokyo\ldots{}.

--Aquest món i totes les seves persones pertanyeran al Mandràgora! --va
dir la Madam Delphi--. Oh, i ja he penjat tota una sèrie de nous
horòscops! És meravellós!

En Wilf havia rescatat un abric vell d'una sala i el va portar a
l'observatori per cobrir el cadàver d'en Melville.

--Per què van matar al pobre home, Doctor?

El Doctor era a la petita sala de control, estudiant les lectures amb
cura, i amb igual cura intentant no tocar res.

--Probablement va posar tot això pel Mandràgora Hèlix, i llavors no el
necessitaven més. Comporta un percentatge del seu poder per controlar
poder, millor estalviar-lo per aquells que necessitin a llarg termini.

--Com per exemple?

El Doctor es va apartar dels controls i va fer que en Wilf s'apropés al
cadàver, apuntant amb el sònic a la porta rere seu.

--Ningú entra o surt fins que no ho digui jo--va murmurar. Llavors va
fer un somriure al Wilf--. M'agradaria tenir una resposta per a tu,
Wilf, però la veritat és que no tinc ni idea. Hi ha sempre un enllaç
entre la gent que esclavitza i la gent que aquesta esclavitza al seu
torn. Ara mateix, no tinc ni idea de què és, o com aquesta Madam Delphi
encaixa però estic suposant que està connectada amb la Mandràgora--de
sobte el Doctor es va fer un cop al cap amb la mà--. Oh, per suposat! Ja
ho veig! Wilf, quina hora és?

--Les dotze trenta-cinc.

--I tu has arribat aquí quan?

--Sobre les nou, volia arribar aquí perquè\ldots{}

El Doctor va aixecar una mà per fer-lo callar. Llavors ell va començar
un compte enrere.

--Cinc. Quatre. Tres. Dos. I\ldots{} u!

I en aquest moment la porta exterior del observatori es va obrir de cop,
enfonsant el terra amb llum solar.

Dempeus allà, emmarcada, estava la Donna Noble.

--La cavalleria, tal i com vaig prometre.

En Wilf va abraçar a la Donna.

--Com sabies que estàvem aquí?

El Doctor estava inclinat contra la paret, amb els braços creuats, tot
tranquil, però prou orgullós.

--Oh, perquè és la teva néta, Wilf, i ella és brillant!

La Donna va ignorar el compliment.

--Sé el què volia dir el paio ahir a la nit. I ell era italià. Són els
italians!

En Wilf va mirar un a l'altre.

--Què?

--Va dir alguna cosa sobre un home que llepava dofins bojos.

El Doctor va fer que sí amb el cap.

--Ho sé.

--Oh.

--Però vejam si coincidim, endavant.

--O--va somriure la Donna--, què passa si tinc més raó que tu?

--Improbable, però sempre possible. Dispara.

--El què va dir era ``L'home. Llepa dofins bojos''. Però no va dir
``dofins bojos'', va dir ``Madam Delphi'' --va somriure la Donna--. Sí,
ja ho havies suposat, oi?

El Doctor va fer que sí amb el cap.

--Encara segueixo donant voltes en el tros sobre llepar-la--va seguir la
Donna.

El Doctor li va somriure.

--Hèlix. És la Mandràgora Hèlix, Donna. Però no sé perquè el tros italià
és tan important\ldots{} Oh, oh, sí per suposat!

--La Itàlia del segle XV. San Martino, potser?

--De què parleu? --va preguntar en Wilf.

El Doctor li va mirar.

--És història, Wilf. El 1492, m'hi vaig trobar amb aquesta energia
alienígena dels Orígens dels Temps. La Mandràgora Hèlix, sempre
desitjosa de dominar les espècies inferiors.

--A qui dius tu inferior? --va preguntar la Donna.

--A la humanitat del segle XV, Donna. No a la humanitat del segle XXI,
oh no. Vosaltres sou molt més sofisticats--va somriure d'una manera que
suggeria que no era així exactament com percebia les coses, però ella ho
va deixar estar.

--Així que, de qualsevol manera--va seguir ell--, vaig portar
accidentalment un fragment d'aquesta energia Hèlix a un petit principat
italià anomenat San Martino. La vaig vèncer, molt llestament,
ensorrant-la. O això pensava. Però és ~això, literalment, el que vaig
fer, ficar-la al fons del terra, on va sobreviure, intentant arreglar-se
a si mateixa. Es va filtrar al terra, a l'aigua i finalment en la gent.
Una petita entitat biològica afegint-se als cromosomes, a l'ADN, el que
sigui. Es va transferir de generació en generació fins que tot el
conjunt de la Mandràgora es va manifestar a si mateixa a l'altre costat
de l'univers i es van connectar. L'últim cop, volia frenar el progrés
humà. Aquest cop, la Mandràgora s'ha adonat de que no hi ha manera
d'aturar-vos, que sereu allà fora, enfonsant les estrelles en poc temps
amb la humanitat, creant colònies, imperis, guerres i treves, fins al
final del temps. Així que la Mandràgora hagi dit, ``Jo prendré una mica
d'això, gràcies'', i es enganxar a si mateixa amb tu per tota
l'eternitat. Un gran pla, pot manipular-vos a tots vosaltres durant el
mil·lenni que ha de venir.

--Així que al voltant del món--va dir Donna--, diluint-se a través dels
fills i tot això, hi ha descendents de San Martino arreu del món.
Centenars d'aquestos ara, probablement la meitat d'ells sense saber tan
sols que tenen sang italiana en ells. I la Mandràgora els està
controlant--es va girar cap al Doctor--. Així es com en Joe Carnes sabia
que series aquí. El seu pare és de San Martino.

--Com has sabut això?

--El Joe li va dir--en Lukas Carnes va treure el cap per la porta--. Ha
acabat al lavabo, Donna--va afegir mentre els dos nois entraven.

--Oh, sí--la Donna va somriure dèbilment al Doctor--. Digues hola al meu
equip d'adjuvants.

El Doctor estava molt emocionat de veure'ls.

--Així es com ens vas trobar, oi? Creia que era improbable que hagués
memoritzat la ruta del taxi d'ahir a la nit.

--El Joe és com un gosset guia--va dir la Donna.

El Doctor va posar la mà sobre l'espatlla del Wilf.

--Hem acabat aquí. Anem a casa. Passant per Greenwich.

--Greenwich? --en Wilf va arrufar les celles--. Oh no. No, Doctor, no
involucris a la Netty, si us plau!

--Crec de veritat que ens pot ajudar, Wilf. Ho sento.

--Qui és aquest? --el Lukas senyalava al cadàver cobert per un abric a
la cantonada.

El Doctor va respirar fons.

--Un molt bon amic meu, Lukas. Ha mort--va mirar en Wilf--. Però és
l'últim amic que mor a les mans de la Mandràgora Hèlix, ho prometo.

La Donna ho va creure, recordant la seva promesa de la furgoneta que li
va fer a en Lukas pel Joe. Esperava que el Doctor no els decebés.

Mentre passava davant seu, li va picar l'ullet i va somriure.

En què pensava? Així era el Doctor.

Per suposat que tot estaria bé.

Com no podria estar-ho?

El viatge de tornada a Londres va ser tranquil, per menys. En Wilf es va
asseure al seient davanter al costat de Donna, girant de cop
ocasionalment quan gairebé arrenca retrovisors dels cotxes aparcats. El
Doctor i els dos nois s'asseien darrere quan es van posar mantes, una
ampolla d'aigua i una capsa d'eines.

El Doctor va trobar un llibre de butxaca sota el seient anomenat ``Una
fosca i tempestuosa nit'', sobre reis rics, pirates, criades espantades,
pastors de ramats forts i una noia jova que va ser trobada a la neu. El
Doctor va simpatitzar amb l'heroi de la història, un jove intern
d'hospital que intentava ajuntar els forassenyats elements tot junts.

Després d'una estona, es va rendir i el va llençar als nois. En Lukas va
començar a llegir-ho i al mateix temps mantenia un ull protector en el
Joe mentre el Doctor li preguntava sobre el seu pare perdut fa temps.

No va obtenir respostes útils. En Joe no podia recordar gaire anar a la
botiga d'electrodomèstics el divendres a la tarda, però si no hagués
estat pel prototip M-TEK gratuït, mai no hauria sabut que hi va ser
allà.

--Té dies com aquest--va murmurar en Lukas.

--Què és un M-TEK quan és a casa, llavors?

En Lukas va tornar al llibre mentre el Joe li ensenyava al Doctor el
petit aparell portàtil.

--És com un reproductor d'MP3 que també reprodueix pel·lícules--va dir
el Joe--. Es connecta amb la xarxa, és com un telèfon i també té una
memòria de 160 GB, així que pots guardar-hi coses dins. Fa servir
Windows i OSX6 molt ràpidament.

El Doctor va fer que sí amb el cap, impresionat.

--Les grans coses venen en petits embolcalls--va dir, i en un moment va
treure el seu tornavís sònic i va analitzar l'M-TEK amb ell.

Reconeixent el soroll, la Donna va dir:

--Espero que li compris un substitut.

Però el Doctor arrufà les celles. El sònic no havia fet res en absolut.
Ni tan sols barrejar la música emmagatzemada.

--Això és\ldots{}

--Estrany? --va oferir la Donna.

--Més que estrany--va coincidir. Es va inclinar a la part del darrere de
la furgoneta, va trobar la capsa d'eines, va treure un pesat martell i
el va colpejar amb l'M-TEK. El cop del martell, el crit de fúria del Joe
i la maledicció molt alta d'en Lukas gairebé va fer que la Donna se'n
anés a la cuneta mentre giraven cap al túnel Blackwall.

--I bé? --va preguntar en Wilf.

El Doctor va aixecar el M-TEK.

--Ni tan sols una esgarrapada, ni una osca, res. Això és bona
tecnologia. Tecnologia alienígena, però bona tecnologia. I també és
impossible--va somriure als espantats nens--. Oh, sempre m'agrada tot el
que és impossible.

--Algú ha notat alguna altra cosa estranya? --va preguntar la Donna.

--Encara seguim vius després de que hagis conduit durant una hora? --va
suggerir en Wilf.

--No hi ha trànsit--va suggerir en Lukas.

El Doctor va aixecar la vista.

--És això veritat?

La Donna va fer que sí amb el cap.

--Hi ha molts cotxes aparcats. De fet he vist només altres tres cotxes
en moviment des que hem abandonat Copper Knickers. Un d'ells no deixava
de seguir-nos, creia que estava perseguint-nos.

--Probablement ho feia--va dir el Wilf--. Però l'has deixat enrere.

--Però creia que també intentava fer-nos fora--la Donna va ignorar al
seu avi--. Perquè això és boig. On és tothom?

--És diumenge? --va suggerir el Doctor.

--És el sud-est de Londres--va informar Donna--, i no som al segle X.
Hauria d'haver centenars de cotxes.

--M'agrada prou--va dir en Wilf--. Tot tranquil. Pren la pròxima
sortida, reina, la Netty viu just a la carretera principal.

Van aparcar just davant de la casa de la Netty en silenci.

En Wilf va sortir i va fer sonar el timbre, però no hi havia ningú. Va
trucar-la a través del forat del correu i, després d'un segon o dos, la
porta es va obrir i ell s'hi va ficar, quedant fora de la vista.

La Donna, davant de la furgoneta, va mirar al Doctor.

--Has vist això?

--Era la Netty--va dir el Doctor.

--Com ho saps?

--Els alienígenes mai no vestirien barrets com aquell.

La porta es va reobrir i va sortir en Wil, seguit per la Netty vestida
amb un barret ver de felpa amb una ploma de paó real, tot dels anys 50.

--Has vist les notícies, Doctor? --va preguntar la Netty, pujant-se a la
furgoneta i asseient-se al costat del Wilf.

Va respondre que no.

--Llavors el millor que podem fer és conduir pel centre de Londres.

Intrigada, la Donna va arrancar la furgoneta i hi van anar.

A través de Greenwich, després de la reconstrució de Cutty Sark i tots
els mercats i botigues. A través del New Cross, per sota de l'antiga
carretera de Kent, al voltant de l'Elephant \& Castle i per sobre el
pont Blackfriars.

--Ni un ànima--va dir en Wilf--. Ningú.

--La BBC deia a tothom que no sortís de casa. Fairchild ha declarat
l'estat d'emergència.

--Fairchild? --va preguntar el Doctor.

--El primer ministre--va dir Lukas amb un sospir--. Que no saps res?

--Conec munts de primers ministres--va dir el Doctor--. Però aquest
segle van i venen anualment, això sembla. ~Aquest clarament no marca a
la història.

La Donna va frenar la furgoneta de cop i va dir, tranquilament:

--Oh.

Aquells a la part del darrere de la furgoneta es van inclinar cap
endavant:

--Oh, prou--va dir el Doctor.

Perquè no podien anar més enllà. Estaven a l'Embankment, just a sota de
l'estació de Charing Cross.

Hi havia possiblement un milió d'altres persones. De peu. Quietes. Amb
els braços aixecats cap al cel. Tots cantaven.

--Hèlix. Hèlix. Hèlix.

--Això no és bo--va dir la Donna.

El Doctor li va passar a ella l'M-TEK del Joe.

--Truca a la teva mare, si us plau.

--Per què?

--Per fer-li saber que estem bé i que la veurem demà al amtí.

--Prioritats? --va preguntar la Donna.

--Mantenir de bon humor a la teva mare és una prioritat, Donna. Per
ambdós. Estarà preocupada--es va girar cap als nois Carnes--. Llavors
trucarem a la vostra mare, haurà d'estar molt preocupada.

--No ho estarà--va dir el Joe--. Serà una d'aquest munt.

En Wilf estava a punt de preguntar perquè, però el Doctor va negar el
seu cap.

--Ara, Joe, just perquè és la teva mare, no està en perill. Cap
d'aquestes persones ho està, pel que sembla.

--Ella el va tenir a ell. Potser també té aquest gen Hèlix, oi? --va
preguntar el Lukas--. Gràcies a Déu que soc el germà gran.

En Joe va mirar fora de la furgoneta.

--Què fem per rescatar-la, Doctor?

El Doctor va somriure.

--Aquest és l'esperit, nois, recordeu que la podem salvar. Podem salvar
tota aquesta gent.

La Donna va passar l'M-TEK enrere.

--Diu que Chiswick està buit. Li he dit que es quedi a dins, begui te i
mantingui la televisió encesa. Li he dit que faci qualsevol el que digui
la BBC, excepte que tingui a veure amb abandonar casa seva i deixar de
beure te. No li ha vist la gràcia.

--No em sorprèn-- va dir el Wilf--. Bé, Doctor, què fem?

El Doctor estava mirant l'M-TEK.

--Ells us van donar això, oi, Joe? Han donat gaires gratuïts?

En Joe va fer que sí amb el cap.

--Al fòrum, deien que donaven un milió gratuïts abans de la seva sortida
a la venta de demà.

--M'aposto el que voleu que l'objectiu també era d'una específica
genealogia. Així que aquesta cosa es vendrà a escala nacional demà?

--A escala mundial--va afegir la Netty--. Jo agafaré un. M'agraden les
coses com aquesta. Anava a esperar un mes o així, per veure si el canal
de televenta els feia més barats.

--Oh, m'agrada això--va dir la Donna--. Bé, m'agradava. Quan tenia
temps. Aquell Anis Ahmed feia coses per mi\ldots{}

Netty va riure.

--És tan sexy\ldots{}

En Wilf va tosir.

--El que sigui, tornant al tema que ens portem entre mans\ldots{}
Doctor, no podem aparcar aquí.

El Doctor seguia jugant amb el seu M-TEK.

--Res està tan ben protegit\ldots{} Si pogués reescriure una part del
software\ldots{} --el sònic va fer un parell de ombres blaves diferents,
llavors el M-TEK va fer un sorollet, i el Doctor va fer un crit
d'alegria. Llavors es va aturar--. Semblo haver accedit al web
d'horòscops. Ah, la nostra vella amiga Madam Delphi.

--Treballa per la get que ha fet l'M-TEK--va dir en Lukas--. Escriu les
coses aquestes dels horòscops per alguns dels seus diaris.

El Doctor va mirar als nens.

--El què?

--MorganTech, fan tot avui dia. Tenen canals de televisió, diaris,
treballs, etc--en Lukas es va encongir d'espatlles--. Pensa en una
barreja de Bill Gates, Ruper Murdoch i Richar dBranson i tindràs a en
Dara Morgan.

--I qui és aquest quan és a casa seva? --va preguntar el Doctor.

--Dirigeix MorganTech. Ha estat per aquí els últims anys. Vam fer
recerca sobre ell a l'escola, però no hi ha gaire d'ell. No li agraden
les biografies no autoritzades.

El Doctor va mirar al grup de la furgoneta.

--Deixeu-me entendre això. Tenim rajos de llum colpejant el terra,
hipnotitzant gent cantant a les estrelles gràcies a una astròloga dels
diaris que et diu que canviarà el món, nous aparells estan sent donats
gratuïtament a gent que són descendents d'italians i ningú pensa que
està tot connectat?

La resta es van mirar els uns als altres. La Donna va parlar al cap
d'una estona.

--No podem esperar tenir aquests salts de lògica que fas tu, ja ho saps.

--No són salts de lògica, són clarament camins definits d'evidències
i\ldots{} oh, no importa. On puc trobar aquest MorganTech?

--Vora on vivim--va dir el Lukas--. A Brentford.

--Oh, és veritat--va dir en Wilf--. Tenen aquella gran oficina i l'hotel
a la Milla Daurada.

--I l'home a càrrec es diu Dara Morgan?

--Sí.

--Per suposat que sí--va murmurar el Doctor--. Clar que sí. Lukas, vull
que intentis recordar tot el que puguis sobre ell, d'acord--el Doctor va
deixar el M-TEK al terra i el Joe es va aixecar per agafar-ho--.
Deixa-ho, Joe, és perillós.

Va apuntar el sònic a la part del darrere de la furgoneta i les portes
es van obrir.

--Vinga va, no podrem passar d'aquesta gent, i necessitem caminar i
trobar transport nou.

--Doctor--va protestar el Wilf--. La Netty està\ldots{}

--Ei--va dir la Netty--. Puc caminar igual que tu pots, Wilfred Mott--va
enllaçar el seu braç amb el d'ell--. Podem ajudar-nos l'un a l'altre.

Li va fer un somriure.

I la Donna anava a fer el mateix quan va veure la mirada a la cara del
Doctor. ~Era la mateixa que va tenir al sopar la nit anterior.

Mirava a la Netty de forma estranya.

La Donna va posar els nois més a prop seu.

--Quedeu-vos amb mi--els va dir--, i ajudarem al Doctor a acabar amb tot
això.

--Donna--va dir el Doctor de cop, i d'una forma a la que la Donna hi
estava acostumada. Era una veu d'advertència.

Entre ells i la multitud cantaire hi havia un grup de persones. Persones
que la Donna va reconèixer de la nit anterior, al Copernicus Array.

--No és bo?

--No és bo.

--Com has reprogramat el M-TEK? --va preguntar l'home menut al capdavant
del grup. La Donna també se'n va recordar d'ell. Ells els havia guiat, i
llavors ella es va adonar que l'accent era, per suposat, italià.

--Talent meu--va dir el Doctor.

--Això no és part del pla--va dir l'home menut--. No podem permetre
trencar el petit eslavó de la cadena.

--Oh, ho sento--va dir el Doctor, indicant amb la seva mà a la resta del
grup que s'apartés, lleugerament darrere d'ell, deixant-se a si mateix
entre el grup poderós per la Mandràgora i la furgoneta--. Ho he deixat
al darrere. Voleu que vagi a buscar-ho?

--Ho deixaràs--va dir l'italià, i es va obrir pas amb el Doctor i es va
ficar dins la furgoneta.

El Doctor va somriure a la resta del grup. Una parella gran a un costat,
quatre joves al final i un home forçut a l'esquerra.

--Em pregunto quants de vosaltres sou vertaders descendents de San
Martino, i quants vosaltres sou només els seus\ldots{} esclaus?
Adjuvants? Participants involuntaris en l'assassinat de professors als
observatoris? Si podeu combatre la Mandràgora, potser podem\ldots{}

El Doctor va fer un cop fort a l'asfalt mentre la furgoneta blava
explotava en flames i deixalles.

La Donna i els nois ja corrien, en Wilf i la Netty anaven darrere seu.

Bé.

Va mirar cap a la pira funerària del menut home italià que havia estat
un cop una furgoneta.

--Aquesta és una forma d'eliminar el dèbil M-TEK, suposo--va dir--. Una
mica ``exagerapp'' si em pregunteu.

I quan es va aixecar, es va veure vorejar pel grup. L'home forçut
semblava ser el seu nou líder i quan va parlar, el Doctor va reconèixer
un fort accent grec.

--La Madam Delphi vol veure't.

--Bé, d'acord, però vull veure si els meus amics estan bé.

--També venen amb nosaltres.

--Oh, no estic segur de poder accedir amb aquesta part del tracte.

--O et matarem ara mateix--va afegir el grec.

I en aquest punt, en Wilf, la Netty, la Donna i els nens van sortir del
seu amagatall i van ser vorejats ràpidament.

El Doctor va fer un sospir.

--Crec que probablement era un bluf--va dir al Wilf--. Em volien amb
vida, recordeu?

--Estem junts en això--va dir en Wilf--. Quan estava a l'exèrcit, mai no
deixàvem a ningú enrere.

El Doctor va fer que sí amb el cap.

--Oh bé, ara estem tots aquí. Hem de contractar un autocar?

--Caminarem--va dir una de les persones grans, una dona americana.

--És un llarg camí--va dir la Donna.

Un dels homes joves es va encongir d'espatlles.

--Així ens mantindrà en forma, llavors.

I van començar a caminar a través de Londres.

Allà on fossin, petits grups de gent estaven juntes, cantant al cel.

Altres es podien veure amagats, espantats, ocasionalment assaltant
botigues, probablement assumint que no acabaria allò mai, i que aquell
menjar es tornaria escàs.

--És com quan el Blitz--va dir la Netty en un indret, mentre caminaven a
través de la plaça Leicester.

--Sense les bombes i els edificis derruïts--va dir el Doctor--. Gràcies
al cel.

El Doctor va permetre que el seu grup es separés lleugerament, va notar
la Donna. Marcava el pas amb els nois i el Wilf s'estava cansant i només
anava unes passes per davant. Rere seu, el grec i els vells americans.
Per davant del Doctor, les quatre persones joves.

El Doctor estava amb la Netty, canviant-se el lloc amb el Wilf, amb el
seu braç unit amb el d'ella.

La Donna no podia escoltar el que deia, però la Donna podria dir que amb
la urgència amb la que ell feia anar la mà i la falta de respostes de la
dona més que un assentiment de cap lleuger, que no discutien gaire sobre
l'arquitectura de Londres.

Gairebé li va preguntar al seu avi de què creia que estaven parlant,
però no ho va fer. Perquè si anava tot malament, si resultava que era
malament, no volia fer al Doctor responsable de res.

La Donna es va adonar de que aquella era la primera vegada que s'havia
trobat a si mateixa qüestionant les accions del Doctor durant molt de
temps. I no li agradava.

Un parell d'hores van passar. Els havien deixat parar ocasionalment, els
joves buscant menjar (i a vegades feien servir els poders de la
Mandràgora per llençar portes avall i agafar les coses).

En un moment determinat, el Doctor i la Netty s'havien assegut junts en
un restaurant de menjar ràpid abandonat, mentre els nois menjaven
patates fregides fredes i magdalenes. La Netty havia trobat una mica de
paper en una taula i li escrivia alguna cosa, i el Doctor feia que sí
amb el cap.

En Wilf va preguntar a la Dona si hi havia alguna opció d'intentar els
microones, i quan va mirar cap a la Netty estava sola, i el Doctor
intentava parlar amb el grec.

No hi havia temps de passar a les hamburgueses pel microones, doncs els
van fer dir que comencessin a caminar de nou, malgrat les protestes del
Doctor.

Els nois es van cansar aviat de nou. La Netty i en Wilf estaven ja molt
cansats. La Donna estava completament exhausta, però el Doctor\ldots{}
va continuar cap endavant. Tenia en Lukas i en Joe davant seu en aquell
moment, intentant distreure'ls explicant-los la història de la carretera
Cromwell i els diferents edificis que hi havia mentre hi passaven.

La vella parella d'americans haurien d'estar morts dempeus, però no,
allí eren sempre, un o l'altre, alguns cops ambdós, amb els seus braços
apuntant cap endavant, llestos per fer servir el seu poder de la
Mandràgora com havien vist al Copernicus Array la nit anterior.

Era fosc quan van arribar a Hammersmith, i la Donna va saber que
trigarien una altra hora o així per arrivar a Brentford. Possiblement
més, ja que la Netty i el Wilf es paraven més i més sovint.

--El meu avi és gran--va dir en un moment al grec, demostrant enuig,
sinó cansament--. Ei, estic bé--va dir el Wilf.

El grec només es va encongir d'espatlles i va dir que la Madam Delphi no
li agradaria esperar.

No era una freda nit, però tampoc no és que fos estiu, i quan van
començar a caminar per la carretera Great West sense gent, ni cotxes,
era gairebé mitjanit.

La Donna estava amb el Doctor. El Wilf i la Netty estaven amb els nois
Carnes.

En Wilf intentava animar els seus esperits decaiguts amb històries de
les seves aventures al règim de paracaigudistes, com feia per la Donna
quan tenia la seva edat, als llargs viatges en cotxe i no per les
doloroses caminades a través de ciutats esgarrifoses.

--Per què no deixes que aquesta gent se'n vagi a casa? --va suggerir el
Doctor, aturant-se de cop--. La Madam Delphi només em vol a mi, estic
segur. Mira, som a Chiswick. Deixa que la Donna s'emporti al Wilf i a la
Netty a casa. I deixa que els nens se'n vagin també. Si us plau?

El grec el va ignorar i va continuar endavant.

--No és que l'avi o jo et vulguem deixar ni un moment--li va xiuxiuejar
la Donna mentre corria per agafar al Doctor--, però per què creus que
ens volen a tots?

El Doctor li va mirar directament.

--Assegurança--va dir senzillament--. Per amenaçar-me a fer-me mal,
segurament. De qualsevol manera, em necessiten viu per qualsevol raó.
Amenaçant-me amb matar-vos, és normal. Ho sento.

--No ho sentis--va dir en Wilf--. Hem escollit d'involucrar-nos amb tot
això. Estic orgullós d'estar al teu costat, Doctor. Igual que els meus
nens soldats aquí.

Els nois Carnes van fer que sí amb el cap, Lukas una mica més
entusiasmat que el Joe, tot havia de ser dit.

El Doctor va mirar a la Netty. Començava a passejar erràticament, vagant
cap a la reserva central d'arbustos.

--És el cansament--va dir el Doctor, amb tristesa mentre en Wilf anava a
guiar-la de volta al grup--. La seva ment se'n va de nou com l'última
nit.

--Llavors per què l'has portada amb nosaltres? --va dir la Donna una
mica més agressivament del que pretenia.

--No m'esperava caminar--va dir el Doctor--. Ho sento.

La Donna es va deixar endarrer un parell de passes. Alguna cosa al pla
del Doctor havia anat malament, i estava preocupat.

Allò no era bon senyal.

De sobte un conjunt de llums els van il·luminar, i el grup del Doctor es
va tapar els ulls tots junts. Un petit minibús va frenar davant seu.

--Ei! --va cridar la Donna--. Ens has ajudat!

El Doctor va anar a frenar a la Donna, però no importava. La porta del
minibus es va obrir i una dona els va cridar.

--Pugeu, col·legues--va dir amb un alegre accent irlandès--. La Madam
Delphi els espera.

Un per un, s'hi van pujar.

--No podries haver arribat fa tres hores? --va grunyir en Wilf mentre
ajudava a una confusa Netty a pujar els esglaons del vehicle.

La dona es va riure.

--Sóc la Caitlin, i en representació de MorganTech, em disculpo pel seu
malestar. Però no és res amb el que vindrà. I no, la Madam Delphi creu
que els presoners cansats són més mal·leables que els sans i capaços.
L'única raó per la que sóc aquí és perquè és gairebé mitjanit. I se'ns
acaba el temps. Pugeu!

~La Caitlin va fer una volta en forma de U i va recórrer l'A4 cap a la
zona de negocis de Brentford coneguda com la Milla Daurada.

--Ja hem arribat--va dir la Caitlin, anant més a poc a poc.

Davant seu, la Donna va veure l'Hotel Oracle brillant entre la foscor,
amb llums a cada finestra.

--Ja! --va riure el Doctro--. Anem a veure Delphi a l'Oracle. Molt agut.
O no.

--És mitjanit--va anunciar la Caitlin mentre obria la porta del
minibus--. Avui és dilluns. L'univers mai no tornarà a ser el mateix.

I va somriure.

I la Donna va tenir un calfred.