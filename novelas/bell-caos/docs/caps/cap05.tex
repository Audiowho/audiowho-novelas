\chapter*{DILLUNS}
\addcontentsline{toc}{chapter}{DILLUNS}

El Doctor, la Donna, en Wilf i els seus amics van ser portats a la suite
de l'àtic per la Caitlin, que mantenia la seva mà descansant sobre la
culata d'un revòlver ficat dins de la cintura dels seus pantalons.

Mentre l'irlandesa obria les portes de l'àtic, el Doctor hi va entrar i
va fer una ullada. Va començar a aplaudir lentament quan va veure el que
hi havia dins.

--Madam Delphi, suposo? --va dir--. Per suposat. No ets una persona
real, oi? Ets un ordinador! Bé, jo en dic ordinador, però ets més una
intel·ligència artificial, hostatjant una antiga malevolència que mai no
hauria d'haver estat alliberada de la seva dimensió. Com estàs,
Mandràgora? Han passat segles.

--Aquesta forma\ldots{} és molt més capaç que un rodanxó cos humà,
Doctor--va dir la Madam Delphi--. Com a Senyor del Temps, com algú que
pot aguantar molt més que un trauma espai-temporal, el teu cos és el que
el Doctor mana, i disculpa el terriblement dolent joc de paraules.

El Doctor no va dir res.

--Ja has escoltat aquesta, oi? --va preguntar la Madam Delphi.

El Doctor i la Donna ara eren dempeus davant del seu grup exhaust, el
Wilf i la Netty, el Lukas i en Joe rondant un parell de passes darrere
seu. Davant seu, en un protector cercle al voltant de l'ordinador de
Madam Delphi, estava en Dara Morgan, la Caitlin i els conversos de la
Mandràgora que havien caminat fins allà.

--Oh, hola--va dir el Doctor, com si s'adrecés a una trobada
internacional--. Això sembla bastant impressionant. Bonica sala. Bonic
hotel. Bonic gest--va senyalar cap on era la vella americana havia
aixecat els seus braços en una ara-identificable posició mandragoriana
de tret foc amb la letal energia Hèlix--. Tot i que una mica poc
amistós.

--Em disculpo. No es pot confiar en el personal--la veu femenina de
l'ordinador va ressonar pels altaveus col·locats al voltant de
l'habitació--. Benvinguts al meu hotel. Us puc recomanar el gimnàs?
M'han dit que tenim una gran piscina.

--Com és el bar? --va preguntar la Donna--. Vull dir, no hi ha un hotel
de cinc estrelles sense un bon bar, oi?

--Ah, Donna Noble, benvinguda tu també. Crec que trobareu que oferim
quatre bars, tres restaurants i un self-service a la carta 24/7--llavors
la Madam Delphi va riure--. Haig de dir, tot i així, que aspirem a un
reconeixement més gran que només cinc estrelles.

El Doctor va fer que sí amb el cap.

--Bé, suposo que busqueu les cinc milions d'estrelles. Què en penses,
Donna?

--Hauran de tenir un bon servei per aconseguir les cinc milions
d'estrelles, Doctor. Recordes aquell hotel a Cassius? Allò sí que era un
hotel cinc-estrelles en condicions.

--Oh sí! --el Doctor va somriure--. I entenien com havia de ser una bona
relació amb el client. Recordes quan vam tenir un petit problema amb
aquell llangardaix?

--Que teniu problemes amb els llangardaixos a Brentford, Madam Delphi?
--va preguntar la Donna--. Perquè si hi ha problemes de llangardaixos
per resoldre, no crec que sigui un hotel tan bo.

--L'Oracle és\ldots{} --va començar en Dara Morgan, però la Madam Delphi
el va fer callar.

--El Doctor i la seva dolça amiga només guanyen temps, Dara. Intenten
saber com guanyar-nos, com sortir de l'Oracle amb vida, com ``ajudar''
al seu preciós planeta Terra--la Madam Delphi va guardar un moment i
llavors va seguir, més dolçament i per tant més lleugerament
amenaçant--. Però no ens detindràs de debò, Doctor. No ofereixo
garanties sobre la gent sortint amb vida d'aquí. I, per la meva
perspectiva, ajudar la Terra és precisament el que estem fent.

El Doctor va caminar cap al grup, i van apartar-se gairebé inclinant-se,
així que ara estava mirant directament a les pantalles de l'ordinador.

--L'últim cop que vam xerrar, et vaig enviar cap a la foscor, llepant-te
les ferides, ho recordes?

--Per suposat--les suaus ones de la Madam Delphi van polsar feroçment--.
He esperat tant de temps per tenir una oportunitat d'arribar a tu
personalment. És hora de fer-t'ho pagar.

--Oh, no facis el truc de la dolenta venjança cap al pobre Senyor del
Temps, Mandràgora? Vull dir, ets millor que això. Vinga va, dóna'ns una
bona raó.

La Madam Delphi va riure.

--No és la primera vegada des del 1492 que la Mandràgora Hèlix ha estat
a la Terra, ja ho saps.

--Sí, això ho sé. La sagrada muntanya de Xi'an, si no recordo malament?
Llavors eren els Orfes del Futur, tots amb aquelles caputxes blanques i
carmesí. Oh, i allò del club nocturn Mandrake també va estar bé, ho haig
de dir. Però cada vegada, només ha estat un fragment de l'energia Hèlix,
no és així? Una petita espurna per a comprovar les aigües. Per què
enviar-me petits missatges pel paper psíquic per involucrar-me, per
portar-me aquí\ldots{} Ah\ldots{} Sí, volies que fos aquí. En aquest dia
i en aquesta hora. Per què?

--Les estrelles estan alineades--va dir en Dara Morgan.

--Estic parlant amb la Madam Delphi, gràcies, no amb l'ajuda llogada.

--Com goses\ldots{}? --va començar en Dara Morgan.

--Oh, tanca el bec! --li va engegar el Doctor--. Vull dir, qui ets tu,
de qualsevol manera?

--Sóc en Dara Morgan. Vaig obrir MorganTech. Vaig crear el M-TEK, jo
vaig dissenyar\ldots{}

--Oh, si us plau; no has fet res que la Mandràgora Hèlix no et digues
que fessis. No, qui ets realment? Qui es va emportar el Hèlix,
distorsionar, manipular i fotre per complert abans de reinventar-te com
Dara Morgan?

--Què?

--Lukas? --va rugir el Doctor--. El meu adjuvant de recerca--va explicar
tranquil·lament a la Madam Delphi--. La Donna estava ocupada. Per
problemes familiars.

La Donna va arrufar les celles. No és perquè fes que en Lukas Carnes fes
les seves recerques, sinó perquè havia dit ``problemes familiars''. Li
va llençar una mirada a en Wilf però es va encongir d'espatlles. Llavors
va mirar a la Netty, mirant intencionadament ~a la distància entre ells.
Quan la Donna va tornar a mirar al Doctor, va reconéixer una mirada als
seus ulls. Una mirada que si li hagués donat veu, hauria estat una
variació de: ``Ho sento, ho sento moltíssim''.

--No--va dir sense fer sorolls--. No gosis! --però l'atenció del Doctor
tornava a estar al Dara Morgan.

--Arreu del món, Dara Morgan, milions de persones cauran víctimes
d'aquesta consciència alienígena a la que has donat accés al món. I això
passarà avui.

--Ho sé--va somriure en Dara Morgan--. Com és de brillant això.

--Bé, és brillant des del punt de vista del teu M-TEK essent una
meravella tecnològica, augmentada per un alienígena que en sap i
distribuïda bastant magníficament per persones que, imagino, no en tenen
ni idea del que voleu fer amb ells.

--Ni idea.

--Hi ha molta sang a les teves mans, Dara Morgan. Si jo fos un policia,
faria que t'arrestessin però, com ara explicarà en Lukas, no és
possible.

--En Dara Morgan va tenir protagonisme vuit anys enrere, fent les seves
primers aparicions sobre el MorganTech a un especial de les notícies,
emès en directe el 31 de desembre de 1999.

--Just al final del mil·lenni.

--Abans d'això, no hi ha cap pista de tal persona. El MorganTech va ser
registrat per una empresa privada a les 5.29 pm del mateix dia.

--Així que qui eres abans que la Mandràgora se t'apropiés? Abans de
reinventar-te com un humà, convertint-te en l'anagrama Dara Morgan?

--Ah, ara ho entenc--va cridar en Wilf--. És molt intel·ligent.

-- Sí, gràcies, avi--va xiuxiuejar la Donna--. Però deixa que el Doctor
es centri.

Abans que ningú el pogués aturar, el Doctor va posar les seves mans a
cada costat del cap d'en Dara Morgan, amb els dits pressionant contra
les temples i va xiuxiuejar:

--Obre les portes tancades i deixa't sortir.

Els acòlits reunits van donar una passa endavant cap al Doctori la Madam
Delphi va polsar amenaçant:

--Atureu-lo--va dir.

A l'ull de la seva ment, el Doctor podia veure una imatge. Una fosca
nit, freda i humida. Ell caminava per un carreró, amb les voreres a cada
costat i la pluja caient pel seu coll.

Va tenir un calfred. Estava enfada\ldots{} No, no estava enfadat. Estava
ferit. Enfurismat. Ella havia dit que no. No a què? Qui era ella? A la
seva mà hi havia una caixa, suau i de folre. I dins d'ella, sí, s'ho
podia imaginar. Una banda de plata i un diamant pla. Tot el que havia
estat capaç de pagar. I ella havia dit que no. Havia dit que necessitava
anar-se'n lluny de Derry, es volia anar a Sydney. O a San Diego. O a
qualsevol altre lloc que vora d'ell. Com és que s'havia portat tan
malament? Com era que no ho havia vist venir? Com era possible estimar a
algú tant, així que cada cop que entrava en una habitació, cada vegada
que parlava, somreia, reia, el seu cor feia un bot? Només saber que era
a la cuina, al lavabo o al vestíbul era prou per enviar aquells
fantàstics, increïbles i meravellosos pensaments a recórrer el seu cos?
Encara que quan ho va fer, quan li va dir que l'estimava, ella li va dir
que volia allunyar-se. No un ``jo també t'estimo, però\ldots{}''. No un
``gràcies, però ho sento''. Només un ``Oh déu meu, ho dius seriosament?
No, me'n vaig d'Irlanda l'abans possible. No vull estar relacionada amb
res aquí!''

Era com si algú l'hagués tret tot el que importava dins seu i s'hi
hagués passejat.

No ets la primera persona en estar enamorada i ser rebutjat, es va dir a
si mateix racionalment.

Però no volia ser racional. ~ Què hi havia de racional en estar
enamorat? Què hi havia de racional a l'oferir-te a tu mateix a algú
només per a ser destruït?

I allà era ell, perdut i sol. Tothom deia que no estava interessat.
Tothom havia intentat dir-li que gastava el seu temps. Però quan estàs
enamorat, esgarrapes qualsevol cosa, creus que un dia et llevaràs i et
diran ``Saps què? M'equivocava i tu tenies raó, ets una altra persona
per a mi''.

Però allò no havia succeït.

Mai succeïa.

En lloc d'allò, havia vist el llampec recórrer el cel de la nit mentre
recorria la carretera, amb les llàgrimes barrejant-se amb la pluja,
pensant en que tot el que volia fer era estar a casa en aquell moment.

Casa.

A deu minuts caminant, com a màxim.

Més llampecs.

Blau, blanc i morat\ldots{}. Morat?

Va colpejar el terra just davant seu, fent-lo caure enrere.

Recordava veure la petita capsa amb l'anell esvaint-se en una ràpida
conflagració, literalment i metafòricament dibuixant una línia sota
aquella part de la seva vida.

Es va sentir com si també estigués en flames. Tot el que podia veure era
una llum morada, rodejant-lo a ell en aquell moment, destrossant les
voreres, destrossant la carretera, la foscor i la pluja.

I llavors la veu. Al seu voltant. Al seu cap. Venint del cel i del seu
cor al mateix temps.

--És el teu moment. Callum Fitzhaugh ja no és rellevant. Ara formes part
d'una causa major.

La veu es va quedar amb ell després de que el foc morat s'hagués anat,
durant els dies i les setmanes i ell es va donar voluntàriament al seu
nou propòsit.

El pròxim matí va tocar el teclat al caixer automàtic i van sortir dues
cent lliures. Vuit caixers automàtics després aquell matí. Llavors més a
diferents ciutats. Llavors va obrir un compte. Va manipular el banc
on-line, moviments que no es podien rastrejar perquè plantava figures
als ordinadors que esborraven rastres de les seves accions.

En tres setmanes, era multimilionari. Tenia edificis per tot el món.
Posseïa companyies que ell tancava i feia emergir i, en un mes,
MorganTech havia cobrat existència gràcies a la manipuladora influència
de la veu a la seva ment que li deia què fer i com fer-ho.

Després va ajuntar el sistema informàtic que canviaria el seu destí.
D'alguna manera la veu el va guiar mentre construïa la Madam Delphi,
sentia la veu al seu cap transferir-se al hardware, d'alguna manera
creant vida artificial en una escala sense ser escoltada abans.

--Et necessito--la veu es va fer suau--. Ara i sempre. Necessito una
interfície humana, una connexió amb el món de carn i ós. Un avatar a la
realitat.

Així que en Dara Morgan va ser creat.

Recordava venir d'una rica família de banquers i accionistes de comerç.
Els seus pares van morir en un accident d'un avió privat, i el
MorganTech li havia passat a ell quan tenia només 21.

Recordava més records falsos, esdeveniments, gent, qualificacions i
festes. Cap d'elles era real, però cada vegada que s'imaginava part de
la seva història fictícia, es tornava realitat. La veu li va mostrat com
una societat que confiava en els ordinadors per cerca informació, que ja
no emprava paper i tinta per mantenir els records, podia ser tan
fàcilment manipulada en acceptar la història, les mentides, les
fabricacions que els deies mitjançant el teclat eren de veritat.

Recordava la veu dient-li com desenvolupar l'M-TEK en un parell d'anys,
així que el mercat hi confiaria. Confiaria en MorganTech. Aquell era un
llarg joc.

I recordava veure-la a un carrer de Dubai una tarda.

Era amb un parell d'homes, mirant paperassa a una taula d'una cafeteria.

Havia escoltat mentre els homes explicaven que necessitaven pensar en el
tracte que fos que estiguessin parlant i se'n van anar. I llavors va
anar a asseure's al seu costat.

Ella va aixecar la mirada, al principi intrigada, llavors sorpresa i
després espantada. Al cap de poc temps va trobar la veu.

--Cal?

--Ara no--va dir--. Soc en Dara Morgan.

Ella va riure, amb aquell riure suau i bell que li va portar l'amor que
havia sentit anys enrere.

Però la veu al seu cap va xiuxiuejar:

--No, recorda-te'n. Recorda les llàgrimes i el dolor. No et rendeixis
ara, Dara Morgan.

--Sí que t'hi assembles, Cal--va dir ella--. Què et porta a Dubai?

--La mandràgora s'empassarà el cel--va dir ell--. Deixa'm mostrar-t'ho,
Cait.

Va agafar la seva mà, i els seus ulls van brillar amb l'energia morada
de Mandràgora. Llavors va obrir les carpetes sobre les que havia estat
parlant amb els homes de negocis moments abans--. Signi aquí, si us
plau, senyor Morgan--i ho va fer, perquè la veu li ho va dir.

En una hora, MorganTech era l'amo d'una cadena d'hotels cinc estrelles
al voltant del món, i la Caitlin s'havia tornat la seva primera
conversa.

Amb un esbufec, el Doctor es va apartar d'en Dara Morgan, qui
immediatament es va desmaiar i va caure al terra.

Tot havia portat menys d'un segon a la vida real però, per al Doctor,
havia semblat per sempre.

Es va apartar d'en Dara Morgan mentre la resta del grup influenciat per
la Mandràgora s'hi apropava, amb els braços aixecats, llestos per
repartir el raig de la mort.

--No! --les ones sinuoses de Madam Delphi feien bots a dalt i avall a
les seves pantalles--. No, necessito aquest cos. Es això perquè he
esperat tants segles per a que el Doctor es presenti a si mateix.
L'últim dels Senyors dels Temps, posseït per l'energia de la Mandràgora
Hèlix, animat per mi!

Els deixebles van abaixar els braços.

I el petit Joe Carnes es va apartar del seu germà i va córrer cap al
Doctor.

--No--va cridar--. Deixeu-lo en pau.

En Lukas estava al seu costat en un segon, i llavors la Donna i el Wilf
també hi eren.

Es van posar entre ell i l'ordinador posseït per la Mandràgora.

--Sí, gràcies a tots--va dir el Doctor--. Però no és necessari--va
somriure a la Madam Delphi--. Quin munt d'opcions. Nens que ningú no es
prendria seriosament, un vell amb condició cardíaca que cauria mort en
un segon, la seva amiga Henrietta, una experta en estrelles.

Va llençar una mirada darrere de tots ells, una mirada només vista per
la Donna.

L'Henrietta Goodhart seguia vora la porta, com si intentés veure-li
sentit a tot el què passava.

El Doctor la mirava amb una barreja de tristesa i\ldots{} què era allò?
Es va preguntar la Donna. Por? Desesperació? Com si ell esperés que ella
digués o fes alguna cosa?

Però no va ser així. La Netty no estava amb ells en aquell moment.

--Les llums estan enceses, però ningú no condueix--era el tipus de coses
que la Donna imaginaria que la seva mare diria. Una frase terrible, però
una amb la que la Donna no podria estar-hi en desacord en aquell moment.
I era com si el Doctor pensés que la Netty l'hagués abandonat, d'alguna
manera.

--Donna--va xiuxiuejar el Doctor--. El teu mòbil. Ara.

El va posar a la seva mà, i mantenint un ull en la Madam Delphi, va
mirar expertament al seu directori de contactes.

--Donna?

--Sí?

--Per què no està el número del teu avi aquí?

--Per què mai no encén la maleïda cosa. Quin motiu hauria de tenir?

--Oh genial. Gràcies.

--Per què el vols trucar? Està aquí dempeus.

--El seu telèfon és a Essex. Necessito trucar-lo.

La Donna va tancar els ulls, imaginant-se els seus dits al teclat i va
xiuxiuejar-li els números. Quan va dir cada número, va pitjar els
botons. Quan va escoltar que la trucada començava, va penjar.

--Espero que ho hagis fet bé, perquè sinó\ldots{}

--Algú ha rebut una estranya trucada?

--I el món acabarà. Però ei, ha estat divertit--i li va passar el
telèfon mòbil.

--No el guanyaràs--el seu avi estava parlant amb l'ordinador--. Aquest
home és brillant, ha salvat aquest planeta, l'univers sencer,
probablement, més cops que hem tingut sopars calents. Hauràs de passar
per sobre de nosaltres per tenir-lo!

Déu beneeixi l'Avi, però la Donna dubtava seriosament que pogués aturar
a la Madam Delphi. El Doctor necessitava alguna cosa de la Netty, i la
Donna estava segura d'això. Així que necessitava aconseguir temps.

--Si vols un cos al que habitar-hi que hagi estat fent volts per la
galàxia, senyoreta--va dir--, pren el meu. Oh, puc no tenir dos cors o
un cabell que desafia a la moda, però aquest cos ha vist una mica de
l'acció de l'espai exterior--va empènyer al Doctor per darrere d'ells,
apropant-lo a la Netty.

Les pantalles de la Madam Delphi polsaven de nou.

--Noble de nom i noble de natura, oi?

--Oh, com si no hagués escoltat això abans. Una nit quan en Neal Bailey
es va posar com un pop a l'Odeon, va xiuxiuejar a la meva orella: ``Ja
que tenim un pastisset Noble, què tal si fa un tomb la dolça Donna?''.
Li vaig ficar una coça allà on fa mal i vaig tocar el dos. Per si
t'importa, coneixia les seves intencions de Shakespeare i hauria
aconseguit punts de Brownie per l'originalitat. el meu pare no estava
d'acord i li va trencar el nas la setmana següent al pub-- la Donna va
somriure dolçament a l'ordinador--. Algun cop has tingut un paio tan a
prop teu? No, per suposat que no, perquè ets tota elèctrica i cables i
això. Estàs sola, no? És això pel que fas tot això, oi? Que cerques
l'amor? Te n'hauries d'haver anat cap al Lonely Hearts Club i no pel
camí de l'astrologia.

En Wilf va agafar l'espatlla de la seva néta.

--Faràs que s'enfadi.

--De veritat, avi? Mai no se m'hagués ocorregut--li va picar l'ullet--.
Sé el que em faig.

La Madam Delphi polsava amb enuig.

--Tant de bo pogués saber per què el Doctor sempre s'envolta d'humans
estúpids. Vull dir, per a què serviu? Res més que xais de sacrifici.
Quantes han viatjat abans que tu al TARDIS, Donna Noble? I què els va
passar? Vull dir, segur que vols viatjar amb ell per sempre. Creus que
ets la primera en creure això? Per suposat que no. Però tu estàs aquí i
la resta no. Em pregunto què els va passar a tots ells?

La Donna no deixaria que allò se li fiqués a dins, més que res perquè
era una pregunta que li havia preguntat al Doctor abans i ella havia
estat més que satisfeta amb la seva resposta.

Però clarament va calar fons al seu avi.

--Reina, aquesta és una bona pregunta.

--De veritat, ara no és el millor moment--va engegar-li.

--Hi ha un cementiri amb tombes, totes en una columna amb els seus noms,
no creus, Donna? --va preguntar la Madam Delphi--. Té ja un tros de
terra guardat per a tu, oi?

--Potser--va respondre la Donna--. No m'importa gaire, per ser honesta.
Visc per l'ara i aquí. I ara mateix, aquí mateix, l'únic que m'importa
és aturar-te a tu i al teu petit exèrcit de zombis.

--Destruiu-los--va dir la Madam Delphi, amb una veu tan normal, tan
casual, que li va costar a la Donna entendre el que havia dit.

Però llavors els deixebles, tots junts com un sol, van aixecar els
braços, llestos per disparar els seus rajos d'energia.

No va passar res.

--Destruiu-los! --va cridar l'ordinador.

Seguia sense passar res.

--Destruiu-lo a ell! --va demanar la Madam Delphi, però els deixebles no
van fer res més que arrufar les celles i mirar al seu voltant com si
estiguessin sorpresos. Era com si s'acabessin de llevar d'un somni.

--Ah--va dir el Doctor--, això haig de ser jo. Bé, de fet si et soc
sincer, ha estat una encantadora jove anomenada senyoreta Oladini, no he
sabut mai el seu nom de pila, què en soc de groller! Malgrat tot, acaba
de trencar el teu alineament una mica, ha cancel·lat el teu poder sobre
els descendents de San Martino, en massa. Closed, kaput!

--Finito--va dir la Donna amb un estrany accent italià.

--I això no és tot!

La Donna va mirar a la seva esquerra. En Dara Morgan estava dempeus a un
costat, amb un portàtil a les seves mans, amb els seus dits volant per
sobre de les tecles mentre teclejava amb una sola mà.

--He enviat una cancel·lació via la xarxa als M-TEKs de tot el món. En
quant es connectin amb els ordinadors, en lloc de descarregar-se les
teves ordres, s'instal·laran un virus, que desfragmentarà la plataforma,
i esborrarà els records per complert--en Dara Morgan va pitjar la tecla
de retorn un últim cop--. I ho he protegit amb una contrasenya.

--Sóc una megaordinador megalomaniaca, connectada amb milions d'aparells
elèctric a través de tot el món, estúpid humà. De veritat creus que
m'has detingut? M'has decebut, Dara Morgan.

En Dara Morgan es va encongir d'espatlles.

--Detenir-te del tot? Ho dubto, però al menys t'he fet anar més
lentament, així que la senyal no serà activada en deu minuts.
Probablement no fins d'aquí un parell de dies ara, amb tot el temps per
al Doctor per a detenir-te--en Dara Morgan va somriure--. Ah, i em dic
Callum Fitzhaugh.

Un profund sospir eletrònic va venir de la Madam Delphi.

--Caitlin?

I l'irlandesa, l'estimada de Callum que l'havia rebutjat deu anys
enrere, va treure el seu revòlver de la seva cintura i el va aixecar.

--Caitlin, no ho facis--va cridar en Callum--. Combat l'influencia de la
Mandràgora! Recorda qui ets de veritat!

La Caitlin va arrufar les celles.

--Cal?

--Sí, soc jo!

La Caitlin es va encongir d'espatlles.

--No m'agradaves gaire llavors, tampoc no és que m'agradis molt ara.

I va disparar un tret que va anar a través del cervell d'en Callum
Fitzhaugh i va sortir per l'altre costat.

Estava mort abans de que el seu cadàver caigués al terra.

Els recent despertats deixebles van cridar i xisclar en confusió i van
començar a córrer per l'habitació.

--Aneu amb ells--va xiuxiuejar la Donna als nois Carnes--. Treieu-los
d'aquí. Lukas, emporta't al Joe a casa. No pareu de córrer fins que hi
arribeu--es va girar cap en Wilf--. Tu també.

--Ni ho pensis, Donna, reina meva. Soc massa vell per córrer i estic
aquí amb tu fins el final. Li vaig dir al teu pare que cuidaria de tu, i
Déu sap que ho faré.

Se li va ocórrer a la Donna que la Caitlin aquella hauria pogut obrir
foc feia estona, així que va mirar a veure què feia. Havia posat la
pistola a l'escriptori i ara s'asseia mirant cap a les pantalles de la
Madam Dephi.

El Doctor va caminar passant per davant de la Donna, gairebé
incidentalment portant la Netty als braços d'en Wilf, murmurant:

--Agafa-la bé, Wilf. Com si la vida depengués d'això--llavors es va
ajupir amb la Caitlin, amb la mà agafant la pistola.

--Agafa-la--va dir amb tranquil·litat--. Callum i jo ja hem et prou dany
per justificar el què vaig fer.

--Estaves sota el control de la Mandràgora--va dir el Doctor--. Jo el
vaig alliberar i ell et va alliberar a tu.

I la Caitlin el va mirar directament, amb una llàgrima caient-li per la
seva galta.

--La Madam Delphi mai em va controlar. La Mandràgora mai no em va
controlar, no feia falta.

--Llavors qui et va dir que disparessis a en Dara Morgan o com fos que
es digués? --va preguntar la Donna.

--La seva ment havia estat relliscant-se durant dies. Començava a
recordar coses\ldots{} era un esglaó dèbil. Havia d'eliminar-lo.

--Havies de fer què? Per què? Podria haver salvat la humanitat! Era allò
per tot el que ha estat això?

I la Caitlin va mirar de cop el Doctor directament als ulls.

--No ho sé--va dir, i la llàgrima seguia caient.

--En què m'he convertit? Què m'ha fet treballar per aquesta cosa? He
matat a algú. Oh déu meu\ldots{} L'he disparat sense pensar-ho.

--Una mica tard per les llàgrimes, maca--va dir la Donna--. Treballant
amb la Mandràgora, probablement hagis matat a centenars de persones.

--Ho sé--va dir la Caitlin amb tranquil·litat--. Estava fora de control,
amb fam de\ldots{} de poder. Volia control per sobre de la meva vida.

--Controlo tot--va respondre amb polsos la Madam Delphi--. Incloent la
teva vida!

--No, no ho fas, estúpida capsa de cables! Vaig escollir aquesta vida
perquè vaig pensar que la volia. Però saps què? Em vaig equivocar--els
seus dits volaven per sobre del teclat--. Estic tancant les connexions
wireless i activant les proteccions firewall.

--Això no m'aturarà.

--No. Però t'aïllarà durant una estona--la Caitlin va mirar tristament
al Doctor--. Ja he aportat el meu granet de sorra, Senyor del Temps. Ara
et toca a tu--i va empènyer enrere la seva cadira i va apartar al
Doctor. Disculpant-se, es va moure al seu voltant i va caminar cap al
cadàver d'en Callum--. Podríem haver tingut el món--va dir mentre s'hi
ajupia.

El Doctor intentava trobar el sentit a les paraules de la Caitlin.
Connexions wireless. Proteccions firewall. Coses sense sentit, la Madam
Delphi era de lluny més poderosa que allò. Va teclejar al teclat i una
ona d'energia morada de Mandràgora gairebé el va electrocutar els dits.

--Ei, ei, relaxa't una mica.

--Et destruiré, Doctor. Seràs\ldots{} -- i va callar.

Llavors va veure el que la Caitlin havia fet en realitat. Havia dit
coses sense sentit, sabent que la Madam Delphi gastaria unes poques
subrutines per rastrejar el que deia haver fet. Havent trobat els
proteccions firewall i les connexions wireless intactes, l'ordinador
buscava en algun altre lloc. La mantindria callada i ocupada
durant\ldots{} bé, no gaire, sincerament.

Però s'estava duent a terme un càlcul, podia veure-ho a una de les
pantalles, era com un mini-virus, una equació matemàtica semireplicativa
que emprava bytes amb cada segon que passava mentre que, intentant
esbrinar la solució, el que feia era multiplicar-se. El Doctor va
somriure. La Caitlin era bona en el què feia, tot i que només pogués
entretenir la Madam Delphi uns segons fins que ho esbrinés. Va mirar cap
el cadàver d'en Callum, esperant veure la Caitlin.

El cadàver estava sol.

El Doctor va tocar-se les butxaques. El revòlver era allà. Però hi havia
alguna cosa que no hi era.

--Donna--va xiuxiuejar-li--. Donna, vull que baixis al vestíbul. Tota
aquella gent estarà confosa, desorientada. La meitat d'ells no parlen bé
angles, pel que sabem. Necessiten algú normal i racional per ajudar-los,
per explicar-los les coses.

--Però no hi ha ningú aquí que encaixi amb aquesta descripció--va dir la
Donna--. Jo hauré de fer-ho.

El Doctor li va fer un somriure.

--Oh Donna, ets la millor que hi ha. Ara, ves-hi. Oh, no tu, Wilf. Tu i
la Netty us quedeu aquí.

--Per què no venen amb mi? --va preguntar la Donna grollerament.

--Problemes familiars--va dir el Doctor--. Confia en mi, estaran sans i
estalvis escales avall amb mi en un parell de minuts.

--Però\ldots{}

En Wilf va donar un pas endavant.

--Ves, Donna, no discuteixis amb aquest home. Quan t'ha decebut?

La Donna se'n va anar.

--Pot haver no estat la millor elecció de paraules, Wilfred.

--L'has decebut algun cop, Doctor?

--Bé\ldots{} --el Doctor s'ho va pensar--. No, de fet, però hi he estat
a prop un o dos cops.

--Doncs si algun cop ho fas, hauràs de respondre davant de mi.

Els seus ulls es van trobar, creuant l'habitació i, per un petit segon,
un fragment d'eternitat, el Doctor va saber que mai de la vida decebria
la Donna Noble.

--No ho faré--va dir--. De fet, Wil, hauria de dir que ``nosaltres'' no
ho farem, perquè tu ets important ara mateix. Per la Donna, per a mi i
per al món sencer. I sobre tot, per a l'Henrietta Goodhart--va
aixecar-se de cop--. No ho facis, Caitlin.

En Wilf es va adonar de que la noia irlandesa estava vora una paret,
vora una capsa de connexions, agafant un bolígraf platejat. Amb una
punta blava que brillava.

--Què vol fer?

--No saps com fer-ho servir, Caitlin--va dir el Doctor, lentament--. I
la Madam Delphi tornarà a estar activa en un segon. Et detindrà.

--Que ho intenti--va dir la Caitlin--. I tens raó, no sé com fer-ho
servir, però suposo que si pressiono això, giro això i el moc per
aquí\ldots{}

--Cat, no!

Era massa tard. Mentre el tornavís sònic grinyolava amb energia, massa
energia, mal emprada, feta servir per mans sense experiència, la Caitlin
ho va moure cap a la ara oberta capsa de connexions, i just directament
cap als cables de fibra òptica que en Johnnie Bates va morir connectant
feia tan sols uns dies abans.

Hi va haver un llampec de foc morat i la Caitlin es va esfumar, reduïda
a àtoms junt amb un tros de la paret, el cablejat i el tornavís sònic
del Doctor.

--Bona idea--va dir el Doctor, ple de dolor--, però havia d'haver una
forma millor.

--L'ordinador s'ha apagat--va dir en Wilf.

El Doctor va mirar a les pantalles, i es va ajupir per examinar el
servidor.

--Mort com un tronc--va confirmar.

--Hem guanyat?

--Oh, encara no--el Doctor va mirar a en Wilf--. He mentit a la
Donna--va confessar.

--Ho sé--va dir en Wilf--. Però t'has assegurat de que estava bé.
gràcies.

--La Caitlin ha tallat l'energia de la Mandràgora de l'ordinador. A
efectes pràctics, la Madam Delphi s'ha anat. Esborrada. Destruïda.

--Però aquesta cosa de l'energia Mandràgora, segueix aquí, oi?

--Atrapada.

--On?

--A aquesta sala. I ara mateix, està cercant per una nova casa. Suposo
que tenim tres minuts.

--No m'escollirà a mi, això és pel que tu m'has conservat aquí. He
escoltat el que deies. Tinc feblesa cardíaca. Vaig a morir.

--Què? --el Doctor va arrufar les celles, i llavors se'n recordà del que
li havia dit abans a la Madam Delphi--. Wilfred, no tinc ni idea sobre
la condició del teu cor. Podràs tenir un parell de dècades més com a
mínim, pel que sé. Només ho deia perquè\ldots{} bé, ara mateix no
importa--va mirar cap al tros de paret que faltava on la Caitlin havia
estat feia un moment--. Ara, dubto que pugui donar vida als morts, així
que espero que vagi cap a l'objectiu més fàcil, el camí de menys
resistència.

En Wilf va seguir el punt de vista del Doctor fins a la Netty, que
estava somrient serenament rere seu.

--No\ldots{}

--És el mitjà més propici.

En Wilf tremolava de tristesa.

--Però és la meva Netty. Anàvem a veure el món junts, anar de creuer,
veure Sud Amèrica, el Canadà, l'Oceà Índic. Doctor, ella és la meva
vida. Mai no vaig pensar que podria substituir la meva dona, que en pau
descansi, però la Netty Goodhart va arribar i em va mostrar que hi ha
més vida que estar assegut a una capsa de verdures escoltant en Dusty
Springfield. No la puc perdre. No puc perdre una altra dona de la meva
vida. L'estimo!

--Sé que ho fas, però ho sento molt per haver de demanar-li-ho en ella,
però ho he de fer.

--No li pots demanar ara mateix, està\ldots{} està apagada ara. És la
seva malaltia, la demència. No pot parlar per si mateixa. ~

El Doctor va mirar la seva butxaca i va treure'n un paper.

En Wilf el va mirar.

``Estimat Wilfred,

Em vas dir un cop que confiaves en el Doctor amb la vida de la Donna.
Ara confia amb ell amb la meva. No sé què farà, ni l'estat en el que
estaré quan ho faci, però si hi confies, és prou per mi.

HG''

--És el teu paper especial--va dir el Wilf--. Em mostra el que vull
veure. La Donna m'hi va parlar, temps enrere.

El Doctor va agrair-li-ho a la Donna en silenci, llavors va treure la
seva cartera de cuir amb el paper psíquic de veritat.

--No, Wilfred, la carta és de veritat. Quan estàvem al restaurant
d'hamburgueses, li vaig explicar a la Netty el que potser hauríem de
fer. Perquè era un objectiu potencial i una potencial\ldots{}

El Doctor va callar, i en Wilf va veure un momentani flaix de llum
morada als seus ulls. Llavors els va tancar i els va tornar a obrir.
Eren marrons. Com sempre.

El Doctor va treure aire de les seves galtes.

--No ha estat divertit, però no ho tornarà a intentar.

--No cal--va dir l'Henrietta Goodhart, amb tranquil·litat però amb una
amenaça familiar al seu to--. Ja tinc una nova llar. Un nou cos. Un que
es pot moure, parlar i sentir.

--Surt de la meva companya--va engegar-li en Wilf.

La Netty només va riure.

--Pobre patètic home crèdul. Aquest ara és el meu mitjà. La Mandràgora
viu. Estendré la destrucció per aquest món, obtindré la meva venjança.
El cosmos sencer caurà a la ruïna i al caos i jo m'hi alimentaré durant
segles. Serà tot un bell caos!

En Wilf va donar una passa endavant cap a la Netty, però el Doctor el va
fer anar cap enrere, i va fer un gairebé imperceptible moviment de cap.

--Espera--el Doctor va mirar cap a la paret cremada on la Caitlin havia
mort. Llavors a l'ordinador, ara inútil i apagat. Al cadàver d'en Callum
Fitzhaugh, consumit per la ràbia i la desesperació temps enrere, li
havia donat socors a una energia de l'univers que anava a destruir la
Terra, si se li donava oportunitat. I per últim al cos de l'Henrietta
Goodhart, habitat ara per un poder extraterrestre tan gran que havia
sobreviscut als Temps Foscos i ara estava llesta per estendre una onada
d'anihilació a través de les galàxies conegudes.

Això si no s'havia equivocat en els càlculs.

La Netty va caminar per l'habitació, com si no hi hagués res de dolent
en ella, una forta dona en forma d'uns seixanta, que hauria pogut estar
comprant pels Yorkshire Dales o prenent el sol a un creuer al Carib, amb
en Wilfred Mott com el seu acompanyant, al seu costat.

En lloc d'això, va tossir. I va trontollar.

En Wilf va anar a ajudar, però el Doctor el va detenir.

--Ho sento, no puc imaginar-me com de difícil és això per tu--li va dir
el Doctor-- , però has deixar que acabi l'espectacle.

La Netty, o més bé la força extraterrestre que en aquell moment habitava
el seu cos i la seva ment, els va somriure. Era un somriure que a cap
dels dos els agradava. Va girar la cara de la Netty d'una forma que
demostrava per complert que aquella no era la Netty.

--Gràcies, Doctor--va complaure's--. M'has donat l'oportunitat d'una
nova vida. Sempre he sabut que l'ordinador tindria un final, que un dia
l'hoste perfecte vindria. Havent estudiat aquest ridícul planeta,
m'hauria agradat que fos un jove i sexy cos masculí. Com una estrella de
telenovel·la o un esportista. ~Però ei, les petites dones velles
serviran per ara. Quan aquesta s'hagi cremat, em mouré cap a una altra.

--Cremat? --el Doctor va mirar cap al Doctor , però la mirada del Senyor
del Temps estava fermament fixada en la Netty. En Wilf no sabia si era
potser com a raó per a no poder aguantar la seva mirada acusadora.

--Oh, no et va dir el teu amic alienígena aquí sobre aquesta part de la
història? Em pregunto si li va dir a l'Henrietta Goodhart quan va fer el
tracte amb ella. Ja veus, Wilf, et puc dir Wilf, oi? Només la Netty
t'aprecia, i crec que fa les coses més fàcils si ens podem comunicar
casualment. Senyor Mott és horriblement formal--el somriure en rictus es
va ampliar--. Bé, un cos humà només pot aguantar l'energia Mandràgora
per un breu temps abans que s'evapori i hauré de trobar un nou
receptacle. Per últim, serà el Doctor.

--Així serà? Oh, vaja.

--Saps que així serà.

--Però primer necessites afluixar les meves defenses, per suposat.
Desmoralitzar-me. Abatre el meu esperit. Com ho vols fer?

La Netty es va riure, era un so estranyament buit de calidesa o alegria.

--Destruint a tothom que et coneix.

Llavors de cop va mirar desequilibradament cap als seus peus, i es va
avançar buscant l'ajut d'en Wilf. Va estar a punt d'apropar-se quan el
Doctor va creuar l'habitació i va apartar el braç d'en Wil.

--Ei!

--Oh, no comencis--va murmurar el Doctro--. Deixa-la caminar amb els
seus propis peus.

La Netty estava dempeus de nou.

--Començant per aquest home vell amb un cor dèbil.

En Wilf va estar a punt de parlar, i llavors va entendre perquè el
Doctor s'havia inventat allò sobre el seu cor. Volia que agafés el cos
de la Netty, no el d'en Wilf. Però per què?

--Fa molt de temps, Doctor, vam veure aquest món com una amenaça, tenia
molt potencial. Vam intentar aturar-lo en el seu desenvolupament,
endarrerir les seves ciències primàries. Però mira-ho! Ens vam
equivocar, hauríem d'haver-les ajudat. Té l'habilitat de comunicar-se.
Amb un petit virus d'ordinador, la Mandràgora pot tocar el món. Hi ha
gairebé set milions de persones a la Terra, Doctor. En un parell d'anys,
dos milions de cases tindran un ordinador personal, fet servir per
mitjana per tres persones. Afegeix-li la infiltració al lloc de treball
i li comportarà a la Mandràgora menys d'una hora de dominar de manera
efectiva la majoria de la població d'aquest planeta, d'emprar la
tecnologia humana per estendre la Mandràgora per la galàxia molt més
eficientment que pugui fer-ho jo sola. En vint anys, podré tenir granges
d'humans a Mart. En cent anys, haurem colonitzat Alfa Centauri. Un nou
Imperi de la Mandràgora, combinant l'energia Hèlix, la qualitat física
~dels humans i les seves ciències comunicatives. I llavors\ldots{}
llavors\ldots{}

--És impressionant, ho haig de reconèixer. I llavors què?

--Què fan les espècies? Floreixen, dominen i\ldots{} tiren del carro.

--Oh, ``tiren del carro'' és molt científic--el Doctor va asseure's a
una cadira davant de l'ara inútil ordinador Madam Delphi--. I? Parla'm
més dels teus plan. En Wilfred està desesperat de saber-ho. I jo també.
I la Donna que és just fora de la sala. Hola, Donna, si us plau, entra.
Estic segur de que ella també vol saber-ho.

La Donna va entrar a la sala.

--He cregut que podria ser de més ajuda aquí a dalt. A tots aquells els
he ficat dins el restaurant i els he dit que tornaria en un minut.

--Eh\ldots{} Donna--va aventurar-se el Doctor--. I això els ha detingut?

La Donna va treure una petita clau de plata i se la va ensenyar davant
dels nassos.

--Perquè sóc brillant i els hi he tancat a dins--va fer un somriure--.
Oh, però he enviat als nois Carnes a casa.

--Bé, bé. Mandràgora ens estava dient sobre com ell/ella/això va a crear
un nou imperi sencer. Com l'Imperi bizantí, o, eh, com era l'altre? El
grec? No\ldots{} Oh, quin era?

La Donna va obrir la seva boca per dir ``Romà'', però el Doctor la va
aturar.

--No, no, Donna deixa que la Mandràgora ho esbrini. Vinga, tens tota la
informació a la punta de la llengua, quin és l'imperi en el que estic
pensant? Alguna cosa tenien a veure un Declivi i una Caiguda, oi?

La Netty/Mandràgora es va aturar. I llavors va dir:

--Romà. L'Imperi romà.

--Oh, sí. molt bé. Quants milions d'ordinadors hi ha a la Terra en
aquest moment? Quin era el mercat d'infiltració del M-TEK, per cert?

La Netty/Mandràgora va arrufar les celles, i es va girar cap a l'avi de
la Donna.

--Wilf? Ajuda'm\ldots{}

--Netty\ldots{}?

--No, Wilfred--va dir el Doctor en veu alta, amb la seva veu de sobte
com un tret--. Asseu-te, deixa que la Netty ho esbrini. Endavant!

I en Wilf es va asseure vora la Donna, al terra.

--D'on és la Mandràgora?

La Netty/Mandràgora va somriure.

--D'una nebulosa, Doctor. Vam escapar dels Temps Foscos, i vam crear una
nova llar al cor del bell caos.

--Per suposat que ho vau fer. Com es deia?

--Era el\ldots{} era\ldots{} no puc recordar-ho--la Netty/Mandràgora va
trontollar lleugerament--. No ens en podem recordar.

--Vinga va, centra't--va cridar el Doctor de cop, i es va aixecar,
rodejant la Netty/Mandràgora, fent-li preguntes mentre caminava:

--Em pots dir la velocitat de la llum? Quant va durar la guerra entre
els Carrionites i els Eterns? Quin és el planeta d'origen dels Judoon?
Quina és la Vint-i-tresena Convenció de la Proclamació de les Ombres?
Qui va guanyar la guerra entre els Bendromes i els Sendromes? Quantes
mongetes fan cinc? Vinga, vinga--va fer petar els dits impacientment--.
Vull dir, no crec que puguis prendre l'univers si no pots pensar per tu
mateixa, oi?

--Dóna'm un moment--li va engegar la Netty/Mandràgora.

--Donar-te un moment? Bé, et podria donar un moment. Suposo, vull dir,
li donaria a l'Henrietta Goodhart un moment o dos alegrement, per què
ella no està bé, oi? Oh, no te n'has adonat? No has esbrinat aquesta
part? --i el Doctor va agafar a la Netty per les espatlles i la va girar
per mirar-la a la cara, amb els nassos gairebé tocant-se--. Has pres el
cos d'una dona increïble, Mandràgora, i t'hi has tancat a tu mateixa a
la seva ment, estenent-te per les seves sinapsis i tot. El problema és,
que les sinapsis li estan fallant. Tots i cada un dels dies les neurones
i les sinapsis del seu còrtex cerebral s'atrofien. I tu estàs accelerant
el procés en la teva pressa per aclimatar-te i, tristament, ja estàs
trencada. Vull dir, no pots ni pensar en les paraules, això és un toc de
la parafàsia. No pots aguantar-te dempeus? Bé, crec que això és
l'apràxia. El lobus temporal i el lobus parietal, tots fent-te mal.

--Tu\ldots{} tu has fet quelcom\ldots{} enganyat\ldots{} m'has
enganyat\ldots{}

--Bé, sí. Això penso. I quant més ho combatis, més energia Mandràgora
gastaràs intentant arreglar aquests trossos del cervell, i més t'hi
estàs perdent, perquè aquesta gran dama té Alzheimer moderat i això no
es pot curar, ni tan sols per tu.

--Llavors hauré de, ja saps, canviar\ldots{} moure'm, canviar el cos
amb\ldots{} hauré de\ldots{}

--Sí? Què? Què faràs? Vinga va, digues-m'ho.

La Donna s'hi va unir.

--O ens podries parlar sobre les estrelles, totes aquestes meravelloses
constel·lacions, totes les coses que la Netty sap. Parla'ns d'on és
l'estrella Sirius? O com trobar el Carro Gran. En quina direcció veuré
Venus en aquesta època de l'any?

En Wilf va agafar el braç de la Donna.

--Atura't, Donna, l'estàs confonent.

--Aquesta és la idea--li va dir--. Això és pel que ho fa el Doctor, fent
la confusió més de pressa.

--Però és la Netty\ldots{} li esteu fent mal a la Netty!

En Wilf no es podia moure. Sabia que la Donna i el Doctor sabien el què
es feien, però no podia parar de voler aturar-los. De fer qualsevol
altra cosa que veure a la Netty passant per allò. Però no podia. Perquè,
com va llegir un cop a una columna de consells del diari, ``algunes
vegades el millor bé pot venir del dolor més petit''. De fet,
probablement deia ``la vida és plena de petits cops'', però era com
havia escollit ell interpretar-ho.

I ho odiava.

Por un petitíssim segon, gairebé va odiar al Doctor i a la Donna per
fer-ho\ldots{} tan fàcil.

I llavors va prendre una decisió.

--Quan les estrelles comencin a caure--va començar a cantar lentament--.
Oh, Senyor! Quin matí! Oh, senyor! Quin matí\ldots{}. --la seva veu es
va trencar lleugerament, així que es va aclarir la gola i va començar de
nou--. Quan les estrelles comencin a caure, oh Senyor! Quin matí! Oh,
Senyor! Quin matí. Quan les estrelles comencin a caure\ldots{}

Gentilment i en silenci, la veu de la Netty va respondre:

--Oh, pecador, què faràs quan les estrelles comencin a caure\ldots{}?
Oh, Senyor! Quin matí\ldots{}

En Wilf li va estendre una mà, intentant amagar les seves llàgrimes al
fer-ho.

Aquest cop el Doctor no el va apartar. En Wilf va començar lentament a
fer passes cap a la Netty, mentre els dos cantaven en silenci junts la
cançó que tant estimava ella.

El Doctor va tocar-li l'esquena a la Donna.

--Ara és cosa del teu avi.

--Te'n recordes on vam escoltar per primer cop aquesta cançó? --li va
dir en Wilf a la Netty mentre ballaven--. Qui la va cantar? Quin era el
nom de l'home que ens va portar el dinar? Pots recordar el cotxe, tot
platejat i brillant? I era aquella vegada el primer cop que vas anar
dins un Rolls Royce? I què vaig dir al final, quan conduíem per aquells
carrers, mirant cap al clar i bell cel? I pots\ldots{}

--Jo\ldots{} no puc\ldots{} no puc recordar\ldots{} Sóc Mandr\ldots{}
Mandràgora\ldots{} Jo dominaré l'univers\ldots{} d'alguna manera\ldots{}
No puc\ldots{}

--Tu ets l'Henrietta Goodhart--va dir en Wilf amablement--. I estàs
malalta, molt i molt malalta, i tinc tanta por de perdre't que no vull
fer-ho. Si us plau, queda't amb mi.

--Wilfred?

El Doctor i la Donna es va aixecar d'immediat quan la Netty va dir el
seu nom.

--Tu ets en Wilfred\ldots{} i joc sóc la Netty\ldots{} No, sóc la
Mandràgora Hèlix, i jo\ldots{} Oh Senyor ! Quin matí\ldots{}

En Wilf la va agafar més fermament, i la va abraçar més del que havia
abraçat mai a ningú des que la seva estimada Eileen va morir--. Oh,
pecador, què faràs--va cantar-hi de tornada.

--El meu cap\ldots{} no entenc\ldots{} no puc recordar res--la Netty es
va apartar--. Per què no puc recordar res? No és just\ldots{} No és
just! No puc recordar res\ldots{} no és just!!

El cap de la Netty va caure cap enrere i mirava cap el sostre, portant
ambdues mans cap a la direcció en la que mirava. Els altres van veure
com una forta llum morada, blava i vermella sorgia del seu cos,
vaporitzant-se a l'espai al sostre mentre l'energia de la Mandràgora
sortia de la forma humana de la Netty i s'esvaïa en l'espai. ~

I llavors el soroll i la llum van acabar.

La Mandràgora se n'havia anat.

Les mans de la Netty van caure netament als costats i el seu cap va
rodar cap endavant. En Wilf va anar a agafar-la, però la Netty va tenir
un calfred i el va mirar, amb un gran somriure de reconeixement a la
seva cara.

--Wilfred? --va mirar al voltant de l'àtic, va veure un forat perfecte
al sostre, llavors va veure al Doctor i a la Donna.

--Verge santíssima--va dir--. M'he anat sota el sol un altre cop? On
dimonis he anat aquest cop?

--Tu, Henrietta Goodhart--va somriure el Doctor--, acabes de salvar el
planeta Terra. Ets brillant.

La Donna el va seguir.

--Sí. I tu no ets tan inútil, home de l'espai.

Cent milles. Mil milles. Un milió de milles. Movent-se gairebé a la
velocitat del pensament, la trencada i sense cos energia Mandràgora
Hèlix viatjava a través de l'univers, cridant per dintre, amb la seva
ment trencant-se, intentant trobar el camí de tornada a casa.

Però hi havia una casa? Segur que hi havia\ldots{} No, no hi
havia\ldots{}

Hi havia una casa? On era aquí?

Qui soc? Què soc? Per què soc? Qui\ldots{} què\ldots{} on\ldots{}
com\ldots{}

Penso, per tant existeixo\ldots{}

Penso, per tant\ldots{}

Penso\ldots{}

Jo\ldots{}

Què és ``jo''?

Res\ldots{}

``Jo'' és res\ldots{}

Jo\ldots{}

Jo\ldots{}

O\ldots{}

\ldots{}

..

.

A la cafeteria, la Donna havia pres el comandament, i en poc temps va
fer que tothom es relaxés i ho va arreglar, pel descans del Doctor. Ella
era molt millor en aquells temes sentimentals dels humans que ell. I el
més important en aquell moment, podia mentir molt més convincentment,
dir-los que havia estat tot part d'un virus d'ordinador enviat per
MorganTech i que podrien tornar a casa en quant es fes de dia.

El Doctor va fer un parell de trucades ràpides a gent que coneixia als
alts comandaments (o potser als baixos) i va anunciar que algú arribaria
ben d'hora per donar a tothom bitllets de vol i reserves de primera
classe on volguessin anar.

--Això és Anglaterra--va dir el vell americà--. Sempre he volgut venir a
Anglaterra. Com dimonis he arribat aquí?

El Doctor no podia respondre a allò però en lloc d'això els va dir a ell
i a la seva encisadora esposa que els havia arreglat una estada a un
hotel (no a aquell, per sort) al centre de la ciutat, i que tenien un
passi de set dies per explorar la ciutat.

--Agafeu el tren, visiteu Bath, o Warwick, o l'Illa de Wight.

--O Hull--va afegir la Donna.

--Donna, perquè voldrien anar a Hull? Què hi ha a Hull que possiblement
vulguin veure?

--No ho sé--va dir--. Mai he estat a Hull però sempre he pensat que sona
interessant.

--Hull és impressionant--va afegir en Wilf--. Hi vaig anar a passar un
pont, per veure un partit. Vaig fer un tomb en vaixell.

El Doctor va afegir:

--D'acord--els va dir als americans--. Aneu a Hull també. Té vaixells.
Pel que sembla.

La parella se'n va anar parlant de Hull, i el Doctor va girar --se cap
als tres estudiants. Els tres nois i una noia.

--Què li ha passat al professor? --va preguntar la noia.

Els dos nois al fons (oh, per tant una parella, va decidir la Donna) van
fer que si amb el cap, però l'altre home semblava trist.

--Ha mort, oi?

El Doctor va fer que sí amb el cap.

--L'home italià? Sí, ho sento.

Els estudiants es van mirar entre ells.

--No veig cap motiu per tornar a Itàlia.

--Jo sí--va dir l'home menut al final, mirant als altres.

Ah, va pensar la Donna.

--Al menys hauríem d'acabar les coses per allà--va dir el tercer noi.

I el grup es va allunyar, parlant entre ells.

El grec estava pesarós, dient que no tenia ni idea del què havia fet,
però suposava que no havia estat bé. El Doctor li va dir que no era
culpa seva i que hauria de tornar amb la seva família i oblidar-se'n de
Londres. L'home es va allunyar, parlant per si mateix.

--Suposa que ha fet alguna cosa a Grècia. Suposa que algun d'ells ho ha
fet. O tots ells--va dir la Donna.

--No puc saber-ho tot, Donna--va fer un sospir--. Només podem esperar
que sigui el que sigui el que ha passat als seus passats, si és que ha
passat, puguin arreglar-ho. I tampoc no crec que ho recordin.

--Penses en el teu amic del Copernicus, oi?

--Un d'ells el va matar. Li va trencar el coll. El nostre amic grec
sembla el més adequat, però no soc un policia. I no puc provar res.

La Donna anava a comentar sobre la seva moralitat quan el seu telèfon
mòbil va sonar. Era un missatge.

--La senyoreta Oladini--la Donna li va passar el mòbil. ~Va llegir el
missatge.

--Està bé? --va preguntar el Doctor.

--Està emocionadíssima. Què li has fet?

--No sé què vols dir.

--Doctor?

--Bé, possiblement mentre arreglava les coses amb UNIT per aquesta gent,
puc possiblement haver mencionat com ha estat necessària ella també.

--Aquí diu--va somriure la Donna--, que el seu passi de visitant ha
estat ascendit i que ara pot anar i venir tant com li plagui. Res
d'amagar-se. Oh, i també diu que et digui que té una gata anomenada
Dolly i que sabries què significaria.

El Doctor va somriure.

--Bé per totes dues.

--Creia que no t'agradaven gaire els gats?

--Sempre m'ha agradat la Dolly. I es mereix una bona llar.

--Doctor? ~Quanta gent va fer servir la Madam Delphi i se'n va desfer?

--La humanitat era només una eina per al Hèlix, eines per a ser emprades
i abandonades.

--Com la Netty?

El Doctor va fer una ganyota de dolor.

--Ho sento--va dir la Donna--. M'he passat.

El Doctor va mirar la seva amiga.

--Però certa i honesta. Haig de prendre riscos, Donna. Temps enrere, ho
hauria fet amb menys consciència.

--Déu meu--va dir la Donna amb fals terror--. Què t'he fet?

El Doctor estava seriós. Li va agafar les mans.

--M'has fet millor persona.

La Donna li va apartar les mans, amb rebot, com sempre, amb les seves
bromes.

--Ei, no toquis el que no pots pagar, home de l'espai.

Van veure com en Wilf i la Netty començaven a caminar cap a la recepció.

--Tornem amb la mare, d'acord?

La Donna va fer que sí amb el cap.

--També vens tu, oi? Vull, ja saps com és.

El Doctor va fer que sí.

--Sí, una versió més gran de la seva filla.

--Ei! --la Donna va riure i va creuar el braç amb el Doctor--. Vinga,
home de l'espai. Has vençut als Sontarans, als Pyroviles i als Pescadors
de Kandalinga. No crec que la meva mare sigui tant esgarrifosa.

--No ho creus?

--No. Malgrat no sigui dilluns. Els dilluns és quan es torna boja. Avui
és dilluns?

--Avui és dilluns, sí.

La Donna el va agafar més fermament.

--Em toca protegir-te, oi?
