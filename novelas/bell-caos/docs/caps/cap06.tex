\chapter*{DIVENDRES}
\addcontentsline{toc}{chapter}{DIVENDRES}

Uns dies després, la humanitat havia, com sempre feia, assimilat i
avançat. MorganTech estava oficialment en fallida i desapareguda, en
absència del seu equip directiu.

Els M-TEKS havien estat reclamats i destruïts i s'havia emès un arrest
per en Dara Morgan, fins que es va descobrir la veritat sobre la seva
identitat (o al menys el fet de que ell no era en Dara Morgan), i en tal
moment tot l'assumpte de MorganTech queia sota la jurisdicció de UNIT i
s'esvaïa del públic. La gent vigilant els cràters s'havien despertat,
completament estranyats per la raó per la que eren allà. Havien estat
arrestats, però serien alliberats sense dubte quan s'hi fiqués UNIT.

La família Noble anava cap a la Reial Societat Planetària, doncs en Wilf
tindria per fi el seu sopar i el seu Nomenament d'Honor. La Netty era
amb ells, ajudant a la Sylvia a preparar-se, emprovant-se barrets amb
unes plomes alarmantment llargues i rient-se de coses sense sentit amb
la mare de la Donna.

En Wilf i el Doctor s'havien sensatament escapat cap al jardí del
darrere, bevent te i discutint en tons silenciosos sobre les diferents
aventures del Doctor amb ``la gent robot de l'espai exterior'', història
la qual acabava normalment amb un riure general.

La Donna va abocar-se a les portes del pati per fer-los callar.

--Faràs que la mare es pregunti de què parleu, i llavors haurà acabat el
joc.

--No li diràs la veritat, oi? --el Doctor va aixecar una cella a ambdós.

Una mirada ràpida es va creuar entre avi i néta, seguida d'una frase a
l'uníson:

--Estàs boig?

--Et mataria definitivament aquest cop--va dir la Donna.

--Després de matar-me a mi per guardar els secrets--va coincidir en
Wilf.

El Doctor es va encongir d'espatlles i va canviar de tema.

--Així que, la festeta d'aquesta nit. A quina hora us anireu?

--Ens hi anirem a les set--va dir la Donna.

El Doctor va obrir la boca per protestar, per dir que l'última cosa que
volia era un altre sopar amb la Reial Acadèmia Planetària, una altra
oportunitat de ser ridiculitzar pel doctor Crossland o establir una
eterna i avorrida conversació amb l'Ariadne Holt sobre dibuixos amb les
mans o la seva absència d'elegància per vestir.

--Brillant--va dir sense gaire entusiasme--. Necessitaré un viatge curt
al TARDIS per a, eh, canviar-me el vestit.

La Donna va fer que no amb el cap.

--Et quedaràs aquí, Speedy Gonzàlez.

--Aquí?

--Aquí.

--I el TARDIS? I el vestit?

--Ni el TARDIS, ni el vestit, ni cap trucada d'emergència de la princesa
~Leia dient-te que ets la seva última esperança--la Donna va treure
tasses de te--. Més te?

El Doctor va fer que sí amb el cap.

La Sylvia va sortir, amb el telèfon a l'orella.

--De veritat? --deia--. Bé ha d'haver estat robada en aquell escàndol de
les llums i tot\ldots{} Oh, d'acord. Bé això és una mica arriscat, això
d'esperar que algú et robi la furgoneta per l'assegurança. Oh, bé,
d'acord. Fins desprès. Adéu, maco--va penjar el telèfon--. La furgoneta
blava del senyor Webb ha estat robada, resulta que volia que se la
robessin, va deixar-hi les claus i tot, per tal de poder reclamar diners
a l'asseguradora. Aparentment ha aparegut, cremada, en algun lloc de
l'East End. No ho sé, la gent\ldots{} --tenia alguna cosa més a les
seves mans i la hi va deixar davant d'en Wilf.

Era una pila de pamflets de residències, trencades per la meitat.

--Crec que la Netty hauria de venir a viure aquí. Amb nosaltres--va
tocar-li la galta al Wil--. Amb tu.

En Wilf es va aixecar i va abraçar la seva filla.

--No--va dir la Netty des de darrere de tots, estant magnífica amb el
seu últim barret--. La meva ment ara està més clara per primer cop en
molt de temps. Però no em puc venir a viure aquí, Sylvia.

--Per què no? -- va preguntar en Wilf.

--Oh, estimat i bon home--la Netty li va fer l'ullet--. Em fas molt
feliç, però no sóc una ximpleta. Estaria molt bé quedar-m'hi aquí mentre
se'm va el cap. Però i si\ldots{} quan me'n vagi de nou, vosaltres dos
no esteu preparats per ajudar-me? La pressió, l'estrès, no és just per
vosaltres. Per a cap dels dos.

Va agafar els pamflets trencats.

--No és per vosaltres, que ho sabeu. Però, tot i així, m'agradaria fer
una volta a veure si podem trobar un que ens agradi a tots.

La Sylvia va agafar-li el braç a la Netty.

--És una decisió molt gran--va dir--. Estàs segura? Perquè no ho deia
només per ser bona. Crec que hauries d'estar aquí, ets part de la
família.

La Netty va mirar al Doctor.

--Què en penses, Doctor?

El Doctor va mirar a la Sylvia, llavors a la Donna i llavors al Wilf. I
finalment de nou a la Netty.

--Crec, Henrietta Goodhart, que ets una dona sàvia, sensata i forta que
coneix les seves pròpies limitacions millor que nosaltres i que farà el
que sigui correcte--va agafar les tasses de te de la mà de la Donna--. I
jo no sóc de la família, i com que m'agradaria evitar aquesta
conversació amb un toc de gràcia, me n'aniré i posaré a escalfar la
tetera.

Va caminar ràpidament cap a la casa, va rentar les tasses i va omplir la
tetera, mirant per la finestra de la cuina cap al grup del jardí i va
fer un somriure silenciós.

--Ei, maco--va dir una veu silenciosa a la porta.

--És la vida del teu avi i la teva mare, Donna--va dir ell--. No té res
a veure amb mi. Les famílies no son gaire cosa meva.

La Donna se li va unir a la pica, mirant per la finestra.

--Sembla tan\ldots{} control·lada ara mateix. Així tan\ldots{}

--Normal?

--Bé, no hauria fet servir aquesta paraula exactament, però sí.

--No durarà.

La Donna no el va mirar.

--Per què no? Potser haver emmagatzemat tota l'energia Mandràgora li ha
netejar les coses neuronals, i ho ha arreglat tot.

~--És al menys un Alzhèimer de segon grau, Donna. Això és
deteriorament--va respondre amb tranquil·litat--. No millora, i en gran
part anirà a pitjor. No hi ha cap cura miraculosa, em temo, ni cap
màgica solució per a la Netty. La seva ment está una mica com el
parabrises d'un cotxe després d'una rentada. En alguns aspectes, la
Mandràgora Hèlix va estar el rentador de cotxes, netejant-ho tot una
mica durant un temps. Però no trigarà abans que la terra, els insectes i
la pols tornin. Ho sento.

--Està tan malament.

--Sí, ho està. Però la vida mai és tan convenient com ens agradaria. Hi
ha milions de xacres, malalties i trastorns. Si hagués cregut que una
cosa tan maligna com la Mandràgora podria haver esborrat només una
d'aquestes, l'hauria deixada. L'hauria permès quedar-s'hi, fer el bé.
Però no hi ha mai cures miraculoses per coses com aquesta. La vida no és
així. Però no hauria d'aturar a la gent de seguir cercant perquè algun
dia, trobaran una resposta.

--I això significarà que t'has equivocat.

El Doctor va riure.

--Sí. A vegades passa. I a vegades m'agrada. M'agradaria poder trobar
una forma d'ajudar, però no puc.

--Què li passarà a l'avi?

--És un home crescut. Prendrà la decisió racional i adulta de cuidar-la
el temps que pugui. Això fa que en Wilfred Mott sigui un Home Molt Bo al
meu diari.

--I al meu també.

--Potser ens podríem quedar una mica, ajudar a la Netty a instal·lar-se
en algun lloc. Podries passar una mica de temps de qualitat amb la teva
mare?

La Donna va fer que no amb el cap.

--Ara estem bé. La setmana que ve estarem barallant-nos, discutint,
cridant i esbufegant.

A través de la finestra, fora al jardí, van veure a la Sylvia i a la
Netty mirant els pamflets.

--On és aquest te, llavors? --va dir en Wilf darrere del Doctor i de la
Donna.

--Avi--va dir la Donna de cop--. Potser hauria de posar-vos a tu i a la
mare primer. Potser m'hi podria quedar, ajudar una mica amb la Netty--va
mirar al Doctor--. Déu sap, que et trobaré a faltar i tot, allò\ldots{}
--va assenyalar cap al cel--. Però potser és hora de créixer una mica.

En Wilf va abraçar a la Donna.

--Reina, què et fa feliç?

Sense pensar-ho, la Donna va mirar al Doctor.

--I creus que jo seria feliç sabent que sóc responsable de deixar tot
això? I creus que la Netty ho seria?

--Però la mare i tu em necessiteu\ldots{}

--Potser, però ens hem arreglat un temps per ara. M'agrada saber que tu
ets allà fora amb el Doctor, fent als altres planetes i a la gent el què
vau fer per la Terra l'altre dia.

--I la Netty?

En Wilf va somriure amb tristesa.

--Està malalta, i al final s'anirà. I jo també. I la mare. I cap de
nosaltres haurà vist i fet totes les coses meravelloses i increïbles que
tu has fet. Tot són records que tens. La malaltia de la Netty podria
prendre-la en cinc anys o el proper dijous. També podria passejar per
davant del número 18 de Kew Gardens. No deixaré que la seva malaltia o
la nostra tristesa per no tenir-te aquí, et deturin de viure la vida que
has escollit allà fora. Amb ell.

El Doctor va posar un braç al voltant de la Donna.

--Jo la cuidaré.

--Oh i tant que ho faràs, amic, o hi haurà problemes, recordes? --en
Wilf va treure la tetera quan va començar a bullir--. Escolta, et
prometo que jo, la teva mare i la Netty, estarem aquí asseguts quan
tornis. No deixaré anar a la Netty enlloc, algú de mantenir-me pel bon
camí--en Wilf va començar a fer el te--. Ets una dona impressionant,
Donna Noble--li va dir--. I estic orgullós de conèixer-te i
estimar-te--li va fer un petó a la galta--. Ara, truca un taxi, no vull
arriscar-me a que la teva mare condueixi després d'haver pres uns quants
glopets de licor aquesta tarda.

Li va passar una tassa de te a ambdós i va aixecar la seva per fer un
brindis.

--Per la família. I pels llaços que mai no es poden trencar.

\begin{center}\rule{3in}{0.4pt}\end{center}

UN DIA\ldots{}\ldots{}. (REPRESA)

Plovia allà al turó, amb el continu tamborineig colpejant el gran
ombrel·la de golf com bales contra una llauna. Per a ser sincers, plovia
arreu, però dalt el turó, allà a l'hort, era l'únic lloc on en Wilfred
Mott es preocupava de veritat si plovia o no.

Va mirar cap a les estrelles, cap a la seva estrella, seguia allà, sense
contemplar la destrucció de la Terra, de la humanitat o de ningú.

Ni de tan sols de la Netty.

--Com està ella? --va dir, escoltant les passes caminant fatigosament a
través de l'hortet mullat rere seu.

La Sylvia es va asseure rere seu, agafant-li el termos, mirant com
estava de calent el te de l'interior.

--Hauria d'haver-te portat un altre--va dir--. No hem anat a veure-la.
La Donna és fora amb la Susie Mair i jo sola no puc anar-hi.

En Wilf va mirar a la seva filla. Hi havia alguna cosa\ldots{}

Li va estirar la mà amb un sobre a dins.

--És una mica tard pel correu, reina--va dir.

La Sylvia no va dir res, només li va donar el sobre.

En Wilf el va agafar. No tenia segell, donat a mà i adreçat a MARE.

--En Lukas Carnes me l'ha donada aquest matí. M'ha dit que li va dir que
ho donés sis setmanes després de veure per últim cop a la Donna, sense
importar perquè.

--La Donna l'ha vist?

La Sylvia va fer que no amb el cap.

--Estava escales amunt amb l'ordinador. Ni tan sols va fer sonar el
timbre. En Lukas viu ara a Reading. Crec que el Doctor li deu haver
arreglat. En Lukas creu que ella segueix\ldots{} --la Sylvia va
assenyalar a les estrelles--. Creu que està allà fora, amb ell. No hi ha
cap motiu per xafar-li els somnis, oi?

En Wilf li va fer una abraçada a la seva filla.

--Ho superarem, reina.

--Hem de fer-ho, oi? Per la Donna, vull dir.

--I pel Doctor.

--Qui ens cuida ara, pare? --va dir la Sylvia de cop--. Vull dir, mai
m'ha agradat aquest home, però fins i tot jo sé quan m'equivoco. Ha
salvat el món, ha fet feliç a la Donna. Ens ha mantingut vius un cop
més. Però si ella no és amb ell, la seva connexió a aquest planeta, què
li importa tornar aquí, a preocupar-se de nosaltres?

--Perquè ell és bo així--va dir en Wilf--. Perquè ell és el Doctor i
quan el necessitem, hi serà. És el que fa.

--Per què passa si no és així? ~Vull dir, jo em sentia segura abans. No
sabia res sobre els Sontaran, la Mandràgora o els Daleks. No saber-ho,
ens manté sans i estalvis. Però ara tots sabem que l'univers és molt més
gran que nosaltres. Que tu o que jo, o fins i tot que la Donna.

En Wilf va tornar a mirar cap a les estrelles, només per si de cas el
meravellós TARDIS passava volant.

Res.

--Bé, sempre hi haurà algú.

Va obrir la carta.

La Sylvia es va aixecar.

--Aniré a buscar-te un altre termos, d'acord? Ara torno--i la Sylvia es
va ajupir a fer-li un petó a la galta, però en lloc d'allò li va fer una
forta, i sincerament li va fer mal i tot, abraçada--. T'estimo, pare--li
va dir amb calma.

I se'n va anar.

La Donna no era l'única a la que el Doctor havia canviat. En Wilf va
pensar amb una mica de tristesa i, al mateix temps, amb alegria. La
Sylvia Noble era una persona més estable, tot sigui dits, aquells dies.
Així que què hi havia a aquella carta que li havia fet estar
tant\ldots{} sensible aquella nit?

Va ficar la mà dins la seva bossa i va treure una llanterna halògena que
feia servir per llegir els seus llibres d'astrologia quan passava les
nits a fora.

Va reconèixer la lletra, per suposat. La de la Donna.

La seva estimada, llesta, bonica i valenta Donna.

``Estimada mare,

Em vas preguntar què faig. Què fem el Doctor i jo. I et vaig mentir. Ho
sento. T'he dit que era algú que arreglava coses, que ens passejàvem pel
país ell i jo arreglant coses. I que jo era la seva assistent personal.
No és cert. Bé, per suposat que no ho és i estic segura de que tampoc no
em creuràs, ets la meva vella mare, ets més eixerida que això. Recordes
el que deia la iaia Mott? No pots guardar els secrets, perquè no
existeixen. Algú sempre ho sap, sinó qui et va dir el secret en primer
lloc? I té molta raó.

Bé, un parell d'anys enrere, jo rondava. De treball en treball, de lloc
en lloc, gràcies als cels que vaig agafar aquell treball a l HC
Clements. Gràcies al cel que m'hi vas ficar (tot i que no era
precisament el treball que tu volies que jo fes), tampoc no és que t'ho
digués llavors, oh no. Això t'haurà fet créixer massa.

Però m'alegro de que ho fessis, mare. Perquè és així com vaig conèixer a
l'home més fantàstic de tots (i no, no parlo del pobre Lance. Algun dia,
t'ho prometo, t'explicaré la seva vertadera història).

Vaig conèixer al Doctor. Ell és un extraterrestre, mare. Però crec que
tu ja ho sabies. No estic segura de perquè no t'agrada gaire, però
sovint em pregunto si és perquè se'm va emportar, i crec que hi ha una
part de tu que no pot acceptar que és qui m'hagi canviat de veritat. Em
va fer feliç. Em va fer una persona millor.

Ho sento, no volia dir-ho així. No t'estic donant la culpa. Tu m'has
donat la millor vida. Ho vas fer de veritat. Però ell m'ha ensenyat que
hi ha molt més.

Em vas preguntar quant em pensava quedar amb ell. Per sempre. El qual,
amb el seu treball, podria significar qualsevol cosa. Però no tornaré a
casa en poc temps. Et prometo que us visitaré més sovint i us enviaré
més cartes. ~També intentaré trucar-vos més sovint. No creuries què li
ha fet al meu mòbil, ha fet que la resta semblin llaunes d'escopinyes.

No, no som una ``parella'', no hi ha res de romàntic en ell. És el meu
amic. És el meu millor amic. Espero que t'estigui explicant això
adequadament. No puc dir-t'ho a la cara, així que per això t'ho escric.

Anava a donar-te un discurs però llavors vaig pensar que t'agradaven les
cartes, i de fet n'he escrit una. És la primera vegada que escric una
que no acaba amb un ``Cordialment'' des que vaig escriure-li aquella
carta de Nadal a la tieta Maureen? Quants en tenia? Catorze? I ja saps
com va acabar, no és que hagi escrit gaire des llavors!

Ell em cuida, mare. Has de confiar amb ell. Jo ho faig. Espero que si jo
hi confio, tu també ho faràs. L'avi ho fa. Ell ho sap, i si us plau no
li cridis, vaig se jor qui li vaig fer-me prometre que no et diria què
fem. Per que et preocuparies.

Oh, mare, hauries de veure el que jo veig. Hem estat a llocs, a móns,
als futurs i als passats amb els que només podries somniar. Crec que la
meitat els he somniat, perquè no poden ser reals. Però ho són. I a cada
lloc al que anem, marquem una diferència. Posem les coses bé, fem a la
gent més feliç. Això és el que fa el Doctor. Troba una forma per a que
l'univers tingui sentit. I l'estimo per això. Perquè és desinteressat, i
crec això ho a tret de mi, però no gaire perquè hauria d'haver pensat en
com t'haurà d'estar ferint això. Hauria d'haver sabut que només venir a
casa per l'aniversari del pare no és prou.

Em necessites, però ell em necessita encara més. T'estimo, mare, i no
ser capaç d'estar allà amb tu per ajudar-te està malament, però
necessito que entenguis la raó per la que no hi sóc allà massa sovint.

Vaig a seguir viatjant amb el Doctor a altres planetes, altres món i
conèixer alienígenes i coses, alguns de bons i altres de dolents, perquè
finalment tinc una vida. Tots aquests anys, he esperat per algú com ell
i mai m'havia adonat. Però ara sé que estic fent el correcte. Em sento
viva.

I ell em cuidarà tant com jo el cuido amb ell. Confia en mi quant dic
que estic sana i estàlvia i sempre ho estaré. I mai em passa res (i serà
millor que no perquè tornaré i perseguiré el seu primet cos per sempre)
sé que no et deixarà sense saber-ho. T'ho dirà sense importar com de
difícil deu ser per ell. Perquè entén com és estar sol i com és de
dolent i no crec que el meu petit home de l'espai ho desitgi a ningú.

T'estimo, mare, i quan t'arribi això (suposant que en Lukas fa el que li
he demanat) farà temps que m'hagi tornat a anar. Però és el divertit
d'estar amb el Doctor. Puc tornar abans que ho sàpigues. Sis setmanes
deuen haver passat per mi i sis minuts per a tu.

Cuida de l'avi. I de l'encantadora Netty, ella li fa molt de bé, i crec
que ja ho saps. No intenta ser una substituita de la iaia Eileen, és una
alternativa. I li dóna algunca cosa a fer més que asseure's a un humit
hortet tota la nit.

T'estimo molt i ens veurem aviat.

Molts petons,

Donna''

Després de llegir-la dos cops més, en Wilf va signar la signatura i va
tornar a ficar la carta dins el sobre amb molta cura.

Va pensar en el Doctor, en el què havia dit la Donna sobre la soledat. I
va recordar aquella trista (molt i molt trista) mirada a la seva cara
sota la pluja aquella nit.

L'havia portada a casa. S'havia enfrontat a allò, igual que la Donna
sabia que ho faria.

La Sylvia tenia raó, tot i així. Sense la Donna per portar-lo de volta,
quina garantia hi havia de que salvaria la Terra la pròxima vegada?

Era massa fàcil dir ``Oh, bé, algú ho farà''. Potser algú altre hauria
d'aixecar-se, i estar llest per fer-ho.

Així que en Wilf es va aixecar i va mirar a les estrelles, sentint la
pluja contra la seva cara.

Va fer un salut al cel fosc.

--No sé si estàs allà fora, Doctor, vigilant-nos. Però sé que ho estàs.
Perquè sé que això és el que fas per tothom, per cada món, a totes
parts. Però crec que també necessitem aprendre per nosaltres mateixos a
defensar-nos. No sempre haver de comptar amb tu.

Es va treure la pluja dels ulls, al menys va decidir dir que era pluja.
Si algú preguntava.

I en Wilfred Mott va mirar cap a l'hortet, i després rere seu cap a
l'oest de Londres per sota, encès a la nit.

No semblava que hi hagués alienígenes envaint-los, no hi havia
súper-ordinadors que intentessin destruir les vides.

I ell només va pensar en l'amistat.

--Torna aviat, Doctor--va murmurar--. No només quan et necessitem.
Passa't a prendre't un te algun dia.
