\chapter*{INTRODUCCIÓ}
\addcontentsline{toc}{chapter}{INTRODUCCIÓ}

La memòria és una cosa divertida\ldots{} resumeix tot el que som, tot el
que vam ser i tot el que possiblement serem, perquè dóna forma a tot el
que fem. He estat obsessionat durant molt de temps amb la memòria i la
seva pèrdua i l'efecte que provoca a la nostra identitat. Ha estat un
tòpic recorrent a les meves novel·les de Doctor Who~d'ençà. Però Bell
Caos~era el primer llibre on volia treure-ho de la ciència ficció i de
la fantasia i fer-ho real. I també volia agosarar-me amb l'Alzheimer i
l'efecte que pot tenir a les persones terriblement afectades i sobre
aquells que els estimen. I la família Noble em va donar la oportunitat
perfecta.

Però fem una passa enrere, anem al món de Doctor Who~el 2008 quan vaig
escriure això.

En David Tennant era el desè Doctor. Havia conegut al David feia molts
anys, molt abans que es convertís en el Doctor, ell era un actor a qui
vaig admirar des que el vaig veure a un episodi especial de The Bill, el
drama policíac de la Thames TV, on va interpretar a un psicòpata que
havia cordat a una noia a la part del darrere d'una furgoneta on
s'asfixiaria tard o d'hora. Va mantenir la seva postura mentre era
interrogat, es va fotre's dels policies i els va portar per carrerons
sense sortida tot manipulant-los mentre ell continuava semblant innocent
i guai. Fins que va ser descobert i la noia salvada. I llavors a una
meravellosa actuació de televisió, va esclatar la escena final.
Metafòrica i literalment doncs la bogeria del seu personatge va
esclatar, deixant palplantats als policies i als espectadors amb tota la
ferocitat, credibilitat i vulnerabilitat que va poder reunir. Vaig saber
que era un actor a qui s'havia de mantenir vigilat. L'he buscat un
parell de cops i també l'he dirigit uns altres (per a mi n'ha fet de
nazis molests, escocesos emotius i soldats sàdics). Llavors se'n va anar
a fer Casanova~a la televisió amb en Russell T Davies i la Julie
Gardner, que van ser qui el van fer el Doctor. Perquè ells tenen molt
bon ull.

Així que jo sabia que seria un Doctor genial i pels volts del 2008, vaig
tenir la oportunitat de veure el seu creixement, no només a la televisió
sinó també al plató i a les sessions de lectura perquè jo treballava
llavors a la sèrie. I volia capturar tota aquella energia, aquella
lleugera imprevisibilitat maníaca però per sobre de tot, aquella
intel·ligència increïble, per a fer un breu resum, per a aquest llibre.
Volia un llibre on es reflexés com veia jo al desè Doctor.

La Donna Noble, oh, com vaig aplaudir quan vaig assabentar-me que seria
la Catherine Tate qui faria d'ella. Va ser un càsting molt inspirador i,
com tots els bons còmics, seixanta cops amb més talent que qualsevol que
la menyspreés sense veure-la. De veritat, no pots aconseguir tot el que
fa ella als seus programes d'sketchos sense ser una actriu brillant, i
no en parlem de ser la millor comediant. ~Me'n recordo de veure-la a una
lectura del «Planeta dels Ood»~ (jo vaig fer d'Ood i ella m'ho va agrair
i jo em vaig enamorar), i per la forma amb la que es podia convertir en
la Donna en un sol segon, al següent fer una ganyota de sarcasme i al
segon de desprès deixar anar una tempesta de emocions que et regira de
cap a peus. No vaig ser l'únic amb els ulls mullats quan va llegir
aquesta escena on escolta la cançó dels Ood a les cel·les. De nou, sabia
que havia de capturar allò a la impremta, la agressiva xerraire que
ocultava un personatge emotiu i sensible que sempre lluitava per
continuar endavant.

Tant el Doctor com la Donna van ser creació de l'escriptura i
creativitat d'en Russell i sempre van ser totalment reals i normals per
a mi. Apartant la falsa excentricitat que caracteritzava al Doctor a la
sèrie clàssica i substituint-la amb un cor i una ànima va funcionar molt
bé per a aquest humil fan.

Però llavors tenim en Wilfred Mott. En Wilf. L'avi. (Conegut
anteriorment com en Sid, el venedor de diaris). Que amb la Sylvia, la
mare de la Donna interpretada meravellosa i tàcitament per la Jacqueline
King, en Bernard Cribbins va entendre el clixé de l'avi lleugerament
xapat a l'antiga i el va fer brillar, el va fer tan completament
meravellós que volies que fos el teu propi avi. ¿No és el millor que pot
resultar de fer un bon guió i una bona actuació? Que desitgis que algú
fos el teu avi, perquè te l'estimes molt i de forma molt ràpida.

En mig de tot això, faig aparèixer el meu propi personatge, la Nettie,
la ``noia'' d'en Wilf que dóna el que rep, completament conscient del
seu paper a la dinàmica de la família (la Donna l'adora i la Sylvia la
veu com una possible amenaça), però mai passant les fronteres. (En
Russell la va esmentar fins i tot a l'episodi de «La fi del temps: part
u»~i no sabeu com va significar per a mi). Tot hauria de ser perfecte
per a ella i per al Wilf, sinó fos per l'Alzheimer què, ella i en Wilf
ho saben, l'acabarà destruint a ella. Ningú no sap quan, però ambdós
saben que acabarà passant. I això és tràgic, en mig de les invasions
alienígenes, els ordenadors grillats i un par de nois de l'espai i la
deliciosament valenta senyoreta Oladini, la Nettie i en Wilf son com
Romeo i Julieta, Calvin i Hobbes i l'Eric i la Hattie tot en un.

Al menys, això és el que he intentat. No tinc forma de saber si també ha
funcionat per al lector, només espero que així sigui. Tot el que he dit
és que aquestes van ser les meves intencions: escriure una història
sobre la pèrdua, la tristor, el triomf i viure l'única vida que tenim,
tot i (i a vegades per causa de) el que el fats ens tenen preparat.

Tot el que sé és que vaig fer plorar en Russell T Davies (això diu ell,
encara que crec que només ho diu per ser amable), i vaig rebre el més
gran dels acompliments de la meva vida de la mà d'en Bernard Cribbins,
que va llegir l'àudio resumen del CD.

--Tu has escrit això? Vaig creure que com que posava Russell al nom de
l'autor era seu sota un pseudònim perquè era molt emotiu i punyent.

No vaig saber què dir (sospito que només intentava ser amable) perquè,
tot i que Bell caos~és, sota els meus ulls al menys, el millor que he
escrit mai, sóc ben conscient que n'hi ha de centenars, de novel·les de
Doctor Who~millors allà fora escrites per autors més ben dotats que jo.

Només volia escriure una història sobre gent de veritat enfrontant-se a
situacions reals amb un fons d'una historia d'aventures i fantasia. Vaig
a prendre a voler això gràcies al treball amb en Russell tots aquells
anys.

Mai li podré agrair suficient per donar-me l'oportunitat de fer-ho. Però
continuaré intentant-ho\ldots{}

Gary Russell

Agost de 2012