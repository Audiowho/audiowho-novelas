\chapter*{UN DIA\ldots (REPRESA)}
\addcontentsline{toc}{chapter}{UN DIA\ldots (REPRESA)}

Plovia allà al turó, amb el continu tamborineig colpejant el gran
ombrel·la de golf com bales contra una llauna. Per a ser sincers, plovia
arreu, però dalt el turó, allà a l'hort, era l'únic lloc on en Wilfred
Mott es preocupava de veritat si plovia o no.

Va mirar cap a les estrelles, cap a la seva estrella, seguia allà, sense
contemplar la destrucció de la Terra, de la humanitat o de ningú.

Ni de tan sols de la Netty.

--Com està ella? --va dir, escoltant les passes caminant fatigosament a
través de l'hortet mullat rere seu.

La Sylvia es va asseure rere seu, agafant-li el termos, mirant com
estava de calent el te de l'interior.

--Hauria d'haver-te portat un altre--va dir--. No hem anat a veure-la.
La Donna és fora amb la Susie Mair i jo sola no puc anar-hi.

En Wilf va mirar a la seva filla. Hi havia alguna cosa\ldots{}

Li va estirar la mà amb un sobre a dins.

--És una mica tard pel correu, reina--va dir.

La Sylvia no va dir res, només li va donar el sobre.

En Wilf el va agafar. No tenia segell, donat a mà i adreçat a MARE.

--En Lukas Carnes me l'ha donada aquest matí. M'ha dit que li va dir que
ho donés sis setmanes després de veure per últim cop a la Donna, sense
importar perquè.

--La Donna l'ha vist?

La Sylvia va fer que no amb el cap.

--Estava escales amunt amb l'ordinador. Ni tan sols va fer sonar el
timbre. En Lukas viu ara a Reading. Crec que el Doctor li deu haver
arreglat. En Lukas creu que ella segueix\ldots{} --la Sylvia va
assenyalar a les estrelles--. Creu que està allà fora, amb ell. No hi ha
cap motiu per xafar-li els somnis, oi?

En Wilf li va fer una abraçada a la seva filla.

--Ho superarem, reina.

--Hem de fer-ho, oi? Per la Donna, vull dir.

--I pel Doctor.

--Qui ens cuida ara, pare? --va dir la Sylvia de cop--. Vull dir, mai
m'ha agradat aquest home, però fins i tot jo sé quan m'equivoco. Ha
salvat el món, ha fet feliç a la Donna. Ens ha mantingut vius un cop
més. Però si ella no és amb ell, la seva connexió a aquest planeta, què
li importa tornar aquí, a preocupar-se de nosaltres?

--Perquè ell és bo així--va dir en Wilf--. Perquè ell és el Doctor i
quan el necessitem, hi serà. És el que fa.

--Per què passa si no és així? ~Vull dir, jo em sentia segura abans. No
sabia res sobre els Sontaran, la Mandràgora o els Daleks. No saber-ho,
ens manté sans i estalvis. Però ara tots sabem que l'univers és molt més
gran que nosaltres. Que tu o que jo, o fins i tot que la Donna.

En Wilf va tornar a mirar cap a les estrelles, només per si de cas el
meravellós TARDIS passava volant.

Res.

--Bé, sempre hi haurà algú.

Va obrir la carta.

La Sylvia es va aixecar.

--Aniré a buscar-te un altre termos, d'acord? Ara torno--i la Sylvia es
va ajupir a fer-li un petó a la galta, però en lloc d'allò li va fer una
forta, i sincerament li va fer mal i tot, abraçada--. T'estimo, pare--li
va dir amb calma.

I se'n va anar.

La Donna no era l'única a la que el Doctor havia canviat. En Wilf va
pensar amb una mica de tristesa i, al mateix temps, amb alegria. La
Sylvia Noble era una persona més estable, tot sigui dits, aquells dies.
Així que què hi havia a aquella carta que li havia fet estar
tant\ldots{} sensible aquella nit?

Va ficar la mà dins la seva bossa i va treure una llanterna halògena que
feia servir per llegir els seus llibres d'astrologia quan passava les
nits a fora.

Va reconèixer la lletra, per suposat. La de la Donna.

La seva estimada, llesta, bonica i valenta Donna.

``Estimada mare,

Em vas preguntar què faig. Què fem el Doctor i jo. I et vaig mentir. Ho
sento. T'he dit que era algú que arreglava coses, que ens passejàvem pel
país ell i jo arreglant coses. I que jo era la seva assistent personal.
No és cert. Bé, per suposat que no ho és i estic segura de que tampoc no
em creuràs, ets la meva vella mare, ets més eixerida que això. Recordes
el que deia la iaia Mott? No pots guardar els secrets, perquè no
existeixen. Algú sempre ho sap, sinó qui et va dir el secret en primer
lloc? I té molta raó.

Bé, un parell d'anys enrere, jo rondava. De treball en treball, de lloc
en lloc, gràcies als cels que vaig agafar aquell treball a l HC
Clements. Gràcies al cel que m'hi vas ficar (tot i que no era
precisament el treball que tu volies que jo fes), tampoc no és que t'ho
digués llavors, oh no. Això t'haurà fet créixer massa.

Però m'alegro de que ho fessis, mare. Perquè és així com vaig conèixer a
l'home més fantàstic de tots (i no, no parlo del pobre Lance. Algun dia,
t'ho prometo, t'explicaré la seva vertadera història).

Vaig conèixer al Doctor. Ell és un extraterrestre, mare. Però crec que
tu ja ho sabies. No estic segura de perquè no t'agrada gaire, però
sovint em pregunto si és perquè se'm va emportar, i crec que hi ha una
part de tu que no pot acceptar que és qui m'hagi canviat de veritat. Em
va fer feliç. Em va fer una persona millor.

Ho sento, no volia dir-ho així. No t'estic donant la culpa. Tu m'has
donat la millor vida. Ho vas fer de veritat. Però ell m'ha ensenyat que
hi ha molt més.

Em vas preguntar quant em pensava quedar amb ell. Per sempre. El qual,
amb el seu treball, podria significar qualsevol cosa. Però no tornaré a
casa en poc temps. Et prometo que us visitaré més sovint i us enviaré
més cartes. ~També intentaré trucar-vos més sovint. No creuries què li
ha fet al meu mòbil, ha fet que la resta semblin llaunes d'escopinyes.

No, no som una ``parella'', no hi ha res de romàntic en ell. És el meu
amic. És el meu millor amic. Espero que t'estigui explicant això
adequadament. No puc dir-t'ho a la cara, així que per això t'ho escric.

Anava a donar-te un discurs però llavors vaig pensar que t'agradaven les
cartes, i de fet n'he escrit una. És la primera vegada que escric una
que no acaba amb un ``Cordialment'' des que vaig escriure-li aquella
carta de Nadal a la tieta Maureen? Quants en tenia? Catorze? I ja saps
com va acabar, no és que hagi escrit gaire des llavors!

Ell em cuida, mare. Has de confiar amb ell. Jo ho faig. Espero que si jo
hi confio, tu també ho faràs. L'avi ho fa. Ell ho sap, i si us plau no
li cridis, vaig se jor qui li vaig fer-me prometre que no et diria què
fem. Per que et preocuparies.

Oh, mare, hauries de veure el que jo veig. Hem estat a llocs, a móns,
als futurs i als passats amb els que només podries somniar. Crec que la
meitat els he somniat, perquè no poden ser reals. Però ho són. I a cada
lloc al que anem, marquem una diferència. Posem les coses bé, fem a la
gent més feliç. Això és el que fa el Doctor. Troba una forma per a que
l'univers tingui sentit. I l'estimo per això. Perquè és desinteressat, i
crec això ho a tret de mi, però no gaire perquè hauria d'haver pensat en
com t'haurà d'estar ferint això. Hauria d'haver sabut que només venir a
casa per l'aniversari del pare no és prou.

Em necessites, però ell em necessita encara més. T'estimo, mare, i no
ser capaç d'estar allà amb tu per ajudar-te està malament, però
necessito que entenguis la raó per la que no hi sóc allà massa sovint.

Vaig a seguir viatjant amb el Doctor a altres planetes, altres món i
conèixer alienígenes i coses, alguns de bons i altres de dolents, perquè
finalment tinc una vida. Tots aquests anys, he esperat per algú com ell
i mai m'havia adonat. Però ara sé que estic fent el correcte. Em sento
viva.

I ell em cuidarà tant com jo el cuido amb ell. Confia en mi quant dic
que estic sana i estàlvia i sempre ho estaré. I mai em passa res (i serà
millor que no perquè tornaré i perseguiré el seu primet cos per sempre)
sé que no et deixarà sense saber-ho. T'ho dirà sense importar com de
difícil deu ser per ell. Perquè entén com és estar sol i com és de
dolent i no crec que el meu petit home de l'espai ho desitgi a ningú.

T'estimo, mare, i quan t'arribi això (suposant que en Lukas fa el que li
he demanat) farà temps que m'hagi tornat a anar. Però és el divertit
d'estar amb el Doctor. Puc tornar abans que ho sàpigues. Sis setmanes
deuen haver passat per mi i sis minuts per a tu.

Cuida de l'avi. I de l'encantadora Netty, ella li fa molt de bé, i crec
que ja ho saps. No intenta ser una substituita de la iaia Eileen, és una
alternativa. I li dóna algunca cosa a fer més que asseure's a un humit
hortet tota la nit.

T'estimo molt i ens veurem aviat.

Molts petons,

Donna''

Després de llegir-la dos cops més, en Wilf va signar la signatura i va
tornar a ficar la carta dins el sobre amb molta cura.

Va pensar en el Doctor, en el què havia dit la Donna sobre la soledat. I
va recordar aquella trista (molt i molt trista) mirada a la seva cara
sota la pluja aquella nit.

L'havia portada a casa. S'havia enfrontat a allò, igual que la Donna
sabia que ho faria.

La Sylvia tenia raó, tot i així. Sense la Donna per portar-lo de volta,
quina garantia hi havia de que salvaria la Terra la pròxima vegada?

Era massa fàcil dir ``Oh, bé, algú ho farà''. Potser algú altre hauria
d'aixecar-se, i estar llest per fer-ho.

Així que en Wilf es va aixecar i va mirar a les estrelles, sentint la
pluja contra la seva cara.

Va fer un salut al cel fosc.

--No sé si estàs allà fora, Doctor, vigilant-nos. Però sé que ho estàs.
Perquè sé que això és el que fas per tothom, per cada món, a totes
parts. Però crec que també necessitem aprendre per nosaltres mateixos a
defensar-nos. No sempre haver de comptar amb tu.

Es va treure la pluja dels ulls, al menys va decidir dir que era pluja.
Si algú preguntava.

I en Wilfred Mott va mirar cap a l'hortet, i després rere seu cap a
l'oest de Londres per sota, encès a la nit.

No semblava que hi hagués alienígenes envaint-los, no hi havia
súper-ordinadors que intentessin destruir les vides.

I ell només va pensar en l'amistat.

--Torna aviat, Doctor--va murmurar--. No només quan et necessitem.
Passa't a prendre't un te algun dia.