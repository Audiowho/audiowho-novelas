\chapter*{DIVENDRES}
\addcontentsline{toc}{chapter}{DIVENDRES}

En Terry Lockworth va mirar el seu telèfon mòbil, però ni hi havia cap
senyal. La Maria anava a cansar-se d'ell, havia de treballar fins tard
però no li podia dir. No dubtava que el seu bol d'espaguetis acabaria a
la paperera aquella nit. Un altre cop. La pobra Maria, no era culpa seva
que es cansés d'ell, però era el que se suposava que hauria de fer.
Portaven casats tres mesos i tenien un nen de camí (si us plau que fos
nena), i els diners anaven justos.

Clar, el seu pare els havia donat un dipòsit pel pis al Boston Manor,
però encara hi havia l'hipoteca, les factures, les classes prenatals, el
menjar\ldots{}

Terry va sacsejar el cap mentre es desava el telèfon a la butxaca. Deixa
de queixar-te, es va dir a si mateix, i posa't a treballar, que llavor
seràs a casa en una hora amb sort. I el que era més important, estaria
fora d'aquest punt mort de cobertura telefònica i així podria al menys
telefonar-la.

Va agafar la seva capsa d'eines i va treure les alicates, va tallar la
cobertura de plàstic del cablejat de coure i va tallar els cables.
Llavors va estirar els cables vells de la caixa de connexió i en va
treure una llarga bobina de fibra òptica de la capsa d'eines. Aquelles
era fibra òptica interessant (bé, d'acord, només en Terry ho trobava
interessant) perquè era més fina que del normal. Un nou sistema,
desenvolupat pels americans (com sempre), i aquell edifici era el primer
del Regne Unit en fer-los servir. Havien enviat a en Terry a un curs de
Nova York sis mesos per a aprendre el nou sistema. Allò havia estat
divertit, moltes nits a la ciutat amb en Johnnie Bates, descobrint que
era de veritat allò que deien de que era la ciutat que mai dormia.
Freqüentment havien fet les classes del curs al dia següent, amb la
ressaca però feliços.

En Terry era massa sensat per saber quan anar-se'n de festa i quan
aplicar-s'hi i tenir el treball fet, tot i així, i ell i en Johnnie
havien tornat a Anglaterra, certificats per treballar instal·lant
aquelles noves fibres òptiques, fet que els havia fet populars a ambdós
amb el seu cap i els havia aportat una pica d'extra.

Els havien promès un altre extra quan fessin el treball i, francament,
era els diners al banc el que feia el món girar. El cablejat era
senzill, el que portava molt de temps era retirar les antigues coses de
coure.

En Johnnie era a un parell de pisos per sobre d'ell, més proper a la
suite d'exhibició. S'ho havien jugat a cara o creu per veure qui dels
dos s'aniria amb els peixos grans i tindria l'oportunitat d'agafar una
tassa de te de les secretaries i els assistents i qui hauria de
treballar ~a les escales i als passadissos de servei. En Terry havia
perdut, per suposat. No hi hauria te per a ell.

Va treure un tornavís del seu cinturó d'eines i va començar a obrir
l'última capsa de connexió, xiulant alguna cosa que havia escoltat a la
ràdio mentre venia conduint. Qualsevol cosa per passar el temps.

Si hagués mirat enrere cap al treball que acabava de fer, es podria
haver sorprès a l'adonar-se de que els cables de fibra òptica que havia
connectat a les prèvies capses de connexió van començar a brillar de
forma estranya.

Els cables que no eren connectats a cap font d'energia no acostumaven a
brillar de forma estranya. Mai, per a ser sincers. Senzillament no
succeïa. Perquè ho feien? Com ho podien fer?

Però estava tenint lloc: petits batecs morats d'energia, titil·lant
breument cap a dalt i a baix pel cablejat. Gairebé com sang bombejant a
través de les venes d'una gran criatura electrònica.

En Terry no se'n va adonar perquè estava mirant cap endavant, mirant a
veure què venia a continuació, no on hi havia estat.

I això va ser desafortunat. No només per en Terry Lockworth, els
espaguetis del qual acabarien sense ser menjats de qualsevol manera
aquella nit, sinó que també per a tota la raça humana.

En Terry va cordar l'últim tros del cablejat de fibra òptica a l'última
capsa de connexió i la va tancar per fi, somrient per a si mateix.
Escales a dalt, en Johnnie hauria d'estar rebent la prova de que el
cablejat estava acabat, i les pantalles li dirien que tot anava bé.

Mentre en Terry acabava d'assegurar l'últim cargol, un gran llampec
d'energia alienígena morat li va travessar pel tornavís, la mà i per tot
el cos sencer. Va anar tan ràpid que, quan els minúsculs flocs
carbonitzats que quedaven d'en Terry van caure al terra, el tornavís
només començava a separar-se del cap del cargol.

Per suposat, en Terry va tenir sort. Al morir tan de cop, tan
violentament i tan eficientment, es va evitar el que venia els propers
dies.

Però probablement ell no ho hagués vist de la mateixa manera.

Escales a dalt, a la suite de l'àtic, en Johnnie Bates estava connectant
tots els ordinadors amb el servidor central del Hotel Oracle, fanal
brillant de la brillantor arquitectònica que era el Western Business
District Development, més conegut com la Milla Daurada, just al costat
esquerre de l'autopista M4 a les afores de Londres.

Però per a l'home que portava l'hotel, en Johnnie no era més que un home
petit entre un munt fent alguna cosa amb cables.

En Dara Morgan, segons les biografies que ell havia mantingut amb cura a
les pàgines web de la companyia, va aconseguir el seu primer milió a
Derry quan només en tenia 26, al crear un popular web de descàrrega de
música torrent que havia permès descàrregues barates a velocitats sis
cops més tradicionals i amb quatre cops més qualitat que un MP3 normal.

La indústria musical l'estimava. Els clients l'estimaven. El govern
l'estimava. La seva mare l'estimava (bé, al menys això suposava, ja que
no parlaven gaire perquè era el que passava quan la seva mare era a una
urna platejada al replà de la xemeneia al costat del seu pare).

I el món dels negocis l'estimava. Quatre anys després, i MorganTech era
la fundació rere el nou WBDD, portant el treball i el desenvolupament a
Hounslow, Osterley i totes aquelles zones de Londres entre Brentford i
l'aeroport de Heathrow que mai havia sentit a parlar a l'hora de comprar
les terres.

Amb una cartera de valors d'uns 65 milions de lliures, era a un pas de
ser una megaestrella, ja dinant i sopant amb els Trumps, els Gateses,
els Rothschilds i els Gettys i una mitja dotzena de accionistes i rics
amb noms impronunciables de tot el món. De fet, els noms eren
impronunciables, però en Dara Morgan tampoc es molestava en
aprendre-se'ls. No li importaven gaire.

El que li importava a ell en aquell moment era tenir les suites del seu
hotel llestes per a la demostració del seu nou ordenador portàtil. I
l'home petit a les parets grises no anava bastant ràpid.

--Cait? --va esclatar els dits i una pèl-roja ben vestida amb unes
ulleres metàl·liques fines i uns talons bojament alts es va apropar.

--Digui, senyor Morgan.

En Dara Morgan va senyalar cap a l'home de las parets grises.

--Quan trigarà? --va preguntar, amb el seu lleuger accent del nord
d'Irlanda estranyament irritat.

La Caitlin va fer que sí amb el cap entenent i es va passejar fins
l'home per a preguntar-li.

En Dara Morgan va somriure lleugerament per a ell, observant la Caitlin
moure's. L'apreciava en molts nivells, per la seva bellesa era als
primers llocs de la llista.

Tothom a l'organització era de Derry o dels districtes de les rodalies
d'Irlanda del nord. I el que era més important era tothom amb qui havia
crescut. Tot escoltant històries dels pares i els germans grans sobre
conflictes, assassinats i honor. Les marxes, les tropes, les
recriminacions i els càstigs.

Allò era història per en Dara Morgan, quelcom gairebé d'una altra era.
La seva generació no havia tingut temps per preocupar-se per les
Lluites, igual que tampoc se'n preocupaven de la manca de la suposada
manca de patates o d'en Oliver Cromwell. Allò era història antiga. En
Dara Morgan i MorganTech eren el futur. De moltes maneres.

Es va recórrer la mà pel seu cabell que li arribava fins les espatlles i
llavors va treure el seu telèfon mòbil, aturant-se per olorar el sabó
dels seus dits.

Era molt important estar net. Estar ben vestit i fer bona olor.

A l'escola, li van diagnosticar un trastorn obsessivocompulsiu, com els
impulsos obsessius de rentar-se les mans cada cop que tocava algú fossin
dolents! La gent transportava gèrmens i, mentre no pensés durant un
moment que anava a ser infectat amb la malària per donar-se les mans amb
algú altre, no era inacceptable intentar mantenir-se net tan sovint.

A l'escola mai el van entendre, recordava vagament. Era massa petit,
probablement massa centrat en els currículums, els horaris i els
esports.

No podia esperar per a sortir, i això va fer en el moment en que va
acabar els seus exàmens. Ni batxillerat, ni universitats o cicles
superiors per a ell. Va anar directe al negoci, directe cap a allò, al
futur del món, directe a crear un sistema MP3 per als troglodites que
creien que Germà Gran i Factor X eren el principi i el fi de la cultura
televisiva. Els necessitava, per suposat, perquè l'ajudarien a arribar
al seu potencial, havia pujat els primers graons de l'escala de l'èxit.
Un pas endavant per a controlar el món, per mitjà dels negocis. No tenia
cap desig de controlar, de fet, el món, era massa ple de gent sufocant
lluitant pel petroli i el territori i Déu per a establir un pla sensat
per a controlar-ho. Però tot i així, els podia controlar a la
tecnologia, despatxar els reconeguts gegants i obrir-se camí a les cases
i els llocs de treball de tothom al planeta.

Allò era suficient.

I a la demostració de premsa de demà, aquell pla donaria el seu primer
pas.

La Caitlin va tornar i va dir que l'home esperava una trucada d'un altre
home a l'àrea de manteniment a la planta no sé quantes i estaria llest.

En Dara Morgan el va mirar: l'homenet intentava en general trucar a
algú.

~--Digues-li al teu amic--en Dara Morgan va dir a la Caitlin--, que no
aconseguirà trobar al seu col·lega. Les àrees de manteniment estan
bloquejades de senyals inalàmbriques. Digues-li que faci servir un
telèfon fix. Si les fibres òptiques estan connectades, el connectarà
directament amb el mòbil del seu company.

La Caitlin va fer que si amb el cap i li va passar el missatge.

En Dara Morgan va observar mentre l'homenet inseria la connexió de fibra
òptica al costat de darrere del seu portàtil i el trucava per allà.

Va haver-hi un flash de llum morada i, on l'home havia estat agenollat,
hi havia una pila de cendres. Una olor agra i a cremar va omplir l'aire,
i en Dara Morgan va arrufar el nas amb disgust. Carn humana cremada,
tela fosa i suor. Nauseabund.

--Bé--va dir la Caitlin--, això pinta bé, senyor.

En Dara Morgan va aplaudir les mans amb intensitat, i tothom a
l'habitació, tots aquells que havien ignorat la mort d'en Johnnie Bates,
es van girar per mirar-lo.

--Gent, sembla que l'hotel ja està cablejat. O hauria de dir ``fibrat''?

Hi va haver una educada ona de riures.

--Demà, dominarem el món.

--Ei!

Una paraula/frase/soroll gutural, balbucejat amb un toc d'indignació, un
caire de sarcasme i un gran glop de volum.

No importava quants cops ho intentés, el Doctor no podia fer res més que
sospirar cada cop que ho escoltava. Normalment perquè la indignació, el
sarcasme i especialment el volum anaven tots dirigits cap a la seva
direcció.

Va sospirar i es va girar per a mirar a la cara a la Donna Noble, la
reina dels ``Ei''s.

I ella no hi era allà.

Només el TARDIS, aparcat entre dos contenidors de l'ajuntament. I amb
molta cura, es va dir a ell mateix.

Oh.

Ah.

És clar.

--Ho sento--va dir-li a la porta del TARDIS, llavors va caminar cap
enrere i la va obrir, mostrant a la Donna dempeus al llindar--. He
cregut que eres fora.

--Quin tros de ``sóc just al teu darrere'' no has entès? --va preguntar
la Donna tot educadament, amb un característic moviment de cap que
donava a entendre que no era massa educada--. Quin tros de ``espera'm''
t'ha esquivat l'oïda? Quin tros de ``m'estic posant una cosa bonica''
s'ha esvaït a l'èter?

No hi havia forma que el Doctor pogués sortir d'aquella. Ell només es va
encongir d'espatlles i va dir:

--He dit que ho sentia.

--Que ho sents?

~--Sí, ho sento. Què més vols?

--Ho sents per no haver-me escoltat? Ho sents per haver-me tancat dins
la teva nau espacial alienígena? O que ho sents per no haver ni tan sols
notat que no hi era amb tu?

Cada cop que la Donna cridava el verb ``sentir'' sonava com si fos la
paraula menys penedida de la llengua catalana i cobrés un nou significat
sobre el qual els lingüistes podrien estar discutint una exacta
implicació durant els propers dotze segles.

--No hi ha forma que pugui guanyar això--va dir el Doctor--, per tant ho
deixaré anar, d'acord?

La Donna va obrir la boca per tornar a parlar, però el Doctor es va
avançar i li va posar un dit sobre els llavis.

--Calla--va dir.

La Donna va callar.

I li va picar l'ullet.

--Jo guanyo!

I llavors li va dedicar l'increïble i fantàstic somriure que ella sempre
li donava quan se li burlava, i ell li va donar el somriure que admetia
que l'havia tornat a enxampar.

Era un joc. Un joc entre dos amics que havien estat tan junts que el
jugaven instintivament amb l'altre.

Familiaritat, amistat i diversió. Les sigles FAD que resumien el temps
compartit per aquells dos aventurers.

Ella va posar el seu braç per sobre de la seva espatlla i el va apropar
cap a ell.

--Quin és el pla, Planet?

El Doctor va assenyalar amb el cap a la carretera major de Chiswick i
les anades i vingudes del trànsit, i ràpidament la va arrossegar pel
carrer principal, llestos per perdre's entre les multituds.

Sinó fos per que no n'hi havia. De fet, no hi havia gaire gent al
voltant, només un parell de nens en monopatins a la vorera contrària i
un home gran passejant el seu gos.

El Doctor va aixecar la seva mà.

--No plou--va dir.

--Molt ben vist, Sherlock--va dir la Donna--. Què és diumenge?

--Volies el divendres quinze de maig del 2009, Donna. És això per al que
he programat el TARDIS.

La Donna va riure.

--Llavors serem a un diumenge del mes d'agost de l'any 1972.

El Doctor va estirar el cap cap al quiosc, somrient a l'home rere el
mostrador, que escoltava el seu reproductor MP3 tot ignorant al seu
possible client per complet.

El Doctor va mirar el diari mes proper:

--Divendres, 15 de maig del 2009--li va confirmar a la Donna.

--Llavors on és tothom?

--Potser és l'hora de dinar--va suggerir el Doctor--. O potser Chiswick
no és ja el centre de la societat com ho era fa un mes. Anem cap a casa
teva?

--Hi vens?

El Doctor semblava com si el pensament de no anar-hi amb la Donna no se
li havia passat pel cap.

--Oh. Um\ldots{} Bé, això anava a fer.

--Doctor, per què hi som aquí?

--És el primer aniversari de la mort del teu pare.

--I, tot i que crec que estarà molt agraïda per salvar el món dels
Sontaran, no crec que ma mare ho estigui de veure't, justament avui.

--El teu pare n'estarà.

--Si? Bé, emporta-te'l per fer una birreta aquesta nit al Shepherd's
Hut, però per començar vull veure'ls jo sola--la Donna seguia agafant-li
de la mà, i ella la va estrènyer carinyosament--. Ho entens, veritat?

Ell va somriure.

--Per suposat que sí. No hi pensava. Ho sento.

--No comencis amb això de nou, d'acord? --Donna va deixar-li anar la
mà--. Vaig a per unes flors i aniré cap a casa. Per que no ens hi trobem
aquí, a la mateixa hora que ara, demà?

--Aquí. Demà. Fet. --el Doctor li va picar l'ullet i es va allunyar--.
Hi ha una floristeria ben bufona a la cantonada per allà--li va
cridar--. Pregunta per la Loretta i diu que t'hi envio jo!

Va girar una cantonada i se n'havia anat.

La Donna va respirar profundament i va caminar en la direcció que havia
assenyalat.

Feia un any tot just avui.

Els Adipose, els Pyrovile, els Oods amb els cervells a les mans, fins i
tot els Sontaran, els Hath i els esquelets parlants tots semblaven
senzills en comparació amb el que anava a succeir aquella tarda.

Perquè aquella tarda la Donna havia de tornar i ser allà per a la seva
mare i probablement reviure no només l'últim any, sinó els dies i les
setmanes que l'havien seguit, els funerals, dir-ho a la gent, els
memorials, les notícies als diaris, evitar el costat financer de les
coses, trobar el testament\ldots{} Res d'allò havia estat fàcil per a la
mare de la Donna. Tampoc ho havia estat per a la Donna, tot sigui dit, i
feia un any aquell pensament l'hauria superat. La Donna Noble, posant-se
a ella mateixa primer. Però no ara, només un poc temps amb el Doctor li
havia ensenyat que no era la dona que havia estat.

I l'Avi, el pobre Avi, recordant la mort de l'Àvia, s'hi havia enfrontat
amb tothom, intentant evitar tot tipus de sol·licitants i directors de
funeràries i gent d'aquest tipus.

No és que la mare fos dèbil o feble, la Sylvia Noble no era així, i
també havien estat preparades per a la mort del pare, bé, tot el que
puguis estar per a això, però tot i així allò la superava. Podia
veure-ho als ulls de la seva mare, era com si algú li hagués tallat un
braç o una cama, i la mare ho va agafar tan bé com podia. Portaven
casats uns trenta-vuit anys.

La Donna va sospirar.

--Et trobo a faltar, pare--va dir en veu alta mentre va aturar-se davant
d'una bugaderia anomenada Loretta. El seu telèfon va vibrar amb un
missatge de text, i el va llegir.

``Ummm. De fet puc haver-me equivocat. Pot ser que la Loretta no sigui
una florista. Ho sento''.

Com ho feia allò? Pel que sabia ella no tenia ni telèfon mòbil. Aquell
tornavís sònic potser? Hi havia res que no pogués fer?

Guardant-se el telèfon mòbil a la butxaca de l'abric, la Donna va
decidir que seria millor anar-hi cap a Turnham Green. Sabia que allà hi
havia una floristeria.

Homes. Homes alienígenes. Inútils, tots ells.

En Lukas Carnes odiava la tecnologia. Fet que el feia una mica estrany,
segons els seus amics. La seva mare tenia un ordinador, però en Lukas
evitava fer-lo servir el menys possible, per res més que per a escriure
els assajos de l'escola un cop els havia fet a mà. Tenia un reproductor
MP3 que el seu germà petit (que tenia vuit) li havia ficat ell mateix
música per ell. I ni tan sols parlar-ne dels problemes relacionats amb
fer servir els DVD-R.

Ell era, va decidir el seu quinzè aniversari, un endarrerit per al seu
temps, quan els nois destres amb la tecnologia eren anomenats bitxos
rars i les noies corrien darrere d'ells. Per desgràcia per a en Lukas,
la majoria de les noies que coneixia volien un xicot que pogués
descarregar-se música a 20 MB i alliberar ~un mòbil que hagués vingut
d'una paradeta del Shepherds Bush Market.

I era per això que en Lukas no tenia xicota.

I era això el que hi afegia un gra més a la seva repulsió per la
tecnologia. Acceptava que la necessitava, però ell no volia entendre-la.
El seu cervell no estava connectat per entendre les compressions MP3 i
el 3G i els sistemes de rastrejament de GPS. Ell només volia prémer el
botó d'encès i tenir-ho tot fet. No era pel que havia passat el grup
d'edat de la seva mare durant la primera/segona/tercera generació de
tecnologies? Per a que ell pogués prémer botons i coses que funcionaven
sense passar-se de moda als sis mesos i redundants als dotze. A la
televisió parlaven sobre els dies quan es podria fer petar els dits i
les portes s'obririen, quan es podria entrar a una habitació i dir
``llums'' i un ordinador ho encendraria tot, just al nivell correcte.

Déu. ¡Era la seva àvia! Tot seguit, estaria dient que no entenia la
música pop i de quin sexe era aquella persona de la Popworld?

Quinze, no cinquanta, Lukas.

Llavors, per què hi era davant la botiga local de Discount Electronics,
veient una exposició del últim Processador de Quarta Generació d'algun
portàtil més fi que un guix?

Pel seu germà, de vuit anys, destre amb la tecnologia, en Joe, que li
havia demanat. Bé, estrictament parlant, la mare li havia demanat. Quan
el pare d'en Joe se'n va anar, igual que el del Lukas abans que ell, el
germà gran s'havia convertit a la pràctica en el pare del seu germà
petit. Que li anava bé a en Lukas, perquè en secret, adorava en Joe,
però mai li diria. I perquè els germans petits necessitaven saber qui
era el cap, i el poder d'en Lukas es perdria al primer senyal de
debilitat.

I en Joe havia actuat bastant malament des que el seu pare se'n va anar,
ficant-se en problemes a l'escola i al barri, i la mare havia estat a la
policia dos cops.

I per això el Lukas s'havia emportat en Joe i li havia explicat el
millor que havia pogut a un nen de vuit anys que no era culpa de la mare
que el seu pare s'hagués anat, ni tampoc la d'en Joe, i que embolicar-se
amb nens més grans, ajudar-los a robar cotxes i aquelles coses no era
forma d'ajudar a la mare.

Després d'uns mesos, en Joe s'havia calmat. Però ara anava darrere d'en
Lukas sempre, i s'enrabiava si el seu germà gran no el portava a totes
parts. Fins i tot en Lukas havia començat a portar en Joe a l'escola
abans d'anar a l'institut Park Vale. Fet que la mare ho apreciava, no
gaire, però ho feia.

Però de vegades, en Lukas volia enrabiar-se ell sol, estar sol, no ser
el responsable.

Aquell dia era un d'aquells dies, però era allà amb en Joe, veient
aquella nova exposició junt amb trenta persones, totes ajuntades a una
botiga on hi cabrien gairebé deu persones. Que Déu els ajudés si hi
havia un incendi.

Una gran (a totes direccions) dona es va moure davant seu, pel que en
Lukas va pujar a en Joe a coll-i-bé per a que pogués veure millor. Allò
significa que en Lukas no podria veure res. Així que mentre en Joe (com
n'havia, de crescut!) ho observava tot amb intensitat, la mirada d'en
Lukas va moure's per la botiga.

Un home llargarut amb un vestit d'home blau estava tafanejant una
exposició de portàtils que probablement es passés de moda aquella
mateixa nit. L'home cercava alguna cosa a Internet, en Lukas podia veure
repetides pantalles mostrant un cercador (oh, els termes tècnics!) i
arrufant les celles. Era obvi que no aconseguia el resultats que volia.

L'home va mirar la seva butxaca i va treure'n un tub brillant, com un
subratllador, i va apuntar-lo contra la pantalla. Al començament, en
Lukas va pensar que anava a escriure a la pantalla del portàtil però en
canvi, la punta del bolígraf va brillar amb un to blau, i en Lukas va
observar sorprès mentre les imatges a la pantalla es descarregaven i
canviaven a una velocitat sorprenent.

No, a una velocitat impossible. L'home del vestit blau va treure un
parell de gruixudes ulleres negres d'una altra butxaca i se les va posar
mentre mirava intensament a les pantalles canviants. Podia llegir tan
ràpid?

Va adonar-se de que en Lukas el mirava i va somriure, gairebé
avergonyit. El bolígraf brillant va tornar a la seva butxaca, i les
ulleres a l'altra.

En Lukas es va adonar de que tenia la boca oberta, i la va tancar de
cop.

L'Home del Vestit Blau li va picar l'ullet a en Lukas i estava a punt de
sortir de la botiga, quan va aturar-se i va agafar un pamflet sobre
l'exposició que el germà petit d'en Lukas estava veient. Llavors va
mirar a la gentada i s'hi va apropar.

En Lukas va dirigir ràpidament la seva atenció de nou a l'exposició. O
al menys a la part de darrere del cap de la dona grassa.

Després d'uns minuts escoltant a un ros estrany sobre com de
revolucionari era el nou sistema informàtic, l'Home del Vestit Blau es
va encongir d'espatlles i va murmurar alguna cosa sobre ``impossible'' i
``no a aquest planeta'' i ``contradiu el divuitè protocol de la
Proclamació de les Ombres'', i llavors en Lukas va decidir que, amb un
bolígraf brillant o no, aquell paio era probablement un boig. Potser
podria apartar en Joe d'ell, només per si el paio portava un ganivet.

En Lukas es va inclinar per a xiuxiuejar-li a l'orella d'en Joe que era
moment d'anar-se'n a casa, just quan l'Home del Vestit Blau li va fer un
copet amb el colze.

--Doncs, tothom és dins d'aquestes botigues electròniques, veritat?

--Perdó?

--Els carrers són bastant buits. Mentre venia cap aquí, m'he adonat que
tothom era a botigues com aquesta, mirant els aparadors.

--Avui és la data de llançament--en Lukas es va trobar a si mateix
explicant-li-ho--. Tothom hi està interessat.

--Tu no--va respondre l'Home del Vestit Blau.

En Lukas es va encongir d'espatlles.

--El meu germà petit n'està.

--Ah, ja veig.

En Lukas va intentar donar un pas enrere, però va ser arraconat entre un
home a un costat i la dona grassa a l'altre.

L'Home del Vestit Blau va treure el seu bolígraf brillant de nou.

--No em malinterpretis--va dir.

Però en Lukas el va malinterpretar. I molt.

--Per què hi ets aquí? --va preguntar.

L'Home del Vestit Blau es va encongir d'espatlles.

--Bé, primer de tot, estic deixant que una amiga vagi a casa seva una
estona. En segon lloc, em preguntava per què tothom és aquí. I en quart
lloc, ara sóc molt conscient de la tecnologia d'aquest portàtil.

En Lukas va saber que se'n penediria:

--I en tercer lloc?

--En tercer lloc? --l'Home del Vestit Blau semblava confós, i llavors va
somriure quan se li va venir al cap--. Oh sí, en tercer lloc, he vingut
buscant-te a tu, Lukas Samuel Carnes--li va allargar la mà--. Sóc el
Doctor i estic aquí per a salvar la teva vida.

En Dara Morgan va beure del seu cafè lentament. En part perquè això deia
que era assenyat, i en part perquè era massa calent per fer res més.
Però probablement semblaria assenyat per al senyor Murakami i a la seva
delegació.

--Llavors, senyor Morgan--deia el banquer japonès--, tenim tracte?

Els ulls blaus d'en Dara Morgan van brillar amb malícia mentre mirava la
Caitlin, dempeus vora la porta de la oficina.

--Què en penses, Cait?

La Caitlin s'hi va apropar, amb les seves llargues cames i la seva
faldilla curta clarament atraient la mirada d'alguns companys del senyor
Murakami però no, va observar en Dara, del mateix senyor Murakami. Bé.

--Crec que és un bon tracte, senyor--va dir ella--. Si en Murakami-san
pot portar el M-TEK cap a l'est aquest diumenge, serà soberbi.

En Dara Morgan es va apartar el cabell dels ulls.

--Justament en dos dies, a les 3.30pm hora de Tòquio. Factible?

El senyor Murakami va arrufar les celles.

--Per què el diumenge? És un avís ridículament curt.

En Dara Morgan només va somriure.

--Diguem que és veritablement on el tracte es recolza. Necessito la
garantia, Murakami-san, o es pot anar a qualsevol altre lloc.

--Però d'aquesta manera, tens menys oportunitats de fer un tracte per a
llavors--va dir l'home japonès.

En Dara Morgan va fer que sí amb el cap.

--Ho sé. Però siguem sincers, amb els diners que el M-TEK farà, unes
companyies més petites com la seva, algunes amb més fam potser, vindran
cap aquí per acceptar els meus\ldots{} requisits de MorganTech--va
tornar a beure del café--. Estarà al contracte, a les clàusules de
penalització.

--Què serà?

--Catastròfic. Per a tot Japó.

La gent del senyor Murakami es van apropar una mica més cap al seu cap.

--Era això una amenaça, senyor Morgan? --li va preguntar suaument.

--No--va dir en Dara Morgan--. No faig amenaces. Els bàrbars en fan. Els
idiotes en fan. Jo només anuncio fets.

--És una gran oportunitat--va interrompre la Caitlin--. Si us plau,
pensi-ho fins al sopar. Aquesta nit. Paguem nosaltres.

--Per cert, no ens podrem reunir amb vostès--va afegir en Dara Morgan--,
però és lliure d'escollir qualsevol restaurant de Londres que sigui del
vostre gust i totes les despeses seran cobertes per MorganTech.
Insisteixo.

--Totes les despeses?

--Relacionades amb el menjar i el beure, sí.

--Ah. En aquest cas, tindrà la meva resposta aquesta nit a mitjanit--el
senyor Murakami es va aixecar i en Dara Morgan va fer el mateix, fent
una lleugera reverència quan ho va er. El senyor Murakami li va
respondre de la mateixa manera, incloent a la Caitlin, i ella va
assentir amb el cap i a la resta.

Un cop passades les formalitats, la delegació japonesa va anar a la
porta de la suite, però el senyor Murakami es va girar un últim cop.

--De veritat, per què el diumenge? Per què a les 3.30 de la tarda?

--Perquè hi ha un gran esdeveniment que succeirà per tot el món el
dilluns a les 3pm hora de Regne Unit. Això es a les 11pm hora de Tòquio.
Però necessitem els tractes arreglats. Jo també he fet un tracte, ja
veu, però es més com una propietat en cadena: una anella es trenca i tot
el tracte es cau pels terres. Llavors tots sofrim.

--Tots?

--Universalment--en Dara Morgan va llançar una mirada a cua d'ull a la
Caitlin, i immediatament es va moure per escortar el senyor Murakami a
l'exterior de la sala.

Un moment més tard, els japonesos s'havien anat i la Caitlin va tornar
al costat d'en Dara Morgan. Ell estava dempeus davant una gran finestra,
amb una gran vista panoràmica per sobre de l'oest de Londres. Podia
veure l'estadi de Wembley, el Centrepoint, el London Eye i les altres
estructures altes de Londres.

--Dilluns--va somriure--, i aquest planeta serà de Madam Delphi.

La Caitlin va fer que si amb el cap.

--Al final. La venjança serà seva.

Es van agafar les mans i les van prémer amb força, i mirant a les
pantalles dels ordinadors, que semblaven taral·lejar un ritme, tan
fluix, creant onades a una de les pantalles per fer polsos de forma
fraccionada en el temps.

--Benvinguda de nou--li van dir a ella, junts.

Donna estava dempeus al capdavant de Brookside Road i va respirar
profundament. No havia passat tan temps des de l'últim cop que hi havia
estat (de fet, per a la seva mare, hauria estat menys temps encara),
però cada cop que tornava a casa tenien moments incòmodes. ``On has
estat?'' i ``Encara continues anant per allà amb aquell horrible paio
Doctor?'' i ``Per què no truques?'' i ``Encara no tens treball?''.

Per suposat, l'avi Wilf sabia on era ella, li havia dit tot des del
començament. Però la seva mare, bé, no era algú qui ho pogués entendre.
No era algú que pensés que salvar els Ood, aturar guerres generacionals
o assegurar-se de que Carlemany conegués el Papa equivalgués a un bon
treball com prendre notes o fer comandes.

Respirant profundament, va caminar cap a la casa que no havia estat casa
seva durant molt de temps.

Després de la seva desastrosa boda i el lleugerament menys desastrós
viatge a Egipte, els seus pares es van anar a viure de la seva casa
adossada on la Donna havia crescut a aquell nou xalet. Havia estat tot
un trasbals, afegint que la Donna no tenia cap treball i que el Wilf
havia estat neguitós al principi per deixar els seus amics astrònoms
enrere. Com va resultar ser, per suposat, els horts eren més fàcils
d'arribar des d'aquella casa, i va acabar sent feliç al fi i a la cap.

Però el pare de la Donna no havia estat bé durant molt de temps, i de
moltes formes el anar-se'n a viure havia estat idea seva, el seu desig
de trobar algun lloc nou on estar-hi, de donar-li reptes nous. S'havia
avorrit a l'antiga casa. Havia fet tots els armaris, posat prestatgeries
a totes les parets, pintat tots els sostres que havia pogut, i ara
necessitava alguna cosa nova a fer per mantenir-lo actiu ja que la seva
malaltia l'havia fet retirar-se massa d'hora. Preparar el nou lloc amb
les exigents especificacions de la mare seria el repte correcte.

Havien estat allà durant tres mesos abans que el pare es morís. La Donna
i el Wilf havien pres el control de tots aquells treballs manuals que el
pare havia estat fent, però mai els feien bé , mai eren ``com el teu
pare ho hauria fet''. El qual no era gaire sorprenent, en Wilf tenia
vint i escaig anys més, i la Donna mai havia aixecat un pinzell o un
martell a la seva vida.

Mare meva. Com de superficial podia haver estat la Donna Noble abans
d'anar amb el Doctor? Abans havia aprés no només a estar dempeus amb
ambdós peus sense adonar-se. La seva vida familiar era a una situació
molt trista. Havia estat inútil a casa perquè els seus pares li havien
permès ser-ho, o la seva mare creia que era inútil perquè ho era de
veritat?

I parlant del tema, parlant de res alhora, amb la Sylvia Noble era una
experiència rarament positiva. La Donna hauria volgut dir que tota
l'amargor de la seva mare i tot el seu ressentiment era per la mort del
pare, però la veritat era que la Sylvia sempre havia estat decebuda amb
la seva filla. Ella difícilment ho amagava. I la Donna mai entendria
perquè. Havia volgut un fill? Havia volgut un advocat d'alts vols o una
filla executiva d'una companyia que seria bastant rica com per enviar
als seus pares a viure al camp a una caseta del segle XVI on podrien
tenir cura de cabres? Havia empitjorat des que el pare havia mort?
Hauria anat tot millor si s'hagués casat amb el Lance? Li hauria haver
dit a la mare la veritat sobre aquell dia? Com ho havia fet amb l'avi?
Probablement no, perquè a la Sylvia no li agradava la gent ser sincera i
honesta. `Aquests cors sagnants que porten els seus cors a les mans' era
una analogia amb la que l'havia torturada temps enrere, i havia resumit
la seva opinió de la gent sent honesta.

La Donna recordava llegir un article d'una revista temps enrere sobre
com els pares mai esperaven entendre de veritat a la canalla adolescent
i la seva millor aposta por una convivència harmoniosa era només tolerar
aquells tres o quatre anys de malsons. Però hi havia un manual per a
fills i filles sobre com tractar amb pares negatius? Era impossible
discutir amb una mare, tenien un botó de ``viatge de la culpa'' innat
per a prémer que prohibeix dir totes les coses que voldries dir-les,
sigui el que sigui el que els diguessin.

La Donna estimava a la seva mare, no hi havia cap dubte. I tampoc tenia
cap dubte de que la Sylvia Noble estimava la seva filla.

Senzillament no estava completament segura de que s'agradessin gaire
l'una a l'altra.

--Hola, Donna! --li va cridar la senyora Baldrey des de la vorera de
davant--. Ha anat bé el viatge?

--Si, gràcies--la Donna li va retornar el somriure--. Com està en
Seymour?

--Oh, bé. Segueix queixant-se sobre la pròstata--li va grunyir la veïna.

La Donna pensava francament que aquella conversació havia anat tot el
lluny que volia que anés i va accelerar el seu pas cap a casa.

Un gat estava assegut sota un fanal, observant atentament, aproximar-se
a la Donna, no molt segur de si era un amic o no. La Donna va fer un
soroll amable per atraure la seva atenció.

Va sortir corrents.

Ah, bé.

El cotxe de la mare era a la carretera (tens un cotxe, mare, fes-lo
servir) i la Donna va tocar el capó al passar. Era fred. La mare no
havia sortit aquell dia. Era divertit com havia agafat aquells petits
detalls al viatjar amb el Doctor per descobrir coses. Com si el cotxe
havia estat fet servir. La vella Donna mai hauria pensat en allò. A la
vella Donna no li hauria importat.

La vella Donna s'havia anat.

Gràcies a Déu, la seva vida era un trilió de vegades millor aquells
dies.

Si només podria involucrar a la seva mare en allò, va pensar. Aquella
última peça del trencaclosques, aquell últim tros d'acceptació de cada
un d'ells.

--Oh, hi ets allà, senyoreta--va dir una veu familiar darrere seu--. Em
preguntava si et veuríem avui.

La Donna no es va molestar en girar-se.

--Hola, mare--va dir.

--Oh, sí. ``Hola, mare'', perquè amb això és suficient, oi? En un minut
l'aire està ple dels fums dels cotxes, la següent nit el cel està en
flames, i ja està. I no tinc ni idea d'on és la meva filla. Sense
trucades, sense missatges, ni tan sols un missatge al teu pare, per a
que em pugui fer callar. Res.

La Donna va aturar-se al carrer i es va girar per mirar a la cara de la
seva mare, automàticament agafant dues de les quatre bosses de la compra
de la seva mà per a ajudar.

La vella Donna no ho hauria considerat.

--També m'alegro de veure't--va dir la Donna--. L'avi ha posat la tetera
a bullir? M'aniria bé una tasseta, moltes per a ser-te sincera.

La Sylvia Noble es va encongir d'espatlles i va passar pel costat de la
seva filla.

--Saps quin dia és avui, veritat? --li va dir.

I la Donna es va quedar quieta.

Per suposat que sabia quin dia era. Per què dimonis pensava que hi era?
Com podia atrevir-se tan sols a preguntar allò?

La porta principal i la Sylvia la va empènyer passant pel costat de
l'avi Wilf i anant sense dir res cap a la cuina.

--Saps el que m'acaba de preguntar? --va xiular la Donna al seu avi
després de fer-li un petó a la galta.

En Wilf va aixecar les celles:

--No serà avui un d'aquests dies, oi?

La Donna va obrir la boca per a respondre-li, però es va aturar.

La vella Donna hauria ignorat aquell comentari, allà i en aquell moment.
La vella Donna hauria començat una discussió amb la seva mare,
llençant-se l'una a l'altra comentaris com ``aptitud'' i ``em dóna
igual'' i ``egoista''.

La nova Donna no ho faria.

Perquè la nova Donna, tal i com n'estava de frustrada, entenia què volia
la Sylvia Noble i que avui el que necessitava era plorar.

Però ésser la Sylvia ``cor sagnant que porta el cor a la mà'' Noble
significava no fer mai allò.

I per desgràcia, tocava ser tasca de la Donna assegurar-se de que ho
fes, abans que l'embombollament de la seva mare li fes més mal del que
ja li feia.

El Doctor baixava per la carretera principal de Chiswick, mirant
diferents botigues on la gent s'hi quedava observant els aparadors dels
nous portàtils.

--Es només un ordinador--va xiuxiuejar--. Per què tant d'interès?

--Ja són antics--va dir una veu jove darrere seu--. El M-TEK, això és el
futur.

El Doctor va mirar darrere seu, després avall. El locutor gairebé li
arribava a l'altura dels genolls. Era un nen petit. El Doctor l'havia
vist en algun lloc abans, llavors corrents darrere seu, va veure en
Lukas Carnes i es va adonar que era el germà petit que portava a
collibé.

--Joe! --cridava en Lukas--. Com t'has anat de mi tan ràpid?

--No sóc el teu presoner--li va cridar en resposta el Joe, i llavors el
Doctor li va llençar una mirada a un Lukas que intentava agafar aire que
deia ``I amb això ho ha dit tot, col·lega''.

En Lukas va apartar en Joe del Doctor.

--Deixa'l en pau--li va dir el Lukas--. Toca'l i cridaré a la policia--i
com per tancar la frase en Lukas tenia el mòbil fora i preparat.

El Doctor no es va molestar tan sols en assenyalar que el Joe li havia
trobat a ell. Ni tampoc que en Lukas només estava sent agressiu pel que
li havia dit a l'altra botiga.

Havia de recordar que la gent no sempre li agraden les pistes sobre el
seu futur. Sempre tenia problemes amb allò.

Spoilers, va dir algú.

--A què et referies amb que em salvaries la vida?

El Doctor es va encongir d'espatlles.

--Ja ho he fet.

--Fer el què?

--Salvar la teva vida.

--Com? Quan i per què?

El Doctor va treure el seu paper psíquic de la butxaca interna de la
jaqueta i li va mostrar a en Lukas. Tenia la data correcta escrita, el
nom de la botiga i el missatge: SALVA LA VIDA D'EN LUKAS SAMUEL CARNES
EVITANT QUE EL SEU GERMÀ COMPRI UN M-TEK.

--No sé qui ho ha escrit--va dir el Doctor--, perquè no reconec la
lletra. Però ha aparegut al paper vint minuts després de que arribés a
Chiswick. Una mala forma d'ignorar missatges, crec jo. Així que ja has
estat salvat, el meu treball està fet, i ara l'únic que haig de fer és
trobar si la Loretta es una bugaderia, una floristeria o un cafè en
aquest període de temps. És una d'aquestes al 2009 però no estic segur
de quina.

--Bugaderia--va dir en Joe.

En Lukas va empènyer el Joe darrere seu.

--No li parlis a aquest home--va renyar al seu germà petit.

--Però tu ho fas amb ell--va protesta en Joe, amb molta raó.

--Això és diferent--va dir en Lukas, adonant-se amb terror que allò era
exactament el que la seva mare li deia quan trobava que ella havia dit
alguna cosa hipòcrita o injusta.

El Doctor es va girar.

--M'alegro de veure-us, nois, però crec que hauria d'anar-me'n cap a un
encantador restaurant a Brentford, cap al canal, i passar-hi el temps
abans de trobar-me de nou amb la meva amiga.

--Què? --va dir en Lukas, penedint-se mentalment de preocupar-se. Només
deixa que el rarot se'n vagi, es va afanyar a si mateix. Però la seva
boca no podia deixar de fer preguntes--. Què hauria passat si Joe
tingués un M-TEK?

I el Doctor li va mirar.

--No tinc ni idea. No sé què es un M-TEK. Suposo que era el portàtil que
estaves pensant en comprar. Vaig fer una cerca, però no vaig poder
trobar cap resultat. Llavors us vaig veure mirant l'aparador, com tothom
sembla fer, així que suposo que és això un M-TEK.

--No--va dir en Joe, fent-se camí cap endavant--. Això era el nou Psiryn
Book Plus. És deixalla. Jo volia un M-TEK.

--Però no pretenia comprar-ne cap--va afegir en Lukas--. Ni pel Joe ni
per a mi.

--Llavors què es un M-TEK? --el Doctor va arrufar les celles.

En Lukas va fer un sospir.

--Com pots haver-me salvat d'això, si no saps què és?

--Si sabés tot salvaria la gent abans d'intentar salvar-les, i salvaria
molt poca gent perquè em passaria el dia buscant de què els salvava, no
creus?

En Lukas i en Joe es van mirar l'un a l'altre.

--Ets divertit--va dir en Joe.

--Gràcies--va dir el Doctor.

En Lukas va fer que no amb el cap.

--Anem-nos a casa--li va dir al Joe--. Vine--va arrossegar al seu germà
petit.

--Adéu, Doctor--li va cridar el Joe.

El Doctor va acomiadar-se dels nois amb la mà mentre s'esvaïen cap a un
carrer perpendicular. Llavors va començar a caminar per la part més
baixa de Chiswick, pel pas elevat de la M4, cap a Brentford. I aquell
italià tan bonic a la plaça. El Luna Piena. No havia tingut un menjar
italià decent des feia anys. Segles potser.

DES DEL 1492 es va escriure al seu paper psíquic.

Fet que era estrany, perquè el paper psíquic no feia allò. Al menys, no
ho havia fet en el passat. Ja era bastant dolent que la gent el fes
servir més i més per a enviar-li missatges últimament, però quan
començava a respondre-li sense invitació prèvia, era hora de donar-li la
jubilació després de dos cents anys de servei.

Va tornar a ficar la cartera de cuir amb el paper de tornada a la
butxaca de la jaqueta i es va intentar oblidar de tot allò.

Al fons de la seva ment, tot i així, seguia tenint una preocupació
persistent, un eco de la pregunta raonable d'en Lukas Samuel Carnes: què
era un M-TEK i com havia salvat en Lukas d'ell?

~Resposta a la qual era òbvia.

Encara no ho havia fet.

Pel que en Lukas seguia en perill (si el paper psíquic era de
confiança), i havia de salvar-lo.

Oh, i una altra pregunta necessitava resposta.

Com havia sabut el Joe, el germà petit d'en Lukas, que es deia Doctor?

Així que\ldots{} el Luna Piena o investigar?

No era una decisió gaire difícil. El menjar estava bé, però els
misteris, eren molts millors.

Es va preguntar com li estaria anant a la Donna i si podria deixar-se
caure i dir-li que estaria ocupat un parell de dies.

Nah, probablement estaria millor sola amb la seva família.

I així fa ver mitja volta i va anar cap al carrer per on havien girat
ambdós nois.

A la suite de l'àtic de l'Hotel Oracle, en Dara Morgan i la Caitlin
observaven un grup de pantalles planes, connectades a un ordinador, pels
cables de fibra òptica que en Terry Lockworth i en Johnnie Bates havien
mort instal·lant temps abans.

A la majoria de les pantalles hi apareixia una ona sinuidal, bategant
rítmicament, com si l'ordinador estigués respirant. I ho feia. Més o
menys.

Però a la més gran i central de les pantalles hi havia una imatge, una
fotografia, presa de les càmeres de seguretat que havia estat hackejada,
redimensionada i adaptada als paràmetres i posada en una resolució
perfecta.

--Madam Delphi--va preguntar en Dara Morgan--. Què és això?

El seu dit va traçar la silueta. Era una alta cabina blava a un carreró
de Chiswick entre dos contenidors.

--El TARDIS--va respondre una fort veu femenina, ressonant per
l'habitació, i les ones sinuoses a les altres pantalles bategant i
canviant mentre parlava.

--Ell és aquí--va dir la Caitlin--. Ja ha arribat.

En Dara Morgan va fer que si amb el cap amb entusiasme.

--Cinc cents anys, com presagien les llegendes. El Portador del Caos.

--Cinc cents disset anys, un mes i quatre dies--li va corregir la Madam
Delphi--. No ens podíem permetre els moviments còsmics fa cinc cents
anys. Allò va ser bastant\ldots{} desafortunat.

La Caitlin es va dirigir a l'ordinador.

--Però Madam Delphi, hi ha hagut altres intents\ldots{}

--I per culpa dels moviments còsmics, per culpa de que l'univers respiri
poc profundament i alhora profundament, els alineaments mai han estat
perfectes.

--Però aquest dilluns tot ho serà--en Dara Morgan va acariciar les
superfícies de la Madam Delphi--. I vostè tindrà la seva venjança.

--Amb el Doctor. Amb la humanitat. Amb tot l'univers--va dir la Caitlin,
emocionada.

--Oh, clar--la Madam Delphi va emetre ones sinuoses--. Absolutament.
Adoro això de la venjança, estimats meus. Sobretot amb el Doctor.

Eren les 5pm al Regne Unit. Així doncs, a Nova York, les ombres de la
Gran Illa es feien més estretes mentre el sol de migdia brillava,
cobrint la ciutat amb una poc usual capa d'humitat.

Allò no eren bones notícies per als treballadors del bloc d'oficines de
MorganTech a la 52ena amb la 7ena. L'aire condicionat havia fallat hores
abans, i les fonts automàtiques havien deixat de treure aigua freda. La
principal raó per tot allò era que l'energia de l'edifici estava
tallada. Les portes principals havien fallat en primer lloc, seguides
pels telèfons, els serveis informàtics, l'aire i tot allò.

Havia portat a la Melisa Carson de recepció uns pocs minuts comprovar
que tot anava malament. Va intentar cridar manteniment. Òbviament, com
tot el manteniment havia fallat i no hi havia forma de que manteniment
mantingués res. Allò havia molestat a la Melisa, pel que havia comés el
crim corporatiu d'abandonar la seva taula per trobar a algú.

En lloc d'això, el que va trobar (més que ascensors tancats probablement
contenint passatgers ràpidament deshidratant-se i portes elèctriques
internament tancades) era una pila de pols al terra vora una caixa de
cables al subterrani. Probablement manteniment havia estat fent alguna
cosa amb els cables i s'havien fos els sistemes. No se li va ocórrer a
la Melisa (per què ho hauria hagut de fer?) que les cendres que
trepitjava per casualitat amb els seus talons de Dolce \& Gabanna havien
estat un home a qui havia saludat amb la mà hores abans aquell dia
anomenat Milo. Però sí es va preguntar on seria en Milo i els nois.

Casualment, mentre tornava arrossegant els peus cap a la seva taula amb
frustració, va tancar la capsa de cables oberta.

D'un cop, els recent instal·lats cables de fibra òptica van cobrar vida,
bategant amb llum morada a través de la seva xarxa. Els ocupants de
l'edifici, ja preocupats pels ascensors que no funcionaven, la manca
d'aire condicionat i els ordinadors espatllats, només van tenir deu
segons per adonar-se que els ordinadors emetien llum i tornaven a
funcionar. Tal i com fan els oficinistes, tothom es va inclinar i va
començar a tocar els teclats.

Un massiu arc de llum morada va bategar a través de l'edifici, tocant a
tothom, no només aquells que feien servir un ordinador. Ni una sola
persona, panerola o arna de l'edifici es va salvar del batec morat
d'energia.

Quaranta-dos segons després de que la Melisa Carson hagués tancat la
capsa de cables, els cent set humans, les divuit rates, les dues cent
criatures de diferents mesures i formes amb sis potes o més i tres
coloms al terrat estaven morts.

--Tenim un petit problema, nois--la Madam Delphi va bategar a en Dara
Morgan i la Caitlin--. L'edifici de MorganTech de Manhattan està fora de
línia. Els terminals estan terminals-- hi va haver un soroll semblant a
un riure elèctric, i unes ones movent-se al seu ritme.

En Dara Morgan va arrufar les celles, colpejant una altra part del
conjunt de monitors i teclats de la Madam Delphi.

--No m'equivoco--va informar l'ordinador.

--Ho sé--va dir ràpidament en Dara Morgan--. Mai t'equivoques. Només
intento saber quin tipus de problema és.

--Error humà--va informar la Madam Delphi--. Què més podria ser? Vull
dir, siguem honestos, sempre sou l'anella més dèbil de la cadena.

--Necessitem Nova York--va dir la Caitlin.

--Bé, ja no tenim Nova York--va dir l'ordinador.

--Pots anular el batec?

--No--va engegar la Madam Delphi--. Es massa tard, de tota manera.
Honestament, reis meus, gasteu el vostre temps. Vaig a veure si puc
moure les fonts d'energia cap a un servidor secundari i començar de nou.

En Dara Morgan va suar per primer cop en molt temps.

--No ho entens? No tenim temps de començar de nou. Hem de tenir els
M-TEKs de Nova York connectats a les 10am de Manhattan el dilluns.

--És la ciutat que mai dorm, si no recordo malament la lletra de la
cançó--va dir l'ordinador.

--Sí, potser mai dorm. Però sí que deixa de treballar a les 5pm del
divendres i no va a treballar els caps de setmana.

--Oh, estimat meu, dóna'm una mica de fe. He tractat amb aquest tipus de
coses per tot l'univers els últims milions d'anys. Només hem de fer una
fusió aquesta tarda. Una presa hostil de MorganTech sobre una companyia
petita i\ldots{} deixeu-me endevinar\ldots{} oh, sí, mireu, aquí n'hi ha
una\ldots{}

A la pantalla gran de la Madam Delphi hi havia una imatge d'un petit
(per a Nova York) edifici d'oficines, tot metall i vidres amb gent
treballant d'un costat a l'altre.

--Estic accedint als seus sitemes\ldots{} ara. Oh, sí. molta gent. Fan
serveis de hardware i reparacions, una firma co-dirigida per la Mafia
italiana, una triada xinesa i inicialment fundada per fer blanqueig de
l'IRA. Cap dels quals no en sap res dels altres òbviament. És senzill
prendre-hi control perquè cap dels tres es queixarà i la reclamarà.

La imatge va canviar a una imatge més allunyada.

--Lexington amb la 3era, bonic lloc--va continuar l'ordinador--. Hi vaig
conèixer un cibercafé un cop. Mai em va tornar les trucades.

Filera rere filera de siluetes van recórrer les pantalles, massa ràpid
per a que en Dara Morgan pogués tan sols contar-les i llavors les ones
van tornar, bategant de nou mentre la Madam Delphi els parlava.

--Kittel Software Inc, ara subsidiària de MorganTech. Espero que no us
importi, però he hagut de permetre que en Harvey Gellar segueixi sent el
director executiu, amb un recompte dels vots.

La Cait va arrufar les celles.

--No serà això un problema?

--En una hora i divuit minuts, el senyor Gellar es ficarà dins
l'elevador, vull dir, l'ascensor, i s'hi muntarà per baixar a la planta
baixa. En una hora i vint-i-dos minuts, l'ascensor es quedarà parat
entre els pisos divuit i dinou. Pitjarà el botó d'alarma. Serà l'última
cosa que faci. Trobaran el cos al seu interior, oh, un parell d'hores
més tard i suposaran que ha estat un atac de cor. Ja he reescrit el seu
testament per a assegurar que a la seva mort el seu cosí tercer
irlandès, Dara Morgan de MorganTech, hereti tot.

--Jo no sóc el seu cosí tercer\ldots{}

--Ara sí segons els arxius de l'FBI--les pantalles de la Madam Delphi es
van enfosquir lleugerament, amb les ones brillant amb un to vermell
mentre bategaven--. El teu constant menyspreu del que puc fer, Dara
Morgan, comença a molestar-me.

--Ho sento.

--Bé. Llavors, ara, aquesta nova branca de MorganTech apareixerà per a
llançar els M-TEK aquest dillus. Tenen els gràfics, els detalls, les
llistes de clients, tot. Tot el que hem de fer és enviar un parell de
paios aquest cap de setmana per a treure els cables i posar les nostres
fibres òptiques. I, llestos, els subcontractats agafats i assignats. És
senzill.

La Caitlin va mirar a en Dara Morgan.

--Sí, Madam Delphi, senzill.

--Brindeu a la meva salut, estimats meus. Ara si no us molesta, nens, em
vaig a descarregar els capítols d'aquesta setmana de Coronation Street.
Qui sap què farà aquest bufó d'en David Platt a continuació?

La Sylvia estava posant la compra a la cuina. Amb cura i tot al seu
lloc. Com sempre. Tot i així, feia que les portes dels armaris de la
cuina es tanquessin una miqueta massa fortament.

--Què he fet ara? --li va xiuxiuejar la Donna al seu avi des de la
butaca que mirava cap a la televisió a la saleta d'estar. On el seu pare
s'asseia i es reia amb Factor X. I veia un cop i un altre Dad's army. I
aquell programa on\ldots{} on\ldots{}

Bé, tots els seus programes, al cap i a la fi.

I de cop, es va voler moure cap al sofà més a prop del seu avi, però ell
en lloc es va asseure més a prop de la butaca on seia ella.

--No sé què vols dir, reina--a dir, sense intercanviar la mirada.

--Clar, perquè la mare està fent servir les coses d'IKEA per assajar la
bateria d'alguna cosa que va escriure en Ozzy Osbourne només perquè ara
resulta que li agrada el heavy rock, veritat?

--Oh, ella només\ldots{} és ta mare. Ja saps\ldots{}

--No, avi. No, no ho sé--la Donna va fer un sospir i va mirar al diari
local a la tauleta del cafè. La portada parlava de l'opertura d'un
supermercat Q-Mart rival tot just enfront d'un Betterworth a Park Vale.

Emoció-landia.

--Estic aquí. A Chiswick, Londres, W3, la terra. Ahir era a un altre
planeta, aturant un robots de fer una guerra civil. Fa una setmana, érem
a un bazar de Garazoone muntant cavalls de sis potes! --va agafar de cop
la mà de l'avi Wilf--. Tenien sis potes! Sis. Vull dir, com anàvem de
ràpids? Era brillant. M'encantava. I el Noi Marcià no deixava de cridar
a ple pulmó: On es l'interruptor per apagar-ho? Perquè creia que
s'apagaven així.

--No és marcià, oi? Creia que va dir\ldots{}

~--No, avi, no és marcià. És només una broma. Te'n recordes de les
bromes? Ja saps, aquell moment on obres la boca i fas ja, ja, ja?
Acostumàvem a fer-ne, a aquesta casa, un cop o dos.

La Donna va observar la cara arrugada del seu avi. Quan s'havia tornat
vell tan de sobte? Era també la pressió de la mort del pare? Què li
havia passat a aquell home amb qui donava voltes al seu vell Aston? Qui
li presentava els seus amics paracaidistes al club social? Quan l'havien
substituït per aquell vell de cabell blanc assegut davant seu?

Quan li havia semblat la idea de tornar a casa tan terrible? Era la part
negativa d'estar amb el Doctor? Aquella normalitat ara li era aliena?

--Haig de dir-te una cosa, reina--li va dir el seu avi--. Espero que
t'alegri. Espero de veritat que ho faci.

Bones noticies a la fi. La Donna va somriure.

--Bé, digue-m'ho. Escup.

El seu avi va obrir la boca per a parlar, però la Sylvia va escollir
aquell moment per arribar a la saleta d'estar i asseure's al sofà al
costat de l'avi.

--Així que, on has estat, Donna Noble?

La Donna va obrir la boca per respondre, però el seu avi es va avançar.

--Ha estat muntant a cavall, Sylv. A Dubai.

--Dubai? Com dimonis t'has permès Dubai? --la Sylvia va fer un sospir--.
Oh, tonta de mi, el Doctor t'hi va portar, oi?

La Donna va fer que si amb el cap.

--Sí. Ell ha pagat tot. Gairebé em vaig casar amb un ric xeic de petroli
i viure al seu harem, però saps què? Vaig creure que era més important
estar aquí avui. Amb vosaltres dos.

--Bé, això és tot un detall per la teva part, n'estic segura--va dir la
Sylvia--. Potser, si passejar-te per allà amb un ric i famós xeic no ha
estat gaire exigent per a tu, ens podries fer una tassa de tè?

--És clar--la Donna es va aixecar però no va ser bastant ràpid com per a
que la Sylvia no deixés anar unes quantes més.

--Encara te'n recordes d'on són les bosses de te? I la tetera?

I allò va ser tot. Era hora de deixar-ho sortir.

--Què t'he fet, mare? Vull dir, de veritat, quan va començar a anar tot
malament? Tot el que em deies era que sortís, que fes coses, que trobés
un treball, que visqués la meva vida. I ho he fet. I encara segueixo
sense ser bastant bona, oi? --es va tornar a asseure--. Segueixo sense
ser bastant bona, veritat? Decebia al pare igual que a tu?

--No t'atreveixis a parlar del teu pare d'aquesta manera--va cridar la
Sylvia, més fort del necessari.

--Ara, ara\ldots{} --va començar a dir en Wilf, però la Sylvia el va fer
callar.

--No, no, ja és hora de que la senyoreta Tonteries, tingui un parell de
veritats familiars--la Sylvia es va inclinar cap endavant, aixecant un
dit en l'aire--. El teu avi i jo estem molt preocupats, ho sabies? Te'n
vas i ens deixes sense dir-nos ni una sola paraula, baixes de la lluna
quan et va bé i en una estona te'n tornes a anar. No sé si ets viva o
morta. No sé si cada cop que sona el telèfon ets tu dient-me que ets a
Timbuctu, o si algú truca a la porta i serà un policia dient que han
trobat el teu cadàver al Tàmesi. Una carta arriba per a tu i la deixo al
replà, esperant que d'alguna manera això signifiqui que vindràs d'hora.
Però després d'un parell de setmanes, l'haig de deixar sobre el teu llit
perquè no funciona. No et porta a casa. Des que vas conèixer al paio
aquell del Doctor, t'has tornat una persona diferent.

La Donna va mirar a la seva mare amb un xoc silenciós. D'on havia vingut
tot allò?

--Per què dimonis has de suposar que he mort? És una bogeria.

--No és una bogeria, tampoc no és irraonable. És el que peso. Cada dia
que no sé de tu, ho crec més. Potser si fossis mare, potser si
tinguessis fills, fills estúpids i egoistes que no pensen per ells
mateixos, ho entendries.

La Sylvia tremolava.

La Donna tenia molta por. Ella, d'alguna manera, sense voler.-ho, sense
saber ben bé perquè, havia fet plorar a la seva mare! Per moltes raons
dolentes! Com si n'haguessin de bones! Es suposa que no has de fer
plorar a la teva mare\ldots{}

--No moriré, mare! Cap policia trucarà a la porta de casa i dirà que sóc
morta.

--Per què no? --va dir la Sylvia gairebé cridant, no d'una forma
enfadada, però amb llàgrimes caient per les galtes, no, queien per la
seva cara, com si fossin ratolins humits--. Per què no? És el que va
passar amb el teu pare!

El silenci que hi va haver de sobte era terriblement estremidor.

Llavors Donna va creuar la sala, abraçant a la seva mare ploranera,
agafant-la, aixafant-la, xiuxiuejant disculpes i paraules dolces,
dient-li que tot aniria bé, que era allà.

Però un pensament li va recórrer el cap. Demà, tornaria a anar-se'n. Amb
el Doctor. Perquè era el que volia.

Però tenia el dret a fer-ho? S'havia merescut el dret a anar-se'n de nou
si era allò el que pensava la seva mare?

Totes aquelles vegades ella i la Sylvia havien barallat, discutit i
cridat. Quan era adolescent (i francament, una major part dels seus
mimats vint anys), la Donna s'havia consolat amb un ``és la meva mare''.
Però la Donna no era ja aquella persona, i podia veure que la seva vídua
mare, any rere any, necessitava la seva filla més que abans.

I la Donna ara també plorava.

Plorar pel dolor de la seva mare, per la pèrdua del seu pare, recordant
aquella trucada a la porta. El policia allà dempeus.

--Se suposava que hauria d'haver mort aquí, als meus braços, amb la seva
família--deia la Sylvia--. No a una maleïda benzinera. Sol.

I llavors, amb el cronometratge exacte de quan les paraules ``perfecte''
i ``inconvenient'' van ser inventades, algú va trucar a la porta.

Sense di una paraula, en Wilf va a anar a respondre, i la Donna va
escoltar-lo dir:

--Ah, no és un bon moment--la Donna sabia, sense escoltar la resposta,
exactament qui era al llindar.

I també ho va fer la Sylvia.

Va mirar amb uns ulls inflats i vermells a la seva filla.

I, pel primer cop que la Donna pogués recordar, la Sylvia Noble va
acariciar la cara de la Donna, amb un dolç i suau moviment de pur amor
maternal.

--Posaré la tetera--i llavors el va cridar--. Endavant, Doctor.

Un moment després, la cara del Doctor va aparèixer vora la porta de la
saleta d'estar, amb les ulleres d'home llest posades i el cabell més
embogit que mai.

--Ei--els va dir a tots--. Oi que no coneixereu a la família Carnes, per
casualitat? Crec que tenen alienígenes a la família.

El comerç turístic a Moscatelli es basava principalment en oliveres,
tarongers, un bonic vinyar i una carrera anual de motos que començava a
tretze milles lluny a Florència i acabava a l'altre costat de les
muntanyes vora aquella petita ciutat poc visitada.

La gent que vivia a Moscatelli eren en gran part italians que havien
estat allà durant trenta o més generacions. Tothom coneixia a tothom i
era amistós, acollidors i alegres.

També era, a meitats de maig, el recipient d'un increïble bon temps, i
la Jayne Greene creia que treia els millors a la gent local. Era per
això que en Tonio es passava la major part del dia durant la excavació
portant res més que un parell de texans estrets tallats que no deixaven
gaire a la imaginació (i la Jeyne podia imaginar-ne bastant). El
Professor havia contractat en Tonio i la seva família per ajudar a
organitzar l'excavació feia una setmana. La Jayne i els seus dos
col·legues estudiants, en Sean i en Ben, havien accedit a acompanyar al
Professor a passar-hi l'estiu perquè els donaria bones notes a les
assignatures., seria una aventura viatjar a una bonica part d'Itàlia i
era genial tenir una forma de posar-se morena.

--Ho tinc! --va cridar en Sean, emocionat.

--Per quant? --va preguntar en Ben, filtrant terra un parell de metres
allunyat d'on descansava el portàtil vora la tenda de campanya amb el
menjar.

--Per setanta-vuit euros.

--Seixanta i escaig lliures, no està malament--en Ben va fer que si amb
el cap--. Molt ben fet.

--Adoro massa l'eBay--en Sean va somriure a la Jayne--. Bé per mi!

--Era el pot egipci?

En Sean la va mirar i va fer que no amb el cap, lentament.

--L'espasa de l'edat de ferro, llavors?

Va continuar fent que no.

La Jayne va deixar les seves eines i es va apropar al portàtil per mirar
en què havia gastat seixanta lliures en Sean.

--Això?

--Això.

--És una joguina.

--Per suposat que és una joguina--va cridar en Ben mentre en Tonio
deixava caure més terra al seu colador--. És que hi ha res més que
compri en Ben per eBay?

La Jayne no podia entendre-ho.

--Vols dir que et gastes tots aquests diners, et passes set dies
frustrat mirant la subhasta per una joguina produïda en cadena?

--Una figureta d'acció--la va corregir en Sean--. D'edició limitada.
Només n'han produït cinc cents, i d'això en fa vuit anys. És un treball
de pintura diferent, ho veus? Ella porta el seu vestit vermell del
Període fosc en lloc del seu verd tradicional.

La Jayne va observar en Sean.

--Ets un adult. Ets un home crescut emocionant-se per una joguina de
plàstic. Una figuera per a nens. Una\ldots{}

--No diguis ``nina'' --va murmurar en Ben per a si mateix.

--\ldots{} una nina?

En Sean va tancar d'un cop el portàtil.

--Els meus diners, la meva elecció. Tu t'emociones per fermalls romans i
~ceràmiques.

--I tu també!

--Sí, perquè es el meu treball. És el que faig aquí i a la universitat.
Però en el meu temps lliure, tinc altres hobbies. Tinc\ldots{}

--No diguis ``una vida'' --va tornar a murmurar en Ben per a si mateix.

--\ldots{} una vida--va acabar en Sean--. Hauries d'intentar trobar-te
una per a tu mateixa abans de criticar a la resta.

La Jayne va observar a en Sean, llavors a en Bean, que s'assegurava de
no mirar a ningú als ulls i va començar a fer moure el dit sense cap
sentit per entre la pols de la terra, en un intent de fer com si
estigués distret amb alguna cosa.

La tensió va ser trencada pel petit professor Rossi, sortint d'entre les
tendes venint d'un viatge a la ciutat buscant llet i bosses de te.

--Bé, bé, os podia escoltar des de la carretera. Què està passant?

--Res--va grunyir en Sean--. Ho sento, professor.

En Rossi va fer que no amb el cap, rascant la cicatriu que feia una
petita marca a la seva galta. A la universitat, tothom feia broma dient
que era una cicatriu de guerra que havia aconseguit barallant-se amb una
dona que estimava, però un dia algú va descobrir la veritat: feia deu
anys s'havia tallat a un accident de cotxe on havia mort la seva dona.
Tothom hi va perdre l'interès imaginant coses romàntiques després
d'allò.

--Què vaig a fer amb vosaltres tres? Us porto aquí des de la universitat
durant les vacances de mitjans de trimestre, per a visitar a la família,
i us dono la oportunitat de millorar les vostres francament pobres notes
d'arqueologia. I tot el que feu és jugar amb la banda ampla, flirtejar
amb el pobre Tonio o avergonyir-lo, o beure massa vi de taronja. Sou
aquí per treballar, ho sabeu, veritat? Ser sociable és l'efecte positiu
però no l'essencial. El que és essencial, de tota manera, és el treball
en equip. Sean i Jayne, no `importa si no us podeu suportar, però
treballareu junts. Ben i Jayne, no m'importa si lluiteu entre vosaltres
per l'atenció d'en Tonio, però treballareu junts. Sean i Ben no
m'importa si us beveu l'un a l'altre sota les taules mentre al dia
següent estigueu frescos i capaços de treballar junts. Està tot entès?
No sóc els vostres pares però sóc l'home que us avaluarà a final de
trimestre, i faríeu bé en recordar que mantenir-me content és positiu--
en Rossi va posar uns cartrons de llet sobre la taula vora el
portàtil--. Així doncs, de qui és el torn de fer el te?

En Sean es va oferir mentre en Rossi agafava el portàtil.

--Amb sort, el Bursar ens haurà donat més fons per a nosaltres per a que
intentem i mirem aquells túnels entre les muntanyes a l'altre costat del
llac.

--Quant fa que té família aquí, Professor?

En Rossi es va encongir d'espatlles.

--Estic en procés de descobrir-ho a la biblioteca. De fet els meus
rebesavis paterns van ser els que es van anar a Ipswich, però sospito
que les seves arrels són aquí des del segle XXV.

En Ben es va apropar amb el seu colador.

--Així que estem buscant més que vasos italians del segle XXV, llavors!
Ja ho deia jo. Vinga, professor, quin és el gran secret?

--Ah--va somriure en Rossi--. Bé, veieu, en algun lloc per aquí un ducat
es va esvair. Una ciutat sencera amb el seu castell i tot té la seva
base per aquí, o als turons o a algun lloc a les rodalies del llac més
enllà dels horts de tarongers. Intento trobar les seves fronteres.

--Com ho sabrem? --va preguntar en Sean mentre la tetera bullia.

--Està als registres de la biblioteca--va dir en Tonio amb un bon però
marcat accent anglès.

La Jayne i en Ben van mirar en un xoc mut i un lleuger terror.

Ell va somriure.

--Oh, clar, vosaltres creieu que no entenia res d'anglès--va riure, un
profund riure greu.

I també ho va fer el professor Rossi.

--Bé, ara--va dir, això és divertit. Cap de vosaltres us havíeu adonat?

Van fer que no amb el cap, sense dir ni mitja paraula, mentre en Sean
s'afanyava amb el te, intentant no intercanviar cap mirada.

--Però això significa\ldots{} --va començar la Jayne.

--Tot el que hem dit\ldots{} --va afegir en Ben.

--Sobre tu\ldots{} --la Jayne de nou.

--Has escoltat\ldots{} has escoltat\ldots{} oh déu meu, mateu-me-- en
Ben va deixar el seu colador i es va asseure al terra.

En Tonio va despentinar el cabell fosc d'en Ben i li va picar l'ullet,
abans de mirar a la Jayne.

--Ho sento, tu perds.

--Per suposat que sí--va dir la Jayne--. Quan ha volgut la vida anar a
la manera de la Jayne Green?

El portàtil va fer un bip i, deixant als estudiants assimilant el te i
la confessió d'en Tonio, el professor Rossi va mirar els correus
electrònics sense respondre. Res del Bursar, però hi havia un missatge:

De: MADAM DELPHI

Per:
\href{mailto:Rossi@Tarminsteruni.ac.uk}{Rossi}\href{mailto:Rossi@Tarminsteruni.ac.uk}{@}\href{mailto:Rossi@Tarminsteruni.ac.uk}{Tarminsteruni.ac.uk}

Tema: SAN MARTINO

Professor Rossi,

Les meves felicitacions, ha redescobert el seu patrimoni, i està de fet
a San Martino, tal i com esperava. Faci clic a l'enllaç per a anar al
meu web per a més informació sobre aquest encisador regne italià i els
seus secrets.

El Professor va estar a punt de cridar als estudiants, però va creure
que seria millor comprovar si no era una enganyifa (tot i que com podia
saber ningú que cercaven San Martino?) No li havia dit a cap dels seus
estudiants el nom del regne). Així que va clicar a l'enllaç.

En lloc d'un nou web, la pantalla va ser omplerta a l'instant d'una bola
bategant d'una brillant llum blanca vorejada per vores morades i
espirals.

Llavors va allargar-hi la mà i la va mirar.

Petant al voltant dels seus dits hi havia vestigis dels batecs morats
d'energia, com petites guspires de pura energia elèctrica. Va girar la
mà, estudiant els petits batecs mentre van semblar esvair-se, absorbits
per la pell. Va ajuntar els dits, i llavors va tornar a mirar la
pantalla. Tornava a mostrar la captura d'en Sean a eBay de nou.

El Professor es va aixecar i va girar la cara cap als seus estudiants i
va aixecar els braços, amb les mans estirades.

--Ho hem fet--va respirar.

Distrets instantàniament pels seus propis problemes, els quatre joves
s'hi van apropar, la Jayne i en Sean donant-se les mans, emocionats pel
gest de la tornada, com si no estiguessin segurs de què celebraven.

Després d'un segon, van alliberar les mans del Professor i en Rossi va
agafar les d'en Ben i en Tonio. I ells van agafar les d'en Sean i la
Jayne, formant tots cinc un cercle.

A l'uníson, tots van aixecar les mans unides cap al cel, creant i petant
una electricitat morada al seu voltant.

La resta van seguir la direcció de la mirada del professor mentre mirava
cap al cel.

--Benvinguda de nou--va dir tranquil·lament.

El sopar va suavitzar les coses a la casa dels Noble.

La Sylvia va omplir en silenci els plats amb menjar. La Donna va passar
els plats en silenci des de la cuina a la taula de dinar. En Wilf va
omplir d'aigua en silenci els gots, tres d'iguals d'una benzinera i un
més gran amb la imatge de l'ànec Donald. Aquest era pel Doctor.

El Doctor hi seia allà, incòmode amb tota la domesticitat del moment i
bastant malalt per la incomoditat.

--Dubai? --va dir la Sylvia, asseient-se.

El Doctor va mirar a la Donna, què se suposava que havia de dir?

--Amb els cavalls--va afegir en Wilf, a l'ajut.

--Cavalls? --el Doctor era com un conill arraconat a la gàbia--.
Cavalls. Sí, coses precioses.

--El xeic de Dubai ens va hostatjar un parell de setmanes--va intercedir
la Donna--. Veritat?

La Sylvia va començar a menjar. Era alguna cosa semblant a uns macarrons
amb formatge va suposar el Doctor, però no estava gaire segur. Alguna
cosa semblant a aquella li havia intentat mossegar un dels dits del peu
a la costa de Kal-Durunt a la galaxia Keripedes.

Gentilment va endinsar-hi la forquilla.

--Ho sento si no és tan ric com estàs acostumat a Dubai, amb els cavalls
i els xeics--va dir la Sylvia--. Però no sabia que vindrieu.

--Oh, bé, no podia deixar que la Donna es perdés el dia d'avui-- va dir
el Doctor amb un somriure. Massa somrient. Era mal moment per a ser el
Doctor Somrient, millor ser el Doctor Callat.

--Creia que els Emirats els portaven emirs i no xeics--va dir la Sylvia,
servint-se més aigua--. Però què hauré de saber jo? Jo només m'assec
aquí cada dia, esperant a que la gent surti del nores, esperant el
sopar.

El Doctor va llençar una mirada a la Donna que volia dir ``ajuda'' però
ella va interpretar-la clarament com un ``està bé, ignoreu-me, ah i ara
aniria bé que comencessis una baralla amb la teva mare''. I així ho va
fer la Donna.

--Quin és el teu problema, mare? Molta gent mataria per tenir família al
seu voltant.

En Wilf va intentar intervenir-hi, però la Donna havia començat a parlar
i no pararia.

--Vull dir, la Mooky se'n va fora dues setmanes i els seus pares fan una
maleïda festa per celebrar el seu retorn. I tot el que va fer va ser
anar-se'n de compres a Glasgow. Jo me'n vaig a veure les gal·là\ldots{},
vull dir, veure el món, coses que mai hauria cregut poder veure, i tot
el que tinc són gemecs.

La Sylvia no va aixecar la vista del seu plat.

--Si, però ells probablement sabien on era la Mooky. Tot el que sé és
quan el teu avi es molesta de dir-me que té una postal teva. I mai em
deixa llegir-les, oh no.

La Donna començava a castigar en Wilf per allò quan va recordar que
aquelles postals eren enviades des d'un altre sistema solar.

--D'acord, mare. Començaré a enviar-te postals també, t'ho prometo.

--Oh, no és només això--va dir la Sylvia--. Ets tota la meva vida. El
teu pare s'ha anat, tu t'has anat, i jo em quedo aquí mentre haig de
cuidar a la tercera en discòrdia del teu pare.

La Donna va obrir la boca per parlar, però llavors la va tancar.
Llavors, mentre el comentari s'endinsava, va obrir la boca de nou, però
no va sortir cap so.

--Tercera en discòrdia? --li va preguntar el Doctor a en Wilf.

En Wilf va mirar de reüll a la Sylvia.

--És una amiga--va dir--. No em casaré amb ella.

--Espero que no--va dir la Sylvia--. La mare es remouria a la seva
tomba.

--Ah, així que és sobre això--en Wilf va fer un sospir--. Creus que
l'Eileen no ho aprovaria. Creus que d'alguna manera una pobra i boja
dona gran faria que l'Eileen s'entristís. Bé, t'equivoques. Ella era la
teva mare, però era la meva dona. La conec millor que tu.

El Doctor va recordar perquè no feia famílies.

--Molts bons aquests macarrons amb formatge, senyora Noble--va dir,
omplint la seva boca--. Mmm\ldots{}

--Són xampinyons ~gratinats--li va engegar.

--No són macarrons?

--Xampinyons.

--És molt\ldots{} molt\ldots{} formatjós. I\ldots{}

--Així que\ldots{} qui és aquesta senyora, avi? --va preguntar la Donna.

En Wilf va somriure.

--És una senyora astrònoma que conec, de Greenwich. Dóna una mà a
l'observatori de per aquí, bé, es va posar malalta i va deixar de
treballar. Vam parlar un parell de cops per telèfon, ens van trobar i
vam sopar junts. Creuries que he començat a sortir amb una cosina
embarassada casada i adolescent per la forma de la que la Sylvia en
parla d'ella.

El Doctor mirava a la Sylvia Noble, de totes les maneres. Veient què va
fer que es posés vermella quan va parlar en Wilf. Havia estat la paraula
``malalta''.

Va tornar a mirar al vell paracaidista.

--Per què va deixar l'Observatori llavors?

--Pregunta-li tu mateix--va dir la Sylvia--. Ella hi serà aquí en
qualsevol minut. Fins i tot al dia de Geoff, la meva illa et porta a tu
per aquí i ell la porta a ella.

I la Sylvia es va aixecar i se'n va anar a la cuina.

La Donna va fer un sospir i va anar rere la seva mare. En Wilf va
intentar seguir-les, però el Doctor li va agafar el braç.

--No sóc un expert, Wilfred, però crec que és millor deixar-les soles.

En Wilf va fer que si amb el cap.

--I la teva amiga?

--La Netty. Henrieta Goodhart--va somriure--. El nom li escau molt bé.
Però li van diagnosticar amb\ldots{} Té Alzheimer, Doctor. I no millora.

--No ho farà--va dir el Doctor, mentre algú trucava a la porta--. És
ella?

En Wilf va fer que si amb el cap i la va deixar passar.

Un moment després i el Doctor somreia a una visió excèntrica, encant i
humor que només algunes dones angleses de certa edat i modals podria
aconseguir tenir.

Vestia des del cap als peus de color marró: una faldilla fins als
genolls de pana, una brusa color canyella i una jaqueta de color
xocolata i portava una bossa color canyella. Al seu cap hi portava un
meravellós barret amb almenys mitja dotzena de plomes marrons de
diferents formes i mides. En Wilf li treia el seu abric fosc, i la Netty
li va oferir la mà al Doctor abans que en Wilf li tragués l'abric,
deixant-la amb una sola màniga, la que li donava al doctor encara seguia
posada.

--Doctor, és meravellós coneixe't. Hurria i visca, és un veritable
plaer.

--Senyora Goodhart.

--Senyoreta, si us plau. Millor encara, només Netty. Mai he estat casada
i, malgrat el que cregui la filla d'en Wilf, no tinc cap intenció de
casar-me mai.

En Wilf va acabar traient-li l'abric, i la Netty es va asseure a una
cadira, agafant un got d'aigua en una òbvia assajada maniobra.

--Mai me he casat--va continuar--. Em semblava una alarmant despesa de
temps. Visc a Greenwich, ja saps, una mica de senderisme per allà, però
la meva línia de taxis local em coneix i coneix les meves manies, així
que mai és un problema si m'oblido de portar diners a sobre. O a on
vaig.

El Doctor va agafar estima a aquella agradable dona a l'instant.

--No et pots fiar d'una línia de taxis, Netty--va dir, intentant fer una
frase abans de que ella continués.

--Ja han començat a discutir? És l'única cosa que fan. I és sobre mi. Jo
estava molt avergonyida quan van començar, ara no més ho faig com un
ritual, i en deu minuts la Sylvia torna al seu encant normal, donant-me
te i pastetes.

El Doctor va somriure.

--La Sylvia Noble? Encantadora? Aquestes paraules no acostumen a anar
juntes.

La mirada que li va llançar la Netty li va mostrar com n'havia estat,
d'equivocat al prejutjar la relació entre les dues dones.

--Oh, no em decebis, Doctor. No després de tot el que m'han dit de tu.
Aquesta dona d'aquí és una santa. Acaba de perdre al seu marit, té una
filla a la que arrossegues qui sap on i ha de tolerar que el meravellós
Wilfred, que pot estar igual de tossut i rondinaire com ella. De fet
hauria de ser-ho més. Ella m'agrada molt. I a més, sé que ella es
queixa, però és només la seva forma de deixar sortir el fum, és genial
amb la meva\ldots{} bé, ja saps--la Netty es va tocar el cap--. La meva
condició. Beneïda sigui, l'últim cap de setmana ella va portar en cotxe
a en Wilf fins a Charlton. Aparentment em van trobar al jardí del
darrere d'algú, intentant convèncer-los de que jo hi vivia allà quan
tenia sis anys!

--I ho feies?

--Déu del cel, no. Em vaig criar a Hampshire--va mirar a la seva bossa i
va treure una llibreta vermella de mida A5 i se la va ensenyar--. La
meva vida--va dir senzillament--. Així puc recordar les coses.

El Doctor la va mirar directament als ulls i va veure, breument, una
dona espantada però orgullosa. I li agradava encara més que abans.

--Sense aquest llibre, sense l'amor de la Sylvia Noble, no sóc res. Em
vaig deixar la bossa a una botiga al carrer major de Greenwich, així que
era al jardí, incapaç de saber qui era, d'on era. La Sylvia va trobar un
tiquet a la meva butxaca, va trobar la botiga, va anar a per la meva
bossa just on la vaig deixar i ho va arreglar tot amb la policia. Ella
em vol posar a una residència, saps? Els pamflets són al calaix vora la
cuina.

--De veritat?

--Sí. En Wilf no ho sap. Diu que ell abans voldria que m'hi quedés aquí
a viure. Aquest vellet bo, com si no tingués suficient amb una casa
difícilment podria suportar una altra. Però una residència amb
infermeres, on em cuidaran. Com de meravellós seria això?

La porta de la cuina es va obrir. La Donna i la Sylvia van entrar i la
Donna es va presentar d'inmediat.

Mentre la Sylvia posava la tetera, el Doctor la va aturar i es va posar
rere seu.

--Sap en Wilf tot el que fas per la seva amiga?

--Sap la Donna que fiques el nas als assumptes de la seva famía? --va
respondre la Sylvia.

--No sóc enemic teu, senyora Noble--va dir el Doctor.

La Sylvia es va girar i li va somriure. Era el somriure més sincer
possible.

--Pel bé de la meva filla, Doctor, et tolero a aquesta casa. Però això
és tot. Pel bé del meu pare, faré tot el que pugui per la Netty
Goodhart. No crec que sigui una dona egoista, Doctor. He treballat dur,
he construït una vida, mai he tingut massa diners, i he intentat donar a
la Donna una vida decent. Però llavors un dia, vaig perdre el meu marit.
La meva roca. I des de llavors he intentat fer el mateix que vam fer
ambdós, però amb una filla que en un minut no té treball, i a l'altre es
pot permetre ser a l'altre hemisferi, però no pot permetre's un segell,
i un pare ancià que sembla haver decidit que és hora de substituir la
meva mare d'un cop.

--Estàs segura de que no et preocupa que et substitueixin a tu? Imagino
que va deixar anar a la teva mare fa temps.

El silenci que va seguir a la bufetada a la cara que va rebre va semblar
durar hores, però probablement només van ser uns segons.

--No volia dir això\ldots{} --va començar el Doctor--. Jo només em
preguntava\ldots{}

La Sylvia el va ignorar.

--Pare--va dir--. Per què no portes al Doctor a l'hort? La Donna i jo
ens posarem al dia amb la Netty mentre vosaltres dos us passeu una hora
o més allà a dalt, d'acord?

En Wilf va entendre la pista i tot el que va fer va ser agafar al Doctor
de la cuina mentre les dones observaven.

L'última cosa que el Doctor va sentir va ser:

--Te per tothom--de la Sylvia, abans que el Wilf els hagués tret al fred
aire nocturn.

--L'hort és per aquí--va dir l'ancià.

En Babis Takis va carregar la bala de palla més gran a la part del
darrere del vagó i es va aturar a descansar. Començava a fer-se gran i
allò era treball del Nikos.

Però el Nikos no hi era allà, i probablement fos al voltant de la
granja, amb la noia Spiros. Molt típic. Hi havia treball per fer: havien
de tenir fet aquell encàrrec cap a Faliraki, on hi havia molt treball de
construcció: hotels, apartaments, centres comercials, el sencer
Dodecanès s'aprofitaria d'allò pel continu turisme que portaria.

En Babis va cridar el nom del Nikos un parell de cops mentre carregava
més bales de palla. Va mirar el seu rellotge. El portaria una hora o
així de conduir a través de la illa cap a Pentaloudes, on agafarien en
Kris, abans d'anar cap a Lindos i per la costa de Faliraki. Deixarien la
palla al dipòsit, i llavors cap a la taverna de l'Erik per la nit.

Encara no hi havia cap senyal del Nikos.

Amb un sospir, en Babis es va apartar del vagó.

--Sóc un home gran, Nikos--va cridar--. Vaig lluitar a la guerra, ja
saps, per a que la gent com tu pogués ser independent i tingués luxes.
Només per un cop, t'agrairia si m'ajudessis.

Havia arribat als voltants de la granja, quan va escoltar el soroll
d'algú dins els estables: un curt crit de dona de sorpresa, gairebé de
por. Segurament d'alarma.

En Babis va ser dins l'estable en un segon.

En Nikos era al terra, agafant-se el cap amb les mans, callat però
clarament marejat.

Dempeus vora ell, amb una pala a les mans hi havia la jova noia maca, a
qui Babis va reconéixer com la Katarina Spiros.

--Estàs bé, noi? --va preguntar en Babis, agafant la pala.

La Katarina es va girar per mirar l'avi d'en Nikos, i ell es va adonar
de com n'estava d'espantada.

--Què ha passat? --va preguntar en Babis.

Estava sorprès quan la Katarina va explicar-ho.

--Va agafar una trucada\ldots{}

Ella senyalava a en Nikos Takis, que començava a aixecar-se, deixant el
telèfon mòbil al terra vora els seus peus. Mirava al seu avi, fent un
pas enrere.

En Babis Takis va lluitar els últims dies de la guerra com un jove, fent
fora els nazis de Creta i mantenint Grècia per als grecs. S'havia
enfrontat a la fúria i al dolor dels seus familiars, que l'havien odiat
i lentament repudiat per enamorar-se, casar-se i tenir fills amb una de
les tan odiades italianes que havien ocupat les illes gregues des feia
set cents anys abans de ser enviats a la guerra. Havia estat algun temps
a la presó per una baralla a Diagoras, i un cop s'havia hagut
d'enfrontar el ressentit fill d'un alemany que havia cordat a una
granada activada durant l'any 1944.

Però res no l'havia espantat tant com la mirada que el seu net li va fer
llavors.

En Nikos ja no hi era. Aquell net despreocupat, divertit i llest que
havia criat després de la mort del seu pare ja no hi era.

En Babis no sabia com ho sabia. No sabia com havia passat. Però mai
havia estat tan segur de res en la seva vida.

En seguia estant segur quan un flash de llum morada, més calent que el
cor del sol, va extingir la seva existència en menys temps que va portar
a la Katarina Sprios agafar aire per a respirar.

Un segon més tatrd, un grapat de cendres van caure al terra on havia
estat la noia jova.

I en Nikos Takis va aixecar els braços cap al cel, amb una electricitat
morada brillant al voltant dels seus dits, mentre movia cap enrere el
seu cap.

--Benvinguda de nou--va cridar, triomfant.

En Donnie i la Portia eren de lluna de mel. Era la primera d'en Donnie,
la segona de la Portia, però ambdós ho gaudien enormement.

El fill d'en Donnie havia estat el seu fadrí. El seu net havia estat el
nen de les arres. La neta de la Portia havia estat la nena de les flors.
Havien tingut dues cerimònies, una completament jueva i una senzilla
cristiana per a reflectir ambdues fes escollides. La Portia sempre havia
tingut una fe jueva, mentre que la família d'en Donnie la havia
abandonat en unes setmanes després d'arribar a l'illa d'Ellis feia un
segle i escaig.

Havien vençut als fats, un ensurt de càncer per a en Donnie, algunes
dificultats dels familiars més tradicionals de la Portia i la mort del
seu gat de vuit anys, el senyor Smokey, una setmana abans de les
cerimònies. Després de coneixe's durant quinze anys, flirtejant els
últims sis, finalment s'havien unit per sempre.

I allà eren, al jeep d'en Donnie, havent recorregut la carretera 8,
passant a través de Danbury, abans d'abandonar l'autopista i el camp de
Connecticut cap a la seva lluna de mel.

Havien agafat una bonica casa colonial a les afores d'Olivertown, a
tretze milles de Danbury. La casa pertanyia a un dels clients de Portia
(era una passeja-gossos, donant voltes tres cops al dia al voltant de
Central Park amb una gran varietat de gossos). Els Carpenters eren a la
radio, dient-li al món com els agradaria ensenyar-los al món a cantar, i
la parella feliç els acompanyaven cantant.

Havien passat per Abba, Dr Hook, els Medicine Show, Jo Stafford i ara
movien el cap amb el compàs de la Helen Reddy cantant sobre com de bo
era estar boig.

--No one asks you to explain--van cantar a l'uníson mentre aparcaven
fora de la casa que havien llogat.

La Portia va mirar al seu nou marit.

--Bé, senyor DiCotta ja hi som.

--És clar que sí, senyora DiCotta--en Donnie li va picar l'ullet--. T'hi
acostumaràs algun cop?

--Mai ho faré, ho reconec--va riure--. Però m'agrada--es va inclinar a
través del cotxe i li va fer un petó a en Donnie mentre ell apagava el
motor.

I la ràdio va seguir funcionant.

Mentre es separaven, ambdós van mirar la caixa de l'estèreo.

--Això no ha de ser bo, Donnie--va dir la Portia DiCotta--. Ha d'haver
un tall a algun lloc.

Ell va fer que sí amb el cap.

--Vaja, serà millor que ho arregli ara, estimada. Sinó no tindrem
energia suficient per a poder portar-te a aquell restaurant de New
Preston. El menjar és excel·lent, la hospitalitat de primera classe i la
vista és per a morir-se. Pots mirar des de dalt de tot el llac Waramaug
i és realment romàntic.

La Portia va fer que sí amb el cap.

--Mentre arregles el cotxe, jo vaig fent un café.

En Donnie va joguinejar per sota el tauler de control buscant un cable
despenjat. Hi va haver una petita guspira de electricitat i la ràdio va
callar.

--Molt ben fet--va somriure la Portia--. Ara pots ajudar-me portant les
maletes.

En Donnie DiCotta no va dir res. Només va mantenir la seva mà sota del
tauler de controls del jeep, mirant cap endavant.

--Donnie?

Res.

La Portia va allargar la mà per tocar la seva espatlla, i ell va girar
la cara per a mirar-la. La Portia va veure els seus ulls: no eren
d'aquell bonic blau dels que s'havia enamorat. S'havien reemplaçat per
dos sòlids orbes de llum morada brillant, minúscules llengües de
guspires elèctriques dels seus lacrimals.

No va poder dir una paraula perquè ell li havia agafat el cap i li va
fer un petó a la boca. Sencer. Dur. Però no del tot apassionat.

Després d'un segon o dos, es van separar.

I ara els ulls de la Portia DiCotta brillaven amb la mateixa inquietant
energia morada.

Sense paraules, van sortir del jeep i van caminar pel porxo, estudiant
el fosc cel de la nit per a sobre d'ells, fins que en Donnie va
assenyalar cap a dalt i a la dreta, cap una brillant estrella, ell havia
estat un expert en aquelles coses, hauria sabut que no s'havien vist per
ulls humans durant molts segles.

Ell i la seva dona es van agafar les mans i van mirar cap a l'estrella.

--Benvinguda de nou--van respirar tot junts.

El Doctor mirava cap Londres.

--Ja puc veure per què t'agrada ser-hi aquí a dalt, Wilf--li va dir a
l'home que es removia al seu costat, buscant un segon tamboret per a que
s'hi assegués--. És terriblement\ldots{} tranquil.

En Wilf Mott va fer que sí amb el cap.

--He estat venint aquí durant uns quants anys. Acostumava a mirar el cel
de nit quan era a l'exèrcit. Els altres nois creien que estava boig,
però saps què, Doctor? Mai m'he perdut. Ni un sol cop.

El Doctor va somriure a l'home gran i va asseure's al seient que li va
donar.

--Gràcies.

En Wilf va asseure's darrere d'ell i li va donar una tassa de te d'un
termos. El Doctor va beure'n agraït.

--Acostumava a abocar-hi una mica del millor de Mr Daniels--va dir en
Wilf--. Però la seva Senyoria ho va trobar i allò va ser el final.
Maleïts doctors que em van dir que no en begués.

--Un grup molt dolents, aquests doctors--va riure el Doctor--. Però a
vegades saben el que és millor, sense importar com d'impopulars es
tornen.

--Subtil--en Wilf va fer que sí amb el cap--. La Sylvia es passarà.
Probablemente.

--De veritat?

--No ho crec--va rugir en Wilf--. Ni de broma, amic.

El Doctor va somriure de nou.

--Llavors no tens cap problema amb mi?

--Oh, fas que la meva Donna sigui feliç. Sigues fent això i per mi
estarà genial. Fes res que la molesti, i em sentiràs, encara que siguis
a Mart.

El Doctor el va mirar, sorprés.

--Quant t'ha dit de tot això?

--Tot. Des del principi--en Wilf a mirar al telescopi--. Jo era aquí,
mirant cap al cel quan ella em va parlar de tu. No me la vaig creure al
principi.

--Bé, ningú no et culpa d'això.

--La veritat és que no crec que entenia el que em deia. Llavors vaig
veure després d'allò de la grassa, volant a través del cel, amb la Donna
saludant-me amb la mà i em vaig adonar de que tot era veritat. M'ha anat
posant al dia sempre que ha pogut. Amb postals, correus electrònics. Un
estrany regal. Encara no sé què fer amb la medalla verroniana. Ni
exactament què es una medalla verroniana!

El Doctor va somriure.

--Una meravellosa raça, els verronians. Tenen un meravellós cos aeri.
Bastant inútil, ja que no han lluitat en una guerra des fa un mil·lenni,
però el seu cos aeri és el seu millor assoliment. És una miqueta com
enviar una condecoració militar al servei distingit a un nen de set
anys--el Doctor es va encongir d'espatlles--. Espera un minuto, quan
t'ho va enviar? D'on ho va aconseguir? I com dimonis va aconseguir que
te la fessin arribar?

En Wilf gairebé va recular del torrent de preguntes del Doctor.

--No tinc ni la menor idea. Per què? No la necessites, oi? Vull dir, la
va robar? La Donna no ha pispat res des d'aquells caramels de Woolies
quan tenia vuit anys. Li vam fer que els retornés i demanés perdó i tot.

--No, no, dubto que l'hagi pispat. Els verronians són molt generosos.
Només em preguntava quan va conèixer un, de verronià.

--No m'agrada pensar que no vigiles prou bé la meva petita, Doctor--va
dir en Wilf, aixecant una cella.

--A Hèlios 5--va dir el Doctor--. Ha hagut de ser allà. O a Ylum.
M'agrada Ylum i a la Donna també li va agradar, és molt cosmopolita. O
potser al Moulin Très Rouge. Hi havia un mercat allà. O suposo que va
ser aquell dia quan\ldots{}

--De qualsevol manera--li va tallar en Wilf--, hem vingut aquí per una
raó.

--Una raó que no sigui comprovar quines són les intencions vers la teva
néta? I per donar una oportunitat per a que la meva galta esquerra es
refredi.

En Wilf va riure.

--Oh, això és l'estil de la Sylvia, no la meva. Sé que ets honorable.
També sé que la Donna es pot cuidar a si mateix en qualsevol cas.
Qualsevol cosa deshonrosa per part teva i mai tornaries a sentir parlar
de tot això. Literalment.

El Doctor va pensar en els ``Ei''s i va decidir que sí, en Wilf coneixia
molt bé la seva néta.

En Wilf va ajustar el seu telescopi.

--Mira per aquí.

El Doctor ho va fer, i va veure una estrella. Un petit foradet de llum,
brillant prou tan poc que semblava anar-se a esvair.

--T'agrada? La 7432MOTT--va dir en Wilf, orgullós.

--Ho sento--va dir el Doctor, mirant-lo de nou--. Li van posar el teu
nom?

--Jo la he descobert. M'he unit a un nòdul. No molt desprès, vaig
descobrir aquesta estrella. La RSP em farà un sopar demà a la nit.

--No dé nes dels nòduls--va dir el Doctor.

--Ah, bé un nòdul és\ldots{} --va començar el Wilf, però el Doctor el va
frenar amb la mà.

--Faig brom. Sé què es un nòdul. I estic molt impressionat, Wilf, de que
hagi una estrella amb el teu nom i encara gaudeixo més amb que la Real
Societat Planetària es posin als teus peus. I segur que t'has comprat
una corbata nova i tot. I ho sento per fer malbé la teva alegria, però
haig d'assenyalar que hi ha una altra nova estrella, just a sota. És
increïblement brillant, just a l'esquerra de l'espassa d'Orió.

En Wilf es va inclinar cap endavant i va apartar al Doctor.

--Aquí?

--No, allà.

--Oh, allà. Sí, a tots ens va agradar aquesta. És molt bonica.

--Sí, molt bonica. Una molt bonica i nova estrella brillant blancament
en una constel·lació que no hauria de ser allà. La cosa és que, les
estrelles d'aquesta magnitud i brillantor no apareixen per cap raó bona.

--Brillantor? És cap terme tècnic?

El Doctor li va llençar una mirada a en Wilf.

--És prou bona per a mi.

--Només sé que tothom està parlant d'aquestes estrelles com els Cossos
del Caos.

El Doctor hi va pensar durant un o dos segons, llavors es va encongir
d'espatlles.

--Així es diuen? És quelcom nou per a mi.

--És com li diuen als diaris.

El Doctor va fer que sí.

--Ah, bé, si els diaris ho diuen, ha de ser veritat, perquè qui dimonis
hauria de discutir amb els diaris? --va mirar cap a la cosa brillant al
cel--. Els cossos del caos. Un bon nom descriptiu, per cert. Amb aquesta
han encertat--el Doctor va posar un braç al voltant de les espatlles
d'en Wilf--. Però saps què? A qui l'importa? Tens una estrella, una
estrella molt menys preocupant, amb el teu nom i n'estic molt orgullós.

--Gràcies. Estic orgullós que ho diguis perquè has resolt un problema
meu.

--Quin problema és?

--Necessitava algú que em portés a Vauxhall!

--Per què?

--Per al sopar.

--Amb qui?

--La Societat.

--Quan?

--Demà a la nit.

--Com hi anirem?

--Amb el TARDIS?

--Sí, perquè estic segur de que això passarà\ldots{}

--D'acord, d'acord, anirem amb metro. La Sylvia creu que no sóc capaç
d'anar-hi tot sol. Vull dir, veu bé que estigui assegut en un fred i
humit hort cada nit, que fa malbé la meva\ldots{}

--Així que la Sylvia no vol que siguis fora molt tard, oi?

--Vull dir, no em perdré, Doctor, però s'ha posat molt protectora i
pesada aquests últims dies. Des que va perdre en Geoff. I amb la Donna
fora tant de temps\ldots{} I ara que la Netty\ldots{}

--Saps què, Wilfred Mott, em sento molt honrat d'acompanyar-te al sopar
a Vauxhall. Anirem amb taxi, com de rics és això? No he anat a un sopar
de la RSP des que en Bernard i la Paula m'hi van portar el 1969 per
veure els aterraments a la Lluna. Segueixen fent aquell pudding de
xocolata tan bo?

--No ho sé. No hi he estat abans.

--Llavors demà a la nit, Wilfredd Mott de Chiswick, suposant que el menú
no hagi canviat en quaranta anys, i si la Real Societat Planetària és
qui crec qui és, dinarem coses d'allò més bones.

El Doctor va mirar a la preocupant nova estrella brillant que no era la
7432MOTT un últim cop a través del telescopi.

--Un bon Cos del Caos.

--Tot i així, és bell. --va murmurar en Wilf.

I ambdós homes van sentir un calfred de sobte. Com si algú s'hagués
passejat per les seves tombes.

O per la tomba de tot el planeta sencer.

--Bell Caos--va dir el Doctor, amb calma.
