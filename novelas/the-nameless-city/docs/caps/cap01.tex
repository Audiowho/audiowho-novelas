\chapter*{Capítulo Uno}
\addcontentsline{toc}{chapter}{Capítulo Uno}

Londres 1968.

Un grito, agudo, aterrorizado.

El sonido estuvo a punto de perderse entre el ruido del
ajetreado tráfico de un sábado por la tarde y la bulliciosa multitud de
Charing Cross Road. Unas pocas personas levantaron la vista y miraron a
su alrededor. Al no ver nada extraño, siguieron su camino.

Un segundo grito resonó, casi completamente ahogado por el
estruendo de las bocinas de los coches.

Sólo un joven alto y moreno que estaba en el exterior de una
cochambrosa librería de libros antiguos continuó mirando, la cabeza
inclinada hacia un lado, los ojos medio cerrados, escuchando
atentamente. Ninguno de los transeuntes le prestó atención y, dado que
se trataba de Londres y la ciudad estaba inundada de las últimas modas,
nadie ni tan siquiera parpadeó ante el gran tamaño de su jersey de
cuello alto o ante el hecho de que llevara un Kilt escocés rojo a
cuadros, completado con un sporran.

El joven usaba un truco que su padre le había enseñado cuando
cazaban urogallos en las Highlands. Se concentraba deliberadamente en
los sonidos - primero los coches y autobuses; lo siguiente, el ruido de
la calle, el murmullo monótono de los gritos, el zumbido de las risas -
y entonces los desconectaba. Esperó algo fuera de lo común, algo
raro. Algo como\ldots{}

El golpe del cuero contra la piedra.

Había venido de su espalda.

Moviéndose ahora rápidamente, siguió el sonido. Le llevó a la
boca de un callejón adoquinado. Miró: estaba vacío. Sin embargo, sabía
con absoluta certeza que este estrecho tubo de piedra habría llevado
cualquier sonido más allá hasta la calle. Sumergiéndose en el callejón,
parpadeó, permitiendo que sus ojos se acostumbrarana la oscuridad,
antes de lanzarse hacia delante. El callejón se curvaba ligeramente
hacia la izquierda y cuando dobló la esquina descubrió la fuente del
ruido.

Un hombre barbudo de pelo gris estaba tendido sobre las sucias
piedras, rodeado de una dispersión de libros antiguos encuadernados en
cuero. Un enorme matón de pelo grasiento se agazapaba sobre la figura
examinando una desvencijada mochila, sacando libros y tirándolos a un
lado.

---Por favor\ldots{} por favor ten cuidado, ---gimió el anciano
cuando cada volumen revestido de cuero caía al suelo con un golpe
característico.

---¿Dónde está el dinero? ---gruñó el enorme matón---. ¿Dónde
está la recaudación de la tienda?

---No hay nada, ---dijo rápidamente el anciano---. Vendemos
libros antiguos. Pero algunos días no vendemos nada\ldots{}

---No te creo. Vacíate los bolsillos.

---No, ---dijo el anciano desafiante.

---¡Sí! ---el ladrón sonrió, los finos labios se despegaron de
unos dientes amarillentos.

La ira brilló en los ojos del joven escocés. Sabía que no
debería involucrarse. Le habían confiado una misión crítica y había
prometido no retrasarse, pero también había sido educado con un estricto
código de honor, que incluía la protección de los débiles y el respeto
hacia los ancianos. Manteniéndose cerca de las paredes, se precipitó
hacia adelante, los zapatos gastados de suelas blandas de cuero no
hacían ruido sobre los adoquines.

---Dije que te vaciases los bolsillos. El matón tiró la mochila
a un lado y se inclinó sobre el hombre que yacía en el suelo.

De repente, un grito cortó el aire: un gruñido gutural que
sorprendió al ladrón que se quedó inmóvil. Alcanzó a ver una sombra por
el rabillo del ojo antes de que un tremendo golpe en su costado le
mandara a estrellarsecontra la pared del callejón. Su cabeza golpeó con
fuerzalas viejas piedras, y puntos fríos de luz rojos y azules bailaron
ante sus ojos mientras caía de rodillas. El ladrón parpadeó, mirando a
una figura con una falda roja --- no, un Kilt --- y sacudió la cabeza
para enfocar la vista. Poniéndose en pie, lanzó un golpe vacilante y
entonces algo le golpeó en el centro del pecho, y cayó sentado, la
columna vertebral retorcida sobre los adoquines.

---Si sabes lo que te conviene, te marcharás corriendo. Y no
mirarás atrás. Aunque el escocés había hablado enun susurro, la amenaza
era clara.

Doblándose en dos, con los brazos alrededor del pecho magullado,
el ladrón retrocedió, dio media vuelta y echó a correr.

El escocés se arrodilló, le tendió la mano al anciano y con
cuidado le ayudó a sentarse.

---¿Está herido?

---Sólo mi orgullo\ldots{} y miis pantalones. ---el hombre del
pelo gris luchó lentamente por ponerse de pie, apartándose el pelo de su
gran frente---. Y mis pobres libros. Se dispuso a recogerlos, pero el
escocés ya se había lanzado a ello, recogiendo los volúmenes dispersos.

---Eres muy valiente, ---dijo el hombre, su voz profunda resonó
en las paredes del callejón.

---Bueno, no podía dejarlo pasar, ¿no?

---Sí, podrías. Otros lo hacen. ---el anciano sacó su mano
enguantada---. Gracias, muchas gracias. Sonrió a través de una cuidada
barba de chivo con motas grises, sus ojos oscuros y curiosos bajo unas
tupidas cejas.

--- Soy el profesor Thascalos.

---Soy Jamie, Jamie McCrimmon.

---Escocés. Me pareció reconocer un grito de guerra gaélico.
Creag un Tuire. ¿Qué es eso --- ¿la Roca del Jabalí?

Jaime le entregó los libros.

---¿Quiere decir que el Kilt no era una pista? ---preguntó con
una sonrisa.

El anciano sonrió. ---Las modas de hoy en día---. Se encogió de
hombros. ---¿Quién sabe lo que lleva la gente joven?

Jaime cogió la mochila y la mantuvo abierta mientras el profesor
sacudía cuidadosamente cada libro y lo devolvía a la bolsa. Algunas de
las encuadernaciones de cuero habían sido rozadas y desgarradas cuando
habían golpeado los adoquines y una cubierta se había desprendido por
completo.

---¿Estás en el ejército? ---preguntó el profesor.

Jaime negó con la cabeza. ---En realidad no.

---Reaccionaste como un soldado, ---dijo el Profesor Thascalos
---. Un grito en el último minuto para desorientar al enemigo, seguido
por un ataque abrumador. Eso sólo viene con la experiencia. Has estado
en batalla.

El joven escocés asintió. ---Sí, bueno, fue hace mucho tiempo,
---dijo con un acento de repente pronunciado---. Y no terminó bien. No
iba a decirle al profesor que suúltima batalla había tenido lugar hace
más de doscientos veinte años. Le entregó el último libro al anciano.
---¿Hay muchos daños?

---Puedo re-encuadernar los que están peor. No debería haberme
metido por el callejón, pero estaba tomando un atajo a mi tienda.
Vendolibros en Charing Cross Road,---añadió, y levantó la bolsa de
libros---. Pero probablemente ya habrás adivinado eso.

---Sí, lo hice ---sonrió Jaime---. ¿Va a denunciar esto a la
policía?

---Por supuesto.

---Si ya está bien, seguiré mi camino.

El profesor metió la mano en su bolsillo y sacóuna cartera.
---Ahora, permíteme darte algo --- Se detuvo al ver la expresión del
rostro de Jaime. ---Dinero no entonces, pero\ldots{} Hurgando en la
bolsa, encontró un pequeño libro, envuelto en un pañuelo de seda negro.

---No quiero pago\ldots{}

---No es un pago - es un regalo,---dijo el librero---. Un
agradecimiento. Le entregó el paquete a Jaime, que lo cogió y le dió la
vuelta con sus grandes manos, doblando hacia atrás la seda para trazar
un contorno de curling en el relieve de la tapa de cuero oscuro del
libro.

---Se ve que es antiguo.

---Lo es. Es uno de los libros más antiguos que poseo.

Jaime lo abrió. Las gruesas páginas estaban cubiertas de bloques
de letras negras en un lenguaje que pensó podría ser alemán.

---Debe ser muy valioso.

---Lo es, ---repitió el profesor---, pero quiero que lo tengas.
Hoy me has salvado la vida, joven, ---dijo con voz ronca---. Es lo menos
que puedo darte.

---No puedo leer la escritura.

---Hay pocos que puedan. Pero quedatelo. Insisto. Siempre puedes
dárselo como regalo a alguien que creas que pueda apreciarlo.

De pronto extendió la mano y se la estrechó a Jaime. ---Yate he
retrasado y abusado demasiado de tu tiempo. Gracias. Eres un orgullo
para tu clan. El ancianoretrocedió y se puso la mochila al hombro, se
volvió y se dirigió hacia el callejón. Alzó su mano enguantada y su voz
resonó en las piedras.

---Cuidaté, Jamie McCrimmon ---llamó---. Disfruta de tu libro. Y
entonces dobló la esquina y desapareció.

Jamie miró el libro negro, frotando sus pulgares sobre su
superficie. El cuero era grasiento y ligeramente húmedo al tacto. Supuso
que se había caído en un charco. Acercándoselo a la nariz aspiró
lentamente. De sus páginas pensó que olía el débil olor a pescado y aire
del mar. Encogiéndose de hombros, lo envolvió de nuevo en su seda y se
lo metió en el cinto mientras se alejaba. Quizá al Doctor le gustaría.

El profesor Thascalos se detuvo en el extremo del callejón.
Podía oir los pasos de Jamie desvanecerse en la dirección opuesta.
Volvió la cabeza para mirar a una enorme figura que estaba al acecho en
las sombras. El ladrón de pelo grasiento dió un paso adelante, su boca
con una amplia sonrisa de dientes separados.

---Lo has hecho bien, ---dijo el profesor en voz baja. Sacó un
fajo de billetes de bolsillo interior de su abrigo ---. Acordamos
cincuenta, pero aquí tienes sesenta. Despegó seis crujientes billetes de
diez libras y se los entregó. ---Un plus por haber sido golpeado.

El hombre miró el grueso fajo de billetes y se lamió los labios.

---Ahora estás pensando cosas tontas, ---dijo el profesor de
nuevo en voz baja, su cara se convirtió en una implacable máscara---.
Cosas tontas y peligrosas, --- añadió con frialdad.

El matón miró a los ojos oscuros del profesor, y lo que vió allí
le hizo dar un paso atrás alarmado.

---Sí\ldots{} sí, cincuenta. Muy generoso. Gracias.

---Buen chico. Ahora, vete. ---el profesor arrojó la bolsa de
libros al hombre grande---. Ydeshazte de estos por mi.

---Pensaba que eran valiosos.

---Sólo uno, ---murmuró el profesor para sí mismo, volviendo la
mirada hacia el callejón ---. Y ese es de un valorincalculable.

Entrando en las sombras, el profesor vioque el ladrón se
deslizaba desapercibido entre la multitud de gente que paseaba. Entonces
sacó un cilindro de metal delgado del bolsillo, lo giró en sentido
opuesto al de las agujas del reloj y loacercó a sus delgados labios.

---Está hecho, ---dijo en una lengua que no se había oído en la
Tierra desde la caída de la Atlantida ---. He completado mi parte del
trato. Confío en que, cuando llegue el momento, usted cumplirá con la
suya.

Un hilo de una tenue música etérea quedó suspendidoen el aire.

El profesor rompió el cilindro cerrado y se alejó, con una
extraña sonrisa en los labios.