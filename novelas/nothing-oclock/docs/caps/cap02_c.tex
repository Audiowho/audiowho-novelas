\chapter*{Capítulo Dos} \addcontentsline{toc}{chapter}{Capítulo Dos}

El miércoles, Polly Browning, de once años de edad, asomó la cabeza por la puerta del despacho de su padre.

---Papá, hay un hombre en la puerta principal con una máscara de conejo que dice que quiere comprar la casa.

---No seas tonta, Polly ---el señor Browning estaba sentado en la esquina de la habitación que le gustaba llamar su oficina, y que el agente inmobiliario había enumerado con optimismo como un tercer dormitorio, aunque era apenas lo

suficientemente grande para un archivador y una mesa, sobre la que descansaba un nuevo ordenador Amstrad. El Sr. Browning estaba introduciendo cuidadosamente los números de una pila de recibos en el equipo, y haciendo una

mueca. Cada media hora debía salvar el trabajo que había hecho hasta ese momento, y el equipo hacía un chirrido durante unos minutos, ya que guardaba todo en un disquete.

---No estoy haciendo el tonto. Dice que te dará setecientas cincuenta mil libras por ella.

---Ahora estás realmente haciendo el tonto. Está a la venta por cincuenta mil libras ---y tendría suerte de conseguirlas en el mercado actual, pensó, pero no lo dijo. Era el verano de 1984, y el Sr. Browning desesperaba por encontrar un comprador para la pequeña casa al final de Claversham Row.

Polly asintió pensativa.

---Creo que deberías ir a hablar con él.

El Sr. Browning se encogió de hombros. De todos modos, tenía que guardar el trabajo que había hecho hasta el momento. A medida que el equipo hizo su sonido gruñon, el señor Browning bajó las escaleras.

Polly, que había planeado ir a su habitación a escribir en su diario, decidió sentarse en las escaleras y averiguar lo que iba a ocurrir a continuación.

De pie en el jardín delantero había un hombre alto con una máscara de conejo. No era una máscara especialmente convincente. Cubría toda la cara, y dos orejas largas se levantaban por encima de su cabeza. Llevaba una gran bolsa de cuero marrón, que le recordó al señor Browning los maletines de los médicos de su infancia.

---Bien, aquí estoy ---comenzó a decir el señor Browning, pero el hombre de la máscara de conejo se llevó un dedo enguantado a los labios de conejo, y el Sr. Browning se quedó en silencio.

---Pregúntame qué hora es ---dijo una voz tranquila que venía de detrás de la boca inmóvil de la máscara de conejo.

---Entiendo que está interesado en la casa ---el señor Browning miró el cartel de ``En venta'' de la puerta principal, que estaba sucio y rayado por la lluvia.

---Tal vez. Puedes llamarme señor Conejo. Pregúntame qué hora es.

El Sr. Browning sabía que debía llamar a la policía. Debía hacer algo para que el hombre se fuera. ¿Qué clase de loco lleva una máscara de conejo?.

---¿Por qué lleva una máscara de conejo?.

---Esa no es la pregunta correcta. Pero estoy usando una máscara de conejo porque estoy representando a una persona muy famosa e importante que valora su privacidad. Pregúntame qué hora es.

---¿Qué hora es, señor Conejo? ---preguntó suspirando.

El hombre de la máscara de conejo se irguió. Su lenguaje corporal era de alegría y placer.

---Es hora de que seas el hombre más rico de Claversham Row ---dijo---. Voy a comprar tu casa , por dinero en efectivo , y por diez veces más de lo que vale, porque es simplemente perfecto para mí ---abrió la bolsa de cuero marrón, y sacó fajos de dinero. Cada fajo contenía quinientos crujientes billetes de cincuenta libras. Saco también dos bolsas de la compra de supermercado de plástico, en los que colocó los fajos de dinero.

El Sr. Browning inspeccionó el dinero. Parecía ser real.

---Yo\ldots{} ---vaciló---. ¿Qué tiene que hacer?. Voy a necesitar un par de días. Para ingresar esto. Asegúrese de que sea real. Y necesitaremos elaborar contratos, obviamente.

---El contrato ya está redactado ---dijo el hombre de la máscara de conejo---. Firma aquí. Si el banco dice que que el dinero no es bueno, puedes quedarte el dinero y la casa. Estaré de vuelta el sábado para entrar en la casa, vacía.

Puedes tenerlo todo fuera para entonces, ¿no?.

---No lo sé ---dijo el señor Browning. Luego añadio---. Estoy seguro de que puedo. Quiero decir, por supuesto.

---Volvere el sábado ---dijo el hombre de la máscara de conejo.

---Esta es una forma muy inusual de hacer negocios ---dijo el señor Browning. Estaba de pie en la puerta de su casa sosteniendo dos bolsas de la compra que contenían 750.000 libras.

---Sí ---asintió el hombre de la máscara de conejo---. Lo es. Nos vemos el sábado, entonces.

Se alejó. EL Sr. Browning se sintió aliviado al ver que se iba.

Había sido capturado por la convicción irracional de que, si hubiera tenido que quitarle la máscara de conejo, no habría encontrado nada debajo.

Polly subió a decirle todo lo que había visto y oído a su diario.

~

El jueves, un joven alto con una chaqueta de tweed y una pajarita llamó a la puerta. No había nadie en casa, por lo que nadie respondió, y, después de caminar alrededor de la casa, se fue.

El sábado, el señor Browning estaba de pie en su cocina vacía.

Había depositado el dinero con éxito y había acabado con todas sus deudas. Los muebles que habían querido mantener habían sido puestos en una furgoneta de mudanzas y enviados al tío del señor Browning, que tenía un enorme garaje que no estaba usando.

---¿Y si todo es una broma? ---preguntó la señora Browning.

---No estoy seguro de que sea gracioso darle a alguien setecientas cincuenta mil libras ---dijo el señor Browning---. El banco dice que el dinero es bueno. No ha sido denunciado como robado. Sólo es una persona rica y excéntrica que quiere comprar nuestra casa por mucho más de lo que vale.

Habían reservado dos habitaciones en un hotel local, aunque las habitaciones de hotel habían resultado más difíciles de encontrar de lo que el señor Browning había esperado. Además, había tenido que convencer a la Sra. Browning, que era enfermera, de que ahora podían permitirse el lujo de alojarse en un hotel.

---¿Qué pasa si no vuelve? ---preguntó Polly. Estaba sentada en la escalera, leyendo un libro.

---Ahora estás haciendo el tonto ---dijo el Sr. Browning.

---No llames tonta a tu hija ---dijo la señora Browning---. Ella tiene razón. No tienes un nombre o un número de teléfono o cualquier otra cosa.

Esto era injusto. El contrato se había firmado, y el nombre del comprador estaba claramente escrito en él: N.M. de Plume. Había una dirección, también, de una firma de abogados de Londres, y el Sr.

Browning les había llamado por teléfono y habían dicho que, a pesar del nombre algo tonto, sí, esto era absolutamente legítimo.

---Es excéntrico ---dijo el señor Browning---. Un excéntrico millonario.

---Apuesto a que él está detrás de esa máscara de conejo ---dijo Polly---. El millonario excéntrico.

Sonó el timbre. El sr. Browning fue a la puerta principal, con su esposa e hija a su lado, cada uno de ellos con la esperanza de cumplir con el nuevo propietario de su casa.

---Hola ---dijo una señora con una máscara de gato en la cabeza. No era una máscara muy realista. Polly vio los ojos brillantes detrás de ella.

---¿Es el nuevo propietario? ---preguntó el Sr. Browning.

---O eso, o lo represento.

---¿Dónde está \ldots{} su amigo?. El de la máscara de conejo.

A pesar de la máscara de gato, la joven (debía de ser una joven ya que su voz sonaba joven) parecía eficiente y casi brusca.

---¿Ha sacado todas sus posesiones?. Me temo que todo lo dejado atrás pasará a ser propiedad del nuevo dueño.

---Tenemos todo lo que importa.

---Bien.

Polly dijo:

---¿Puedo venir a jugar al jardín?. No hay un jardín en el hotel ---había un columpio en el árbol de roble en el jardín trasero, y a Polly le encantaba sentarse en él y leer.

---No seas tonta, amor ---dijo el señor Browning---. Vamos a tener una casa nueva, y entonces tendrás un jardín con columpios. Voy a poner columpios para ti.

La dama de la máscara de gato se agachó.

---Soy la señorita Gata. Pregúntame qué hora es, Polly.

Polly asintió.

---¿Qué hora es, señorita Gata?

---El tiempo para que tu y tu familia salgáis de este lugar y nunca volvais la vista atrás ---dijo la señorita Gata, pero ella lo dijo amablemente.

Polly se despidió de la señora con la máscara de gato, cuando llegó a la final del camino del jardín.