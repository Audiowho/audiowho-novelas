\chapter*{Capítulo Cuatro} \addcontentsline{toc}{chapter}{Capítulo Cuatro}

---1984 ---musitó Amy Pond---. Pensé que de alguna manera se sentiría más, no lo sé. Histórico. No se siente que fue hace mucho tiempo. Pero mis padres no se habían siquiera conocido ---ella vaciló, como si estuviera a punto de decir algo acerca de sus padres, pero su atención se desvió. Cruzaron la carretera.

---¿Cómo eran? ---preguntó el Doctor---. ¿Tus padres?

Amy se encogió de hombros.

---Lo de siempre ---dijo ella, sin pensar---. Una madre y un padre.

---Suena probable ---asintió el Doctor---. Necesito que mantengas los ojos abiertos.

---¿Qué es lo que buscamos?

Era un pequeño y pintoresco pueblo inglés, y se veía como un pequeño pueblo inglés por lo que Amy estaba preocupada. Al igual que el que había dejado en 2010, con una zona verde y árboles y una aldea y una iglesia, sólo que sin las cafeterías o las tiendas de telefonía móvil.

---Fácil. Estamos buscando algo que no debería estar aquí. O estamos buscando algo que debería estar aquí, pero no lo está.

---¿Qué tipo de cosas?

---No estoy seguro ---dijo el Doctor. Se frotó la barbilla---.

Gazpacho, tal vez.

---¿Que es el gazpacho?

---Sopa fría. Pero se supone que debe estar fría. Así que si nos fijamos en todo 1984 y no podemos encontrar gazpacho, eso sería una pista.

---¿Siempre fuiste así?

---¿Cómo?

---Un loco. Con una máquina del tiempo.

---Oh, no. Tardé una eternidad hasta que llegué a tener una máquina del tiempo.

Caminaron por el centro de la pequeña ciudad, en busca de algo inusual, y no encontraron nada, ni siquiera el gazpacho.

~

Polly se detuvo en la puerta del jardín en Claversham Row, mirando a la casa que había sido su hogar desde que se habían mudado aquí cuando tenía siete años. Se acercó a la puerta principal, tocó el timbre y esperó, y se sintió aliviada cuando nadie contestó. Echo un vistazo por la calle, luego se dirigio a toda prisa alderedor de la casa, más allá de los contenedores de basura, en el jardín trasero.

La vidriera que daba al pequeño jardín de atrás tenía una trampilla que no se fijaba correctamente. Polly pensó que era muy poco probable que los nuevos dueños de la casa la hubieran arreglado. Si lo hubieran hecho, habría regresado cuando estuvieran aquí, y habría tenido que preguntar, y sería incómodo y embarazoso.

Ese era el problema con las cosas que se ocultan. A veces, si vas a toda prisa, las dejas atrás. Incluso las cosas importantes. Y no había nada más importante que su diario.

Ella lo había tenido desde que habían llegado a la ciudad. Había sido su mejor amigo: había confiado en él, le dijo lo de las chicas que la había intimidado, con las que había hecho amistad, sobre el primer chico que nunca le había gustado. Ella volvía a él en momentos de dificultad o confusión y dolor. Era el lugar donde derramó sus pensamientos.

Y estaba escondido debajo de una tabla suelta en el gran armario en su dormitorio.

Polly golpeó la puerta francesa de la izquierda fuertemente con la palma de su mano, golpeando al lado de la ventana, y la puerta se tambaleó, y luego se abrió.

Entró. Se sorprendió al ver que no habían sustituido los muebles que su familia había dejado atrás. Todavía olía a su casa. Se quedó en silencio: nadie en casa. Bueno. Se apresuró a subir las escaleras, preocupada de que todavía pudiera estar en la casa cuando el Sr. Conejo o la señorita Gata volvieran.

En el rellano algo rozó su cara, la tocó suavemente, como un hilo o una tela de araña. Ella levantó la vista. Era extraño. El techo parecía peludo: hilos como cabellos o pelos filiformes, bajaban de él. Dudó, pensó en correr, pero podía ver la puerta de su dormitorio. El cartel de Duran Duran estaba todavía en ella. ¿Por qué no lo habían quitado?

Tratando de no mirar hacia el techo peludo, abrió la puerta de su dormitorio.

La habitación era diferente. No había muebles, y donde estuvo su cama había hojas de papel.

Miró al suelo: las fotografías de los periódicos, rostros quemados hasta de tamaño natural. Los agujeros para los ojos se habían cortado ya. Ella reconoció el príncipe Carlos, Ronald Reagan, Margaret Thatcher, el Papa Juan Pablo II, la Reina\ldots{}

Tal vez iban a hacer una fiesta. Las máscaras no se veían muy convincentes. Fue al armario empotrado en el extremo de la habitación.

Su diario Smash Hits estaba en la oscuridad, bajo el suelo. Abrió la puerta del armario.

---Hola, Polly ---dijo el hombre en el armario. Llevaba una máscara, al igual que los otros. Una máscara de animal, se trataba de una especie de gran perro negro.

---Hola ---dijo Polly. Ella no sabía qué más decir---. Yo\ldots{} dejé mi diario.

---Ya lo sé. Lo estaba leyendo ---levantó el diario. No era la misma persona que el hombre de la máscara de conejo o de la mujer de la máscara de gato, pero todo lo que Polly había sentido en ellos, la maldad, se intensificó aquí---. ¿Lo quieres?

---Sí, por favor ---dijo Polly al hombre perro enmascarado. Se sentía herida y violada: este hombre había estado leyendo su diario.

Pero lo quería de vuelta.

---¿Ya sabes lo que tienes que hacer, para conseguirlo?

Ella negó con la cabeza.

---Pregúntame la hora.

Ella abrió la boca. Estaba seca. Se humedeció los labios y murmuró:

---¿Qué hora es?

---Y mi nombre ---dijo---. Di mi nombre. Yo soy el señor Lobo.

---¿Qué hora es, señor Lobo? ---preguntó Polly. Un juego de patio de recreo le vino espontáneamente a la mente. El señor Lobo sonrió (pero ¿cómo puede sonreír una máscara?), y abrió su enorme boca para mostrar filas y filas de dientes afilados.

---Hora de la cena.

Polly empezó a gritar cuando él se acercó, pero no llegó a gritar durante mucho tiempo.

