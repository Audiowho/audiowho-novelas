\chapter*{CAPÍTULO VEINTIUNO}
\addcontentsline{toc}{chapter}{CAPÍTULO VEINTIUNO}

{El Cartago tenía, al menos en términos reales, cientos de años de edad.
 No le llevó mucho partirse en dos cuando los motores astrónicos ardieron
 bajo el repentino esfuerzo de la ignición. Un río de plasma
 supercaliente surgió a través de los pasillos y las cubiertas, empujando
 una incandescente pared de llamas ante ella y dejando nada más que ruina
fundida tras ella.}

{Corrieron hacia la TARDIS con las llamas quemándoles los talones y la
 puerta de la cabina de policía se cerró de golpe con un ruido sordo
 cuando el plasma golpeó. Una energía ardiente engulló la nave del tiempo
 en una tormenta de iones cargados. Los últimos vestigios de la atmosfera
 incineradora llevaban consigo el rugido final de la explosión y con un
 ligero gruñido y quejido la cabina de policía comenzó a
 desmaterializarse. La ablación del torrente de plasma desapareció un
 momento antes de que el mismo Cartago se convirtiera en una nube
expandida de átomos acelerados.}

{Fuera de la cegadora tormenta de fuego salió disparada una pequeña
 chispa de materia azul, girando sobre sí misma, con la lámpara en su
 tejado brillando ocupada mientras se desvanecía del universo material y
 se deslizaba hacia el misterioso vórtice donde el tiempo y el espacio
 eran uno.}

{El Doctor se aferraba a la consola de
 control, una mano yendo de un lado a otro entre palancas e
 interruptores. Los brillantes filamentos naranjas de la columna de
 cristal en el centro de la consola se alzaban grácilmente y bajaban,
 emanando un cálido y hogareño brillo en su anciana cara. Por encima de
 él, los gigantescos rotores temporales giraban mientras la TARDIS
calculaba su posición exacta en el continuo del espacio-tiempo.}

{Clara se levantó del suelo. Tibby Vent y Hobbo también estaban
 poniéndose en pie al mismo tiempo que Raymond Balfour. Todos parecían
tan débiles como Clara se sentía.}

{--Lo hemos conseguido, entonces --dijo Balfour, sujetándose a la consola
en busca de apoyo.}

{--Por supuesto que sí --dijo el Doctor, finalmente retrocediendo de los
 controles y flexionando sus largos dedos. Los juntó e hizo crujir sus
nudillos--. Nunca lo he dudado.}

{Clara le lanzó una mirada pero no dijo nada. Su piel le hormigueaba y
 había un olor a quemado en el aire. Supuso que todos se habrían
chamuscado un poco al final.}

{El Doctor hizo girar el monitor de la TARDIS y le dio un toque.}

{--Mirad eso.}

{Observaron la imagen. Mostraba una profusión de círculos concéntricos
 sobrepuestos y elegantes ecuaciones hexagonales. En el centro del
 diagrama había una brillante esfera de energía chispeante,
desapareciendo en un largo y turbulento vórtice.}

{--¿Es esa la nave de los Faerón? --se preguntó Clara.}

{--Sí.}

{--Se ha ido --dijo Tibby--. Llevándose al Glamour con ellos.}

{--Como una araña por un desagüe --dijo el Doctor. Apagó el monitor con
una floritura--. Se ha ido para siempre.}

{--Pero Jem\ldots{} --dijo Clara.}

{--Tuvo que hacerlo. Era la única que podía asegurarse de que el Glamour
fuera con la nave de los Faerón.}

{--Debemos de haber recibido un buen golpe de radiación cuando los
motores del Cartago explotaron --dijo Hobbo.}

{El Doctor hizo un ademán, desechando la idea.}

{--No es un problema. En el momento en el que entrasteis dentro de la
 TARDIS eso se disolvió. No hay efectos secundarios. Solo unas pocas
marcas de quemadura.}

{--Esta nave tuya es verdaderamente un milagro, ¿verdad, Doc? --preguntó
Hobbo. Miró a su alrededor con puro asombro.}

{--No es un milagro --dijo el Doctor--. Solo se\ldots{}}

{--Mofa del tiempo y el espacio, sí, lo pillo.}

{El Doctor frunció el ceño.}

{--De hecho, iba a decir que se trata de un soberbio ejemplar de
ingeniería dimensional relativa.}

{Balfour dijo:}

{--Debemos agradecértelo, Doctor\ldots{} pero hemos perdido tanta gente.
 Jem dio su vida para asegurarse de que el Glamour fuera destruido. Tanya
 fue asesinada. Mitch fue asesinado. Y Marco\ldots{}}

{El Doctor le observó severamente.}

{--Al principio de la misión dije que sería peligroso. No estaba
bromeando.}

{--Debería haber prestado más atención --dijo Balfour, con tristeza.
 Parecía débil, casi desolado--. Debería haber abandonado la misión al
principio.}

{--¿Bajo mi palabra? --el Doctor estaba escandalizado-- No seas ridículo.
 El Alejandría era una nave hermosa, una buena aventura. Piénsalo:
 viajamos por el último agujero de gusano de los Faerón. Descubrimos un
 nuevo planeta, perdido en un vacío intergaláctico. Encontramos el
 Cartago. Conocimos a los mismos Faerón. Y, lo más importante, ayudamos a
 detener el Glamour de una vez por todas. Créeme, el universo es un lugar
 mejor sin él.}

{La TARDIS se alzaba en el vestíbulo
 principal desde el que se veían las bahías de embarque públicas de la
 Estación Lejana. Clara observó con asombro cómo los humanos, los
 alienígenas, los robots y cosas que ni siquiera podía clasificar se
 paseaban, apresurándose hacia las puertas de salida y esperando a los
 vehículos espaciales o simplemente observando las vistas. Podía ser
 salvaje. Podía ser atemorizante. Pero jamás era algo menos que
increíble.}

{--Qué bueno que nos hayas traído --le dijo Balfour al Doctor. Estaba de
pie con Hobbo y Tibby ante la TARDIS.}

{El Doctor estaba asomado por la puerta de la cabina.}

{--Bueno, no os puedo tener por ahí paseándoos por la TARDIS.\@ Ya habéis
visto cómo es por dentro. Apenas hay espacio para mí y para Clara.}

{Balfour sonrió.}

{--Puede que me lleve un poco de tiempo sobreponerme a la experiencia,
 Doctor, pero me alegro de haberla tenido. He aprendido mucho sobre mí
 mismo y sobre otras personas. De hecho, ya he hablado con Tibby y Hobbo
 sobre otra idea: Tibby dice que ha ganado un nuevo punto de vista sobre
 unas nuevas ruinas de los Faerón en otras partes de la galaxia. Estamos
 pensando en comprar una nueva nave espacial y\ldots{}}

{--No me digáis nada más --dijo el Doctor, levantando una mano a modo de
rendición.}

{Clara sonrió y le pegó un codazo.}

{--Creo que hay algo más que echar un vistazo a unas ruinas, Doctor.}

{--¿De qué estás hablando?}

{El Doctor miró a Tibby y a Balfour, y de repente vio que estaban dándose
la mano.}

{--¿Por qué estáis haciendo eso? Dándoos la mano así. ¿Para qué?}

{--Ahora estamos juntos, Doctor --dijo Tibby, alegremente.}

{--Ya lo veo. Estáis justo delante de mí. Pero, ¿por qué os estáis dando
la mano así?}

{--Te lo explicaré más tarde --le dijo Clara.}

{--Pensábamos que quizá Clara y tú quisierais veniros con nosotros --dijo
Balfour.}

{--Hay un objeto en particular en Ursa Minor --dijo Tibby--. Es un
genuino icono de los Faerón. Necesita ser estudiado.}

{El Doctor suspiró.}

{--Si queréis mi consejo, y por supuesto lo hacéis, dejadlo bien solo.
 Olvidaos de Ursa Minor. Olvidaos de los Faerón. Olvidaos de cualquier
 cosa que requiera estudio, o parezca demasiado bueno como para
 resistirse, o que parezca perfecto. Id y descubrid otra cosa --el Doctor
 se detuvo, observó sus manos juntas durante un momento largo y añadió--.
Como el uno al otro, por ejemplo.}

{--Qué idea más buena --dijo Clara.}

{El Doctor abrió la puerta de la TARDIS.}

{--Clara, vamos, dejemos de interferir. Es hora de volver a casa.}

{Tibby comenzó a protestar, pero el Doctor levantó un dedo para pedir
silencio y les fijó con una mirada oscura.}

{--Escuchadme. El Glamour fue arrastrado al olvido por los Faerón. Gente,
 amigos, murieron en el proceso. Por respeto a su memoria, no persigáis
todo lo que brilla o requiera vuestra atención.}

{--Pero creía que el Glamour se había ido para siempre --dijo Clara.}

{--``Para siempre'' es un tiempo muy largo --dijo el Doctor--. Pero para
las cosas como el Glamour, es raramente suficiente.}
