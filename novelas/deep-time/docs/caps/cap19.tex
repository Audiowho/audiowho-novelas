\chapter*{CAPÍTULO DIECINUEVE}
\addcontentsline{toc}{chapter}{CAPÍTULO DIECINUEVE}

{--¿Misión suicida? --las palabras parecían pesadas y feas en la boca de
 Clara y sintió que su estómago se torcía un poco ante la idea. De
 repente, la colada y corregir exámenes jamás le había parecido tan
apetecible.}

{--Solo hay una forma de estar seguros de que el Glamour vaya por todo el
 camino hasta lo más profundo del pozo que excavaron --dijo el Doctor--.
Llevándoselo con ellos.}

{--Pero, aun así, ¿suicida?}

{--Ya les has oído --dijo Tibby, con tristeza--. El viaje final.}

{--Pero podemos irnos, ¿no? --preguntó Balfour-- En el Cartago, quiero
decir.}

{--He hecho unas pocas comprobaciones y diría que podría ser --dijo
 Hobbo. Estaba apoyada contra uno de los paneles de controles, con los
 brazos cruzados. Un repentino espasmo de tristeza le cruzó la cara y
frunció el ceño--. Me sentiría mejor si Mitch estuviera aquí.}

{--Pero, ¿qué vamos a hacer? --preguntó Balfour, pareciendo más y más
 preocupado-- Quiero decir, ya les habéis oído. Están en una misión
suicida en los albores del tiempo. No queremos unirnos a ellos, ¿no?}

{--¿Hay algo que podamos hacer, Hobbo? --preguntó Tibby.}

{--Tengo que intentar encenderlo primero --dijo Hobbo--. No lo puedo
decir hasta que no lo hayamos intentado.}

{--No me apetece la única opción --dijo Clara. Se giró al Doctor--.
 Vamos. ¿Cómo podemos hacer que esta nave vuelva a volar? Siempre piensas
en algo.}

{El Doctor apretó los labios.}

{--Bueno, si tuviéramos la TARDIS, las cosas se volverían mucho más
simples, por supuesto.}

{--Pero no es así --dijo Balfour.}

{--Un milagro seguro que sería muy útil --dijo Hobbo.}

{--Pero no tenemos la TARDIS --repitió Balfour.}

{--No hemos visto la TARDIS desde que hemos dejado el Alejandría --dijo
Clara--. Y tú dijiste que la señal había desaparecido, Doctor.}

{--Sí, eso he dicho --los ojos del Doctor se volvieron más huecos--. Pero
 eso era antes de que supiera que los Faerón seguían aquí, aunque solo
sea en espíritu.}

{Te conocemos, Señor del Tiempo. Recordamos la llegada de los Señores del
Tiempo.}

{--Estaba esperando que no lo hicierais --dijo el Doctor.}

{Recordamos la llegada del Nuevo Universo. La caída del Tiempo de los
Faerón.}

{--Eso fue antes de mi época, si veis lo que quiero decir --dijo el
 Doctor--. Mucho antes. Los Señores del Tiempo cambiaron. Lucharon sus
 propias guerras y Gallifrey se perdió. Quizá se ha perdido para
siempre.}

{Gallifrey conocía la Imperfección.}

{--Sí habláis del Glamour, sí. Era una poderosa arma para el mal. Los
 Señores del Tiempo no podían permitir que influyera en el universo de la
 forma que quería. Y con vuestro poder, el Glamour, la Imperfección,
 podría haber causado gran daño para las civilizaciones que vendrían
después.}

{Lo sabemos.}

{--Oh, vale, lo siento.}

{Gallifrey advirtió a los Faerón.}

{El Doctor pareció estupefacto.}

{--¿Qué? Pero creía que los Señores del Tiempo os obligaron a cerrar las
 carreteras de los Faerón para detener al Glamour. Creía que provocaron
vuestra destrucción.}

{Los Señores del Tiempo no podían destruirnos.}

{--Ah --el Doctor se giró al resto-- ¿Veis? Dije que eran más poderosos
de lo que pensaba.}

{Los Faerón se enfrentarán a su propia Imperfección.}

{--¿Aunque eso signifique destruiros a vosotros en el proceso?}

{El último viaje. Tiempo Profundo.}

{--Eso es horrible --dijo Tibby, en voz baja--. ¿Se mataron para librar
el universo del Glamour?}

{--Auto sacrificio a nivel cósmico --dijo el Doctor.}

{--Es tan triste --susurró Clara. Miró a las parpadeantes figuras azules,
 y de repente ya no le parecían ni terroríficas ni misteriosas, solo
trágicas.}

{--La nobleza obliga --dijo el Doctor--. Eran la raza perfecta. Hicieron
 lo correcto para el universo, fueran cuales fueran las consecuencias.
 Crearon el Glamour, lo que vieron como una Imperfección, así que
lidiaron con ello. O, al menos, eso intentaron.}

{El viaje final debe ser completado.}

{--Pero están atrapados. El Cartago voló directamente en su último
 agujero de gusano cuando estaban a punto de deshacerse del Glamour. Han
 estado atrapados aquí desde entonces. Y el Glamour sigue por ahí,
rondando el universo, provocando el caos.}

{--Así que, ¿la misión fracasó? ¿Todo esto para nada?}

{--No del todo. El Glamour es la Imperfección, recordad, es parte de la
 raza de los Faerón. Para nosotros parece algo perfecto, algo que
 deseamos más que nada. Es así como usa a la gente. Pero nuestra
 perfección es su imperfección. Es parte de ellos, está atado a ellos, es
por eso por lo que tienen que ir en el viaje final con ello.}

{--Así que, ¿si la misión final se completa, el Glamour se irá con
ellos?}

{--Le guste o no, sí.}

{Debéis marcharos antes de que el viaje final se complete.}

{El Doctor juntó sus manos y frunció el ceño.}

{--Bueno, eso es muy amable por vuestra parte. Y nos iríamos, aunque no
tenemos forma de hacerlo.}

{Te conocemos, Señor del Tiempo.}

{--Eso es muy halagador, pero\ldots{}}

{Te conocemos\ldots{} Doctor.}

{Las palabras de los Faerón resonaron en la mente de Clara y de repente
 tomaron un nuevo significado. Los ojos del Doctor brillaban como
diamantes cuando se giró para mirar a los Faerón.}

{--¡Sí! Me conocéis. Debéis conocerme. Porque tenéis mi TARDIS,
¿verdad?}

{Los Faerón harán su último viaje.}

{Tras los Faerón, un fresco brillo azul apareció en la oscuridad,
 convirtiéndose en la silueta de una alta y gran cabina con una lámpara
 en lo alto. Las puertas y las ventanas aparecieron en el parpadeante
rectángulo y las palabras ``CABINA DE POLICIA'' brillaron en el techo.}

{--Es un regalo --dijo el Doctor. Sus palabras estaban bañadas de
 tristeza y humildad, y cerró los ojos de alivio--. Están agradecidos con
los Señores del Tiempo por señalar su propia Imperfección.}

{Al cabo del tiempo, la luz azul se desvaneció y con ella los Faerón
 también desaparecieron, como rayos de luz de luna tras una nube. El
puente se volvió de repente más oscuro, silencioso y vacío.}

{--Eh, Doctor\ldots{} --dijo Clara--. ¿Está la TARDIS bien?}

{Los ojos del Doctor se abrieron de golpe. La TARDIS se alzaba orgullosa
 en el centro, pero el viaje parecía haberse cobrado peaje. La pintura
 estaba agrietada, pelándose en largas hileras. Las ventanas congeladas
estaban oscuras y bañadas con polvo y moho.}

{--Parece que ha estado en la guerra, Doc --dijo Hobbo.}

{Con cara mustia, el Doctor se acercó lentamente a la cabina de policía.
Tocó una de las puertas, sintió la pintura azul derrumbarse.}

{--Bueno\ldots{} --dijo, muy flojo--. Ha pasado un tiempo, ¿no? Me
pregunto cuántos millones de años habrás estado esperando.}

{Clara tragó saliva incómodamente cuando el Doctor recorrió sus largos y
 temblorosos dedos por la madera de la TARDIS, recorriendo el viejo y
agrietado cartel en la puerta que contenía el teléfono.}

{--¿Cómo puede haber envejecido así cuando ha estado yendo hacia atrás en
el tiempo? --se preguntó ella.}

{--El tiempo es relativo, Clara. La TARDIS ha estado rastreando el origen
 de los flujos temporales, el aquí y el ahora. Así que debe de haber
 tomado el camino largo. Los Faerón no lo sabían. Lo hicieron lo mejor
que pudieron.}

{--Sin embargo, no cabemos todos ahí --dijo Balfour.}

{--Sí que podemos --dijo Clara--. Es más grande por dentro. Mucho más
grande.}

{Balfour se encogió de hombros.}

{--Si tú lo dices.}

{--Debe de haberse desmaterializado cuando cayó por el precipicio con el
 Alejandría --dijo el Doctor--. Los Faerón la percibieron en el continuo
espacio-tiempo y la recogieron. La habrán guardado hasta ahora.}

{--Pero\ldots{} es tan vieja.}

{--Eso no significa nada para los Faerón, Clara. Existen fuera y en un
espacio y tiempo.}

{Clara nunca había visto al Doctor tan perdido. Descansó su mano sobre la
 TARDIS y cerró los ojos. Clara alargó la suya y tocó la máquina del
 tiempo. Parecía tan fría como una lápida. No había ninguna ligera
vibración, ningún zumbido. Estaba sin vida.}

{--¿Deberíamos entrar? --preguntó ella. Aquella idea le hizo sentirse
nerviosa.}

{Lenta y pesadamente, el Doctor sacó la llave de la TARDIS de su bolsillo
 y la insertó en el cerrojo. Al principio estaba demasiado duro como para
 moverse. Pero al cabo del tiempo, el cerrojo se abrió a duras penas y el
Doctor empujó la puerta. Crujió con sus goznes oxidados.}

{Estaba oscuro dentro. Aquello fue lo primero que asustó a Clara. El
 exterior de la TARDIS era duro, eso lo sabía, prácticamente
 indestructible, como decía el Doctor. Supuso que podría sobrevivir a un
poco de tiempo. Era una máquina del tiempo, después de todo.}

{Pero el interior era distinto. El interior se suponía que debía ser
 inviolable. Intocable. Pero el tiempo se había metido dentro de la
 TARDIS y la había asolado con una ferocidad que le robó el aliento a
 Clara. No había ni una sola luz brillando en la consola central, o en
 cualquiera de los paneles que les rodeaban. La columna central estaba
 oscura, los filamentos de cristal en su interior estaban grises y
 agrietados. Normalmente estarían ardiendo con un cálido brillo naranja,
 un brillo fogoso, hogareño y lleno de energías y promesas. Pero ahora no
había más que vacío y silencio.}

{Sus pasos resonaron en el suelo de metal mientras siguió al Doctor hasta
 la escotilla de la zona de control. Caminó lentamente alrededor de la
 consola, dejando que sus dedos recorrieran los controles oscurecidos,
dibujando unas gruesas líneas en el polvo.}

{--¿Qué ha pasado? --preguntó Clara, estupefacta.}

{--La TARDIS está muerta, Clara --respondió el Doctor.}

{--No es culpa tuya.}

{--¿Ah, no? --el Doctor la miró desoladamente. Sus ojos eran como
 carámbanos de hielo azul-- Yo nos traje aquí. A ti, a mí y a la TARDIS.\@
Sabía que iba a ser peligroso. Yo lo dije, justo desde el principio.}

{--Siempre es peligroso. Bueno, casi siempre. Pero normalmente\ldots{}}

{--Normalmente escapamos. Sobrevivimos --el Doctor suspiró--. Bueno, no
 tienes que preocuparte por eso. Seguimos pudiéndolo hacer. Sigue estando
el Cartago, recuerda.}

{--Lo sé, pero Hobbo ha dicho que puede que no vuele.}

{--Oh, estoy seguro de que encontrará una manera. Tiene que hacerlo. O
 sino todos acabaremos así, al cabo del tiempo --el Doctor apretó un
 nudillo contra la consola de la TARDIS.\@ Y entonces, con una repentina y
 explosiva furia golpeó su puño contra el metal. Vio a Clara estremecerse
 ante el ruido resonando alrededor de la sala y sonrió con tristeza--. Lo
siento, no debería haber hecho eso.}

{--Está bien, en serio\ldots{}}

{--No, creo que me he roto un dedo --el Doctor levantó su palma con una
mueca de dolor.}

{Clara le tomó la mano amablemente, y cerró sus dedos alrededor de los de
 él. Su mano estaba fría y pensó que podía detectar un ligero temblor. De
repente su piel parecía pálida y fina, como la de un anciano.}

{--Lo siento --dijo ella.}

{--Yo también.}

{--Ey, Doc --dijo Hobbo. Estaba en el umbral, asomando su cabeza en la
 sala de la consola--. He intentado encender el Cartago y, ¿adivinas qué?
Creo que puede que sea una nave, después de todo.}

{El Doctor asintió, ausentemente.}

{--Bien, bien. Eso está bien.}

{--Quizá la TARDIS se ponga mejor cuando la saquemos del pozo temporal
 --sugirió Clara--. Quizá una vez nos hayamos liberado del Cartago y de
los Faerón, se ponga mejor. Ya sabes, regenerarse o algo.}

{El Doctor asintió, ausentemente.}

{--Sí, posiblemente --dijo él, pero no había convicción en su voz.}

{--Y no podemos abandonar a los otros, ¿verdad?}

{--¿Otros?}

{--Van a necesitarte para hacer que el Cartago funcione, ¿no?}

{--Sí, probablemente.}

{Le alejó de la consola y salieron de la TARDIS.\@ Tuvo cuidado al cerrar
 la puerta tras ellos, estremeciéndose cuando los goznes protestaron al
 hacerlo.}

{Hobbo estaba ocupada en las estaciones de
 control principales, tumbada boca arriba con su cabeza dentro de una
 escotilla de inspección. Jem estaba sentada en uno de los asientos de
 vuelo. Balfour ayudaba a Tibby a aclarar las telarañas y la maleza de
las superficies.}

{--Parece que vamos a poder irnos --dijo Balfour.}

{Tibby le observó con incertidumbre. Le dijo al Doctor:}

{--Siento lo de tu TARDIS.}

{--No pasa\ldots{} nada --dijo el Doctor, con pesadez. Parecía estar
 moviéndose con lentitud, como si estuviera en un sueño, pensó Clara. O
 en una pesadilla. Echó un vistazo hacia la TARDIS y cerró sus ojos como
si le doliera.}

{Había luces parpadeando en todos los instrumentos del Cartago y parecía
 más el puente de una nave espacial que un mausoleo. Clara pensó que
 podía detectar una vibración bajo sus pies a medida que la energía
 comenzaba a inundar a través de la nave. Más luces se encendieron en las
 consolas encima de sus cabezas. Toda la nave comenzó a parecer más
cálida y más brillante a medida que más sistemas se ponían en línea.}

{--Desearía que Dan estuviera con nosotros --dijo Jem cuando recorrió la
secuencia de comprobación de luces--. Sabía mucho más de esto que yo.}

{--No bromeo --dijo Hobbo, al incorporarse. Había una mancha de suciedad
en su cara--. Sigo pensando qué haría Mitch con todo esto.}

{--¿Creéis que podéis hacer volar al Cartago? --preguntó Clara.}

{--No debería ser un problema. No es muy distinta al Alejandría. El
 despegue está automatizado. Solo espero que los sistemas de auto
reparación vuelvan a estar en línea.}

{Hobbo cerró de golpe la escotilla de inspección.}

{--Están todos encendidos y arreglándose mientras hablamos. Todos los
sistemas funcionan. Menuda vieja gloria, esta nave.}

{--¿Así que nos vamos ya? --preguntó Balfour, impacientemente.}

{--Sí --dijo Hobbo--. No tenemos mucha energía así que cada segundo nos
cuesta más.}

{--De acuerdo --dijo Balfour, juntando sus manos--. Que todo el mundo
encuentre un asiento y se ate los cinturones. Nos vamos.}

{--Me parece que no --dijo el Doctor.}

{Hubo un afilado e inesperado silencio cuando las palabras del Doctor
 resonaron por la cubierta de vuelo. Los ojos de todo el mundo se giraron
 para mirarle. Se alzaba en el centro de la sala, con los brazos cruzados
y resuelto.}

{Balfour frunció el ceño.}

{--¿Qué? Átate, Doctor. Ya has oído a Hobbo: podría ser un viaje
movidito.}

{--Oh, eso lo sé --dijo el Doctor--. Pero no voy a hacer ningún viaje.
Ninguno de nosotros se va.}
