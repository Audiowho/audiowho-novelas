\chapter*{CAPÍTULO VEINTE}
\addcontentsline{toc}{chapter}{CAPÍTULO VEINTE}

{--¿Qué quieres decir? --Balfour parecía confuso, casi irritado.}

{--¿Doctor? --dijo Clara, incierta.}

{El Doctor caminaba lentamente por el puente, dando vueltas alrededor de
Balfour. Sus ojos eran duros.}

{--No vamos a marcharnos de este pozo temporal --dijo él.}

{--¿De qué estás hablando, Doctor? --preguntó Balfour-- ¡Mira a tu
 alrededor! ¡Tenemos la oportunidad perfecta de marcharnos de este lugar!
¡Deja de retrasarnos!}

{--¿La oportunidad perfecta? --repitió el Doctor. Miró por el puente-- Sí
 que lo es, ¿no? Una vieja y estrellada nave que vuelve a estar
completamente funcional. ¡Qué suerte!}

{--Doctor --dijo Balfour, forzando la paciencia en su voz--, se nos acaba
 el tiempo\ldots{}}

{--¿Ah, sí? Creía que estábamos atrapados en el tiempo. Atascados como
 una espina en su garganta --el Doctor parecía indignado--. El tiempo ha
 estado tosiendo durante la eternidad en este planeta, intentando
 desalojar el bloqueo. Es eso todo lo que eran esas fluctuaciones
 temporales, la pobre Madre Tiempo, teniendo arcadas y asfixiándose hasta
morir.}

{Los motores de la nave ahora eran un murmullo profundo y de fondo, como
un trueno antes de una tormenta. Clara dijo:}

{--Doctor\ldots{}}

{El Doctor la ignoró a medida que su voz se endurecía.}

{--El Cartago entró en el agujero de gusano y chocó directamente contra
 la nave de los Faerón, fusionando una discreta área del continuo
 espacio-tiempo y por ello ninguno de los dos se podía mover. Lo que
 significa que los Faerón jamás completaron su viaje, por lo que sabemos.
 Lo que significa que el Glamour sigue libre para merodear por ahí, hasta
que su viaje se complete.}

{--Sí, ya sabemos eso. Así que vamos a ayudarles a hacerlo --dijo
 Balfour, con ansias--. Ayudémosles a acabar su misión. Si nos marchamos
ahora en el Cartago, el bloqueo se liberará.}

{--El bloqueo se liberará, sí --dijo el Doctor, de acuerdo con él.}

{--Todos podremos sobrevivir --añadió Balfour.}

{--No --el Doctor negó con la cabeza--. No, me temo que no podemos. El
 Cartago no va a volar, no decentemente. Puede que consigamos suficiente
 energía como para despegar, pero eso será todo. Si tenemos suerte
 explotaremos en el espacio profundo y todos y cada uno de nosotros se
evaporará al instante.}

{--¿Y si no tenemos suerte? --preguntó Clara, incrédula.}

{--La integridad del casco se mantendrá, pero el soporte vital fracasará
y nos asfixiaremos. Lentamente.}

{Hubo un silencio incómodo durante unos pocos segundos.}

{--Pero la nave está funcionando --dijo Hobbo--. Está bien.}

{--¿Lo está? --el Doctor dio una vuelta y fijó en la joven ingeniera su
 mirada más fiera-- ¿De verdad crees eso, Hobbo? Oh, mira, todo va bien y
 es brillante y genial --el Doctor se giró para mirar al resto, a cada
 uno de ellos, con sus cejas alzadas como un profesor desafiando a su
clase--. ¿Es eso lo que pensáis de verdad?}

{--Bueno, es un poco afortunado, supongo --dijo Clara, incierta--.
 Pero\ldots{}}

{--Pero no es imposible --dijo Hobbo--. Simplemente no lo es.}

{--Vuelve a mirar, entonces, Hobbo --dijo el Doctor con amabilidad--.
Todos vosotros, mirad de nuevo. ¿Esto os parece una nave?}

{Clara parpadeó y miró alrededor del puente una vez más. Podía ver los
 paneles de controles, las luces parpadeantes en los instrumentos. Podía
 percibir el zumbido de los motores, percibir la energía contenida dentro
 de la nave. Quería decir que sí, pero algo se lo impedía. Algo en la
 parte trasera de su mente, como el pitido lejano de un timbre. Algo en
la voz del Doctor. Algo en lo que confiaba, casi más que en sí misma.}

{Miró de nuevo. Los paneles eran pedazos fosilizados de roca, enredados y
 agrietados con maleza y plantas. No había luces parpadeantes, solo unas
 finas y grises arañas correteando furtivamente a través de las ruinas.
 No había vibración, ni ruido de motores, solo un fétido y apestoso hedor
y niebla merodeando por el suelo.}

{--El Cartago no va a ir a ningún lugar --dijo el Doctor.}

{Tibby había estado observando el intercambio con una ansiedad en
 aumento. Se aclaró la garganta, el sonido fue incómodo en el silencio.
Miró a Balfour y dijo:}

{--Tú no eres Ray Balfour\ldots{} ¿verdad?}

{--Estás loca --dijo Balfour.}

{--Eres el Glamour --dijo el Doctor.}

{Con un repentino rugido de furia, Balfour lanzó al Doctor a través del
 puente. Entonces se abalanzó contra los controles de vuelo del Cartago,
rascando entre la tierra para encender los motores.}

{--¡No dejéis que despegue! --gritó el Doctor.}

{Jem intentó agarrar a Balfour pero tenía la fuerza de un animal salvaje,
 deshaciéndose rudamente de ella. Se abalanzó sobre los controles,
 comenzó a golpear unos interruptores enterrados y unos paneles de
 activación con una urgencia maníaca. Unas luces cubiertas de tierra
 parpadearon y unos circuitos olvidados hace tiempo se encendieron. Un
 murmullo antinatural hizo temblar el puente como si una antigua bestia
 se hubiera despertado tras una larga siesta. Unas grietas aparecieron
 por todo el suelo, mandando nubes de polvo por el suelo. El metal brilló
 bajo la calcificada capa como los huesos que se podían ver a través de
la piel rota.}

{--¡Os voy a llevar a todos conmigo! --gritó Balfour a través de sus
 dientes apretados. La saliva volaba de sus labios-- ¡Os voy a llevar a
todos conmigo!}

{El Doctor clavó sus largos dedos en el cuello de Balfour y apretó
 fuerte. Balfour apretó sus dientes por el dolor, con sus ojos dando
 vueltas alocadamente, pero no había forma de moverle. El Cartago comenzó
 a vibrar como si un gigantesco enjambre de avispas furiosas estuviera
arrastrándose por el puente, buscando una forma de salir.}

{Clara y Hobbo sujetaron a Balfour, cada una intentando arrancar una de
 las manos como garras de los controles, pero parecía ser capaz de
combatir a ambas y al Doctor.}

{--¡No dejéis que despegue! --volvió a gritar el Doctor. Su voz casi se
 perdió en el rugido de los antiguos motores cuando se les forzó a entrar
en acción.}

{Hubo una repentina y brillante luz de luz morada que hizo un arco a
 través de la sala y golpeó a Balfour directamente entre los omoplatos.
 Lanzó su cabeza hacia atrás con un jadeo y entonces se hundió de
rodillas.}

{Tibby Vent estaba sujetando el fusor de iones ante ella con las dos
manos y su cara sorprendida.}

{--¡Lo siento! --dijo-- ¡Es lo único en lo que he podido pensar!}

{--También has hecho un buen trabajo --dijo Hobbo, sin aliento. Empujó a
 Balfour lejos de los controles y el hombre aterrizo a cuatro patas, con
la saliva cayéndole de la boca.}

{Jem le observaba con una mirada de fascinación y horror.}

{--Ese no es Balfour --susurró ella.}

{--No le miréis --ordenó el Doctor--. No le miréis, ninguna de
vosotras.}

{Clara apartó la mirada de forma obediente. Miró al Doctor, en su lugar,
 aunque era aún consciente de Balfour, agachado en el suelo, jadeando
como un perro.}

{--Tibby y Jem tienen razón, ese no es Raymond Balfour --dijo el Doctor,
 alisándose la chaqueta--. Es algo llamado el Glamour, tomando su forma.
 La Imperfección que los Faerón están intentando enterrar. Está
 intentando escapar de sus captores subiéndose a una salida del pozo
temporal.}

{--Pero creía que habías dicho que el Cartago no podía volar --dijo
Tibby.}

{--No puede, al menos no lo bastante lejos. Esta nave ya no puede
 soportar el espacio. Millones de años de decadencia entrópica no pueden
quitarse de encima en unos pocos minutos.}

{Hobbo parecía estupefacta.}

{--Pero yo realmente creía que podría volar. Todos lo hicimos.}

{El Doctor asintió.}

{--Ese es el Glamour trabajando, haciendo lo que mejor hace: cumpliendo
 tus deseos, tomando lo que tú quieres creer y amplificándolo y
 controlándolo. Absorbe cada deseo que tienes, el deseo en tu corazón y
 te lo lanza en la cara. Todos queríamos creer que había una salida del
 pozo temporal. Todos queríamos creer que el Cartago, habiendo estado
 perdido y esperando durante millones de años, podía de repente poder
 soportar el espacio y llevarnos volando hasta ponernos a salvo. Qué
romántico. Qué conveniente. Qué imposible.}

{--¿Pero qué vamos a hacer? --preguntó Clara, intentando mantener la
 desesperación fuera de su voz. Podía notar su corazón golpeándole en el
 pecho. El Cartago estaba desmoronándose a su alrededor y no había forma
de salir-- ¿Cómo podemos escapar?}

{--En la TARDIS, por supuesto.}

{Clara se giró hacia donde la TARDIS se alzaba en la parte trasera de la
 habitación. La cabina de policía se alzaba resplandecientemente sólida y
 azul, sin ningún rastro de pintura despegada o matojos de hongos. Las
 ventanas estaban despejadas. El cartel por encima de las puertas
brillaba con su normal reafirmación constante.}

{--Vuelve a estar ahí --dijo Clara, con sus ojos húmedos ante la
imagen--. Como debería ser.}

{--La forma en la que siempre ha estado. El Glamour no quería irse en la
 TARDIS.\@ No sería capaz de mantener la ilusión de sí mismo dentro de la
 TARDIS.\@ Su forma verdadera e intenciones se revelarían al instante. No,
 tenía que convencernos de no usar la TARDIS y de que el Cartago era la
única forma de escapar.}

{--Bueno --dijo Clara, sorbiéndose los mocos y limpiándose los ojos--. A
mí me convenció.}

{--¿Cuándo te has dado cuenta, Doctor? --preguntó Tibby.}

{--En cuanto la TARDIS volvió a mí gracias a los Faerón. No veía lo que
 veíais vosotras. La TARDIS y yo somos demasiado cercanos como para
 dejarnos engañar por ese tipo de basura y, además, es una máquina del
 tiempo. ¿Creéis que un millón de años podrían siquiera tocarla?
¡Miradla! ¡Grande, gruesa, hermosa y azul! Nunca me abandonará.}

{--Pero\ldots{} --dijo Clara.}

{--¿Por qué le seguí el juego a la ilusión del Glamour? --los profundos
 ojos del Doctor se estremecieron durante un momento-- Tenía que obligar
al Glamour a mostrarse. En ese punto, no sabía quién era de verdad.}

{--Yo sí --dijo Tibby, en voz baja--. Sabía que algo no iba bien. No era
 Ray Balfour. No se parecía en nada al hombre que yo\ldots{} al hombre
que conozco.}

{--Me temo que todo se volvió demasiado obvio al final --admitió el
Doctor.}

{--¿Pero y si el Cartago hubiera despegado? --preguntó Hobbo-- ¿Qué
habría pasado?}

{--Con suerte habría escapado del pozo temporal, pero sus motores habrían
 fallado y el soporte vital con ellos. Se habría vuelto un naufragio, a
la deriva en el espacio profundo. Y todas vosotras habríais perecido.}

{--¿Qué hay del Glamour?}

{--El Glamour no necesita aire para respirar, ni comida ni agua. Podría
 sobrevivir cientos de años, un centenar de miles, durante toda la
 eternidad, quizás. Pero alguien o algo lo habría acabado recogiendo,
 probablemente buscando el deseo más profundo de su corazón. El Glamour
 se habría convertido en lo que quiera que quisieran, y más. Y entonces
 se habría alimentado de esos deseos, y se había fortalecido, más
 poderoso e influyente --el Doctor parecía sombrío--. Lo peor de los
Faerón se habría quedado a sus anchas en un universo desconfiado.}

{--Pero,. ¿qué hay del Ray Balfour real? --preguntó Tibby-- ¿Dónde
está?}

{--No quiero ni pensarlo. Quizá deberíamos preguntárselo al Glamour.}

{Pero cuando el Doctor bajó la mirada no había rastro de la cosa que
había tomado la forma de Raymond Balfour.}

{--¿A dónde ha ido?}

{--No lo sé --dijo Clara--. Nos dijiste que no le miráramos.}

{--Doctor, también Jem se ha ido --dijo Hobbo, señalando la silla vacía
del piloto.}

{La cara del Doctor se volvió una expresión de horror.}

{--Oh, no. Eso no es bueno, para nada lo es.}

{--Tenemos que encontrarla --dijo Clara.}

{Jem siguió a la
 figura coja a través del pasillo oscurecido. Se movía lentamente, con
 una cojera irregular, y estaba claramente herido. Jem no sabía a dónde
 se dirigía, pero parecía estar llevándola más y más profundo en la nave.
 El ruido de los motores del Cartago estaba aumentando en todo momento.
 Gruñía bajo el suelo, haciendo vibrar todo con un daño que hacía que
castañetearan los dientes.}

{--¿A dónde estás yendo? --preguntó ella, apresurándose tras la sombría
figura.}

{Él no se detuvo, no se giró. Siguió cojeando, arrastrando un pie tras
 él, sujetándose en las paredes para tener apoyo. Cada vez que Jem
 pensaba que estaba a punto de atraparle, desaparecía en la siguiente
 esquina, o bajaba por unas escaleras. En un momento, tuvo que obligarse
 a cruzar una cortina de vegetación que había crecido por las paredes y
 por el techo colgando en unas secas y largas raíces. Más allá, había una
 sala más grande, llena de gruesos cilindros como montones de monedas
 gigantescas. El ruido allí era peor, y parecía como si el mundo entero
 estuviera temblando. El humo se arremolinaba alrededor de la base de los
cilindros mientras latían con luz azul.}

{Había una figura tumbada en el suelo, acurrucada. Jem vio la mata de
 pelo rubio y se dio cuenta de que era Raymond Balfour. Quizá estuviera
muerto. No importaba. Balfour no era a quien estaba buscando.}

{--¿Dónde estás? --gritó Jem.}

{Una sombra se movió tras los cilindros.}

{--Sé que estás ahí --dijo Jem--. ¿Por qué no sales? Sé quién eres.}

{La sombra vaciló, se giró. Caminó lentamente hacia adelante, hacia la
luz de los cilindros.}

{--Siempre sabía quién eras --dijo Jem.}

{--No me he dado cuenta de que eras tú --dijo Dan Laker. Parecía cansado,
 demacrado, y había moratones en su cara. Seguía llevando su traje
 espacial, aunque parecía dañado y por supuesto le faltaba el casco,
 enterrado en la nieve en un millón de años. Laker se sujetó
dolorosamente, se acercó, herido, aun cojeando.}

{--¿Creías que te iba a dejar? --preguntó Jem, dando un paso más cerca.}

{--No --dijo él--. Jamás.}

{Se acercó, enderezándose un poco. Bajo la luz parpadeante los moratones
 parecían menos obvios. Se pasó una mano a través del pelo de la forma
que ella recordaba y le sonrió amablemente.}

{--Te he echado de menos --dijo Jem.}

{--Nunca he pretendido hacerte daño --le dijo Laker--. Los Faerón me
 quitaron del Alejandría cuando caí por el precipicio. Nunca creí que te
vería de nuevo.}

{Jem dio un paso delante de nuevo, lo bastante cerca como para tocarle.
 Cuanto más se acercaba, mejor parecía: los moratones desaparecieron, los
cortes desaparecieron. Se alzaba más alto y más fuerte.}

{--El Cartago va a explotar --dijo Jem--. Tienes que salir de esta
nave.}

{--Déjame ir contigo --dijo él, alargando una mano.}

{--¡Jem! --la voz del Doctor sonó claramente desde detrás de ella-- No
dejes que te toque.}

{Ella se giró para encontrarse al Doctor y a Clara, tras ella, en la
 entrada de la sala de motores. Tibby y Hobbo estaban con ellos. Tibby
 corrió hacia donde Ray Balfour estaba tumbado en el suelo y le
examinó.}

{--Está respirando. Está vivo. ¡Gracias a Dios!}

{Hobbo ayudó a Tibby a levantarle. Parecía mareado pero sin heridas.}

{--¿Qué está pasando?}

{--Algo está a punto de golpear el ventilador --dijo Hobbo--. Preparaos
para agacharos.}

{El Doctor tenía sus manos extendidas hacia Jem, sus dedos estirados,
como si pudiera controlarla como un titiritero.}

{--Jem, espera. Piénsalo con cuidado.}

{--Lo he hecho --respondió ella.}

{Laker retrocedió, entrando un poco en las sombras.}

{--Dejadnos solos --dijo Jem--. Por favor, marchaos.}

{--Pero ese no es Dan Laker --le dijo Clara--. Es el Glamour. Está
haciéndote creer que es Dan, pero no lo es.}

{--No les escuches, Jem --dijo Laker.}

{Uno de los motores se abrió agrietándose y una llama brillante como el
 sol apareció en su interior, enviando unos largos dedos de luz para
tocar cada superficie.}

{--Los motores del Cartago se están sobrecalentando --gritó Hobbo.}

{El agudo zumbido de una maquinaria antigua había ido in crescendo
mientras hablaban.}

{--¡Esos propulsores astrónicos van a encenderse literalmente por primera
vez en eones!}

{El Doctor, convertido en una silueta con el brillo alocadamente cegador,
se adelantó y agarró a Jem por el brazo.}

{--¡Debemos irnos ahora!}

{--No, Doctor --Jem tuvo que gritar por encima del ruido--. ¡Tengo que
estar aquí! ¡Me quedo con Dan!}

{--¡Ese no es Dan Laker! --repitió el Doctor, fieramente.}

{El suelo se sacudió de repente y otro motor comenzó a encenderse. Unos
 dedos ardientes de energía cayeron por los suelos, encontrando motores
 cercanos, comenzando a hundirse en ellos. El Doctor comenzó a moverse
 hacia adelante pero un brillante arco de energía blanco se agrietó entre
 él y Jem. Clara agarró al Doctor y le echó para atrás, gritando para que
 se alejara. Jem alargó su mano hacia Laker. Él la cogió en brazos y
sonrió al Doctor y a Clara.}

{--¡No! --gritó Clara.}

{--No es Dan Laker --dijo el Doctor, de nuevo, con su voz llena de
desesperación.}

{--Lo sé --dijo Jem. Y entonces endureció su agarre en Laker, hundiendo
sus dedos para asegurarse de que no pudiera alejarse.}

{Un brillo azul les envolvió a ambos, unas raíces de energía emanando
 para conectarse con los motores. El brillo se profundizó, y fuera del
brillo azul salieron tres figuras altas y encapuchadas.}

{--¡Los Faerón! --dijo Clara, conteniendo el aliento.}

{Hacemos el viaje final.}

{El Doctor parecía asombrado.}

{--No, no puedes\ldots{}}

{--Tengo que asegurarme --dijo Jem, aun sujetando a Laker--. Tengo que
 asegurarme de que los Faerón se lo llevan con ellos. Tenemos que ir más
profundo en el tiempo, juntos.}

{Los Faerón alargaron sus brazos hacia Laker y Jem. Laker rugió,
 girándose de repente de un lado a otro, pero Jem le contuvo rápidamente
 en sus brazos. La cara de Laker se alargó, volviéndose aquilina y
 afilada como un pico, y sus ojos desaparecieron en unos anchos agujeros
 negros a cada lado de su cabeza sin piel. Clara había visto una vez la
 calavera de un cuervo y no parecía muy diferente. El afilado pico se
 agrietó y los hombros se alzaron tras él, de repente tenía joroba y era
 más delgado. Las cuencas de los ojos se llenaron con unas bolas de cosas
 grises húmedas y un fino y entusiasta quejido se alzó por encima del
ruido atronador de los motores.}

{Y entonces los Faerón retrocedieron hasta el brillo azul, llevándose a
 la cosa entusiasta con ellos y también a Jem, y entonces más motores se
abrieron y la energía desatada se sacudió por la sala.}

{--¡Por el amor de Dios, salgamos de aquí! --Hobbo agarró tanto al Doctor
 como a Clara y les arrastró. Se giraron y, junto con Tibby y Balfour,
 corrieron por sus vidas mientras el Cartago se desintegraba a su
alrededor.}
