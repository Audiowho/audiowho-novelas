\chapter*{Un día}
\addcontentsline{toc}{chapter}{Un día}

Llovía allí en la colina, el continuo chop, chop, chop golpeando la gran
sombrilla de golf como balas en una lata. Para ser sinceros, llovía en
todas partes, pero arriba en la colina, allí en el huerto, era el único
lugar dónde Wilfred Mott se preocupaba de verdad por si llovía o no.

Siempre que llovía, no podía evitar recordar. Aquél terrible, terrible
día cuando él llegó a casa, trayendo a Donna consigo. Inconsciente,
incapaz de recordar nada. Por su propia seguridad.

Wilf recordó aquella última vez que vio al Doctor, empapado por la
lluvia, con su cara surcada de agua, el pelo chorreando y las ropas
húmedas y pegadas a su cuerpo larguirucho. Y sus ojos, sus ojos parecían
tan atormentados, tan tristes, tan perdidos. Y tan, tan viejos. Parecían
los ojos de un hombre mayor, atrapado en un ridículo cuerpo joven. Tan
mísero, tan solo y solitario.

Entonces esa maravillosa cabina azul se había desvanecido mientras Wilf
le saludaba.

Y nunca había vuelto a ver al Doctor.

Pero eso no le impedía mirar, allí arriba. Arriba en el cielo nocturno,
a las estrellas que sólo seguían allí gracias a Donna. Hacia las
estrellas que calentaban e iluminaban incontables planetas, con
incontables vidas que debían su existencia a Donna Noble.

No quería pensar en eso. No lo entendía del todo, él sólo confiaba en el
Doctor con su vida. Y el Doctor merecía esa confianza porque les había
salvado a todos.

De las naves espaciales del cielo, de la estrella de Navidad, del
gigantesco Titanic, de los adiposos, de los Sontaran y de los Dalek.

Y aquellos eran los únicos que él conocía. Sabía por Donna, la Donna que
había sido antes, que había habido incontables más.

Negó con la cabeza al pensar en todo lo grandioso de aquello. Y en lo
pequeño e insignificante que era él en comparación. Pero a él no le
importaba. Porque el honor era haber conocido al Doctor.

Rebuscó en su bolsillo empapado y sacó una cartera de cuero. Y de su
interior, sacó unas cuantas fotografías.

Una mostraba a Donna el día de su boda, la boda que nunca tuvo lugar.
Donna creía que nunca había llegado al altar porque el pobre Lance se
había visto en medio de todo lo que pasó con la estrella de Navidad
atacando las calles de Londres. Lance murió entonces, esa es la historia
que le habían dicho. De hecho, es la historia que Wilf contó a todo el
mundo. Y también le dijo a Donna que había estado tan traumatizada por
la muerte de Lance, que se había ido a Egipto para superar el mal trago.

Fuera lo que fuera lo que el Doctor había hecho a sus recuerdos, hizo
que el cerebro aceptara la historia y encontrara una forma de encajarlo,
por lo que estaba convencida de que eso era lo que realmente había
pasado. Quizás era eso lo bueno que había salido del ``accidente'', que
su cerebro hacía eso para enfrentarse a ello, más que a un año en
blanco, si le dejabas caer una idea, ella era capaz de racionalizarlo
sin preguntar. Como un rompecabezas donde no encajaran las piezas sino
que se transformaban para encajar, y formaban una imagen ligeramente
distinta que Donna nunca se atrevería a preguntar.

Otra fotografía era de Donna con su madre y su padre en la cena del
cumpleaños de su padre en la ciudad. Resultó ser su último cumpleaños.

La última imagen era de una mujer mayor en una silla de mimbre, con un
vaso de cerveza negra en la mano, brindando al fotógrafo.

Wilf suspiró. Demasiada tristeza en la casa de los Noble durante el
último par de años.

Apartó las fotografías y volvió a echar un vistazo al telescopio. Nada
que ver.

---¿Cómo está el cielo esta noche? ---preguntó una voz por detrás.

---Hola, cariño ---dijo Wilf, indicando a la recién llegada que se
uniera a él bajo la sombrilla---. ¿Qué haces aquí? Te vas a enfríar.

---Oh, estoy bien, Papá ---dijo Sylvia Noble, pasándole un termo---. Te
he traído un poco de té y una tableta de chocolate.

Wilf cogió agradecido el termo, y se sentaron en silencio durante un
rato, dejando que la lluvia creara una sinfonía por encima de ellos.
Entonces destapó el termo y se lo ofreció a su hija. Ella negó con la
cabeza.

---Donna hacía esto ---murmuró.

Sylvia asintió.

---Lo sé. Quizá necesite que le estimule la memoria y vuelva a hacer lo
mismo.

Wilf se encogió de hombros con tristeza.

---Mejor que no, ¿eh? Sólo por si acaso.

Sylvia cambió de tema.

---¿Entonces no hay cabinas azules en el cielo esta noche?

---Hoy no. Pero un día le veré.

Hubo una pausa.

---¿Te importa de verdad? ¿Después de todo lo que ha hecho por
nosotros?

---Sí, cielo, me importa ---dijo Wilf---. Necesito saber que sigue ahí
afuera, vigilando todavía por nosotros.

Vigilando el universo. Porque entonces sabré que Donna sufrió por algo
que vale la pena. Pero sin él, no estamos seguros.

---Eso es demasiada fe para poner en un solo hombre, Papá ---dijo Sylvia
con tranquilidad---. Y mucha responsabilidad.

Wilf sabía que a Sylvia no le gustaba el Doctor, y no sólo por lo que le
había pasado a Donna. Ella pensaba que si el Doctor nunca hubiera venido
a la Tierra quizá aquellos Dalek tampoco lo hubieran hecho. Era una
larga discusión, y ellos nunca se pondrían de acuerdo. Por ello,
intentaban no hablar demasiado del Doctor.

---Está ahí fuera, cariño. Protegiéndonos. Y a los marcianos y a los
venusianos. Y Dios sabe a quien más ---tomó un sorbo de té---. Deberías
volver para calentarte.

Sylvia asintió y se levantó.

---¿Te vas a quedar mucho rato?

---No, sólo quiero quedarme hasta las once, entonces me iré.

---Donna ha sugerido ir a visitar a Netty mañana.

Wilf bajó el té.

---No gracias ---dijo rápidamente.

---Papá, tendrás que verla en algún momento ---Sylvia se estiró y apretó
su mano---. Si no es por tu bien, hazlo por el suyo.

---No deberías dejar que Donna fuera ---dijo Wilf---. No es seguro. ¿Qué
pasa si dice algo del Doctor?

---No lo creo. Y aunque lo hiciera, Donna no lo entenderá y Netty no
podrá explicarlo ---Sylvia se levantó y caminó por la lluvia. Entonces
miró atrás a su viejo padre---. Hemos pasado por cosas por las que nadie
debería pasar, Papá ---dijo con calma---. No provoquemos más nosotros.
Ven, por favor.

---Me lo pensaré. Ahora vete, antes de que te enfríes.

Sylvia señaló al cielo.

---El Doctor lo querría ---dijo.

Wilf se giró, frunciendo el ceño.

---Eso no es cosa tuya, cielo. Por favor no digas eso.

Sylvia asintió.

---Lo siento, Papá ---y bajó por el huerto y la colina.

Wilf vio su silueta empequeñeciéndose hasta que se perdió de vista,
entonces desenvolvió su tableta de chocolate y mordió un pedazo. Se giró
para volver a mirar otra vez por el telescopio, preocupado porque Sylvia
había hablado del Doctor. Preocupado porque era un golpe bajo. Y
preocupado porque Sylvia tenía razón.

Una lágrima cayó por la mejilla de Wilf.

Por muchas razones.
