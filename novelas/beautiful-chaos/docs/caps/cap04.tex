\chapter*{Domingo}
\addcontentsline{toc}{chapter}{Domingo}

Cuando era niña, Donna había oído la frase ``el disparo que se escuchó
en todo el mundo'' usada para describir el efecto que el asesinato del
presidente estadounidense John Kennedy tuvo en toda la civilización
occidental. La gente siempre decía que podían recordar dónde estaban
cuando ocurrió.

Como hija de los 70, creció oyendo hablar de cosas como el aterrizaje en
la Luna, los asesinatos de ambos Kennedys y Martin Luther King y el
funeral de Estado de Winston Churchill, pero nunca los entendió del
todo. Con una infancia de Pelotas de Goma, Donny Osmond, bicicletas
chopper y sellos Green Shield, palabras como `Bombardeo',
`racionamiento' y `puré de patatas mezclado con col' sólo significaba
que los ancianos estaban recordando cosas de veinte años atrás y,
probablemente, quejándose de que la juventud de hoy nunca sabría como lo
tenían de bien.

La primera vez que Donna se había encontrado a sí misma diciendo
aquello, a uno de los hijos de los vecinos que había hecho un rasguño al
último coche de su padre, se quedó boquiabierta. Por fin se había
convertido exactamente en lo que se mofaba en sus padres y abuelos
cuando tenía su edad. Hoy en día no había nada que la gustara más que
escuchar el Abuelo Wilf hablar de la guerra, su vida en el regimiento de
paracaidistas o los días de la abuela Eileen como miembro del ejército
terrestre femenino.

Hoy era un día como ese 22 de noviembre 1963, un día en que otro disparo
se oyó en todo el mundo.

El Domingo, para ser honesta, había comenzado bastante mal. Donna había
despertado en su cama (esto era una buena cosa), aunque sólo había
tenido unas dos horas de sueño (esto era una mala cosa).

El vestido hecho jirones de Veena estaba hecho una pelota en el suelo
(mala cosa); junto a él, un recibo que le había dado el taxista (mala
cosa, pues ¿a ver a quién le reclamaría todo aquello?) que la había
llevado desde un lugar llamado South Woodham Ferrers de regreso a
Chiswick. El recibo era de 225 libras (muy, muy malo, su cuenta debía
estar vacía ahora). Wilf había estado despierto cuando volvió (muy buena
cosa) y la había escuchado cuando le contó todo lo que había sucedido en
el Observatorio Copérnico. La había abrazado fuerte, le había prometido
que encontrarían una manera de rescatar al Doctor y para rematar la
envió a dormir.

Donna se había sentido furiosa consigo misma, había dejado al Doctor
allí, le había abandonado de una forma que él nunca le haría a ella
(mala cosa). Pero también había sido práctica. Él le había dicho que se
fuera, y eso había sido un acierto, porque de lo contrario la habrían
matado (definitivamente una mala cosa). Donna se sentía como muerta,
pero era necesario que consiguiese ayuda y volviera al Observatorio.

Quizá podría localizar a Martha Jones y sus compañeros de UNIT, le
gustaba Martha y sabía que lo dejaría todo para ayudar. Al llegar al pie
de la escalera, cogió las Páginas Amarillas y ya estaba a la mitad de la
U cuando se dio cuenta de que sería poco probable que UNIT apareciera
ahí, en la categoría de Organizaciones Militares Dedicadas a Hacer
Desaparecer a los Marcianos (ni bueno ni malo, sólo un poco de gatillo
fácil).

---Buenos días, señorita ---le espetó Sylvia Noble desde el otro lado
del pasillo---. Estaba preocupada por ti anoche.

---¿Por qué?

---Oh, no lo sé. Volviendo de puntillas al amanecer tras una noche de
discoteca con el Doctor? ¿A tu edad? Quiero decir, ir de discoteca es
genial cuando tienes 21, pero cuando la mediana edad está a sólo unos
pocos cumpleaños de distancia\ldots{}

---Ey.

---Es Igual. La cosa es, que es hora de crecer, jovencita.

---Gracias, mamá. Demasiado mayor para irse de discoteca, pero no
demasiado mayor para vivir con mi madre. ¡Genial!

---Nadie te obliga a vivir aquí ---dijo Sylvia, poniendo una taza de té
en manos de Donna---. No es que lo hagas mucho estos días. Te vas con
Papá Piernaslargas semana sí y semana también ---Sylvia tiró de la bata
de Donna, apretando el cinturón, enderezando el cuello---. ¿Y dónde está
él ahora? Seguro que se lo ha llevado tu abuelo al huerto, sin duda. Y
no es una cálida mañana, y se ha dejado aquí el termo ---Sylvia se lamió
el dedo índice y borró una marca de la mejilla izquierda de Donna---.
Aunque me imagino que estarán de vuelta para la hora de comer. ¿La carne
asada de cada domingo y toda la pesca? ¡Ja! Qué mas quisieran. Te diré
qué: un viaje al Jolly Lock Keeper y su todo-lo-que-puedas-comer por
diez libras es lo que haremos hoy, chiquilla. Déjame decirte que
aquellos días de tenerme esclavizada con la ternera y el Pudding de
Yorkshire se fueron con tu padre ---Sylvia puso el pelo rojo de Donna
detrás de las orejas y le apartó el flequillo---. Y había una nota para
ti anoche, la encontré cuando tu abuelo me despertó, tambaleándose
cuando le dejaste. Yo no la leí, pero es de esos chicos Carnes sobre los
que el Doctor estaba hablando.

Donna quería preguntarle cómo sabía de quién era si no la había leído,
pero de esa manera eso le llevaría a otras cuestiones sobre notas de los
directores cuando Donna tenía doce años a cartas de Martyn Hart cuando
tenía quince y quién las había abierto cuando estaban dirigidas a otra,
así que se mantuvo callada. Donna también tuvo una punzada de la carne
asada y del Pudding de Yorkshire, evocados por su madre, cubiertos con
un fantástico jugo de carne.

Papá cortandolo. El abuelo y la abuela visitándolos los domingos,
hablando de horarios de trenes y de clubs de coches y largos paseos por
el Parque Windsor Great. De repente quería volver a tener diez años. Y
quería llorar.

---¿Mamá?

---¿Qué?

---Echo de menos a papá.

Y Sylvia Noble abrazó a su hija de una manera que no lo hacía desde
hacía mucho tiempo. Después se apartó, casi como si hubiese recordado
que la programación de Sylvia Noble era ser gruñona e intransigente y no
cercana ni cálida. Especialmente con su descarriada hija.

Pero fue suficiente momento para hacer feliz a Donna. Porque había sido
un verdadero gesto instintivo.

---¿Puedes llamar a tu abuelo, por favor, averiguar cuándo va a regresar
y si tu Doctor se unirá a nosotros en el pub?

---¿Y Netty? (Ooh, algo muy malo)

---Si debemos\ldots{}

Deseosa de cambiar el estado de ánimo, Donna abrió la puerta para
comprobar el tiempo.

---Mamá, ¿por qué dices que el abuelo se ha ido al huerto?

---¿Qué hace sino? Está o con las verduras o con Netty. Cuando no son
las dos lo mismo. Donna le lanzó una mirada y Sylvia tuvo la decencia de
disculparse.

---Lo Siento. Estoy acostumbrada a estar tanto sola, que me olvido de
que no debería decir a otras personas las cosas que me digo a mi misma
en alto.

---Ya nos preocuparemos de ti y Netty más tarde. El abuelo se ha llevado
el coche. Que no necesita para ir al huerto.

Sylvia frunció el ceño.

---No me dijo que se lo llevaba. Dijo que iba a reunirse con el Doctor.

---¿Así que has supuesto que estaba en el huerto?

---Como la otra noche, sí.

Donna llamó a su abuelo con su móvil, pero estaba apagado. El muy tonto
estaba de camino al Observatorio Copérnico, ¿verdad? Y había dejado
atrás a Donna. Y en ese momento se produjo ``el disparo que se oyó en
todo el mundo''.

Donna siempre recordaría que estaba vestida con su bata, de pie en una
puerta ligeramente abierta, con una nota de los chicos Carnes en la mano
(no leída, al menos no por ella), mirando a un espacio en el que un
coche debería haber estado, con su madre justo detrás de ella. Había una
taza de té recién hecho junto a las páginas amarillas. Por el camino, el
anciano Señor Lyttle estaba paseando a su perro. Una pequeña cosa negra
de raza indeterminada que siempre olía a pelo mojado. A la izquierda,
con una de esas cosas de visión periférica de verdad, había una caravana
azul aparcada. Y dominando el cielo azul había una enorme, atroz y
terrible columna de pura luz brillante, envuelta con un tenue brillo
púrpura. Donna oyó a su madre decir:

---Oh, Dios mío, otra vez no, el cielo en llamas otra vez no.

Pero no estaba totalmente en llamas. Era sólo una columna, acompañada de
un sonido parecido a la llama de gas de la cocina, pero intensificada en
diez mil decibelios.

Donna sólo sabía que algo terrible estaba sucediendo en esa columna de
luz que golpeaba el suelo, en algún lugar al oeste de Chiswick. Aunque
ella no lo sabía entonces, en todo el mundo, no importaba en qué zona
horaria, similares columnas de energía calorífica estaban haciendo lo
mismo. Y en el cielo, esa cara lasciva de estrellas todavía seguía allí,
la horrible mueca aparentemente más ancha y más amplia que nunca.

---Mamá, entra dentro, ahora. Cierra las puertas, que no entre nadie
excepto yo. O el Abuelo. O el Doctor. Especialmente el Doctor.

---¿Y por qué es tan especial?

---Oh, sólo di que lo harás, ¿vale?

---De acuerdo ---murmuró Sylvia---. ¿Y a dónde vas?

---Tengo que intentar encontrarle. En realidad a los dos.

---¿No están juntos?

---Eso es lo que me preocupa.

---Bueno, no puedes salir así. Donna se dio cuenta de que todavía no
estaba vestida y corrió escaleras arriba, tirando su bata apenas cruzó
la puerta de su habitación.

---No dejes eso tirado en el suelo ---dijo Sylvia.

Donna lo recogió, lo colgó en el gancho de la puerta y suspiró.

---Prioridades, mamá ---murmuró.

Vestida con una camiseta y un viejo chándal, que no podía creer que lo
hubiera llevado alguna vez, pero sabía que iba a estar caliente, Donna
volvió a bajar las escaleras, cogiendo un pesado abrigo. Fuera en la
calle, podía oír a la gente y a los coches acelerando. Todo el mundo
había visto la columna de luz y ahora estaban viendo esa cara horrible
en el cielo.

Dales una media hora y entrarán en pánico por las calles, provocando
disturbios, saqueos y con policía en todas partes. Tenía que salir de
Londres rápidamente.

---No tengo coche ---se maldijo para sus adentros.

Abrió la puerta de entrada y vio esa furgoneta azul. Eso está muy mal.
Muy mal, Donna Noble. Muy mal. Después estaba en la ventanilla del
conductor, mirando el asiento. El salpicadero. La falta de llaves. La
puerta cerrada. La pobre señorita Oladini había hecho que pareciera tan
fácil cuando había puenteado ese coche, pero Donna no tenía ni idea de
por dónde empezar.

---¡Genial!

Intentó abrir la puerta por si acaso. Estaba abierta. Levantó la vista a
la calle, pero nadie en la turba de gente le gritaba o reclamaba la
furgoneta como suya. Se tiró en el asiento del conductor y metió la mano
bajo el asiento para ajustarlo. Dios sabía por qué lo hizo, no iba a ir
a ninguna parte, porque nadie en esta época era tan estúpido como para
poner las llaves debajo del asiento en una camioneta abierta. Sacó las
manos, con un manojo de llaves del coche en ellas.

---Y quiero un triciclo, un pony y suministro de chocolate con leche
para toda la vida ---dijo en voz alta, poniendo su mano atrás bajo el
asiento por si acaso sus deseos de Navidad de cuando tenía ocho años se
hacían también realidad. No hubo ponis, ni bicicleta, ni siquiera una
barra de chocolate derretido. Pero sí las llaves, eso era bueno.

Salio con la camioneta marcha atrás, y segundos después estaba de
regreso a Chiswick High Road, planeando su segundo viaje hacia Essex en
doce horas. Echó una última mirada a su casa por el espejo retrovisor
mientras daba media vuelta con la furgoneta y después salía disparada,
esperando que su madre no la hubiera visto hacerlo. Porque después lo
pagaría caro. ¡Y con razón! Antes de que incluso hubiese llegado a la
carretera principal, la multitud estaba en la calle, mirando y
señalando, y podía oír sirenas de ambulancias, la policía y camiones de
bomberos a su alrededor, todos dirigiéndose al oeste, hacia la M4.

Hacia donde la columna de luz había golpeado el suelo.

Se dirigía hacia Londres y ese lado de la calle estaba relativamente
vacío, incluso para un domingo. La atención de Donna se desvió a causa
del número de personas que estaban fuera de las diversas tiendas de
electrónica que salpicaban ambos lados de la calle. Chiswick High Road
había tenido en su mayoría cafés y tiendas de ropa cuando era pequeña,
pero esta invasión de tiendas de chismes era extraña. Recordó al Doctor
comentando que había conocido a los chicos Carnes en una. Todo esto pasó
por su mente por un breve segundo, probablemente porque enfrente de ella
en la calle se encontraban Lukas y Joe Carnes.

Como si la hubieran estado esperando. Literalmente. De pie en la calle.
Un momento, la carretera estaba vacía. Al siguiente, dos muchachos
estaban justo enfrente de ella.

Donna pisó el freno, y justo evitó derrapar en seco, realizando de hecho
una parada bastante grácil, aunque un hombre detrás de ella hizo sonar
el claxon.

---¿Sí? ¿Qué coño recibiste por Navidad, rey? ---le gritó---. ¿Por qué
no te lo metes por \ldots{}?

La puerta del pasajero se abrió, y Joe y Lukas entraron.

---Joe dice que tenemos que ir a un sitio llamado ``Copérnicus'' ---dijo
Lukas tranquilamente---. También sabía que estarías aquí. En este
momento.

---Por supuesto que sí ---respondió Donna, avanzando mientras el
conductor furioso la adelantaba, con una mano fuera del volante y
haciéndole gestos. Encogiéndose de hombros, Donna siguió conduciendo
hacia Hammersmith.

---Buenos Días, Joe ---le dijo al chico, que ahora estaba en la parte de
atrás.

Joe no respondió, pero sacó algo de su bolsillo.

---¿Qué es eso? ¿Un nuevo chisme MP3?

---Es un M-TEK ---respondió Lukas en nombre de Joe.

---¿Un qué? ---Donna trató de sonar interesada, pero no lo estaba.
Estaba más centrada en cómo habían sabido que ella estaría allí.

---Le dijo dónde estarías ---continuó Lukas---. Le habla.

Esto de alguna manera respondía a su pregunta, decidió Donna, pero
inquietantemente le planteaba un par de docenas más.

---Entonces, ¿es así cómo supo el nombre del Doctor el otro día?

Lukas se encogió de hombros.

---No sé. El hombre en la tienda se lo dio a él. Dijo que era una
versión de prueba. Dio unos diez de ellos. Dijo que Joe era la persona
perfecta para tener uno. No me lo dijo hasta que llegamos a casa y me lo
encontré descargándose música.

Eso tenía sentido para Donna, aunque en realidad no tenía ningún sentido
en absoluto. Cuando viajabas con el Doctor, comenzabas a aceptar que las
cosas que no tenían sentido realmente tenían sentido en una forma de
no-tiene-sentido-para-la-gente-normal. Así que aquel aparato del M-TEK
hacía que Joe Carnes supiera cosas. O le decía cosas. Cosas que atraían
la atención del Doctor.

---¿No os dijo vuestro padre que no se aceptan regalos de hombres
desconocidos?

---Mi padre lo hizo ---dijo Lukas, mirando a Joe---. El padre de Joe no
se quedó el tiempo suficiente.

``Bien'', pensó Donna, ``esto mata una conversación''. Entonces hizo un
giro repentino hacia la rotonda de Hammersmith que provocó que alguien
hiciera sonar el claxon. Quizá fuese el mismo conductor de antes, pero
ni lo sabía ni le importaba. Giró hacia la carretera de Talgarth.

Estaba vacía. Realmente vacía. Era una gran carretera de seis carriles
hacia el centro de Londres, a través de Earls Court después
Knightsbridge, a continuación, Hyde Park y, finalmente, Piccadilly.
Debería haber tardado veinte minutos, tal vez treinta llegar a
Piccadilly en un domingo a la hora de comer, y eso era sin ningún tipo
de obra en la carretera. Donna lo hizo en diez y tampoco es que fuera a
demasiada velocidad

Era como si todas las personas en Londres huyeran\ldots{} no, que fuesen
hacia algún lugar. Esa luz. Todos se estaban dirigiendo hacia eso.
Mirones, ansiosos por sacar fotos con sus móviles y decir `¡Eh, mirad,
vimos la carnicería!'. ¿O era algo más siniestro? En ese caso, ¿por qué
no estaba afectada?

---Perdonadme chicos, más infracciones de la ley\ldots{} ---Donna sacó
su móvil mientras conducía y llamó a su madre. No tuvo respuesta.

Eso no era una buena noticia. Y allí estaba ella, en una furgoneta
robada, conduciendo a través de un Londres desierto, hacia el Essex más
oscuro para rescatar a su abuelo y a su amigo de unos asesinos, incapaz
de contactar con su casa, con los Hijos de los Malditos a su lado.

---Salud, Doctor ---dijo a nadie en particular.

A unos veinte kilómetros de distancia de Donna y los chicos, había una
gran presencia policial y de ambulancias alrededor del área de Ruislip
Woods, con todavía más servicios de emergencia llegando desde la cercana
RAF Hillingdon. El gigantesco rayo blanco de energía había golpeado la
arboleda,uno de los bosques más protegidos de Gran Bretaña, aunque no
había mucho que proteger en estos momentos. Había un enorme cráter en
forma de cuenco gigantesco de un cuarto de milla de ancho, diezmando los
árboles, los pastos y los arbustos. Una pequeña ráfaga de humo flotaba
en el aire de la mañana y la multitud de sorprendidos espectadores se
amontonaban en las cercanías, en parte boquiabiertos, en parte en shock,
pero en gran parte por miedo.

¿Había sido un accidente de avión? ¿Una bomba de al-Qaeda? ¿Algo de la
base de la RAF había salido mal? ¿Habría víctimas? Oh, Dios mío, ¿mis
hijos estaban jugando aquí? ¿Alguien ha visto a mi perro, una mezcla de
Labrador? Disculpen, ¿han visto a mi marido, que estaba fuera
practicando jogging? ¿Has visto esa cara horrible en el cielo? ¿Es
algún efecto especial de película? Nunca confié en ese alto el fuego
del IRA\ldots{}~ La Sargento de Policía Alison Pearce estaba intentando
controlar a la multitud y a sus propios oficiales y dejar pasar a los
equipos de emergencia. El turno de mañana del Domingo le había parecido
una buena idea. Tres niños significaban que hacer las guardias nocturnas
no era posible, pero su madre les podía cuidar el domingo mientras ella
estaba de guardia. Normalmente, llegaría a casa a las diez, los veía
dormidos y los preparaba para la escuela por la mañana. Ya había llamado
a casa y advertido a su querida madre que esos cuidados de la abuela
deberían durar un par de días más. Sólo el papeleo de esto la mantendría
ocupada. Y eso suponiendo que alguna vez realmente se fuese de aquel
lugar.

---Ey, tu, ¿perdona? ---le gritó a un joven que estaba intentando pasar
bajo la cinta roja y blanca---. ¿Señor? No se puede pasar\ldots{}.~

El hombre no le hizo caso. La Sargento Pearce cogió la radio y llamó a
un par de colegas mientras ella misma pasaba bajo la cinta y corría tras
él.

---Bienvenido de nuevo ---dijo a\ldots{} bueno, a la nada. Sólo a las
finas cenizas blancas que una vez habían sido árboles y Dios sabe qué
más.


--Señor, tengo que pedirle que vuelva detrás de la cinta. Esto es la
escena de un crimen.

El hombre siguió ignorándola, y la sargento Pearce se dio cuenta de que
cinco personas más habían hecho lo mismo alrededor del perímetro.

--Chicos --dijo a la radio--, ¿qué está ocurriendo?

Uno de sus agentes le respondió.

--No los hemos podido parar, sargento, se han deshecho de nosotros.

La sargento Pearce suspiró y llegó dónde estaba el hombre, pero ahora
estaba de rodillas, estirado hacia el suelo lleno de cenizas.

--Bienvenida de nuevo --dijo otra vez.

Y tenía los dedos conectados con la tierra mientras la sargento Pearce
alargaba la mano para tocarlo.

Sintió algo parecido a un shock, pequeño y eléctrico, pero poderoso y se
encontró a sí misma a un par de metros de distancia, tumbada boca
arriba, moviendo la cabeza para aclararse.

El hombre estaba de pie, de espaldas al cráter de ceniza. La sargento se
dio cuenta de que el resto de la gente, ahora eran siete en total,
habían hecho lo mismo. Era como si estuvieran vigilando el lugar.

El joven agente que le había hablado por la radio estaba a su lado.

--¿Se encuentra bien, sargento? --dijo, ayudándola a levantarse.

Lo apartó de un empujón.

--Estoy bien, Steve. ¿Qué demonios es esto? --el agente de policía Steve
Douglas se encogió de hombros. La sargento Pearce intentó hablar por su
radio pero todo lo que oía era un ruido estático.

El agente de policía Douglas intentó con la suya. Lo mismo.

--Vale, esto está muerto y es raro --dijo.

La sargento Pearce se alejó y volvió a cruzar la cinta, diciéndole a
Douglas que se quedara allí y mantuviera un ojo en ellos:\\--Pero no te
acerques.

Ella se dio prisa en ir hacia un grupo de bomberos y policías, que ahora
incluían a su superintendente.

--Señor, tenemos un problema --informó, y explicó que siete personas
estaban vigilando el cráter.

El superintendente Shakiri frunció el ceño y comenzó a moverse hacia
adelante, hacia el perímetro.

--Haga que el público se aleje más, sargento. Mueva la cinta otros seis
metros.

Ella asintió, pero su radio seguía sin funcionar. Shakiri intentó con la
suya. Nada.

--Funcionaba hace diez minutos --murmuró.

--Y la mía igual --dijo la sargento--. Tiene que ser algo eléctrico.

--¿Por qué dice eso?

Ella le explicó que había tocado al hombre y el shock que había
recibido.

--Vaya a que la vea uno de los médicos, sargento.

--Estoy bien, señor\ldots{} --comenzó, pero él le hizo ir.

--Reacción desaprobada, sargento. Usted le diría a cualquiera que
hiciera lo mismo. Y si dicen que se encuentra bien, la veré en cinco
minutos --le sonrió--. ¿Por favor?

La sargento Pearce se encogió de hombros y caminó hacia una de las
ambulancias, mientras oía a Shakiri gritando órdenes para que la cinta
se retrasara manualmente.

Mientras llegaba el médico, algo\ldots{} algo instintivo hizo que mirara
hacia atrás. Era como uno de esos momentos a cámara lenta en una
película, tantas cosas pasaron en un momento, que no pudo decir si lo
había visto todo o sólo había tenido lugar en su cerebro que lo había
puesto todo junto después. Un flash de luz morada, como un rayo de un
tiro eléctrico a través de la multitud de curiosos, abatiéndolos a todos
y a cada uno de ellos.

El agente de policía Steve Douglas se desvaneció, aún así, durante un
pequeño segundo, la sargento estaba convencida de que le vio subir los
brazos para protegerse del flash morado y ella le pudo ver -- no a él,
sino a su esqueleto --, sólo un segundo, cuando se fue.

Los siete ``guardias'', no más escondidos en la multitud, habían
estirado los brazos apuntando los unos hacia los otros, y la
electricidad morada los conectaba a todos, como una cuerda.

El superintendente Shakiri se lanzó al suelo, llevándose a un par de
otros oficiales con él con una táctica de rugby, probablemente salvando
sus vidas.

Hubo otro flash en el cielo, como un rayo solar, sólo por un segundo, y
la sargento juró que el cielo entero brillaba de color morado.

Y entonces acabó. Más o menos.

La gente se estaba levantando y huía. Nadie quería estar cerca de lo que
fuera eléctrico. Aquello era bueno en el sentido de que el público se
iba, pero estaba desorganizado, y aquello era peligroso. Si una sola
persona caía\ldots{} recordaba la historia del desastre en un túnel del
metro en el este de Londres durante la guerra cuando se usaba como
refugio para esconderse de los bombardeos aéreos. Mientras la gente
asustada corría escaleras abajo una mujer cayó llevándose a la multitud
entera matando a casi doscientas personas de golpe.

El pánico huyendo en aquel momento, no era tan abundante, pero podía ser
igual de letal. Vio a Shakiri, arrastrándose, gritando a los oficiales a
su alrededor que intentaran ayudar al público. Lanzó una mirada hacia
dónde Steve Douglas tendría que estar, era obvio que también lo había
presenciado todo, y entonces hacia ella.

Apartando al médico, corrió para unírsele en la escena.

--¿Qué demonios era eso?

Él señaló hacia los siete ``guardias'' alrededor del cráter.

--Imagino que quieren que nos alejemos --miró hacia las multitudes
huidizas.

--¿Alguna baja?

La sargento sólo miró hacia donde su joven agente de policía había
estado.

--¿Lo sabremos algún día? --dijo--. No queda nada de Steve Douglas.

Shakiri vio su mirada.

--Y es por eso que necesitamos saber si ha habido otros. Si les hacemos
responsables de una muerte, necesitamos hacerles responsables de
cualquier otra.

Ambas radios hicieron un ruido y se encendieron.

--Buenos días a todos, en todas partes alrededor del mundo --era una voz
femenina, hablando un claro y preciso inglés--. Me llamo Madam Delphi y
soy la única voz que nunca necesitaréis oír. Os estoy hablando a todos
por cada señal, por cada radio, televisión, ordenador y PD alrededor del
mundo. Ahora habéis visto lo que puedo hacer y seguiré haciéndolo. Este
planeta es mío. Podéis volver todos a vuestras tristes y pequeñas vidas
y esperad a que yo os diga qué hacer a continuación. Ahora os volveré a
vuestras programaciones establecidas. Oh, lo siento, excepto en aquellos
países donde están emitiendo cualquier versión de Gran Hermano. Lo
siento, todos los concursantes y presentadores de este programa, donde
quiera que estén, están muertos. Ya me lo agradeceréis luego.

Los dos oficiales de policía miraron hacia las siete personas que
vigilaban el cráter, con aquella electricidad morada que les seguía
uniendo.

--Le diré una cosa, señor --dijo la sargento Alison Pearce, mientras
miraba hacia arriba, donde todo había empezado.

--¿Qué ocurre, sargento?

--Aquella cara horrible en el cielo ha desaparecido.

La señorita Oladini estaba pensando seriamente en presentar su dimisión.
Aquél no era un trabajo lo bastante bueno como para que mereciera todas
aquellas molestias. La última noche había sido perseguida, había sido
electrocutada, casi volada del mapa en un coche, y lo peor de todo,
alguien le había robado la bicicleta. Esperaba que hubiera sido aquella
mujer pelirroja que había estado en el coche con ella, porque eso
significaría que había huido de la explosión.

La señorita Oladini no estaba del todo segura de cómo se lo había hecho
ella misma, pero sabía que había involucrado mucho de hacer la croqueta
por el suelo, sin hacer caso al calor y correr hacia el arbusto y
aguantándose la respiración por lo que pareció casi una hora pero que
podrían haber sido sólo uno o dos minutos hasta que sus perseguidores
asumieron que las dos mujeres estaban muertas.

No tenía idea de lo que pasaba en el Copernicus, pero su cuerpo había
estado en shock y había caído inconsciente en el suelo de la casa de la
vieja mansión, levantándose de nuevo, fría y húmeda y con mucha hambre.
Y sin bicicleta.

Esperó hasta que vio que nadie la veía, entonces se abrió camino de
nuevo al interior de la casa buscando calor. Después de un par de
minutos encontró un par de abrigos abandonados. Sabía que estaba en
shock. Su cuerpo necesitaba protección y calor. Se puso los abrigos, uno
encima del otro, entonces volvió hacia el armarito. Se podría esconder
allí, y sus condiciones estrechas le ayudarían a retener el calor que
necesitaba. Encontró una botella de agua en una mesa y también la cogió.

Después de un par de horas, se sintió lo bastante fuerte como para
aventurarse fuera del armario y ver si la gente seguí allí, al menos
para ver si el profesor Melville seguía con ellos.

Se estuvo arrastrando en silencio por un pasillo cuando dio un salto
porque un montón de radios y televisiones y un par de portátiles
cobraron vida, y escuchó la aportación de Madam Delphi sobre amenazas,
sintiéndose fría de nuevo.

Y cuando se recuperó lo suficiente del susto, se hizo de día y era hora
de alejarse del Copernicus. Olvidarse del profesor Melville y de aquella
gente, aquello era demasiado con lo que tener que tratar, y sospechaba
que la emisión de radio tenía algo que ver. La policía, quizá el
ejército, necesitaban saber que algo estaba sucediendo allí. Comenzó a
arrastrarse lentamente hacia la gran escalera cuando una mano llegó
salida de la nada y le tapó la boca, silenciando cualquier ruido que
pudiera hacer.

La señorita Oladini lo pensó, allí terminaba todo.

--No haga ruido, por favor, cielo --dijo una voz en su oreja--. Me llamo
Wilfred Mott y no quiero hacerte daño.

Apartó la mano de su cara, y la señora Oladini se apartó. Miró al
hombre, anciano, pero no débil, obviamente. Sus ojos ardían de
inteligencia, pero no había nada de amenazador.

--¿Por qué está aquí? --preguntó con valor.

--Mi nieta estuvo aquí anoche. Me dijo lo que ocurría. Estoy buscando al
Doctor.

--¿Nieta? ¿Es pelirroja?

--Esa es Donna. ¿Tú debes ser la señorita Oladini? Ella pensaba que
estarías muerta, así que se alegrará de que estés bien.

La señorita Oladini no estaba segura de aquello. Aquella gente que le
había atacado podrían saber todo aquello. ¿Pero sabrían sobre\ldots{}?

--¿Cómo se fue Donna?

--Con una bicicleta. ¿Era tuya? La dejó cerca de una comisaría en algún
lugar. Me parece que en South Woodham Ferrers --le sonrió--. Llegó
anoche muy tarde y yo la esperaba. Me dijo todo lo que había pasado y
después la mandé a dormir. He decidido comprobar el lugar por mí mismo.

La señorita Oladini frunció el ceño.

--¿No la creía?

--¡Por supuesto que sí! Donna no se inventa las cosas. Pero quería
encontrar al Doctor y mantener a Donna segura al mismo tiempo. Ya ha
pasado por bastantes cosas. Así que la he dejado durmiendo y me he
escapado de casa esta mañana --miró su reloj--. Ahora mismo debe de
haber adivinado esto y estará subiéndose por las paredes, supongo.

La señorita Oladini aún no estaba convencida pero no parecía tener el
carácter zombie del resto de la gente de allí.

--¿Está buscando un doctor? ¿Cuál de todos? Todos son doctores y
profesores aquí.

--Él vino con Donna. Es un tipo alto y con el pelo ridículo. Habla de
muchas cosas sin sentido.

--Francamente, eso definiría a la mayoría de las personas que trabajan
en el Copernicus, señor Mott.

--Llámame Wilf. Y no, él no trabaja aquí. Le pidio que viniera aquí un
tal profesor Melville. Es por eso que él y Donna llegaron tan tarde.

--He pasado gran parte de la noche escondiéndome y siendo perseguida, no
vi a Donna hasta que escapamos. No tengo ni idea de si había otra
persona con ella, lo siento.

Pareció que Wilf se deshinchaba.

--Oh, estaba tan seguro de que estaría aquí. Creo que es el único que
nos puede salvar de esta Madam Delphi que hemos oído hace poco.

--¿Por qué piensa eso?

--Es el tipo de cosas que hace. Salvarnos.

--¿Es algún tipo de cura?

Wilf rió.

--No, no del todo. Así que, ¿dónde está todo el mundo?

La señorita Oladini se encogió de hombros y le explicó que pensaba irse.

--Dame quince minutos --dijo Wilf--. Si no encontramos a mi amigo, te
llevaré en coche a casa, ¿vale?

La señorita Oladini valoró las opciones y entonces accedió. Aún y así,
no había otra forma de volver a casa. Y Wilf Mott no parecía demasiado
amenazador.

Le llevó escaleras abajo, a través de las ventanas francesas hechas
añicos y fuera hacia el jardín trasero, señalando hacia el
radiotelescopio, explicando que aquello era lo que importaba de allí.

Wilf asintió.

--Aquella cara del cielo, estaba hecha de estrellas, ¿verdad? Supongo
que este observatorio es dónde el Doctor tendría que estar.

La señorita Oladini tembló y se agarró más fuerte a los abrigos.

--No estoy segura --dijo con tranquilidad--. No quiero volver.

--¿Por qué no?

Pero la señorita Oladini no podía explicarlo. Había algo sobre aquello,
algo sobre cómo el telescopio le había parecido siempre un lugar seguro
para trabajar pero ahora\ldots{}

Wilf le cogió del hombro.

--De acuerdo, espérame aquí y yo me pasaré a ver si el Doctor está allí.
No tardaré.

La señorita Oladini le vio mientras se acercaba. Volvió a temblar. Por
un breve instante se había sentido segura con aquel anciano extraño, y
ahora volvía estar sola\ldots{}

Le alcanzó en pocos segundos.

--La entrada es por aquí --dijo.

Él le sonrió.

--Bien por ti, niña --dijo--. Te tengo que reconocer que tampoco es que
me gustara demasiado ir solo.

Se sonrieron el uno al otro.

--Así que Donna es tu nieta, ¿verdad? --dijo la señorita Oladini--. Me
alegro de que se fuera.

--Yo también. Estaría perdido sin ella. La familia es algo que se tiene
que mantener.

La señorita Oladini lo reflexionó.

--No sé dónde está mi familia ahora mismo --dijo--. Probablemente en
Nigeria.

--¿Cómo es que habéis perdido el contacto?

Ella sonrió.

--Oh, ya sabes cómo es todo, llegué a Reino Unido por la universidad,
perdí mi estatus, me quedé aquí escondida, entré en una agencia para
encontrar un trabajo bajo un nombre falso, cosas normales.

--Eso es muy valiente por tu parte --dijo Wilf--. Y también arriesgado,
trabajar aquí.

--Justo debajo de la nariz del gobierno --respondió--. Es la forma más
fácil de desaparecer del radar: esconderse a simple vista. Es eso lo que
me dijo mi padre la última vez que hablé con él.

Wilf estaba de acuerdo.

--Yo acostumbraba a decir eso de los espías, durante la guerra --dijo--.
La mejor forma de infiltrarse es ser visto, así nadie sospecha de ti.
Conviértete en un miembro de la sociedad.

--Eso es lo que hice. Mira dónde me he metido. Asustada de por vida.

Wilf le guiñó el ojo.

--Estarás bien.

Estaban en la puerta del radiotelescopio. Estaba abierta a medias y se
metieron. El profesor Melville estaba muerto. No había ninguna duda, su
cuello estaba en un ángulo tan extraño que aunque la señorita Oladini no
había visto nunca a nadie muerto, lo sabía. Su mano cubría su boca,
silenciando el grito que quería salir de su garganta. Wilf comprobó el
pulso del pobre hombre, pero dejó caer con amabilidad su mano.

Parecía que había estado trabajando en los sistemas de guía de la antena
cuando murió. Cuando fue asesinado, pensó la señorita Oladini. Después
de todo, la gente no se rompía sus propios cuellos.

Wilf estaba subiendo la pequeña escalera de mano que llevaba al piso
superior, dónde estaba el mismo telescopio. No era un anticuado
telescopio tubular, sino un conjunto de ordenadores ajuntados a través
de la sala, conectados a la antena de radio en lo alto del edificio.

Aquello siempre había decepcionado a la señorita Oladini cuando llegó a
trabajar por primera vez para el profesor. De alguna manera un
telescopio gigante habría parecido más romántico que un conjunto de
ordenadores.

Volvió a mirar su cuerpo, con los ojos abiertos mirando hacia el techo,
y pensó en su vieja madre. Y en su gata. Y cómo había estado de
asustada, la última vez que le había visto. Y ahora lo único en lo que
podía pensar era en su gata. Y comenzó a llorar por primera vez desde
que todo había ido mal.

Donna daba golpes a los botones de la radio hasta que encontró un canal.
No quería música, quería noticias. No fue demasiado difícil encontrarlo.
Alrededor del globo, masivos rayos de luz habían caído, y estaban
rodeados de personas. Algunos observadores decían que eran terroristas
vigilando un lugar bomba, algunos creían que eran fanáticos religiosos
vigilando algo sagrado y especial. Otros creían que eran alienígenas,
viniendo para reclamar a aquellos que habían estado abducidos y entonces
habían vuelto con microchips en sus cabezas durante los últimos
cincuenta años. El extraño mensaje de Madam Delphi, que todo el mundo
asumía que sólo era un hacker de Internet intentando hacerse el
gracioso, se decía que lo había comenzado todo.

--La ironía es --le dijo a la radio--, que esta es la que puede ser
verdad.

No parecía que hubiera demasiadas victimas pero nadie se podía acercar a
la gente que estaba vigilando los cráteres. Inglaterra, Estados Unidos,
Rusia, Oriente Medio, Asia, Nueva Zelanda, África y Groenlandia: no
había sitio sin tocar. No parecía haber ninguna conexión entre la gente
uniéndose alrededor de los cráteres: diferentes edades, sexos,
ideologías y pasados.

--Me pregunto si los polos habrán sido atacados --murmuró después de
escuchar los informes un poco más mientras conducían cerca de la Torre
de Londres.

--Hubo una mención de uno en Alemania --dijo Lukas.

Donna sonrió.

--No hablaba de los polacos de Polonia, hablaba de los Polos Norte y
Sur.

--¿Es que importa?

--Sí, probablemente, sí. Parecen ser zonas bastante pobladas, más que
desoladas. Así que tiene que haber algo significativo en ello.

--¿Qué?

--No tengo ni idea. Pero estoy pensando --vio un cartel que decía A13
Tilbury--. Essex es hacia allí --entonces se encogió de hombros--. No es
que sepa exactamente dónde estoy yendo. Era negra noche y pensaba
demasiado en el Doctor como para fijarme en los carteles.

--Tienes que coger la A127--añadió Joe--. Tres millas hacia adelante
después del cruce con la M25, entonces hacia la izquierda hacia Meadow
Lane, media media más allí y a la derecha hacia Gorsten Road. Sigue por
allí seis millas y media, entonces hacia la izquierda, hacia South
Woodham Ferrers.

--Oh sí, recuerdo ese nombre --dijo Donna--. ¿Cómo lo sabes?

--Después de pasar el puente de la vía, tienes que ir ocho millas por la
carretera Tributary y mientras vas hacia la B8932, gira a la izquierda
en el callejón Allcomb. La Copernicus está a dos millas de allí.

Donna miró a Lukas, que se encogió de hombros.

--Sabía dónde encontrarte --le dijo--. Y en qué furgoneta estarías.

--Eso asusta --dijo Donna, con calma, mirando a Joe por el retrovisor.


---Ese es Joe ---dijo Lukas---. Gracias a Dios solo somos medio
hermanos.

---No digas eso ---le reprendió Donna---. Sigue siendo tu hermano.

---Sí ---coincidió Lukas---. Pero si fuéramos hermanos completos, quizá
ambos seríamos raros. Así podría traducir si empieza a hablar en
italiano.

---¿Por qué iba a hacer eso?

---Porque, según mi mama, de ahí es de donde era su papa.

Y algo pasó por la mente de Donna. Algo que el Doctor había dicho
durante la cena la noche anterior, cuando estaba ayudando a Netty y un
anciano Crossland había pensado que estaba ladrando. Cuando estaba
hablando sobre la Mandragora.

Le hice frente por primera vez en la Italia del siglo XV.

Algo se desató en la cabeza de Donna. ¡Delfines locos! ¡Por supuesto!
Sin duda, no podía ser tan simple.. pero claro\ldots{} dijo que fue hace
500 años. Tiempo suficiente como para que gente de Italia viajase por el
mundo, generaciones de niños\ldots{} Ese hombre, la última noche en el
telescopio que había electrocutado al Doctor, su acento podría haber
sido italiano Y había hablado de genealogía\ldots{}

---¿Alguna idea de dónde en Italia?

---No ---dijo Lukas.

---San Martino ---añadió Joe.

---Pensé que quizá lo sabrías ---dijo Donna.

---¿Por qué? ---preguntó Lukas.

---Porque no creo que tenga poderes de GPS, detrás hay una coincidencia.
Creo que algo le está usando para llevarnos a ese telescopio por alguna
razón.

---¿Quieres decir que mi hermano pequeño es un alien?

---No te emocionas mucho con la idea.

---Nah, mola bastante ---Lukas se acercó---. Siempre le dije a mamá que
era raro.

---No es un alien. Pero puede que haya algo en su pasado que ayude al
Doctor a arreglar esto. Lukas miró a su hermano, que estaba escuchando
su M-TEK de nuevo.

---No quiero que le pase nada.

Donna le sonrió.

---No le pasará nada, el Doctor se asegurará de que esté a salvo.

Pero por dentro, no estaba tan segura de que pudiera garantizarlo.

El Doctor abrió los ojos y Wilfred Mott apareció en su campo de visión.
Sonrió.

---Hola, Wilf.

---Hola, Doctor ---dijo el anciano, ayudándolo a levantarse---. ¿Qué
estas haciendo en el suelo?

---Estaba tirado. Depositado. Abandonado. ¡Que grosería! ---murmuró el
Doctor. Después agarró Wilf por ambos brazos---. ¿Dónde está Donna?

---Está bien. Segura en casa con Sylvia, pensando que los dos estamos en
el huerto.

---Bien. Excelente. Brillante incluso. Ahora, ¿por qué me dejaron aquí?

---¿Ese montón de gente extraña con la electricidad púrpura?

---Sí, esos. Caray, Donna no se dejó casi nada, ¿no?

---Hice que me lo dijera todo. ¿Doctor?

---¿Sí?

---Hay un hombre muerto en la zona de oficinas.

El Doctor abrió la boca para hablar, pero se detuvo.

---Me lo temía ---siguió a Wilf fuera de la sala de control, echando un
último vistazo, y entrecerrando los ojos a algunos números de una
pantalla. Un momento después, estaba dejando el cadáver de Melville en
el suelo, comprobándolo.

---¿Tú debes ser la Señorita Oladini? ---dijo.

La Señorita Oladini asintió.

---¿Cómo\ldots{}?

---Sabía que la estaban buscando. El Profesor Melville me pidió que
intentase encontrarla. Mantenerla a salvo. ¿Y algo acerca de un gato?

---¿Entonces el Profesor Melville estaba vivo?

---Oh, sí. Le estaban obligando a mover el radio telescopio para
alinearlo con el Cuerpo del Caos de allí arriba.

Wilf le informó sobre los acontecimientos alrededor del mundo.

---¿Madam Delphi?, escribe columnas de astrología en los
periódicos---dijo la Señorita Oladini.

El Doctor le lanzó una mirada.

---Lo siento ---dijo---. Información inútil, lo sé.

---Oh no, no lo era, Señorita Oladini. En realidad es una información
brillante. Nos explica mucho. Si eres un super-ser alienígena cuya poder
Helix está regida por las estrellas, entonces quien mejor que un
astrólogo cuyas palabras son leídas y devoradas por millones que la
emplean como medium. Tenemos que encontrar a esta Madam Delphi y
preguntarle de dónde obtiene su información.

---¿Por qué matarían al Profesor Melville? ---preguntó la Señorita
Oladini.

---La Mandrágora es bastante buena para las herramientas. ¿Toda aquella
gente que viste anoche? Eran herramientas. Herramientas de las que
puede deshacerse una vez que no sean de utilidad. Imagino que el pobre
profesor Melville hizo lo que tenía que hacer para ellos y que
simplemente se lo sacaron de encima.

Puso una mano en el hombro de la Señorita Oladin

--- Una pérdida estúpida y sin sentido de un buen hombre y un amigo. Lo
siento.

Ella le sonrió.

---¿Hay algo que pueda hacer para ayudar a detenerlos?

El Doctor volvió a mirar la sala de control.

---Wilf, ¿algún rastro de alguien más aquí?

---Nadie.

---Creo que tuvieron que irse sobre las nueve de la mañana ---añadió la
Señorita Oladini---. No pude verlos mucho, pero les oí hablar.

---Wilf, ¿tienes coche?

---En la puerta principal.

---Bien, dáselo a la Señorita Oladini.

---¿Por qué?

---Sí, ¿por qué?

---Porque está viva, señorita Oladini, e hice la promesa a un hombre de
que la mantendría así. Vayase a casa. Wilf, ¿tienes móvil?

---Está muerto, nunca lo cargo.

---¿Móvil?

Wilf lo sacó del bolsillo de su chaqueta, y el Doctor lo soniqueó,
después volvió corriendo a la sala de control. Un momento después,
regresó y le dio el móvil a la Señorita Oladini.

---Cargado, debería de durar un par de semanas. Wilf, lo siento, tuve
que borrar la tarjeta SIM. ---¿La tarjeta qué?

---Si no lo sabes, no importa. Señorita Oladini, almacenada en esta
tarjeta SIM están las coordenadas en las que está posicionado
actualmente el telescopio. También he hecho algunas cosas con él,
significa que cuando te llame a este móvil no tienes que responder.

---¿Cómo sabré que eres tú?

---Porque es poco probable que nadie más llame a este teléfono porque
saben que su tonto propietario siempre lo tiene apagado y deja la
batería muerta.

Wilf carraspeó.

---Así que ---continuó el Doctor--- cuando te llame, no respondas,sino
que en lugar de eso, presiona la tecla almohadilla. Y pase lo que pase,
no la aprietes accidentalmente antes de que te llame ---Wilf quiso saber
por qué, pero el Doctor negó con la cabeza---. Es más seguro si no
preguntas. Señorita Oladini, haz todo eso y puede que seas responsable
de salvar el mundo.

Posiblemente todo el universo. Wilf, las llaves.

Wilf, de mala gana, se las entregó y la Señorita Oladini se metió el
móvil en uno de los bolsillos de sus muchos abrigos.

---Buena suerte, señorita Oladini. Y gracias ---dijo el Doctor---.
Vamos. Echo una última triste mirada a Melville---. Era encantador
---dijo simplemente---. Le conocí en 1958. Tenía una banda de jazz
callejera llamada The Geeks, toqué el teclado para él. Joe Meek iba a
producir el álbum. Le llamábamos ``Ahab'' porque su apellido era
Melville. A día de hoy, no sé cual era su nombre de pila.

---Brian ---dijo la Señorita Oladini---. Eso es lo que decía su
expediente personal. ---lanzó una triste sonrisa al Doctor y se fue.

---Brian ---le dijo el Doctor al cadáver---. Adiós, Brian. ~

Wilf miró a la señorita Oladini.

---¿Estará bien? ¿Si esa gente sigue vigilándonos?

---Nah, se han ido. Estará bien, vive en el vecindario, podrás recoger
el coche la semana que viene. Si seguimos vivos.

---Encantador.

---Siempre hay un riesgo ---echó un vistazo al reloj---. Calculo que
tenemos alrededor de una hora para averiguar por qué me dejaron con
vida.

---¿Qué pasará después?

---La caballería.

Dara Morgan estaba en el ático del Hotel Oráculo, mirando a la
autopista.

---Parecen hormigas ---Caitlin estaba a su espalda.

---Qué es más bien lo que son.

Dara Morgan abrió la boca, como si fuera a hablar, pero la cerró.

---¿Estás bien? ---preguntó Caitlin.

Se encogió de hombros.

---Yo\ldots{} me parece recordar algo. Coches de juguete. Veo un montón
de pequeños coches de metal jugando y un niño que juega con ellos. El
niño parece\ldots{}

---¿Familiar?

---En realidad iba a decir, feliz ---Dara Morgan se apartó de la
ventana---. Madam Delphi ---dijo a las pantallas de ordenador montados
por el escritorio alineados en una larga pared---. ¿Que tal lo
llevamos?

---Es fantástico ---respondió el ordenador, formas de onda brillando
positivamente---. Por todo el mundo, los hijos de la Mandragora se están
uniendo, protegiendo los lugares de llegada. Y MorganTech controla el
ochenta y siete por ciento de las franquicias mundiales de informática
---Madam Delphi rió---. Pegue eso en su pipa y fumalo, William Henry
Gates III.

Caitlin empezó a leer a algunos informes de Internet.

---Hay un nuevo culto a Mandrágora en Sudamérica ---rió---. ¿Hasta
cuándo se remonta la Mandrágora?

Las pantallas de Madam Delphi brillaron.

---Muy lejos.

---Ooh mira, el rastro de todo un linaje en Noruega. ¿Hay algún lugar al
que no hayamos llegado? ---Caitlin apretó unas cuantas teclas más---. ¡Y
otro en Zaire!

---¡Verdaderamente este es el amanecer de la Era de Acuario! ---vitoreó
Madam Delphi.

Dara Morgan estaba observando de nuevo los coches. Y en la ventana,
estaba inconscientemente dibujando letras con el dedo.

Caitlin miró. Y frunció el ceño. Dara Morgan estaba dibujando una C y
una F. Instantáneamente se levantó y se puso a su lado.

---¡Hey! ---dijo y le hizo retroceder---. Madam Delphi tiene algo que
enseñarnos. Una décima parte de la población mundial se nos unirá. Y
cuando el M-TEK salga a la venta esta semana, ¡tendremos seis veces esa
cantidad! Los prototipos gratuitos que repartimos ya han activado a
nuestros camaradas durmientes ---Dara Morgan echo una última mirada por
la ventana, hacia la autopista M4 y luego a las ventas potenciales del
M-TEK---. Una vez que Murakami haya arreglado las cosas en Tokio\ldots{}

---¡Este mundo y todos sus habitantes pertenecerán a la Mandragora!
---dijo Madame Delphi---. Oh, y acabo de subir una nueva remesa de
horóscopos. ¡Qué maravilloso!

Wilf había rescatado un viejo abrigo de una habitación y lo llevó al
observatorio para cubrir el cadáver de Melville.

---¿Por qué mataron al pobre hombre, Doctor?

El Doctor estaba en la pequeña sala de control, estudiando
cuidadosamente las lecturas, cuidando de no tocar nada.

---Probablemente preparó todo esto para la Mandrágora Hélix, y después
ya no le necesitaban. Se necesita un porcentaje de su poder para
controlar a la gente, mejor guardarlo para aquellos que lo necesitan a
largo plazo.

---¿Como por ejemplo?

El Doctor se alejó de los controles y guió a Wilf de vuelta al cadáver,
soniqueando la puerta tras él.

---Nadie entra ni sale hasta que yo lo diga ---murmuró. Entonces sonrió
a Wilf---. Desearía tener una respuesta para ti, Wilf, pero la verdad es
que no tengo ni idea. Siempre hay un vínculo entre las personas que
esclavizan y las personas esclavizadas. Ahora mismo, no tengo ni idea de
lo qué es, o cómo Madam Delphi encaja, pero supongo que está vinculada a
la Mandrágora ---de repente, el Doctor se dio en la cabeza---. Oh, ¡por
supuesto! ¡Ahora lo veo! Wilf, ¿qué hora es?

---Las doce y treinta y cinco.

---¿Y a qué hora llegaste aquí?

---Sobre la nueve, quería llegar porque\ldots{}

El Doctor alzó una mano para hacerlo callar. Entonces, empezó una cuenta
atrás.

---Cinco. Cuatro. Tres. Dos. Y\ldots{} ¡uno! ---en ese momento la puerta
exterior del observatorio se abrió, inundando la habitación con luz
diurna. Allí de pie, enmarcada en ella, estaba Donna Noble.

---La caballería, como había prometido.

Wilf abrazó Donna.

---¿Cómo sabías que estábamos aquí?

El Doctor estaba apoyado contra la pared, con los brazos cruzados, todo
indiferente, pero tan orgulloso.

---Aww, porque es tu nieta, Wilf, y es brillante.

Donna ignoró el cumplido.

---Sé lo que que ese tipo quiso decir anoche. Y era italiano. ¡Son los
italianos!

Wilf miró de uno a otro.

---¿Qué?

---Dijo algo sobre un hombre que lame delfines locos.

El Doctor asintió.

---Lo sé.

---Oh.

---Pero veamos si estamos de acuerdo, continua.

---O ---sonrió Donna--- ¿si estoy más en lo cierto que tú?

---Improbable, pero siempre es posible. Dispara.

---Lo que dijo fue ``El hombre. Lame delfines locos''. Pero no dijo
``delfines locos'' dijo "Madam Delphi ---Donna sonrió---. Sí, habías
adivinado eso, ¿verdad?

El Doctor asintió.

---Sigo trabajando en el trozo sobre lamerla ---continuó Donna.

El Doctor le sonrió.

---Hélix. Es la Mandrágora Hélix, Donna. Pero no sé por qué la parte
italiana es importante\ldots{} Oh, oh, sí, ¡por supuesto! Italia siglo
XV. San Martino, ¿quizás?

---¿De que estáis hablando? ---preguntó Wilf.

El Doctor le miró.

---Versión corta, Wilf. En 1492, me encontré con esta energía alienígena
del Amanecer del Tiempo. La Mandrágora Hélix, siempre luchando por
dominar a las especies inferiores.

---¿A quién llamas inferior? ---preguntó Donna.

---A la humanidad del siglo XV, Donna. No como la humanidad del siglo
XXI, oh no. Sois mucho más sofisticados ---sonrió de forma que sugería
que no era exactamente así como percibía las cosas, pero lo dejó
pasar---. Así que, de todos modos ---continuó---, accidentalmente traje
un fragmento de energía Hélix a un pequeño principado italiano llamado
San Martino. Lo derroté, muy inteligentemente, enterrándolo. O eso
creía. Pero eso es literalmente lo que hice, la metí en la tierra, donde
sobrevivió, tratando de repararse. Se metió en la tierra, en el agua y,
finalmente, en la gente. Una diminuta entidad biológica adhiriéndose a
los cromosomas, al ADN, a lo que sea. Transferida de generación en
generación hasta que todo el conjunto de la Mandrágora se manifestó a sí
misma en el otro lado del universo y se conectaron. La última vez,
quería frenar el progreso humano. Esta vez, la Mandrágora se ha dado
cuenta de que no hay manera de deteneros, que estaréis ahí fuera,
inundando a la humanidad a través de las estrellas en poco tiempo,
creando colonias, imperios, guerras y treguas, hasta el final del tiempo
. Así que la Mandrágora ha dicho, ``Cogeré un poco de eso, gracias'', y
se enganchó a sí misma con vosotros por toda la eternidad. Un gran plan,
puede manipularos a todos vosotros durante el milenio que está por
venir.

---Así que en todo el mundo ---dijo Donna---, diluyéndose a través de
los hijos y todo eso, hay descendientes de San Martino todo el mundo.
Cientos de estos ahora, probablemente la mitad de ellos sin saber
siquiera que tienen sangre italiana en sus venas. Y la Mandrágora les
está controlando ---se volvió hacia el Doctor---. Así es como Joe Carnes
sabía que eras quien eras. Su padre es de San Martino.

---¿Cómo sabes eso?

---Joe se lo dijo ---Lukas Carnes asomó la cabeza por la puerta---. Ha
terminado en el baño, Donna ---agregó mientras los dos chicos entraban.

---Ah, sí ---Donna sonrió débilmente al Doctor---. Saluda a mi equipo de
ayudantes.

El Doctor estaba muy contento de verlos.

---Así es como nos encontraste, ¿verdad? Pensé que era poco probable
que hubieras memorizado la ruta en taxi de anoche.

---Joe es como un perro guía ---dijo Donna.

El Doctor puso una mano en el hombro de Wilf.

---Hemos terminado aquí. Vamos a casa. Pasando por Greenwich.

---¿Greenwich? ---Wilf frunció el ceño---. ¡Oh no! No, Doctor, no
involucrarás a Netty. ¡Por favor!

---Realmente creo que ella puede ayudar, Wilf. Lo siento.

---¿Quién es ese? ---Lukas estaba señalando el cuerpo cubierto con el
abrigo en la esquina.

El Doctor suspiro.

---Un buen amigo mío, Lukas. Murió ---y lanzó una mirada a Wilf---. Pero
es el último amigo que muere a manos de la Mandrágora Helix, lo prometo.

Donna lo creyó, recordando su promesa de la furgoneta que le hizo a
Lukas por Joe Esperaba que el Doctor no les decepcionara. Mientras
pasaba frente a él, le guiñó un ojo y sonrió.

¿En qué estaba pensando? Era el Doctor. Por supuesto que todo estaría
bien. ¿Cómo podría no estarlo? El viaje de regreso a Londres
transcurrió sin incidentes, al menos. Wilf se sentó en el asiento de
delante junto a Donna, haciendo una mueca ocasional cuando casi arranca
los espejos retrovisores de los coches aparcados El Doctor y los dos
chicos se sentaron en la parte de atrás una vez que se habían movido
mantas, una botella de agua y una caja de herramientas.

El Doctor había encontrado un libro de bolsillo debajo de un asiento
llamado ``Una oscura y tormentosa noche'',sobre reyes ricos, piratas,
criadas asustadas, pastores de fuertes rebaños y una chica joven
encontrada en la nieve El Doctor simpatizaba con el héroe de la
historia, un joven interino de hospital que trataba de juntar a los
elementos dispares todos juntos. Después de un rato, se había dado por
vencido y se lo tiró a los chicos. Lukas ansiosamente había comenzado a
leerlo, manteniendo al mismo tiempo, un ojo protector sobre Joe mientras
el Doctor le preguntaba por su padre perdido hacia mucho tiempo. No
consiguió respuestas útiles. De hecho Joe no podía recordar haber ido a
la tienda de electrónica el viernes por la tarde, si no hubiera sido por
la prototipo de M-TEK gratuito, nunca habría sabido que fue allí.

---Tiene días como este ---murmuró Lukas.

---Entonces, ¿qué es un M-TEK cuando está en casa?

Lukas volvió al libro, mientras que Joe le enseñaba al Doctor el pequeño
aparato portátil.

---Es como un reproductor MP3 que también reproduce películas ---dijo
Joe---. Se conecta a la red, es un teléfono móvil y tiene una memoria de
160 gigas, para que puedas grabar cosas en él. Utiliza Windows y OSX 6
realmente rápido.

El Doctor asintió, impresionado.

---Las grandes cosas vienen en envoltorios pequeños ---dijo, y
rápidamente sacó su destornillador sónico y analizó el M-TEK con él.

Reconociendo el ruido, Donna gritó.

---Espero que le compres uno de repuesto.

Pero el Doctor tenía el ceño fruncido. El destornillador sónico no había
hecho absolutamente nada. Ni siquiera mezclar la música almacenada.

---Eso es\ldots{}

---¿Raro? ---ofreció Donna.

---Más que raro ---coincidió. Se inclinó a la parte trasera de la
furgoneta, encontró la caja de herramientas, sacó un pesado martillo y
lo descargó sobre el M-TEK. El golpe del martillo, el grito de furia de
Joe y la maldición muy alta de Lukas casi hizo que Donna se subiera al
bordillo mientras giraban hacia el túnel Blackwall.

---¿Y bien? ---preguntó Wilf.

El Doctor levantó el M-TEK.

---Ni siquiera un arañazo, ni una muesca, nada. Eso es buena tecnología.
Tecnología alienígena, pero buena tecnología. Y también es imposible
---sonrió a los asustados chicos---. Oh, me gusta un poco de imposible.

---¿Alguien ha notado otra cosa extraña? ---preguntó Donna.

---¿Todavía seguimos vivos después de conducir por una hora? ---sugirió
Wilf.

---No hay tráfico ---sugirió Lukas.

El Doctor levantó la vista.

---¿Es eso cierto?

Donna asintió.

---Hay un montón de coches aparcados. De hecho he visto sólo otros tres
coches en movimiento desde que hemos dejado Copperknickers.


---Uno de ellos no dejaba de seguirnos, creía que estaba
persiguiéndonos.

---Probablemente lo hacía ---dijo Wilf---. Le has dejado atrás.

---Pero creo que también intentaba echarnos ---Donna ignoró su
abuelo---. Porque esto es de locos. ¿Dónde está todo el mundo?

---¿Es domingo? ---sugirió el Doctor.

---Es el sudeste de Londres ---contratacó Donna---, y no estamos en el
siglo X. Debería haber cientos de coches.

---Me gusta bastante ---dijo Wilf---. Todo tranquilo. Toma la próxima
salida, cariño, Netty vive justo en la carretera principal.

Se detuvieron frente a la casa de Netty en silencio. Wilf se bajó y tocó
el timbre, pero no había nadie. Llamó a través del agujero del correo y,
después de un segundo o dos, la puerta se abrió y fue arrastrado dentro,
quedando fuera de la vista. Donna, en la parte delantera de la
furgoneta, miró al Doctor.

---¿Has visto eso?

---Era Netty ---dijo el Doctor.

---¿Cómo lo sabes?

---Los alienígenas nunca llevarían sombreros como ese.

La puerta se abrió de nuevo y Wilf salió, seguido de Netty con un
sombrero de fieltro verde con una pluma de pavo real, todo muy años 50.

---¿Has visto las noticias, Doctor? ---preguntó Netty, subiendo a la
furgoneta y sentándose junto a Wilf.

Respondió que no.

---Entonces lo mejor que podemos hacer es conducir por el centro de
Londres.

Intrigada, Donna arrancó la furgoneta y se fueron. A través de
Greenwich, pasado el reconstruido Cutty Sark y todos los mercados y
tiendas. A través del New Cross, por debajo de la antigua carretera de
Kent, alrededor del Elephant \& Castle y por sobre el puente
Blackfriars.

---Ni un alma ---dijo Wilf---. Nadie.

---La BBC decía a todos que no salieran de casa. Fairchild ha declarado
el estado de emergencia.

---¿Fairchild? ---preguntó el Doctor.

---El Primer Ministro ---dijo Lukas con un suspiro---. ¿No sabes nada?

---Conozco a muchos primeros ministros ---dijo el Doctor---. Pero en
este siglo van y vienen cada año, creo. Claramente esté no dejará huella
en la historia.

Donna frenó la furgoneta de golpe y dijo en voz muy baja:~

---Oh.

Los que estaban en la parte trasera de la furgoneta se inclinaron hacia
adelante:

---Oh, efectivamente, ---dijo el Doctor.

Porque no podían ir más allá. Estaban en el Embankment, justo debajo de
la estación de Charing Cross. Había posiblemente un millón de otras
personas. De pie. Quietas. Con los brazos levantados hacia el cielo. Y
todos cantando en voz baja.

---Helix. Helix. Helix.

---Eso no es bueno ---dijo Donna.

El Doctor le pasó el M-TEK de Joe.

---Llama a tu madre, por favor.

---¿Por qué?

---Hazle saber que estamos a salvo y que la veremos mañana.

---¿Prioridades? ---preguntó Donna.

---Mantener de buen humor a tu madre es una prioridad, Donna. Para los
dos. Va a estar preocupada ---se volvió hacia los chicos Carnes---.
Después llamaremos a vuestra madre, debe estar muy preocupada.

---No lo estará ---dijo Joe en voz baja---. Será una de este montón.

Wilf estaba a punto de preguntar por qué, pero el Doctor negó con la
cabeza.

---Ahora, Joe, justo porque es tu madre, no está en peligro. Ninguna de
estas personas lo están, al parecer.

---Ella le dio a luz a él. Quizás también tiene este gen Helix, ¿verdad?
---dijo Lukas---. Gracias a Dios que soy el mayor.

Joe miró fuera de la furgoneta.

---¿Qué haremos para rescatarla, Doctor?

El Doctor sonrió.

---Ese es el espíritu, muchachos, recordad que podemos salvarla. Podemos
salvar a todas estas personas.

Donna pasó el M-TEK hacia atrás.

---Dice que Chiswick está vacío. Le he dicho que se quede en casa, beba
té y mantenga encendido el televisor. Le he dicho que haga lo que diga
la BBC, salvo que implique abandonar la casa y dejar de beber té. No le
ha visto la gracia.

---No me sorprende ---dijo Wilf---. Bueno, Doctor, ¿qué hacemos?

El Doctor estaba mirando el M-TEK.

---Ellos te dieron eso, ¿verdad, Joe? ¿Han dado muchos gratis?

Joe asintió.

---En el foro dijeron que estaban regalando un millón antes del
lanzamiento de mañana. ---Apuesto a que el público objetivo también era
de una genealogía muy específica. ¿Así que esta cosa se venderá a nivel
nacional mañana?

---A nivel mundial ---añadió Netty---. Yo conseguiré uno. Me gustan este
tipo de cosas. Iba a esperar un mes o así, para ver si el canal de la
teletienda los vendía más baratos.

---Oh, me gusta eso ---dijo Donna---. Bueno,me gustaba. Cuando tenía
tiempo. Aquel Anis Ahmed hacía cosas por mí\ldots{}~

Netty se rió.

---Tan sexy\ldots{}

Wilf tosió.

---De todos modos, volviendo al tema que nos ocupa. Doctor, no podemos
aparcar aquí.

El Doctor seguía jugando con el M-TEK.

---Nada está tan bien protegido\ldots{} Si pudiera reescribir una parte
del software\ldots{} ---el destornillador sónico hizo un par de brillos
azules de tonos diferentes, después el M-TEK hizo un ruidito, y el
Doctor lanzó un grito de alegría. Después se detuvo---. Parece que he
accedido a un web de horóscopos. Ah, nuestra vieja amiga Madam Delphi.

---Ella trabaja para la gente que fabricó el M-TEK ---dijo Lukas---.
Escribe las cosas estas de los horóscopos para algunos de sus
periódicos.

El Doctor se quedó mirando a los jóvenes.

---¿Qué?

---MorganTech. Hacen de todo hoy en día. Tienen canales de televisión,
diarios, trabajos, etc ---Lukas se encogió de hombros---. Piensa en una
mezcla entre Bill Gates, Ruper Murdoch y Richard Branson y tendrás a
Dara Morgan.

---¿Y quién es él cuando está en casa? ---preguntó el Doctor.

---Dirige MorganTech. Hicimos un trabajo sobre él en la escuela, pero no
hay mucho sobre él. No le gustan las biografías no autorizadas.

El Doctor miró al grupo de la furgoneta.

---Dejadme a ver si he entendido eso. Tenemos rayos de luz golpeando el
suelo, gente hipnotizada cantando a las estrellas gracias a una
astróloga de un periódico que les dice que están cambiando el mundo,
nuevos aparatos distribuidos gratuitamente a personas que son de
ascendencia italiana y ¿nadie piensa que todo está conectado?

El resto se miraron unos a otros. Donna habló finalmente.

---No puedes esperar que tengamos estos saltos de lógica que haces tú,
ya lo sabes.

---No son saltos de lógica, son claramente caminos definidos de
evidencias y\ldots{} oh, no importa. ¿Dónde puedo encontrar este
MorganTech?

---Cerca de donde vivimos ---dijo Lukas---. En Brentford.

---Oh, es verdad ---dijo Wilf---. Tienen un gran complejo de oficinas y
un hotel en la Milla de Oro. ---¿Y el hombre al cargo se llama Dara
Morgan?

---Sí.

---Por supuesto que sí ---el Doctor murmuró---. Claro que sí. Lukas,
quiero que intentes recordar todo lo que puedas acerca de él, ¿de
acuerdo? ---el Doctor dejó el M-TEK sobre el piso,en el suelo y Joe fue
a recogerlo---. Deja eso, Joe, es peligroso ---apuntó con el
destornillador sónico a la parte trasera de la furgoneta y las puertas
se abrieron---. Venga. Va, no podremos pasar por este montón y
necesitamos encontrar un nuevo transporte.

---Doctor ---protestó Wilf---. Netty\ldots{}

---Ey ---dijo Netty---. Puedo andar igual que tú puedes, Wilfred Mott
---le agarró del brazo---. Podemos ayudarnos el uno al otro.

Él le sonrió. Y Donna iba a hacer lo mismo, hasta que vio la mirada en
el rostro del Doctor.

Era la misma que tuvo la cena la noche anterior. Estaba mirando a
Netty\ldots{} de una forma extraña. Donna atrajo a los chicos más cerca
de ella.

---Quedaos conmigo --- les dijo---, y ayudaremos al Doctor a acabar con
todo esto.

---Donna ---dijo de repente el Doctor y de una forma en la que la Donna
estaba acostumbrada. Era una voz de advertencia. Entre ellos y la
multitud cantante había un grupo de personas. Personas que Donna
reconoció de la noche anterior, en el Observatorio Copernico.

---Eso no es bueno, ¿verdad?

---No es bueno.

---¿Cómo has reprogramado el M-TEK? ---preguntó el hombre menudo al
frente del grupo. Donna también se acordó de él Los había guiado, y se
dio cuenta de que su acento era, por supuesto, italiano.

---Talento ---dijo el Doctor.

---Eso no forma parte del plan ---dijo el hombrecillo---. No podemos
permitir un eslabón débil en la cadena.

---Oh, lo siento ---dijo el Doctor, indicando con la mano al resto de su
grupo que se apartasen ligeramente por detrás de él, dejando a sí mismo
entre el grupo poderoso de la Mandrágora y la furgoneta---. Lo he dejado
atrás. ¿Quieres que vaya a buscarlo?

---Lo dejarás ---dijo el italiano, mientras pasaba empujando al Doctor y
subía a la furgoneta.

El Doctor sonrió al resto del grupo. Una pareja de ancianos a un lado,
cuatro jóvenes al final y un hombre forzudo a la izquierda.

---Me pregunto cuántos de vosotros sois descendientes reales de San
Martino, y cuántos son sólo sus\ldots{} ¿esclavos? ¿Ayudantes?
¿Participantes involuntarios en el asesinato de profesores en los
observatorios? Si podéis combatir a la Mandrágora, quizá
podamos\ldots{} El Doctor dio un golpe fuerte en el asfalto mientras la
furgoneta azul explotaba en llamas y escombros.

Donna y los chicos ya estaban corriendo, con Wilf y Netty, tambaleándose
trás ellos. Bien.

Echó un vistazo a la pira funeraria del hombrecillo italiano que había
sido una vez una furgoneta.

---Esa es una forma de eliminar a los débiles M-TEK, supongo ---dijo---.
Sin embargo, un poco exagerado si me preguntáis.

Y cuando se levantó, se vio rodeado por el grupo. El hombre forzudo
parecía ser su nuevo líder y cuando habló, el Doctor reconoció un fuerte
acento griego.

---Madam Delphi quiere verte.

---Bueno, está bien, pero quiero comprobar que mis amigos están bien.

---También vienen con nosotros.

---Oh, no estoy seguro de estar de acuerdo con esa parte del trato.

---O te matamos ahora mismo ---añadió el griego.

En ese momento Wilf, Netty, Donna y los chicos salieron de su escondite
y rápidamente fueron rodeados. El Doctor suspiró.

---Creo que probablemente era un farol ---le dijo a Wilf---. Me quieren
vivo, ¿recuerdas? ---Estamos en esto juntos ---dijo Wilf---. Cuando
estaba en los paracas, nunca dejamos a nadie atrás.

El Doctor asintió.

---Bueno, ahora que estamos todos aquí. ¿Debemos contratar un autocar?

---Caminaremos ---dijo uno de los ancianos, una mujer americana.

---Es un largo camino ---dijo Donna.

Uno de los hombres más jóvenes se encogió de hombros.

---Así nos mantendremos en forma.

Y empezaron a caminar a través de Londres. Dondequiera que iban,
pequeños grupos de personas estaban juntos, cantando a los cielos. Otros
se podían ver escondidos, asustados, ocasionalmente asaltando tiendas,
probablemente asumiendo que esto no terminaría pronto, y que la comida
se volvería escasa.

---Es igual que en el bombardeo ---dijo Netty en un momento dado,
mientras caminaban por Leicester Square.

---Sin las bombas y los edificios derrumbados ---dijo el Doctor---.
Gracias a Dios.

El Doctor permitió que su grupo se separara ligeramente, notó Donna.
Estaba en la retaguardia con los chicos y Wilf se estaba cansando, y
sólo iba unos pasos por delante. Trás ella, el hombre griego y los
ancianos estadounidenses. Por delante del Doctor, los cuatro más
jóvenes.

El Doctor estaba con Netty, cambiándose el lugar con Wilf, con su brazo
unido con el de ella.

Donna no podía oír lo que estaba diciendo, pero podría decir con la
urgencia con la que él daba a sus gestos y la falta de respuestas de la
mujer más que un asentimiento de cabeza ligero, que no discutían mucho
sobre la arquitectura de Londres.

Casi le preguntó a su abuelo de que creía que estaban hablando, pero no
lo hizo. Porque si salía mal, si todo salía mal, no quería que él
culpase al Doctor de todo. Donna se dio cuenta de que aquella era la
primera vez que se había encontrado a sí misma cuestionando las acciones
del Doctor durante mucho tiempo. Y no le gustaba. Pasaron un par de
horas. Les habían permitido pararse ocasionalmente, los jóvenes buscando
comida (a veces utilizaban los poderes de la Mandrágora para tirar
puertas abajo y llevarse cosas). En un momento determinado, el Doctor y
Netty se habían sentado juntos en un restaurante de comida rápida
abandonado, mientras los chicos comían patatas fritas frías y
magdalenas. Netty había encontrado algo de papel en una mesa y escribía
algo, y el Doctor asentía con la cabeza.

Wilf preguntó a Donna si tenía sentido probar los hornos microondas y
cuando volvió a mirar hacia atrás Netty estaba sola, y el Doctor estaba
intentando hablar con el hombre griego.

No había tiempo para cocinar hamburguesas en el microondas, pues les
dijeron que comenzaran a caminar de nuevo, a pesar de las protestas del
Doctor. Los chicos se cansaron pronto de nuevo. Netty y Wilf estaban
realmente muy cansados. Donna estaba completamente agotada, pero el
Doctor\ldots{} simplemente continuó hacia adelante. Tenía a Lukas y Joe
ante él en ese momento, intentando distraerlos explicándoles la historia
de la carretera Cromwell y los diferentes edificios que había mientras
pasaban. La pareja estadounidense de ancianos por lógica deberían haber
estado muertos de cansancio, pero no, siempre estaban allí, uno u otro,
a veces ambos, con sus armas apuntando hacia delante, dispuestos a usar
su poder de Mandragora como había visto en el Observatorio Copérnico la
noche anterior.

Era de noche cuando llegaron a Hammersmith y Donna calculó que tardarían
una hora o más en llegar a Brentford. Posiblemente más, ya que Netty y
Wilf se paraban más y más a menudo.

---Mi abuelo es muy mayor ---dijo en un momento dado al hombre griego,
provocando un indignado y agotado, ``Ey, estoy bien'' de Wilf.

El hombre griego se encogió de hombros y dijo que a Madam Delphi no se
la hacía esperar.

No era una noche fría, pero tampoco era pleno verano y cuando comenzaron
a caminar por la carretera Great West sin gente, ni coches, era casi
medianoche. Donna estaba con el Doctor. Wilf y Netty estaban con los
chicos Carnes. Wilf intentaba animar a sus espíritus decaídos con
historias de sus aventuras en el regimiento de paracaidistas, como hacía
por Donna cuando tenía su edad, en los largos viajes en coche más que
por las dolorosas caminatas a través de ciudades espeluznantes.

---¿Por qué no dejas que esta gente se vaya a casa? ---sugirió el
Doctor, deteniéndose repentinamente---. Madam Delphi sólo me quiere a
mi, estoy seguro. Mira, estamos en Chiswick. Deja que Donna se lleve a
Wilf y a Netty a casa Y deja que los niños se vayan también. ¡Por
favor!

El griego no le hizo caso y siguió su camino.

---No es que el abuelo o yo te queramos dejar ni un momento ---le
susurró Donna mientras corria a alcanzar al Doctor---, pero ¿por qué
crees que nos quieren a todos?

El Doctor la miró a los ojos.

---Un seguro ---dijo simplemente---. Amenazar con hacerme daño no sirve.
De todos modos, me necesitan con vida por alguna razón. Amenazar con
matarte, es una ventaja. Lo siento. ---No lo sientas ---dijo Wilf---.
Hemos escogido involucrarnos en todo esto. Estoy orgulloso de estar a tu
lado, Doctor. Al igual que mis niños soldados aquí.

Los chicos Carnes asintieron, Lukas con un poco más entusiasmo que Joe,
era preciso decir.

El Doctor miró a Netty. Empezaba a pasear erráticamente, vagando hacia
la reserva central de arbustos.

---Es el agotamiento ---dijo el Doctor con tristeza mientras Wilf se
dirigía a guiarla de nuevo al grupo---. Su mente se va de nuevo como la
última noche.

---¿Entonces por qué la has traído? ---Donna lo dijo un poco más
agresivamente de lo que había previsto.

---No esperaba caminar ---dijo el Doctor---. Lo siento.

Donna se dejó caer hacia atrás un par de pasos. Algo en el plan del
Doctor había salido mal, y estaba realmente preocupado. Eso no era una
buena señal.

De repente una serie de luces los iluminaron y, como si fueran uno, el
grupo del Doctor se tapó los ojos. Un pequeño minibús frenó delante de
él.

---Ey ---gritó Donna---. ¡Tienes que ayudarnos!

El Doctor fue a detener a Donna, pero no importó. La puerta del minibús
se abrió y una mujer les llamó.

---Hop, adentro gente ---dijo con un alegre acento irlandés---. Madam
Delphi está esperando. ~Uno por uno, se subieron.

---¿No podría haber llegado tres horas antes? ---gruñó Wilf mientras
ayudaba a una confusa Netty a subir los peldaños del vehículo. La mujer
se echó a reír.

---Soy Caitlin y en nombre de MorganTech les pido disculpas por las
molestias. Pero eso no es nada en comparación a lo que viene. Y no,
Madam Delphi cree que los prisioneros exhaustos son mucho más maleables
que los sanos y capaces. La única razón por la que estoy aquí es porque
es casi medianoche. Y se nos acaba el tiempo. ¡Agarraos fuerte!

Caitlin hizo un giro en U y recorrió rugiendo la A4, hacia la zona de
negocios de Brentford, conocida como la Milla de Oro.

---Ya hemos llegado ---dijo Caitlin, yendo más despacio.

Frente a ella Donna vio el hotel Oraculo brillante entre la oscuridad,
con las luces encendidas en todas las ventanas.

---Ja ---El Doctor se echó a reír---. Vamos a ver a Delphi en el
Oráculo. Muy ingenioso. Oh no. ---Es medianoche ---anunció Caitlin
mientras abría las puertas del minibús---. Ya es lunes. El universo
nunca volverá a ser el mismo.

Y sonrió.

Y Donna se estremeció.
