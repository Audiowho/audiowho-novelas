\chapter*{Viernes}
\addcontentsline{toc}{chapter}{Viernes}

Terry Lockworth comprobó su teléfono móvil, pero todavía no había
cobertura. María iba a estar tan harta de él, pues había tenido que
trabajar hasta tarde, pero no podía hacérselo saber. Sin duda el bol de
espaguetis acabaría en la papelera esa noche. De nuevo. Pobre María, no
era culpa suya que terminara harta de él pero, ¿qué se suponía que tenía
que hacer? Habían estado casados durante tres meses, tenían un hijo
en camino (por favor, que sea una niña), e iban escasos de dinero.

Sí, claro, su padre les había dado un depósito para el apartamento de
Boston Manor, pero aún quedaba la hipoteca, las facturas, las clases
pre-natales, la comida\ldots{} Terry negó con la cabeza mientras se
guardaba el teléfono en el bolsillo. Deja de quejarte, se dijo, y sigue
adelante con el trabajo, entonces estarás en casa en una hora con un
poco de suerte. Y lo que era más importante aún, estaría fuera de ese
punto sin cobertura en menos que eso, por lo que al menos la podría
llamar.

Cogió su caja de herramientas y sacó los cortaalambres, recortó el
recubrimiento de plástico del cableado de cobre y cortó los cables.
Luego tiró los viejos cables de la caja de conexiones y sacó un rollo
largo y delgado de fibra óptica de la caja de herramientas. Aquellas
eran fibras ópticas interesantes (bueno, vale, sólo Terry las encontraba
interesantes) porque eran aún más finas de lo normal. Un nuevo sistema
desarrollado por los americanos (siempre son los mismos), y este
edificio era el primero en Reino Unido en utilizarlos. Habían enviado a
Terry a un curso en Nueva York hacía seis meses para aprender sobre el
sistema. Aquello había sido divertido: un montón de noches en la ciudad
con Johnnie Bates, descubriendo que era de verdad la ciudad que nunca
dormía. Frecuentemente habían conseguido llegar a las clases de
entrenamiento al día siguiente, con resaca pero felices.

Sin embargo, Terry era lo bastante sensato como para saber cuándo ir de
fiesta y cuándo aplicarse y hacer el trabajo, y él y Johnnie habían
vuelto a Inglaterra, certificados para trabajar en la instalación de
esta nueva fibra óptica, lo que hizo a los dos populares con su jefe y
les ganó un poco de dinero extra.

Les habían prometido otro extra si hacían este trabajo y, francamente,
era el dinero lo que les movía a hacerlo. El cableado fue fácil, fue la
eliminación de las cosas viejas de cobre lo que estaba tomando más
tiempo.

Johnnie estaba un par de pisos por encima de él, más cerca de la suite
de demostración. Lo habían echado a suertes con una moneda para ver a
quién se paseaba por ahí con los peces gordos y tenía la oportunidad de
robar una taza de té de las secretarias y las APs y a quién le tocaba
las escaleras traseras y los pasillos de servicio. Terry había perdido,
por supuesto. No habría té para él.

Sacó un destornillador de su cinturón de herramientas y empezó a abrir
la última caja de conexiones, silbando algo que había oído en la radio
del coche. Cualquier cosa para pasar el tiempo.

Si hubiera mirado por encima de la obra que acababa de hacer, podría
haberse sorprendido al darse cuenta de que los cables de fibra óptica
que había conectado en las cajas de conexión anteriores brillaban
extrañamente.

Los cables que no estaban conectados a fuentes de energía rara vez
brillaban. Nunca, para ser francos. Simplemente no pasaba. ¿Por qué iban
a hacerlo? ¿Cómo podía ser posible? Pero estaba ocurriendo: pequeños
impulsos de color púrpura de energía, brevemente parpadeando arriba y
abajo por el cableado. Casi como el bombeo de la sangre por las venas de
una enorme criatura electrónica.

Terry no se dio cuenta porque estaba mirando hacia delante, mirando
dónde iba a hacer lo siguiente, no dónde acababa de estar.

Lo cual fue desafortunado. No sólo para Terry Lockworth, cuyos
espaguetis a la boloñesa quedarían sin comer esa noche, sino también
para casi toda la raza humana.

Terry entrelazó el último pedazo de cableado de fibra óptica en la
última caja de conexión y atornillándola, la cerró por última vez,
sonriendo para sí. Arriba, Johnnie debería estar recibiendo una prueba
de que el cableado estaba terminado, y sus monitores le dirían que todo
estaba listo para salir.

Mientras Terry terminaba de apretar el último tornillo, un masivo rayo
de energía alienígena púrpura se precipitó a través de su
destornillador, por su mano y por todo su cuerpo. Se movió tan rápido
que, en el momento en que los minúsculos copos carbonizados que eran
todo lo que quedaba de Terry cayeron al suelo, el destornillador sólo
estaba empezando a separarse de la cabeza del tornillo.

Por supuesto, Terry tuvo suerte. Al morir tan repentina, violenta y
eficientemente, se salvó de lo que estaba por venir en los próximos
días.

Pero él probablemente no lo hubiera visto exactamente así.

Arriba, en la suite del ático, Johnnie Bates estaba vinculando todos los
ordenadores con el principal servidor de administración del Hotel
Oracle, faro del esplendor arquitectónico que era el Western Development
Business District, comúnmente conocido como la Milla de Oro, justo al
lado izquierdo de la autopista M4 saliendo de Londres.

Pero para el hombre que era dueño del hotel, Johnnie era sólo un pequeño
hombre con un mono gris haciendo algo con cables.

Dara Morgan, según las biografías que mantenía cuidadosamente en las
páginas web de sus empresas, hizo su primer millón en Derry, cuando
tenía sólo 26 años, con la creación de una popular web de descarga de
música que permitía descargas baratas seis veces más rápido que las
velocidades tradicionales y con cuatro veces más calidad que los MP3
tradicionales.

La industria de la música le amaba. Los consumidores le amaban. El
gobierno le amaba. Su madre le amaba (bueno, eso pensaba, no hablaban
mucho estos días, es lo que tenía que estuviera en una urna de plata en
la repisa de la chimenea junto a su padre). Y el mundo de los negocios
le amaba. Cuatro años más tarde, MorganTech era la financiación tras el
nuevo Día Mundial de la Donación de Sangre, trayendo trabajo y
desarrollo a Hounslow, Osterley y a todas esas otras zonas de Londres
entre Brentford y el aeropuerto de Heathrow de las que nunca había oído
hablar antes de la compra de los terrenos.

Con un portafolio personal de alrededor de 65 millones de libras, a un
paso de ser una megaestrella, ya cenaba con los Trump, los Gates, los
Rothschild, los Gettys y media docena más de los que movían los hilos
con nombres impronunciables de todo el mundo. En realidad, los nombres
no eran impronunciables, pero Dara Morgan no se molestaba en
recordarlos. Ellos simplemente no importaban lo suficiente.

Lo que le importaba ahora era conseguir que las suites de su nuevo hotel
estuvieran listas para la demostración de su nuevo ordenador portátil. Y
el pequeño hombre del mono gris mugriento no iba lo bastante rápido.

---¿Cait? --- chasqueó los dedos y una potentemente vestida pelirroja
con unas finas gafas metálicas y unos tacones increíblemente altos se
acercó.

---Señor Morgan, ¿señor? ---Dara Morgan señaló hacia el hombre del mono
gris.

---¿Cuánto tiempo más hay que esperar? ---preguntó, con su acento
norirlandés suave espetando inusualmente.

Caitlin asintió comprendiendo y se acercó a preguntarle al hombre para
obtener información.

Dara Morgan sonrió para sus adentros, mirando a Caitlin mientras se
movía. Él la apreciaba en muchos aspectos, pero su belleza estaba muy
alta en la lista.

Todos en su organización eran de Derry o de los distritos vecinos de
Irlanda del Norte. Y lo que era más importante aún, todos ellos eran
gente con la que había crecido. Todas las historias de los padres y
hermanos mayores sobre las luchas, los asesinatos, el honor. Las
marchas, las tropas, las recriminaciones y los castigos.

Era la historia para Dara Morgan, algo de otra época, por lo menos. Su
generación no había tenido tiempo de preocuparse por las luchas, más de
lo que se preocupaban por las hambrunas de patata o de Oliver Cromwell.
Esa era historia antigua. Dara Morgan y MorganTech eran el futuro. En
muchos sentidos.

Pasó una mano por su pelo largo hasta los hombros y luego sacó su móvil,
haciendo una pausa para oler el aroma de champú de sus dedos. Era muy
importante estar limpio. Para verse bien y oler mejor.

En el colegio, le habían diagnosticado como una forma de trastorno
obsesivo compulsivo, ¡como si la compulsión obsesiva de lavarse las
manos cada vez que entraba en contacto con otra persona fuera algo
malo! Las personas llevaban gérmenes y, aunque no creía ni por un
momento que iba a ser derribado por la malaria sólo con darle la mano a
un extraño, no era tan irracional cuidarse uno mismo de vez en cuando.

En el colegio nunca le entendieron, o eso recordaba vagamente. Era
demasiado pequeño, probablemente demasiado centrado en los planes de
estudios, en los horarios y en los deportes.

No podía esperar para irse, y así lo había hecho en el momento en que
terminó sus exámenes. No habría bachillerato, universidad o formación
profesional para él. Directo a los negocios, directamente a la
tecnología de la información, el futuro del mundo, directamente a la
creación de un sistema de MP3 para los trogloditas que pensaban que Gran
Hermano y Factor X eran el principio y el fin de toda la cultura
televisiva. Él los necesitaba, por supuesto, porque le habían ayudado a
alcanzar su potencial, habían sido los primeros peldaños en la escalera
hacia el éxito. Para poder gobernar el mundo, a través de los negocios.
No tenía ningún deseo de gobernar el mundo en realidad, pues estaba
lleno de demasiadas gruesas personas peleándose por el petróleo, el
territorio y Dios para ser un plan bastante sensato como para poder
controlarlo.

Pero podría dominar la tecnología, despedir a los llamados actuales
gigantes y comprar el acceso a los hogares y lugares de trabajo de todo
el mundo en el planeta.

Eso era suficiente. Y en la demostración de prensa de mañana, el plan
daría su primer paso.

Caitlin volvió y dijo que el hombre estaba esperando la llamada de otro
hombre en algún área de servicio en la entreplanta y estaría listo. Dara
Morgan miró hacia allí, el renombrado hombre estaba tratando de llamar.

---Dile a tu amigo ---le dijo Dara Morgan a Caitlin---, que no va a
conseguir hablar con su colega. Las áreas de servicio están bloqueadas a
las señales de cobertura. Dile que tiene que usar un terminal. Si las
fibras ópticas están conectadas, le pondrán directamente con el móvil de
su socio. Caitlin asintió y transmitió el mensaje.

Dara Morgan vio como el renombrado hombre insertaba la conexión de fibra
óptica en la parte posterior de su ordenador portátil y marcaba con
aquello. Hubo un destello púrpura y, donde el hombre estaba arrodillado,
había ahora sólo un montón de cenizas. Un acre olor a quemado flotaba en
el aire, y Dara Morgan arrugó la nariz con disgusto. Carne quemada,
tela fundida y sudor. Asqueroso.

---Bueno ---dijo Caitlin---, eso indica una buena señal, señor ---Dara
Morgan dio una palmada con fuerza, y todos los demás en la sala, todos
los que habían ignorado la muerte de Johnnie Bates, se giraron hacia él.

---Gente, al parecer el hotel está cableado O ``fibrado'', diría yo
---hubo un cortés murmullo de risas.

---Mañana, dominaremos el mundo.

---¡Ey! Una palabra / frase / ruido gutural, balbuceado con un toque de
indignación, un toque de sarcasmo y un gran trago de volumen.

No importaba lo mucho que lo intentara, el Doctor no podía dejar de
suspirar cada vez que lo oía. Por lo general, la indignación, el
sarcasmo y, especialmente, el volumen tenían como objetivo su dirección.

Él suspiró y se volvió hacia Donna Noble, la reina de los Ey. Pero ella
no estaba allí.

Sólo la TARDIS, aparcada entre dos contenedores de basura municipales.
Muy cuidadosamente, se dijo a sí mismo. Oh. ¡Ah! Cierto.

---Lo siento ---dijo a la puerta de la TARDIS, luego dio un paso hacia
atrás y la abrió, revelando a Donna que estaba en el umbral.

---Supuse que ya estabas fuera. ¿Qué parte de ``Estoy justo detrás de
ti'' no te ha quedado clara? ---Donna preguntó educadamente, con ese
giro de cabeza que denotaba que no decía lo que realmente sentía, es
decir, que no era del todo educada---. ¿Qué parte de ``Espérame'' te ha
esquivado el oído? ¿Qué pedazo de ``Me estoy poniendo algo cómodo'' se
ha desvanecido en el éter?

No había manera de que el Doctor se pudiera escabullir de aquello. Así
que se encogió de hombros.

---Te he dicho que lo sentía.

---¿``Lo sientes''?

---Sí, ``lo siento''. ¿Qué más quieres?

---¿``Lo sientes'' por no escucharme? ¿``Lo sientes'' por haberme
encerrado dentro de tu nave espacial extraterrestre? ¿O ``lo sientes''
por no haberte dado cuenta de que no estaba contigo?

Cada vez, que Donna escupía las palabras ``lo sientes'' sonaba como las
palabras menos arrepentidas de la lengua y adquirían un nuevo
significado que los lingüistas podrían discutir sobre su implicación
exacta los próximos doce siglos.

---No hay manera que pueda ganar esto ---dijo el Doctor ---, pero voy a
dejarlo ir, ¿de acuerdo?

Donna abrió la boca para hablar de nuevo, pero el Doctor se adelantó y
puso un dedo en sus labios.

---Calla ---dijo.

Donna calló. Y le guiñó el ojo.

---¡He ganado! ---y entonces le dedicó esa fantástica e increíble
sonrisa que siempre ponía cuando le estaba tomando el pelo, y él suspiró
admitiendo que le había pillado otra vez más.

Era un juego. Un juego entre dos amigos que habían vivido tantas cosas
juntos que jugaban instintivamente entre ellos. Familiaridad, amistad y
diversión. Las tres palabras que resumían el tiempo compartido por los
dos aventureros. Ella puso su brazo alrededor de él y lo atrajo hacia
sí.

---Así que, ¿cuál es el plan, Plano?

El Doctor hizo un ademán con la cabeza hacia la carretera principal de
Chiswick y su ajetreo y el bullicio del tráfico, y rápidamente la
arrastró por la acera, dispuesto a perderse por entre la multitud. Lo
único que no había ninguna. De hecho, no había mucha gente a su
alrededor, sólo un par de niños en un monopatín en la acera de enfrente
y un anciano que paseaba a su perro. El Doctor levantó su otra mano.

~---No llueve ---dijo.

---Bien visto, Sherlock ---dijo Donna

---¿Es domingo? Querías el viernes 15 de mayo de 2009, Donna. Y ese es
el día para el que programé la TARDIS.

Donna se rió.

---En ese caso es probable que sea un domingo de agosto de 1972. El
Doctor asomó la cabeza en un quiosco, sonriendo al hombre detrás del
mostrador, que estaba escuchando a su reproductor de MP3 y haciendo caso
omiso de su cliente en potencia. El Doctor miró el periódico más
cercano.

---Hoy es viernes, 15 de mayo de 2009 ---confirmó a Donna.

---Entonces, ¿dónde está todo el mundo?

---Tal vez sea la hora de comer ---sugirió el Doctor---. O tal vez
Chiswick ya no sea el centro neurálgico de la sociedad como lo era hace
un mes. ¿Vamos a tu casa?

---¿Tú vienes?

El Doctor puso una expresión como si la idea de no ir con Donna no se le
hubiera pasado por la cabeza.

---Oh. Umm\ldots{} Bueno, sí.

---Doctor, ¿por qué estamos aquí?

---Es el primer aniversario de la muerte de tu padre.

---Y, aunque estoy segura de que estará agradecida por haber salvado al
mundo de los Sontaran, no creo que mi madre se alegre de verte hoy,
entre todos los días.

---Tu abuelo lo estará.

---¿Sí? Bueno, llévatelo a tomar una pinta esta noche a Shepherd's Hut,
pero para empezar quiero verlos por mi cuenta.

Donna aún sostenía su mano, y la apretó suavemente.

---Lo entiendes, ¿no?

Él sonrió.

---Por supuesto que sí. No lo he pensado. Lo siento. \\---No empecemos
de nuevo, ¿vale? ---Donna le soltó la mano---. Voy a conseguir algunas
flores y caminaré hasta casa. ¿Por qué no nos encontramos de nuevo aquí,
a esta hora, mañana?

---Aquí. Mañana. Vendido.

El Doctor le guiñó un ojo y comenzó a alejarse caminando.

---Hay una bonita floristería en la esquina de ahí ---gritó ---.
Pregunta por Loretta y dile que yo te mando.

Dobló una esquina y desapareció. Donna respiró hondo y se dirigió en la
dirección que había señalado.

Hacía un año. Ese día. Los Adiposos. Los Pyroviles. Los Oods con
cerebros en las manos.

Incluso los tubos de escape Sontaran, los Hath y los esqueletos
parlantes, todo parecía sencillo en comparación con lo que iba a pasar
esta tarde.

Porque esa tarde Donna tenía que volver y estar allí para su madre y
probablemente no sólo revivir el año pasado, sino los días y semanas que
siguieron, los funerales, decírselo a la gente, los memoriales, los
avisos en los periódicos, la parte financiera de las cosas, encontrar el
testamento\ldots{} nada de eso había sido fácil para la madre de Donna.
No había sido tan fácil para Donna, la verdad sea dicha, y hace un año
ese habría sido su pensamiento predominante. Donna Noble, poniéndose a
sí misma en primer lugar.

Pero ahora no: sólo un breve tiempo con el Doctor le había demostrado
que ella no era la mujer que había sido entonces.

Y el abuelo, el pobre abuelo, recordando la muerte de la abuela, él
había seguido valientemente adelante por el bien de todos los demás,
tratando de coordinar a los abogados y a los directores de funerarias y
similares.

No es que mamá hubiera sido débil, pues Sylvia Noble no era así, y se
habían preparado para la muerte de papá, bueno, todo lo que se puede
estar, pero aún eso la perseguía. Podía verlo en los ojos de su madre,
era como si alguien le hubiera cortado un brazo y una pierna, y mamá lo
asimiló lo mejor que pudo. Habían estado casados durante treinta y ocho
años.

Donna suspiró.

---Te echo de menos, papá ---dijo en voz alta mientras frenó en seco en
una lavandería llamada Loretta.

Su teléfono vibró al entrar con un mensaje de texto, y lo leyó.
«Eh\ldots{} De hecho, podría estar equivocado. Puede que Loretta no sea
una florista. Lo siento.»

¿Cómo hacía eso? Ni siquiera tenía un móvil por lo que Donna sabía. ¿El
destornillador sónico, tal vez? ¿No había nada que no pudiera hacer?
Metiendo el teléfono en el bolsillo del abrigo, Donna decidió que sería
mejor ir en dirección a Turnham Green. Sabía que había una floristería
allí.

Hombres. Hombres alienígenas. Inútiles, todos ellos.

Lukas Carnes odiaba la tecnología. Lo que le hacía un poco raro, según
todos sus compañeros. Su madre tenía un PC, pero Lukas evitaba usarlo,
si era posible, a menos que fuera para escribir ensayos del colegio una
vez los había escrito a mano. Tenía un reproductor de MP3 al que su
hermano menor (que tenía ocho) tuvo que ponerle la música por él. Y ni
siquiera le hablaras de los problemas asociados con el uso de un DVD-R.

Él era, había decidido en su decimoquinto cumpleaños, un atrasado de una
época anterior, por allá cuando a los chicos con conocimientos
tecnológicos les llamaban frikis y las niñas salían huyendo de ellos.
Tristemente para Lukas, la mayoría de las chicas que conocía querían un
tipo que pudiera descargarse la música a veinte revoluciones y
desbloqueara un móvil que había comprado de un puesto chungo del mercado
de Shepherds Bush.

Por lo que Lukas no tenía novia.

Lo que añadió combustible a su odio apasionado por la tecnología.
Aceptaba que la necesitaba, simplemente no quería entenderla. Su cerebro
no estaba programado para entender compresiones MP3, el 3G y sistemas de
localización GPS. Sólo quería presionar un botón de encendido y que todo
funcionase. ¿No era eso por lo que el grupo de edad de su madre había
pasado durante la Primera/Segunda /Tercera Generación de tecnologías?
Así que podría presionar botones y las cosas funcionarían sin estar
obsoletas en seis meses ni inútiles en doce. En la televisión se hablaba
de que un día podrías chasquear los dedos y las puertas se abrirían,que
entrarías en una habitación y dirías ``luces'' y un ordenador lo
encendería todo, justo en el nivel adecuado.

Dios mío. ¡Era como su abuela! ¿Lo siguiente sería decir que no podía
entender la música pop y que el sexo era esa persona en Popworld?
Quince, no cincuenta, Lukas.

Así que, ¿por qué estaba de pie en la sucursal local de Discount
Electronics, viendo una demostración del más reciente Procesador de
Cuarta Generación en un ordenador portátil más fino que un pedazo de
cartón? Porque su hermano, su hermano Joe, de 8 años de edad con
conocimientos técnicos, se lo había pedido. Bueno, estrictamente
hablando, mamá se lo había pedido. Con el padre de Joe ausente, igual
que el de Lukas antes que él, el hijo mayor se había convertido en el
padre de facto para su hermano pequeño. Lo que venía bien a Lukas,
porque adoraba a Joe en secreto, pero nunca se lo diría. Y porque los
hermanitos pequeños necesitaban saber quién era el jefe, y el poder de
Lukas se perdería a la primera señal de debilidad.

Y Joe había dado bastante guerra después de que su padre se fuese,
metiéndose en problemas en la escuela y en el barrio, y a mamá le había
visitado la policía dos veces.

Así que Lukas se había llevado aparte a Joe y había explicado lo mejor
que pudo a un niño de 8 años que no era culpa de mamá que su padre se
hubiese ido, ni era la de Joe, y que perdiendo el tiempo con los chicos
mayores, ayudándoles a mangar coches y esas cosas, no estaba ayudando a
mamá.

Después de unos meses, Joe se había calmado. Pero ahora se aferraba a
Lukas en todo momento, y se enrabietaba si su hermano mayor no le
llevaba a todas partes. Incluso Lukas había empezado a llevar a Joe a la
escuela secundaria antes de dirigirse al Instituto Park Vale. Hecho qué
mamá agradecía que no acabase , así que eso era bueno.

Pero de vez en cuando, Lukas mismo quería enrabietarse, estar solo, no
ser el responsable.

Hoy era un día así, pero aquí estaba con Joe, viendo esta nueva
demostración junto con otras treinta personas, todos metidos en una
tienda en la que probablemente cabían de forma segura diez personas como
máximo. Qué Dios los ayudase si había un incendio.

Un mujer enorme (en todos los sentidos) se colocó delante de ellos, por
lo que Lukas izó a Joe en sus brazos para que pudiera ver mejor. Esto
significaba que Lukas no podía ver nada. Así, mientras que Joe (¡qué
mayor se estaba haciendo!) miraba con atención, la mirada de Lukas
erraba por la tienda.

Un tipo flaco vestido con un traje azul estaba tecleando en un ordenador
portátil de demostración, que probablemente iba a estar obsoleto al
final del día. El tipo estaba buscando algo en Internet, Lukas podía ver
pantallas repetidas mostrando un motor de búsqueda (¡ooh, un término
técnico!), y frunciendo el ceño. Era evidente que no estaba consiguiendo
los resultados que quería.

El hombre metió la mano en el bolsillo y sacó un tubo brillante,
parecido a un rotulador, y apuntó a la pantalla. Al principio, Lukas
pensó que iba a escribir en la pantalla del ordenador portátil, pero en
cambio el extremo del bolígrafo brilló azulado y Lukas observó con
asombro como las imágenes de la pantalla se descargaban y cambiaban a un
ritmo extraordinario. No, a un ritmo imposible. El hombre vestido de
azul sacó un par de gruesas gafas negras de otro bolsillo y se los puso
mientras miraba fijamente a las cambiantes pantallas. ¿En serio podía
leer tan rápido? Se dio cuenta de que Lukas lo miraba y sonrió, casi
con timidez. El bolígrafo brillante volvió a un bolsillo, las gafas a
otro.

Lukas se dio cuenta de que tenía la boca abierta, así que la cerró. El
Tipo del Traje Azul le guiñó un ojo a Lukas y estaba a punto de salir de
la tienda, cuando se adelantó y cogió un folleto sobre la demostración
que el hermano pequeño de Lukas estaba viendo. Después miró a la
multitud y se acercó. Lukas volvió rápidamente su atención de nuevo a la
demostración. O al menos a la parte posterior de la cabeza de la Señora
Gorda.

Tras unos minutos escuchando a una rubia hablando de lo revolucionario
que era el nuevo sistema informático, el Tipo del Traje Azul se encogió
de hombros y murmuró algo así como ``imposible'', ``no en este planeta''
y ``contradiciendo el Protocolo Decimoctavo de la Proclamación de las
Sombras'', y en ese momento Lukas decidió que, con o sin bolígrafo
brillante, este hombre era probablemente un loco. Quizá debería alejarle
de Joe, por si acaso el tipo tuviese un cuchillo.

Lukas se inclinaba para susurrar al oído de Joe que tal vez ya era hora
de irse a casa, cuando el Tipo del Traje Azul le dio un codazo.

---Así que todo el mundo está dentro de todas las tiendas de
electrónica, ¿no?

---¿Perdón?

---Las calles estaban bastante vacías. Mientras caminaba, me he dado
cuenta de que todo el mundo está en este tipo de tiendas, viendo estas
demostraciones.

---Hoy es el lanzamiento ---se encontró explicando Lukas---. Todo el
mundo está interesado. ---Tú no ---respondió el Tipo del Traje Azul.

Lukas se encogió de hombros.

---Mi hermano pequeño lo está.

---Ah. Ya veo.

Lukas trató de alejarse, pero fue cercado entre otro hombre por un lado
y la Señora Gorda por delante. El Tipo del Traje Azul volvió a sacar su
bolígrafo brillante.

---No me hagas caso ---dijo.

Pero Lukas le hizo caso. Mucho.

---¿Por qué estás aquí? ---preguntó.

El Tipo del Traje Azul se encogió de hombros.

---Bueno, en primer lugar, estoy dejando que una amiga vaya a su casa.
En segundo lugar, me preguntaba por qué todo el mundo está aquí. Y en
cuarto lugar, ahora estoy realmente preocupado por la tecnología de ese
portátil.

Lukas sabía que iba a arrepentirse de esto.

---¿Y en tercer lugar?

---¿En tercer lugar? ---el Tipo del Traje Azul parecía confundido, y
después sonrió como si algo le hubiera vuelto a la memoria de
repente---. Ah, sí, en tercer lugar, vine a buscarte, Lukas Samuel
Carnes ---le tendió la mano---. Soy el Doctor y estoy aquí para salvarte
la vida.

Dara Morgan bebía su café lentamente. En parte para demostrar que tenía
buenos modales, y en parte porque estaba demasiado caliente para hacer
nada más. Pero probablemente les parecerían buenos modales al señor
Murakami y su delegación.

---Entonces Señor Morgan ---estaba diciendo el banquero japonés---,
¿tenemos un trato? Los ojos azules de Dara Morgan brillaron con
picardía mientras miraba a Caitlin, de pie junto a la puerta de la
oficina.

---¿Qué piensa usted, Cait?

Caitlin se acercó, sus piernas largas y su falda corta se dibujaban
claramente en los ojos de algunos miembros del séquito del señor
Murakami, pero, observó Dara Morgan, no en los del propio señor
Murakami. Bien.

---Creo que es un buen trato, señor ---ronroneó---. Si Murakami-San
puede llevar a cabo el M-TEK en todo Oriente para el domingo,
sería\ldots{} magnífico.

Dara Morgan se apartó el pelo de los ojos.

---Justamente en menos de dos días, a las 15:30 hora de Tokio.
¿Factible?

El Señor Murakami frunció el ceño.

---¿Por qué el domingo? El plazo es ridículamente corto.

Dara Morgan se limitó a sonreír.

---Digamos sólo que todo el trato depende de eso. Necesito esa garantía,
Murakami-San, o me iré a otro sitio.

---Pero de esa forma usted tiene aún menos posibilidades de un acuerdo
para llevarlo a cabo para entonces ---dijo el japonés.

Dara Morgan asintió.

---Lo sé. Pero afrontémoslo, con el dinero que proporcionará el M-TEK,
las empresas más pequeñas que la suya, algunas quizá más hambrientas,
estarían más dispuestas a satisfacer mis\ldots{} los requisitos de
MorganTech ---tomó otro sorbo de café---. Estará en el contrato, con
cláusulas de penalización.

---¿Cuáles serían?

---Catastróficas. Para todo Japón.

El séquito del señor Murakami se acercó a su jefe un centímetro más
cerca.

---¿Eso era una amenaza, Señor Morgan? ---preguntó en voz baja.

---No ---dijo Dara Morgan---. Yo no amenazo. Los bárbaros amenazan. Los
idiotas amenazan. Simplemente expongo hechos.

---Es una gran oportunidad ---intervino Caitlin---. Por favor, piense en
ello durante la cena. Esta noche. A nuestras expensas.

---Por desgracia, no podremos acompañarle ---añadió Dara Morgan---, pero
es libre de escoger cualquier restaurante de Londres, el que más le
guste, y todos los gastos serán cubiertos por MorganTech. De hecho,
insisto en ello.

---¿Todos los gastos?

---Los relativos a comida y bebida, sí.

---Ah. En ese caso, tendrá mi respuesta antes de la medianoche de hoy.

El Señor Murakami se levantó y Dara Morgan hizo lo mismo, inclinando
ligeramente la cabeza mientras lo hacía. El Señor Murakami respondió del
mismo modo, incluyendo a Caitlin y ella le saludó con la cabeza a él y a
todo su séquito. Terminadas las formalidades, la delegación japonesa se
dirigió a la puerta de la suite, pero el Señor Murakami se volvió una
ultima vez.

---En serio, ¿por qué el domingo? ¿Por qué las 3:30 de la tarde?

---Porque algo grande va a pasar en todo el mundo el lunes a las 15:00
hora de Reino Unido. Eso son las 23:00 hora de Tokio. Pero necesitamos
todos los tratos en marcha Yo también he hecho un trato, ya ve, pero es
más bien como una cadena de adquisición de propiedades: un eslabón que
se rompa y todo el trato se derrumba. Y entonces todos sufriremos.
---¿Todos?

---Universalmente.

Dara Morgan miró de reojo a Caitlin, y de inmediato fue a escoltar al
Señor Murakami fuera de la habitación. Momentos después, los japoneses
se habían marchado y Caitlin estaba de regreso al lado de Dara Morgan.
Él estaba de pie frente a un enorme ventanal con una gran vista
panorámica del Oeste de Londres. Podía ver el nuevo estadio de Wembley,
el Centrepoint, el London Eye y otras estructuras altas de Londres.

---El lunes ---sonrió---, y este planeta será de Madam Delphi.

Caitlin asintió.

---Al fin. La venganza es suya.

Se cogieron de las manos y las apretaron con fuerza, mirando a las
pantallas del ordenador, que parecían estar tarareando una melodía, muy
levemente, causando ondas en una de las pantallas para hacer pulsos de
forma fraccionada en el tiempo.

---Bienvenida de nuevo ---le dijeron juntos.

Donna estaba de pie al final de Brookside Road y respiró profundamente.
No había pasado mucho tiempo desde la ultima vez que había estado aquí
(de hecho, para su madre, había sido menos tiempo aún), pero cada vez
que llegaba a casa, había momentos incómodos. ``¿Dónde has estado?'' y
``¿Sigues dando vueltas con ese horrible Doctor?'' y ``¿Por qué no
llamas?'' y ``¿Tienes trabajo ya?''. Por supuesto, el abuelo Wilf sabía
dónde estaba, se lo había dicho todo desde el principio. Pero su madre,
bueno, ella no era una persona que lo pudiera entender. No era de las
que pensaran que salvar Oods, detener guerras generacionales o
asegurarse de que Carlomagno se reuniese con el Papa realmente equivalía
a un ``buen'' trabajo como era tomar notas o hacer pedidos.

Respirando profundamente, caminó hacia la casa que en realidad no había
sido su hogar mucho tiempo. Después de su desastrosa boda y un viaje un
poco menos desastroso a Egipto, sus padres se habían mudado de la casa
adosada donde Donna había crecido a este nuevo chalet. Había sido una
revolución, agravada por una Donna sin trabajo y Wilf estando en
principio enfadado porque pensaba que tendría que dejar a sus colegas
astrónomos abandonados. Al final resultó que, por supuesto, era más
fácil llegar a los huertos desde la nueva casa, así que terminó siendo
feliz después de todo.

Pero el padre de Donna no había estado bien por mucho tiempo, y en
muchos sentidos el traslado había sido idea suya, su deseo de encontrar
algún lugar nuevo donde vivir, que le diese nuevos retos.

Se había aburrido en la vieja casa. Había construido todos los armarios,
puesto estanterías en todas las paredes, pintado todos los techos que
fue capaz, y necesitaba algo nuevo para mantenerlo activo desde que su
enfermedad le había hecho jubilarse anticipadamente. Preparar el nuevo
lugar con las exigentes especificaciones de mamá sería exactamente el
desafío adecuado. Llevaban viviendo allí tres meses cuando su padre
murió.

Donna y Wilf habían tomado el relevo de hacer todos esos trabajos que
papá había estado haciendo, pero nunca fueron buenos del todo, nunca
fueron ``cómo tu padre lo habría hecho''. Lo que no era del todo
sorprendente, Wilf era veintitantos años más viejo, y Donna nunca había
levantado una brocha o un martillo en toda su vida.

Dios mío. ¡Qué superficial era Donna Noble antes de volverse a
encontrarse con el Doctor! Había aprendido no solo a valerse por sí
misma, sino a darse cuenta de que podía. Su vida familiar era una
verdadera situación del huevo o la gallina. ¿Habría sido una inútil en
casa porque sus padres siempre le habían permitido serlo, o su madre
pensaba que era inútil porque lo era? Y hablando de ello, hablando de
cualquier cosa, con Sylvia Noble era raramente una experiencia positiva.
A Donna le encantaría decir que la amargura y el resentimiento de su
madre era a causa de la muerte de papá, pero la verdad era que a Sylvia
siempre le había decepcionado su hija. Rara vez lo escondía. Y Donna
nunca entendió el por qué. ¿Habría deseado un hijo? ¿Habría deseado una
abogada de altos vuelos o una ejecutiva de empresa que fuera lo
suficientemente rica como para enviar a sus padres a vivir en el campo
en una pequeña casa rural del siglo XVI donde podrían cuidar cabras?
¿Había empeorado desde que papá había muerto? ¿Habría sido mejor si
ella se hubiera casado con Lance? ¿Debería haberle dicho la verdad de
ese día a mamá como había hecho recientemente con el Abuelo?
Probablemente no, porque a Sylvia no le gustaba que la gente fuese
abierta y honesta. ``Esos corazones sangrantes que llevan sus corazones
en las manos'' era una analogía con la que la había torturado una vez, y
resumía su opinión real sobre la gente que era honesta.

Donna recordó haber leído una vez en un artículo de una revista acerca
de cómo los padres nunca esperaban entender realmente a los adolescentes
y su mejor apuesta para una convivencia armoniosa era sólo tolerar esos
tres o cuatro años de pesadilla. Pero, ¿había algún manual para hijos e
hijas sobre cómo tratar con padres negativos? Era realmente imposible
discutir con una madre, tenía el botón ``culpabilidad'' incorporado para
presionarlo y prohibir que les dijeses todas las cosas que querías
decirles, sea lo que sea que te lanzasen. Donna amaba a su madre, no
había vuelta de hoja. Y no tenía ninguna duda de que Sylvia Noble amaba
a su hija. Simplemente no estaba segura del todo de que ambas en
realidad se gustasen mucho.

---Hola, Donna ---dijo el Señor Baldrey desde el otro lado---. ¿Has
tenido un buen viaje? ---Sí, gracias ---Donna le devolvió la
sonrisa---. ¿Cómo está Seymour?

---Oh bien. Quejándose todavía de su próstata ---gimió el vecino.

Donna pensó que la conversación había llegado tan lejos como francamente
quería que fuera y aceleró el paso hacia la casa. Un gato se sentó junto
a un poste de luz, mirando con recelo como Donna se aproximaba, no muy
seguro de si era amiga o enemiga. Donna hizo ruidos chirriantes para
atraer su atención. Salio corriendo disparado. Ah bien.

El coche de mamá estaba fuera en la calle (tienes un coche, mamá, úsalo)
y Donna tocó el capó al pasar. Frío. Entonces Mamá no había salido hoy.
Es curioso cómo había adquirido estas pequeños hábitos para descubrir
cosas al viajar con el Doctor. Como lo de si un coche había sido
conducido. La antigua Donna nunca habría pensado en eso. A la antigua
Donna no le habría importado. La antigua Donna se había ido. Gracias a
Dios. Su vida era un trillón de veces mejor estos días. Si tan sólo
pudiera implicar a su madre en ella, pensó. Ese último pedacito del
rompecabezas, ese último trozo de aceptación de cada uno de ellos.

---Oh ahí estás, Señorita ---dijo una voz familiar detrás de ella---. Me
preguntaba si volvería a verte hoy.

Donna no se molestó en darse la vuelta.

---Hola, mamá ---dijo.

---Ah, sí, ``Hola, mamá'', porque eso es suficiente, ¿no? Un instante
el aire está lleno de gases asfixiantes y al siguiente el cielo está en
llamas y ya está. Ni idea de donde está mi única hija. Ninguna llamada,
ni mensajes, ni siquiera un mensaje a tu abuelo, así podría hacerme
callar. Nada.

Donna se detuvo en la calle y se volvió hacia su madre, cogiendo
automáticamente dos de las cuatro bolsas de la compra que llevaba en la
mano para ayudarla. La antigua Donna no habría considerado eso tampoco.

---Encantada de verte también ---dijo Donna---. ¿El Abuelo tiene puesta
la tetera? Me iría bien una tacita, muchas, para serte sincera.

Sylvia Noble se encogió de hombros y adelantó a su hija.

---¿Sabes qué día es hoy, ¿verdad? ---le replicó.

Y Donna se detuvo en seco. Por supuesto que sabía qué día era hoy. ¿Por
qué diablos creía que estaba allí? ¿Cómo se atrevía incluso a
preguntarlo? La puerta principal se abrió y Sylvia pasó al lado del
Abuelo Wilf y se fue sin decir nada a la cocina.

---¿Sabes lo que me acaba de preguntar? ---Donna le siseó después de
besarle en la mejilla.

Wilf elevó los ojos al cielo.

---Va a ser uno de esos días, ¿no?

Donna abrió la boca para responder pero se detuvo. La antigua Donna se
habría marchado, en el acto. La antigua Donna hubiera comenzado una
pelea con su madre, lanzando palabras como ``actitud'' y ``sea lo que
sea'' y ``egoísta''. La nueva Donna no lo hizo.

Porque la nueva Donna, frustrante como era, comprendió que lo que Sylvia
Noble probablemente quería y necesitaba hoy era llorar un rato. Pero
siendo Sylvia ``Yo no llevo mi corazón en la mano'' Noble significaba
que nunca haría eso. Y tristemente iba a ser el trabajo de hoy de la
nueva Donna asegurarse que lo hacía, antes de que la represión de su
madre le hiciera más daño del que ya le hacía.

El Doctor bajaba a grandes zancadas por la calle principal de Chiswick,
mirando dentro de varias tiendas donde la gente se quedaba mirando las
nuevas demostraciones de portátiles. ---Es sólo un ordenador
---murmuró---. ¿Por qué tanto interés?

---Ya están obsoletos ---dijo una voz joven a su lado---. El M-TEK, eso
es el futuro.

El Doctor miró entonces hacia abajo. El que había hablado apenas le
llegaba a la altura de las rodillas. Era un niño pequeño. El Doctor lo
había visto antes en alguna parte. Entonces, corriendo hacia él, vio a
Lukas Carnes y se dio cuenta que era el hermano pequeño que había estado
llevando a hombros.

---¡Joe! ---gritó Lukas---. ¿Cómo te alejaste tan rápido?

---Yo no soy tu prisionero ---replicó Joe, y el Doctor le dio a un Lukas
recien llegado y exhausto una mirada que decía ``te lo dije,
compañero''.

Lukas empujó a Joe lejos del Doctor.

---Déjalo en paz, tócale y tendré a la policía aquí ---como para
subrayar esto, Lukas tenía su móvil en la mano y listo.

El Doctor no se molestó en señalar que Joe le había encontrado. Tampoco
señaló que Lukas fue claramente agresivo solo por lo que había dicho en
la otra tienda. Tenía que recordar que a las personas no siempre les
gustaba conseguir pistas sobre su futuro. Siempre tropezaba con eso.
Spoilers, una vez dijo alguien.

---¿Que significa eso de que vas a salvarme la vida?

El Doctor se encogió de hombros

---Ya lo he hecho.

---¿Hecho qué?

---Salvé tu vida.

---¿Cómo \ldots{}? ¿Cuándo y por qué?

El Doctor sacó su papel psíquico del bolsillo interior de la chaqueta y
se la mostró a Lukas. Tenía la fecha correcta en él, el nombre de la
tienda y el mensaje ``SALVAR LA VIDA DE LUKAS SAMUEL CARNES AL IMPEDIR
QUE SU HERMANO COMPRE UN M-TEK''.

---No sé quién lo escribió, porque no reconozco la letra. Pero apareció
en el papel unos veinte minutos después de llegar a Chiswick. Siempre he
pensado que ignorar mensajes es una mala cosa. Asi que, has sido salvado
y mi trabajo está hecho, todo lo que necesito hacer ahora es averiguar
si el local Loretta es una lavandería, una floristería o una tienda de
café en este período. Es uno de los tres en 2009, pero no estoy seguro
de cual. ~---Lavandería.

Lukas puso a Joe detras suya.

---No hables con ese hombre ---le regañó a su hermano pequeño.

---Pero tu estás hablando con él ---Joe protestó, no sin razón.

---Eso es diferente ---dijo Lukas, dándose cuenta con horror que esto
era exactamente lo que su madre solía decir cuando ella había dicho algo
que él consideraba hipócrita o injusto.

El Doctor se volvió.

---Encantado de conoceros, muchachos, pero pensé que podría desviarme a
un encantador restaurante en Brentford, por el canal de allí, para pasar
el tiempo antes de reunirme con mi amiga.

---¿Qué? ---dijo Lukas, dandose patadas mentales por preocuparse.
Simplemente deja que el bicho raro se vaya, se instó a sí mismo. Pero su
boca no paraba de hacer preguntas. ---¿Qué habría pasado si Joe hubiese
conseguido un M-TEK?

Y el Doctor le miró.

---Ni idea. No sé lo que es un M-TEK. Supuse que era ese portátil que
estabas pensando en comprar. Hice una búsqueda, pero no pude encontrar
ninguna referencia. Entonces vi que estabas viendo la demo, como todo el
mundo parecia hacer, así que supongo que eso es M-TEK.

---No ---dijo Joe, empujando hacia delante---. Ese es el nuevo Psiryn
Book Plus. Es basura. Yo quería un M-TEK. Pero no estaba pensando en
comprar uno, ni para Joe ni para mí. ---Entonces, ¿qué es un M-TEK?
---el Doctor frunció el ceño.

Lukas suspiró.

---¿Cómo puedes salvarme de ello, si no sabes lo que es?

---Si supiese todas las cosas de las que estaba salvando a la gente
antes de que intentase salvarlas, salvaria a muy pocos, ya que pasaria
todo el dia averiguando sobre la cosa de la que los salvaria, ¿no?

Lukas y Joe se miraron el uno al otro.

---Eres divertido ---dijo Joe.

---Gracias.

Lukas negó con la cabeza.

---A casa ---dijo a Joe---. ¡Vamos! ---casi arrastró a su hermano
pequeño

---Adiós, Doctor ---gritó Joe.

El Doctor saludo mientras los dos niños desaparecian en una calle
lateral. Luego empezó a vagar hacia el extremo inferior de Chiswick,
hacia Brentford. Y ese bonito restaurante italiano en la plaza. Luna
Piena. No había tenido una comida italiana decente en años. Quizás
siglos.

``DESDE 1492'' apareció en el papel psíquico.

Lo cual era extraño, porque el papel psíquico no funcionaba así. Al
menos, no antes. Ya era bastante malo que la gente lo estaba usando más
y más para enviarle mas y mas mensajes esos días, pero cuando le comenzó
a contestar espontáneamente, era el momento tal vez para darle un
servicio de dos mil hojas.

Empujó la cartera de cuero con el papel en ella de nuevo en el bolsillo
de su chaqueta y trató de olvidarse de ella. En el fondo de su mente,
sin embargo, todavía tenía una preocupación persistente, un eco de la
pregunta de Lukas Samuel Carnes: ¿qué era un M-TEK y cómo había salvado
a Lukas de ello? A lo que la respuesta era obvia. No lo había hecho.

Así que Lukas todavía estaba en peligro (si el papel psíquico era de
fiar), y tenia que salvarlo.

Ah, y otra pregunta necesitaba una respuesta. ¿Cómo había sabido el
hermano pequeño de Lukas, Joe, que se llamaba Doctor? Así que\ldots{}
¿Luna Piena o involucrarse? No era una decisión complicada, ¿no? La
comida estaba bien, pero un misterio, era mucho mejor.

Se preguntó cómo estaría Donna y si deberia pasar por ahí y decirle que
podría estar ocupado por un par de días. No, ella probablemente estaba
mejor haciendo cosas de familia.

Y así se dio la vuelta y se dirigió hasta la calle lateral detrás de los
dos chicos.

En la planta suite del ático del Hotel Oráculo, Dara Morgan y Caitlin
estaban mirando un banco de monitores de pantalla plana, conectados a la
computadora, por los cables de fibra óptica por los que Terry Lockworth
y Johnnie Bates habian muerto mientras los conectaban.

En la mayoría de las pantallas habia una onda sinusoidal, pulsando
rítmicamente, como si el equipo estuviera respirando. Algo que estaba
haciendo. De alguna manera.

Pero en la pantalla central, la más grande, estaba una imagen, una foto,
tomada de cámaras CCTV que habían sido automáticamente hackeadas y
mejoradas para tener una resolución casi perfecta, de acuerdo con los
parámetros de los ajustes de la computadora.

---Madam Delphi ---preguntó Dara Morgan---. ¿Qué es esto?

Su dedo trazó el contorno. Era una caja azul alta en un callejón de
Chiswick entre dos contenedores de basura.

---La TARDIS ---contestó una voz fuerte y femenina, haciendo eco a
través del cuarto,mientras las ondas sinusoidales en las otras pantallas
pulsaban y cambiaban a medida que hablaba.

---Él está aquí ---dijo Caitlin.

---Ya ---Dara Morgan asintió con entusiasmo---. Quinientos años, como
las leyendas predijeron. El Portador del Caos.

---Quinientos diecisiete años, un mes, cuatro días ---corrigió Madam
Delphi---.No permitiamos el cambio cósmico hace quinientos años. Eso fue
un poco\ldots{} desafortunado. ~Caitlin se dirigió a la computadora.

---Pero, Madam Delphi, ha habido otros intentos\ldots{}

---Y debido a ese cambio cósmico, debido a que el universo da respiros
superficiales, así como profundos, las alineaciones nunca han sido
perfectas.

---Pero el lunes todo será perfecto ---Dara Morgan acarició las
superficies de Madam Delphi---. Y tu tendrás tu venganza.

\begin{center}\rule{3in}{0.4pt}\end{center}

---En el Doctor. En la humanidad. En todo el universo ---dijo Caitlin
emocionada.

---Oh, por supuesto ---Madam Delphi pulsó sus ondas sinusoidales---.
Absolutamente. Amo la parte de la venganza, mis queridos. Pero
especialmente para el Doctor.

Eran las 5 PM en el Reino Unido. Así que, en la soleada Nueva York, las
sombras de la Gran Manzana se estiraban mientras el sol del mediodía se
ponia, cubriendo la ciudad en una inusualmente húmeda manta. Esto no era
una buena noticia para los habitantes de la torre de oficinas MorganTech
en la 52 con la Séptima. El aire acondicionado había fallado un par de
horas antes, y las fuentes de agua potable automáticas había dejado de
bombear agua fría en los enfriadores de agua. La razón principal de esto
era que toda la electricidad en la manzana estaba apagada. Las puertas
principales habían fallado en primer lugar, seguido de los móviles,
equipos informáticos, aire acondicionado y así sucesivamente.

Le había costado a Melissa Carson, en la recepción, unos minutos el
darse cuenta de que todo había salido mal. Trató de llamar a
mantenimiento. Obviamente, como todo lo que mantenia Mantenimiento habia
fallado, no había manera de conseguir que los de Mantenimiento
mantuvieran algo. Esto molestó a Melissa, asi que cometió el crimen
corporativo de dejar su escritorio para encontrar a alguien.

En cambio, lo que encontró, aparte de ascensores estancados
probablemente conteniendo pasajeros rápidamente deshidratandose, y las
puertas electronicas internamente cerradas, era un montón de polvo en el
suelo cerca de una caja de conexiones en el sótano. Es de suponer que
Mantenimiento había estado haciendo algo con el cableado y habían
fusionado los sistemas. No se le ocurrió a Melissa (y por qué debería
hacerlo?) que las cenizas que estaba quitando casualmente de sus tacones
Dolce \& Gabbana habían sido una vez un chico al que había saludado
llamado Milo. Pero si que se preguntó donde estaban Milo y los chicos.
Casualmente, mientras pisoteaba de vuelta a su escritorio frustrada,
cerró la caja de conexiones abierta.

Sin esperar, la fibra óptica instalada recientemente volvio a la vida,
latiendo luz púrpura a lo largo de su red. Los ocupantes del edificio,
ya quejandose acerca de ascensores atascados, aires acondicionados rotos
y computadoras muertas, tenían diez segundos para registrar que sus PCs
se habian encendido. Como hacen los trabajadores de oficina, todo el
mundo se adelantó y tocó sus teclados. Un arco masivo de luz púrpura
latía en todo el edificio, dandoles a todos, no sólo los que utilizaban
los PCs.

Ni una persona, cucaracha o polilla en el sótano se salvó del pulso
púrpura de energía.

Cuarenta y dos segundos después de que Melissa Carson había cerrado esa
caja de conexiones, todos los ciento siete seres humanos, dieciocho
ratas, dos mil criaturas de diferentes tamaños y formas, pero con seis
patas o más y tres palomas en el techo, estaban todos muertos.

---Tenemos un pequeño problema, chicos ---Madam Delphi pulsó a Dara
Morgan y Caitlin---. El edificio MorganTech en Manhattan está
desconectado. Los terminales estan terminales. Hubo un ruido como el de
una risa electrónica, y las ondas sinusoidales pulsaron acorde a este.

Dara Morgan frunció el ceño, tocando en otra parte del grupo de
monitores y teclados de Madam Delphi.

---Yo no estoy equivocado ---informó el equipo.

---Lo sé ---dijo Dara Morgan rápidamente---. Nunca estás equivocado.
Estoy tratando de averiguar cuál es el problema.

---Error humano ---informó Madam Delphi---. ¿Qué otra cosa podria ser?
Quiero decir, seamos honestos entre nosotros, ustedes siempre son los
eslabones más débiles de la cadena.

---Necesitamos a Nueva York ---dijo Caitlin.

---Bueno, no tenemos a Nueva York ---dijo el equipo.

---¿Puedes anular el pulso?

---No ---espetó Madam Delphi---. Demasiado tarde de todos modos.
Honestamente, queridos, están perdiendo el tiempo. Voy a ver si puedo
transferir recursos a una copia segura del servidor y volver a empezar.

Dara Morgan perdió la calma por primera vez en mucho tiempo.

---¿No lo entiendes? No tenemos tiempo para empezar de nuevo. Tenemos
que tener los M-TEKs de Nueva York en línea a las 10 am hora de
Manhattan el lunes.

---Es la ciudad que nunca duerme, si recuerdo la letra ---dijo el
equipo.

---Sí, tal vez nunca duerme. Pero prácticamente para de trabajar a las 5
pm en un viernes y no trabaja los fines de semana.

---Oh, mi querido muchacho, ten un poco de fe. He estado en este tipo de
cosas a través del universo por unos pocos millones de años ya.
Tendremos que hacer una fusión esta tarde. Una adquisición hostil por
MorganTech de una pequeña empresa en\ldots{} ooh déjame adivinar\ldots{}
ah, sí, mira, aquí hay uno ---en la pantalla grande de Madam Delphi
apareció una imagen de un edificio de oficinas más bien pequeño (para
Nueva York), todo cromo y cristal con gente pululando dentro---. Estoy
accesando a sus sistemas\ldots{} ahora. Ooh sí, un montón de gente.
Hacen servicio y reparaciones de hardware, una co-firma propiedad de la
Mafiosa, una tríada china y en un principio establecido por el blanqueo
del IRA. Ninguno de ellos saben el uno del otro, obviamente. Fácil de
tomar el control porque ninguno de ellos va a ponerse de pie y gritar al
respecto. Lexington con la Tercera, bonito lugar, conocí a un cibercafé
allí una vez. Nunca volvió a llamar ---fila tras fila de cifras pasaron
a través de las pantallas, demasiado rápido para que Dara Morgan las
contara y luego las ondas sinusoidales regresaron, pulsando de nuevo
mientras Madam Delphi ronroneaba a los dos---. Kittel Software Inc,
ahora una subsidiaria de MorganTech. Espero que no te importe, tuve que
estar de acuerdo en permitir que Harvey Gellar permanezca como CEO, con
un conjunto de acciones y un voto en el Consejo.

Cait frunció el ceño.

---¿No será un problema?

---En una hora dieciocho minutos, Mr. Gellar entrará en el
elevador\ldots{} lo siento, ascensor\ldots{} para ir a la planta baja.
En una hora veintiun minutos, el ascensor se atascara entre los pisos
diecinueve y dieciocho. El presionará el botón de alarma. Va a ser la
última cosa que haga.

Van a encontrar el cuerpo dentro de, oh, un par de horas y se asumira
que fue un ataque al corazón. Ya he reescrito su testamento para que
cuando muera, su primo tercero de Irlanda, Dara Morgan de MorganTech, lo
herede todo.

---Yo no soy su primo tercero\ldots{}

---Ahora lo eres, de acuerdo con los archivos del FBI ---las pantallas
de Madam Delphi se oscurecieron ligeramente, la onda sinusoidal tomando
una tonalidad roja mientras latía---. Tu subestimación constante de lo
que puedo hacer, Dara Morgan, está empezando a aburrirme. ---Lo siento.

---Bien. Ahora entonces, esta nueva rama de MorganTech parecera haber
estado planeando el lanzamiento del M-TEK del lunes todo este tiempo.
Tienen las especificaciones, los detalles, la base de clientes, todo.
Todo lo que necesitamos hacer es conseguir un par de lacayos el fin de
semana que eliminen el cableado y lo reemplacen con nuestra fibra
óptica. Y\ldots{} ahi, subcontratistas reservados y asignados. Fácil.

Caitlin miró a Dara Morgan.

---Sí, Madam Delphi. Fácil.

---Alegrense mis queridos. Ahora, si está bien con ustedes niños, yo
sólo voy a descargar el Coronation Street Ómnibus de esta semana. ¿Qué
hará ese dulce y pequeño David Platt ahora?

Sylvia estaba guardando las compras en la cocina. Cuidadosamente, todo
en su lugar. Como siempre. Ella estaba, sin embargo, dejando que el
cajón o puerta del armario extraño se cerrara un poco ruidosamente.

---¿Qué he hecho ahora? ---Donna susurró a su abuelo desde el sillón que
daba a la televisión en la sala de estar. Cuando su padre se sentaba y
se reia de ``The X-Factor''. Y de las repeticiones de ``Dad's Army''. Y
ese programa donde\ldots{} donde\ldots{} bueno, todos sus programas, de
hecho. Y de repente, ella quería ir al sofá al lado de su abuelo, pero
el se trasladó a lo largo, por lo que se sentó cerca de su silla.

---No sé lo que quieres decir, querida ---dijo, sin mirarla.

---Claro, porque mamá esta usando lo mas fino de IKEA para hacer el solo
de bateria de algo que Ozzy Osbourne escribió porque ella tiene de
repente un interés en el rock.

---Oh, ella sólo\ldots{} esta siendo tu madre. Ya sabes\ldots{}

---No, abuelo. No, no lo sé.

Donna suspiró y miró a un periódico sobre la mesa auxiliar. Su historia
principal era sobre Q-Mart y Betterworth abriendo supermercados rivales
en Park Vale.

Misterioville.

---Estoy aquí. Chiswick. Londres W3. La Tierra. Ayer estaba en otro
planeta, deteniendo a robots que luchaban una guerra civil. Una semana
antes de esto, estábamos en el Bazar Garazone montando caballos de seis
patas. De repente agarró la mano de Wilf. ¡Tenían seis patas! Seis.
Quiero decir, ¿a que velocidad estábamos galopando? Fue brillante. Me
encantó. Y el marciano estaba gritando a todo pulmón: ``¿Dónde está el
interruptor de apagado?''. Porque pensó que podían ser detenidos así
como así.

---Él no es un marciano, ¿o si? Pensé que dijo\ldots{}

---No, abuelo, él no es marciano. Es una broma. ¿Te acuerdas de las
bromas? Ya sabes, ese momento en que abres la boca y dices ja-ja-ja.
Solíamos hacerlo, incluso en esta casa una o dos veces ---Donna se quedó
mirando el rostro arrugado de su abuelo. ¿Cuándo habia envejecido tan de
repente? ¿Acaso influyó la muerte de papá también? ¿Qué pasó con el
hombre que solía llevarla a dar una vuelta en su viejo Aston? ¿Quién
presumia de ella con sus viejos compañeros paracaidistas en el Social?
¿Cuándo fue reemplazado por el anciano de pelo blanco, sentado frente a
ella? ¿Cuando pasó que la idea de volver a casa la llenara de tanto
miedo? ¿Era éste el inconveniente de estar con el Doctor? ¿Lo normal
ahora era alienigena?

---Tengo algo que decirte, cariño ---dijo su abuelo---. Creo que va a
animarte. Espero que lo haga.

Buenas noticias al final. Donna sonrió.

---Bueno,sigue, entonces. Cuentamelo.

Su abuelo abrió la boca para hablar, pero Sylvia eligió ese momento para
entrar en la sala de estar y acomodarse en el sofá junto a él.

---Así que, ¿dónde has estado, Donna Noble?

Donna abrió la boca para contestar, pero su abuelo llegó primero.

---Ella ha estado haciendo equitación, Sylv. En Dubai.

---¿Dubai? ¿Cómo demonios has pagado eso? ---Sylvia suspiró---. Oh
tonta de mí, el Doctor te llevó, ¿sí?

Donna asintió.

---Sí. Él pagó por ello. Casi me casé con un rico jeque petrólero y viví
en su harén, pero ¿sabes qué? Pensé que era más importante estar aquí
hoy. Con vosotros dos.

---Bueno, eso es bueno, estoy segura ---dijo Sylvia---. Tal vez, si
salir con los ricos y famosos de la OPEC no ha sido demasiado exigente,
tal vez podrias hacernos una taza de té?

---Claro ---Donna se puso de pie, pero no fue lo suficientemente rápida
como para impedir que Sylvia soltara una burla mas.

---¿Puedes recordar dónde estan las bolsitas? ¿Y la jarra?

Y eso fue todo. Hora de sacarlo todo.

---¿Qué he hecho yo, mamá? Quiero decir, realmente, ¿dónde me
equivoqué? Todo lo que alguna vez me has dicho era que saliera, que
hiciera cosas, consiguiera un trabajo y viviera mi vida. Y lo he hecho.
Y todavía no es lo suficientemente bueno, ¿verdad? ---se sentó de
nuevo---. Todavía no soy lo suficientemente buena, ¿verdad? ¿Estaba
papá tan decepcionado conmigo como tu lo estás?

---No te atrevas a hablar de tu padre así ---Sylvia gritó, mucho más
fuerte de lo que parecía necesario.

---Ya, ya\ldots{} ---Wilf comenzó, pero Sylvia le hizo callar
bruscamente.

--- No, no, ya va siendo hora de que la señorita Tonterías tenga un par
de verdades --- Sylvia se inclinó hacia adelante, levantando un dedo en
el aire---. Tu abuelo y yo estamos muy preocupados, ¿lo sabías? Vas y
vienes sin apenas una palabra, bajas de la Luna cuando te viene bien y
en un rato te vuelves a ir. No sé si estás viva o muerta. No sé si cada
vez que suena el teléfono serás tú diciéndome que estás en Timbuctú o si
alguien llama a la puerta sea un policía que han encontrado tu cadáver
en el Támesis. Llega una carta para ti y la dejo en la repisa, esperando
que de alguna manera signifique que volverás pronto. Pero después de un
par de semanas, la tengo que tirar en tu cama porque no funciona. No te
devuelven a casa. Desde que conociste al tipo ese del Doctor, te has
vuelto una persona distinta. Donna se quedó mirando a su madre en un
estado de shock silencioso. ¿De dónde venía todo eso? --- ¿Por qué
demonios tienes que suponer que estoy muerta? Es una locura. --- No es
una locura, tampoco no es irrazonable. Es lo que pienso.

Y cada día que pasa sin saber de ti, lo pienso más. Tal vez si fueras
madre, tal vez si tuvieras hijos, hijos estúpidos y egoístas que no
piensan por sí mismos, lo entenderías. Sylvia estaba temblando.

Donna tenía mucho miedo. ¡Ella, de alguna manera, sin querer, sin saber
exactamente porqué, había hecho llorar a su madre! ¡Por todas las
razones equivocadas! ¡Cómo si hubieran buenas! Se supone que no tienes
que hacer llorar a tu madre\ldots{} --- ¡No me voy a morir, mamá! Ningún
policía llamará al timbre y te dirá que estoy muerta. ---¿Por qué no?
--- dijo Sylvia casi gritando, no de una manera enfadada, pero con
lágrimas cayéndole por las mejillas, no, caían por su cara, como si
fueran ratones húmedos---. ---. ¿Por qué no? ¡Es lo que pasó con tu
padre! El silencio que hubo de repente era terriblemente estremecedor.

Y entonces Donna cruzó la habitación, abrazando a su madre mientras ésta
lloraba, cogiéndola, aplastándola, susurrándole disculpas y palabras
dulces, diciéndole que todo iría bien, que estaba allí.

Pero un pensamiento se le ocurrió. Mañana, se habría vuelto a ir. Y con
el Doctor. Porque era lo que quería.

Pero, ¿qué derecho tenía a hacerlo? ¿Se había merecido el derecho a
volverse a ir de nuevo si era eso lo que pensaba su madre? Todas las
veces que ella y Sylvia habían peleado, discutido y gritado. Cuando era
adolescente (y francamente, una mayor parte de sus mimados veinte años),
Donna se consoló con un ``así es mi madre''.

Pero Donna ya no era esa persona, y podía ver que su viuda madre, año
tras año, necesitaba a su hija más que antes.

Y Donna también comenzó a llorar.

Lloraba por el dolor de su madre, por la pérdida de su padre, recordando
aquella llamada a la puerta. Con el policía allí de pie.

--- Se suponía que tenía que haber muerto aquí, en mis brazos, con su
familia --- decía Sylvia---. No en una maldita gasolinera.

Solo Y entonces, con el cronometraje exacto de cuando las palabras
``perfecto'' e ``inconveniente'' fueron inventadas, alguien llamó a la
puerta.

Sin decir una palabra, Wilf fue a responder, y Donna le escuchó decir:
--- Ah, no es un buen momento. Donna supo, sin escuchar la respuesta,
exactamente quién estaba en el umbral.

Y Sylvia también.

Miró con unos ojos rojos e hinchados a su hija.

Y, por primera vez que Donna pudiera recordar, Sylvia Noble acarició la
cara de Donna, con un dulce y suave movimiento de puro amor maternal.
--- Voy a poner la tetera --- y entonces le llamó---. Adelante, Doctor.
Un momento después, la cara del Doctor apareció cerca de la puerta de la
salita de estar, con las gafas de listillo puestas y el pelo más alocado
que nunca.

--- Hola --- les dijo a todos---. ¿No conoceréis la familia Carnes, por
casualidad? Creo que hay alienígenas en la familia.

~ ~ ~ ~\textsuperscript{\hyperref[cmnt1]{{[}a{]}}}

El comercio turístico en Moscatelli se basaba principalmente en olivos,
naranjos, uno bonito viñedo y una carrera anual de motos que empezaba a
trece millas de Florencia y acababa en el otro lado de las montañas
cerca de aquella pequeña ciudad poco visitada.

La gente que vivía en Moscatelli era en gran parte italianos que habían
estado allí durante treinta o más generaciones.

Todo el mundo conocía a todo el mundo y eran amistosos, acogedores y
alegres.

También era, a mediados de mayo, el recipiente de un increíble buen
tiempo, y Jayne Greene creía que sacaba lo mejor de la gente local. Era
por eso que Tonio se pasaba la mayor parte del día durante la excavación
trayendo nada más que un par de tejanos apretados y cortados que no
dejaban demasiado para la imaginación (y Jeyne se podía imaginar
bastante). El Profesor había contratado a Tonio y a su familia para
ayudar a organizar la excavación hacía una semana.

Jayne y sus dos colegas estudiantes, Sean y Ben, habían accedido a
acompañar al Profesor a pasar el verano porque les daría buenas notas en
sus asignaturas, y sería una aventura viajar a una hermosa parte de
Italia y sería genial tener una forma de ponerse morena.

--- ¡Lo tengo! --- gritó Sean, emocionado.

--- ¿Por cuánto? --- preguntó Ben, tamizando tierra a un par de metros
alejado de dónde descansaba el portátil cerca de la tienda de campaña
con la comida.

--- Por setenta y ocho euros. --- Sesenta y pico libras. No está mal.
--- Ben asintió ---. Muy bien hecho. --- Adoro muchísimo Ebay --- Sean
sonrió a Jayne ---. ¡Bien por mí! --- ¿Era la vasija egipcia? Sean la
miró y negó con la cabeza, lentamente.

--- ¿La espada de la Edad de Hierro, entonces? Sacudió la cabeza de
nuevo.

Jayne dejó caer sus herramientas y se acercó al portátil para ver en qué
se había gastado Sean las sesenta libras.

--- ¿Eso? --- Eso. --- Es un juguete. --- Por supuesto que es un juguete
--- gritó Ben mientras Tonio dejaba caer más tierra en su filtro---. ¿Es
que hay algo más que compre Ben por eBay? Jayne no podía creérselo. ---
¿Quieres decir que te has gastado todo ese dinero, te pasas siete días
frustrado mirando la subasta por un juguete fabricado en serie?? --- Una
figura de acción --- la corrigió Sean ---. De edición limitada.

Sólo han hecho quinientos, y de eso hace ocho años.

\textsuperscript{\hyperref[cmnt2]{{[}b{]}}}

Es un trabajo de pintura distinto, ¿lo ves? La chica lleva su traje rojo
de la Edad oscura en lugar del verde tradicional. Jayne observó a Sean.
--- Eres un un adulto. Un hombre hecho y derecho emocionándose con un
juguete de plástico. Una figura para los niños. Una\ldots{}~ --- No
digas muñequita --- murmuró Ben para sí mismo.

---\ldots{}¿una muñeca? Sam cerró de un golpe la pantalla del portátil.
--- Mi dinero, mi elección.

Tú te emocionas con alhajas romanas y cerámicas llenas de tierra. --- ¡Y
tú también! --- Sí, porque es mi trabajo. Es lo que hago aquí y en la
universidad.

Pero en mi tiempo libre, tengo otros hobbies. Tengo\ldots{} --- No digas
``una vida'' --- se dijo Ben a sí mismo de nuevo.

--- \ldots{} una vida --- acabó Sean ---. Deberías intentar encontrarte
tú una antes de criticar a los demás. Jayne observó a Sean, entonces a
Ben, que se aseguraba de no mirar a nadie a los ojos y comenzó a mover
el dedo sin ningún sentido entre el polvo de la tierra, como si
intentara hacer cómo si estuviera distraído con alguna cosa.

La tensión se rompió por el pequeño profesor Rossi, saliendo de entre
las tiendas viniendo de un viaje a la ciudad yendo a buscar leche y
bolsas del té.

--- Bien, bien, se os podía escuchar desde la carretera.

¿Qué está pasando? --- Nada --- gruñó Sean. ---. Lo siento, profesor.
Rossi negó con la cabeza, rascándose la cicatriz que hacía una pequeña
marca en su mejilla. En la universidad, todo el mundo hacía broma
diciendo que era una cicatriz de guerra que había conseguido peleándose
con una mujer a la que amaba, pero un día descubrió la verdad: hacía
diez años que se había cortado en un accidente de coche en el que había
muerto su mujer. Todo el mundo perdió el interés imaginando cosas
románticas después de aquello.

--- ¿Qué voy a hacer con vosotros tres? Os traigo aquí de la universidad
durante las vacaciones de mediados de trimestre, para visitar a la
familia, y os doy la oportunidad de mejorar vuestras francamente pobres
notas de arqueología. Y todo lo que hacéis es jugar con la banda ancha,
flirtear con el pobre Tonio o avergonzarle, o beber demasiado vino de
naranja. Estáis aquí para trabajar, ¿lo sabéis, verdad? Ser sociable es
un agradable efecto secundario pero no el esencial. Lo que es esencial,
de cualquier forma, es el trabajo en equipo. Sean y Jayne no me importa
si no os podéis aguantar, pero trabajaréis juntos. Ben y Jayne, no me
importa si lucháis entre vosotros por la atención de Tonio, pero
trabajaréis juntos. Sean y Ben no me importa si os bebéis el uno al otro
bajo las mesas mientras el día siguiente estéis frescos y capaces de
trabajar juntos. ¿Ha quedado todo entendido? No soy vuestros padres pero
soy el hombre que os pondrá nota al final del trimestre, y haríais bien
recordando que mantenerme contento es positivo --- Rossi puso unos
cartones de luz sobre la mesa cerca del portátil --- . Así que, ¿a quién
le toca hacer el té? Sean se ofreció voluntario mientras Rossi cogía el
portátil.

--- Con suerte, el Bursar nos habrá dado más fondos para que podamos y
miremos en aquellos túneles entre las montañas al otro lado del lago.
--- ¿Cuánto hace que tiene familia aquí, Profesor? --- preguntó Jayne.

Rossi se encogió de hombros. --- Estoy en proceso de descubrirlo en la
biblioteca. De hecho mis bisabuelos paternos fueron los que se fueron a
Ipswich, pero sospecho que mis raíces están aquí des del siglo XV. Ben
se acercó con su colador. --- ¡Así que entonces estamos buscando más que
vasos y platos italianos del siglo XV! ¡Ya os lo dije! Venga, Profesor,
¿cuál es el gran secreto? --- Ah --- Rossi sonrió ---. Bueno, verás, en
algún lugar de esta zona un ducado entero desapareció. Toda una ciudad
con un castillo y todo tenía su base aquí alrededor, o en las colinas o
algún lugar en las orillas del ese lago detrás de los cultivos de
naranjos. Intento encontrar sus límites. --- ¿Cómo se sabe? --- preguntó
Sean mientras la tetera hervía.

--- Está en los registros de la biblioteca --- dijo Tonio con un buen
inglés pero con un fuerte acento.

Jayne y Ben miraron a Tonio en estado de shock y con un ligero horror.

Éste sonrió. --- Ah, sí, los dos pensasteis que no entendía inglés ---
rió, con una profunda y grave risa.

Y también lo hizo el Profesor Rossi. --- Eso sí ---dijo---, es
divertido.

¿Ninguno de los dos se ha dado cuenta? Sin decir nada, ambos negaron con
la cabeza mientras Sean se ocupaba del té, decidido a cruzar sus
miradas.

--- Pero eso significa que\ldots{}--- comenzó Jayne.

--- Todo lo que hemos dicho\ldots{} --- añadió Ben.

--- Sobre ti\ldots{} ---dijo Jayne de nuevo.

--- Y tú\ldots{} has oído\ldots{} oh, Dios\ldots{} matadme ahora
mismo\ldots{} --- Ben dejó caer su colador y se sentó en el suelo.

Tonio despeinó el oscuro cabello de Ben y le guiñó un ojo, antes de
lanzarle una mirada a Jayne. --- Lo siento, tú pierdes. --- Por supuesto
que sí --- dijo Jayne ---. ¿Cuándo la vida irá al estilo de Jayne
Greene? El portátil pitó y, dejando a los estudiantes con el té y la
confesión de Tonio, el profesor Rossi accedió a su bandeja de entrada de
los correos electrónicos. Nada de Bursar, pero había un mensaje:

~

De: Madam Delphi

Para: \href{mailto:Rossi@Tarminsteruni.ac.uk}{Rossi@Tarminsteruni.ac.uk}

Asunto: SAN MARTINO~

~

Profesor Rossi, Mis felicitaciones, ha redescubierto su patrimonio, y
que es, en efecto, de San Martino, tal como esperaba. Haga clic en este
enlace para ir a mi web para obtener más información acerca de este
encantador reino italiano y sus secretos.

~ \textsuperscript{\hyperref[cmnt3]{{[}c{]}}}

El profesor estuvo a punto de llamar para que se acercaran sus
estudiantes, pero creyó que sería mejor comprobar que no fuera un bulo.

(Aunque ¿cómo iba a saber nadie que estaban buscando San Martino? Ni
siquiera le había dicho a los estudiantes el nombre del ducado). Así que
hizo click en el enlace.

En lugar de una nueva página web, la pantalla se llenó al instante con
una bola pulsante de una brillante luz blanca, resaltada con bordes de
color lila y espirales.

Instintivamente dejó que su mano se acercara para tocar la
pantalla\ldots{} para entrar en la pantalla, para ir a través de la
pantalla\ldots{} como si su mano derecha estuviera siendo consumida por
la bola de energía.

Luego retiró la mano, y la miró.

Crujiendo alrededor de las puntas de los dedos había vestigios de unos
morados pulsos de energía, como unos diminutos destellos de energía
primaria. Giró la mano, estudiando los pequeños impulsos hasta que se
desvanecieron para siempre, absorbidos en su piel. Se frotó los dedos, y
luego volvió a mirar a la pantalla. Simplemente mostraba la victoria en
eBay de Sean otra vez.

El profesor se puso de pie, se volvió hacia sus alumnos y extendió sus
brazos, con las palmas de las manos hacia afuera. --- Lo hemos hecho ---
suspiró.

Distraídos al instante de sus propias y pequeñas preocupaciones, los
cuatro jóvenes se acercaron, Jayne y Sean cogiéndole cada uno la mano
tendida, devolviendo el gesto con entusiasmo, sin estar seguros de lo
que estaban celebrando.

Después de un segundo, sin decir palabra soltaron las manos manos al
Profesor, y Rossi entonces agarraron las manos de Ben y Tonio. Y ellos,
a su vez tomaron las Sean y Jayne, los cinco ahora formando un círculo.

Al unísono, todos levantaron sus manos unidas al aire, haciendo una
crepitante electricidad púrpura que les rodeaba.

Los demás siguieron la dirección de su mirada cuando el Profesor miró
hacia el cielo.

--- Bienvenida de nuevo --- dijo en voz baja.

La cena estaba servida en el hogar de los Noble.

Sylvia silenciosamente ponía comida en los platos. En silencio, Donna
pasaba los platos de la encimera a la mesa. En silencio, Wilf ponía agua
en los vasos: tres a conjunto de una gasolinera y uno con la cara del
pato Donald. El Doctor tenía ese último.

El Doctor estaba allí sentado, incómodo con la domesticidad en su mejor
momento, encontrándose peor por momentos.

--- ¿Dubai? --- dijo Sylvia de repente mientras se sentaba.

El Doctor lanzó una mirada a Donna: ¿qué era lo que se suponía que tenía
que decir? --- Con los caballos --- añadió amablemente Wilf.

--- ¿Caballos? --- el Doctor era como un conejo atrapado entre dos focos
---. Caballos. Sí, seres maravillosos. --- El jeque de Dubai nos retrasó
un par de semanas --- intervino Donna ---, ¿no fue así? Sylvia empezó a
comer Algo que se parecía a macarrones con queso, supuso el Doctor, pero
no estaba seguro. Algo parecido a eso una vez le intentó morder los
dedos de los pies en la costa de Kal-Durunt, en la abadía de Kerípedes.

Introdujo su tenedor en ello suavemente.

--- Siento que no sea tan elegante como lo que os darían en Dubai, con
caballos y jeques --- dijo Sylvia ---. Pero no me habían avisado de que
ninguno de los dos ibais a venir. --- Oh, bueno, no podía ser que Donna
se perdiera el día de hoy---dijo el Doctor alegremente. Demasiado
alegremente. Una ocasión mala para ser el Doctor Sonrisas, mejor ser el
Doctor Ojitos aquella noche.

--- Pensé que los Emiratos estaban en manos de los emires y no de jeques
--- dijo Sylvia, sirviéndose más agua ---. Pero, ¿qué sabré yo? Me voy a
sentar aquí cada día, esperando a que la gente aparezca un buen día,
esperando a ser alimentada El Doctor acababa de lanzar una mirada a
Donna que creyó que decía ``Ayúdame'' pero que Donna tomó claramente en
sentido de ``No, no pasa nada, no me hagas caso , oh y ahora sería el
mejor momento para una buena pelea con tu madre''.

Y eso hizo Donna.

--- ¿Cuál es tu problema, mamá? La mayoría de las personas matarían por
tener familia a su alrededor Wilf intentó intervenir, pero Donna ya
estaba hablando.

--- Quiero decir, Mooky se fue durante dos semanas, sus padres montaron
una maldita fiesta para celebrar su regreso. Y lo único que hizo fue
irse de compras a Glasgow. Consigo poder ver la gala\ldots{} bueno, ver
el mundo, cosas que nunca creí que podría tener la oportunidad de ver, y
todo lo que consigo son quejas. Sylvia no levantó la vista de su comida.
--- Sí, pero probablemente sus padres sabían dónde estaba Mooky. Todo lo
que yo sé es cuando tu abuelo aquí se molesta en decirme que tiene una
postal. Y nunca se me permite leerlas, oh no. Donna iba a castigar a
Wilf por ello cuando se acordó que tales postales las solía mandar desde
otro sistema solar completamente diferente.

--- Vale, mamá, también voy a empezar a enviarte las postales.

Prometido. --- Oh, no es sólo eso --- dijo Sylvia ---. Es toda la vida
que tengo. Tu padre se fue, tú te has ido, y yo me quedo aquí de niñera
de la novia de tu abuelo. Donna abrió la boca para decir algo, pero la
cerró de nuevo.

Luego, a medida que fue encajando el comentario, volvió a abrir su boca
de nuevo, pero no salió ningún sonido.

--- ¿Novia? --- preguntó el Doctor a Wilf.

Wilf fulminó con la mirada a Sylvia. --- Es una amiga ---dijo ---. No me
voy a casar con ella. --- Eso espero --- dijo Sylvia---. Mamá se
revolvería en su tumba. --- Ahhh, así que eso es de lo que se trata todo
esto --- suspiró Wilf ---. Crees que Eileen no lo aprobaría. Crees que
verme con una pobre y enferma anciana entristecería a Eileen. Bueno,
pues te equivocas. Ella era tu madre, pero fue mi esposa.

La conocía mejor. El Doctor recordó por qué no le gustaban las familias.

--- Los macarrones con queso están buenísimos, señora Noble --- dijo,
llenándose la boca---. Mmmmm\ldots{} --- Es raclette de setas --- le
espetó.

--- ¿No son macarrones? --- Setas. --- Es\ldots{} genial\ldots{} tiene
mucho queso. Y\ldots{} --- Entonces, ¿quién es esa señora, abuelo? ---
preguntó Donna.

Wilf sonrió. --- Es una astrónoma que conozco, de Greenwich. Ayuda el
observatorio de allí, lo ha hecho durante años. Pero hace unos tres
años, ella\ldots{} bueno, ella enfermó y tuvo que dejar de trabajar.
Hablamos por teléfono un par de veces, nos conocimos y cenamos.
Pensarías que he comenzado a salir con una prima adolescente casada y
embarazada por la manera que Sylvia habla de ella. Sin embarg, el Doctor
estaba mirando a Sylvia Noble.

Viendo qué le hizo enrojecer de furia cuando Wilf habló. Había sido la
palabra ``enfermó''.

Miró de nuevo al viejo paracaidista. --- ¿Entonces por qué dejó el
Observatorio? --- Pregúntaselo tú mismo --- dijo Sylvia ---. Va a estar
aquí en cualquier momento. Incluso en el día de Geoff, mi hija te trae y
él la trae a ella --- y Sylvia se levantó y salió de la cocina.

Donna suspiró y fue detrás de su madre. Wilf estuvo a punto de
seguirlas, pero el Doctor le cogió del brazo.

--- No soy un experto, Wilfred, pero supongo que mejor dejar en paz a
las damas --- Wilf asintió.

--- ¿Y tu amiga? --- Netty. Henrietta Goodhart ---Sonrió---. El nombre
más apropiado que creo que podría tener. Pero le diagnosticaron\ldots{}
Ella tiene Alzheimer, Doctor. Y no mejora. --- Y no lo hará --- dijo el
Doctor en voz baja, justo cuando sonó el timbre. --- ¿Es ella? --- Wilf
asintió y fue a hacerla entrar.

Instantes después, el Doctor estaba sonriendo ante la visión de la
excentricidad, encanto y humor que solo ciertas mujeres inglesas de una
particular edad y porte podrían mostrar.

Estaba vestida de pies a cabeza con una falda de pana marrón que llegaba
a la rodilla, una blusa canela, una chaqueta color chocolate y un bolso
de mano canela. En su cabeza había un impresionante sombrero con al
menos media docena de plumas marrones de distintos tamaños y formas.
Wilf le estaba quitando su oscuro largo abrigo y Netty le ofreció su
mano al Doctor antes de que Wilf hubiera tenido la oportunidad de
quitarle el abrigo, eso significaba que una manga, con la que se hacían
los apretones de manos, seguía puesta.

---Doctor, qué maravilloso es el conocerte. Viva y hurra, es un
verdadero placer.

---Señora Goodhart.

---Señorita, por favor. Mejor aún, sólo Netty. Nunca he estado casada y,
a pesar de lo que la hija de Wilfred cree, no tengo intención de casarme
nunca.

Wilf finalmente le quitó el abrigo, y Netty se deslizó hacia una silla,
tomando un vaso de agua en una maniobra obviamente bien ensayada.

---Nunca me he casado ---continuó---. Parecía un tan alarmante
desperdicio de tiempo. Vivo en Greenwich, sabes, a un poco más de una
caminata, pero mi empresa local de taxis, me conocen, así que nunca es
un problema si me olvido llevar dinero. O dónde voy.

Al Doctor le había caído bien esta sociable mujer instantáneamente.

---No se puede batir a una empresa de taxis fiable ---dijo, consiguiendo
pronunciar una frase entera antes de que continuara alegremente.

---¿Ya están discutiendo? Es todo lo que hacen. Y es sobre mi. Estaba
tan avergonzada al inicio, ahora me lo tomo como un ritual, y en diez
minutos Sylvia volverá a ser igual de encantadora que siempre, llena de
té y galletas ---el Doctor sonrió---. ¿Sylvia Noble? Encantadora.
Palabras que no a menudo van en la misma frase ---la mirada que recibió
de Netty le mostró cómo había malinterpretado la situación entre las dos
mujeres.

---Oh, no me decepciones, Doctor. No después de todo lo que me han
contado sobre ti. Esa mujer de ahí es una santa.

Acaba de perder a su marido, tiene una hija que arrastras a medio camino
a Dios sabe dónde por capricho, y ha tolerado al maravilloso Wilfred,
que puede ser tan terco y gruñón como ella. Más aún, de hecho. Me gusta
mucho. Además, sé que se queja, pero esa es forma de desahogarse, es
fantástica con mi\ldots{} ya sabes\ldots{} ---Netty golpeó el lateral de
su cabeza---. Mi estado. Bendita sea, la semana pasado llevo a Wilf
hasta Charlton. Al parecer, me encontraron en el jardín de alguien,
¡tratando de convencerlos de que solía vivir allí cuando tenía seis
años!

---¿Lo hacías?

---¡Dios mío, no! Me crié en Hampshire ---se zambulló en su bolso y
sacó un cuaderno rojo A5 y se lo enseñó---. Mi vida ---dijo
simplemente---. Así puedo recordar cosas ---el Doctor la miró fijamente
a los ojos y vio, brevemente, una mujer muy anciana y muy orgullosa. Y
le gustó incluso más que antes.

---Sin ese libro, sin la gente como Sylvia Noble, no soy nada. Me
dejaría el bolso en una tienda en la calle principal de Greenwich, así
que estoy en este jardín, sin saber quien soy o de dónde soy. Sylvia
encontró un recibo en mi bolsillo, encontró la tienda, recuperé el bolso
de donde me lo dejé, todo apañado con la policía. Quiere que me meta en
una residencia, sabes. Los panfletos están en ese cajón al lado de la
estufa.

---¿En serio?

---Sí. Wilf no quiere saber nada. Dice que antes hará que me mude aquí.
Viejo tonto, como si me fuera a ir de una casa con la que apenas puedo
lidiar a otra. Pero una encantadora residencia de ancianos, ¿dónde
estaré cuidada? ¿Qué maravilloso es eso? ---la puerta de la cocina se
abrió. Donna y Sylvia entraron, y Donna inmediatamente se presentó.

Mientras Sylvia ponía la tetera, el Doctor se puso detrás de ella.

---¿Sabe Wilf todo lo que haces por su amiga?

---¿Sabe Donna que estás metiendo tus narices en los asuntos de su
familia? ---respondió Sylvia.

---No soy tu enemigo, Sra Noble ---dijo el Doctor.

Sylvia se giró y le sonrió. La sonrisa más poco sincera posible.

---Por el amor de mi hija, Doctor, te tolero en esta casa. Pero eso es
todo. Por el amor de papá, lo haré lo mejor que pueda con Netty
Goodhart. No creo que sea una mujer egoísta, Doctor. He trabajado duro,
he construido una vida, nunca he tenido mucho dinero, y he tratado de
dar a Donna una vida digna. Pero entonces, un día, perdí a mi marido. Mi
roca. Y desde entonces he tratado de hacer lo que ambos hacíamos, pero
con una hija que un minuto no consigue un trabajo, y el siguiente puede
darse el lujo de estar en un hemisferio diferente, pero no puede
permitirse un sello, y un anciano padre que parece haber decidido que es
hora de una vez por todas de remplazar a mamá.

---¿Estás segura de que no estás preocupada por que te está
reemplazando? Me imagino que dejó ir a tu madre hace mucho tiempo ---el
silencio que siguió a la bofetada en la cara que recibió pareció
continuar durante horas, pero probablemente fueron solo unos cuantos
segundos.

---Yo no quería decirlo así\ldots{} ---comenzó el Doctor---.
Sinceramente me preguntaba\ldots{}

Sylvia le ignoró.

---Papá ---dijo---. ¿Por qué no te llevas al Doctor a la parcela?, ¿eh?
Donna y yo podemos ponernos al día con Netty mientras vosotros estáis
una hora o\ldots{} unas cuantas allí arriba, ¿sí? ---Wilf captó la
indirecta y estaba casi arrastrando al Doctor fuera de la cocina
mientras las mujeres observaban. Lo último que el Doctor oyó fue ``Té,
para todos'' de Sylvia, antes de que Wilf les sacara ambos a la noche.

---Parcela. Por aquí ---dijo el anciano hombre.

Babis Takis lanzó el mayor fardo de heno en la parte trasera de la
furgoneta y paró a descansar. No estaba rejuveneciendo y esto era el
trabajo de Nikos. Pero Nikos no estaba allí, probablemente estaba en la
parte trasera de la granja, liándose con esa chica Spiros. Típico. Había
trabajo que hacer, tenían que conseguir este envío para Faliraki, donde
aun había mucha construcción. Hoteles, apartamentos, centros
comerciales, todo de lo que el Dodecaneso se beneficiaría por el
continuo turismo que traería. Babis gritó el nombre de Nikos un par de
veces mientras izaba más fardos de heno. Echó un vistazo a su reloj.
Costaría una hora más o menos conducir por la isla hasta Petaloudes,
donde recogerían a Kris, antes de pasar por Lindos y a lo largo de la
costa a Faliraki. Harían la entrega de heno en el almacén y luego a la
Taberna de Erik para el resto de la noche.

Aún no había señales de Nikos. Con un suspiro, Babis se alejó de la
camioneta.

---Soy un hombre mayor, Nikos ---llamó---. Luché en la guerra, sabes,
para que gente como tú pueda ser independiente y tener vuestros lujos.
Por una vez, estaría bien que pudieras trabajar como el resto ---había
alcanzado la parte trasera de la granja, cuando escuchó un ruido del
interior de uno de los establos, un corto suspiro femenino de sorpresa,
casi miedo. Ciertamente de alarma. Babis en un segundo estuvo dentro del
establo.

Nikos estaba en el suelo, sosteniendo su cabeza entre las manos,
silencioso pero claramente aturdido. De pie sobre él, con una pala en
sus manos había una chica muy joven, que Babis reconocido como Katarina
Spiros.

---¿Estás bien, chica? ---preguntó Babis, alcanzando la pala.

Katarina se dio la vuelta para afrontar al abuelo de Nikos, y se dio
cuenta de lo asustada que parecía.

---¿Qué ha pasado? ---preguntó Babis.

Se sorprendió cuando Katarina se explicó.

---Recibió una llamada\ldots{} ---estaba señalando a Nikos Takis, quien
ahora estaba empezando a levantarse, abandonando el teléfono en el suelo
junto a su pie. Miró a su abuelo, haciendo que Babis involuntariamente
retrocediera.

Babis Takis de joven había combatido en los últimos días de la guerra,
echando a los nazis de Creta y manteniendo a Grecia para los griegos. Se
había enfrentado a la ira y al dolor de sus padres, quienes le habían
odiado y al final le habían rechazado por enamorarse, casarse y tener
hijos con una de los odiados italianos que habían ocupado las Islas
Griegas durante 700 años antes de ser enviados a hacer las maletas
después de la guerra. Había pasado algo de tiempo en prisión por una
pelea de bar en Diagoras, y una vez tuvo que enfrentarse al amargado
hijo de un alemán al que había atado a una granada en 1944.

Pero nunca nada lo había asustado tanto como la mirada que su nieto le
echaba.

Nikos ya no estaba allí. Ese despreocupado, divertido, inteligente nieto
que había criado después de que su padre muriera. Babis no sabía cómo lo
sabía. No sabía cómo había ocurrido. Pero nunca había estado tan seguro
de nada en su vida.

Todavía estaba seguro de esto cuando un destello de fuego púrpura, más
caliente que el corazón de un sol, extinguió su existencia en menos
tiempo del que le tomó a Katarina Spiros coger aire para gritar. Un
segundo después, un puñado de cenizas cayeron al suelo, donde la joven
había estado. Y Nikos Takis alzó los brazos hacia el cielo, con un
zumbido eléctrico púrpura alrededor de sus dedos, mientras echaba la
cabeza hacia atrás.

---Bienvenido ---gritó triunfante.

Donnie y Portia estaban de luna de miel. Era la primera de Donnie, la
segunda de Portia, pero ambos se estaban divirtiendo enormemente. El
hijo de Donnie había sido su padrino. Su nieto había sido el paje. La
nieta de Portia había sido la dama de honor. Habían realizado dos
ceremonias, una judía y una cristiana más simple para reflejar ambas
creencias elegidas. Portia siempre había abrazado la fe judía, mientras
que la familia de Donnie la había abandonado bastante pocas semanas
después de llegar a las islas Ellis hacía poco más de un siglo.

Habían superado las circunstancias, el cáncer de Donnie, algunos severos
fruncimientos de ceño de los parientes más tradicionales de Portia y la
muerte de su gato de ocho años, el señor Smokey, una semana antes de las
ceremonias. Después de conocerse desde hacia quince años, saliendo los
últimos seis, finalmente estarían juntos para siempre.

Y ahí estaban, en el jeep de Donnie, yendo a toda marcha por la Octava,
pasando por Danbury, antes de dejar la autopista y hacia la campiña de
Connecticut para su luna de miel.

Habían alquilado una casa colonial muy bonita en la afueras de
Olivertown, a 20 kilómetros de Danbury. La casa pertenecía a uno de los
clientes de Portia (era una paseadora de perros, paseando tres veces al
día por Central Park con una variedad de canes). Los Carpenters estaban
en FM, diciendo al mundo como les gustaría enseñar al mundo a cantar, y
la feliz pareja estaba cantando.

Habían pasado por Abba, el Dr. Hook y Medicine Show, Jo Stafford, y
asentían suavemente con Helen Reddy cantando sobre lo bueno que era
estar loco: ``Nadie pide que te expliques'' cantaron al unísono cuando
se detuvieron en la casa que habían alquilado. Portia miró a su nuevo
marido.

---Bueno, señor DiCotta, aquí estamos.

---Lo estamos, señora DiCotta ---Donnie le guiñó un ojo.

---¿Todavía no te has acostumbrado?

---Creo que nunca lo haré ---se rió---. Pero me gusta ---se inclinó
sobre el coche y besó a Donnie mientras lo apagaba. Y la radio siguió
sonando. Al separarse, ambos miraron la carcasa del estéreo.

---Eso no es bueno, Donnie ---dijo Portia DiCotta---. Debe de haber un
cortocircuito en algún lado.

---Mierda, será mejor que lo arregle ahora, cariño. De lo contrario no
tendremos batería para mañana, cosa que no sería bueno si quieres que te
lleve a ese restaurante en New Preston. La comida es magnifica, el
servicio de primera clase y las vistas son mortales. Puedes ver sobre el
Lago Waramaug y es verdaderamente romántico ---Portia asintió.

---Tu arregla el coche, haré café ---Donnie se acercó para buscar bajo
el salpicadero un cable suelto. Hubo una pequeña chispa de electricidad
y la radio se silenció.

---Bien hecho ---sonrió Portia---. Ahora puedes ayudarme a sacar las
maletas del maletero. Donnie DiCotta no dijo nada. Matuvo su mano bajo
el salpicadero del jeep, mirando al frente.

---¿Donnie? ---nada.

Portia le tocó el hombro, y él giró la cabeza para mirarla. Portia vio
sus ojos, no los hermosos ojos azules de los que se había enamorado.
Estos habían sido reemplazados por dos sólidos orbes de ardiente luz
violeta, minúsculas lenguas de electricidad chispeando por sus conductos
lacrimales. No puedo decir nada porque él le cogió la cabeza y la beso
en la boca. Abundante. Duro. Pero para nada apasionado. Después de uno o
dos segundos, se separaron. Y ahora los ojos de Portia DiCotta ardían
con la misma escalofriante energía purpura. Sin mediar palabra, bajaron
del jeep y caminaron hacia el porche, estudiando el cielo nocturno sobre
ellos, hasta que Donnie señaló a la derecha, hacia una estrella
brillante, si hubiera sido un experto en tales cosas hubiera sabido que
no había sido vista por ojos humanos en muchos siglos. Él y su nueva
esposa se cogieron de las manos y miraron a la estrella.

---Bienvenido ---respiraron juntos.

El Doctor estaba mirando Londres.

---Puedo ver porque te gusta este sitio, Wilf ---dijo al anciano
quejándose tras él, preparando un segundo asiento para que se
sentara---. Es terriblemente\ldots{} pacífico ---Wilf Mott asintió.
---Llevo años viniendo aquí. Solía mirar al cielo por la noche cuando
estaba en el ejército. Solía navegar por las estrellas al igual que por
los mapas y esas cosas. Los otros chicos pensaban que estaba loco, pero
¿sabes qué, Doctor?, nunca nos perdimos. Ni una sola vez. El Doctor
sonrió al anciano y se sentó en el asiento ofrecido.

---Gracias ---Wilf se sentó junto a él y le sirvió una taza de té del
termo. El Doctor bebió un sorbo agradecido

---Solía meter un poco de lo mejor del señor Daniels ---dijo Wilf---.
Pero los malditos doctores se lo dijeron a mi hija y tuve que dejarlo.

---Una panda terrible los doctores ---rió el Doctor---. Pero a menudo
saben lo que es mejor, por impopular que les haga.

---Sutil ---asintió Wilf.

---Sylv cambiará de opinión. Probablemente.

---¿De verdad?

---No ---y Wilf rugió---. Ni en sueños, amigo ---el Doctor volvió a
sonreír.

---¿Entonces no tienes un problema conmigo?

---Oh, tu haces feliz a mi Donna. Sigue así y ya estás bien conmigo. Sin
embargo, haz cualquier cosa que la moleste, y sabrás de mi, incluso en
Marte ---el Doctor parecía sorprendido ante esto.

---¿Cuánto te ha contado?

---Todo. Desde el inicio ---Wilf señaló a su telescopio---. Estaba aquí,
mirando al cielo cuando me habló de ti. Al principio no la creí.

---Bueno, nadie podría culparte de ello.

---La verdad es que creo que no entendí lo que realmente me estaba
diciendo. Entonces os vi después de ese asunto con la grasa, volando por
el cielo, Donna saludándome, y me di cuenta de que todo era cierto. Me
mantiene al día cuando puede. Postales, correos. El regalo extraño.
Todavía no sé qué hacer con la medalla de Verron. ¡O exactamente lo que
es una medalla de Verron! ---el Doctor sonrió.

---Maravillosa raza, los Verrons. Tienen un cuerpo aéreo brillante.
Completamente inútil, no han librado una guerra en unos pocos milenios,
pero su cuerpo aéreo es su mayor logro. Es un poco como si le enviaras a
alguien una DSO de tres barras ---el Doctor se encogió de hombros---.
Espera, ¿cuándo te ha enviado ella eso?, ¿de dónde lo sacó?, ¿y cómo
diablos te lo hizo llegar? ---Wilf casi retrocedió ante la andanada de
preguntas del Doctor.

---Ni idea. No necesitas que te lo devuelva, ¿verdad? Quiero decir, ¿no
fue robado? Donna no ha robado nada desde esos dulces en Woolies cuando
tenía ocho años. Le hicimos devolverlos y pedir perdón y de todo.

---No, no, no creo que lo haya robado. Los Verrons son muy generosos.
Sólo quiero saber cuando conoció a un Verron.

---Doctor, no me gustaría pensar que mi pequeña no estaba siendo cuidada
adecuadamente ---dijo Wilf alzando una ceja.

---Helios 5 ---dijo el Doctor---. Tuvo que ser ahí. O Ylum. Me gusta
Ylum, y a Donna también, muy cosmopolita. O quizás de Moulin Très Rouge.
Allí había un mercado. O supongo que fue ese día en\ldots{}~

---De todos modos ---Wilf le cortó--- hemos venido aquí por una razón.

---¿Otra razón aparte de querer comprobar mis intenciones con tu nieta?
Y para darle a mi mejilla izquierda la oportunidad de enfriarse ---Wilf
rió.

---Oh, ese es el método de Sylvia, no el mío. Sé que eres honorable.
También sé que Donna puede cuidar de sí misma en ese aspecto. Cualquier
deshonorabilidad de tu parte y nunca escucharás el final del mismo.
Literalmente ---el Doctor pensó en ``¡Ey!'' y decidió que sí, Wilf
ciertamente conocía muy bien a su nieta.

Wilf ajustó su telescopio.

---Mira.

El Doctor lo hizo, y vio una estrella. Un diminuto punto de luz,
parpadeando lo suficiente y que a menudo parecía desvanecerse.

---¿Te gusta? 7432MOTT ---dijo Wilf orgulloso.

---Disculpa ---dijo el Doctor, retrocediendo y mirándolo de nuevo---.
¿La nombraron por ti?

---Yo la descubrí. Me uní a un nódulo. Poco después, descubrí esa
estrella. La RPS hará una cena en mi honor mañana por la noche.

---Nodo adintiendo con los nódulos ---dijo el Doctor.

---Ah, bueno, un nódulo es ---comenzó Wilf, pero el Doctor hizo un gesto
a un lado.

---Estoy bromeando. Sé lo que es un nódulo. Y estoy totalmente
impresionado, Wilf, que haya una estrella nombrada en tu honor y estoy
aún más contento de que la Real Sociedad Planetaria te organice una
fiesta. Y apuesto a que has comprado un corbata nueva y de todo. Y
siento mucho tener que aguarte la fiesta, pero puede que haya una nueva
estrella, justo ahí abajo. Es increíblemente brillante, justo a la
izquierda de la espada de Orión.

Wilf se inclinó y apartó al Doctor.

---¿Ahí?

---No, aquí.

---Oh, ahí. Sí, a todos también nos gusta esa. Es muy bonita.

---Sí, muy bonita. Una nueva estrella muy bonita brillando intensamente
en una constelación que no debería existir. La cosa es que las estrellas
de esa magnitud y brillantez simplemente no aparecen sin una buena
razón.

---¿Brillantez? ¿Es ese un término técnico?

El Doctor le echó una mirada a Wilf.

---Es lo suficientemente bueno para mí.

---Sólo sé que todo el mundo que ha estado hablando de ella
aparentemente llama a estrellas como estas Cuerpos de Caos ---el Doctor
pensó por un segundo o dos, luego se encogió de hombros.

---¿Lo hacen? Eso es nuevo para mi.

---Es como lo llaman en los periódicos ---el Doctor asintió.

---Ah, bueno, si los periódicos lo dicen debe de ser cierto, porque
¿quien podría discutir con los tabloides? ---miró a la cosa en el
cielo---. Cuerpos del Caos. Buen nombre descriptivo, eso lo han acertado
---el Doctor pasó un brazo sobre el hombro de Wilf---.

¿Pero sabes qué?, ¿a quién le importa? Tienes una estrella, una
estrella mucho menos preocupante, nombrada en tu honor y estoy muy
orgulloso.

---Gracias. Me alegro de que hayas dicho eso porque me acabas de
resolver un problema. ---¿Qué problema?

~---Necesito a alguien que me lleve a Vauxhall.

---¿Por qué?

---La cena.

---¿Con quién?

---La Sociedad.

---¿Cuando?

---Mañana por la noche.

---¿Cómo?

---¿Con la TARDIS?

---Sí, porque eso va a pasar.

---Vale, de acuerdo, entonces iré en el metro. Sylvia cree que no soy
capaz de ir solo. Quiero decir, estoy bien sentado en una fría y húmeda
parcela cada noche, que causa estragos en mi\ldots{}

---¿Así que Sylvia no quiere que vayas solo bien entrada la noche,
verdad?

---Quiero decir, no voy a perderme, Doctor, pero está siendo muy
protectora y tonta últimamente. Desde que perdió a Geoff. Y con Donna
tanto tiempo fuera. Y ahora Netty\ldots{} ---¿Sabes qué? Wilfred Mott
me encantaría acompañarte a cenar en Vauxhall. En taxi ¿Qué pijo es
eso? No he tenido una comida en el RPS desde que Bernard y Paula me
llevaron en 1969 para ver el alunizaje. ¿Todavía hacen ese pudín con
chocolate?

---No lo se. No he estado antes.

---Entonces, mañana por la noche, Wilfred Mott de Chiswick, asumiendo
que el menú no ha cambiado en cuarenta años, y siendo la Real Sociedad
Planetaria creo que estoy en terreno seguro, estás apuntado a una
delicia culinaria ---el Doctor miró por última vez a la preocupantemente
brillante nueva estrella que no era 7432MOTT por el telescopio---.
Ciertamente un Cuerpo de Caos.

---Sin embargo es bonita ---murmuró Wilf.

Y ambos se estremecieron repentinamente. Como si alguien hubiera pasado
por encima de sus tumbas. O la tumba de todo el planeta.

---Hermoso Caos ---dijo el Doctor en voz baja.
