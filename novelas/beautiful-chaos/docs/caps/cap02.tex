\chapter*{Viernes}
\addcontentsline{toc}{chapter}{Viernes}

Terry Lockworth comprobó su teléfono móvil, pero todavía no había
cobertura. María iba a estar tan harta de él, pues había tenido que
trabajar hasta tarde, pero no podía hacérselo saber. Sin duda el bol de
espaguetis acabaría en la papelera esa noche. De nuevo. Pobre María, no
era culpa suya que terminara harta de él pero, ¿qué se suponía que tenía
que hacer? Habían estado casados durante tres meses, tenían un hijo
en camino (por favor, que sea una niña), e iban escasos de dinero.

Sí, claro, su padre les había dado un depósito para el apartamento de
Boston Manor, pero aún quedaba la hipoteca, las facturas, las clases
pre-natales, la comida\ldots{} Terry negó con la cabeza mientras se
guardaba el teléfono en el bolsillo. Deja de quejarte, se dijo, y sigue
adelante con el trabajo, entonces estarás en casa en una hora con un
poco de suerte. Y lo que era más importante aún, estaría fuera de ese
punto sin cobertura en menos que eso, por lo que al menos la podría
llamar.

Cogió su caja de herramientas y sacó los cortaalambres, recortó el
recubrimiento de plástico del cableado de cobre y cortó los cables.
Luego tiró los viejos cables de la caja de conexiones y sacó un rollo
largo y delgado de fibra óptica de la caja de herramientas. Aquellas
eran fibras ópticas interesantes (bueno, vale, sólo Terry las encontraba
interesantes) porque eran aún más finas de lo normal. Un nuevo sistema
desarrollado por los americanos (siempre son los mismos), y este
edificio era el primero en Reino Unido en utilizarlos. Habían enviado a
Terry a un curso en Nueva York hacía seis meses para aprender sobre el
sistema. Aquello había sido divertido: un montón de noches en la ciudad
con Johnnie Bates, descubriendo que era de verdad la ciudad que nunca
dormía. Frecuentemente habían conseguido llegar a las clases de
entrenamiento al día siguiente, con resaca pero felices.

Sin embargo, Terry era lo bastante sensato como para saber cuándo ir de
fiesta y cuándo aplicarse y hacer el trabajo, y él y Johnnie habían
vuelto a Inglaterra, certificados para trabajar en la instalación de
esta nueva fibra óptica, lo que hizo a los dos populares con su jefe y
les ganó un poco de dinero extra.

Les habían prometido otro extra si hacían este trabajo y, francamente,
era el dinero lo que les movía a hacerlo. El cableado fue fácil, fue la
eliminación de las cosas viejas de cobre lo que estaba tomando más
tiempo.

Johnnie estaba un par de pisos por encima de él, más cerca de la suite
de demostración. Lo habían echado a suertes con una moneda para ver a
quién se paseaba por ahí con los peces gordos y tenía la oportunidad de
robar una taza de té de las secretarias y las APs y a quién le tocaba
las escaleras traseras y los pasillos de servicio. Terry había perdido,
por supuesto. No habría té para él.

Sacó un destornillador de su cinturón de herramientas y empezó a abrir
la última caja de conexiones, silbando algo que había oído en la radio
del coche. Cualquier cosa para pasar el tiempo.

Si hubiera mirado por encima de la obra que acababa de hacer, podría
haberse sorprendido al darse cuenta de que los cables de fibra óptica
que había conectado en las cajas de conexión anteriores brillaban
extrañamente.

Los cables que no estaban conectados a fuentes de energía rara vez
brillaban. Nunca, para ser francos. Simplemente no pasaba. ¿Por qué iban
a hacerlo? ¿Cómo podía ser posible? Pero estaba ocurriendo: pequeños
impulsos de color púrpura de energía, brevemente parpadeando arriba y
abajo por el cableado. Casi como el bombeo de la sangre por las venas de
una enorme criatura electrónica.

Terry no se dio cuenta porque estaba mirando hacia delante, mirando
dónde iba a hacer lo siguiente, no dónde acababa de estar.

Lo cual fue desafortunado. No sólo para Terry Lockworth, cuyos
espaguetis a la boloñesa quedarían sin comer esa noche, sino también
para casi toda la raza humana.

Terry entrelazó el último pedazo de cableado de fibra óptica en la
última caja de conexión y atornillándola, la cerró por última vez,
sonriendo para sí. Arriba, Johnnie debería estar recibiendo una prueba
de que el cableado estaba terminado, y sus monitores le dirían que todo
estaba listo para salir.

Mientras Terry terminaba de apretar el último tornillo, un masivo rayo
de energía alienígena púrpura se precipitó a través de su
destornillador, por su mano y por todo su cuerpo. Se movió tan rápido
que, en el momento en que los minúsculos copos carbonizados que eran
todo lo que quedaba de Terry cayeron al suelo, el destornillador sólo
estaba empezando a separarse de la cabeza del tornillo.

Por supuesto, Terry tuvo suerte. Al morir tan repentina, violenta y
eficientemente, se salvó de lo que estaba por venir en los próximos
días.

Pero él probablemente no lo hubiera visto exactamente así.

Arriba, en la suite del ático, Johnnie Bates estaba vinculando todos los
ordenadores con el principal servidor de administración del Hotel
Oracle, faro del esplendor arquitectónico que era el Western Development
Business District, comúnmente conocido como la Milla de Oro, justo al
lado izquierdo de la autopista M4 saliendo de Londres.

Pero para el hombre que era dueño del hotel, Johnnie era sólo un pequeño
hombre con un mono gris haciendo algo con cables.

Dara Morgan, según las biografías que mantenía cuidadosamente en las
páginas web de sus empresas, hizo su primer millón en Derry, cuando
tenía sólo 26 años, con la creación de una popular web de descarga de
música que permitía descargas baratas seis veces más rápido que las
velocidades tradicionales y con cuatro veces más calidad que los MP3
tradicionales.

La industria de la música le amaba. Los consumidores le amaban. El
gobierno le amaba. Su madre le amaba (bueno, eso pensaba, no hablaban
mucho estos días, es lo que tenía que estuviera en una urna de plata en
la repisa de la chimenea junto a su padre). Y el mundo de los negocios
le amaba. Cuatro años más tarde, MorganTech era la financiación tras el
nuevo Día Mundial de la Donación de Sangre, trayendo trabajo y
desarrollo a Hounslow, Osterley y a todas esas otras zonas de Londres
entre Brentford y el aeropuerto de Heathrow de las que nunca había oído
hablar antes de la compra de los terrenos.

Con un portafolio personal de alrededor de 65 millones de libras, a un
paso de ser una megaestrella, ya cenaba con los Trump, los Gates, los
Rothschild, los Gettys y media docena más de los que movían los hilos
con nombres impronunciables de todo el mundo. En realidad, los nombres
no eran impronunciables, pero Dara Morgan no se molestaba en
recordarlos. Ellos simplemente no importaban lo suficiente.

Lo que le importaba ahora era conseguir que las suites de su nuevo hotel
estuvieran listas para la demostración de su nuevo ordenador portátil. Y
el pequeño hombre del mono gris mugriento no iba lo bastante rápido.

---¿Cait? --- chasqueó los dedos y una potentemente vestida pelirroja
con unas finas gafas metálicas y unos tacones increíblemente altos se
acercó.

---Señor Morgan, ¿señor? ---Dara Morgan señaló hacia el hombre del mono
gris.

---¿Cuánto tiempo más hay que esperar? ---preguntó, con su acento
norirlandés suave espetando inusualmente.

Caitlin asintió comprendiendo y se acercó a preguntarle al hombre para
obtener información.

Dara Morgan sonrió para sus adentros, mirando a Caitlin mientras se
movía. Él la apreciaba en muchos aspectos, pero su belleza estaba muy
alta en la lista.

Todos en su organización eran de Derry o de los distritos vecinos de
Irlanda del Norte. Y lo que era más importante aún, todos ellos eran
gente con la que había crecido. Todas las historias de los padres y
hermanos mayores sobre las luchas, los asesinatos, el honor. Las
marchas, las tropas, las recriminaciones y los castigos.

Era la historia para Dara Morgan, algo de otra época, por lo menos. Su
generación no había tenido tiempo de preocuparse por las luchas, más de
lo que se preocupaban por las hambrunas de patata o de Oliver Cromwell.
Esa era historia antigua. Dara Morgan y MorganTech eran el futuro. En
muchos sentidos.

Pasó una mano por su pelo largo hasta los hombros y luego sacó su móvil,
haciendo una pausa para oler el aroma de champú de sus dedos. Era muy
importante estar limpio. Para verse bien y oler mejor.

En el colegio, le habían diagnosticado como una forma de trastorno
obsesivo compulsivo, ¡como si la compulsión obsesiva de lavarse las
manos cada vez que entraba en contacto con otra persona fuera algo
malo! Las personas llevaban gérmenes y, aunque no creía ni por un
momento que iba a ser derribado por la malaria sólo con darle la mano a
un extraño, no era tan irracional cuidarse uno mismo de vez en cuando.

En el colegio nunca le entendieron, o eso recordaba vagamente. Era
demasiado pequeño, probablemente demasiado centrado en los planes de
estudios, en los horarios y en los deportes.

No podía esperar para irse, y así lo había hecho en el momento en que
terminó sus exámenes. No habría bachillerato, universidad o formación
profesional para él. Directo a los negocios, directamente a la
tecnología de la información, el futuro del mundo, directamente a la
creación de un sistema de MP3 para los trogloditas que pensaban que Gran
Hermano y Factor X eran el principio y el fin de toda la cultura
televisiva. Él los necesitaba, por supuesto, porque le habían ayudado a
alcanzar su potencial, habían sido los primeros peldaños en la escalera
hacia el éxito. Para poder gobernar el mundo, a través de los negocios.
No tenía ningún deseo de gobernar el mundo en realidad, pues estaba
lleno de demasiadas gruesas personas peleándose por el petróleo, el
territorio y Dios para ser un plan bastante sensato como para poder
controlarlo.

Pero podría dominar la tecnología, despedir a los llamados actuales
gigantes y comprar el acceso a los hogares y lugares de trabajo de todo
el mundo en el planeta.

Eso era suficiente. Y en la demostración de prensa de mañana, el plan
daría su primer paso.

Caitlin volvió y dijo que el hombre estaba esperando la llamada de otro
hombre en algún área de servicio en la entreplanta y estaría listo. Dara
Morgan miró hacia allí, el renombrado hombre estaba tratando de llamar.

---Dile a tu amigo ---le dijo Dara Morgan a Caitlin---, que no va a
conseguir hablar con su colega. Las áreas de servicio están bloqueadas a
las señales de cobertura. Dile que tiene que usar un terminal. Si las
fibras ópticas están conectadas, le pondrán directamente con el móvil de
su socio. Caitlin asintió y transmitió el mensaje.

Dara Morgan vio como el renombrado hombre insertaba la conexión de fibra
óptica en la parte posterior de su ordenador portátil y marcaba con
aquello. Hubo un destello púrpura y, donde el hombre estaba arrodillado,
había ahora sólo un montón de cenizas. Un acre olor a quemado flotaba en
el aire, y Dara Morgan arrugó la nariz con disgusto. Carne quemada,
tela fundida y sudor. Asqueroso.

---Bueno ---dijo Caitlin---, eso indica una buena señal, señor ---Dara
Morgan dio una palmada con fuerza, y todos los demás en la sala, todos
los que habían ignorado la muerte de Johnnie Bates, se giraron hacia él.

---Gente, al parecer el hotel está cableado O ``fibrado'', diría yo
---hubo un cortés murmullo de risas.

---Mañana, dominaremos el mundo.

---¡Ey! Una palabra / frase / ruido gutural, balbuceado con un toque de
indignación, un toque de sarcasmo y un gran trago de volumen.

No importaba lo mucho que lo intentara, el Doctor no podía dejar de
suspirar cada vez que lo oía. Por lo general, la indignación, el
sarcasmo y, especialmente, el volumen tenían como objetivo su dirección.

Él suspiró y se volvió hacia Donna Noble, la reina de los Ey. Pero ella
no estaba allí.

Sólo la TARDIS, aparcada entre dos contenedores de basura municipales.
Muy cuidadosamente, se dijo a sí mismo. Oh. ¡Ah! Cierto.

---Lo siento ---dijo a la puerta de la TARDIS, luego dio un paso hacia
atrás y la abrió, revelando a Donna que estaba en el umbral.

---Supuse que ya estabas fuera. ¿Qué parte de ``Estoy justo detrás de
ti'' no te ha quedado clara? ---Donna preguntó educadamente, con ese
giro de cabeza que denotaba que no decía lo que realmente sentía, es
decir, que no era del todo educada---. ¿Qué parte de ``Espérame'' te ha
esquivado el oído? ¿Qué pedazo de ``Me estoy poniendo algo cómodo'' se
ha desvanecido en el éter?

No había manera de que el Doctor se pudiera escabullir de aquello. Así
que se encogió de hombros.

---Te he dicho que lo sentía.

---¿``Lo sientes''?

---Sí, ``lo siento''. ¿Qué más quieres?

---¿``Lo sientes'' por no escucharme? ¿``Lo sientes'' por haberme
encerrado dentro de tu nave espacial extraterrestre? ¿O ``lo sientes''
por no haberte dado cuenta de que no estaba contigo?

Cada vez, que Donna escupía las palabras ``lo sientes'' sonaba como las
palabras menos arrepentidas de la lengua y adquirían un nuevo
significado que los lingüistas podrían discutir sobre su implicación
exacta los próximos doce siglos.

---No hay manera que pueda ganar esto ---dijo el Doctor ---, pero voy a
dejarlo ir, ¿de acuerdo?

Donna abrió la boca para hablar de nuevo, pero el Doctor se adelantó y
puso un dedo en sus labios.

---Calla ---dijo.

Donna calló. Y le guiñó el ojo.

---¡He ganado! ---y entonces le dedicó esa fantástica e increíble
sonrisa que siempre ponía cuando le estaba tomando el pelo, y él suspiró
admitiendo que le había pillado otra vez más.

Era un juego. Un juego entre dos amigos que habían vivido tantas cosas
juntos que jugaban instintivamente entre ellos. Familiaridad, amistad y
diversión. Las tres palabras que resumían el tiempo compartido por los
dos aventureros. Ella puso su brazo alrededor de él y lo atrajo hacia
sí.

---Así que, ¿cuál es el plan, Plano?

El Doctor hizo un ademán con la cabeza hacia la carretera principal de
Chiswick y su ajetreo y el bullicio del tráfico, y rápidamente la
arrastró por la acera, dispuesto a perderse por entre la multitud. Lo
único que no había ninguna. De hecho, no había mucha gente a su
alrededor, sólo un par de niños en un monopatín en la acera de enfrente
y un anciano que paseaba a su perro. El Doctor levantó su otra mano.

~---No llueve ---dijo.

---Bien visto, Sherlock ---dijo Donna

---¿Es domingo? Querías el viernes 15 de mayo de 2009, Donna. Y ese es
el día para el que programé la TARDIS.

Donna se rió.

---En ese caso es probable que sea un domingo de agosto de 1972. El
Doctor asomó la cabeza en un quiosco, sonriendo al hombre detrás del
mostrador, que estaba escuchando a su reproductor de MP3 y haciendo caso
omiso de su cliente en potencia. El Doctor miró el periódico más
cercano.

---Hoy es viernes, 15 de mayo de 2009 ---confirmó a Donna.

---Entonces, ¿dónde está todo el mundo?

---Tal vez sea la hora de comer ---sugirió el Doctor---. O tal vez
Chiswick ya no sea el centro neurálgico de la sociedad como lo era hace
un mes. ¿Vamos a tu casa?

---¿Tú vienes?

El Doctor puso una expresión como si la idea de no ir con Donna no se le
hubiera pasado por la cabeza.

---Oh. Umm\ldots{} Bueno, sí.

---Doctor, ¿por qué estamos aquí?

---Es el primer aniversario de la muerte de tu padre.

---Y, aunque estoy segura de que estará agradecida por haber salvado al
mundo de los Sontaran, no creo que mi madre se alegre de verte hoy,
entre todos los días.

---Tu abuelo lo estará.

---¿Sí? Bueno, llévatelo a tomar una pinta esta noche a Shepherd's Hut,
pero para empezar quiero verlos por mi cuenta.

Donna aún sostenía su mano, y la apretó suavemente.

---Lo entiendes, ¿no?

Él sonrió.

---Por supuesto que sí. No lo he pensado. Lo siento. \\---No empecemos
de nuevo, ¿vale? ---Donna le soltó la mano---. Voy a conseguir algunas
flores y caminaré hasta casa. ¿Por qué no nos encontramos de nuevo aquí,
a esta hora, mañana?

---Aquí. Mañana. Vendido.

El Doctor le guiñó un ojo y comenzó a alejarse caminando.

---Hay una bonita floristería en la esquina de ahí ---gritó ---.
Pregunta por Loretta y dile que yo te mando.

Dobló una esquina y desapareció. Donna respiró hondo y se dirigió en la
dirección que había señalado.

Hacía un año. Ese día. Los Adiposos. Los Pyroviles. Los Oods con
cerebros en las manos.

Incluso los tubos de escape Sontaran, los Hath y los esqueletos
parlantes, todo parecía sencillo en comparación con lo que iba a pasar
esta tarde.

Porque esa tarde Donna tenía que volver y estar allí para su madre y
probablemente no sólo revivir el año pasado, sino los días y semanas que
siguieron, los funerales, decírselo a la gente, los memoriales, los
avisos en los periódicos, la parte financiera de las cosas, encontrar el
testamento\ldots{} nada de eso había sido fácil para la madre de Donna.
No había sido tan fácil para Donna, la verdad sea dicha, y hace un año
ese habría sido su pensamiento predominante. Donna Noble, poniéndose a
sí misma en primer lugar.

Pero ahora no: sólo un breve tiempo con el Doctor le había demostrado
que ella no era la mujer que había sido entonces.

Y el abuelo, el pobre abuelo, recordando la muerte de la abuela, él
había seguido valientemente adelante por el bien de todos los demás,
tratando de coordinar a los abogados y a los directores de funerarias y
similares.

No es que mamá hubiera sido débil, pues Sylvia Noble no era así, y se
habían preparado para la muerte de papá, bueno, todo lo que se puede
estar, pero aún eso la perseguía. Podía verlo en los ojos de su madre,
era como si alguien le hubiera cortado un brazo y una pierna, y mamá lo
asimiló lo mejor que pudo. Habían estado casados durante treinta y ocho
años.

Donna suspiró.

---Te echo de menos, papá ---dijo en voz alta mientras frenó en seco en
una lavandería llamada Loretta.

Su teléfono vibró al entrar con un mensaje de texto, y lo leyó.
«Eh\ldots{}..De hecho, podría estar equivocado. Puede que Loretta no sea
una florista. Lo siento.»

¿Cómo hacía eso? Ni siquiera tenía un móvil por lo que Donna sabía. ¿El
destornillador sónico, tal vez? ¿No había nada que no pudiera hacer?
Metiendo el teléfono en el bolsillo del abrigo, Donna decidió que sería
mejor ir en dirección a Turnham Green. Sabía que había una floristería
allí.

Hombres. Hombres alienígenas. Inútiles, todos ellos.

Lukas Carnes odiaba la tecnología. Lo que le hacía un poco raro, según
todos sus compañeros. Su madre tenía un PC, pero Lukas evitaba usarlo,
si era posible, a menos que fuera para escribir ensayos del colegio una
vez los había escrito a mano. Tenía un reproductor de MP3 al que su
hermano menor (que tenía ocho) tuvo que ponerle la música por él. Y ni
siquiera le hablaras de los problemas asociados con el uso de un DVD-R.

Él era, había decidido en su decimoquinto cumpleaños, un atrasado de una
época anterior, por allá cuando a los chicos con conocimientos
tecnológicos les llamaban frikis y las niñas salían huyendo de ellos.
Tristemente para Lukas, la mayoría de las chicas que conocía querían un
tipo que pudiera descargarse la música a veinte revoluciones y
desbloqueara un móvil que había comprado de un puesto chungo del mercado
de Shepherds Bush.

Por lo que Lukas no tenía novia.

Lo que añadió combustible a su odio apasionado por la tecnología.
Aceptaba que la necesitaba, simplemente no quería entenderla. Su cerebro
no estaba programado para entender compresiones MP3, el 3G y sistemas de
localización GPS. Sólo quería presionar un botón de encendido y que todo
funcionase. ¿No era eso por lo que el grupo de edad de su madre había
pasado durante la Primera/Segunda /Tercera Generación de tecnologías?
Así que podría presionar botones y las cosas funcionarían sin estar
obsoletas en seis meses ni inútiles en doce. En la televisión se hablaba
de que un día podrías chasquear los dedos y las puertas se abrirían,que
entrarías en una habitación y dirías ``luces'' y un ordenador lo
encendería todo, justo en el nivel adecuado.

Dios mío. ¡Era como su abuela! ¿Lo siguiente sería decir que no podía
entender la música pop y que el sexo era esa persona en Popworld?
Quince, no cincuenta, Lukas.

Así que, ¿por qué estaba de pie en la sucursal local de Discount
Electronics, viendo una demostración del más reciente Procesador de
Cuarta Generación en un ordenador portátil más fino que un pedazo de
cartón? Porque su hermano, su hermano Joe, de 8 años de edad con
conocimientos técnicos, se lo había pedido. Bueno, estrictamente
hablando, mamá se lo había pedido. Con el padre de Joe ausente, igual
que el de Lukas antes que él, el hijo mayor se había convertido en el
padre de facto para su hermano pequeño. Lo que venía bien a Lukas,
porque adoraba a Joe en secreto, pero nunca se lo diría. Y porque los
hermanitos pequeños necesitaban saber quién era el jefe, y el poder de
Lukas se perdería a la primera señal de debilidad.

Y Joe había dado bastante guerra después de que su padre se fuese,
metiéndose en problemas en la escuela y en el barrio, y a mamá le había
visitado la policía dos veces.

Así que Lukas se había llevado aparte a Joe y había explicado lo mejor
que pudo a un niño de 8 años que no era culpa de mamá que su padre se
hubiese ido, ni era la de Joe, y que perdiendo el tiempo con los chicos
mayores, ayudándoles a mangar coches y esas cosas, no estaba ayudando a
mamá.

Después de unos meses, Joe se había calmado. Pero ahora se aferraba a
Lukas en todo momento, y se enrabietaba si su hermano mayor no le
llevaba a todas partes. Incluso Lukas había empezado a llevar a Joe a la
escuela secundaria antes de dirigirse al Instituto Park Vale. Hecho qué
mamá agradecía que no acabase , así que eso era bueno.

Pero de vez en cuando, Lukas mismo quería enrabietarse, estar solo, no
ser el responsable.

Hoy era un día así, pero aquí estaba con Joe, viendo esta nueva
demostración junto con otras treinta personas, todos metidos en una
tienda en la que probablemente cabían de forma segura diez personas como
máximo. Qué Dios los ayudase si había un incendio.

Un mujer enorme (en todos los sentidos) se colocó delante de ellos, por
lo que Lukas izó a Joe en sus brazos para que pudiera ver mejor. Esto
significaba que Lukas no podía ver nada. Así, mientras que Joe (¡qué
mayor se estaba haciendo!) miraba con atención, la mirada de Lukas
erraba por la tienda.

Un tipo flaco vestido con un traje azul estaba tecleando en un ordenador
portátil de demostración, que probablemente iba a estar obsoleto al
final del día. El tipo estaba buscando algo en Internet, Lukas podía ver
pantallas repetidas mostrando un motor de búsqueda (¡ooh, un término
técnico!), y frunciendo el ceño. Era evidente que no estaba consiguiendo
los resultados que quería.

El hombre metió la mano en el bolsillo y sacó un tubo brillante,
parecido a un rotulador, y apuntó a la pantalla. Al principio, Lukas
pensó que iba a escribir en la pantalla del ordenador portátil, pero en
cambio el extremo del bolígrafo brilló azulado y Lukas observó con
asombro como las imágenes de la pantalla se descargaban y cambiaban a un
ritmo extraordinario. No, a un ritmo imposible. El hombre vestido de
azul sacó un par de gruesas gafas negras de otro bolsillo y se los puso
mientras miraba fijamente a las cambiantes pantallas. ¿En serio podía
leer tan rápido? Se dio cuenta de que Lukas lo miraba y sonrió, casi
con timidez. El bolígrafo brillante volvió a un bolsillo, las gafas a
otro.

Lukas se dio cuenta de que tenía la boca abierta, así que la cerró. El
Tipo del Traje Azul le guiñó un ojo a Lukas y estaba a punto de salir de
la tienda, cuando se adelantó y cogió un folleto sobre la demostración
que el hermano pequeño de Lukas estaba viendo. Después miró a la
multitud y se acercó. Lukas volvió rápidamente su atención de nuevo a la
demostración. O al menos a la parte posterior de la cabeza de la Señora
Gorda.

Tras unos minutos escuchando a una rubia hablando de lo revolucionario
que era el nuevo sistema informático, el Tipo del Traje Azul se encogió
de hombros y murmuró algo así como ``imposible'', ``no en este planeta''
y ``contradiciendo el Protocolo Decimoctavo de la Proclamación de las
Sombras'', y en ese momento Lukas decidió que, con o sin bolígrafo
brillante, este hombre era probablemente un loco. Quizá debería alejarle
de Joe, por si acaso el tipo tuviese un cuchillo.

Lukas se inclinaba para susurrar al oído de Joe que tal vez ya era hora
de irse a casa, cuando el Tipo del Traje Azul le dio un codazo.

---Así que todo el mundo está dentro de todas las tiendas de
electrónica, ¿no?

---¿Perdón?

---Las calles estaban bastante vacías. Mientras caminaba, me he dado
cuenta de que todo el mundo está en este tipo de tiendas, viendo estas
demostraciones.

---Hoy es el lanzamiento ---se encontró explicando Lukas---. Todo el
mundo está interesado. ---Tú no ---respondió el Tipo del Traje Azul.

Lukas se encogió de hombros.

---Mi hermano pequeño lo está.

---Ah. Ya veo.

Lukas trató de alejarse, pero fue cercado entre otro hombre por un lado
y la Señora Gorda por delante. El Tipo del Traje Azul volvió a sacar su
bolígrafo brillante.

---No me hagas caso ---dijo.

Pero Lukas le hizo caso. Mucho.

---¿Por qué estás aquí? ---preguntó.

El Tipo del Traje Azul se encogió de hombros.

---Bueno, en primer lugar, estoy dejando que una amiga vaya a su casa.
En segundo lugar, me preguntaba por qué todo el mundo está aquí. Y en
cuarto lugar, ahora estoy realmente preocupado por la tecnología de ese
portátil.

Lukas sabía que iba a arrepentirse de esto.

---¿Y en tercer lugar?

---¿En tercer lugar? ---el Tipo del Traje Azul parecía confundido, y
después sonrió como si algo le hubiera vuelto a la memoria de
repente---. Ah, sí, en tercer lugar, vine a buscarte, Lukas Samuel
Carnes ---le tendió la mano---. Soy el Doctor y estoy aquí para salvarte
la vida.

Dara Morgan bebía su café lentamente. En parte para demostrar que tenía
buenos modales, y en parte porque estaba demasiado caliente para hacer
nada más. Pero probablemente les parecerían buenos modales al señor
Murakami y su delegación.

---Entonces Señor Morgan ---estaba diciendo el banquero japonés---,
¿tenemos un trato? Los ojos azules de Dara Morgan brillaron con
picardía mientras miraba a Caitlin, de pie junto a la puerta de la
oficina.

---¿Qué piensa usted, Cait?

Caitlin se acercó, sus piernas largas y su falda corta se dibujaban
claramente en los ojos de algunos miembros del séquito del señor
Murakami, pero, observó Dara Morgan, no en los del propio señor
Murakami. Bien.

---Creo que es un buen trato, señor ---ronroneó---. Si Murakami-San
puede llevar a cabo el M-TEK en todo Oriente para el domingo,
sería\ldots{} magnífico.

Dara Morgan se apartó el pelo de los ojos.

---Justamente en menos de dos días, a las 15:30 hora de Tokio.
¿Factible?

El Señor Murakami frunció el ceño.

---¿Por qué el domingo? El plazo es ridículamente corto.

Dara Morgan se limitó a sonreír.

---Digamos sólo que todo el trato depende de eso. Necesito esa garantía,
Murakami-San, o me iré a otro sitio.

---Pero de esa forma usted tiene aún menos posibilidades de un acuerdo
para llevarlo a cabo para entonces ---dijo el japonés.

Dara Morgan asintió.

---Lo sé. Pero afrontémoslo, con el dinero que proporcionará el M-TEK,
las empresas más pequeñas que la suya, algunas quizá más hambrientas,
estarían más dispuestas a satisfacer mis\ldots{} los requisitos de
MorganTech ---tomó otro sorbo de café---. Estará en el contrato, con
cláusulas de penalización.

---¿Cuáles serían?

---Catastróficas. Para todo Japón.

El séquito del señor Murakami se acercó a su jefe un centímetro más
cerca.

---¿Eso era una amenaza, Señor Morgan? ---preguntó en voz baja.

---No ---dijo Dara Morgan---. Yo no amenazo. Los bárbaros amenazan. Los
idiotas amenazan. Simplemente expongo hechos.

---Es una gran oportunidad ---intervino Caitlin---. Por favor, piense en
ello durante la cena. Esta noche. A nuestras expensas.

---Por desgracia, no podremos acompañarle ---añadió Dara Morgan---, pero
es libre de escoger cualquier restaurante de Londres, el que más le
guste, y todos los gastos serán cubiertos por MorganTech. De hecho,
insisto en ello.

---¿Todos los gastos?

---Los relativos a comida y bebida, sí.

---Ah. En ese caso, tendrá mi respuesta antes de la medianoche de hoy.

El Señor Murakami se levantó y Dara Morgan hizo lo mismo, inclinando
ligeramente la cabeza mientras lo hacía. El Señor Murakami respondió del
mismo modo, incluyendo a Caitlin y ella le saludó con la cabeza a él y a
todo su séquito. Terminadas las formalidades, la delegación japonesa se
dirigió a la puerta de la suite, pero el Señor Murakami se volvió una
ultima vez.

---En serio, ¿por qué el domingo? ¿Por qué las 3:30 de la tarde?

---Porque algo grande va a pasar en todo el mundo el lunes a las 15:00
hora de Reino Unido. Eso son las 23:00 hora de Tokio. Pero necesitamos
todos los tratos en marcha Yo también he hecho un trato, ya ve, pero es
más bien como una cadena de adquisición de propiedades: un eslabón que
se rompa y todo el trato se derrumba. Y entonces todos sufriremos.
---¿Todos?

---Universalmente.

Dara Morgan miró de reojo a Caitlin, y de inmediato fue a escoltar al
Señor Murakami fuera de la habitación. Momentos después, los japoneses
se habían marchado y Caitlin estaba de regreso al lado de Dara Morgan.
Él estaba de pie frente a un enorme ventanal con una gran vista
panorámica del Oeste de Londres. Podía ver el nuevo estadio de Wembley,
el Centrepoint, el London Eye y otras estructuras altas de Londres.

---El lunes ---sonrió---, y este planeta será de Madam Delphi.

Caitlin asintió.

---Al fin. La venganza es suya.

Se cogieron de las manos y las apretaron con fuerza, mirando a las
pantallas del ordenador, que parecían estar tarareando una melodía, muy
levemente, causando ondas en una de las pantallas para hacer pulsos de
forma fraccionada en el tiempo.

---Bienvenida de nuevo ---le dijeron juntos.

Donna estaba de pie al final de Brookside Road y respiró profundamente.
No había pasado mucho tiempo desde la ultima vez que había estado aquí
(de hecho, para su madre, había sido menos tiempo aún), pero cada vez
que llegaba a casa, había momentos incómodos. ``¿Dónde has estado?'' y
``¿Sigues dando vueltas con ese horrible Doctor?'' y ``¿Por qué no
llamas?'' y ``¿Tienes trabajo ya?''. Por supuesto, el abuelo Wilf sabía
dónde estaba, se lo había dicho todo desde el principio. Pero su madre,
bueno, ella no era una persona que lo pudiera entender. No era de las
que pensaran que salvar Oods, detener guerras generacionales o
asegurarse de que Carlomagno se reuniese con el Papa realmente equivalía
a un ``buen'' trabajo como era tomar notas o hacer pedidos.

Respirando profundamente, caminó hacia la casa que en realidad no había
sido su hogar mucho tiempo. Después de su desastrosa boda y un viaje un
poco menos desastroso a Egipto, sus padres se habían mudado de la casa
adosada donde Donna había crecido a este nuevo chalet. Había sido una
revolución, agravada por una Donna sin trabajo y Wilf estando en
principio enfadado porque pensaba que tendría que dejar a sus colegas
astrónomos abandonados. Al final resultó que, por supuesto, era más
fácil llegar a los huertos desde la nueva casa, así que terminó siendo
feliz después de todo.

Pero el padre de Donna no había estado bien por mucho tiempo, y en
muchos sentidos el traslado había sido idea suya, su deseo de encontrar
algún lugar nuevo donde vivir, que le diese nuevos retos.

Se había aburrido en la vieja casa. Había construido todos los armarios,
puesto estanterías en todas las paredes, pintado todos los techos que
fue capaz, y necesitaba algo nuevo para mantenerlo activo desde que su
enfermedad le había hecho jubilarse anticipadamente. Preparar el nuevo
lugar con las exigentes especificaciones de mamá sería exactamente el
desafío adecuado. Llevaban viviendo allí tres meses cuando su padre
murió.

Donna y Wilf habían tomado el relevo de hacer todos esos trabajos que
papá había estado haciendo, pero nunca fueron buenos del todo, nunca
fueron ``cómo tu padre lo habría hecho''. Lo que no era del todo
sorprendente, Wilf era veintitantos años más viejo, y Donna nunca había
levantado una brocha o un martillo en toda su vida.

Dios mío. ¡Qué superficial era Donna Noble antes de volverse a
encontrarse con el Doctor! Había aprendido no solo a valerse por sí
misma, sino a darse cuenta de que podía. Su vida familiar era una
verdadera situación del huevo o la gallina. ¿Habría sido una inútil en
casa porque sus padres siempre le habían permitido serlo, o su madre
pensaba que era inútil porque lo era? Y hablando de ello, hablando de
cualquier cosa, con Sylvia Noble era raramente una experiencia positiva.
A Donna le encantaría decir que la amargura y el resentimiento de su
madre era a causa de la muerte de papá, pero la verdad era que a Sylvia
siempre le había decepcionado su hija. Rara vez lo escondía. Y Donna
nunca entendió el por qué. ¿Habría deseado un hijo? ¿Habría deseado una
abogada de altos vuelos o una ejecutiva de empresa que fuera lo
suficientemente rica como para enviar a sus padres a vivir en el campo
en una pequeña casa rural del siglo XVI donde podrían cuidar cabras?
¿Había empeorado desde que papá había muerto? ¿Habría sido mejor si
ella se hubiera casado con Lance? ¿Debería haberle dicho la verdad de
ese día a mamá como había hecho recientemente con el Abuelo?
Probablemente no, porque a Sylvia no le gustaba que la gente fuese
abierta y honesta. ``Esos corazones sangrantes que llevan sus corazones
en las manos'' era una analogía con la que la había torturado una vez, y
resumía su opinión real sobre la gente que era honesta.

Donna recordó haber leído una vez en un artículo de una revista acerca
de cómo los padres nunca esperaban entender realmente a los adolescentes
y su mejor apuesta para una convivencia armoniosa era sólo tolerar esos
tres o cuatro años de pesadilla. Pero, ¿había algún manual para hijos e
hijas sobre cómo tratar con padres negativos? Era realmente imposible
discutir con una madre, tenía el botón ``culpabilidad'' incorporado para
presionarlo y prohibir que les dijeses todas las cosas que querías
decirles, sea lo que sea que te lanzasen. Donna amaba a su madre, no
había vuelta de hoja. Y no tenía ninguna duda de que Sylvia Noble amaba
a su hija. Simplemente no estaba segura del todo de que ambas en
realidad se gustasen mucho.

---Hola, Donna ---dijo el Señor Baldrey desde el otro lado---. ¿Has
tenido un buen viaje? ---Sí, gracias ---Donna le devolvió la
sonrisa---. ¿Cómo está Seymour?

---Oh bien. Quejándose todavía de su próstata ---gimió el vecino.

Donna pensó que la conversación había llegado tan lejos como francamente
quería que fuera y aceleró el paso hacia la casa. Un gato se sentó junto
a un poste de luz, mirando con recelo como Donna se aproximaba, no muy
seguro de si era amiga o enemiga. Donna hizo ruidos chirriantes para
atraer su atención. Salio corriendo disparado. Ah bien.

El coche de mamá estaba fuera en la calle (tienes un coche, mamá, úsalo)
y Donna tocó el capó al pasar. Frío. Entonces Mamá no había salido hoy.
Es curioso cómo había adquirido estas pequeños hábitos para descubrir
cosas al viajar con el Doctor. Como lo de si un coche había sido
conducido. La antigua Donna nunca habría pensado en eso. A la antigua
Donna no le habría importado. La antigua Donna se había ido. Gracias a
Dios. Su vida era un trillón de veces mejor estos días. Si tan sólo
pudiera implicar a su madre en ella, pensó. Ese último pedacito del
rompecabezas, ese último trozo de aceptación de cada uno de ellos.

---Oh ahí estás, Señorita ---dijo una voz familiar detrás de ella---. Me
preguntaba si volvería a verte hoy.

Donna no se molestó en darse la vuelta.

---Hola, mamá ---dijo.

---Ah, sí, ``Hola, mamá'', porque eso es suficiente, ¿no? Un instante
el aire está lleno de gases asfixiantes y al siguiente el cielo está en
llamas y ya está. Ni idea de donde está mi única hija. Ninguna llamada,
ni mensajes, ni siquiera un mensaje a tu abuelo, así podría hacerme
callar. Nada.

Donna se detuvo en la calle y se volvió hacia su madre, cogiendo
automáticamente dos de las cuatro bolsas de la compra que llevaba en la
mano para ayudarla. La antigua Donna no habría considerado eso tampoco.

---Encantada de verte también ---dijo Donna---. ¿El Abuelo tiene puesta
la tetera? Me iría bien una tacita, muchas, para serte sincera.

Sylvia Noble se encogió de hombros y adelantó a su hija.

---¿Sabes qué día es hoy, ¿verdad? ---le replicó.

Y Donna se detuvo en seco. Por supuesto que sabía qué día era hoy. ¿Por
qué diablos creía que estaba allí? ¿Cómo se atrevía incluso a
preguntarlo? La puerta principal se abrió y Sylvia pasó al lado del
Abuelo Wilf y se fue sin decir nada a la cocina.

---¿Sabes lo que me acaba de preguntar? ---Donna le siseó después de
besarle en la mejilla.

Wilf elevó los ojos al cielo.

---Va a ser uno de esos días, ¿no?

Donna abrió la boca para responder pero se detuvo. La antigua Donna se
habría marchado, en el acto. La antigua Donna hubiera comenzado una
pelea con su madre, lanzando palabras como ``actitud'' y ``sea lo que
sea'' y ``egoísta''. La nueva Donna no lo hizo.

Porque la nueva Donna, frustrante como era, comprendió que lo que Sylvia
Noble probablemente quería y necesitaba hoy era llorar un rato. Pero
siendo Sylvia ``Yo no llevo mi corazón en la mano'' Noble significaba
que nunca haría eso. Y tristemente iba a ser el trabajo de hoy de la
nueva Donna asegurarse que lo hacía, antes de que la represión de su
madre le hiciera más daño del que ya le hacía.

El Doctor bajaba a grandes zancadas por la calle principal de Chiswick,
mirando dentro de varias tiendas donde la gente se quedaba mirando las
nuevas demostraciones de portátiles. ---Es sólo un ordenador
---murmuró---. ¿Por qué tanto interés?

---Ya están obsoletos ---dijo una voz joven a su lado---. El M-TEK, eso
es el futuro.

El Doctor miró entonces hacia abajo. El que había hablado apenas le
llegaba a la altura de las rodillas. Era un niño pequeño. El Doctor lo
había visto antes en alguna parte. Entonces, corriendo hacia él, vio a
Lukas Carnes y se dio cuenta que era el hermano pequeño que había estado
llevando a hombros.