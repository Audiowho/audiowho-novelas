\chapter*{Sábado}
\addcontentsline{toc}{chapter}{Sábado}

Caitlin estaba esperando en la Terminal 5 del Aeropuerto de Heathrow a
los vuelos que deberían haber ido a la Terminal 2. Esa estaba siendo
demolida actualmente, preparándose para la Terminal del Este que
reemplazaría a aquella adonde solían llegar los vuelos europeos. Se
encontraría con diferentes vuelos, incluyendo uno que venía desde el JFK
a mediodía, que llegaría antes que los europeos. Vio aterrizar al enorme
jet 787-9, la luz del sol brillaba sobre su fuselaje nuevo mientras
frenaba lentamente, antes de cruzar rodando la puerta de llegada y ser
llevado a su posición por el pequeño camión remolcador.

Caitlin estaba vestida con un elegante uniforme del personal de la
Terminal, con su pase de acceso a todas las áreas, con la más alta
credencial de seguridad que Madam Delphy pudo conseguir, colgando de una
correa alrededor del cuello. Sonrió dulcemente a otros dos miembros
de la plantilla, ninguno de los cuales la detuvieron y ni siquiera
pensaron que pudiera ser posible que no la hubieran visto antes. El pase
era el que hacía eso, aunque el largo pelo moreno y los ojos azules de
Caitlin habrían sido suficientes para distraer a cualquiera que quisiera
verla de cerca. Una rápida sonrisa era todo lo que le costaba a Caitlin
conseguir lo que quería.

De hecho fue un agudo guardia de seguridad quien la vio mientras
atravesaba las puertas de Acceso Restringido hacia la sala de llegadas,
exactamente en dirección a donde realmente no tenía derecho a estar.

---¿Perdone? ---gritó.

---¿Hay algún problema\ldots{}? ---Caitlin arrugó los ojos para
leer su nombre en la etiqueta---. ¿Hay algún problema, Keith?

Keith Brownlow la miró fijamente, con la cabeza inclinada
ligeramente hacia un lado como evaluándola.

---No te he visto antes ---dijo en voz baja.

---No estoy aquí desde hace mucho ---respondió rápidamente.

---Es curioso ---replicó---. Conozco a todos en esta Terminal. Y
a los que no conozco, no entran aquí. Me temo que tendrás que irte hasta
que pueda verificar que eres quien dices ser.

Caitlin se detuvo por un momento, intentando recordar un nombre.

---Estoy segura que si lo compruebas con el Señor Golding te lo
confirmará. Me uní a la plantilla el jueves.

Keith se encogió de hombros.

---Lo haré, pero sólo si vuelves por dónde has venido y esperas en la
Zona Amarilla.

---Eres muy bueno en esto, Keith, ¿verdad?

---Nos tomamos la seguridad muy en serio ---dijo.

---Por supuesto que sí ---ronroneó Caitlin---. No es para menos. Pero
verás, solo estamos aquí tú y yo, has visto mi pase, puedes ver que
tengo autorización. Estoy segura que han sido los de administración los
que no te han informado de que estoy aquí. Realmente tengo que ir al
otro lado de esta puerta para saludar a algunos VIP.

Keith la ignoró, la mano derecha apoyada en el revólver enfundado en la
cadera y la mano izquierda levantando su walkie-talkie.

---Oh querido, y yo que pensé que podríamos ser amigos ---dijo Caitlin
levantando su brazo derecho y lanzando un rayo de energía púrpura al
pecho de Keith, reduciéndole a cenizas antes incluso de que se diera
cuenta del movimiento.

Continuó hacia adelante, sus tacones altos aplastaron las pocas cenizas
que quedaban en la alfombra. Con un rápido y profundo suspiro, abrió las
puertas de la zona VIP y salió a saludar a sus invitados.

Estaban allí de pie, esperándola, ninguno de ellos parecía darse cuenta
de dónde estaban o de qué estaban haciendo. Dos ancianos americanos
llegados de Nueva York.

---¿El Señor y la Señora DiCotta? Felicidades por su boda y bienvenidos
a su luna de miel.

Donnie y Portia DiCotta no dijeron nada, sólo asintieron con la cabeza
mientras Caitlin les llevaba hasta uno de los cochecitos de servicio con
los que los pasajeros ancianos o con discapacidad se movían por las
terminales de los aeropuertos.

Un confuso encargado de equipajes miró su pase mientras ella se lo
mostraba, pero después se apartó, permitiéndoles coger el cochecito. Los
DiCotta subieron a bordo mientras el encargado se giraba.

---¿Tienen equipaje? ---gritó.

Caitlin sonrió con su sonrisa más dulce, que esperaba que no sugiriese
lo que pensaba (``Oh, lárgate, patán inútil''), después dijo en voz
alta.

---Se ha retrasado en el JFK así que volveremos a por él por la mañana.

Puso en marcha el cochecito y condujo hacia delante.

---Sois los primeros ---dijo en voz baja, ya sin ser todo dulzura y
sonrisas.

---¿Dónde están los otros?

---El Griego llegará en breve, el grupo italiano no tardará más de una
hora o así. Esperaremos a recogerlos y después iremos a ver a Madam
Delphi juntos. Tiene una misión para vosotros esta noche ---Caitlin dejó
escapar una risita---. Me gustaría preguntar sobre el jetlag, pero
imagino que Madam Delphi se ha asegurado de que no sentís nada de eso.

---Nos sentimos bien ---dijo Portia DiCotta.

---Bien ---dijo Caitlin---. Perfecto.

El cochecito continuó hacia delante, lejos de las rutas principales y a
través de otra puerta por donde llegaría el vuelo del Aeropuerto
Internacional de Atenas.

---Y después iremos a encontrarnos con el genial 787-3 del Aeropuerto
Leonardo Da Vinci de Fiumicino ---dijo---. Espero que disfruten de su
estancia. Será breve pero muy dramática.

El pequeño cochecito atravesaba los largos pasillos preparado para
recoger a más componentes del ejército que Madam Delphi utilizaría para
llevar a la raza humana a una destrucción total.

Donna estaba ajustándole la corbata al Doctor. Lo había hecho unas cien
veces para su padre, especialmente hacia el final, lo que le había dado
una escusa a él para estar de mal humor y sentirse inútil. El Doctor no
tenía esa escusa, pero eso no le había impedido quejarse al respecto.

---Donna, puedo atarme una corbata, ¿sabes?

---¿En serio? Porque no he visto ninguna evidencia de eso ---fue su
respuesta, seguido por un tirón demasiado entusiasta del nudo, demasiado
cerca de la nuez de Adán del Doctor---. Ups ---le sonrió---. Lo siento.

Deslizó un dedo por el cuello de la camisa y de algún modo ese simple
gesto no sólo aflojó el nudo, sino que además desabrochó el botón
superior y después arrugó también la camisa.

Era arte, dijo. Chic friki, dijo. Arturo Desaliñado lo llamaba ella,
renunciando a esa causa perdida.

---¿Cómo sigue Netty? ---preguntó.

---Está bien hoy ---dijo Donna---. El abuelo se encarga de ella. Mama
estaba empezando a quejarse por ello, así que se marchó antes de tiempo
y dijo que nos encontraríamos aquí. ¡Creo que se ha ido a comprarse otro
sombrero estrafalario!

---Tu madre se preocupa por Wilf, eso es todo. Netty es muy divertida
pero implica una gran responsabilidad ---dijo el Doctor.

---Lo sé, pero Mamá no tiene que ser tan negativa respecto a eso, ¿no?

El Doctor se encogió de hombros.

---Es una madre, Donna. Es su trabajo criticar a cada miembro de su
familia. Está en el manual.

---¿Tu madre criticaba todo lo que hacías? La manera en que te
peinabas, la ropa que llevabas, los amigos con los que salías, la
música\ldots{}

---Sí, bueno, fue un poco diferente para mí ---dijo con calma.

Donna le miró y sonrió felizmente.

---Por supuesto que lo era. Lo siento. No lo he pensado.

---Oh, está bien. Solo estoy diciendo que le des a tu madre algo de
crédito. Es mucho por asimilar, ha cuidado sola de su abuelo durante
mucho tiempo. Ahora él tiene a alguien en su vida, ella está obligada a
estar un poco disgustada.

---Oh, no te pongas Spock conmigo.

---¿Spock?

---Sí, aquel psicólogo infantil, o quien sea que fuera. El que hablaba
sobre relaciones entre padres e hijos.

---Ah, el Doctor Spock. Por supuesto.

---¿Por qué? ¿Hay algún otro Spock que conozcas? ---se rió Donna
mientras salía de la habitación de invitados. Aunque sospechaba que el
Doctor no había pegado ojo allí, parecía que nunca necesitaba dormir
como una persona normal.

El Doctor se miró en el espejo. Siempre pensó que le quedaría bastante
bien una chaqueta de vestir y una corbata negra, odiaba las pajaritas,
le hacían parecer un camarero, y a juzgar por lo que había pasado en
otras fiestas, lo mejor para esta noche era una adecuada corbata negra.
Por supuesto, eso significaba que ahora parecía que iba a ir a un
funeral, pero jey-jo. Y luego estaba lo que le pasaba con las chaquetas:
no importaba cómo se abrochase los botones, siempre parecían o demasiado
pequeñas o demasiado ceñidas, y los pantalones nunca le llegaban hasta
los talones. Ah bueno, era la noche de Wilf, no la suya. Llamaron a la
puerta.

---Sí, ya voy, Donna, danos un minuto.

La puerta se abrió. Era la madre de Donna.

---Ah, hola ---dijo. Era una mujer intimidante y, como la mayoría de las
madres, claramente él no le agradaba mucho. ¿Eran imaginaciones suyas o
la mejilla le estaba empezando a doler de nuevo?

A algunas madres se las había ganado por su propio encanto (Ah, Jackie
Tyler, ¿qué estarás haciendo ahora?), o demostrando que la fe de su hija
en él estaba justificada (todavía tienes un buen gancho de derecha
 Francine Jones, bendita seas).

Pero ahora era el turno de Sylvia Noble. Llena de tanto orgullo,
templada con tanta rabia, tanta frustración. Era como si nunca hubiera
sentido el control de su vida, que tanto se esforzaba en decirse que
tenía, y eso le hacía estar realmente enfadada.

Por supuesto, no debía de haber sido fácil perder a Geoff. El Doctor
sólo le había visto una vez, en la boda de Donna, donde parecía ser el
más\ldots{} calmado de los padres Noble. Ahora la pobre Sylvia estaba
haciendo todo lo posible para hacer frente a una hija rebelde, que no
era ni de lejos tan rebelde como Sylvia imaginaba (todavía no tenía ni
idea de adonde llevaba realmente a Donna) y un padre anciano, que estaba
tan decidido a no ser una carga para su hija que se convirtió en una más
grande por defecto. Wilf Mott quería demostrar que era independiente,
fuerte y veinte años más joven de lo que era, creyendo que así aliviaría
la presión a Sylvia, simplemente no se daba cuenta de que Sylvia veía a
través de ello y estaba dos veces más preocupada de lo que estaría si
sólo se sentara en un sillón viendo todo el día ``Countdown''.

¿Cuánto había pasado desde que el Doctor se había sentado a ver
``Countdown''? Solía gustarle ``Countdown''.

---Cuidarás de él.

---¿No vienes con nosotros?

Sylvia miró como si estuviera a punto de decir algo, pero se limitó a
sacudir un poco la cabeza.

---No es cosa mía.

---Pero es tu padre\ldots{}

---Tuvimos una\ldots{} discusión sobre ello.

---Ah ---el Doctor solo podía empezar a imaginar cómo había sido---.
¿Estás segura? Porque estoy seguro de que a él, a Donna y a Netty les
encantaría que tu\ldots{}

Se detuvo. Sylvia estaba a un paso de estremecerse ante la mención de
Henrietta Goodhart.

---Por favor, Doctor, cuida de él ---le dijo con tranquilidad---. No se
está haciendo más joven a pesar de lo que piensa.

El Doctor sonrió de forma encantadora.

---Por supuesto que lo haré.

---No hay ningún ``por supuesto que lo haré'' contigo, Doctor, así que
no me vengas con tu supuesto encanto y palabrería. No te lo estaba
pidiendo, te lo estaba diciendo.

El Doctor trató de no sonreír, era como si estuviera siendo regañado por
una directora. Entonces, el brillo de los ojos de Sylvia le recordó que
eso no era ni remotamente divertido.

---Lo prometo ---dijo, añadiendo mentalmente, ``y escribiré trescientas
veces: No perderé miembros de la familia Noble en Londres''.

---Son todo lo que tengo ---dijo y salió de la habitación.

El Doctor se chupó el dedo índice y lo levantó. Sí, efectivamente la
temperatura ambiente de la habitación había bajado unos cuantos grados.

Media hora más tarde, se amontonaban dentro de un taxi. Wilf y Donna se
sentaron en el asiento, el Doctor en uno de esos asientos de repuesto
abatibles. Se retorció incómodamente en él durante todo el viaje.

Wilf estaba vestido de punta en blanco, corbata de seda, una buena
camisa, chaqueta ligeramente ceñida que probablemente se había comprado
a principios de los años setenta, pero fallaba por el par de deportivas
que llevaba en los pies.

Como si sintiera que el Doctor le estaba mirando, Wilf levantó una cutre
bolsa de plástico. Dentro de ella había un par de zapatos negros de
vestir.

---Me cortan la circulación de los pies, así que los llevaré tan poco
como me sea posible ---explicó.

El Doctor asintió.

---Te ves muy bien ---le dijo a Donna.

---Gracias. Tuve que pedir prestado algo a Veena, que le había prestado
a Mooky, así que ha tenido que pasar esta tarde mientras te estabas
poniéndo el traje y\ldots{} ¿qué?

El Doctor sonrió.

---Tus amigas tienen nombres sorprendentes ---se rió suavemente---.
Mooky.

Donna levantó una ceja ante eso.

---Mira quién habla. Señor Ood. Señor Matrona Cofelia. Señor Ventraxian
Gol-Zeeglar. ¿De dónde te sacas pensar que los nombres de mis amigas
suenan divertidos, eh?

---Bien visto, aunque Ood no es realmente un nombre, es más un tipo de
designación de especie, y yo\ldots{} um\ldots{}

Donna le estaba dando una de sus miradas ``¿te parece que me importe?'',
así que en su lugar se volvió hacia Wilf.

---¿Emocionado?

---Demasiado, Doctor. He descubierto una nueva estrella. Y le han puesto
mi nombre. ¡Es genial!

---Es más que genial, Abuelo ---Donna le apretó la mano---. Es
jodidamente maravilloso. Aquí al Hombre del Espacio, pregúntale si le
han puesto su nombre a una estrella.

---¿Lo han hecho, Doctor?

Si lo habían hecho o no, ni era aquí ni ahora.

---Absolutamente no ---le dijo a Wilf---. Y estoy muy celoso.

---Sí ---Wilf se inclinó hacia adelante para que el conductor no le
oyera---. Sí, pero ¿y tú? Tienes la oportunidad de visitarlas, ¿no?
Tienes la oportunidad de ir hasta allí ---se volvió a Donna y le dio un
codazo juguetonamente---. Aprovéchalo al máximo, mi chica.

---Oh, lo hago, no te preocupes ---dijo.

El Doctor asintió.

---Oh, lo hace, no te preocupes en absoluto.

---Quiero decir que he visto y hecho algunas cosas en mi vida, Doctor,
pero nada comparado a lo que estás mostrando a mi pequeña, ¿eh?

---Ey, no soy solo una pasajera, ¿sabes? ---Donna sonrió---. He tomado
un montón de decisiones sobre a dónde vamos, a quién vemos, la rapidez
con la que tenemos que marcharnos de nuevo porque ha molestado a alguien
que estaba al cargo. Con un ejército. Y una gran hacha. Y doce piernas.

---Diez piernas ---le corrigió el Doctor automáticamente.

---Oooh, de acuerdo entonces. Diez piernas y dos brazos que le colgaban
hasta el suelo. Si eres pedante, que es algo que estas siendo claramente
esta noche.

El Doctor les sonrió a los dos.

---Quizá deberíamos llevarte con nosotros en una excursión de un día,
Wilf.

---¡No!

Tanto Donna como Wilf lo habían dicho a la vez, después se miraron el
uno al otro.

---Es peligroso, Abuelo.

---¡Tú vas!

---Yo\ldots{} puedo cuidar de mi misma. Si te ocurriera algo, ¿qué haría
mamá?

---¿Matarte?

---Bueno, lo mataría ---Donna señaló al Doctor con la cabeza---. Yo iría
después siendo despellejada viva. Probablemente.

---Tu dijiste ``no'' también, Wilf ---dijo el Doctor.

Wilf levantó la vista hacia las estrellas mientras circulaban hacia el
sur de Londres, cruzando el Puente de Vauxhall.

---Me gusta mirar, Doctor. Me gusta mirar e imaginar y soñar. Pero, ¿la
realidad? Todos esos monstruos, armas y esas cosas. Ná, prefiero mis
ideas.

El Doctor asintió.

---Muy sabio. Fíjate, serías una influencia tranquilizadora para ella.

---Oh, lo sé. Ella sigue adelante, ¿no?

---Estoy sentada aquí. Justo aquí ---dijo Donna.

El Doctor seguía hablando con Wilf.

---Y obviamente hay algo en su infancia sobre ciempiés, pero no dice el
qué. Porque el día que fuimos a ese lugar\ldots{}

---¡Ey!

Wilf se rió.

---Tengo que contarte eso. Cuando tenía unos ocho años, su padre y yo la
llevamos a Norfolk. ¿A las Marismas? De todos modos, ella estaba
remando cuando\ldots{}

---¡EY!

Ambos miraron a Donna. Estaba señalándose a sí misma. Con un dedo en
cada mano. A su cabeza.

---Como ya he dicho. Estoy sentada aquí. Escuchando. Y no me gusta lo
que escucho ---los dos hombres se sonrieron uno al otro.

---Más tarde ---dijo el Doctor.

---Más tarde ---confirmó Wilf.

Donna les separó.

---Ya hemos llegado.

El taxi se detuvo frente a la Sociedad, un enorme edificio de ladrillo
rojo, construido a principios de la época victoriana, al igual que el
vestíbulo principal de la estación de Vauxhall.

Donna pagó al conductor (apoquinarás en el camino de vuelta, le dijo al
Doctor). Cuando el taxi se fue rugiendo, se enderezó el vestido,
comprobó sus tacones y le dio un codazo al Doctor, que estaba mirando
hacia el cielo nocturno y las estrellas.

A la nueva estrella que Wilf le había mostrado la noche anterior. Que
ahora era más brillante que antes. Y parecía que había otra pareja de
estrellas que creía que no deberían estar ahí\ldots{}

Donna le dio de nuevo un codazo.

---¿Qué?

Señalo con la cabeza hacia Wilf, que estaba agarrado a un poste de la
luz tratando de quitarse las deportivas y agarrando su bolsa de plástico
al mismo tiempo.

---No puedo agacharme con esta cosa ---siseó---. Si lo rompo, Veena me
pegará la semana que viene. Practica todo tipo de artes marciales.

El Doctor le cogió la bolsa a Wilf y dejó que el anciano se apoyase en
él mientras se quitaba las deportivas y las reemplazaba por los zapatos
de vestir.

---Sylvia me los compró ---le dijo al Doctor---. Estas malditas cosas
son tres tallas más pequeñas.

---No, no lo son ---dijo Donna automáticamente---. No lo estas
intentando.

---Caray, ¿Cuándo te has convertido en tu madre? ---dijo Wilf.

Donna abrió la boca para replicar, pero el Doctor, percibiendo que una
retirada era mejor que un ataque, los cogió a ambos y se puso en medio.

---La cena nos espera ---dijo y entraron en el edificio.

La gran puerta de madera se abrió cuando llegaron al último escalón y un
caballero elegantemente vestido, de unos treinta años, con la piel
olivácea, cabello negro y los ojos brillantes, les saludó.

---Buenas tardes, Señor Mott ---dijo con un ligero acento europeo---.
Señorita Noble. Doctor Smith. Soy Gianni, Relaciones Públicas.

Donna hizo una mueca de sorpresa.

---Lo tiene muy bien ensayado.

---Somos invitados de honor ---dijo Wilf---. Tenía que decir a quién
traía.

Gianni les levó dentro, a una pequeña área en la que una pareja de entre
unos ochenta a doscientos once años estaba apoyada en un bar. O
posiblemente el bar era el que los aguantaba a ellos. Sea como fuere,
parecían formar parte del mobiliario.

Donna arrugó la nariz ante el fuerte olor a whisky y miró detrás de
ellos, donde otra puerta conducía a una zona amplia de comedor y una
algarabía de ruidos.

---Pasen ---dijo el Relaciones Públicas, así que Donna abrió la marcha.

Cuando el trío entró en el comedor, el murmullo se detuvo y fue
sustituido por una ronda de aplausos conducidos por, y Donna se alegró
de comprobarlo, Henrietta Goodhart, resplandeciente con otro extraña
pero inusualmente conjuntada pamela.

Ella caminó hacia ellos, con los brazos extendidos, besó a Donna, luego
al Doctor y finalmente a Wilf, dándoles a cada uno un beso en cada
mejilla, al estilo continental. Entonces le plantó uno rápido en los
labios a Wilf y le guiñó los ojos.

---Hoy estoy bien ---le respondió a la pregunta sin formular.

Un hombre de unos tardíos cincuenta años se acercó, y apretó la mano de
Wilf.

---Crossland. Cedric Crossland. Doctor Cedric Crossland. Doctor Cedric
Crossland, de la Orden del Imperio Británico. Pero me debe llamar Rick,
señor Mott.

---Oh, sólo Wilf está bien ---dijo Wilf, lanzándoles una mirada en busca
de ayuda o rescate a Donna, al Doctor y a Netty.

Donna dio un pase adelante, pero Netty la detuvo.

---No, no, déjale. Es su noche a las verdes y a las maduras y es capaz
de todo y más. Además, por el pudding de chocolate merece la pena haber
venido ---miraron cómo Wilf se relacionaba con todas las celebridades
---. Parece tan feliz ---dijo.

---Entiendo que tú tienes gran parte de la culpa ---dijo el Doctor,
añadiendo---. No es que pretenda entender cosas como esta.

Netty sonrió.

---Por supuesto que no, Doctor. Siendo del espacio exterior.

El Doctor se la quedó mirando boquiabierto, y entonces sonrió.

---De hecho, soy de Nottingham\ldots{}

---Es de Walthamstow---dijo Donna al unísono.

---Nací en Nottingham---dijo el Doctor.

---Pero se crió en Walthamstow---añadió Donna, un poco más tarde.

---Wilf me lo contó todo, Doctor. Sobre ti. Sobre lo del ATMOS. Sobre
dónde está Donna cuando está contigo. No tenemos secretos, ya ves.

El Doctor sacó el aire de sus mejillas:

---Bueno, no estoy seguro de lo que Wilf te ha dicho pero, yo
soy\ldots{} eh\ldots{} bueno\ldots{}

Netty le apretó la mano.

---Está bien. La mayor parte de los días apenas puedo recordar quién
soy, imagínate en qué planeta estáis tú y Donna desde el que mandáis
postales. Vuestro secreto está a salvo conmigo.

---Creo que le voy a matar esta vez---dijo Donna, mirando hacia su
abuelo al que estaban sirviendo un extraordinariamente gran brandy por
un grupo de ancianos y ancianas.

Netty negó con la cabeza.

---Está muy orgulloso de vosotros dos, por favor, no os enfadéis con él.
Además, eso me da la oportunidad de hablaros sobre los Cuerpos del Caos.
Ya sabes, mientras pueda.

Donna frunció el ceño.

---Lo siento, Donna ---dijo Netty---. ¿Te avergüenza que hable de mi
condición? No tienes por qué, no hay nada que tenga que esconder a
nadie. Y mucho menos a mí misma.

---No es eso---dijo Donna---. Es sólo que\ldots{} bueno, es un poco
triste.

---Lo es. Muy triste, créeme. Pero tengo que acostumbrarme a vivir con
ello y disfrutar la mayor parte de los días lúcidos, porque los días en
los que no estoy son más y más frecuentes.

---¿Cómo de frecuentes?

---¡Doctor! No tiene que explicarlo todo.

Pero le mandó callar.

---¿Cómo de frecuentes, Henrietta?

---Si puedo llegar al viernes recordando qué hice el martes, es toda una
victoria.

Gianni estaba a su lado, justo donde un buen Relaciones Públicas tendría
que estar, con las bebidas en una bandeja de plata para ellos y éstos
cogieron los vasos rápidamente, como si intentaran llenar el vacío de la
conversación.

---Así que---dijo Netty---, los Cuerpos del Caos.

---¿Cuándo aparecieron? El primero, quiero decir.

---Ah ---dijo Netty---, has visto los otros. Los acabo de ver yo esta
noche y nadie de aquí parece haberlos mencionado.

---Eso es porque no estaban anoche. O cuando dejamos Chiswick, de hecho.

Netty rió.

---Sé que sabes más sobre el espacio exterior que este puñado de aquí
juntos pero, científicamente, las estrellas no se mueven tan rápido. Y
si pudieran, la devastación sería espectacular.

El Doctor brindó con ella.

---Ah, pues entonces ya no son estrellas. No son estrellas de verdad. Un
pedazo del caos, y aún así, es lo que se puede ver.

---¿Qué son entonces? ---preguntó Donna.

El Doctor se encogió de hombros.

---Tengo una sospecha. La primera, la original, que parece como una
estrella de verdad y es ciertamente una bola de energía combustible
supercaliente que comparte propiedades menores con una estrella, pero
las demás, son como satélites. Pero no de los estelares.

---¿Hechos por el ser humano?

---Bueno, hechos por alguien, sí. Y en algún rincón de mi cabeza hay una
vocecita intentando decirme dónde los he visto antes.

Hubo un ruidito de alguien golpeando un cristal con una cucharita.

---Damas y caballeros, antes de cenar, debería presentarles a nuestro
invitado de honor---decía el Doctor Crossland---. En honor de ser el
primero en ver la nueva estrella M7432·6, oficialmente conocida como
7432MOTT, este es\ldots{} ¡Wilfred Mott!

Hubo un estruendoso aplauso e incluso se escuchó un ``Viva, viva'',
entonces uno de los camareros guió al Doctor, a Donna y a Netty a la
mesa para unirse al resto de los comensales.

Wilf enganchó al Doctor y lo puso entre él y Crossland, mientras Netty
estaba al otro lado de Wilf y Donna al lado de ésta.

El Doctor miró expectante al sitio vacío de su derecha, preguntándose
quién se iba a sentar allí. Era, pensó, un poco como sentarse en un tren
esperando que el asiento vacío a tu lado no fuera a ser ocupado por un
loco con unos altavoces o un niño que pegara patadas o, lo peor de todo,
un malhumorado hombre de negocios que se pasaría el viaje entero
hablando en voz alta por su teléfono móvil. Y, cada vez que colgara,
alguien le volvería a llamar, dejando sonar el molesto tono de llamada
un buen rato hasta responder.

El Doctor siempre se preguntaba últimamente cuándo aquellas pequeñas
cosas triviales habían comenzado a molestarle tanto. Debió de haber sido
al pasar la marca de los 900.

El asiento fue empujado hacia atrás por una mujer cuya ropa podría ser
descrita como excéntrica, o como alocada, con un horrible choque de
colores, estilos y bueno, de todo.

El mayor crimen contra la moda era la blusa que llevaba, la cual tenía
el mapa celeste de Galileo bordado a mano en ella. Era el tipo de ropa
que te pondrías si te vistieras con los ojos cerrados después de que
alguien hubiera cogido tu armario y le hubiera metido una buena
sacudida.

Tampoco es que quien la llevaba pareciera remotamente consciente de
su\ldots{} única y espectacular pieza de ropa de alta costura. Y lo que
era más alarmante, ninguno de los demás miembros pareció
inmutarse\ldots{} solo Wilf y sobre todo Donna reaccionaron, Wilf con
incredulidad y Donna para aguantarse la risa hizo que el encontrar el
vaso de agua delante de ella fuera la cosa más fascinante de la Tierra.

---Ariadne Holt---dijo con un tono que sugería al Doctor que aquello no
era sólo una introducción, sino que era de hecho una completa
explicación para la razón por la que parecía lo que era.

---¿Qué hay? ---dijo él, ofreciéndole la mano. Ella le ofreció la suya
como si le sugiriera darle un beso, o como mínimo, inclinarse
ligeramente. No hizo ninguna de las dos cosas, intentando en cambio,
cambiarlo por un apretón de manos que él había comenzado.

Ella le lanzó una mirada que parecía decir ``Oh, claro, te comportarás
así, ¿verdad?'' y acercó su silla a la mesa y muy ligeramente lejos de
él.

---Así que---dijo Ariadne---. ¿Cuál es su campo?

---El de los diez acres---sonrió él.

Le lanzó una mirada gélida.

---Es una broma interna---balbuceó.

---Una muy mala.

Otra mirada gélida.

---Soy el Doctor, por cierto.

---Lo sé---dijo Ariadne Holt.

---¿Lo sabe?

---Crossland me lo dijo. Me sugirió que me sentara, aquí, a su lado,
para cenar.

---¿Eso hizo?

---Sí.

---Cierto. Bien, lo siento. De hecho, no conozco a este señor Crossland.

---Doctor.

---¿Sí?

---¿Qué?

---¿Qué?

---És el Doctor Crossland, no el ``señor''.

---Ya veo.

---Usted es Smith. John Smith. Usted escribe libros con fotografías
sobre constelaciones, ¿verdad?

---Ah, no. No soy yo. Lo lamento. Aunque cuando era niño pintaba cuadros
con los dedos del cielo nocturno. Solía añadirle pedazos de\ldots{}
bueno, usted lo llamaría pasta para hacer los planetas en tres
dimensiones y purpurina, mucha purpurina. Me gustaba la purpurina.

De nuevo otra mirada gélida.

---No le importan mucho mis dibujos con los dedos, ¿verdad, Ariadne? No
la culpo. No eran demasiado buenos. Una D baja, muy baja. Cuatro de
diez. Eran duros, pero siempre pensé que posiblemente estuvieran
justificados. Así que, ¿qué estudia usted?

Ariadne Holt le ignoró y miró directamente hacia el doctor Crossland.

---No es el correcto Smith, viejo loco---dijo ella---. Este no es el
autor.

Crossland miró al Doctor.

---¿Qué es él entonces?

---Estoy aquí sentado, por cierto---dijo el Doctor.

---¡Ya! ---rió Donna, recordando el viaje en taxi.

---Un loco---respondió Ariadne Holt---. Hablando monótonamente de
pinturas con los dedos y comida italiana.

Donna levantó al Doctor los dos pulgares y le lanzó un guiño de forma
exagerada.

Él le lanzó una mirada que tendría que haberla convertido en hielo. Ella
le sonrió ampliamente.

Crossland miró al Doctor y entonces a Wilf.

---Pensaba que dijo que el chaval era un experto, señor Mott.

Ahora era el turno de Wilf de parecer un conejo atrapado entre dos
focos, porque estaba fuera de su zona de confort. Lo mejor que pudo
inventarse fue un:

---Lo es.

Crossland mostró indignación tosiendo.

---De hecho---dijo el Doctor---. Creo que encontrará que ``el chaval''
es un poco un experto en tus tan conocidos Cuerpos del Caos.

Aquello le llamó la atención.

El Doctor respiró profundamente.

---No es una estrella, ya sabe.

---Sabemos que no lo son---dijo Ariadne.

---Sólo no sabemos qué son en realidad---dijo Crossland.

---Yo lo sé.

Todo el mundo se giró para mirarle.

---Es la encantadora Ariadne la que me dio la pieza final del
rompecabezas con esta\ldots{} única blusa que lleva puesta.

---Adelante---dijo.

Pero el Doctor se detuvo momentáneamente, porque miraba a Henrietta
Goodhart.

Netty no estaba formando parte de la conversación, aunque nadie se había
dado cuenta. En cambio, ella estaba mirando hacia adelante, como si se
hubiera quedado en blanco por un momento.

Donna captó la mirada del Doctor y asintió lentamente con la cabeza, y
el Doctor le lanzó una triste sonrisa. Entonces volvió a captar la
atención hacia él mismo de nuevo, dando a Donna tiempo para levantar a
Netty, brevemente reposando una mano sobre el hombro de Wilf para evitar
que les acompañara. Mientras se alejaban, el Doctor captó la mirada de
Donna y le hizo con la boca un ``gracias''. Entonces resumió su
explicación.

---No es una estrella igual que no es una bola de súper calor de energía
psiónica que actúa como un campo de protección contenedor alrededor de
una inteligencia maligna. Una inteligencia que quiere dominar y
expandirse y sobrevivir. Una cosa que está completamente autorizada para
hacer la suya en su pequeño rincón el espacio dónde normalmente no
podría herir a nadie. Pero cuando se arrastra fuera de su pequeña
dimensión y entra en la nuestra, cuando se arrastra a sí misma a mitad
de camino a través del universo en este planeta, en este tiempo,
entonces es cuando me intereso. Me intereso y me intrigo. Bien, cuando
digo intrigado quiero decir enfadado. Y un poco asustado. Pueden ver que
el mapa estelar de Galileo me ha recordado a cuando me la encontré por
primera vez, alrededor del siglo XV y estúpidamente, me llevé una
esquirla de allí hacia aquí, a la Tierra. A Italia, de hecho. Cerca de
Florencia. Bien, en aquella región. De cualquier manera, unas cuantas
esquirlas se han abierto camino hasta aquí y de nuevo desde entonces
porque están bastante fascinados por la Tierra y este sistema solar.

---Usted está loco---dijo Ariadne con tranquilidad.

---Ha sido lo del siglo XV, ¿no?

Crossland asentía con la cabeza.

---Eso y todo lo demás.

Wilf le tocó el brazo a Crossland.

---Tendría que escucharlo, señor Doctor Crossland de la Orden del
Imperio británico, Sir, porque el Doctor normalmente tiene razón.

---Oh, gracias, Wilf---le dijo el Doctor.

---El placer es mío, Doctor---respondió.

---Así que, Doctor---dijo Crossland---, ¿cómo se llaman estas divertidas
bolas de energía, entonces?

---Oh, eso es sencillo---dijo el Doctor---. Una palabra. Mandrágora.

Cuando el Doctor dijo ``Mandrágora'', el Cuerpo del Caos comenzó a latir
en el espacio. Las otras luces cercanas latieron en una respuesta
rítmica. Y se acercaron, con una misión.

Cuando el Doctor dijo ``Mandrágora'', siete personas estaban sentadas
tensamente en un 4x4 aparcado en el parking del aeropuerto de Heathrow,
como si les acabaran de conectar: los recién casados DiCotta, tres
estudiantes, su profesor y su asistente, recién llegados de Roma, y un
granjero griego, recién llegado de Atenas. El granjero griego estaba en
el asiento del conductor. Giró la llave de encendido y conectó las luces
y el 4x4 comenzó a moverse, con una misión.

Cuando el Doctor dijo ``Mandrágora'', Donna Noble estaba sentada en la
barra exterior de la Real Sociedad Planetaria con Netty, sujetándole la
la mano, cuando vio a Gianni, el Relaciones Públicas, que acababa de
dejar de servir una bebida. Abrió su boca como si estuviera a punto de
hablar. Las palabras eran formadas pero no salía ningún sonido, hasta
que contuvo el aliento y finalmente habló. Pero habló tan flojo que
Donna ni siquiera estaba segura de lo que decía.

Cuando el Doctor dijo ``Mandrágora'', el señor Murakami estaba sentado
en Primera Clase en un avión de aerolíneas japonesas dirigiéndose a
Narita, con la bebida en la mano, con los ojos cerrados, escuchando una
recopilación de canciones de los 60 de Hibari Misora que se había
descargado en su prototipo del M-TEK.

Cuando se dio cuenta de que bajo la música había mensajes subliminales
sobre MorganTech, era demasiado tarde. Algo en su mente estaba gritando,
chillando, dándose exactamente cuenta del increíble ingrediente que
había en el M-TEK, pero también supo que nunca sería capaz de liberarse
y alertar al mundo.

---¿Una bebida, Murakami-San? ---la azafata de vuelo le ofreció una
selección de bebidas o refrescos.

Con una sonrisa optó por un vaso de vino rojo. Se daba golpecitos en los
auriculares.

---Un cantante maravilloso, pero murió joven. Siempre he dicho que es
una tragedia cuando tanta gente tiene que morir con tanto potencial sin
explotar.

Con un gesto de cabeza, la azafata se movió al siguiente pasajero.

El señor Murakami continuó escuchando música, con su mente gradualmente
corrompiéndose por todos los mensajes subliminales que le daban, y no
había nada que pudiera hacer.

Cuando el Doctor dijo la palabra ``Mandrágora'', Madam Delphi creó ondas
llenas de vida y emoción en la suite del ático del Hotel Oracle, en la
Milla Dorada de Brentford.

---¡Dara Morgan! ---exclamó---. ¡Sabe que existo!

Dara Morgan creyó que el ordenador se habría removido emocionado si
hubiera sido posible.

---¿Quién?

---El Doctor. Después de todos estos eones, después de toda esta
distancia, las estrellas estaban alineadas tal y como yo predije. Los
horóscopos de mañana serán ahora muy distintos.

Y en la web de Madam Delphi, leída por gente de todo el mundo, las
predicciones del domingo para cada signo del zodíaco se reescribieron.
Todo lo que ahora decían era ``¡Bienvenidos de nuevo! Vuestra vida
cambiará durante las próximas 48 horas de maneras que nunca podríais
imaginar. Aprovechad este cambio y preparaos para la siguiente y mejor
fase de tu vida mientras Mandrágora se traga los cielos y te sonríe
desde las alturas''.

En quince minutos, un hombre en Ciudad del Cabo había puesto estas
palabras en una camiseta. Una mujer de París estaba creando un grupo de
Facebook para Mandrágora. Y en Milwaukee, tres jóvenes hacían un
graffiti con la palabra ``Mandrágora'' en las paredes de su colegio.

El Doctor se exasperaba con Cedric Crossland, y la otra mujer loca,
Ariadne no sé qué, que se encargaba de dar dolores de cabeza a Wilf, así
que los dejó a todos y se fue a dar una vuelta por la zona de la barra
exterior.

Vio a Donna y a Netty, y supo de inmediato que Netty se había ido, de
vuelta a su propio mundo. Y el corazón de Wilf latió con un poco más de
dificultad porque, aunque la había visto así un montón de veces, cada
vez que lo hacía, se preguntaba a sí mismo: ¿Qué haría si aquella fuera
la última vez? ¿Y si se iba y nunca volvía?

Donna le sonrió mientras se acercaba, con su brazo abrazado fuertemente
alrededor de la anciana que apenas conocía pero que había acogido bajo
su ala porque a su alocado abuelo le gustaba.

Wilf acercó un taburete y se sentó mirándolas.

---Eres muy buena, Donna---dijo---. No tienes porque hacer esto. No es
tu familia.

---Sí, quizá no lo sea, pero es importante para ti. Y supongo que, en el
fondo, también para mamá.

---Oh, ya conoces a tu madre, siempre lloriqueando, siempre quejándose
pero en el fondo de todo hay\ldots{}

---¿Más lloriqueos, más quejas? ---rugió Donna, riéndose---. Cielos, la
quiero, abuelo, pero hay veces que me saca de quicio.

---Verás, no seré yo quien hable de esto, Donna. Tu madre es muy buena.
Habla con sinceridad, pero eso no es malo. Y ella también te quiere.
Ella sencillamente no sabe cómo luchar con muchas cosas desde que
Geoff\ldots{} ya sabes.

---¿Murió? Puedes decirlo.

---Desde que tu padre se fue, sí.

Donna levantó el brazo de Netty y lo alargó para cogerle la mano a su
abuelo.

---Así que, ¿cómo conociste a Netty?

Wilf sonrió.

---Ella me escribió a mí después de que me publicaran una carta en el
Journal.

---Ah, el Journal. ¿Aún funciona?

---Harán sesenta años el año que viene. Este es el Año Internacional de
la Astronomía, así que todo es el doble de importante. ¡Les escribí una
carta sobre la Triple Conjunción! ¡Júpiter y Neptuno! Netty la vio, no
estuvo de acuerdo con mis pensamientos sobre lo difícil que sería verlo
con un telescopio como el mío y ¡BANG!, tuvimos una pequeña guerra.
Entonces un día me telefoneó de la nada, tomamos un café en Londres para
arreglar nuestras diferencias y una semana después hizo servir sus
influencias para hacerme miembro de la Real Sociedad Planetaria. Y aquí
estamos.

Donna le sonrió.

---Así que, ¿cuándo descubriste lo de su Alzheimer?

---Oh, me lo dijo en nuestra segunda\ldots{} en nuestro segundo
encuentro.

---Ibas a decir, ``segunda cita'', ¿verdad? ¡Oh, viejo zorro astuto!
---Donna se le quedó mirando---. Estoy feliz por ti, abuelo. De que
hayas encontrado una amiga con quien te guste estar.

--\ldots{} y seguro que mamá también lo está.

--Oh, lo sé. Solo está preocupada por su enfermedad, y por  toda la
presión que puede provocar eso en mí. Ella y Netty hablaban de
residencias, pero no lo aceptaré.

Ella miró a Henrietta Goodhart.

--Es una gran dama, Donna. Me gustaría que la vieras como yo la veo.

--Lo hago. Ayer en casa y esta mañana. Es encantadora y creo que no la
tendrías que dejar escapar.

Y Wilf se sintió tan y tan triste que dijo:

--Pero un día, la perderé. Es inevitable. Lo busqué en Internet.

--Oh, bueno. Tendrá que ser verdad, entonces.

--De verdad, no es bueno. No quiero decir que se vaya a morir, pero la
perderé un día que se quede atrapada dondequiera que se vaya y no
volverá. No hay medicinas, ni el conocimiento para curarlo. No es justo
--entonces se le ocurrió algo--. Allí fuera, en las estrellas, seguro
que la pueden tratar. Seguro que hay algo\ldots{}

Y Donna le apretó la mano.

--No funciona así, abuelo. Dios sabe todo lo que he visto, toda la gente
que he conocido, ha habido veces que he creído que tiene que haber
soluciones a las enfermedades, al hambre y a todos los tipos de cosas
molestas. Creía que si gritaba al Doctor lo bastante alto encontraría
una manera. Pero no importa si eres de Chiswick o de Cestus Minor, no
hay respuestas fáciles. Sencillamente tenemos que luchar con lo que el
destino nos da\ldots{}

--No es justo --repitió él.

--No. No lo es. Y lo siento mucho, muchísimo por ti, abuelo. Porque te
quiero y porque de verdad me gusta Netty y si pudiera encontrar una
manera de hacerlo todo sencillo para vosotros dos, lo haría, de verdad.
¿Y sabes qué? El Doctor no os conoce tan bien, pero seguro que lo
intentaría diez veces más que yo. Y probablemente no implicaría ninguna
diferencia, así que no hay ninguna razón para darle más vueltas a  algo
que no controlamos.

Wilf miraba a Donna, y se preguntaba qué le había pasado a aquella
estúpida y violenta chica que había querido pero por la que se había
preocupado todos aquellos años. Ahora era una fina, valiente, brillante
y joven mujer. Y la quería aún más. Y entonces estaba Netty. Apartó la
mano de Donna y le cogió las dos a Netty entre las suyas.

--Ey, tú --dijo con calma--. Henrietta Goodhart, creo que es hora de una
canción, como nos gustaba cantar en los viejos tiempos, ¿vale?

Donna, confusa, frunció el cejo, pero él le guiñó un ojo.

--Sé lo que me hago.

Suavemente comenzó a tararear una melodía. Un antiguo himno de góspel.

--Cuando las estrellas comiencen a caer --comenzó a cantar
tranquilamente--\ldots{} ¡Oh, Señor! ¡Qué mañana, oh Señor! ¡Qué
mañana\ldots{}!

Él miró a Donna.

--Ella me dijo que su marido cantaba esto cuando estaba con ella,
durante la guerra.

--Creía que no estuvo casada.

Wilf sonrió amargamente.

--No le digas nunca a nadie lo que te diré ahora, cielo. Se casó.
Durante tres días. Y él fue asesinado en Singapur, cuando comenzó el
bombardeo. Me dijo que cantaron esto en el día de su boda, por el
camino, en un gran Rolls plateado. Tiene una foto de ello y me la
enseñó: era precioso. Y entonces, cuando intentaron huir de Singapur, él
murió cogiéndole de la mano y ella le cantó esto mientras él
moría  --volvió a mirar a Netty--. No le digas nunca a nadie lo que te he
explicado, y mucho menos a ella. Prométemelo. Ella le quería muchísimo y
se juró que nunca se volvería a casar.

--Por supuesto que lo prometo, abuelo. Por supuesto que sí.

Él comenzó de nuevo.

--Oh, pecador, ¿qué harás cuando las estrellas comiencen a caer? Oh,
Señor, ¡qué mañana!

Los ojos de Netty parecieron centrarse, y respiró hondo, como si se
despertara.

--Me he ido, ¿verdad? Oh, no, no me digáis que he estado por las calles
sin llevar nada más que mi ropa interior --miró a Donna y parpadeó--.
¡Otra vez!

Wilf sonrió y una lágrima casi le caía del ojo, así que se la apartó
antes de que ninguna de las dos la pudiera ver.

--Creo que tenemos que volver a la fiesta, para rescatar al Doctor, ¿de
acuerdo?

Netty se levantó y dejó que Wilf fuera primero. Se paró un poco y se
apoyó en Donna.

--Me canso cada vez más --dijo--. Ah, y gracias.

--¿Por qué?

--Se llamaba Richard Phillip Goodhart. Y tu abuelo es la única persona
que he conocido que se le haya acercado más.

Cuando llegaron al vestíbulo principal, la cena ya había terminado. Wilf
y el Doctor se acercaron a la pared más alejada y Wilf se disculpó
porque Ariadne Holt y Cedric Crossland se habían negado a tomarse en
serio al Doctor.

--Me avergüenzo de conocerles.

El Doctor le miró con tristeza.

--No lo estés. Son buena gente. Algunos son un poco extraños, pero de
corazón son maravillosamente normales. ¿Por qué tendrían que creerme?

--Bien, hemos tenido naves espaciales y Sontarans y estas cosas los
últimos años.

--No hay justificación para la habilidad de la humanidad de racionalizar
las cosas, Wilf. De lo que un grupo de gente se asusta, otro grupo no ve
peligro porque está dentro de su zona de confort. Esta gente son
maravillosos pioneros, amantes de las estrellas, de las constelaciones y
observan y anotan y catalogan los cielos. ¡Igual que tú! Nada de esto
tendría que acabar nunca, es demasiado imporante, aún si las cosas
pueden reconocerse durante un par de siglos. Nadie se tomó en serio a
Galileo, Copérnico o el Organon de Aristóteles en sus propios tiempos.

--Te tomas su grosería demasiado bien, Doctor.

El Doctor se encogió de hombros.

--No es personal. La gente como el doctor Crossland no quiere contemplar
las cosas que caen fuera de su esfera de referencia. En los peores de
los casos, imprudentes,  y en los mejores, ignorantes --entonces miró el
vaso de limonada en su mano--. Normalmente.

--Pero eso de la Mandrágora, no es normal, ¿verdad?

El Doctor negó con la cabeza.

--Es una entidad maligna, Wilf. La última vez que estuvo aquí fue con
fuerza, murió mucha gente. Pero intentaba parar el Renacimiento
italiano, para evitar que la ciencia llegara al estado en el que está
hoy en día. No puedo saber qué intentará hacer hoy. Vete cuarenta años
atrás y para el transistor o el microchip y sí, habrás hecho fracasar la
próxima generación del progreso humano. ¿Pero aquí? Nada
particularmente especial pasa en este año, ni siquiera en esta década,
que realmente pueda afectar al futuro de la Tierra. Espérate un par de
siglos o así. Llegaréis a Marte, un par de vuelos espaciales más largos
y\ldots{}

--¿Marte? ¿Llegamos a Marte? ¿Encontraremos marcianos?

--Spoilers --le guiñó el ojo--. Mis labios están sellados --se tragó la
limonada--. Así que no estoy seguro de si dejar sola aquí a Mandrágora
Hélix y asumir que solo está vigilando las cosas, o prepararme para una
gran batalla.

--Quizá tenga algo que ver con esos chicos Carnes, Doctor. Has dicho que
creías que tenían alienígenas en la familia.

--¡Oh, sí! ¿Y cómo es que Joe Carnes sabía mi nombre? --el Doctor
suspiró--. ¡Oh, Wilf, Wilf, Wilf! Acabas de arruinar una noche
perfectamente agradable.

--¿Yo? ¿Cómo? Y, eh, lo siento.

--Porque acabas de ver un pequeño y diminuto, pero que muy diminuto
defecto de mi lógica. Mandrágora, está enlazada, en una forma muy
extraña, con la astrología, no solo con la astronomía.

--La astrología no tiene ningún sentido.

--Bueno, la mayor parte de ello se crea en los periódicos. Pero tiene su
origen en la Época Oscura, así que probablemente haya algo en ello.
Vuelve al nacimiento del Universo y verás que cada sociedad, cada
civilización tiene alguna forma de zodíaco, una creencia en un poder de
luces antiguas enlazadas a algún tipo de sistema de creencias basadas en
el movimiento de los planetas y las estrellas y los cambios de las
constelaciones. Los astrólogos del planeta Hynass juran ciegamente que
no hay nada como la coincidencia y tienen absoluta fe en el conocimiento
de que desde que el Big Ben se divinizó, es cuestión del destino
preestablecido de que nadie puede escapar. Ahora, puedes pensar que no
tiene sentido y yo puedo pensar que tampoco tiene mucho sentido, pero
Mandrágora se aferra a esta creencia, en este sistema sin probar, y lo
usa. Causa y efecto.

Wilf frunció el cejo.

--Pero sigue sin tener sentido.

--¡Oh, sí! Sí, por supuesto que no lo tiene. ¡Sin sentido! Bien,
probablemente. Esto no detiene a Mandrágora de ser capaz de aprovecharse
de estas energías.

Wilf se encogió de hombros.

--Lo que digas, Doctor.

Donna se acercó.

--Abuelo, creo que Netty necesitaría un poco de apoyo en contra de las
opiniones de esa vieja loca bruja sobre el lugar de una mujer en la
sociedad moderna.

Wilf hizo que sí con la cabeza.

--Brindo por ti, Doctor. Espero que te equivoques, por cierto  --y se
alejó.

--¿De qué iba todo esto? ¿Estabas preocupando a mi abuelo?

El Doctor negó con la cabeza.

--No, Donna, no del todo. Me ha hecho pensar sobre la coincidencia y la
casualidad --miró a Netty--. ¿Cómo está?

--No estoy segura. Se ha ido durante un rato pero entonces ha vuelto,
sonriendo y me ha parado un momento.

--Pasa, me temo --dijo, sin dejar de observarla mientras ponía un brazo
alrededor de la cintura de Wilf--. Y no te olvides, ella ya está
acostumbrada.

Donna le cogió de la mano.

--Hay una cosa más. En la barra. ¿Sabes el tipo apuesto que apareció
antes?

--¿Gianni?

--Sí, ese. Iba a decir una cosa sobre alguien.

--¿Quién?

--No lo sé, no estoy segura de que estuviera hablando pero ha parecido
ser algo sobre un hombre   lamiendo un delfín loco.

El Doctor se encogió de hombros.

--Podría ser cualquier cosa. Probablemente haya bebido demasiado.

--O trabajar con esta gente le haya enloquecido --sonrió Donna--. Oh,
bueno. No estoy segura de por qué un hombre lamería un delfín loco, por
cierto.

El Doctor rió.

--Ni yo tampoco. Tendríamos que pensar en irnos pronto, por cierto.

--¿Por qué?

--Tiene algo que ver con una entidad alienígena antigua y peligrosa
suspendida no demasiado lejos de tu planeta que no parece que se quiera
esconder demasiado.

--¿Cómo de peligrosa?

--Bueno, estará esperando algo como un eclipse lunar el cual, mirando a
la Luna esta noche, no parece que sea especialmente inminente.

--Siempre estará la Triple Conjunción.

--¿La qué?

--El abuelo me ha hablado de ella, es por eso por lo que todos están tan
emocionados por el descubrimiento de la nueva estrella. Es el Año
Internacional de la Astronomía, y están todos esperando a ver la primera
triple conjunción entre Júpiter y Neptuno.

Donna estaba bastante orgullosa de haber retenido toda aquella
información, pero el Doctor se dirigía a grandes pasos hacia el doctor
Crossland.

--La triple conjunción --le gritó--, ¿cuándo es?

--¿Perdón?

--¿Este año, verdad? ¿Pero cuándo en este año?

Crossland suspiró.

--Creía que se suponía que eras más listo.

--Lo soy, pero como todas las personas inteligentes, solo puedo aprender
las cosas cuando la gente me da respuestas directas a las preguntas
directas y no sarcasmo.

El doctor Crossland parecía triunfante. Había superado en inteligencia
al Doctor.

--Bien, si supieras tanto sobre astronomía como dices saber, sabrías que
está teniendo lugar ahora mismo. Comenzó hace poco.

--¿Cuando fue visto el primer Cuerpo del Caos?

--Eso supongo, sí.

--¿Y cuándo tiene su punto más álgido?

--Es por eso por lo que estamos celebrando esta cena, Doctor. Los
eventos principales tienen lugar el lunes, sobre las tres en punto de la
tarde, hora local.

--Pareces estar muy engreído, doctor Crossland --dijo el Doctor--, para
ser un hombre que podría estar muerto en cuarenta y ocho horas. Calculo
que en unas cinco o seis horas.

El doctor Crossland frunció el ceño.

--¿Es una amenaza?

--Sí --dijo el Doctor--. No mía, sino que mío es el consejo. La amenaza
es de Mandrágora. De tu Cuerpo del Caos. Está aquí para matarnos a
todos.

Se giró a toda velocidad hacia Wilf, Netty y Ariadne Holt.

--Siento romper la fiesta, Wilf, pero tengo que marcharme. ¿Puedes
llevar a Netty a casa sana y salva, Donna?

--Oh, Doctor, vete, pero no te preocupes por nosotros. Tengo un taxi
alquilado para llevarme a casa a las once --dijo Netty. Tocó el brazo de
Wilf--. Y ni intentes discutir conmigo, Wilfred Mott. Esta es tu noche,
así que no quiero que te sientas responsable de mí.

Wilf la miró y luego al Doctor.

--Wilf no puede marcharse --dijo Ariadne Holt--. Aún no hemos hecho las
presentaciones.

--Sólo nos vamos nosotros --dijo Donna--. El abuelo se queda.

--Al cuerno que lo haré --dijo Wilf--. Voy con vosotros.

--Nadie vendrá conmigo --dijo el Doctor, pero nadie le escuchaba.

Donna se acercó a su abuelo.

--Abuelo, Netty ya ha tenido un\ldots{} hechizo esta noche. Tendrías que
estar con ella. Asegúrate de que vuelve a casa. Ve con ella en el taxi,
entonces quédate en el taxi y vuelve a casa, ¿vale? --buscó algo en su
bolso y sacó tres billetes de diez--. No sé si es bastante, pero tendría
que ayudar.

Wilf rechazó el dinero.

--Puedo pagarlo yo mismo, gracias, cielo.

--Sí, estoy segura de que puedes --dijo Donna--. Pero cógelo, sólo para
que no tenga en mi conciencia que te has quedado en algún sitio y tienes
que sacar a mamá de la cama para ir a buscarte, ¿vale?

Wilf miró a su nieta, entonces al Doctor, que intentaba fingir que había
encontrado algo interesante en el techo. Cogió el dinero.

--Llámame --dijo--. Tengo mi móvil.

--Sé que lo tienes. Y estará apagado o tendrá poca batería, igual que
siempre. Estaremos bien, te veré mañana por la mañana --le dio un beso,
y entonces le dio otro a Netty y le cogió la mano al Doctor--. Venga, es
hora de irnos.

El Doctor se despidió de Netty, de Wilf y de Ariadne Holt mientras Donna
le arrastraba a través de la puerta y de nuevo hacia la entrada, pasando
al portero y de vuelta al exterior, al frío aire nocturno.

--Iba a ir yo solo --protestó, pero Donna ya había parado un taxi,  en
realidad se había puesto en medio de la carretera de South Lambeth y
silbó a uno que escogió la opción correcta entre pararse, ignorarla o
atropellarla.

Donna entró, arrastrando al Doctor con ella.

--¿Hacia dónde? --preguntó el taxista--. Acabo mi turno de aquí a nada,
así que será mejor que no sea lejos.

--A la carretera principal de Chiswick --dijo el Doctor,  añadiendo para
Donna--. Necesito la TARDIS.

El conductor arrancó y condujo hasta el puente de las vías del tren y
fue hacia Nine Elms y el oeste de Londres.

El primer informe llegó a las 23.04, era del observatorio Clemenstry en
Australia occidental. Informó que la nueva estrella, la que había
aparecido en los cielos hacía una semana más o menos, parecía estar
moviéndose en conjunción con otra estrella, la M84628·7.

Lo cual era bastante inusual, declaró el profesor Melville, apuntando en
la pantalla de su ordenador con su pluma. Él estaba en su oficina en la
Antena Copernicus de Essex, pero probablemente querría estar en la sala
de control del radiotelescopio. Normalmente así lo hacía.

--Ese es el problema con estas nuevas estrellas, estos Cuerpos del Caos
--dijo a su joven ``ayudante'', la señorita Oladini--. Son caóticas y no
tienen sentido, científicamente hablando. ¿No estás de acuerdo?

--Exactamente, profesor --dijo ella, sin tener ni idea de lo que
hablaba. Ella sólo estaba allí por un contrato a corto plazo de la
Agencia Lovelace en Brentwood, encontrando lugares de trabajo temporal
para aprender nuevas habilidades. ``Nuevas habilidades''\ldots{} tenía
25 años y aún necesitaba ``nuevas habilidades''. Era vergonzoso de
alguna manera, no estaba ni remotamente interesada en la astronomía pero
no tenía la fuerza suficiente para decírselo a Melville. En su lugar, la
alimentaba y le daba de beber con barritas de chocolate y té y le
escuchaba hablar sobre su gata y su madre. Él vivía con una y hablaba de
haber puesto a la otra a dormir porque tenía mal los riñones, pero la
señorita Oladini aún no estaba totalmente segura de cuándo hablaba de
cuál. Tenía la ligera sospecha, aún así, de que la gata era la que tenía
mejor salud.

El profesor Melville era un anciano muy simpático. Enfatizando
``simpático'' y ``anciano''. Decía que había sido una estrella del pop
en los sesenta, pero no estaba segura de creerlo. A la señorita Oladini
le gustaba, aunque sospechaba que él sólo estaba contratado en la Antena
Copernicus (porque estaba claro que había pasado de la edad de
jubilación) sólo por compasión. Probablemente porque se encargaba de las
guardias nocturnas, era mejor tenerlo en su sitio.

La misma Antena Copernicus era un radiotelescopio construido en los
jardines de una antigua mansión de estilo georgiano que había sido
convertida en las oficinas, salas de reunión y de todo de la Antena. Una
lástima, creía la señorita Oladini. Era una encantadora casa antigua, y
a menudo le gustaba visitar las casas antiguas y aunque mucho había
estado invertido en preservar los elementos fijos y el mobiliario
originales, aquel lugar parecía estéril y faltado de carácter natural. A
menudo se preguntaba quién habría vivido allí hacía centenares de años,
qué les habría pasado y cómo se habrían   sentido al saber que sus
cajoneras, sus cocinas, sus habitaciones y sus salas de baile se habían
convertido en salas llenas de científicos aburridos y administradores.

A las 23.09, el informe había llegado del Observatorio Griffin en
Maryland. También mencionaba un Cuerpo del Caos moviéndose hacia la
alineación con otra estrella. Pero era una distinta a la M84628·7 de
Clemenstry. Aquella era la M97658·3. Lo que era evidentemente absurdo.

--¿Es que han estado todos   bebiendo? --se preguntó Melville.

Aquello le pareció muy buena idea a la señorita Oladini, aunque ella
solo quería irse a su casa y a su cama. No es que estuviera demasiado
dispuesta a pedalear en la oscuridad y, a aquellas horas de la noche,
había mucha gente en las carreteras que eran peores que el miedo.

A las 23.17, el informe llegó del Proyecto Tycho, cerca de Beaconsfield.
El Cuerpo del Caos se acercaba a la M29034·1.

Por supuesto, todo aquello no fue simultáneo, después de todo, no estaba
oscuro en California o en Perth en aquel momento, era sólo que Melville
hacia una guardia nocturna y sólo había encendido su ordenador.

--Todo lo que necesitamos ahora, señorita Oladini, es que Minsk nos
ofrezca alguna cosa ridícula y\ldots{}

Y justo a las 23.19, el Colossus de Minsk dio detalles de cómo el Cuerpo
del Caos estaba en alineación con la M23116·3.

Ahora, Melville estaba atento y lleno de curiosidad. La señorita
Oladini, a pesar de su escaso interés en astronomía, también, debido a
su pasado como estudiante de matemáticas (pasado que le había hecho
acabar ahí), y calculaba cuantas probabilidades existían de que un
cuerpo del caos empezara sorpresivamente a formar una nueva constelación
con cuatro estrellas ya formadas en una misma noche\ldots{} bueno, más
probabilidades que números cabían en su calculadora. Melville colocó las
últimas fotos en la gran pantalla que ocupaba una de las paredes de su
oficina. En verdad era una pantalla enorme, una obra de arte tecnológica
por la cual otros observatorios y radiotelescopios del Reino Unido
habrían dado más de un riñón. Todo lo que tenía que hacer era trazar una
línea invisible desde la pantalla de su ordenador hasta la pantalla en
la pared para que imágenes y   palabras viajaran de una a otra como si de
una película de ciencia ficción se tratase. Melville estaba muy
orgulloso de este software, aunque no tenía ni idea de cómo funcionaba.
Solo sabía que lo hacía y que ello implicaba poder mover imágenes hasta
la gigantesca pantalla sin siquiera levantarse de su silla, justamente
como estaba haciendo ahora.

Primero puso en el centro su foto de la estrella del caos, y a
continuación fue colocando encima la de Clemenstry, la de Griffin, la de
Tycho y, finalmente, la de Colossus.

--- ¿Profesor\ldots{}?

---Lo sé.

---Pero eso es\ldots{}

---Lo sé.

---Quiero decir, ¿cómo\ldots{}?

---  No tengo ni idea.

---Nunca he visto nada igual.

---Lo sé.

Melville agarró el teléfono de su escritorio. Era de color rojo.
Conforme empezó a pulsar números lanzó una mirada a la señorita Oladini.

---  ¿Ha firmado usted el ASO, señorita Oladini?

---  ¿El qué? ---dijo frunciendo el ceño.

---El Acta de Secretos Oficial. ¿Le hicieron firmarla en la agencia
cuando recibió los formularios de empleo para este lugar?

La señorita Oladini pensó por un segundo. Melville la estaba asustando
con la pregunta. Normalmente era un viejo simpático, un poco atolondrado
y algo divagador, e igual fumaba demasiado en su pipa. Pero de repente
se encontraba atento y totalmente entregado, sereno y sin una pizca de
excentricidad. Y fue entonces cuando la señorita Oladini se dio cuenta
de que el torpe, bobo y alocado viejo no era más que un personaje.
Debajo de todo ello, el profesor era listo como un zorro. Quizá sí que
había sido un cantante de música pop.

--- ¿Y bien?

Ella asintió. Había firmado algo en lo que ponía ``oficial'', de eso
estaba segura. Francamente no había prestado mucha atención a los
papeles que la señora Lovelace le había entregado en la agencia. Todos
los trabajadores temporales tenían que firmar papeleo sobre sanidad,
seguridad, renuncia al seguro y ese tipo de cosas cuando conseguían un
destino. Uno más no le habría llamado la atención. Y ahora parecía algo
sumamente importante y peligroso.

--- ¿Por qué?

--- Porque sin su firma a pie de página en dicho formulario, lo que
estoy a punto de hacer y lo que está a punto de escuchar hará que
acabemos ambos en la cárcel durante el resto de nuestras vidas.

Y la señorita Oladini se concentró. Se le daban bien los números.

---Formulario KD62344 ---dijo de pronto---. Lo firmé dos veces y puse
mis iniciales en un recuadro al fondo a la izquierda.

Melville le guiñó el ojo.

---Gracias.

Pulso el último número del teléfono.

---  Por favor, con Aubrey Fairchild. Aquí el Observatorio Copérnico,
código 18. Me llamo Melville.

La señorita Oladini volvió a mirar el collage de imágenes. Tanto el
Cuerpo del Caos como el resto de estrellas carecían de interés de forma
individual, pero ahora que Melville las había juntado formaban una
imagen. Y no un sinsentido abstracto en el que la gente veía dos peces o
un carrito o una montaña rusa. En esta se veía claramente y sin lugar a
duda una cara. Una cara con la mueca de una carcajada.

Un escalofrío recorrió a la señorita Oladini. Estaba claro que esa
carcajada no era de felicidad: era pura maldad.

--- ¿Primer Ministro? Aquí Melville, profesor administrativo en el
Observatorio Copérnico. Le estoy enviando una imagen de código 18
---hubo una pausa---. Sí, señor. No señor, las imágenes solo han sido
combinadas aquí, de momento solo se sabe en el Reino Unido, pero dentro
de unas horas\ldots{}

El profesor lazó una mirada a Oladini.

---No señor, únicamente mi asistente y yo. Permaneceremos aquí hasta
tener noticias suyas, señor Primer Ministro. No, absolutamente.
Aislamiento total, sin enviar o recibir comunicación alguna bajo ninguna
circunstancia en el Observatorio. Buenas noches señor ---dijo Melville
antes de colgar.

--- ¿De verdad era el Primer Ministro?  ---preguntó la señorita Oladini.
El profesor asintió.

---Lo siento querida, pero me parece que va a ser una noche muy larga.
¿Podría comprobar si nos queda té y leche?

La señorita Oladini se dirigió a la puerta cuando, por un instante, miró
atrás, solo para ver como Melville sacaba el móvil del bolsillo de su
chaqueta. Ni siquiera sabía que tenía un móvil.

--- ¿Profesor, no acaba de prometer al señor Fairchild que nos
mantendríamos incomunicados?

Él sonrió.

---Y por ello necesito que compruebe si nos queda algo de té. Si no se
encuentra en la habitación no se le puede achacar ninguna
responsabilidad en caso de que rompa dicha promesa, cometa un acto de
traición y más que probablemente realice un suicidio profesional. Ahora,
por su propio bien, salga de aquí.

Confundida, la señorita Oladini salió de la oficina, pero esperó detrás
de la puerta para intentar oír a quién estaba llamando. Escuchó cómo
marcaba un número y, tras unos momentos que se hicieron eternos, comenzó
a hablar.

---Buenas noches, soy el profesor Melville. ¿Podría hablar con el
Doctor, por favor?

El taxi se encontraba en medio de Prince Albert Drive, entre Vauxhall y
Chiswick, yendo a unos 20 kilómetros hora por culpa de los baches. En la
parte trasera del coche, Donna cogió su móvil y miró fijamente al
número, sin reconocerlo. Pasando por alto ese detalle, respondió.

--- ¿Hola?

Escuchó lo que le decían y se lo pasó rápidamente al Doctor. Este
sonrió.

---He desviado las llamadas de la TARDIS a tu móvil. Debería habértelo
comentado, perdona.

Ella suspiró.

--- ¿Cómo te las apañabas cuando no había móviles? Un tal profesor
Melville, parece ser.

El Doctor sonrió de nuevo.

--- Y he aquí otra coincidencia. Es una de esas noches ¿eh? ---dijo
antes de coger el teléfono---. ¡Ahab! ¿Qué puedo hacer por el
Observatorio Copérnico esta velada? Como si no lo supiera. ¿No tendrá
algo que ver con el Cuerpo del Caos, no?

La noche no dejaba de sorprender a la señorita Oladini. Primero, las
estrellas formando imágenes; luego, el viejo y alocado profesor Melville
llamando al Primer Ministro; y, además, ver por primera vez desde su
llegada que aún quedaba algo de leche. Ah, e incluso una llamada de su
novio cuando faltaba poco para medianoche.

--- ¿Spencer? ¿Qué es lo que quieres?

---  ¿No me digas que no has usado el maldito telescopio gigante ese para
mirar ahí arriba?

Oladini cruzó la puerta de la cocina que llevaba fuera y se asomó
lentamente al frío aire de la noche, sin esperar ver nada a simple
vista. Así que su sorpresa fue mayúscula cuando sí lo hizo.

--- ¿Son fuegos artificiales o algo así? ---preguntó Spencer.

No sabía qué decir. Diez minutos antes no había ni una nube y el
inquietante rostro solo podía verse mediante fotografías. Ahora era
totalmente visible en el firmamento, y si hasta el idiota de su novio se
había percatado --sin achacarlo a contaminación o alguna cosa así--, era
hora de actuar. Recordó el Acta de Secretos Oficial y le dijo a Spencer
que sí, que probablemente se trataría de algo proyectado desde la azotea
de alguno de los cines por Dagenham.

---Publicidad de una peli o algo así ---añadió---. Nos llamaron ayer
para comentárnoslo ---lo cual era una excusa barata, pero pareció dejar
tranquilo a Spencer.

Después de colgar se le pasó por la cabeza que igual el resto del país
no sería tan ingenuo como para creerse algo así, sobre todo porque no
podía ir llamando uno por uno a todo el mundo para informar de ello.

Quizás el Primer Ministro sí podría hacerlo, o Patrick Moore.

--- ¿Cómo va ese té?

La señorita Oladini se giró para ver a Merville en la puerta sonriendo
de oreja a oreja. Le contó que el rostro podía verse ahora con claridad
en el cielo y el profesor se aproximó poco a poco, echando un vistazo.

---No se preocupe ---dijo ---. Ya hay un experto al tanto.

--- ¿De Downing Street?

--- ¡Oh no! Alguien muchísimo más preparado que cualquiera de ahí
---volvió a echar un vistazo por la ventana---.  Realmente sobrecogedor,
¿no le parece? ---dijo mientras el hervidor de agua volvía a ponerse en
ebullición---.  Con leche y sin azúcar, ¿no es así, señorita Oladini?

Ella asintió.

---Ahora no estoy preocupado por nada ---añadió---. El Doctor se ocupará
de todo.

Unas luces aparecieron en el exterior y se escucharon un par de coches
aparcar en el parking de empleados.

--- ¡Qué rapidez! ---murmuró Melville, frunciendo ligeramente el ceño.

--- ¿Los de Downing Street o ese amigo Doctor suyo? ---preguntó la
señorita Oladini intentando no parecer nerviosa---.  ¿O quizás alguien de
por aquí preocupado por esa cosa del cielo?

Pero Melville no estaba prestándole atención. Estaba intentando ver
quién había en el coche. Las luces seguían encendidas, penetrando por la
ventana de la cocina, impidiéndoles ver quién salía del coche y
únicamente escuchando como se abrían y cerraban las puertas. La señorita
Oladini se estremeció. Algo iba\ldots{} mal. Y parecía que Melville
también lo creía así, porque de pronto se lanzó hacia la puerta
intentando echar el cerrojo. Pero era tarde. Alguien derribó la puerta
de un golpetazo. Melville se plantó frente a la señorita Oladini,
preparado para protegerla de quienes quieran que fuesen los recién
llegados. Y eran\ldots{} bueno, eran muchos. De diferentes edades y sin
parecer especialmente\ldots{} gubernamentales. O, al menos, amenazantes.
Pero había algo en ellos, pensó la señorita Oladini, algo en la forma en
la que escudriñaban todos los rincones, la manera tan extraña en que se
movían sus cabezas, como si nunca hubiesen visto una cocina. Un hombre,
viejo y gordo, estaba al frente de todos ellos, con otro de mediana edad
detrás. Más atrás se encontraba una mujer entrada en años y, en la
retaguardia, un trío que por la edad debían ser estudiantes, dos hombres
y una mujer.

--- ¿Dónde están los otros? ---preguntó el primer hombre con un marcado
acento americano.

El profesor Melville se aclaró la garganta.

---Esto es propiedad privada. El Observatorio Copérnico. No podéis
entrar aquí   a estas horas de la madrugada y\ldots{}

La joven se abrió paso entre el resto de sus como si no entendiese
exactamente qué hacía Melville ahí parado.

---Tú\ldots{} ¿trabajas aquí? ---preguntó.

Esta parecía inglesa. Él asintió.

---Estoy al cargo del radiotelescopio, y esta noche estoy al cargo de
todo el proyecto. Debería advertiros que habéis hecho saltar una alarma
y que la policía llegará en cuestión de minutos.

Otro de los estudiantes empezó a hablar. Su voz tenía un acento europeo,
español o, quizás, italiano.

---No detecto ninguna alarma. El humano miente.

¿Humano? Menuda expresión más extraña. Pero al profesor Melville
parecía no perturbarle. De hecho, permanecía impasible.

--- ¿A qué planeta representáis? ¿Tenéis alguna relación con la
Constelación del Caos?

El americano volvió a hablar.

---Tú podrías sernos útil.

Al segundo de decir esto, Melville levantó el dedo señalando a Oladini.

---Y mi asistente. Si queréis mi ayuda, también os hará falta. Todo en
el Copérnico requiere a dos operadores experimentados.

``Cómo una cabra'', pensó la señorita Oladini para sus adentros.

---Es una física altamente cualificada y una experta en el campo de las
ciencias del cosmos.

``Definitivamente loco''.

---Aquí pasa algo ---dijo la mujer, también americana, dando un paso
adelante.

Sin previo aviso, se lanzó sobre el profesor Melville y le puso la mano
en el hombro. Al principió Oladini pensó que ambos se habían
electrocutado o algo así porque parecía que una chispa azulona había
saltado entre ambos. El profesor se tambaleó retorciéndose de dolor.

---  ¿Profesor? ---dijo la señorita Oladini con un hilo de voz, solo para
arrepentirse de haber hablado.

Melville se giró, y en su mirada aún se veían ese azul eléctrico
recorriéndole por dentro.

---De acuerdo, he mentido. Es solo una asistente temporal ---dijo
Melville casi sin poder respirar---. No tiene ni idea de qué es el
Observatorio Copérnico, es inofensiva.

---Podemos absorberla.

---Dejadla marchar, por favor\ldots{} ---suplicó el profesor antes de
caer inerte en el suelo.

Los tres estudiantes empezaron a avanzar hacia ella, pasando por encima
de Melville. Oladini, presa del pánico, hizo lo que ninguna charla en
Brentwood con la señora Lovelace, ningún formulario sobre salud y
seguridad o ninguna charla de 100 palabras por minuto le habían
enseñado: dar media vuelta y correr por su vida.

El taxi acababa de pasar el Centro de Exhibiciones Earls Court cuando el
Doctor aporreó el cristal que separaba a los pasajeros del conductor.

--- ¿Podría pararse un momento, por favor? ---preguntó.

--- ¿Y ahora qué? ---dijo Donna mientras el taxista reducía la
velocidad.

---Pues\ldots{} la verdad, será mejor que vayas para casa, porque le voy
pedir a este buen hombre que me lleve a otra parte.

--- ¿A Chiswick High Road?

El Doctor pensó unos momentos dónde había dejado la TARDIS.

---No, no, mejor un poquito más lejos.

--- ¿Vas a ver al tipo ese de la materia que te ha llamado antes, no?
El del telescopio.   Coponico o algo así, ¿no? ---le recriminó Donna.

--- ¿Entonces dónde, amigo? ---preguntó el taxista, bastante exasperado.

--- A Essex, por la A127.

---  ¡¿A estas horas?! ---exclamó el conductor ---.Vivo en Bounds Green.
Ya les dije que me iba para casa.

---Pues entonces no tardarás tanto en llegar a casa desde ahí, ¿no,
``amigo''? ---  le respondió Donna a mala gana ---.  ¿Cuánto va a costar?
Porque créeme que el Doctor no lleva mucho encima. Y ya que ya he pagado
por mi abuelo, me da que también tendré que pagar esto.

---  100 libras.

---  Vale ---dijo el Doctor.

---  ¿Cómo que vale? ---chilló Donna ---. ¿Desde cuándo pagar 100 libras
``vale''?

El Doctor la ignoró y se dirigió de nuevo al conductor.

---Pero necesito llegar rápido.

---Pues entonces serán 120.

---Lo que tú digas ---dijo el Doctor en un suspiro---.  Te veo mañana
Donna. ¿Puedes pillar otro taxi desde aquí, no? Ah, por cierto\ldots{}

--- ¿Sí?

--- ¿Me das 120 libras?

Por un momento parecía que Donna iba a estrangularle, pero luego se
acercó al conductor.

---A Essex ---le dijo con mala cara---. Y pare en un cajero, haga el
favor.

El Doctor levantó una ceja.

--- ¿Donna?

---Cielo, si te voy a dejar más de cien libras para ir a un flamante
radiotelescopio, será mejor que te acompañe, ¿no? ---le dijo con una
sonrisa.

---Esperaba que dijeras eso ---dijo devolviéndosela.

Y el taxi comenzó su andadura hacia Kensington mientras el Doctor y
Donna hacían planes.

Wilf y Netty estaban despidiéndose, pidiendo disculpas por tener que
irse antes de lo previsto.

---Tranquilos, lo repetiremos pronto ---les dijo Ariadne Holt---, y
haremos la verdadera presentación.

---Siento que tengamos que irnos tan de repente ---mintió Wilf---, pero
ha surgido algo y tengo que ir a casa de la señorita Goodhart. Espero
que me disculpe, por favor.

---No hay nada que disculpar ---respondió Crossland, dándole una palmada
en la espalda.

---Cualquier excusa es buena para organizar una cena, ¿eh? ¿Qué tal en
la reunión de la semana que viene?

Por fuera Wilf intentaba sonreír, pero en sus ojos podía verse otra cosa
bien distinta.

---Me encantaría ---dijo.

Netty se le acercó al oído y le susurró:

---Estarán bien. El Doctor sabe lo que hace.

Wilfred le sonrió, intentando parecer más alegre de lo que realmente
estaba. No dejaba de preguntarse si debería haber ido con ellos. Y se
preguntaba aún más qué diría Sylvia cuando apareciera pasada la
medianoche y sin ellos.

El Observatorio Copérnico estaba totalmente a oscuras cuando el taxi
entró en el parking. Donna pagó al malhumorado taxista, el cual puso
rápidamente dirección a Londres.

--- ¿Podrías decirme cómo piensas volver a casa desde aquí? ---siseó al
Doctor mientras se adentraban en la oscuridad---. Estamos muy lejos.

--- ¿Qué tal caminando?

Donna puso su mano sobre el hombro del Doctor y, mientras se giraba,
ella dio un paso atrás, señalando a su ropa.

---Estás preciosa ---dijo él sonriendo levemente mientras se ponía las
gafas.

---Bueno, gracias Casanova, pero no era eso lo que quería decir. Estoy
emperifollada hasta las cejas con un vestido que no está justamente
pensado para hacer caminatas de 50 kilómetros por Inglaterra ---dijo en
un suspiro ---.  Lo he tenido que pedir prestado, ¿sabes? Venna nunca me
perdonará si se rompe. Y si   a mí nunca me perdona, Dios sabe qué te
hará a ti cuando le diga que fuiste tú quien me obligó a hacer toda la
caminata con él puesto.

---Pues entonces no se lo digas.

Donna se rindió y sacó otro tema de conversación.

---Vale, entonces dime, ¿por qué estamos aquí? ¿Y por qué están todas
las luces apagadas? Si es un observatorio y es de noche, este lugar
debería estar lleno de gente trabajando, ¿no?

---Buena observación, bien pensado. Y por eso hemos aparcado en el
parking, para acercarnos poco a poco en lugar de ir a la entrada
principal.

---Pero pensaba que tu amigo trabajaba aquí.

---Y lo hace.

---  ¿Y entonces por qué no vamos a la entrada principal y decimos ``hola
amigo del Doctor, nos has llamado y ya hemos llegado''?

El Doctor estaba mirando fijamente un coche en el parking para
empleados, aparcado encima del césped.

--- ¿Qué pasa con ese coche? ---dijo Donna escudriñando hacia la
oscuridad.

---  ¿Qué quien lo lleva no lo sabía aparcar?

--- ¿Y?

--- ¿Qué van a destrozar el césped?

--- ¿Y?

--- ¿Y\ldots{} qué se le están apagando las luces?

--- ¿Y?

El Doctor sonrío.

---Piénsalo. ¿Cuántos años crees que tiene ese coche?

---Parece nuevo. Debió haber hecho algún sonido para indicar que se
habían dejado las luces encendidas. Y lo ignoraron. Y si es nuevo la
cosa que hace funcionar a las luces no debería apagarse tan rápido. Mi
padre se dejó las luces encendidas una noche hace unos años y aún
seguían funcionando por la mañana, y el coche iba bien. Y eso que era
viejo.

El Doctor avanzó hacia el coche con su destornillador sónico en la mano
para escanear el vehículo.

---Una enorme pérdida de energía a nivel local ---susurró.

--- ¿Has visto eso?  ---dijo tocando el hombro al Doctor para captar su
atención.

El Doctor miró donde Donna le estaba señalando. En el cielo se podía ver
una cara sonriente hecha de estrellas.

---Las luces de la ciudad deben de haber evitado que lo viéramos
mientras veníamos hacia aquí ---dijo en voz baja.

--- ¿Fuegos artificiales?

---Esperemos que la mayoría de la gente piense eso ---respondió.

---No son fuegos artificiales, entonces.

---No son fuegos artificiales ---confirmó---. Es un Cuerpodel
Caos en funcionamiento.

---¿Qué es un Cuerpo del Caos cuando está en casa?

El Doctor se encogió de hombros.

---Ni idea. Tu abuelo me introdujo el concepto anoche. Pero, ¿una nueva
estrella que puede atraer a otras estrellas por el cielo para formar una
alineación como esa? Bastante caótico, diría yo.

---Yo también ---Donna subió hasta la puerta principal del
Edificio del Observatorio Copernicus---. Bueno, no vamos a aprender
mucho aquí fuera, ¿verdad? ---encontró varios timbres y tocó el que
decía SOLO PARA EMERGENCIAS.

No pasó nada.

---Una pérdida local de energía, supongo ---Donna le sonrió.

El Doctor asintió y la alcanzó, pasando el destornillador sónico
por la puerta y abriéndola de un empujón.

Oscuridad Total.

---Así que, entonces, ¿quién es ese Profesor colega tuyo?

El Doctor estaba usando la luz azul del destornillador sónico a
modo de linterna, tratando de identificar los nombres de las placas.

---Le conocí hace años. Una de las varias personas de este planeta en
las que puedo confiar de tanto en tanto si necesito ayuda profesional. Y
viceversa, pueden llamarme si me necesitan ---señaló hacia arriba---. La
oficina del profesor Melville está un piso por encima, la sala de
control del Observatorio está fuera en la parte de atrás, cruzando los
jardines y girando a la izquierda.

---¿Oficina sosa o telescopio excitante?

El Doctor sonrió, pareciendo un poco espeluznante bajo la luz azul del
destornillador sónico.

---¿Intentas influenciar en mis decisiones, Donna Noble?

---Por telescopio que no, Doctor. Ni en sueños telescopio intentaría
telescopiarte lo que tienes que hacer o adónde ir.

El Doctor se la quedó mirando ante el último comentario y ella rió.
---Bueno, está bien ---dijo---. Pero no en ese sentido.

---Por alguna razón creo que deberíamos investigar el telescopio.

---Es como si Derren Brown estuviera en cada habitación ---se rió Donna.

Siguieron por el pasillo de la puerta principal hasta que
llegaron a una serie de ventanas francesas que daban a un patio. El
Doctor las abrió con el destornillador sónico, y volvieron a salir al
aire frío de la noche.

---Nos están observando ---dijo Donna tras caminar un par de
minutos.

---¿Cómo lo sabes?

---Mi pelo se me riza ---replicó---. Eso y el hecho de que puedo
verles delante de nosotros.

El Doctor se asomó más en la oscuridad.

Había un grupo de hombre y mujeres de pie a la entrada del
Observatorio mismo.

Donna levantó la vista hacia la estructura de frio metal de
escaleras y pasadizos que llevaban hasta una cabina que estaba dominada
por un radar con forma de plato ovalado que formaba el Observatorio.

---Estoy impresionada ---dijo.

---¿Por ellos?

---No, por el Observatorio. Nunca había visto un radiotelescopio
antes, excepto por la tele. Bonito ---de repente le gritó al grupo---.
Qué bonito plato tenéis, gracias por permitirnos venir a verlo. ¿Hay
alguna tienda de regalos? Me encantaría comprarle a mi abuelo una taza
con un dibujo. O un imán para el frigorífico.

---O ---se unió el Doctor---, una servilleta de té. ¿Hacéis servilletas
de té? Todo el mundo hace servilletas de té hoy en día. Nunca he estado
seguro de porqué compraríais una servilleta de té, pero ahí lo tienes.

No hubo respuesta del grupo.

---En realidad estoy buscando al Profesor Melville ---continuó
el Doctor---. ¿Está aquí? ¿Está bien?

Un hombre pequeño con una cicatriz en la mejilla dio un paso
adelante.

---¿Es ese él? ---susurró Donna.

El Doctor negó con la cabeza.

---Nop. Lástima.

---Excelente ---entonó el hombre, con un ligero acento unido a
su por otro lado preciso inglés---. ¿Eres el Doctor?

---¿Qué te hace decir eso?

---Puedo sentir que no eres\ldots{} del todo humano.

---¿Del todo humano? ---el Doctor parecía ofendido---. No soy ni
remotamente humano, gracias. Una especie espantosa. Siempre peleando y
quejándose.

---¡Ey! ---dijo Donna.

---¿Ves lo que digo? Apenas se irguieron por encima del uso de
sonidos guturales y, ¡ostras!, ya tienen eso que llaman música pop.
Quiero decir, sé que en parte me gusta, pero llega un momento en que
realmente no puedes diferenciar a los chicos de las chicas o entender
las letras, ¿verdad? Después está la comida, por supuesto. ¿Has probado
alguna vez una de sus hamburguesas?

El más anciano alzó la mano para detener la cháchara del Doctor,
así que lo hizo, pero sonrió.

---Podría seguir toda la noche, pero sospecho que eso podría aburrirte,
así que sólo diré que en su lugar charlemos sobre por qué os estáis
apoderando de las mentes de estas pobres gentes.

---Es el linaje ---replicó el hombre pequeño.

El Doctor le lanzó una mirada a Donna y ella pudo ver que no era
la respuesta que estaba esperando.

---¿Cómo en la genealogía? ---respondió.

---Cuatro de los humanos aquí presentes se pueden rastrear hacia
atrás en su árbol familiar hasta un lugar específico en el tiempo. Eso
nos da poder sobre ellos.

---¿Y los otros?

---Zanganos. Esclavos. Sirvientes voluntarios.

---¿Voluntarios? ¿De verdad? Encantador. Sin embargo no es totalmente
cierto, ¿no? Pero si eso es lo que quieres creer, es justo ---el Doctor
comenzó a caminar hacia ellos, así que Donna tuvo que seguirle---. De
acuerdo entonces, ¿de dónde sois originalmente? ---señaló hacia la cara
formada en los cielos---. Algo que ver con eso, imagino.

El hombre pequeño de la cicatriz se encogió de hombros.

---Todo a su debido tiempo, Doctor. Madam Delphi necesita hablar
contigo.

Levantó el brazo en la dirección del Doctor y, antes de que Donna
pudiese lanzar ni un grito ahogado, una densa y crepitante ráfaga de luz
rojo-purpura se disparó de sus dedos, dándole de lleno en el pecho y
lanzando su cuerpo hacia atrás un par de metros. Para cuando Donna llegó
hasta él, estaba casi inconsciente.

---Márchate\ldots{} advierte a la gente\ldots{} ---eso fue todo lo que
consiguió decir antes de que sus ojos se pusieran en blanco y el ritmo
subiera y bajara de su pecho quemado, ralentizándose hasta quedar
inconsciente.

Donna sabía que no había nada que pudiera hacer excepto lo que
le había pedido. Tenía que buscar ayuda.

---¿Y la humana? ---preguntó un miembro del grupo.

---Matadla ---respondió el hombre, y Donna vio que todos los
demás levantaban los brazos en un gesto similar.

---Hoy no, gracias ---gritó Donna y volvió corriendo hasta las
ventanas francesas, zigzagueando mientras corría, consciente de que
rayos de energía purpura se estrellaban contra los arboles justo a su
lado.

Estaba casi en las ventanas francesas cuando explotaron en una
lluvia de metal y cristal a su alrededor.

Cubriéndose la cara con los brazos, Donna corrió directa a los
escombros y entró en la oscura mansión. Estaba pasando por el vestíbulo,
cuando vió una enorme escalera de madera que llevaba a los pisos
superiores.

---He visto las películas ---murmuró. Se dirigió a la parte de
atrás de los escalones, donde, estaba segura, había una puerta que
llevaba a un sótano.

La abrió dando un tirón, contó hasta tres y la cerró de un
portazo, haciendo mucho ruido. Después fue de puntillas hasta una de las
oficinas y se deslizó dentro.

Segundosdespués, alertados por su artimaña, un pequeño grupo
de le gente del exterior llegó a la puerta del sótano. Vio como dos de
ellos bajaban las escaleras, dejando a uno de guardia. Maldición, había
esperado que todos bajasen y así pudiese encerrarlos a todos allí abajo.

El que estaba de guardia parecía estar mirando a su alrededor, y por un
segundo Donna pensó que había descubierto donde estaba escondida.
Entonces una mujer a la que no había visto antes se lanzó de repente
contra el hombre, enviándole volando pesadamente a la pared.

La recién llegada incrustó una silla en la manija de la puerta
del sótano, dejando a los dos atrapados dentro como Donna había planeado
y después le gritó directamente.

---¡Vamos!

Donna corrió al encuentro de su nueva aliada y permitió que la
guiase por un pasillo.

---Cocina\ldots{} ---dijo la recién llegada sin aliento---. Por
aquí\ldots{}

Hubo una pequeña explosión y Donna supuso que la pareja se había
abierto camino a disparos fuera del sótano.

---Fue una buena idea mientras duró ---sonrió---. Soy Donna Noble.

---Soy la Señorita Oladini ---dijo la Señorita Oladini
apresuradamente---. Encantada de conocerte. ¿Coche?

---Vinimos en taxi.

---No, ¿quiero decir su coche?

---Vale la pena intentarlo.

Se colaron por la puerta de la pequeña cocina y entraron en el
aparcamiento, directamente en el coche abandonado. Donna se sentó en el
asiento del conductor.

---¿Y las llaves? ---preguntó la Señorita Oladini.

---No hay llaves ---dijo Donna---. Porque la vida nunca es tan
conveniente.

La Señorita Oladini se agachó bajo el salpicadero y sacó de un
tirón algunos cables.

---Una juventud desaprovechada ---dijo.

El motor se puso en marcha lentamente y después murió de nuevo.

---¡Una vez más! ---le instó Donna---. ¡Pero si no funciona,
déjalo porque sabrán donde estamos!

La Señorita Oladini intentó hacer el puente de nuevo, sin
suerte.

Dos de sus perseguidores aparecieron en la puerta de la cocina
con las armas levantadas.

---¡Corre! ---gritó Donna y salió lanzada fuera y lejos del
coche mientras la energía purpura se estampaba contra él y el coche
explotaba.

Donna estaba tirada sobre unos arbustos con el vestido de Veena
hecho jirones.

---Oh, estoy muerta ---murmuró.

No había señal de la Señorita Oladini, pero era difícil ver algo
con el coche en llamas impidiéndole ver. Donna miró a su alrededor y vio
una bicicleta apoyada contra una pared lejana.

---Tiene que ser una broma ---murmuró y después se miró las
ropas---. Quién no se arriesga, no gana.

Esperando que las mismas llamas que le impedían ver a sus
atacantes la mantendrían fuera de su vista, se deslizó por el césped
hacia la pared.

Con una última mirada por si veía a la Señorita Oladini y
dándose cuenta tristemente de que el coche había sido seguramente su
pira funeraria, Donna agarró la bicicleta y se subió a ella. Se tambaleo
ligeramente, pero poco a poco consiguió acostumbrarse a montar otra vez
y después salió disparada camino abajo y hacia la calle principal.

De ninguna manera conseguiría volver a Chiswick en bicicleta, no le
sobraban tres días, pero podría llegar al pueblo más cercano.

Mientras Donna peladeaba furiosamente, el coche en llamas iluminaba la
parte delantera del Observatorio Copernicus, las llamas se reflejaban en
las grandes ventanas de la mansión. Pero no pudo ver ninguna señal de
personas.