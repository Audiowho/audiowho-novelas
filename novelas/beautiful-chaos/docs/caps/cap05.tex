\chapter*{Lunes}
\addcontentsline{toc}{chapter}{Lunes}

El Doctor, Donna, Wilf y sus amigos fueron conducidos hasta la suite del
ático por Caitlin, quien mantuvo la mano apoyada en la culata de un
revólver que llevaba escondido en la cintura de los pantalones. Mientras
la mujer irlandesa abría las puertas del ático, el Doctor entró y miró a
su alrededor. Comenzó a aplaudir lentamente cuando vio lo que había
dentro.

---Madam Delphi, supongo, ---dijo---. Por supuesto. No eres una persona
real, ¿verdad? ¡Eres un ordenador! Bueno, más que un ordenador, diría
una inteligencia artificial, que alberga un antiguo mal que nunca debió
haberse liberado de su dimensión. ¿Cómo estás, Mandragora? Han pasado
siglos.

---Esta\ldots{} forma es, oh, mucho más capaz que un carnoso cuerpo
humano, Doctor ---dijo Madam Delphi---. Como Señor del Tiempo, como
alguien que puede aguantar mucho más que un trauma espacio-temporal, tu
cuerpo es lo que el Doctor manda, y disculpa el terrible juego de
palabras.

El Doctor no dijo nada.

---Ya has oído eso antes, ¿verdad? ---preguntó Madam Delphi.

El Doctor y Donna estaban ahora de pie delante de su grupo exhausto,
Wilf, Netty, Lukas y Joe merodeaban a unos cuantos pasos detrás. Frente
a ellos, en un círculo protector alrededor del ordenador Madam Delphi,
estaban Dara Morgan, Caitlin y los conversos de la Mandragora que habían
caminado hasta allí.

---Oh, hola ---dijo el Doctor, como si se dirigiera a una reunión del
Instituto de la Mujer---. Esto parece bastante impresionante. Bonita
sala. Bonito hotel. Bonito gesto ---señaló hacia la vieja señora
americana que había levantado el brazo en una ahora reconocible posición
Mandragoriana para disparar un rayo letal de energía Helix---. Aunque
algo poco amistoso. ~---Pido disculpas. No te puedes fiar del personal
---la voz femenina del ordenador resonó por los altavoces repartidos por
toda la habitación---. Bienvenidos a mi hotel. ¿Os puedo recomendar el
gimnasio? Tengo entendido que tenemos una gran piscina.

---¿Cómo es el bar? ---preguntó Donna---. Quiero decir, no hay un hotel
de cinco estrellas que se precie sin un buen bar, ¿verdad?

---Ah, Donna Noble, bienvenida tú también. Creo que encontrarás que
tenemos cuatro bares, tres restaurantes, un restaurante a la carta y
servicio de habitaciones las 24 horas del día, 7 días a la semana
---después Madam Delphi se echó a reír---. Debo decir, sin embargo, que
aspiramos a un reconocimiento más grande que sólo cinco estrellas.

El Doctor asintió.

---Bueno, supongo que estás buscando las cinco millones de estrellas.
¿Qué piensas, Donna? ---Deberán tener un buen servicio para conseguir
las cinco millones de estrellas, Doctor. ¿Recuerdas aquel hotel en
Cassius? Aquel sí que era un hotel de cinco estrellas como Dios manda.

---¡Oh, sí! ---El Doctor le sonrió---. Y entendían cómo debía ser una
buena relación con el cliente. ¿Recuerdas cuando tuvimos un pequeño
problema con el lagarto?

---¿Crees que tienes problemas de lagartos en Brentford, Madam Delphi?
---preguntó Donna---. Porque si hay problemas de lagartos que hay que
resolver, no creo que sea un hotel tan bueno. ---El Oráculo es\ldots{}
---comenzó Dara Morgan, pero Madam Delphi le hizo callar.

---El Doctor y su dulce amiga están intentando ganar tiempo, Dara.
Tratando de averiguar cómo detenernos, cómo salir del Oráculo con vida,
la manera de ``ayudar'' a su precioso planeta Tierra ---Madam Delphi
hizo una pausa y después continuó, más dulcemente y por lo tanto un poco
más amenazador---. Pero realmente no vas a detenernos, Doctor. No
ofrezco ninguna garantía de que la gente salga con vida. Y, bajo mi
punto de vista, ayudar a la Tierra es precisamente lo que estamos
haciendo.

El Doctor se dirigió hacia el grupo y se separaron, casi con reverencia,
así que ahora estaba mirando directamente a las pantallas del ordenador.

---La última vez que tuvimos una charla, te envié a la oscuridad,
lamiéndote las heridas. ¿Lo recuerdas?

---Por supuesto ---las suaves ondas de la Madam Delphi pulsaron
ferozmente---. He esperado mucho tiempo para tener la oportunidad de
llegar hasta ti personalmente. Para hacértelo pagar. ---Oh, no, ¿el
viejo truco de vengarse del pobre Señor del Tiempo, Mandragora? Quiero
decir, eres mejor que eso. Venga va, danos una razón mejor. ~

Madam Delphi se rió.

---No es la primera vez desde 1492 que la Mandragora Helix ha estado en
la Tierra, ¿sabes? ---Sí, eso lo sé. La Montaña Sagrada de Xi'an, si
mal no recuerdo. Después estaban los Huérfanos del Futuro, todos con
aquellas capuchas blancas y carmesí. Oh, y lo del club nocturno
Mandrake, ahora eso estaba bastante bien, tengo que admitirlo. Pero cada
vez ha sido sólo un fragmento de la energía Helix, ¿no es así?, una
pequeña chispa para tantear el terreno. Esta vez tenemos toda la
hoguera. Así que, ¿por qué ahora? ¿Por qué enviarme pequeños mensajes
por el papel Psíquico para involucrarme, para atraerme aquí?\ldots{}
ahhh\ldots{} Sí, querías que estuviera aquí. En este día exacto y a esta
hora exacta. ¿Por qué?

---Las estrellas están alineadas ---dijo Dara Morgan.

---Estoy hablando con Madam Delphi, gracias, no con el personal
contratado.

---¿Como te atreves\ldots{}? ---comenzó Dara Morgan.

---¡Oh, cierra el pico! ---espetó el Doctor---. Quiero decir, ¿quién
eres tú, de todas formas? ---Soy Dara Morgan. Fundé MorganTech. Cree el
M-TEK, diseñé\ldots{}~

---Oh, por favor, no hiciste nada que la Mandragora Helix no te dijese
que hicieras. No, ¿quién eres realmente? ¿A quién cogió, deformó,
manípuló y destrozó por completo la Helix antes de que te reinventara
como Dara Morgan?

---¿Qué?

---¿Lukas? ---ladró el Doctor---. Es mi asistente de investigación
---explicó en voz baja a Madam Delphi---. Donna estaba ocupada. Asuntos
de familia.

Donna frunció el ceño. No era porque estuviera consiguiendo que Lukas
Carnes hiciera su investigación, pero, ¿por qué había dicho ``asuntos de
familia''? Lanzó una mirada a Wilf pero éste se encogió de hombros.
Luego se la lanzó a Netty, mirando fijamente a la distancia entre ellos.
Cuando Donna miró de nuevo al Doctor, reconoció una mirada en sus ojos.
Una mirada que, si se le hubiera dado voz, habría sido alguna variación
de ``Lo siento. Lo siento muchísimo''. ---No ---murmuró---. ¡No te
atrevas!

Pero la atención del Doctor volvía a estar en Dara Morgan.

---En todo el mundo, Dara Morgan, miles de millones de personas serán
víctimas de esta consciencia alienígena a la que has dado acceso al
mundo. Y eso va a suceder hoy.

---Lo sé ---sonrió Dara Morgan---. ¡Qué brillante es eso!

---Bueno, brillante desde el punto de vista de tu M-TEK, siendo una
maravilla tecnológica, aumentada por un alienígena que sabe y
distribuida bastante magníficamente por personas que, imagino, no tienen
ni idea de lo que eso haría con ellos.

---Ni idea.

---Hay mucha sangre en tus manos, Dara Morgan. Si fuera policía, haría
que te arrestasen, pero, como ahora explicará Lukas, eso no es posible.

---Dara Morgan saltó a la palestra hace ocho años, haciendo sus primeras
afirmaciones sobre MorganTech en un especial informativo, transmitido en
vivo el 31 de diciembre de 1999. Justo al final del milenio. Antes de
eso, no hay rastro de tal persona. MorganTech fue registrada como una
Sociedad Limitada a las 17:29 de ese mismo día.

---Entonces, ¿quién eras antes de que la Mandragora se apoderase de ti?
¿Antes de volver a reinventarse como un ser humano, convirtiéndose en el
anagrama Dara Morgan?

---Oh, ahora lo entiendo ---dijo Wilf---. Eso es muy inteligente.

---Sí, gracias, abuelo ---silbó Donna---. Pero deja que el Doctor se
centre.

Antes de que nadie pudiera detenerlo, el Doctor puso las manos a cada
lado de la cabeza de Dara Morgan con los dedos presionando contra las
sienes y susurró:~

---Abre las puertas cerradas y déjalo salir.

Los acólitos reunidos dieron un paso adelante hacia el Doctor, y Madam
Delphi pulsó amenazadoramente.

---Detenedlo ---dijo.

Por el ojo de su mente, el Doctor pudo ver una imagen. Una noche oscura,
fría y húmeda. Caminaba por una calle con unos setos altos a ambos
lados, la lluvia le caía por el cuello. Se estremeció. Estaba
enfadado\ldots{} No, enfadado no. Dolido. Desconcertado. Ella había
dicho que no. ¿No a qué? ¿Quién era? En su mano llevaba una caja suave
y aterciopelada. Y dentro de ella, sí, se lo podía imaginar. Una banda
de plata y un diamante plano. Todo lo que había sido capaz de pagar. Y
ella había dicho que no. Había dicho que necesitaba irse lejos de Derry,
que quería ir a Sydney. O a San Diego. O a cualquier otro lugar cerca de
él. ¿Cómo es que se había llevado tan mal? ¿Cómo es que no lo había
visto venir? ¿Cómo era posible amar a alguien tanto, que cada vez que
entraba en la habitación, cada vez que hablaba, sonreía o se reía, su
corazón daba un bote? Que sólo el hecho de saber que ella estaba en la
cocina, en el baño, en el pasillo era suficiente para enviar esos
fantásticos, increíbles y maravillosos pensamientos circulando a través
de él. Aunque cuando lo hizo, cuando le dijo que la amaba, ella le dijo
que quería alejarse. No un ``Yo también te quiero, pero \ldots{}''. No
un ``gracias, pero lo siento''. Sólo un ``Oh dios mío, ¿lo dices en
serio? No, me voy de Irlanda lo antes posible. ¡No quiero estar atada a
nada aquí!''. Era como si alguien le hubiera arrancado todo lo que
importaba y lo hubiera pisoteado.

No eres la primera persona que se ha enamorado y ha sido rechazada, se
dijo a si mismo de manera racional. Pero no quería ser racional. ¿Qué
había de racional en estar enamorado? ¿Qué había de racional el
ofrecerte a ti mismo a alguien sólo para ser destruido? Y allí estaba,
perdido y solo. Todo el mundo le había dicho que no estaba interesada.
Todo el mundo intentaba decirle que estaba perdiendo el tiempo. Pero
cuando estás enamorado, te agarras a cualquier cosa, crees que un día te
levantarás y te dirán ``¿Sabes qué? Me equivocaba y tú tenías razón, tú
eres la persona indicada para mí''. Pero eso no había sucedido. Nunca
sucedió.

En lugar de ello, había visto el relámpago recorrer el cielo de la noche
mientras recorría la carretera, con las lágrimas mezclándose con la
lluvia, pensando en que todo lo que quería hacer era estar en casa en
ese momento. Casa. A diez minutos andando, como máximo.

Más relámpagos. Azul, blanco y púrpura\ldots{}. ¿Púrpura? Golpeó el
suelo justo delante de él, haciéndolo caer hacia atrás. Recordaba ver la
pequeña caja con el anillo desvaneciéndose en una repentina
conflagración, literal y metafóricamente dibujando una línea bajo
aquella parte de su vida. Se sintió como si también estuviera en llamas.
Todo lo que podía ver era una luz púrpura, rodeándolo en ese momento,
destrozando las aceras, destrozando la carretera, la oscuridad y la
lluvia. Y entonces oyó la voz. A su alrededor. Dentro de su cabeza.
Viniendo del cielo y de su corazón al mismo tiempo.

---Es tu momento. Callum Fitzhaugh ya no es relevante. Ahora formas
parte de una causa mayor.

La voz se quedó con él mucho tiempo después de que el fuego púrpura se
hubiese ido, durante días y semanas mientras él voluntariamente se
entregaba a un nuevo propósito. A la mañana siguiente tocó el teclado de
un cajero automático y de él brotaron 200 libras. Ocho cajeros
automáticos más esa mañana. Después más en diferentes ciudades. Después
abrió una cuenta. Manipuló el banco on-line, movimientos que no se
podían rastrear porque introducía cifras en los ordenadores que borraban
los rastros de sus acciones. En tres semanas ya era multimillonario.
Poseía edificios por todo el mundo. Era dueño de compañías que después
cerraba o hacía emerger y, en un mes, MorganTech había surgido gracias a
la manipuladora influencia de la voz dentro de su mente que le decía qué
hacer y cómo hacerlo. Después elaboró el sistema informático que
cambiaría su destino. De alguna manera la voz le guió mientras construía
a Madam Delphi, oía la voz en su cabeza transferirse al hardware, de
alguna manera, creando vida artificial en una escala desconocida hasta
ahora.

---Te necesito ---la voz se había calmado---. Ahora y para siempre.
Necesito una interfaz humana, una conexión con el mundo de la carne y la
sangre. Un avatar en el mundo real. Así había sido creado Dara Morgan.

Recordaba venir de una rica familia de banqueros y comerciantes de
inversión. Sus padres murieron en un accidente de un avión privado, y
MorganTech habían pasado a él cuando tenía sólo 21 años. Recordó más
falsos recuerdos, eventos, personas, calificaciones y fiestas. Ninguno
de ellos era real, pero cada vez que se imaginaba parte de su historia
ficticia, se volvía realidad. La voz le mostró cómo una sociedad que
confiaba en los ordenadores, que ya no usaba papel y tinta para mantener
registros, podía ser fácilmente manipulada en aceptar que la historia,
las mentiras, la invención que contabas a través del teclado eran
ciertas. Recordó la voz que le decía cómo desarrollar el M-TEK en pocos
años, así que el mercado podría confiar en él. Confiaría en MorganTech.
Aquel era una jugada larga. Y recordaba haberla visto en una calle de
Dubai una tarde.

Estaba con un par de hombres, inspeccionando un fajo de documentos en un
café de carretera.

Había escuchado mientras los hombres explicaban que necesitaban pensar
sobre el trato que fuera que estuvieran hablando y se fueron. Y entonces
fue a sentarse a su lado. Ella levantó la mirada, al principio
intrigada, después sorprendida y después asustada. Al poco tiempo
encontró la voz.

---¿Cal?

---Ahora no ---dijo---. Soy Dara Morgan.

Ella se rió, una magnífica risa hermosa y suave, que le trajo todo ese
amor que había sentido años atrás. Pero la voz en su cabeza siseó: ``No.
Recuerda el anillo. Recuerda las lágrimas y el dolor. No te rindas
ahora, Dara Morgan''.

---Te pareces a él, a Cal ---dijo ella---. ¿Qué te trae a Dubai?

---La Mandragora se tragará el cielo ---dijo---. Te lo voy a enseñar,
Cait.

Le cogió la mano, y sus ojos brillaron con la energía púrpura de la
Mandragora. Después ella había abierto las carpetas de las que había
estado hablando con los hombres de negocios momentos antes.

---Firme aquí por favor, Señor Morgan.

Y lo hizo, porque la voz se lo dijo. En una hora, MorganTech era dueña
de una cadena de hoteles de cinco estrellas en todo el mundo, y Caitlin
se había convertido en su primer convertido.

Con un resoplido, el Doctor se alejó de Dara Morgan, quien de inmediato
se desplomó en el suelo. Todo había durado menos de un segundo en tiempo
real, pero, para el Doctor, le había parecido eterno. Se apartó de Dara
Morgan mientras el resto del grupo influenciado por la Mandrágora se
acercaba, con los brazos levantados, listos para disparar el rayo
mortal.

---No ---las ondas sinuosas de Madam Delphi daban saltos arriba y abajo
en sus pantallas---. No, necesito ese cuerpo. Es lo que he estado
esperando estos largos siglos, a que el Doctor se presentase. ¡El último
de los Señores del Tiempo, poseído por la energía de la Mandragora
Helix, animado por mí!

Los discípulos bajaron los brazos. Y el pequeño Joe Carnes se apartó de
su hermano y corrió hacia el Doctor.

---¡No! ---gritó---. Déjalo en paz.

A los pocos segundos Lukas estaba a su lado, y después Donna y Wilf
también estaban. Se interpusieron entre él y el ordenador poseído por la
Mandrágora.

---Sí, gracias a todos ---dijo el Doctor---. Pero no es realmente
necesario ---sonrió a Madam Delphi---. Qué montón de opciones. Niños que
nadie se tomaría en serio, un viejo con una afección cardíaca que caería
muerto en un segundo, su amiga Henrietta, una experta en estrellas.

Lanzó una mirada a todos ellos, una mirada sólo observada por Donna.
Henrietta Goodhart todavía estaba en la puerta, como si tratara de dar
un sentido a lo que estaba pasando. El Doctor estaba mirandola con una
mezcla de tristeza y\ldots{} ¿qué era eso?, Donna se preguntó. ¿Pánico?
¿Desesperación? Como si estuviera deseando que ella dijese o hiciese
algo Pero no era bueno. Netty no estaba con ellos en ese momento.

``Las luces estaban encendidas, pero no hay nadie conduciendo''. El tipo
de cosas que Donna imaginaría que su madre diría. Una frase terrible,
pero una con la que Donna no podría estar en desacuerdo en ese momento.
Y era como si el Doctor pensara que Netty le había fallado, de alguna
manera.

---Donna ---murmuró el Doctor---. Tú móvil. Ahora.

Lo puso en su mano, y manteniendo un ojo en Madam Delphi, se desplazó
hábilmente por su libreta de contactos.

---¿Donna?

---Sí.

---¿Por qué no está aquí el número de tu abuelo?

---Porque nunca enciende el maldito móvil. ¿Cuál es el problema?

---Oh, genial. Gracias.

---¿Por qué quieres llamarle? Está aquí de pie.

---Su teléfono está en Essex. Tengo que llamarle.

Donna cerró los ojos, imaginandose los dedos en el teclado y le susurró
los números Mientras ella decía cada número, él presionaba la tecla.
Cuando oyó que comenzaba la llamada, colgó.

---Espero que lo hayas hecho bien, porque sino\ldots{}

---¿Alguien acaba de recibir una llamada extraña?

---Y el mundo terminará. Pero ey, ha sido divertido ---le pasó el
teléfono móvil.

---No lo conseguirá ---estaba diciendo su abuelo al ordenador---. Este
hombre es brillante, ha salvado este planeta, el universo entero,
probablemente, más veces de las que hemos tenido cenas calientes.
¡Tendrás que pasar por encima de nosotros para cogerle!

Dios bendiga al abuelo, pero Donna dudaba seriamente que pudiera detener
a Madam Delphi. El Doctor necesitaba algo de Netty, Donna estaba segura
de eso. Así que necesitaba ganar tiempo.

---¿Quieres un cuerpo en el que habitar y que haya estado dando vueltas
por la galaxia, señora? ---dijo ella---, toma el mío. Oh, puedo no tener
dos corazones o un pelo que desafíe a la moda, pero este cuerpo ha visto
algo de acción del espacio exterior ---empujó al Doctor por detrás de
ellos, acercándolo a Netty.

Las pantallas de Madam Delphi vibraron de nuevo.

---Noble de nombre, noble de naturaleza, ¿verdad?

---Oh, como si no hubiera oído eso antes. Una noche, cuando Neal Bailey
decidió ponerse juguetón en el Odeon, murmuró en mi oído: ``Hay grietas
en una tarta noble, ¿qué tal una buena noche, dulce Donna?''. Le solté
uno donde duele y me fui. Eso sí, reconozco que conocía a Shakespeare y
debería de haberse anotado un tanto por la originalidad. Mi padre no
estaba de acuerdo y la semana siguiente se topó con su nariz en el pub
---Donna sonrió dulcemente al ordenador---. ¿Alguna vez te ha entrado un
tio? No, por supuesto que no, porque eres toda eléctrica y cables y
esas cosas. Completamente sola, ¿no es así? Eso es por lo que estás
haciendo todo esto, ¿verdad? ¿Buscando el amor? Deberías de haber ido
a Lonely Hearts, en lugar de la astrología.

Wilf tiró del brazo de su nieta.

---Harás que se enfade.

---¿En serio abuelo? Nunca se me hubiera ocurrido ---le hizo un
guiño---. Sé lo que estoy haciendo.

Madam Delphi latía furiosamente

---Me gustaría saber porque el Doctor siempre se rodeó de estúpidos
seres humanos. Quiero decir, ¿para qué servís? Aparte de cómo chivos
expiatorios. ¿Cuántos han viajado en su TARDIS antes que tú, Donna
Noble? ¿Y qué les sucedió? Quiero decir, crees que vas a viajar con él
para siempre. ¿Crees que eres la primera en creer eso? Por supuesto que
no. Pero tú estás aquí y ellos no. Entonces, ¿me pregunto qué pasó con
todos ellos?

Donna no iba a dejar que esto le afectara, principalmente porque esa era
una pregunta que le había preguntado antes al Doctor y había estado más
que satisfecha con su respuesta. Pero claramente tocó la fibra de su
abuelo.

---Cariño, esa es una buena pregunta.

---Realmente, no lo es ahora ---le espetó.

--- Donna, ¿crees que hay un cementerio con lápidas, todas alineadas con
sus nombres? ---dijo Madam Delphi---. ¿Tiene una parcela de tierra
reservada para ti?

---Tal vez ---respondió Donna---. Para ser honesta, no me importa mucho.
Vivo el aquí y el ahora. Y aquí mismo, ahora mismo, lo único que me
preocupa es detenerte a ti y a tu pequeño ejercito de zombis.

---Destruidlos ---dijo Madam Delphi, tan natural, tan casual, que le
llevó a Donna un segundo el darse cuenta. Pero se dio cuenta cuando los
discípulos, al unísono, alzaron sus brazos listos para disparar sus
rayos de energía. Nada ocurrió.

---¡Destruidlos! ---gritó el ordenador.

Siguió sin ocurrir nada.

---Destruidles ---exigió Madam Delphi, pero los discípulos no hicieron
más que fruncir el ceño y mirarse los unos a los otros sorprendidos. Era
como si hubieran despertado de un sueño.

---Ah ---dijo el Doctor---, eso será cosa mía. Bueno, en realidad, si
soy honesto, sería una encantadora mujer llamada Señorita Oladini, nunca
me quedé con su nombre, que maleducado. De todos modos, acaba de tocar
tu alineamiento un poco, canceló todo el poder que tienes sobre los
descendientes de San Martino. Cerrado, kaput.

---Finito ---dijo Donna con acento italiano.

---¡Y eso no es todo! ---Donna miró a su izquierda. Dara Morgan estaba
de pie a un lado, con un portátil en sus manos y sus dedos volando sobre
las teclas mientras escribía con una mano---. He enviado una señal de
cancelación a través de la red a todos los M-TEK. En cuanto se
sincronicen con los ordenadores, en lugar de descargar tus ordenes
instalaran un virus que desfragmentará la plataforma y borrará por
completo sus memorias ---Dara Morgan tecleó por última vez la tecla de
retorno. Y lo tengo protegido por contraseña.

---Soy un superordenador megalómano, vinculado a miles de millones de
puntos de venta electrónicos en todo el mundo, tonto hombrecillo. ¿De
verdad crees que me has detenido? Me decepcionas, Dara Morgan.

Dara Morgan se encogió de hombros.

---¿Detenerte de una vez por todas? Lo dudo, pero ciertamente te he
retrasado, así que la señal no será activada en 10 minutos.
Probablemente ni siquiera en unos cuantos días, tiempo de sobra para que
el Doctor te detenga ---Dara Morgan sonrió---. Y mi nombre es Callum
Fitzhaugh ---un profundo suspiro electrónico salió de Madam Delphi.

---¿Caitlin? ---y la chica irlandesa, la amada de Callum que le había
rechazado hace casi diez años alcanzó el revolver de su pretina y lo
alzo.

---Caitlin, no ---gritó Callum---. Lucha contra la influencia de
Mandrágora. ¡Recuerda quien eres! ---Caitlin frunció el ceño.

---¿Cal?

---Sí, ¡soy yo!

Caitlin se encogió de hombros.

---Nunca me gustaste entonces, tampoco me gustas mucho ahora ---y
disparó una bala que pasó por el cerebro de Callum Fitzhaugh y salió por
el otro lado. Estaba muerto antes de tocar el alfombrado suelo.

Los recién despertados discípulos aullaron y gritaron confusos y
comenzaron a salir de la habitación.

---Id con ellos ---susurró Donna a los Carnes---. Fuera de aquí. Lukas,
lleva a Joe a casa. No dejéis de correr hasta que lleguéis allí ---se
volvió a Wilf---. Tú también.

---Olvídate de eso, Donna. Estoy demasiado mayor como para correr y
estoy aquí contigo hasta el final. Le dije a tu padre que cuidaría de
ti, y por Dios que lo haré.

Se le ocurrió a Donna que la tal Caitlin podría haber abierto fuego, así
que miró para ver lo que estaba haciendo. Había puesto la pistola en el
escritorio y estaba sentada frente a las pantallas de Madam Delphi. El
Doctor pasó junto a Donna, casualmente dejando a Netty en los brazos de
Wilf, murmurando.

---Sostenla Wilf. Como si tu vida dependiera de ello ---entonces se
agacho junto a Caitlin, buscando el arma con su mano.

---Tómala ---dijo en voz baja---. Callum y yo ya hemos causado
suficiente daño para justificar lo que he hecho.

---Estabas bajo el control de Mandrágora ---dijo el Doctor.

---Me liberé, te liberó ---y Caitlin le miró a los ojos, con una lágrima
rodando por su mejilla---. Madam Delphi nunca me controló. Mandrágora
nunca me controló, no lo necesitaba.

---¿Entonces quién te dijo que dispararas a Dara Morgan o como se
llamara? ---preguntó Donna.

---Su mente ha estado\ldots{} patinando durante dias. Estaba empezando a
recordar cosas\ldots{} era una conexión débil. Tuve que eliminarlo.

---¿Tenías que qué? ¿Por qué? ¡Podría haber salvado a la raza humana!
¿No es eso por lo que ha sido todo esto?

Caitlin miró repentinamente al Doctor a los ojos.

---No lo sé ---dijo, una lágrima comenzó a aflorar---. ¿En qué me he
convertido? ¿Qué me ha hecho trabajar para esta cosa? Acabo de matar a
alguien. Oh, Dios mío. Le disparé sin pensar.

---Un poco tarde para lágrimas, chata ---dijo Donna---. Trabajando con
Mandrágora, seguramente hayas matado a mucha gente.

---Lo sé ---dijo Caitlin en voz baja---. Estaba fuera de control, con
hambre de\ldots{} poder. Quería control sobre mi vida.

---Yo controlo todo ---Madam Delphi latió de nuevo---. ¡Incluyendo tu
vida!

---No, no lo haces, estúpida caja de cables. Elegí esta vida porque
creía que la quería. ¿Pero sabes qué?, me equivoqué ---sus dedos volaban
por le teclado---. Estoy apagando el wi-fi, levantando los firewalls.

---Eso no me detendrá.

---No, pero te aislará ---Caitlin miró con tristeza al Doctor---. He
hecho mi parte Míster Señor del Tiempo. Ahora todo depende de ti. Echó
su silla hacia atrás y chocó con el Doctor. Disculpándose le rodeó y se
acercó al cadáver de Callum---. El mundo podría haber sido nuestro
---dijo mientras se arrodillaba a su lado.

El Doctor trató de dotar de sentido a las palabras de Caitlin.

Wi-fi. Firewalls. Cosas inútiles, Madam Delphi era un ordenador mucho
más potente que eso. Tocó el teclado y una ráfaga de energía purpura de
Mandrágora casi le arranca los dedos.

---No te pongas gruñona.

---Aún te destruiré, Doctor. Serás\ldots{} ---y se silenció.

Entonces vio lo que realmente había hecho Caitlin. Había dicho
chorradas, sabiendo que Madam Delphi gastaría algunas subrutinas
rastreando lo que clamaba haber hecho. Después de haber encontrado los
firewalls y el wi-fi, el ordenador estaba buscando en otro lado. La
matendría en silencio y ocupada durante\ldots{} bueno, no mucho,
francamente. Pero no estaba realizando un cálculo, podía verlo en una de
las pantallas, era como un nimi-virus, una ecuación matemática auto
replicante que estaba usando bytes pasando, tratando de resolver la
ecuación, pero, realmente, la multiplicaba. El Doctor sonrió. Caitlin
era buena en lo que hacía, incluso si sólo le tomaría a Madam Delphi un
par de segundos para contrarrestarlo. Miró al cuerpo de Callum,
esperando ver a Caitlin. El cuerpo estaba solo. El Doctor se palmeó los
bolsillos. El revólver seguía allí. Pero algo más no.

---Donna ---susurró---. Donna, quiero que bajes al hall. Toda esa gente
estará confusa, desorientada.

Por lo que sabemos, puede que la mitad ni siquiera hablen inglés.
Necesitarán a alguien frío y racional para aclararse, explicarles las
cosas.

---Pero como nadie que encaja en esa descripción está disponible ---dijo
Donna--- yo tendré que valer.

El Doctor le sonrió.

---Oh Donna, eres la mejor que hay. Ahora, ve. No, tú no, Wilf. Tu y
Netty quedaos aquí.

---¿Por qué no pueden venir conmigo? ---preguntó bruscamente Donna.

---Cosas de casa ---dijo el Doctor---. Confía en mí, ambos bajarán sanos
y salvos conmigo en unos minutos.

Wilf se acercó a la placa

---Vamos, Donna, no discutas con el hombre. ¿Cuándo te ha decepcionado?

Donna se fue.

---Puede que no haya sido la mejor elección de palabras, Wilfred.

---Doctor, ¿alguna vez la has decepcionado?

---Bueno\ldots{} ---consideró el Doctor---. No, pero en realidad he
estado cerca una o dos veces.

---Porque si alguna vez piensas en decepcionar a mi pequeña, tendrás que
responder ante mi ---sus ojos se encontraron, a través de la habitación
y, durante una milésima de segundo, un fragmento de eternidad, el Doctor
supo que nunca, jamás debía de decepcionar a Donna Noble.

---No lo haré ---dijo---. De hecho, Wilf, debería decir que no lo
haremos porque ahora mismo eres importante. Para Donna. Para mí. Para
todo el mundo. Y, sobre todo, para Henrietta Goodhart ---se levantó
repentinamente---. No lo hagas, Caitlin ---Wilf se dio cuenta de la
chica irlandesa estaba en la pared, junto a la caja de conexiones,
sosteniendo una pluma plateada. Con una punta azul que brillaba. ---¿Qué
trama?

---No sabes como usarlo ---dijo lentamente el Doctor---, y Madam DElphi
va a estar en funcionamiento en un segundo. Te detendrá.

---Que lo intente ---dijo Caitlin---. Y tienes razón, pero creo que si
aprieto esto, giro aquello y lo meto todo aquí\ldots{}

---¡Cait, no! ---era demasiado tarde. A medida que el destornillador
sónico gritó con repentino poder, demasiado poder, mal manejado, usado
por manos inexpertas, Caitlin lo introdujo en la, ahora expuesta, caja
de conexiones, y justo en los cables de fibra óptica que Johnnie Bates
había muerto juntandolos solo unos pocos días antes. Hubo un destello de
fuego púrpura y Caitlin había desaparecido, reducida a átomos con un
fragmento de la pared, el cableado y el destornillador sónico del
Doctor.

---Buena idea ---dijo el Doctor tristemente---, pero tenía que haber una
forma mejor.

---Se ha apagado el ordenador ---dijo Wilf.

El Doctor miró a las pantallas, y se lanzó a examinar el servidor.

---Más muerto que muerto.

---¿Ganamos?

---Oh, para nada ---el Doctor miró a Wilf---. Mentí a Donna ---confesó.

---Lo sé ---dijo Wilf---. Pero te aseguraste de que estuviera a salvo.
Gracias.

---Caitlin cortó la energía de Mandrágora del ordenador. Para todos los
efectos, Madam Delphi se ha ido. Borrada. Destruida.

---Pero esa cosa de la energía de Mandrágora, sigue ahí, ¿verdad?

---Atrapada.

---¿Dónde?

---En esta sala. Y en este momento, está en busca de un nuevo hogar.
Creo que tenemos unos tres minutos.

---No me va a elegir, por eso es por lo que me mantuviste aquí. Escuché
lo que dijiste. Tengo una enfermedad del corazón. Voy a morir.

---¿Qué? ---el Doctor frunció el cejo, entonces recordó lo que le había
dicho antes a Madam Delphi---. Wilfred, no tengo ni idea de la condición
de tu corazón. Por lo que sé, podrías tener al menos un par más décadas.
Estaba diciendo que debido a que, bueno, no importa ahora ---miró el
fragmento desaparecido de pared donde Caitlin había estado---. Bien,
dudo que pueda traer de vuelta a los muertos, así que espero que escoja
el objetivo más sencillo, el camino de menor resistencia ---Wilf siguió
la mirada del Doctor a Netty, que estaba a su lado sonriendo
serenamente.

---No\ldots{}

---Es la nave más probable

Wilf temblaba de tristeza.

---Pero es mi Netty. Íbamos a ver el mundo, en un crucero, ir a América
del Sur, Canadá, el Océano Índico. Doctor, es mi vida. Nunca pensé que
nadie podría sustituir a mi esposa, que en paz descanse, pero llegó
Netty Goodhart y me mostró que hay más en la vida que sentarse en un
terreno y escuchar a Dusty Springfield. No puedo perderla. No puedo
perder otra mujer en mi vida. ¡La amo!

---Sé que lo haces, y realmente siento mucho tener que preguntarle esto,
pero tengo que hacerlo.

---No le puedes preguntar, está\ldots{} ahora está apagada. Es la
enfermedad, la demencia. No puede hablar por sí misma ---el Doctor buscó
en su bolsillo y sacó un pedazo de papel.

Wilf miró.

``Mi querido Wilfred.

Una vez me dijiste que confiaste al Doctor la vida de Donna. Ahora le
confío la mía. No sé lo que va a hacer, ni en qué estado estaré cuando
lo haga, pero confío en él, eso es suficiente para mí.

HG''

---Es tu papel especial ---dijo Wilf---. Me muestra lo que quiero ver.
Hace siglos Donna me habló de él.

El Doctor pronunció un silencioso ``oh muchas gracias, Donna'', entonces
sacó su cartera de cuero con el auténtico papel psíquico.

---No, Wilfred, la carta es auténtica. Cuando estábamos en lo de las
hamburguesas le expliqué a Netty lo que podía que necesitáramos. Por qué
era un blanco potencial y un\ldots{} ---el Doctor se detuvo, y Wilf vio
momentáneamente un destello purpura por sus ojos. Entonces, los cerró y
los abrió de nuevo. Marrones. Como siempre. El Doctor sopló---. Eso no
ha sido divertido, pero no lo intentará conmigo de nuevo.

---No lo necesita ---dijo Henrietta Goodhart, en silencio pero con un
tono amenazador familiar---. Tengo un nuevo hogar. Un nuevo cuerpo. Uno
que se puede mover, y hablar, y sentir.

---Fuera de mi amiga ---espetó Wilf.

Netty sólo se rió.

---Pobre patético hombre. Ahora esta es mi nave. Mandrágora vive. Traeré
destrucción a este mundo, tendré mi venganza. Todo este cosmos caerán en
ruinas y caos y me alimentaré de él durante siglos. ¡Hermoso caos! .

Wilf dio un paso hacia Netty, pero el Doctor tiró de él, dio una
sacudida casi imperceptible de la cabeza.

---Espera ---el Doctor miró a la pared quemada donde había muerto
Caitlin. Luego, al ordenador, inútil y muerto. Al cuerpo de Callum
Fitzhaugh, que una vez consumido por la rabia y desesperación, había
ayudado a una plaga universal que estaba a punto de destruir la Tierra
si se le daba la oportunidad. Y el cuerpo de Henrietta Goodhart,
habitado ahora por un poder alien tan grande, que había sobrevivido
desde la Edad Media y ahora estaba listo para causar una ola de
aniquilación por las galaxias conocidas.

A menos que hubiera hecho los cálculos correctamente.

Netty caminó por la habitación, como si no pasara nada, una mujer
fuerte, apta a finales de sus sesenta, que debería de haber estado
caminando por los valles de Yorkshire o bronceandose en un crucero por
el Caribe, con Wilfred Mott como su compañero, a su lado. En cambio,
tosió. Se tambaleó. Wilf fue a ayudar, pero el Doctor le dio un tirón.

---Lo siento, no puedo imaginar lo difícil que es para ti ---le dijo el
Doctor---, pero tienes que dejar que siga.

Netty, o más bien la fuerza alien que actualmente habitaba su mente y
cuerpo, le sonrió. Era una sonrisa que a ninguno de los dos le gustó
mucho. Retorcía la cara de Netty de una manera que demostró que esta no
era para nada Netty.

---Gracias, Doctor ---se regodeó---. Me has dado un nuevo aliciente en
la vida. Siempre supe que el ordenador era un medio para un fin, que un
día el huésped adecuado aparecería. Habiendo estudiado este ridículo
planeta, hubiera preferido que fuera un cuerpo masculino, joven y sexy.
Como un deportista. Pero bueno, por ahora me valen las ancianitas.
Cuando ésta se queme, me iré a otro.

---¿Quemado? ---Wilf miró al Doctor, pero la mirada del Señor del Tiempo
estaba firmemente fijada en Netty. Si era por otro motivo aparte de no
ser capaz de tratar con el mirada acusadora de Wilf, no tenía ni idea.

---Oh, ¿no te comentó tu colega alien sobre este lado de las cosas? Me
pregunto si se lo dijo a Henrietta Goodhart cuando hizo el trato con
ella. Verás, Wilf, ¿Te puedo llamar Wilf, no? Netty está terriblemente
encariñada contigo y creo que facilitará las cosas si nos comunicamos de
forma casual.

---El señor Mott es terriblemente formal .

La sonrisa de rictus se amplió aún más.

---De todos modos, el cuerpo humano sólo puede soportar la energía
Mandrágora durante un corto periodo de tiempo antes de que se evapore y
tenga que encontrar un nuevo depósito. En última instancia, será el
Doctor.

---¿Lo será? Oh, que alegría.

---Sabes que lo serás.

---Pero primero necesitas debilitar mis defensas.

---Derribame, quiebra mi espíritu. ¿Cómo vas a hacer eso? ---Netty rió,
era un sonido completamente desprovisto de calor o alegría genuina.

---Destruyendo a todos los que conoces.

De repente parecía inestable, y se acercó a Wilf en busca de apoyo.
Estaba a punto de suministrarlo cuando el Doctor saltó del otro lado de
la habitación, golpeando el brazo de Wilf.

---¡Ey!

---Oh, no empieces ---murmuró el Doctor---. Déja que se sostenga ella
sola.

Netty estaba firme de nuevo.

---Empezando con este anciano con mal corazón.

Wilf estaba a punto de hablar, y entonces se dio cuenta de por qué el
Doctor se había inventado lo de su corazón. Hubiera querido tomar el
cuerpo de Netty, no el de Wilf. ¿Por qué?

---Erase una vez, Doctor, vimos este mundo como una amenaza, tenía tanto
potencial. Tratamos de detener su desarrollo, reteniendo sus primeras
ciencias. ¡Pero míralo ahora! Nos equivocamos, deberíamos haberlo
alentado. Tiene la capacidad de comunicarse. Con un pequeño virus
informático, Mandragora puede tocar el mundo. Hay casi siete mil
millones de personas en la Tierra, Doctor. En un par de años, dos mil
millones de hogares tendrán un ordenador, accedidos por una media de
tres personas. Añade a eso la infiltración de Mandrágora en los puestos
de trabajo, y le llevará a Mandrágora menos de una hora dominar
efectivamente a la mayoría de la gente en este planeta, usar tecnología
humana para expandir Mandrágora a través de la galaxia mucho más
eficiente de lo que puede hacerlo solo. En veinte años, podría tener a
la humanidad construyendo granjas en Marte. Dentro de cien años,
podríamos colonizar Alpha Centauri. Un nuevo imperio Mandrágora,
combinando la energía de la Hélix, la calidad física humana y la ciencia
de las comunicaciones. Y entonces\ldots{} entonces\ldots{} ---Es
impresionante, te lo concedo. ¿Entonces qué? ¿Que hace cualquier
especie?

---Florece, domina\ldots{} más o menos\ldots{} sigue adelante.

---Oh, ``más o menos sigue adelante'' eso es muy científico ---el Doctor
se sentó en la silla frente a la ahora inútil instalación de la Señora
Delphi---. ¿Y? Quiero saber más de tus planes. Wilfred está desesperado
por saber. Y yo también. Y Donna, que está fuera de la puerta, hola
Donna, entra. Estoy seguro de que ella también quiere saber ---Donna
apareció.

---Creía que sería de más ayuda aquí arriba. Todos esos bichos raros,
les llevé al restaurante del personal, les dije que volvería en un
minuto.

---Umm, Donna ---le advirtió el Doctor--- ¿que hay de impedir que
huyan?

Donna sostuvo una pequeña llave de plata frente a su nariz.

---Porque soy brillante y los encerré ---sonrió---. Oh, pero envié a los
Carnes a casa.

---Bien, bien. Mandragora me estaba contando cómo él/ella/eso va a crear
todo un nuevo imperio, como el Imperio Bizantino, o, um, ¿cual era el
otro? ¿Griego? No. Oh, ¿cual era? Donna abrió la boca para sugerir
`romano' pero el Doctor la detuvo.

---Vamos, vamos, Donna, deja que Mandragora lo averigüe. Vamos, tienes
toda la información a tu alcance, ¿en qué imperio estoy pensando? Algo
que ver con una decadencia y una caída. Netty/Mandragora se detuvo.

---Romano. El Imperio Romano.

---Oh, sí, muy bien. ¿Cuántos miles de millones de ordenadores hay en la
Tierra? ¿Por cierto, cuál era el mercado de infiltración del M-TEK?

Netty/Mandragora frunció el ceño y se giró hacia el abuelo de Donna.

---¿Wilf? Ayúdame\ldots{}

---¿Netty\ldots{}?

---No, Wilfred ---dijo el Doctor, su voz de repente pareció un
disparo---. Siéntate, y deja que Netty lo resuelva. ¡Ahora! ---y Wilf se
instaló junto a Donna en el suelo---. ¿De dónde es la Mandrágora?

Netty/Mandragora sonrió.

---Una nebulosa, Doctor. Escapamos de los Tiempos Oscuros y creamos un
nuevo hogar en el corazón del hermoso caos.

---Por supuesto que sí. ¿Cómo se llamaba?

---El\ldots{} era\ldots{} No puedo recordarlo\ldots{}
---Netty/Mandragora se tambaleó ligeramente---. No podemos
recordar\ldots{}

---Vamos, centráte ---gritó el Doctor repentinamente, y estaba de pie,
rodeando a Netty/Mandragora, disparando preguntas mientras caminaba
alrededor---.

¿Dime la velocidad de la luz? ¿Cuántos años duró la guerra
Carrionite-Eternos? ¿Dónde está el planeta natal de los Judoon? ¿Cuál
es la Vigésima Tercera Convención de la Proclamación de la Sombras?
¿Quién ganó la Guerra Sendrome/Bendrome? ¿Cuántas son cinco habas?
Vamos, vamos\ldots{} ---chasqueó los dedos con impaciencia---. Quiero
decir, no puedes tomar el universo si apenas puedes pensar, ¿verdad?

---Dame un momento\ldots{} ---escupió Netty/Mandragora.

---¿Darte un momento? Bueno, podría darte un momento, supongo, quiero
decir, le daría a Henrietta Goodhart un momento o dos felizmente, porque
ella no está bien, ¿o lo está? Oh, ¿no te diste cuenta? ¿Acaso no te
diste cuesta de eso? ---y el Doctor agarró bruscamente los hombros de
Netty y le dio la vuelta para que estuvieran cara a cara, con las
narices casi tocándose---. Has tomado el cuerpo de una mujer bastante
increíble, Mandrágora, y te has encerrado en su mente, expandiéndote en
su sinapsis. El problema es que sus sinapsis están fallando. Cada día
las neuronas y las sinapsis en su corteza cerebral se están atrofiando.
Y tú estás acelerando el proceso en tu prisa por aclimatarte y, por
desgracia, ya te estás rompiendo. Quiero decir, no puedes pensar
palabras, ese es un toque de parafasia. ¿No te puedes mantener en pie?
Bueno, creo que eso es apraxia. Lóbulo temporal, lóbulo parietal,
decayendo. ---Tú\ldots{} hiciste algo\ldots{} engaño\ldots{} me
engañaste\ldots{}

---Bueno\ldots{} sí, creo que lo hice. Y cuanto más luchas, cuanta más
energía de Mandrágora utilizas tratando de reparar esas parte de
cerebro, más te pierdes, porque esta mujer tiene Alzheimer y eso no es
curable, ni siquiera por ti.

---Entonces deberé, ya sabes, cambiar\ldots{} mover, cambiar cuerpos
con\ldots{} Debo\ldots{}

---Sí. ¿Qué? ¿Qué harás? Vamos, dime.

Donna se unió.

---O podrías contarnos todo sobre las estrellas, todas esas maravillosas
constelaciones, todas las cosas que Netty conoce. Dinos, ¿dónde está la
Estrella del Perro? O cómo encontrar la Osa Mayor. ¿En qué dirección
veré a Venus en en esta época del año?

Wilf agarró el brazo de Donna.

---Para, Donna, la estás confundiendo.

---Esa es la idea ---le dijo---. Eso por lo que el Doctor está haciendo
esto, acelerar la confusión. ---Pero es Netty\ldots{} ¡estás hiriendo a
Netty! ---Wilf no se podía mover. Sabía que Donna y el Doctor sabían lo
que estaban haciendo, pero eso no le impidió querer que se detuvieran.
Querer hacer otra cosa que no fuera usar y abusar de Netty de esta
forma. Pero no lo hizo, porque, como una vez había leído en una columna
de consejos del periódico `a veces el mayor bien sale del pequeño de los
dolores'. En realidad, probablemente dijera que `la vida está llena de
duros golpes', pero así era como había decidido interpretarlo. Lo
odiaba. Por un pequeño segundo, casi odiaba al Doctor y a Donna por
hacerlo tan\ldots{} fácilmente. Y entonces tomó una decisión.

---Cuando las estrellas comienzan a caer ---comenzó a cantar--- Oh
Señor, ¡qué mañana! Oh Señor, qué mañana\ldots{} ---su voz se quebró
ligeramente, así que se aclaró la garganta u comenzó de nuevo---. Cuando
las estrellas comienzan a caer, oh Señor, ¡Qué mañana! Oh Señor, ¡qué
mañana! Cuando las estrellas comienzan a caer\ldots{} ---suavemente, en
voz baja, Netty respondió---. Oh pecador, ¿qué harás? Cuando las
estrellas comiencen a caer\ldots{} Oh Señor, Qué mañana\ldots{} ---Wilf
extendió y le tomó la mano, tratando de esconder sus lágrimas mientras
lo hacía. Esta vez, el Doctor no le apartó. Wilf comenzó, lentamente a
bailar con Netty, mientras los dos cantaban en voz baja la canción que
ella tanto amaba. El Doctor calmó a Donna. ---Ahora depende de tu abuelo
---susurró.

---¿Recuerdas cuando escuchamos por primera vez esta canción? ---le dijo
Wilf a Netty mientras bailaban---. ¿Quién la cantó? ¿Cuál era el nombre
del hombre que nos trajo la cena? ¿Puedes recordar el coche, todo
plateado y brillante? ¿Y fue esa la primera vez que habías estado en un
Rolls Royce? ¿Y lo que dije al final, mientras íbamos por las calles,
mirando hacia arriba a esa claro y precioso cielo azul? ¿Y
puedes\ldots{}

---No\ldots{} No puedo\ldots{} No puedo recordar\ldots{} Soy
Mandr\ldots{} Mandrágora \ldots{} Gobernaré el universo\ldots{} de
alguna forma\ldots{} No puedo\ldots{}

---Eres Henrietta Goodhart ---dijo Wilf suavemente---. Y estás enferma,
terriblemente enferma, y estoy tan asustado por que te voy a perder y no
quiero perderte. Por favor, quédate.

---¿Wilfred? ---el Doctor y Donna inmediatamente reaccionaron cuando
Netty dijo su nombre---.

Tú eres Wilfred\ldots{} yo soy Netty\ldots{} no, soy la Mandrágora
Hélix, y yo\ldots{} Oh Señor, qué mañana\ldots{} ---Wilf la abrazó
fuertemente contra él, y la abrazó más de lo que había abrazado a nadie
desde su querida Eileen había fallecido---. Oh pecador, ¿qué harás?
---cantó.

---Mi cabeza\ldots{} No lo entiendo\ldots{} No puedo recordar
nada\ldots{} ---Netty lo apartó---. ¿Por qué no puedo recordar? No es
justo\ldots{} ¡No es justo! No puedo recordar nada\ldots{} ¡¡no es
justo!! ---la cabeza de Netty cayó hacia atrás y miró hacia el techo,
con ambas manos señalando la dirección en la que estaba mirando. Los
otros observaban mientras luces púrpuras, azules, y rojas rugían de su
cuerpo, vaporizando completamente el espacio en el techo mientras la
energía de Mandrágora emanaba de la forma humana de Netty y surcaba
hacia el cielo.

Entonces, el ruido y la luz se detuvieron. Mandrágora se había ido.

Las manos de Netty cayeron sin fuerzas a sus lados, y su cabeza colgaba
hacia delante.

Wilf fue a cogerla, pero Netty se sacudió y alzó la vista hacia él, con
una enorme sonrisa de reconocimiento en su rostro.

---¿Wilfred? ---miró alrededor el ático del hotel, vio el agujero
perfecto en el techo, entonces vio al Doctor y Donna.

---Caray O'Reilly ---dijo---. ¿He estado en Sundowning de nuevo? ¿Dónde
he estado vagando esta vez?

---Usted, Henrietta Goodhart ---sonrió el Doctor--- acaba de salvar el
planeta Tierra. Es brillante.

Donna le dio un codazo.

---Sí. Y tu no eres tan inútil, hombre del espacio.

Un centenar de millas. Un millar de millas. Un millón de millas.
Moviéndose casi a la velocidad del pensamiento, el roto e incorpóreo
Mandrágora Hélix disparó a través del universo, gritando en el interior,
con su mente cayéndose, tratando de encontrar el camino a casa.

Pero ¿dónde estaba casa? Sin duda, estaba\ldots{} No, estaba\ldots{}
¿Dónde estaba casa? ¿Dónde estaba aquí? ¿Quién soy? ¿Qué soy? ¿Por
qué soy? ¿Quién\ldots{} qué\ldots{} dónde\ldots{} cómo\ldots{} Pienso,
luego existo\ldots{} Pienso, luego\ldots{} Pienso\ldots{} Yo\ldots{}
¿Qué es ``Yo''? Nada. ``Yo'' es nada\ldots{} Yo\ldots{} Yo\ldots{}
yo\ldots{}

En la cafetería, Donna se había hecho cargo, y pronto había hecho que
todo el mundo se relajara, cosa que era un alivio para el Doctor, ella
era mucho mejor con las cosas sentimentales humanas que él. Lo más
importante en este momento, era que ella podía mentir de forma mucho más
convincente, decirles que todo había sido parte de un virus informático
enviado desde MorganTech y que podrían volver a casa tan pronto como se
hiciera de día.

El Doctor había hecho un par de llamadas rápidas a gente en las altas
esferas (o quizás en las bajas) y anunció que alguien llegaría muy
pronto para darles a todos billetes de avión y reservas de primera clase
para donde sea que quisieran ir.

---Esto es Inglaterra ---murmuró el anciano americano---. Siempre quise
venir a Inglaterra. ¿Cómo diablos llegué aquí?

El Doctor no podía responder a eso, pero en su lugar les dio largas a él
a su encantadora mujer diciendo que había dispuesto su estancia en un
hotel (no este, gracias a Dios), en el centro, y que tenían siete días
para explorar la cuidad.

---Tomen el tren, visiten Bath, o Warwick o la Isla de Wight.

---O Hull --- añadió Donna.

---Donna, ¿por qué querrían ir a Hull? ¿Qué hay en Hull que podrían
querer ver?

---No sé ---dijo---. Nunca he estado en Hull. Pero siempre he pensado
que sonaba interesante. ---Hull es precioso ---se unió Wilf---. Fuimos
una vez durante un largo fin de semana para ver un partido. Dimos un
paseo en barco.

El Doctor cedió.

---Muy bien ---dijo a los americanos---. Vayan también a Hull. Tienen
barcos. Al parecer.

La anciana pareja se fue murmurando sobre Hull, y el Doctor volcó su
atención sobre los estudiantes. Tres chicos y una chica.

---¿Qué le ha pasado al profesor? ---preguntó la chica.

Los dos chicos en la parte posterior (oh, una pareja, Donna decidió)
asintieron, pero el otro parecía abatido.

---Ha muerto, ¿no es así?

El Doctor asintió.

---¿Italiano? Sí, lo siento ---todos los estudiantes se miraron el uno
al otro.

---Ya no le veo sentido regresar a Italia.

---Yo sí ---dijo el chico más pequeño en la parte de atrás, mirándose
mutuamente. Ahhh, pensó Donna.

---Al menos deberíamos de atar los cabos sueltos por aquí ---dijo el
tercer chico.

Y el grupo se alejó murmurando. El hombre griego se disculpó, diciendo
que no tenía ningún recuerdo de lo que había hecho, pero supuso que no
había sido bueno. El Doctor le explicó que no era culpa suya y que
debería de volver con su familia y olvidarse de Londres. El hombre se
alejó, murmurando.

---¿Supongamos que hizo algo en Grecia.? ¿Supongamos que alguno lo
hizo? ¿O todos? ---dijo Donna.

---No puedo arreglar todo, Donna ---suspiró---. Sólo podemos esperar que
todo lo que haya sucedido en sus pasados, en todo caso, puedan lidiar
con ello. Y es poco probable que recuerden.

---Estás pensando en tu amigo en Copérnico, ¿verdad? Uno de ellos lo
mató. Le rompió el cuello. Parece que fue nuestro amigo griego, pero no
soy policía. Y no puedo probar nada ---Donna reflexionó sobre la
moralidad de ello cuando el móvil pitó. Un mensaje.

---La señorita Oladini ---Donna agitó el móvil. Leyó el texto.

---¿Está bien? ---preguntó el Doctor.

---Está eufórica. ¿Qué hiciste?

---No sé lo que quieres decir.

---¿Doctor?

---Bueno, quizás mientras estaba apañando las cosas con UNIT puede que
haya mencionado lo mucho que estábamos en deuda con ella.

---Aquí dice ---Donna sonrió---, que su estado de visitante ha sido
mejorado y que ahora puede ir y venir como le plazca. Sin esconcerse.
Ah, y también me ha dicho que te diga que tiene un gato llamado Dolly, y
tu sabrías lo que significa ---el Doctor sonrió.

---Bien por las dos.

---¿Creía que no te gustaban mucho los gatos?

---Siempre me gustó Dolly. Y se merece un buen hogar.

---¿Doctor? ¿Cuántas otras personas usó Madam Delphi y luego se deshizo
de ellas?

---La humanidad solo era una herramienta para la Hélix, herramientas
para ser usadas y abandonadas.

---¿Como Netty? ---el Doctor se estremeció visiblemente---. Lo siento
---dijo Donna---. Ese ha sido un golpe bajo ---el Doctor miró a su
amiga.

---Pero cierto y honesto. Tuve que arriesgarme, Donna. Por una vez,
podría haberlo hecho con menos consciencia.

---Dios mío ---dijo Donna con fingido horror---. ¿Qué te he hecho? ---el
Doctor estaba serio. Puso sus manos entre las suyas.

---Me has hecho una persona mejor ---Donna apartó sus manos,
recurriendo, como siempre, a sus bromas estándar.

---No toques lo que no te puedas permitir, hombre del espacio.

Observaron como Wilf y Netty empezaron a caminar hacia la zona de
recepción

---Vayamos con tu madre, ¿eh?

Donna asintió.

---¿Entonces, tu también vienes? Quiero decir, ya sabes cómo es.

El Doctor asintió.

---Sí. Una versión más antigua de su hija.

---¡Ey! ---Donna rió y cogió al Doctor del brazo---. Vamos, hombre del
espacio. Te has enfrentado a Sontarans, Pyroviles y el Gyojin de
Kandalinga. Realmente no creo que mi madre de tanto miedo.

---¿No lo crees?

---Nah. A menos que sea lunes. Los lunes se convierte en una pesadilla.
¿Hoy es lunes? ---Ciertamente, hoy es lunes.

Donna le cogió un poco más fuerte.

---Entonces es mi turno de protegerte, ¿eh?