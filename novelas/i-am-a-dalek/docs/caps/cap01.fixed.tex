\chapter*{Capítulo Uno}
\addcontentsline{toc}{chapter}{Capítulo Uno}

Rose comprobó el seguro de su casco espacial, entonces miró a través de
los controles de la TARDIS hacia el Doctor.

---Desactivando el aire ---dijo, con su mano enguantada apretando uno de
los controles en el panel. Su voz le llegaba a Rose a través de una
conexión de radio aplicada a sus cascos---. Apagando la gravedad ---apretó
otro botón y le sonrió. Entonces recordó algo---. Oh, y la presión de
equilibrio ---añadió, bajando una palanca---. Porque no queremos arder
espontáneamente. ¡Vamos para arriba, Mary Poppins!

Rose sintió cómo el peso abandonaba su cuerpo y alargó las manos para
mantenerse en pie al borde de su lado del panel.

---No me lo puedo creer ---dijo. Lanzó una mirada hacia las puertas de la
cabina de policía, imaginándose lo que le esperaba en el exterior---, que
vaya a pasear por la Luna.

---Más que pasear, pegar botes ---dijo el Doctor alegremente. Para
demostrárselo, dio un paso adelante y se dejó llevar a sí mismo por el
vacío, aterrizando con la gracia de un bailarín de ballet a unos buenos
dos metros más allá---. Práctica, ¡venga! ---le dijo a Rose---. ¡No querrás
caerte de culo ahí fuera! ¡Salta!

Rose se soltó del panel y siguió su ejemplo, recordando empujarse con
gentileza, y aterrizando solo un poco menos expertamente justo a su
lado.

---¡Salto gigante! ¡Y salta! ---la animó el Doctor, y eso hicieron,
flotando y rebotando juntos por la TARDIS.

Rose agarró uno de los montantes de las paredes, pegó un empujón con las
piernas e hizo una voltereta perfecta, observando la gran sala dando
vueltas a su alrededor. El Doctor le sonrió.

---¿Lo has pillado? Bien ---agarró una larga vara blanca y una vieja
bolsa maltrecha que había atado a una de las placas del suelo antes de
desconectar la gravedad. De la bolsa sacó una larga sucesión de las
banderas de todas las naciones puestas una tras otra.

---El pedazo en el que hemos aterrizado no va a ser explorado durante
unos cuantos siglos, así que démosles una gran sorpresa cuando lleguen
aquí ---miró la secuencia de banderas, observándolas y se detuvo ante una
bandera verde y azul con una gruesa línea negra y amarilla en el
medio---. ¿Tanzania? ---dijo, con picardía. Entonces sus ojos se abrieron
de par en par en la siguiente bandera, que mostraba una cimera y las
iniciales IM---. ¡No, va a tener que ser esta! Instituto de Mujeres ---se
quedó pensativo unos segundos---. No podemos ---entonces volvió a sonreír
de nuevo y ató la bandera a la vara---. ¡Sí podemos! ¿O es que acaso
estos pies no se pasearon por las verdes montañas de la Luna? Esto
mantendrá a unos cuantos historiadores ocupados en el siglo ID.

La abandonada hilera de banderas flotaba en el aire delante de la cara
de Rose. De repente la importancia de lo que estaba a punto de pasar la
golpeó.

---Espera un momento ---le dijo al Doctor, poniéndole una mano en su
hombro justo cuando iba a saltar hasta las puertas---. Voy a ser la
primera mujer en pisar la Luna. Sé que lo fui hace mucho tiempo, pero es
increíble. La Luna, nunca te paras a pensarlo, es que está\ldots{} ahí
arriba. Y ahora estamos en ella ---ella estudió su cara---. Apuesto a que
para ti es como ir de paseo hasta Calais o algo así.

El Doctor se giró para mirarla a la cara. Sus rasgos estaban vivos con
maravilla y emoción. No por primera vez, Rose sintió como si pudiera ver
a través de sus ojos, y se preguntó si aquella era una de las razones
por las que necesitaba viajar con alguien.

---Rose, la Luna es increíble. Todo el mundo abajo en la Tierra depende
de ella. Las ratas saltan buscándola. Las mareas se rigen por ella. Los
humanos se besan bajo ella. Sin ella no habría nada que iluminara las
noches. Y esto sólo paso por casualidad, con trillones de posibilidades
en su contra, un poco de polvo de estrella se encuentra con otro poco de
polvo de estrella.

Rose saltó hacia las puertas y alargó el brazo para coger el pomo y
entonces se detuvo.

---Debería pensar en algo en lo que decir.

---Simplemente sal ahí fuera ---dijo el Doctor, colgándose una bolsa llena
de palos de golf a su espalda---. ¡Salta!

Rose cerró los ojos, empujó la puerta para abrirse y saltó. Se hundió
con un ruido sordo, chocando contra una mesa de madera. Había sido un
salto ordinario, para nada ingrávido. Levantándose, el protector del
traje le había evitado lo peor de la caída, miró a su alrededor. Había
más mesas, taburetes y sillas, un par de máquinas de fruta, una pizarra
con MARTES DE QUIZ A LAS 8 p.m. EL ESPECIAL DE HOY ES POLLO AL CURRY
escrito con tiza, y una larga barra de bar con trapos. Todo ello estaba
iluminado por la luz de un sol de un tímido y temprano verano británico.
El edificio era viejo, soportado por postes de madera.

Se giró para mirar a la TARDIS, que se erguía aún más fuera de lugar de
lo normal en la esquina del interior de un pub. El Doctor vestido con el
traje espacial estaba de pie en las puertas, mirando a todas partes
menos a ella.

---Guau ---dijo---. ¡Alguien ha construido una réplica exacta de un pub en
la Luna!

Rose se rió, se quitó el casco e hizo que le pegaba un puñetazo.

---¡Ríndete! Eres muy malo.

---No me he ido muy lejos ---dijo el Doctor con un poco con infelicidad,
sacándose su propio casco---. Si la Luna está en el terrestre Calais en
Dover ---frunció el ceño---. Aún así, es raro. He comprobado los
controles cuando estábamos llegando y estábamos totalmente yendo hacia
la Luna. Incluso la vi en el escáner antes de que aterrizáramos, toda
gris y polvorienta, la vieja Luna lunera, esa pequeña y anciana cerilla
en el cielo.

Rose supo que lo decía en serio, que no era simplemente una excusa que
se estaba inventando.

---Ve y comprueba la TARDIS, entonces.

El Doctor asintió.

---Iré a comprobar la TARDIS, entonces ---pero se detuvo en la puerta,
mirando por la ventana más cercana hacia un pueblo verde y una iglesia
que eran demasiado típicos---. Parece Mayo. Parece Inglaterra ---olisqueó
el aire---. No demasiado lejos del mar. Hmm, me ha venido una ráfaga de
agua marina\ldots{}

Rose rió y señaló hacia la TARDIS.

---Entonces ve y compruébalo.

El Doctor cogió su bandera y la bolsa de palos de golf y se desvaneció
dentro de la TARDIS.

Rose estaba a punto de seguirle cuando vio en la barra un diario. No
pudo evitar cogerlo con su mano enguantada y echarle un vistazo,
comprobando la fecha. El Doctor estaba en lo cierto: era Mayo.

Siempre que volvía a la Tierra, a Rose le gustaba enterarse de lo que
había pasado. Aquel era un diario local, con la primera plana preocupada
con nada más que una disputa entre un aparcamiento y un plan para un
supermercado, pero algo hizo que Rose se quitara los guantes y pasara
las páginas mientras caminaba sin pensar hacia la TARDIS.

Tras unas páginas se quedó quieta. Notó cómo se le paraba el corazón.

El titular decía: RUINAS ROMANAS EN CREDITON VALE. Debajo de ello había
una imagen en color de un hombre de mediana edad con un sombrero y un
chaleco amarillo, de pie junto a un gran aparador que contenía una
sección rota de un mosaico romano de unos tres metros. Representado en
el mosaico había un retrato de cuerpo entero de un hombre y una mujer,
ambos atractivos, con el pelo rizado y oscuro, vestidos con ropas
moradas. A su alrededor había una jarra y un racimo de uvas verdes. A la
derecha en el extremo más alejado, hecho con unas pequeñas teselas
doradas, había una silueta familiar con forma de salero. Tres formas
sobresalían de la silueta: un ojo alargado de la cúpula de su cabeza,
una ventosa y una pistola del centro. Su parte baja estaba dotada de
unas brillantes formas circulares.

Un Dalek.

Rose corrió hacia la TARDIS, y la puerta de la cabina de policía se le
cerró en los morros. Hubo un ruido fuerte. La luz en lo alto comenzó a
brillar y los antiguos motores en lo más profundo de su centro gruñeron
volviendo a la vida.

---¡Doctor! ---gritó Rose---. Doctor, ¿qué haces?

Cinco segundos más tarde, la TARDIS se había ~ido. Una profunda marca
cuadrada en la alfombra de flores del pub era la única señal de que una
vez había estado ahí.