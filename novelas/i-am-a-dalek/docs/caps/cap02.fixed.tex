\chapter*{Capítulo Dos}
\addcontentsline{toc}{chapter}{Capítulo Dos}

Kate Yates simplemente sabía que iba a ser un mal día.

Estaba soñando que había vuelto al colegio. Todo el mundo en la clase
tenía dieciséis años, mientras que ella tenía veintiocho, y había unas
risitas infantiles y susurros que decían: «¿Por qué está ella aún
aquí?». Entonces oyó a su padre gritando escaleras arriba:

---¡Son las ocho de la mañana!

En ese mismo momento, la radio de su mesita de noche se encendió. Unos
segundos más tarde oyó la puerta principal cerrarse tras sus padres que
se marchaban a trabajar.

Entonces las noticias acabaron y cuando Wogan comenzó a hablar, la
gentil charlatana e irlandesa Kate había sabido desde la infancia
calarse hasta los huesos. Él hablaba de pasta de dientes, de lo que
dieron anoche en la televisión\ldots{} pequeñas y divertidas cosas. Pero
para Kate sólo decía: Sólo cinco minutos más. Cinco minutos más en tu
cama, Kate Yates, en la más suave y cómoda cama del mundo entero.

Él dejó de hablar y puso un poco de música.

---Esta es la canción Snowbird de Anne Murray.

Kate sabía que era una canción diseñada específicamente para evitar que
la gente saliera de la cama y fuera a trabajar. Era una lenta y
soñolienta canción. Pero no pudo resistirse, y giró la cara hacia su
profunda almohada, cerró los ojos y sintió que, igual que el pájaro de
las nieves, también debería extender sus pequeñas alas y volar lejos.

Un segundo más tarde oyó otra voz. Una voz escocesa. Ken Bruce. Wogan le
pasaba la conexión a Ken Bruce, lo que sólo podía significar que no era
un segundo más tarde de las nueve y media.

Kate se sentó en la cama y miró el reloj.

---¿Qué? ---gritó---. ¿Cómo puede ser? ¿Qué les ha pasado a estos noventa
minutos?

Retiró su edredón y corrió hacia el lavabo, se arrancó el pijama, se
pasó el desodorante bajo sus brazos, agarró una doblada blusa del
armario, se metió en su falda de trabajo y sus zapatos y corrió a toda
prisa escaleras abajo. Una carta descansaba en la alfombra para ella:
otro aviso de la tarjeta de crédito que podría añadir a la carpeta
sobrepoblada bajo su cama. La lanzó por encima de su hombro, agarró su
bolso, se comió medio croissant que había dejado su madre en la mesa del
teléfono y salió corriendo por la puerta principal hacia el que había
sido a menudo descrito como uno de los pueblos más bonitos del Reino
Unido. Pero para Kate, Winchelham era una bonita trampa.

Porque tenía veintiocho y estaba de vuelta. De vuelta a la habitación en
la que había crecido, despertándose cada mañana en la misma cama
individual donde, de adolescente, había soñado con marcharse. Yendo a
hurtadillas por el pueblo por miedo a encontrarse con alguien del
colegio y tener que explicar por qué estaba allí. La niña con grandes
sueños de la ciudad, había vuelto de Londres bajo una nube de deudas,
viviendo con su madre y su padre. Intentando averiguar qué hacer con su
vida mientras trabajaba en una centralita cerca de la reserva natural,
en un escritorio que hacía esquina, lejos de los zarapitos y los
martines pescadores, con una vista de unos cubos de basura y un
aparcamiento de coches.

Los pensamientos sobre la centralita apresuraron el paso de Kate por las
amplias calles hacia la plaza. Su jefa, Serena, estaría ahora mismo
observando su escritorio vacío, poniéndose su cárdigan por encima de sus
enormes e implacables pechos y chasqueando la lengua. Serena, la que no
abría los cajones por miedo a romperse una uña. Serena, que desaprobaba
las llamadas personales de Kate, pero que aún así parecía pasar la mitad
de su jornada laboral llamando a su amiga Sheila para discutir sobre su
obstinado marido en un tono plano y apagado.

---Y yo le dije: ``Si ella está fuera de tu cama y de tu vida, ¿cómo es
que estaban esos dos billetes hacia Gambia en el cajón de tu mesita?''.

Llegaban llamadas de todo el mundo a través del país, furiosos porque
sus camas no habían sido enviadas tal y como habían prometido, o porque
habían llegado sin cabecero o sin ruedas. Aquellas llamadas ahora
estarían en su contestador.

Kate no podía creer que estuviera realmente corriendo hacia Serena,
corriendo hacia las voces enfadadas.

El pueblo del que conocía cada detalle (cada poste de luz, cada losa
suelta, cada papelera molestando en su maldita vida) se emborronó a su
alrededor mientras corría hacia la plaza y hacia el bus de las 9.40.
Eran las 9.39. Los buses siempre llegaban tarde, pero Kate sabía que
aquel en particular estaría girando la esquina de la iglesia exactamente
a tiempo, justo entonces. Eso significaría una larga caminata hasta el
trabajo por un callejón oscuro y mugriento.

Así que corrió aún más rápido.

~

Rose salió de su traje espacial. Podía oír ruidos de movimiento que
venían escaleras arriba. La última cosa que quería en aquel momento era
tener que explicarse al dueño, así que soltó el cerrojo de una ventana,
la abrió de un golpe y se escurrió a través del agujero hacia la soleada
y vacía calle del pueblo.

Sabía que el Doctor no la habría abandonado por voluntad propia. Estaría
de vuelta pronto con alguna extraña y técnica explicación. Pero entonces
recordó el Dalek del mosaico. Seguro que tenía que haber alguna conexión
entre ello y la repentina desaparición del Doctor\ldots{}

Se distrajo de aquellos oscuros pensamientos por la belleza de lo que se
hallaba ante ella. Las nubes se iban y la luz del azulado cielo de Mayo
formaba una escena idílica: una oficina de correos, un pequeño museo, la
plaza de un pueblo y una iglesia. El Doctor tenía razón: tras la iglesia
y detrás de unas bajas colinas vio el reflejo del mar. Un autobús giraba
la esquina de la plaza cerca por la iglesia y la atravesó lentamente.
Parecía imposible que la frenética y peligrosa vida del Doctor pudiera
afectar tal lugar, donde las cosas seguían igual que habían estado
durante siglos.

Rose se sentó en el banco y sacó la llave de la TARDIS del bolsillo de
sus tejanos, esperando que brillara y la alertara de la vuelta del
Doctor. En la distancia escuchó el sonido de unos tacones corriendo.
Alguien tenía prisa.

~

Kate giró la esquina de la plaza igual que lo había hecho millones de
veces antes, pegándole una patada y haciendo volar la botella de leche
de la puerta principal de alguien. Pudo oír el ruido del motor del
autobús a su izquierda y supo en su corazón que era demasiado tarde,
pero siguió corriendo.

Una gran bola de amargura, en parte producido sólo por el croissant que
se acababa de comer, se formó en su estómago. ¿Era eso lo que quería?
Un año antes había estado en Londres, vendiendo sus chanclas en el
mercado de Camden, tan segura de poder devolver su crédito de negocio al
banco que había estado usando su tarjeta de crédito para pagarse el
alquiler. Había pensado que sólo estaba empezando. ¿Qué pasaba si ya
estaba acabada, aplastada y quemada? ¿Qué pasaba si era inútil? ¿Qué
pasaba si la vida era inútil?

Vio la parte trasera del autobús, al otro lado de la plaza, cerca del
pub, alejándose. Se detuvo en seco en medio de la carretera. Una
fracción de segundo más tarde un brillante deportivo rojo giró la
esquina y se chocó contra ella.

Tuvo un pequeño instante para darse cuenta de que iba a morir. La deuda
de la tarjeta de crédito nunca se iba a pagar. Nunca volvería a caminar
por el mugriento callejón en sus tacones, con los excrementos de gato
pegándose a su chaqueta. Serena nunca le echaría la bronca por llegar
dos horas tarde. Nunca haría las maravillosas cosas que había planeado
hacer. Aquello era el final de todo. Un estúpido y tonto accidente.

Se dio un golpe contra el duro asfalto mientras el coche frenaba en
seco. La botella de leche seguía rodando.

~

El repentino golpe de metal contra carne hizo que a Rose se le parara el
corazón. No había ningún otro sonido como el de un alma dejando un
cuerpo. Con su mente llena de pensamientos sobre su padre, se levantó
del banco y corrió a través de la plaza.

El conductor del deportivo estaba de pie, sorprendido, cerca del cuerpo
de una joven pelirroja.

---¡No la he visto! ---le gritó a Rose con una voz seca---. Apareció
corriendo y se detuvo\ldots{}

---¡Llama a una ambulancia! ---gritó Rose.

El conductor sacó su teléfono y comenzó a marcar. Rose se agachó junto a
la mujer y le cogió la ~mano. Los párpados de la joven se abrían y
cerraban. Aún podía haber una oportunidad. Recordó haber visto un vídeo
de primeros auxilios de su antiguo trabajo. Después de un accidente,
tienes que hacer que la persona hable.

---¡Escúchame! ¡Háblame! Me llamo Rose Tyler. ¿Cómo te llamas?

La mujer dijo suavemente:

---Kate\ldots{}

---¿Cuál es tu apellido? Kate, ¿cuál es tu apellido? ¡Háblame! Todo va
a estar bien. Hay una ambulancia en camino.

Rose apretó su mano en la suya, pero el centro del cuerpo de Kate estaba
horriblemente contorsionado y una profunda mancha morada de sangre teñía
la blusa. Rose apretó su mano tan fuerte que le dolía.

---¡Kate!

Sus ojos se pusieron en blanco.

---Yates\ldots{} Soy Kate Yates\ldots{} ---entonces Rose vio la luz salir
de sus ojos.

De repente algo quemó la mano de Rose. Hizo una mueca y la apartó. Al
mismo tiempo el cuerpo de Kate comenzó a temblar. Su espalda se arqueó.
Un aura verde salió de su herida, extendiéndose para cubrir todo su
cuerpo. Rose tragó saliva. El aire alrededor de Kate tenía el ambiente
de una tormenta, crujía de poder.

El aura desapareció tan rápidamente como hubo llegado, como si acabaran
de cerrar el interruptor. El pelo naranja de Kate ahora era rubio. Rose
se inclinó hacia adelante.

---¿Kate?

Con la blusa aún manchada, Kate se puso en pie tranquilamente y recogió
su bolso. Rose miró hacia donde había estado tumbada, al charco de
sangre fresca.

---Todo está bien, gracias. Estoy bien ---dijo Kate.