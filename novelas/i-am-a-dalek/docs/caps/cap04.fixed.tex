\chapter*{Capítulo Cuatro}
\addcontentsline{toc}{chapter}{Capítulo Cuatro}

---Parece un robot ---dijo Frank.

La cosa había sido desenterrada a toda prisa por los estudiantes. En su
emoción habían olvidado que la primera norma de la arqueología era la
paciencia. Su base aún seguía cubierta de tierra y sus lados estaban
manchados con pedazos de tierra. Parecía exactamente lo mismo que había
en el mosaico. Su carcasa dorada había perdido su color pero seguía
entera. Su ojo, ventosa y pistola estaban alzados con arrogancia. El
Doctor zarandeó su mano delante del ojo. No hubo ninguna reacción.

Pareció considerarlo durante un segundo. Entonces, mientras Frank se
adelantaba para tocarlo, gritó:

---¡Es una bomba! ¡Aléjate de ello, Frank!

Frank retiró la mano. Uno de sus estudiantes miró al Doctor de arriba
abajo y preguntó:

---¿Quién es este?

Frank y el Doctor se miraron el uno al otro. De alguna manera, Frank
confiaba en aquel joven extraño.

---Es el tipo de Londres ---se oyó decir a sí mismo, aunque sabía que no
era verdad.

El Doctor golpeó la mano de un estudiante mientras éste la alzaba hacia
la pistola.

---¡Y el tipo de Londres os dice que os retiréis! ---entonces agarró un
pico del suelo y gritó---. ¡Evacuad la zona! Tengo autoridad de Londres
y todo esto. ~¡Salid a la superficie, ya!

Frank no se sorprendió de que sus estudiantes obedecieran. Pero se
encontró a sí mismo quedándose.

~

La tetería del museo abrió más pronto. Kate, su única clienta, masticaba
embobada una pasta de te mientras hablaba por teléfono con Serena.
Enfadarse con Serena no tenía sentido, pero aún así, Kate se estaba
enfadando.

---Sí, casi me atropellan. Justo ahora.

---¿Te han atropellado mientras intentabas llegar al bus en el que ibas
tarde, no? ---preguntó la aburrida y sosa voz de Serena.

---¡La parte importante es lo de que casi me atropellan! ---le espetó
Kate.

Sintió una ola de rabia recorriéndole de arriba abajo. ¿Por qué siquiera
pretendía ser educada con aquella idiota? El significado de la
expresión ``verlo todo rojo'' de repente se esclareció para ella. Sentía
que si Serena hubiera estado allí, habría cogido el cuchillo de la
mantequilla y se lo habría clavado en el pecho. Pero no estaba allí, así
que apagó su teléfono móvil de un golpe y agarró el diario de la
cafetería del mostrador. Sin pensar, avanzó hacia la página de los
pasatiempos. Quizá si intentara el crucigrama fácil se relajaría un
poco.

En vez de eso, los sudokus le llamaron la atención. Antes apenas se
había molestado en mirarlos, siempre había sido muy mala para las
matemáticas, pero aquella mañana los números parecían bailar en el aire.
Sin siquiera pensar en ello, llenó todos los recuadros vacíos, para los
tres niveles: fácil, difícil y extremo, con los dedos recorriendo la
página. Entonces miró los crucigramas. Llenó los espacios vacíos con
letras de manera muy fácil, resolviendo incluso las más difíciles en
cuestión de un segundo.

Era fácil. ~Realmente fácil. ¿Por qué no se había dado cuenta antes?
Miró a su alrededor, respirando profundamente. Algo en el mundo había
cambiado, ¿o era en su interior?

Podía ver los átomos bailando por la habitación. Sabía la exacta
temperatura de su café. Vio y entendió los procesos químicos llevándose
a cabo en el interior de la taza. Pero no era como pensar. No tenía que
concentrarse ni hacer un esfuerzo. Lo sentía tan natural como respirar.
Y con ello vino el sentido de la fuerza y el poder. Alargó su mano para
coger la bolsita de edulcorante en un bol. Lo cogió con gentileza entre
su pulgar y el índice y lo observó romperse en una pequeña chispa de
electricidad estática.

Dio otra respiración profunda y miró hacia arriba. Alguien había entrado
en la pequeña tienda, la joven rubia hermosa que le había cogido de la
mano en la carretera, Rose. Aquello parecía estar sacado de un sueño.
Quería reír. Ni un coche a toda velocidad podría detenerla.

---¿Estás bien ahora? ---preguntó Rose.

Kate sonrió.

---Estoy bien, gracias. Quiero acabar esto e irme a trabajar. Gracias.

Rose se sentó a su lado, inclinándose hacia ella.

---El coche te pasó por encima. Te estabas muriendo. ¿Qué ha pasado?
Puedes decírmelo.

Kate se controló.

---Perdona, ¿puedes apartarte? Me gusta mi espacio personal.

Rose señaló hacia la blusa de Kate.

---Estás cubierta de sangre. Deberías estar muerta.

Había algo amable y lleno de confianza en los ojos marrones y profundos
de la chica. Kate tragó saliva, un pensamiento muy cruel le vino a la
cabeza. Tales emociones eran de débiles.

Rose siguió:

---Sé cómo te sientes. Algo pasa que no puedes explicar. Inventas
cualquier excusa para dejar de pensar en ello.

---¿Cómo te llamabas? ---preguntó Kate, aunque se acordaba.

---Rose. Rose Tyler ---le alargó la mano.

Kate la cogió y la apretó. Muy fuerte.

---Genial. Ahora bien, Rose Tyler, lárgate. Ya tengo bastante con mi
plato.

Rose hizo una mueca y apartó la mano.

~

Frank observaba al Doctor pasar el tubo metálico lentamente por el
objeto que había descrito como una bomba. Entonces el Doctor soltó un
suspiro hondo. Algo de aquella emotiva luz volvió a sus ojos. Miró hacia
Frank.

---¿Hay alguna posibilidad de que me hagas caso si te digo que vuelvas a
casa?

---Ninguna ---dijo Frank. Señaló hacia la sección de la bomba donde la
cabeza abovedada tenía una rejilla oxidada de metal rodeada de unos
listones metálicos---. Podría haber una bisagra ahí.

El Doctor sonrió.

---Me gustas, Frank Openshaw, eres listo.

Aplicó la punta de su tubo a la bisagra y entonces, con cuidado, alzó la
cúpula. Frank se acercó. En su interior había un amasijo de partes
metálicas y cables. Parecía como si faltara el espacio central, algo
sobre el tamaño de una bola de fútbol que habría estado allí tiempo
atrás. El Doctor introdujo la mano y sacó un puñado de polvo. Lo dejó
caer por entre sus dedos y le pegó un soplido.

---Tan muerto como puede estar ---dijo. Parecía estar aliviado, pero
también, o eso pensó Frank, quizá un poco triste, como si estuviera
pensando en el pasado.

Frank profirió un bufido.

---¿Una bomba? ¿En una tierra que no ha sido tocada durante 2000 años?

El Doctor se limpió el polvo de su mano y sonrió.

---De acuerdo, don listo Frank Openshaw, me has pillado. No es
estrictamente una bomba ---dio golpecitos en la carcasa---. Es lo que
queda de la criatura más terrorífica del universo.

---Nunca lo había visto antes ---dijo Frank.

---Y no sabes la suerte que tienes ---pegó un silbido y señaló por encima
de su hombro con su pulgar---. Ahora en serio, vete ---y se giró para
seguir estudiando el objeto.

Frank no se movió. Pensó en las palabras del Doctor.

---Has dicho ``universo''.

---¿Qué pasa con ello? ---preguntó el Doctor.

---Nadie diría ``la criatura más terrorífica del universo''. A no ser que
estés loco y no estás loco.

El Doctor frunció el ceño.

---Vete a casa, Frank. Tienes el día libre. Quítate los zapatos, cómete
un par de salchichas con patatas y ponte a ver Brainteaser. Vuelve
mañana.

---Solo dirías ``universo'' si fueras, no sé, del espacio ---dijo Frank, y
se rió de sí mismo después de decirlo.

El Doctor parpadeó.

---No seas tonto.

Frank señaló hacia el objeto.

---Y eso sería también del espacio. Y lo que has dicho sobre Nerón, y la
pizza\ldots{} solo lo sabrías si hubieras estado ahí ---se volvió a reír
ante la locura de lo que estaba diciendo.

El Doctor volvió a parpadear. Por una vez no decía nada.

---Lo siento, ¿estoy siendo molesto? ---preguntó Frank. Sabía que su
teoría no podía ser cierta.

El Doctor se rió y le pegó unas palmadas en el hombro.

---No. Ahora sí que de verdad me gustas ---señaló al objeto---. Eso es un
Dalek. No, perdón, era un Dalek. Del planeta Skaro. Una vez, sí, fue la
criatura más terrorífica del universo. Estaban dotados para la guerra.
Ahora están todas muertas, las criaturas de su interior. Eso es solo la
carcasa, un montón de chatarra. ¡Hay más vida en el chaleco de un
vagabundo!

Era la conversación más extraña de la vida de Frank. El Doctor estaba
obviamente de broma, inventándose todo aquello, pero aún así Frank
decidió seguirle el juego.

---¿Y qué los mató? ---preguntó.

---Yo ---dijo el Doctor---. Muchas batallas y una guerra final ---le pegó
una patada a la base del Dalek---. Ya no hay nada de lo que asustarse.

~

---Quiero que conozcas a un amigo mío ---dijo Rose, siguiendo a Kate
mientras ésta salía de la tetería---. Puede ayudarte.

Kate suspiró.

---Gracias por preocuparte, pero estoy bien, de verdad.

Rose la agarró del hombro y la giró para hacer que se mirara en una de
las ventanas del museo.

---Eres rubia. Cuando corriste hacia la carretera, te vi. Tenías el pelo
pelirrojo y rizado y ahora, ¡mírate!

Kate se vio a sí misma en la ventana. Su pelo estaba liso y era de un
rubio brillante, como si fuera el de una supermodelo sueca. Le recorrió
un escalofrío, dando un paso atrás. No podía aceptar lo que estaba
viendo.

---Kate, ven y conoce al Doctor ---dijo Rose.

Kate giró la cabeza de golpe. El movimiento era totalmente instintivo.
¡Doctor! ¡El Doctor!

---Vamos ---dijo Rose, cogiéndola con amabilidad de la mano---. Está en un
lugar llamado Crediton Vale. ¿Lo conoces?

Kate asintió. Otro bus estaba girando por la plaza. Señaló hacia allí:

---Podemos subirnos y estaremos allí en cinco minutos.

---No te asustes. Sabrá qué hacer ---dijo Rose, llevándola a la parada de
autobús.

Mientras caminaban por la tranquila calle del pueblo de su infancia,
unas imágenes terribles recorrieron la memoria visual de Kate. Las
memorias de otra persona. Mundos enteros ardiendo, planetas cayendo por
el espacio como bolas empujadas en una mesa de billar. La palabra Doctor
resonaba en su cabeza. Vio la silueta de un hombre iluminada por el
fuego. Hubo un nudo de rabia en su interior, algo vil y seguro y agudo.
Entonces otra emoción le recorrió: miedo.

Una palabra comenzó a recorrerle la cabeza. Sus cuatro sílabas exigían
salir por su boca y gritarlas una y otra vez.

¡EXTERMINAR!