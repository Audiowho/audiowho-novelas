\chapter*{Capítulo Tres}
\addcontentsline{toc}{chapter}{Capítulo Tres}

El Doctor miró hacia la gran columna central de la TARDIS. En cuanto
tocó un par de controles, las puertas se cerraron y la nave había
decidido despegar.

---¡Hola! ¡Debería haber dos pasajeros en esta nave! ---gritó.

Cruzó la sala hacia la pantalla del escáner, que mostraba un conjunto de
símbolos que no había visto antes. Una cosa era segura: la TARDIS no
estaba bajo el control de una influencia externa. Había cambiado el
rumbo de la Luna y les había llevado a la Tierra. Ahora le estaba
llevando a otro lugar. Incluso después de nueve siglos de viajes a
través del Tiempo y el Espacio, le seguía sorprendiendo.

---¿Qué intentas decirme? ¡No seas tan enigmática! ¿No me lo puedes
decir y ya está? ¿Y dónde estamos yendo ahora? ¿A Northampton? ---tocó
un par de botones, sin resultado alguno---. ¡Para, para!

Un segundo más tarde, la columna se detuvo en seco, la gran sala
parpadeaba y temblaba bajo sus pies. Cambió la pantalla para ver una
vista exterior de la nueva localización. Mostraba una oscura y vacía
sala de hormigón. Se quitó su traje espacial y cogió su americana de
rayas de una percha. Poniéndosela, cogió una linterna de un armario,
entonces abrió las puertas y salió. Dondequiera que la TARDIS le había
llevado, y fuera cual fuera la razón, sólo había sido un vuelo de unos
segundos. No podía estar muy lejos de dónde había dejado a Rose.

Ciertamente, se veía y olía muy distinto a la última parada. El aire
estaba estancado y olía a cerrado, con ese especial frío llano que puede
encontrarse bajo tierra. El rayo de luz de su linterna luchaba por entre
la oscuridad. Recorrió unas cuantas columnas de hormigón para
encontrarse con un cartel de metal que decía ÁREA 3 con esas típicas
letras oficiales. A su lado había un pequeño armarito donde una vez
habría estado un extintor.

Detrás de aquello había una oscura y gran puerta metálica de color
verde, abierta de par en par. La atravesó para llegar a un largo y vacío
pasillo.

---¡¿Hola?! ¿¡Hay alguien cerca?! ---gritó, sin esperar una respuesta,
pues el lugar parecía desierto y abandonado.

Caminó por el pasillo y giró hacia otra sala. La linterna iluminaba dos
hileras de unas viejas y oxidadas camas de hierro. En la pared cerca de
la puerta había un teléfono, el Doctor lo cogió y escuchó. No había
línea. La suela de su zapato se topó con algo en el suelo. Se agachó y
cogió un raído panfleto con el título ``Protégete y sobrevive'' y una
fecha de 1980.

---``Come sólo productos enlatados'' ---leyó---. ``Si usted vive en una
caravana o en algún otro alojamiento similar que le provea de muy poca
protección para refugiarse, su autoridad local estará capacitada para
aconsejarle sobre qué debe hacer'' ---rió para sí---. Hola, somos el
ayuntamiento y le decimos que salga corriendo como si no hubiera mañana.

Así que aquél era un refugio nuclear, uno en desuso al parecer. Pero,
¿por qué la TARDIS le había traído allí?

Antes de que tuviera tiempo para pensárselo más, oyó algo que no
esperaba. Se irguió para escuchar. Sí, estaba en lo cierto. Alguien, en
algún lugar de aquel búnker, estaba escuchando la radio.

Salió en busca de aquella persona.

~

Frank Openshaw se recostaba orgullosamente en su silla, observando la
excavación, dando golpecitos con los dedos de los pies al ritmo de la
radio. El lento y paciente negocio de su mayor proyecto hasta la fecha
se extendía delante de él. Voluntarios, en mayor parte estudiantes del
colegio local, trabajaban con cuidado en el hoyo, que estaba iluminado
por varias lámparas enormes. Pegó un sorbo al café de la taza de su
termo, sintiéndose seguro y exitoso. Aquel yacimiento iba a darle un
nombre. No le importaba mucho la fama, pero la seguridad del trabajo
garantizado era otra cosa. Nunca dejaría a Sandra.

Alguien le cogió del hombro.

---Perdone, ¿puedo dejarme su teléfono? ---preguntó una voz en un
ligeramente extraño acento de Londres pero que no acababa de ser de
Londres.

Frank levantó la mirada. El dueño de la voz era demasiado mayor para ser
un estudiante, era alto y muy delgado, tenía el pelo oscuro y estaba
vestido con un tanto desaliñado traje. Frank parpadeó. Era como si
alguien hubiera encendido una luz brillante. El extraño se erguía con
confianza y entusiasmo y se encontró a si mismo dándole su teléfono
móvil sin ni siquiera pensar en ello.

---No conseguirá cobertura aquí abajo ---le advirtió.

---Ya verá cómo sí ---dijo el extraño. Sacó un fino tubo de metal de su
bolsillo, apretó un interruptor a un lado y le pasó la punta del lado
por el teléfono. Entonces marcó el número.

Frank parecía fascinado. Escuchó la voz de una mujer al teléfono.

---Vale, ¿qué ha pasado?

---La culpa es de la TARDIS ---dijo el extraño---. Sí, es todo culpa de la
TARDIS. Tiene esos sistemas de emergencia. Los apagué todos años atrás.
No dejaban de sonar y no me dejaban oír mis propios pensamientos. Deben
de haberse vuelto a encender. Estoy en\ldots{} ---miró a Frank---. ¿Dónde
estoy?

---Crediton Vale ---dijo Frank.

---Crediton Vale, búnker en desuso, quizá a un par de quilómetros y medio
de allí. Debe de ser un paseo encantador para ti. Te envidio. Nos vemos
en un rato.

---Espera, Doctor ---dijo la voz de la mujer con urgencia---. Hay algo raro
e importante. Dos cosas de hecho. Primero, esta esa excavación, y
han\ldots{}

---Sí, estoy ahí ahora mismo. Nos vemos luego. No puedo hablar porque
estoy usando el teléfono de otra persona ---apagó el teléfono y se lo
pasó a Frank. Se fregó las manos y miró hacia el hoyo---. Excavando
---dijo---. No sé si me gusta excavar. Excavar puede ser bueno, excavar
puede ser malo. Depende en lo que los excavadores estén excavando ---se
giró hacia Frank y le lanzó una sonrisa de oreja a oreja---. Lo sé. Quizá
debería dejar de hablar durante un momento.

Frank miraba la pantalla de su móvil. No había barritas.

---No hay cobertura ---dijo.

---¿Ah, no? ---respondió el extraño, inocentemente.

Frank señaló el tubo metálico en las manos del extraño:

---¿Qué es eso? ¿Cómo has hecho eso?

---No preguntes ---dijo el extraño---. Fue el regalo de cumpleaños de mi
cuñada. Yo quería una corbata ---señaló por encima del hombro de Frank
hacia una larga pieza de madera astillada, uno de sus mayores hallazgos
hasta la fecha, que estaba etiquetada y descansaba en una mesa de
trabajo---. Esa es la barra de giro de un pozo romano, alrededor del 70
a.C. Ata a tu caballo ahí, y gira y gira. Cinco minutos más tarde tienes
un precioso cubo de agua y un caballo muy mareado.

Frank se levantó y le siguió hasta la mesa, rascándose la cabeza.

---Creía que era un poste de soporte ---dijo. Algo en aquel tipo le hacía
sentir inferior, como si fuera un principiante.

---No, mira los bordes. Demasiado lisos para ser eso ---alargó la mano y
se la estrechó a Frank firmemente---. Soy el Doctor, por cierto.

---Frank Openshaw. Me dijeron que vendría alguien de Londres\ldots{}

---¿Eso hicieron? ---el Doctor vio otro hallazgo en la mesa, una sucia
moneda romana---. Oh, mira eso. Nerón. Siempre vuelve a mí ---se agachó,
sacó un par de gafas y observó el perfil masculino de la moneda---. Era
mucho más gordo que aquí ---señaló hacia arriba---. Así que había una
ciudad romana aquí, ¿eh? Y fue destruida en la revuelta de Boadicea.
Los bretones la arrasaron y metieron todo aquí abajo en las cuevas.
Alrededor de los 50, el gobierno británico construye aquí abajo un gran
búnker en las cuevas: el centro del gobierno regional. Parece un gran
búnker de los más secretos. Cuando acaba la Guerra Fría, alguien viene a
rellenar este espacio y construye unos edificios en la superficie.
Entonces encuentran estas cosas y te llaman a ti. ¿Me equivoco?

Frank tragó saliva.

---Para nada. Vale, venga aquí y échele un vistazo a esto ---llevó al
Doctor a un montón de los hallazgos más recientes y le pasó un triangulo
metálico---. ¿Cree que es una herramienta de jardinería?

El Doctor negó con la cabeza, tristemente.

---No, el mango está mal. Es un cortador de pizza. Lo único que no tenían
tomate en aquella época. Así que era algo como una tostada con queso y
hierbas. Un naan~de queso, de hecho. Delicioso ---se quitó las gafas, las
guardó y miró a Frank---. Lo lamento, ¿estoy siendo molesto?

---No me ha quedado claro su nombre ---dijo Frank.

---Sólo el Doctor. El. Doctor ---se rascó la parte trasera del cuello---.
Ahora, ¿me equivoco si pienso que has desenterrado algo que no entiendes
para nada en absoluto?

Frank suspiró.

---Y supongo que usted sabrá qué es.

El Doctor se encogió de hombros.

---Puede que sí. Lo siento. A todo el mundo le encanta un tipo listo como
yo.

Frank señaló hacia un estrecho pasillo que llevaba hacia la excavación
principal.

---La imagen de la derecha del mosaico. Aquí abajo. Siga las luces.

El Doctor levantó los pulgares y se marchó. Frank se le quedó mirando y
reflexionó. Y cuánto más reflexionaba, más pensamientos le llenaban la
mente. Uno de los estudiantes le despertó de su reflexión.

---¡Frank! ---le llamó desde el hoyo---. Hay algo metálico aquí abajo. ¡Y
es muy raro!

~

El Doctor se paseó por el pasillo. Una lámpara normal mostraba un
aparador con un gran mosaico de bordes desgastados en su interior. El
Doctor supuso que cuando los bretones hubieron saqueado la ciudad romana
encima de ellos, se lo habrían llevado también a las cuevas. Vio lo que
estaba representado y se le pararon los corazones. En ese mismo momento
escuchó gritos de emoción y sorpresa de la excavación principal. La
radio se había apagado. Corrió hacia allí.

---¡Frank! ¡Señor Openshaw!

Salió hacia una sala y saltó al hoyo, corriendo hacia dónde Openshaw y
sus trabajadores estaban reunidos en una esquina alejada.

---¡Apartaos! ---les gritó, apartando a un par de alumnos para poder
pasar.

Y se encontró a sí mismo encarándose con un Dalek.