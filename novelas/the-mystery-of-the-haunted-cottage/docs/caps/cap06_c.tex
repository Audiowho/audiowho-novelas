\chapter*{Capítulo Seis} \addcontentsline{toc}{chapter}{Capítulo Seis}

---Me encanta una buena persecución ---dijo Cotterill---. Todas las grandes historias terminan con una ---miró a su alrededor mientras el bosque desaparecía al llegar a la cima de la montaña---. Por lo que veo has leído un montón de libros.

---Sí ---afirmó el Doctor, levantando la cabeza---. He leído un montón de libros. Si viajas tanto como yo y pasas tanto tiempo solo, sin dormir apenas\ldots{} tiendes a leer, ¿no crees?. Tengo libros en mi mente desde hace 900 años, Cotterill, y tu los vas a conseguir todos. A la vez.

Su entorno comenzó a cambiar más rápido, mar abierto, campo, un estéril hall, mundo alíen, edificio alíen, silo de misiles\ldots{}como si se hojeara un folioscopio.

---¡Detente! ---exclamó Cotterill, con los ojos muy abiertos.

---Creo que no ---dijo el Doctor---. Mira, ahora que estás aquí\ldots{} ---se tocó la cabeza ---no te voy a dejar salir.

El mundo a su alrededor se desdibujó. Cotterill comenzó a tambalearse con el rostro pálido.

---¡Detenedle! ---gritó---. ¡Matadlo!.

Martha corrió al lado del Doctor mientras los No-Hombres los cercaban.

---¡Doctor!. ¡Ahora vendría bien un arma!.

Su vacilante entorno se frenó repentinamente, y Martha encontró una gruesa rama en su pie. La cogió y el mundo se volvió a desdibujar.

---Una rama ---dijo---. ¿Eso es todo?.

---Es lo mejor que pude hacer ---murmuró el Doctor. Su rostro estaba tenso por la concentración y sus ojos estaban cerrados.

Martha se giró hacia el No-Hombre más cercano golpeándolo en la cabeza.

Dio un paso hacia atrás. Ella no sabía si el ser sentía dolor, pero era un bípedo, por lo que la mayor parte de lo que sabía acerca de la anatomía humana podía ser aplicada, o eso esperaba.

Se giró de nuevo, golpeando bajo esta vez, la rama crujiendo en el lateral de su rodilla. Cayó y se dio la vuelta, la rama crujió en el codo de otro No-Hombre que estaba alcanzando al Doctor. Cargó contra él, rebotó, pero se las apañó para enviarlo atrás unos cuantos pasos.

---¡Vamos! ---rugió mientras que los otros No-Hombres la rodeaban---.

¡Me enfrentaré a todos!.

Ellos no parecían sentirse intimidados en lo más mínimo y siguieron adelante.

---No ---Cotterill jadeó volviendo la cabeza. Martha miró. El paisaje siempre cambiante la hacía sentirse mareada, pero a lo lejos había una especie de agujero negro que estaba creciendo cada vez más.

Uno de los No-Hombres cogió a Martha y ella se giró, moviendo la rama, pero justo antes de que hiciera contacto, la rama se desvaneció de sus manos.

---Lo siento ---murmuró el Doctor.

Los dedos del No-Hombre rodearon el cuello de Martha y empezó a apretar, y de repente el No-Hombre ya no estaba ahí. Martha se giró, jadeando, y observó a los restantes No-Hombres desvanecerse.

El agujero negro se estaba acercando. Se unió a otros cercanos. Martha vio más Nada en la periferia, más allá de los árboles y los edificios y las paredes y rocas que parpadeaban.

---¡Detente! ---rugió Cotterill---. ¡Por favor!. ¡Este es mi hogar!.

El paisaje había perdido sus características. Ahora era algo pequeño y curvado, como si estuvieran de pie en un planeta del tamaño de un campo de fútbol, rodeado por todos lados por un vacío cada vez más grande.

Hasta el propio Cotterill estaba cambiando, encogiéndose, perdiendo su forma física, convirtiéndose en una cosa retorcida de sombras y luz.

El Doctor se puso en pie, lentamente.

---Bueno ---dijo---, creo que alguien le debe a alguien una disculpa.

---Lo siento ---dijo el ser que fue una vez Cotterill. Su voz resonó en los oídos de Marta. Por favor, déjame salir de tu mente. Duele.

---Lo se ---dijo el Doctor. El destornillador estaba en sus manos de nuevo, escaneando. Miró los resultados y gruñó-. ¿Quién eres?.

---Mi pueblo no tiene nombres ---explicó---. Soy un Ch'otterai. Las leyendas de mi pueblo dicen que una vez fuimos dioses, que podíamos modificar la realidad con nuestras mentes. Pero los Ch'otterai nunca han tenido una imaginación particularmente vibrante. Si hubiéramos tenido ese potencial haría tiempo que lo hubiéramos desperdiciado en pequeños actos de exhibicionismo.

---¿Cómo has llegado hasta aquí?.

---Mi nave no funcionaba correctamente, explotó, destruyó mi cuerpo físico y lanzó mi conciencia a la deriva.

Martha entrecerró los ojos.

---¿Eres un fantasma?.

---En cierto modo ---explicó el Doctor---. Entonces, ¿qué pasó?. Pasan unos siglos, te aburres, alguien tiene la mala suerte de estar demasiado cerca\ldots{}. ¿y los atrapas?.

---Pensé que me podrían llevar a casa ---dijo el Ch'otterai---. Todo lo que necesitaba era la fuerza suficiente para tomar forma física\ldots{} pero no podía dejar este punto en el espacio. Me quedé atrapado aquí, donde morí. Así que les di historias y me alimenté del poder que generaban. Luego atraje otra nave hasta aquí, y otro\ldots{} hasta un planeta. Reconfiguré realidad. Me convertí en un Dios.

---Un gran Dios en un pequeño estanque ---dijo el Doctor---. Usar la imaginación de otras personas porque la tuya crece muy despacio. ¿Y qué pasó con esas naves, eh?. ¿Qué pasó con esas personas?.

El Ch'otterai vaciló.

---Yo les di sus historias.

---Entonces, ¿qué pasó con ellos?.

---Se fueron.

---No te creo.

---Se fueron. Te lo prometo, los dejé marchar. Soy un dios benevolente.

---En seguida te enfadas. En seguida gritas: ``¡Que les corten la cabeza!''.

---No, yo\ldots{}

---¿Qué hiciste con los cuerpos?. ¿Y las naves?. ¿Los empujaste al gran lago de Nada y dejaste que los engullera?.

La forma de remolino del Ch'otterai frenó durante un momento.

---Tu no lo entiendes. No sabes lo que ha sido para mi. Se iban a ir, Doctor. Sus historias se volvieron obsoletas y se aburrieron. Yo los necesitaba y me iban a abandonar, iban a advertir a otros para que se alejaran de aquí. No\ldots{} no podía dejar que eso sucediera.

---Así que los mataste.

---Les di a elegir. A todos y a cada uno de ellos. Si se quedaban, les abastecería. Si trataban de irse, los destruiría.

---Todos ellos intentaron huir ---dijo Martha en voz baja.

La voz del Ch'otterai se intensificó con la ira.

---Mintieron. Se comprometieron a quedarse y luego trataron de llegar a sus naves. Trataron de robar mi energía. Se merecían la destrucción.

El remolino se convirtió en algo tan violento que Martha dio un paso atrás, y luego, como si se hubiera dado cuenta de que había ido demasiado lejos, el remolino desaceleró, y la voz del Ch'otterai se suavizó.

---Pero, Doctor\ldots{} tu eres diferente. Si nos tomamos el tiempo suficiente, podrías hacerme lo suficientemente fuerte como para construir una galaxia, y a partir de ahí, un universo. Vosotros dos podríais gobernar ese universo, como rey y reina.

---Contigo como nuestro Dios ---murmuró el Doctor---.

---Sí ---respondió el Ch'otterai---.

---No ---dijo el Doctor---. Extendió la mano en dirección a Martha, y ambos empezaron a caminar hacia atrás. Ahora estaban en una superficie del tamaño de una piscina.

La forma del Ch'otterai estaba girando tan rápido que se estaba desdibujando.

---¡Detente! ---chilló---. ¡Da un paso más y te arrojaré al vacío. ¡Ni siquiera podrás llegar a tu nave!.

Martha y el Doctor se quedaron inmoviles.

---Tendrás que sacrificar hasta el último suspiro de tu poder ---dijo el Doctor.

---¡Lo haré!. ¡Te mataré antes de dejar que te vayas!. ¡Empezaré de nuevo!. ¡Encontraré a alguien nuevo!.

Martha miró trás ella. La TARDIS estaba a tres pasos de distancia.

Demasiado lejos.

---No nos vamos a quedar ---dijo el Doctor---. No podemos.

La voz del Ch'otterai se agudizó.

---Entonces, encontraré la manera de ir contigo. O mejor aún, la manera de transportarme a la Tierra de Ficción. Uno de mis No-hombres te escuchó hablar de ella. Llévame allí, Doctor, y dejaré que ambos viváis.

---La Tierra de Ficción ni siquiera está en esta realidad.

---Eres un ser inteligente ---dijo el Ch'otterai. Su tono de voz era más tranquilo ahora que estaba creciendo en confianza---. Estoy seguro de que podrás encontrar la manera. Hazlo o muere.

Martha soltó al Doctor y se preparó para saltar hacia la TARDIS. El Doctor metió sus manos en los bolsillos.

---Puede que haya algo que yo pueda hacer ---dijo---. Tengo una unidad de emergencia a bordo. No te aburriré con los detalles, pero nos podría permitir deslizarnos hacia los lados atravesando las dimensiones.

Técnicamente, ni siquiera dejarías este espacio. Sin embargo, es muy peligroso. No hay garantías de que Martha y yo pudiéramos volver una vez que hayamos cumplido nuestra parte. No creo que sea una buena idea.

---Hazlo o muere, Doctor ---dijo el Ch'otterai---. Es así de simple.

El Doctor se tomó un momento, y Martha vio a su mandíbula tensarse.

---Muy bien ---dijo. Se acercó a la TARDIS y abrió la puerta. El Ch'otterai se metió dentro antes de que el Doctor pudiera cambiar de opinión\ldots{}

Y entonces el Doctor cerró la puerta.

Martha lo miró fijamente.

---¿Qué estás haciendo?. ¿le has encerrado en la TARDIS?. ¿Qué pasa si despega?.

El Doctor apoyó un hombro contra la puerta y se cruzó de brazos.

---El Ch'otterai es un ser de energía psíquica pura. Es cierto que tiene el potencial de ser infinitamente poderoso, pero ahora mismo apenas puede mantenerse unido.

Por una vez, la actitud calmada de Doctor no tranquilizó a Martha.

---Explícame cómo encerrarlo en la TARDIS es una buena idea.

---La TARDIS no es una máquina, Martha. Bueno, si lo es, pero también es mucho más. Ya has visto como se sobrecargó el Ch'otterai con los libros que he leído. Mi imaginación lo redujo de un fantasma con un planeta a un fantasma con una parcela de jardín. Ahora piensa en lo que la TARDIS hará con él.

Martha frunció el ceño.

---¿La TARDIS tiene imaginación?.

El Doctor se rió.

---Desde cierto punto de vista, la TARDIS es imaginación.

Metió la llave en la cerradura, la giró e hizo un gesto hacia la puerta abierta.

Martha la atravesó. La sala de la consola estaba tranquila. No había ni rastro del Ch'otterai. El Doctor pasó a su lado, se dirigió a la consola, movió unos interruptores y echó un vistazo a un par de lecturas.

---¿Dónde está? ---preguntó Martha.

---Se ha ido ---dijo el Doctor---.O prácticamente lo ha hecho, al menos.

Lo que puedo decir, es que ha sido reducido a un único pensamiento, y probablemente no a uno bueno.

Martha miró a su alrededor.

---Entonces, ¿dónde está?.

El Doctor agitó una mano en el aire.

---Aquí. Ahí. No lo sé.

---Pero cuando nos vayamos, no nos lo llevaremos con nosotros, ¿verdad?.

---¿Qué le pasa, señorita Jones?. ¿No le gusta la idea de una TARDIS embrujada?.

---No mucho.

---No te preocupes ---dijo el Doctor empujando palancas---. Volaremos a tierras lejanas, pero lo que queda del Ch'otterai se quedará aquí.

Probablemente.

No le gustó ese `probablemente', pero no dijo nada mientras la TARDIS empezaba a resollar y a zumbar. Se acercó a un monitor, vio la Nada fuera, doblada sobre si misma y parpadeada fuera de existencia. Por un momento no hubo más que un espacio vacío y estrellas, y luego la pantalla se quedó en blanco.

