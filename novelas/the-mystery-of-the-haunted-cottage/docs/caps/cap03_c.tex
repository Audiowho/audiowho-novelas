\chapter*{Capítulo Tres}
\addcontentsline{toc}{chapter}{Capítulo Tres}

Corrieron de vuelta a las escaleras y las subieron rápidamente, con Martha a la cabeza. Emergió a través de la puerta en lo alto, y entró en la brillante luz solar que se filtraba a través de los árboles. Estaba empapada y tiritando. El Doctor la siguió, cerró la puerta y usó el destornillador para bloquearla.

---¿Qué eran esas cosas? ---preguntó Martha.

---No tengo ni la más remota idea ---dijo el Doctor, cuyo pelo mojado colgaba sobre sus ojos. La cogió del brazo y la condujo a través de los árboles---. ¿Supongo que no aparecen en el libro?.

---Creo que los habría recordado.

---Era casi como si fuera una sección sin terminar. Nadie se había molestado en aventurarse por ahí en el libro, así que ¿qué sentido tiene que haya algo ahí abajo?. Todo aquí es artificio. Está aquí para mostrarse, pero no hay nada real o sustancial al respecto.

---¿Quieres decir como el escenario de una obra?.

Él giró la cabeza hacia ella, sonriendo.

---Eso es exactamente lo que quiero decir. Oh, eres inteligente. No tan inteligente como yo, pero, bueno\ldots{} ¿quién lo es?. Un escenario.

Para el público parecen construcciones y árboles, pero todo está sostenido por trozos de madera. Estamos en un escenario, Martha.

---¿Uno diseñado para parecerse a un libro para niños de la Tierra?. Un poco aleatorio, ¿no?.

El Doctor frunció el ceño, sacó el destornillador sónico de su chaqueta y empezó a escanear todo a su alcance.

---¿Qué estás haciendo? ---preguntó Martha.

El Doctor hizo algunos sonidos ininteligibles mientras corría de un lado a otro, entusiasmándose cada vez más mientras escaneaba. Emergió por la línea de árboles sobre la hierba, en la radiante y cálida luz del sol.

Martha corrió tras él. Podía sentir como ya se estaba empezando a secar.

---Wa-ey ---dijo, con los ojos muy abiertos---. ¡Wa-hey!.

---¿Doctor?.

---¡Martha!.

---Doctor, por favor dime que sabes qué está pasando.

El destornillador desapareció de vuelta al interior de su chaqueta.

---Podría saberlo ---dijo, barriendo su pelo mojado de la cara---. Es posible que pudiera saberlo. Tengo una idea. Pero no sé cómo\ldots{} Bueno, supongo que si\ldots{} A menos que\ldots{} No. Sí. ¿En serio?.

¡Sí!.

---¿Doctor?.

Se dio la vuelta, con el pelo colgando en locos ángulos.

---Esto, Martha Jones, esto, todo a nuestro alrededor. No creo que sea real.

---Pero dijiste que era real. Dijiste que todo se escaneó como real.

---Bueno, sí, pero está lo real y luego está lo real, ¿sabes?.

---No, no lo sé.

---¿Qué somo sino nuestros sentidos Martha?. Nuestros ojos nos dicen que estamos parados sobre una superficie plana, podemos sentir el suelo bajo nuestros pies, ¿pero y si nuestros sentidos nos están mintiendo?. Quita nuestra capacidad para tocar, saborear, escuchar, ver y oler, ¿y no cambia nuestro mundo en consecuencia?.

---Sí ---dijo Martha lentamente ---excepto que no lo hace, ¿cierto?.

Quita nuestros sentidos y estamos justo donde estábamos hace un momento, sólo que ahora no tenemos nuestros sentidos.

Él la miró.

---Le quitas toda la diversión a la filosofía, ¿lo sabías?.

---Soy una estudiante de medicina ---dijo Martha---. Trato con hechos.

Veo un problema y lo arreglo. Explícame esto en términos prácticos.

---De eso se trata, no creo que pueda. Todo este planeta parece ser una idea, un concepto hecho sólido.

---Así que todo este mundo en el que estamos de pie es\ldots{} ¿qué?.

¿Una historia?. No es un planeta en absoluto, ¿sino una historia?. ¿Cómo podemos estar de pie en una historia?. ¿Cómo puede una historia tener gravedad, o luz, o aire para que respiremos?.

El Doctor se encogió de hombros.

---Toda buena historia tiene una atmósfera.

---Voy\ldots{} voy a hacerte un favor y fingiré que no acabas de decir eso.

---Cierra los ojos.

---¿Por qué?.

---Martha\ldots{}

---Vale ---dijo, y cerró los ojos.

---Imagínate que eres una partícula de polvo flotando en el espacio ---dijo el Doctor mientras la rodeaba lentamente---. A tu alrededor, las estrellas nacen y mueren. Los planetas orbitan. Los meteoritos pasan, asteroides a la deriva, y de vez en cuando, si eres muy, muy afortunada, hay un destello de vida distante.

---¿Estoy sola? ---preguntó Martha.

---Eres una partícula de polvo ---dijo el Doctor---. Por supuesto que no estás sola.

---Sueno solitaria.

---Bueno, no lo estás, te lo estás pasando genial. Así que ahí estás, en el espacio, todas las cosas están ocurriendo, y alguien aparece con una agenda. Alguien aparece con un propósito. Llamémosle\ldots{} Bob. Y ello, a una pequeña partícula de polvo como tu, es esa maravillosa cosa nueva, de la que no tienes suficiente. Y por ello eres atraído al propósito de Bob, y giras a su alrededor con todas las otras partículas de polvo y todos los otros minúsculos elementos del cosmos, y de repente eres parte de algo mayor. Eres parte de una idea. Y creces y creces y cuando has acabado de crecer te das cuenta de que te has convertido en una idea.

---¿Puedo abrir ya los ojos?.

---Claro.

Martha lo miró. Estaba sobre un tronco, mirando a su alrededor. Mientras sus ojos estaban cerrados se había arreglado el pelo.

---Así que lo que estás diciendo es que Bob hizo este mundo con fuerza de voluntad y polvo.

---Fundamentalmente.

---¿Y por qué, por el amor de Dios por qué, lo hizo como un libro de los Buscaproblemas?.

---No creo que lo hiciera ---dijo el Doctor---. Nuestros sentidos nos dicen que estamos en un libro de los Buscaproblemas, mis sentidos me dicen que mi destornillador sónico me está diciendo que estamos en un libro de los Buscaproblemas, pero asumo que diferente gente de diferentes culturas son expuestas a una información sensorial diferente.

---¿Así que tiene este aspecto para nosotros porque yo he leído esos libros?- Pero he leído muchos más y mucho mejores, que los Buscaproblemas. ¿Por qué elegir este?. Y tú no has leído esos libros, ¿así que cómo puedes ver lo que yo estoy viendo?.

El Doctor saltó del tronco.

---Este mundo debe ser capaz de tomar solo una forma a la vez. Eligió una serie de historias que han estado en tu memoria durante más tiempo.

Tal vez sea porque fuiste la primera en salir de la TARDIS. Si yo hubiera salido primero, ahora podríamos estar en un antiguo cuento gallifreyano.

---Eso suena bien.

---En realidad no ---dijo el Doctor---. Nuestros cuentos tenían dientes.

---¿Quién es Bob?. ¿Cómo descubrimos quién está detrás de esto?.

---Elemental, mi querida Jones ---dijo el Doctor, metiendo sus manos en los bolsillos---. Usamos nuestros poderes de deducción. Ya tenemos nuestra lista de sospechosos.

---¿Crees que es uno de los personajes?.

---Un ser capaz de formar un mundo entero en torno a la ficción.

¿Realmente crees que un ser así pudo resistirse a insertarse en la historia?.

---Esta bien\ldots{} pero ¿quién es?. Si fuera yo, sería o bien el héroe o el villano. Y teniendo en cuenta que los héroes son un grupo de insufribles niños, diría que es el villano. Así que Bob es Cotterill.

---¿Tienes pruebas?. Esto es un misterio. Tienes que tener pruebas ¿Qué has observado sobre él?.

---Uh, bueno, él es\ldots{} tiene bigote y\ldots{} es un contrabandista, sabemos eso. Es igual que los demás. Excepto\ldots{}

---¿Excepto?.

Martha frunció el ceño.

---Excepto que él nos vio. Nos vio inmediatamente. Todo el mundo aquí necesita un momento para fijarse en nosotros. Pero él no. Él no es como los demás. Solo fingía serlo.

El Doctor sonrió.

---Sabía que lo tenías en ti, Martha Jones.

---¿Pero por qué esas cosas trataron de matarnos?. Todavía no hemos hecho nada.

---Quizás nos hemos desviado demasiado de la historia. Después de todo, fuimos donde se suponía que no debíamos ir. Así que esas cosas\ldots{}

---Podrían ser el sistema inmunológico de este planeta ---terminó Martha---. Fue detectada una infección en una zona vulnerable, y estos pequeños soldados espeluznantes fueron enviados para detenernos.

---Precisamente. Llegamos al límite y seguimos empujando. Te sorprendería cuanta gente reconoce que soy mucho más problemático de lo que merece la pena.

Martha se encogió de hombros sin comprometerse, entonces volvió al tema.

---Entonces, ¿qué vamos a hacer?. Lo más inteligente sería volver a la TARDIS e irse de aquí, pero sé que no vas a hacer eso.

Levantó la ceja.

---Crees que me conoces muy bien, ¿cierto?. Da la casualidad que pienso que esa es una maravillosa idea. No sabemos con qué estamos tratando, y como siempre digo: más vale prevenir que curar.

Martha frunció el ceño de nuevo.

---Sé que solo he estado viajando contigo unos meses, pero, literalmente, nunca te he escuchado decir eso.

---Tonterías. Lo digo todo el tiempo.

Ella sacudió la cabeza.

---Ni una sola vez.

---Puede que estuvieras en otra habitación cuando lo dijera ---dijo el Doctor---, pero lo he dicho, y lo he dicho muchas veces.

---¿Cuándo fue la última vez que lo dijiste?.

---La semana pasada. En el\ldots{} ooh, no me acuerdo. Fue con la cosa.

Bueno, cuatro cosas. Bueno, cuatro cosas y un lagarto. Lo dije entonces.

No es mi culpa si tu\ldots{} ¡Hey!.

Martha miró a su alrededor y vio a Gordito escondiéndose en los arbustos. Los ojos del pequeño niño gordo se abrieron cuando se dio cuenta que había sido descubierto, gritó y salió corriendo.

---Es realmente irritante ---murmuró Martha.

---En realidad ---dijo el Doctor---, él es exactamente lo que necesitamos.

Ella le miró.

---Sabías que estaba ahí.

---Por supuesto. Y ahora se ha ido a chivarse sobre nosotros, a dejar que su amo sepa que estamos a punto de irnos. Eso debería de incitarlo a la gran revelación.

---Oh, eres listo.

---¿Qué te he estado diciendo?. Vamos.

