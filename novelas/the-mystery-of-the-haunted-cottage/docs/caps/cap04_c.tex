\chapter*{Capítulo Cuatro} \addcontentsline{toc}{chapter}{Capítulo Cuatro}

Salieron del bosque y empezaron a caminar hacia la TARDIS. En la esquina de la cabaña, Cotterill estaba con una de las criaturas sin facciones.

Les estaban esperando.

---Allá vamos ---murmuro el Doctor, caminando. Martha lo siguió.

---¿Con que yéndose antes de que la historia se acabe? ---dijo Cotterill con una sonrisa en su rostro---. Debo admitir que estoy un poco decepcionado.

El Doctor le devolvió la sonrisa, con las manos en los bolsillos, una vez más.

---¿Esperando a que todo el mundo se reuniera en el salón para el gran desenlace?. Lo siento, pero estoy guardando eso para un misterio que se lo merezca. Me gusta tu amigo, por cierto. ¿Tiene nombre?.

---Los nombres están sobrevalorados ---dijo Cotterill. Ahora había otros, saliendo de entre los árboles, acercándose---. Estos son mis No-Hombres. Tienen sus usos. Aunque no son demasiado buenos para tener una conversación inteligente, a no ser que les de un personaje, aun así son muy limitados\ldots{} No como vosotros dos. Vosotros dos sois divertidos.

---También inteligentes ---dijo el Doctor---. Lo único que aún no hemos descubierto es lo que sacas tú de todo esto.

La sonrisa de Cotterill se ensanchó.

---Yo vivo.

---Así que te montas un sustento a partir\ldots{} ¿de qué?. ¿De ilusión?. ¿O de la gente que atrapas aquí?.

---Ambas. Y ninguna. La ilusión permite que la gente me de la fuerza que necesito. Dime, Doctor, ¿qué requiere cada historia de su lector?.

---La suspensión voluntaria de la incredulidad.

Cotterill sonrió.

---Exactamente. No tienes ni idea de la energía generada cada vez que alguien cuenta una historia. Cuando una mente consciente y sensible ignora voluntariamente lo que es real, lo que es la realidad, y en su lugar opta por invertir en personas y lugares que nunca existieron\ldots{} ---se estremeció de alegría---. Es magnífico. Viene a ser nada menos que un rechazo de la realidad. Y cuando la realidad es apartada, aunque sea brevemente, deja un espacio, crepitando con potencial, con lo que-podría-ser. Y lo que-podría-ser me hace fuerte.

Martha miró a su alrededor. Los No-Hombres se estaban acercando peligrosamente. El Doctor, por supuesto, apenas pareció darse cuenta.

---Así que tomaste esa fuerza e hiciste un planeta ---dijo el Doctor.

Cotterill se echó a reír.

---¿Esto? Esto es sólo el principio. Esto es un peldaño. Lo siguiente es construir un sistema solar. Luego, una galaxia. A continuación, un universo. Y cuando mi universo sea lo suficientemente grande, la realidad se romperá, y yo estaré allí para reemplazarla con la mía.

---Contigo como su dios.

---Exactamente.

---¿Y debería siquiera molestarme en preguntar qué pasará con la gente?.

---Estarán cuidados ---dijo Cotterill---. No soy un bárbaro. Después de todo, necesitaré gente que me alimente.

---¿Puedo decir algo? ---preguntó Martha dando un paso adelante---.

¿Puedo?. Gracias. Estás loco. Quiero decir, en serio, esta es una idea descabellada. No me refiero a que sea una locura porque nunca funcionará o porque te detendremos, me refiero a que es simplemente descabellado.

No es cuerdo. Vio venir a lo cuerdo y cruzó la calle para evitarlo.

Cotterill, estás loco. Doctor, tú estás loco sólo por tener esa conversación que acabas de tener. No te vas a apoderar del universo sólo con historias Cotterill. Es demasiado absurdo. No lo permitiré. Sobre esa base, más el hecho de que has estado intentando matarnos, vamos a poner fin a tu plan demente y enviarte a hacer las maletas. Doctor. Ya es hora de que tengas un plan para detenerlo. ¿Lo tienes?.

---Desde luego.

---¿Va a hacer falta correr?.

---Desde luego.

---Entonces vamos.

Empezaron a esprintar.

---¡Cogedlos! ---gritó Cotterill---. ¡Cortadles la cabeza!.

Dos No-Hombres se acercaron, con sus cuerpos estirándose y allanándose, convirtiéndose en cartas con miembros, como en Alícia en el País de las Maravillas pero llevando espadas. Martha se agachó bajo sus oscilantes hojas y el Doctor la cogió de la mano, tirando de ella. Corrieron a través de la puerta de la cabaña, y emergieron en el gran comedor de Hogwarts y entonces, antes siquiera de que una varita pudiera alzarse en su dirección, estaban huyendo por la puerta del otro extremo.

---¿Qué es esto? ---exclamó Martha.

---Cotterill está buscando por los libros de su memoria ---respondió el Doctor---, tratando de encontrar algo que nos detenga. ¡Por aquí!.

Corrieron hacia el exterior y la tierra se movió, convirtiéndose en ladrillos que giraban y dejando al descubierto una parte inferior de oro. Martha oyó el batir de alas y los chillidos de monos desde arriba, y el Doctor tiró de ella cuando algo grande y oscuro y peludo rozó su mejilla.

Perdió el equilibrio y se cayó por una colina cubierta de hierba, despatarrándose en la nieve. Se puso en pie la primera, mirando hacia atrás, asegurándose de que los No-Hombres no estaban a punto de abalanzarse sobre ellos.

---¿El león, la bruja y el armario? ---preguntó el Doctor, poniéndose de pie.

---No lo he leído ---dijo Martha. Corrieron hacia un edificio que estaba delante. Uno grande, como un hotel. Vio el nombre, el Overlook, y cambió de dirección---. Por ahí no.

El Doctor la siguió, dejaron la nieve atrás y se adentraron en un túnel.

Sus pasos resonaban en la oscuridad, disminuyeron su velocidad y miraron hacia atrás, tratando de recuperar el aliento.

---Este plan tuyo ---dijo Martha, jadeando en voz baja---. ¿De verdad tienes uno o lo dijiste sólo para parecer inteligente delante de Cotterill?.

---Tengo uno ---dijo el Doctor---. Y no necesito parecer inteligente.

Soy inteligente. El hecho de que parezca inteligente no es más que un plus.

---¿Podrías por favor decirme qué vamos a hacer?.

Él la agarró y Martha se quedó inmóvil. Delante de ellos, unos faros se encendieron como dos soles gemelos. La vibración de un motor potente resonó a su alrededor.

---¿Christine? ---dijo el Doctor.

---Peor ---dijo Martha, casi llorando---. Chitty Chitty Bang-Bang.

El coche rugió y avanzó hacia adelante. Martha y el Doctor se dieron la vuelta y echaron a correr.

---¡Me encantó la película, así que tuve que leer el libro! ---gritó Martha, la última palabra convertida en un grito cuando el Doctor tiró de ella hacia un lado. El coche se precipitó mientras ellos se toparon con una puerta estrecha y se internaron en un bosque.

El Doctor se llevó un dedo a los labios y Martha asintió y le siguió lo más silenciosamente que pudo. Las hojas mojadas se aplastaban bajo sus pies. Hubo un movimiento delante: dos adolescentes, un chico pálido y una chica nerviosa, entraban en un claro. El sol apareció por entre las nubes y el muchacho comenzó a brillar.

Martha sintió los ojos del Doctor en ella y se sonrojó.

---No me juzgues.

---El juicio es para más tarde ---dijo, y continuaron, dando un gran rodeo a los jóvenes amantes.

Salieron de los árboles, de nuevo a la luz del sol. La TARDIS estaba justo en frente de ellos.

Martha echó a correr, con el Doctor detrás de ella. Estaban a mitad de camino y no había señales de ningún No-Hombre. Iban a conseguirlo. Iban a\ldots{}

Una columna de roca brotó de la tierra, llevándose la TARDIS con ella.

Martha se abalanzó ---no sabía por qué, simplemente se abalanzó---, y se agarró a ella, dándose cuenta de la estupidez de su decisión sólo cuando la columna empezó a retorcerse a la vez que se alargaba. Siguió creciendo, el terreno se convirtió en algo lejano, el ladrillo se convirtió en roca, la columna se convirtió en una torre. Finalmente, el crecimiento se ralentizó y se calmó. La torre tenía a la TARDIS colgada precariamente en la parte superior y a Martha colgando de un estrecho alféizar. Miró hacia abajo. Oh, había un largo camino hacia abajo.

---En retrospectiva ---gritó el Doctor desde muy por debajo de ella---, probablemente eso fue una mala idea.

