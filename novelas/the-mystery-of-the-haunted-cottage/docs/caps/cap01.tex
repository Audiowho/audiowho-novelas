\chapter*{Capítulo Uno}
\addcontentsline{toc}{chapter}{Capítulo Uno}

---Bueno ---dijo el Doctor mirando al monitor con los ojos muy abiertos---, esto es interesante.

Martha Jones se apresuró a unirse a él en la consola, mientras que la TARDIS resollaba y zumbaba a su alrededor.

---¿El qué? ---preguntó mirando a la pantalla---. ¿Niebla?. ¿Qué tiene de interesante la niebla?. ¿Dónde estamos?.

---De eso se trata ---dijo el Doctor, hablando tan bajo que casi estaba murmurando---. No estamos donde podríamos estar.

---¿No querrás decir que no estamos donde deberíamos estar?.

---No. Significa lo que dije. No podemos estar aquí. No podemos haber aterrizado aquí. La última vez que pasé por aquí, este era un espacio vacío ---sacó el destornillador sónico de su chaqueta, que hoy era la de rayas marrones, con la corbata azul. Echó un vistazo a los instrumentos y frunció el ceño---. No hay mal funcionamiento ---dijo---. Las lecturas son correctas, así que esto es definitivamente un planeta. Atmósfera respirable, también.

Las lecturas del escáner no significaban nada para Martha, y el monitor todavía mostraba niebla.

---¿Cuándo fue la última vez que pasaste por aquí? ---preguntó---. Tal vez el planeta fue destruido en los años intermedios, o cambió su órbita o algo así.

El Doctor estaba demasiado atrapado en sus propios pensamientos para responder, así que Martha suspiró, cogió su chaqueta y se la puso mientras se dirigía a la puerta. Ya lo conocía lo suficiente como para saber que, de todas forma, iba a salir.

Giró el pestillo, y el Doctor giró la cabeza.

---Martha, espera.

Abrió la puerta y salió, y su cabeza le dio vueltas durante un momento.

Cuando se aclaró, la niebla había desaparecido.

---Huh ---dijo.

El Doctor se le unió. Sol, hierba verde, cielo azul. Algunas nubes blancas y esponjosas. Los pájaros cantaban. Era cerca del mediodía, a juzgar por la posición del único sol.

Martha miró a su alrededor.

---Esa niebla desapareció rápido.

El Doctor no respondió. En su lugar, empezó a caminar hacia la colina más cercana, siguiendo el sonido de unas voces. Llegaron a la cima y vieron a cuatro niños caminando hacia ellos, dos niñas y dos niños, todos de unos once o doce años. El chico más alto llevaba una cesta de picnic, y estaban vestidos como si pertenecieran a la década de 1950, de la Tierra.

---Hola ---dijo el Doctor sonriendo alegremente.

Los niños se detuvieron sin mirarlo y fruncieron el ceño, como si hubieran oído algo muy lejano.

Martha se puso delante de ellos, agitando la mano.

---¿Hola?. ¿Hooola?. ¿Podéis vernos?.

Los ojos del chico más alto se movieron, encontraron su mano saludando y se centraron. Luego alzó la vista, la vio y sonrió.

---Oh, ¡hola!.

---Hola ---dijo Martha.

Ahora los demás la miraban a ella y al Doctor, y también sonreían.

---¡Cielos!. ¡gente nueva! ---dijo la chica de pelo oscuro---. ¡Hace siglos que no vemos gente nueva!. ¿Cuáles son vuestros nombres?.

Había algo en su cara que se enganchó en la memoria de Martha. Todas sus caras, en realidad.

---Soy Martha ---dijo---, y este es el Doctor.

---Hola, Martha ---dijeron todos juntos---. Hola, Doctor.

Martha les sonrió y miró hacia atrás.

---Esto es un poco raro, ¿no?.

---Mucho ---dijo el Doctor---. ¿A dónde vais niños?.

---¡Nos vamos de picnic! ---dijo orgullosamente la niña más pequeña---.

Madre y Padre suelen venir con nosotros, ¡pero este año dijeron que eramos lo suficientemente mayores como para ir solos!. Nunca hemos ido tan lejos por el camino. ¡Espero que seamos capaces de encontrar el camino de vuelta!.

---Deberíamos de ser capaces de hacerlo ---dijo el niño más pequeño---.

Es un camino recto.

---Entonces os dejaremos volver a vuestro picnic ---dijo el Doctor, e inmediatamente fueron acometidos por un coro de despedidas que hicieron a Martha dar un paso hacia atrás. Y al momento siguiente los niños estaban caminando de nuevo, y charlando y riendo entre sí como si nunca hubieran sido interrumpidos. El camino por el que iban continuó por la colina, viró ligeramente alrededor de una antigua cabaña y desapareció en el bosque de detrás. Todo el asunto parecía un cuadro de una postal barata, y con los niños en primer plano a Marta le recordó a\ldots{}

---Los Buscaproblemas ---dijo.

El Doctor la miró.

---¿Disculpa?.

---Los libros. Los libros de los Buscaproblemas. ¿Nunca los has leído?.

---Los Buscaproblemas ---dijo el Doctor---. Treinta y dos libros para niños, escritos por Annette Billingsley a lo largo de quince años, desde 1951 hasta 1966. No, nunca los he leído. Eran basura. Plagio de los Los Cinco y Los Siete Secretos. Ah, Enid Blyton. La conocí una vez, ya sabes. Una mujer rara. Inusuales orejas.

---Bueno ---dijo Martha, hablando con rapidez antes de que el Doctor se pudiera ir por otra de sus tangentes---, yo he leído los Buscaproblemas.

Los devoraba. Desde el Juramento de los Buscaproblemas impreso en la página del título a la lista de los otros libros en la parte posterior, leía cada parte. Y esto, esto aquí, es la portada de El Misterio de la Cabaña Encantada. Es el primero que leí. Esta casa, este punto de vista, esta hora del día\ldots{} todo. Y esos niños. Los conozco a todos. El más alto es Humphrey; es el sensato líder. La chica con el pelo oscuro es Joanne, pero ella insiste en que todo el mundo le llame Jo porque es la marimacho. Luego está Simon; está siempre tratando de probarse a sí mismo, por lo que es el que más se mete en problemas. Y la más joven es Gertie. Hace bollitos con mermelada.

---Bollitos con mermelada. Ya veo.

---Y por lo general son seguidos por\ldots{} ah, ahí está ---Martha señaló con la cabeza a un árbol cercano, donde un niño se escondía, asomándose ocasionalmente---. El pequeño niño gordo.

El Doctor levantó una ceja.

---¿Qué? ---dijo Martha a la defensiva, manteniendo su voz baja---. Así es como se le describe en los libros. No me culpes. Era 1951. Todo en aquel entonces era de miras estrechas, sexista y siempre ligeramente racista. Era una época retrograda.

---Ah, sí ---dijo el Doctor---, porque 2007 no tiene ninguna de esas cosas.

Martha le ignoró.

---El pequeño\ldots{} niño con sobrepeso quería unirse a los Buscaproblemas, pero siempre demostró ser demasiado molesto, y cada vez que lo despachaban corría y se chivaba de ellos. ¿Cuál era su nombre?.

Lo tengo en la punta de la lengua. Era un apodo, algo que todos lo llamaban, incluso su tía y su tío\ldots{}

El Doctor suspiró.

---¿Por casualidad no sería Gordito?.

---Eso ---dijo Martha asintiendo---. Gordito. Sí, es él.

---Los niños pueden ser crueles ---dijo el Doctor---. Los escritores infantiles pueden ser aún peor.

---Doctor ---dijo Martha, que no tenía más remedio que hacer la pregunta---, ¿estamos\ldots{} estamos en un libro?.

---No estamos en un libro. No podemos estar en un libro ---el Doctor miró a su alrededor---. Podríamos estar en un libro ---empezó a caminar.

Martha le siguió.

---¿Cómo es eso posible?.

---Ha pasado antes ---dijo. Luego se encogió de hombros---. Bueno, más o menos. Hace mucho tiempo. Bueno, no hace tanto, relativamente hablando, y dado que el tiempo es relativo, relativamente hablando es como hablamos, ¿no? ---el Doctor cogió una margarita mientras caminaba, y la sostuvo en alto hacia el sol, examinándola---. Esencialmente era un universo de bolsillo, donde los personajes de ficción eran reales.

Conocí a todo tipo de personas. Gulliver, Cyrano de Bergerac, los tres mosqueteros, Medusa. Incluso Rapunzel. Aunque\ldots{}

---¿Aunque?.

---Viajar a la Tierra de Ficción significaría dejar nuestro propio universo, y no lo hemos hecho.

---¿Cómo lo sabes?.

Él se detuvo y dejó caer la margarita.

---El aire tiene un sabor raro en otros universos. Parecido a col hervida y perro mojado. Por lo tanto, no estamos en la Tierra de Ficción, pero parece que sí estamos en una tierra que está replicando una obra de ficción. ¿Puedes recordar cómo terminaba esta historia en particular?.

---Lo siento ---dijo Martha---. Lees más de tres libros de los Buscaproblemas y todos se desdibujan en un gran lío de color rosa.

---Excelente ---dijo el Doctor, sonriendo---. Entonces resolveremos este nosotros mismos. Vamos.

Bajaron la colina tras los Buscaproblemas, que ni siquiera miraban alrededor al sonido de sus pasos. Sin duda no eran unos niños muy despiertos, en opinión de Martha. Quizás el Doctor tenía razón. Quizás los libros eran una absoluta basura.

A Martha no le importaba. Los había amado cuando era más joven. Después de los Buscaproblemas y cosas así, era todo Harry Potter y La materia oscura, y posteriormente estaba leyendo libros de adultos y probando los clásicos\ldots{} pero los Buscaproblemas es donde todo empezó.

Ella echó un vistazo detrás de ellos. Gordito los estaba siguiendo, corriendo de árbol en árbol y de árbol a arbusto, haciendolo lo mejor posible para mantenerse oculto y fallando miserablemente. De repente sintió pena por él, ese pobre chico queriendo desesperadamente ser parte de los Buscaproblemas.

---¡Paremos aquí para el picnic! ---anunció Humphrey.

---Oh, sí, ¡vamos! ---chilló Gertie.

---¡Me muero de hambre! ---gritó Jo.

---¡Elegante! ---gritó Simon.

---Siete infiernos ---murmuró el Doctor.

---Simón, pongamos la manta ---dijo Humphrey---. Chicas, empezad a desempaquetar la cesta, ¡y luego todos a engullir!.

Martha se puso al lado del Doctor y observaron a los niños acomodarse en un festín de bollitos y cerveza de jengibre y manzanas y tarta y sándwiches de jamón y tartas y empanadas de Cornualles y pasteles de crema. Gertie y Jo sacaron tanta comida que Martha empezó a sospechar que la cesta tenía algunas cualidades de la TARDIS.

---Tengo hambre ---dijo ella.

---Deberíamos haber traído nuestro propio picnic ---respondió el Doctor asintiendo.

---Esas tartas parecen deliciosas.

---¿Verdad?. Y nunca he querido una cerveza de jengibre más de lo que lo hago ahora.

Algo llamó la atención del Doctor y empezó a caminar. Martha miró fijamente a unas tartas de mermelada que parecían las mejores tartas de mermelada que nunca se habían horneado, luego se apartó de los Buscaproblemas y Gordito, y de mala gana lo siguió. Había una mujer mayor en una anticuada bicicleta pedaleando hacia ellos. Tenía el pelo gris, un vestido de flores con un cárdigan claro sobre él, y una sonrisa de satisfacción en su rostro.

El Doctor se puso en medio del camino, con las manos en las caderas y las piernas separadas. La anciana siguió pedaleando.

---Um ---dijo Martha.

Una mariposa pasó frente a la anciana, y ella giró la cabeza y la vio revolotear.

---Hola ---dijo Martha en voz alta---. ¿Hola?. Estamos aquí. ¡Hola!.

---Le va a tomar un momento registrar nuestra presencia ---dijo el Doctor---. Después de todo, no pertenecemos a esta historia.

La mujer mantuvo su lento ritmo de pedaleo y se acercó más y más. Sólo en el último momento sus ojos se centraron, y vio al Doctor bloqueando el camino. Ella gritó y retorció el manillar, se desvió del camino y cayó por la colina. Marta corrió hasta el Doctor, y observaron a la anciana caerse de la bici con un graznido de pánico y rodar hasta la base de la colina, donde paró de forma moderada.

---¿Estas bien? ---chilló Martha.

---Estoy bien ---dijo el Doctor---. Ni siquiera me golpeó.

Martha lo fulminó con la mirada, y luego se apresuró a bajar hacia la anciana conforme ella se sentaba.

---Ten cuidado ---dijo Martha---. Puede que estés herida.

La anciana miró a su alrededor, frunciendo el ceño, y se quedó mirando a Martha por unos momentos antes de lograr verla. Una sonrisa temblorosa emergió.

---Oh, no te preocupes por mí, querida. En mis tiempos tuve peores caídas, y me atrevo a decir que las tendré peores. Pero si me pudieras ayudar a levantarme te estaría muy agradecida.

Martha se puso detrás de ella y la levantó mientras el Doctor se les unió, llevando la bici a su lado.

---Mira lo que he encontrado ---dijo alegremente.

---¡Es mi bicicleta! ---dijo la anciana---. Muchas gracias, joven.

---Oh, es lo menos que puedo hacer ---dijo el Doctor, todo encanto y sonrisas---. Soy el Doctor y esta es Martha. ¿Usted es\ldots{}?.

---La señora O'Grady ---dijo la anciana---. Mucho gusto.

---La señora O'Grady ---repitió el Doctor---. ¿Por casualidad no tendrá un nombre?.

---Sí, lo tengo ---dijo la señora O'Grady---. Señora.

---Que personaje más completamente desarrollado ---dijo el Doctor, alzando una escéptica ceja a Martha.

Ella le ignoró.

---¿Vive cerca? ---preguntó, y la anciana asintió.

---Vivo en la siguiente cabaña.

---Entonces tal vez nos pueda ayudar ---dijo el Doctor---. Hay un misterio que resolver en algún lugar de por aquí y no podemos encontrarlo. Cierto es que apenas acabamos de empezar a buscar, pero usted parece ser alguien que podría saberlo.

---¿Un misterio? ---dijo la señora O'Grady---. Cielos, no, me temo que no puedo ayudarte. Me esfuerzo por mantenerme alejada de los misterios, joven. Cosas terribles. Conducen a todo tipo de\ldots{} respuestas ---un escalofrío le recorrió el cuerpo.

---En efecto ---dijo el Doctor---.¿Así que no ha notado nada inusual?.

---¿Inusual?.

---Cosas raras ---dijo Martha---. Sucesos inexplicables. Ruidos.

Comportamiento criminal. ¿Cualquier cosa fuera de lo común?.

---No ---dijo la señora O'Grady---. Nada. Aparte de las extrañas luces en el bosque.

---¿Extrañas luces? ---dijo el Doctor.

---¿En el bosque? ---dijo Martha.

---Oh, sí ---dijo la señora O'Grady---. Cada noche veo extrañas luces flotando entre los árboles. Hay un montón de gente diciendo que son fantasmas, pero yo no creo en fantasmas. Finalmente llegué a la conclusión de que eran luces extrañas.

El Doctor frunció el ceño.

---¿Esa fue su conclusión?.

---Sí.

---¿Le satisfizo?.

---Por supuesto ---dijo la señora O'Grady---. Porque eso es lo que son.

Luces extrañas.

El Doctor la miró de arriba abajo.

---¿Que se siente al ser tú? ---murmuró---. ¿Qué se siente al tener una curiosidad tan limitada?.

---Me las arreglo ---dijo la señora O'Grady, riéndose entre dientes.

---¿Nunca pensó en investigarlas? ---preguntó Martha.

---No ---dijo la señora O'Grady, abriendo mucho los ojos, como si la misma pregunta la llenara de terror---. Fíate de mis palabras, nada bueno puede venir de entrar en ese bosque por la noche. Luces extrañas significan cosas extrañas. ¿Y quién querría cosas extrañas, aparte de gente extraña?. Y no somos gente extraña, ¿verdad?.

Martha dudó y miró al Doctor.

El Doctor sonrió.

---No ---dijo---. No somos para nada gente extraña.

Dejaron a la señora O'Grady continuar su feliz camino, y Martha siguió al Doctor a la cabaña embrujada. Ahora que podía verla de cerca, no parecía ni remotamente encantada. En realidad era bastante bonita. Era bonita y estaba pintada de blanco y tenía un grueso techo de paja.

---La primera regla de ser un detective ---dijo el Doctor mientras llamaba a la puerta---, es observar. Observar lo obvio, y observar lo no tan obvio. Observar lo no tan obvio no es tan fácil como observar lo obvio, pero si fuera fácil todo el mundo sería detective.

La puerta se abrió. Martha observó que un hombre de unos cuarenta años estaba allí de pie. Observó que su cara era larga y que su bigote estaba cuidadosamente cuidado.

---¿Puedo ayudaros? ---preguntó.

---Hola ---dijo el Doctor, sonriendo ampliamente y moviendo la mano del hombre en el más entusiasta apretón de manos que Martha nunca había observado---. Ella es Martha Jones, y yo soy el Doctor. Encantado de conocerle. Debe de ser alguien importante para estar trabajando aquí.

¿Quizás una maniobra de distracción?. ¿O el malo de la película?. No, no, no me lo diga, deje que lo descubra. ¿Tu nombre?.

El hombre hizo todo lo posible para recuperar su mano.

---Cotterill ---dijo, un tanto desdeñoso---. Soy el cuidador.

---El cuidador ---dijo el Doctor con los ojos abiertos por el asombro---. En ese caso, es doblemente bueno conocerle ---se volvió hacia Martha---. Cuidadores y mayordomos. Cuidado con ellos.

Martha adoptó una sonrisa que era mucho menos maníaca.

---Señor Cotterill, nos preguntábamos si podríamos hacerle unas cuantas preguntas sobre las extrañas luces en el bosque.

Inmediatamente la cara de Cotterill se puso tensa.

---¿Extrañas luces?. ¿Qué extrañas luces?. No hay nada extraño en las luces. Son luces. ¿Qué tiene de extraño una luz?.

---¿Qué están haciendo en el bosque?.

---Nunca dije que las viera en el bosque. Nunca dije que sabía de lo que estabas hablando. ¿De qué estás hablando?. ¿Extrañas?. ¿Luces?.

¿Bosque?. ¿Qué?. Soy el cuidador. Cuido de la cabaña y del campo. No del bosque. No de algunas luces extrañas. No es que las haya. Pero si hubiera alguna, no sabría nada de ellas. Y no lo hago, porque no las hay ---su cara se crispó---. En absoluto.

El Doctor sonrió de nuevo.

---Yo te encuentro para nada sospechoso ---se giró hacia Martha y susurró---. Lo encuentro increíblemente sospechoso.

---He oído eso ---dijo Cotterill.

---Se suponía que debía hacerlo ---dijo el Doctor, girándose hacia él---. Le estaba dando una falsa sensación de seguridad. Ahora que ya está suficientemente seguro, aumentaré la intensidad de mi interrogatorio. Señor Cotterill, esas presuntamente extrañas luces en el bosque a la que sigue haciendo referencia, ¿qué son?.

---No sé de lo que\ldots{}

---Responda a la pregunta, señor Cotterill ---dijo el Doctor, repentinamente enfadado---. Las luces. En el bosque. Las extrañas. ¿Qué son?.

---¡No lo sé!.

---¡Así que admite que existen!.

---¿Qué?. ¡No!.

---¿Es usted el responsable de las luces señor Cotterill?. ¿Sabes quién lo es?. ¿Cuál es su plan secreto?. ¿Qué quiere?. ¿Tras qué anda?. ¿Qué esconde, señor Cotterill, si ese es su verdadero nombre?. ¡Responda a la pregunta señor Cotterill!.

---¡Lo es!.

---Lo\ldots{} ---el Doctor se detuvo y parpadeó---. De acuerdo, he perdido la pista a la pregunta a la que esa era una respuesta.

---Señor Cotterill ---dijo Martha, sonriendo mientras avanzaba---.

¿Puedo preguntarle algo?. Como cuidador, ¿alguna vez ha sido testigo de algo\ldots{} inusual?. Actividad inusual, acontecimientos inusuales, ¿visitantes inusuales?.

---¿Quieres decir como vosotros dos?.

---Como nosotros ---dijo Martha asintiendo---, pero más inusuales aún.

---No ---Cotterill enfáticamente---. Aparte de vosotros dos no he visto nada inusual, especialmente nada de fantasmas.

---Nunca mencionamos fantasmas.

---¡Genial! ---dijo Cotterill---. ¡Porque no hay tal cosa!.

Superstición, ¡eso es lo que es!. Ahora, si me disculpáis, ¡tengo un montón de trabajo que hacer! ---retrocedió y dio un portazo.

---No estoy segura ---dijo Martha---. ¿Fue bien o no?.

---Fue brillante ---dijo el Doctor dijo, radiante---. Ahora, ¿qué te parece si vamos a investigar ese bosque?.

