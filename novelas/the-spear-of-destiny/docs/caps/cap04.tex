\chapter*{Capítulo Cuatro}
\addcontentsline{toc}{chapter}{Capítulo Cuatro}

---¡Bueno, aquí estamos! ---anunció el Doctor--- Segundo piso, sala de
exposiciones de la Colección Moxon. ¡Voilà!

Tiró de la puerta abierta de la TARDIS teatralmente, con una amplia
sonrisa a Jo, quien frunció el ceño y alzó su dedo al aire, apuntando
hacia el exterior.

El Doctor se volvió.

---¡Maldición! ---dijo en voz alta y luego en voz más baja--- ¿No
podrías aterrizar dónde se suponía, vieja amiga? ¿Sólo una vez?

Jo se asomó y contempló las vistas.

---Parece que estamos en un techo. El techo del museo, en realidad. No
está nada mal.

---Bien, es cierto ---dijo el Doctor.

Fuera estaba el nocturno horizonte de Londres. Podían ver las luces del
Piccadilly Circus y, un poco más adelante, la Columna de Nelson
entablada en la oscuridad.

---¡Muy bien! ---declaró el Doctor--- Aún estamos en 1973, después de
todo. Podemos deslizarnos al interior desde aquí arriba muy fácilmente
---buscó en su bolsillo y sacó el destornillador sónico---. Tiene que
haber algún tipo de trampilla para el acceso a la azotea ---agregó.

Jo le tiró de la manga.

---¡Ahí! Mira.

---Excelente ---dijo el Doctor---. Jo, ¿te importaría llevar nuestro
señuelo?

A pocos metros, en el techo del edificio, había una pequeña puerta que
daba a la azotea. El Doctor puso el destornillador sónico contra la
cerradura durante un segundo y esta se abrió.

Bajaron por la estrecha y oscura escalera, al final de la cual había una
puerta más grande. Una vez más, el destornillador hizo su trabajo, y
entraron en el museo.

---Un piso más ---dijo el Doctor en voz baja---. Estate atenta a
cualquier ruido, por si acaso.

Jo asintió, agarró la lanza con un poco más de fuerza, y se dispuso a
bajar las escaleras las anchas y enmoquetadas escaleras.

Cerca del final de la escalera el Doctor hizo una pausa, y luego señaló
a la puerta de la habitación que habían visitado esa tarde. Se quedó
quieto en el último escalón, tenso, escuchando atentamente. Un momento
después se relajó y sonrió.

---Bueno, ---dijo--- creo que no hay moros en la costa.

Bajó al descansillo y el estruendo de una alarma estalló sobre ellos,
ensordecedor y estridente.

Por las baldosas de mármol de la planta baja resonaron pasos, cada vez
más cerca, y una voz les gritó

---¡Quédense exactamente donde están o disparo!

Se dieron la vuelta y vieron a un guardia apuntándoles con una pistola.
No era uno de los guardias de seguridad que habían visto durante su
visita aquella tarde, sino uno vestido de uniforme casi militar que
adoptaba una postura como si fuera a disparar en cualquier momento.

---¿Qué quieres decir con que vas a disparar? ---rugió el Doctor--- ¡No
seas rídiculo! ¡Esto es un museo, no un campo de tiro!

Se volvió hacia Jo.

---Vamos. Creo que deberíamos irnos.

---No te muevas ---vociferó el guardia. Se oían más guardias subiendo
por las escaleras, y el Doctor tomó la mano de Jo.

---¡Dispararé! ---gritó el guardia.

---No lo harás ---dijo el Doctor, confiado, mientras retrocedía por la
escalera.

La pared detrás de su cabeza explotó en una nube de de yeso que parecía
llegar a ellos antes de que oyeran el disparo.

--- ¡Corre! ---exclamó el Doctor, y echaron a correr por las escaleras
en dirección al tejado. Sonaron más disparos y la pared que había encima
de sus cabezas estalló mientras corrían, pasando en cuclillas por la
puerta de la pequeña escalera.

Los disparos de las pistolas quedaron repentinamente ahogados por las
ráfagas metálicas de una ametralladora.

---¡Absurdo! ---exclamó el Doctor, subiendo de dos en dos los escalones
hacia la azotea. Se oyeron más gritos y el sonido de las botas en la
escalera resonaron tras ellos.

Los disparos rebotaban en el techo mientras pasaban por la pequeña
puerta y salían de nuevo al aire frío de la noche.

---¡A la TARDIS, Jo! ---gritó el Doctor--- ¡Rápido!

Entraron rápidamente y cerraron la puerta. El Doctor se abalanzó sobre
la consola central y los encerró de forma segura en el interior. El
distante sonido de los disparos fuera de la TARDIS se acercó a ellos,
como abejas chocando contra el grueso cristal de una ventana.

---No alarguemos nuestra bienvenida ---dijo el Doctor mientras se
afanaba por establecer las coordenadas.

---Creo que ya lo hemos hecho ---dijo Jo. Dejó la lanza junto a la
puerta y corrió hacia donde estaba el Doctor.

El sonido de los disparos fue sustituido por el chirrido familiar de la
desmaterialización, y Jo sintió una enorme sensación de alivio. Se dio
la vuelta y se apoyó en el borde de la consola.

---Sí, ha ido de poco ---dijo el Doctor---. Aún así eso prueba una cosa.

---¿Que es\ldots{}?

---Que la lanza es algo inusual. Nadie llegaría a tales extremos para
protegerla si sólo fuera un viejo pedazo de madera y oro.

---Tal vez Moxon es muy protector de su colección.

---¿Ametralladoras? Eso es llevar la custodia del museo un poco lejos,
¿no crees?

---Supongo que sí ---dijo Jo---. En fin, ¿dónde vamos?

El Doctor sonrió

---Una muy buena pregunta.

---Con una muy buena respuesta, espero.

---No podemos robar la lanza ahora, pero podemos hacerlo en el pasado.
Por tanto, vamos a viajar a su único otro lugar confirmado en el
espacio-tiempo.

---¿Que es\ldots{}?

---¿No leíste la descripción que había junto a la urna?

Jo negó con la cabeza.

---Estaba demasiado ocupada tratando de entender a Futhark.

---Bueno, ¿todavía tienes el folleto del museo?

Jo metió la mano en su bolsillo trasero y sacó un trozo de papel
arrugado. En él encontró una breve descripción de la lanza.

Lanza ceremonial. Encontrada en Gamla Uppsala, Suecia. Se cree que fue
utilizada en festivales durante el equinoccio de primavera en el siglo
II de nuestra era. La inscripción sobre la cabeza decía Gungnir. En la
mitología nórdica, Gungnir era la lanza mágica de Odín.

---¿Nos estás llevando a ver a los vikingos? ---preguntó Jo con
incredulidad.

---¡Lo sé! Maravilloso, ¿no? ---dijo el Doctor con una sonrisa.

---Esa no es la palabra que usaría ---dijo Jo---. Hey, espera un minuto,
¿cómo sabes a dónde ir?

---Fácil ---dijo el Doctor---. Echa un vistazo a tu folleto. Uppsala,
Suecia central. O la antigua Uppsala para ser exactos. Centro de poder
de los reyes suecos durante más de mil años hasta que los cristianos
llegaron. Ahí es donde. Cuándo es un poco más difícil. Sabemos que
debemos dirigirnos hacia el equinoccio de primavera\ldots{} Bien por
parte de los vikingos por datar las cosas alrededor de los fenómenos
astronómicos. Hace la vida mucho más fácil.

---Pero ¿en qué año?

---Bueno, estoy tratando de averiguar. En el Museo Británico hay una
piedra rúnica que menciona la única otra referencia conocida a Gungnir.
Se refiere a una ceremonia en la Vieja Uppsala que menciona la aparición
de un segundo sol a través del cielo. Los estudiosos siempre han asumido
que eso es una referencia al cometa Halley, cuya única aparición
conocida fue en el siglo II fue en 141 d.C., según el antiguo calendario
juliano, eso era el veintidós de marzo, el día siguiente al equinocio.
Así que eso es cuándo y dónde estamos yendo.

---Oh ---dijo Jo---. Ya veo.

---Bien\ldots{}

---Sólo tengo una pregunta.

---¡Dispara!

---Oh, Doctor, por favor. No después de lo que pasó en el museo.

El Doctor levantó la mano.

---Lo siento. ¿Cuál es tu pregunta?

Jo tragó saliva.

---Así que, escucha. Esta lanza. La lanza mágica de Odín. Puede que me
equivoque, ¿pero Odín no era un Dios?

---Eso es lo que dicen.

---Bueno, ¿eso no te preocupa en absoluto?

---Al contrario. Bastante divertido, diría yo.

---¿Divertido? ---preguntó Jo, observando la lanza por la puerta,
nerviosa--- ¿De verdad crees que el dueño de Gungnir era un dios?

El Doctor volvió a sonreír.

---Supongo ---dijo--- que estamos a punto de averiguarlo.