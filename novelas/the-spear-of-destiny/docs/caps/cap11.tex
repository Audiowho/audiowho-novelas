\chapter*{Capítulo Once}
\addcontentsline{toc}{chapter}{Capítulo Once}

El Doctor se alejó del asentamiento de Njord camino del río y el templo,
y, más allá, hacia el pueblo de Odín.

Se movía en silencio y, a pesar de que la noche debería haber sido
oscura con el último cuarto de luna en el cielo, éste estaba iluminado
por un resplandor misterioso. Al mirar hacia el cielo vio un cometa y
supo cuánto confiaban los vikingos en tales portentos: el momento
perfecto para que Odín hiciera un sacrificio que asegurara la victoria
en la guerra que se estaba gestando.

Se apresuró por la pendiente, a través de los árboles y del río. En la
penumbra, las ruedas hidráulicas seguían girando, y aunque el Doctor
sabía que tenían algún propósito tramado por el Maestro, no había tiempo
para detenerse e investigar. A lo lejos las luces de la aldea de Odín
brillaban, y el Doctor avanzó con rapidez.

Al llegar a la cima de la colina donde estaba el templo, tropezó con
algo. Miró hacia abajo y vio más cableado eléctrico serpenteando a
través de los árboles, y aunque el tiempo estaba en su contra, lo
siguió.

No pasó mucho tiempo hasta que averiguó hacia donde se dirigía. Iba
hacia un pequeño túnel excavado en la ladera, justo debajo del lateral
del templo.

~

Se adentró arrastrándose y, aunque en teoría debería estar oscuro, vio
una luz. Mano sobre mano se adentró siguiendo el cable hasta que, de
repente, el túnel se abrió en una pequeña cueva artificial. En su centro
había otra caja de metal, como la que había en la cámara del timonel, si
bien ésta tenía unas luces en su lateral que parpadeaban como latidos
del poder que hubiera en su interior.

Jo había logrado cortar sus ataduras sin mayores problemas, dándose
cuenta de que los cuchillos de mesa vikingos eran más armas letales que
cubiertos.

La noche era tranquila y el pueblo estaba extrañamente tranquilo, aunque
podía oír a la gente de las casas al pasar a su lado. Sabía a dónde iba.
Como el circuito camaleón de TARDIS del Doctor todavía estaba roto,
debía de estar en un edificio lo suficientemente grande como para
albergar una caseta de policía de 1963, y sólo había uno de esos: la
sala en la que había conocido a Njord antes.

Se arrastró hacia dicho edificio y vio que las luces estaban apagadas.
Encontró una puerta y entró, la sala central estaba rodeada de
corredores y galerías. Jo avanzó a lo largo de la primera de ellas en
busca de la forma familiar de la TARDIS.

Giró una esquina, y luego otra, pero no encontró nada hasta llegar a una
habitación en la que un poco de luz luna se derramaba por una ventana
sin cristal situada en lo alto de la pared.

La luz le mostró algo muy hermoso, un gran modelo de madera de un barco
vikingo, casi tan largo como alta era ella. Era de cubierta ancha, bajo
y elegante, y con muchos detalle y, a pesar de sí misma, se quedó
mirándolo como si alguna magia la atrajera.

Alargó una mano para acariciar la cabeza del dragón tallada en la proa,
y vio que el interior del modelo tenía incluso más detalles: los bancos
en los que los remeros se sentaban, la barra que controlaba el timón. No
pudo resistir la tentación de tocar el timón y al hacerlo soltó una
exclamación porque éste se quebró en su mano.

---¡Tu! ---gritó una voz en la oscuridad---. ¡Tu! ¿Qué haces aquí?

Alguien se acercaba.

Jo se dio la vuelta y se metió el pequeño trozo de madera en el bolsillo
de atrás, al tiempo que se giraba y le dedicaba una mirada culpable al
enorme escandinavo que la miraba con severidad.

---Oh, no ---dijo ella.

---Te acabo de atar ---dijo el vikingo con un gruñido---. Ven conmigo.
Njord y Frey querrán saber de tu escapada. Njord y Frey querrán saber de
tu huida.

El Maestro se rió.

---¿No le gustan sus aposentos, Señora Grant? No importa, venía a
recogerla a usted y al Doctor de todas maneras. Lamentablemente parece
que también ha considerado conveniente dejarnos aunque, sin duda, será
sólo por poco tiempo. ¿No le importaría decirnos dónde está, verdad?

---No tengo ni idea ---dijo Jo, preguntándose si sería verdad.

---Oh, no me lo creo ---dijo el Maestro---. Pero no importa. De todas
maneras, su tiempo se acaba y usted servirá igual de bien que él como
sacrificio. Venga deberíamos ir yendo.

Dos hombres agarraron a Jo.

Njord se adelantó y aplaudió, y de pronto apareció una multitud de
guerreros vikingos armados para la batalla.

El Maestro señaló a la esquina de la sala que había detrás de Jo, ella
se volvió y vio la TARDIS.

El Maestro sacó la llave de su bolsillo, se acercó a la cabina de
policía y abrió la puerta. Le hizo una seña a Njord, que lo siguió a su
interior junto con sus hombres y Jo.

---Siempre fui mejor piloto que el Doctor, ¿sabe usted? ---dijo el
Maestro, con las manos sobre los mandos de la consola central, mientras
la TARDIS empezaba a desmaterializarse---. No tengo ni idea de por qué
le cogen esas pataletas --- unos segundos más tarde se materializaron.

El Maestro salió de la consola y el grupo de guerreros vikingos en
silencio se hizo a un lado dándole acceso a la puerta. Se acercó con
paso tranquilo y la abrió.

---O tal vez le gusto más que el Doctor, ¿no?

Fuera de la TARDIS había una amplia habitación, más grande incluso que
el salón de Njord, que a Jo le había parecido una iglesia o una
catedral.

El salón estaba lleno de gente y, aunque algunos parecían alarmados por
la repentina llegada de la TARDIS, la mayoría no parecía preocupado por
ella, era como si hubieran visto algo similar antes.

Jo miró a su alrededor. Allí estaba, el rey tuerto, el dios viviente.
Odín, Alto Rey de toda Suecia. En su mano sostenía a Gungnir y estaba
flanqueado por sus hijos, Thor con su martillo, y otro hombre, que Jo
supuso debía ser su hermano menor, Balder. Los guerreros de Odín, por
rango, se situaban a su lado, enfrentando así a los Aesir y los Vanir.

Entre ellos había un hombre solo, vestido de pies a cabeza en una túnica
blanca hasta los pies, en sus manos sostenía un cuchillo largo y de
aspecto siniestro; era el sacerdote y el verdugo, listo para hacer su
sacrificio.

---¡Así pues! ---rugió Odín---. Vienes aquí, Njord, a observar nuestra
antigua y noble bendición en el templo, en esta noche de entre todas las
noches.

Njord se adelantó.

---Sí.

Odín sonrió, pero en su sonrisa había un rastro de traición.

---¿Y tú también vienes, Frey? ¿Has venido para mantener las promesas
que me hiciste?

Njord inclinó la cabeza hacia el Maestro, como preguntando.

El Maestro asintió.

---Lo hago.

---¿Qué es esto? ---gritó Njord--- ¿Qué has acordado con este hombre?

Odín se rió.

---¡Y aún así tendremos un sacrificio! ¡Sí!

El pueblo rugió y sus hombres se separaron mientras el Doctor era
conducido al centro de la asamblea.

---Tu, ¡Sanador! ---declaró Odín--- ¡Propietario de la nave mágica que
ahora me pertenece!

Njord rugió, un grito de traición e ira.

---¿Qué? ¿Qué es esto?

Hizo un gesto a sus hombres, pero, antes de que pudieran reaccionar, más
guerreros de Odín emergieron de las sombras del templo y los rodearon
con las armas desenfundadas.

---Y ahora, Odín, ¡Oh gran rey! ---dijo el Maestro---. Ha llegado el
momento de que haga honor a nuestro trato. La lanza, por favor.

Una vez más, Odin se rió.

---¡Tú, pequeño hombre tonto! Tengo la nave y tengo la lanza. ¿Para qué
más puedo necesitarte?

La cara del Maestro se ensombreció.

---¡Tú! ¿Cómo te atreves? ¡No sabes nada! Me necesitas, y me darás la
lanza, o no te enseñaré cómo manejar la nave.

---¡Me enseñarás! ---gritó Odín, que a continuación susurró de forma
amenazante---. O te mataré junto con el Sanador y la mujer.

El Doctor miró al Maestro.

---Buen trabajo, muchacho ---le dijo, y se volvió hacia Odín---. Ahora,
mi querido señor, tiene que escucharme, tiene que escucharme con mucha
atención.

---Odín se volvió hacia el Doctor con una mueca de sorpresa y diversión
en su cara---. ¿Debo hacerlo? ¿Y por qué?

---Porque este hombre le ha traicionado de la misma manera en que usted
le ha traicionado a él, si bien su traición es mucho más potente que la
suya. Esas ruedas hidráulicas que le ha hecho construir, seguro que le
ha dicho que le beneficiarían de alguna manera.

Odín entornó los ojos, pasando la mirada del Doctor al Maestro, que
ahora se encontraba indefenso con dos de los hombres de Odín sujetándole
los brazos.

---Nos dijo que eran un gran poder, un poder que iluminaría y calentaría
nuestras casas y estancias.

---Y así lo hará, sólo que no en la forma en que imagináis. Lo que usted
ha creado está enviando potencia a una caja escondida debajo de este
templo. Este dispositivo ha estado almacenando la energía de las ruedas
durante mucho tiempo, y va a explotar cuando él quiera, cuando lo
ordene. Si lo hace, no sólo le destruirá a usted y a todos los que están
aquí, en el templo y en el pueblo, sino que creará una herida en la
línea temporal de forma que nadie podrá entrar o salir de aquí jamás. No
cabe duda de que esa era su intención, robar la lanza y cubrir sus
huellas.

Odín miró al Doctor.

---Lo que dices no tiene sentido, Sanador.

---Estoy diciendo la verdad, puede que no entienda todo lo que digo,
pero tenga en cuenta esto: ¡Frey le ha traicionado! ¡No confíe en él!

---¡Basta ya! ---exclamó Njord---. ¡Es Odín quien nos ha traicionado!

---¡No! ---rugió Odín---. Tú y Frey habéis intentado provocar la guerra
desde el principio, ¡y la tendréis!

Agitó Gungnir sobre su cabeza como si fuera a arrojarlo hacia los Vanir.

---¡Doctor! ---gritó Jo--- ¡Es el comienzo de la guerra!

---Aún no, no lo es ---dijo---. ¡Odin! ¿No olvida algo? ¿No debería
hacer un sacrificio por la victoria antes de ir a la guerra? Eso debe
ser lo primero, ¿no?

Odín dio media vuelta con los ojos llenos de rabia, pero bajó su lanza
al hablar con el Doctor.

---Sí ---dijo---. Tienes razón, tienes razón \ldots{} ¡De acuerdo! ¿A
quién de vosotros debería sacrificar primero? ¿Al traidor Frey? ¿O a la
mujer?

---¡No! ---exclamó el Doctor--- Me sacrificará a mi. Me entrego a usted,
con una condición.

---¿Qué es?

El Doctor señaló hacia el sacerdote.

---No me insulte con este hombre, si he de ser un sacrificio, pido, no,
exijo, el honor de morir a manos del rey ¡Y por el poder del propio
Gungnir!

---¡Doctor! ¡No! ---gritó Jo, luchando, sin mucho éxito, para liberarse
de los hombres que la sujetaban.

Odín se quedó mirando al Doctor mientras los segundos pasaban, y luego
se echó a reír. Levantó su lanza por encima de su cabeza y gritó su
nombre:

---¡Gungnir!

Su grito resonó en las voces de sus hombres, y Odín dio un paso
adelante. El Doctor se inclinó hacia un lado, retrocediendo,
retrocediendo. Odín sonrió disfrutando del juego (el gato y el ratón, el
cazador y la presa) mientras el Doctor ponía distancia entre ellos.

---¡No! ---gritó Jo de nuevo--- ¡No!

Y a continuación enmudeció, porque vio lo que el Doctor había hecho y el
lugar en el que se había situado.

Odín echó el brazo hacia atrás y arrojó Gungnir con todas sus fuerzas.
La lanza voló y se dirigió directamente hacia el Doctor. No podía
fallar. No lo haría.

Jo gritó cuando vio que el Doctor la miró durante medio segundo, fue
entonces cuando dio dos pasos hacia atrás y empujó la puerta abierta de
la TARDIS, desapareciendo de la vista.

La lanza se precipitó tras él y también desapareció.

Todo el mundo dio un grito ahogado y quedaron en silencio.

Nada ni nadie se movió durante tres largos latidos, tras los cuales el
Doctor salió de la TARDIS sosteniendo la lanza.

Su voz era profunda y fuerte.

---Lanzaste la lanza que no puede fallar, Odín. Y fallaste. Ahora me
toca a mi.

Echó el brazo hacia atrás, hizo una pausa y sonrió.

---O puedo devolverle Gungnir con una condición

Odín gritó disgustado, pero el Doctor hizo amago de tirar la lanza

---¿Tan seguro estás de que yo también fallaré? ---preguntó.

El rugido de Odín se transformó en un ruido sordo, y apretando los
dientes susurró

---¿Qué es lo que quieres, Sanador?

El Doctor se volvió hacia Jo y sonrió.