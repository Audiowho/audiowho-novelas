\chapter*{Capítulo Cuatro}
\addcontentsline{toc}{chapter}{Capítulo Cuatro}

Abajo, en la raíz corazón del Heligan, el hombre que se había quedado
para vigilar la TARDIS estaba aburriéndose. Caminó en torno a ese
matorral de nuevos troncos, asomándose a través de los huecos entre
ellos, pero apenas podía distinguir la Caja Azul, y por los trozos que
podía ver no parecía tan temible ni impresionante como sonaban las
viejas historias. Se suponía que era `más grande por dentro',
significase lo que significase, pero no podía ver a través de las
ventanas.

Pequeños ruidos venían ahora constantemente desde los huecos del suelo:
salpicaduras y deslizamientos y extraño, ásperos susurros. Los ignoró.
El viejo árbol estaba inquieto esta noche, y ¿quién podría culparlo?
Estaba temblando y sacudiéndose, lleno de nuevos sonidos.

~

Sumido en sus pensamientos, y estudiando la TARDIS, no se dio cuenta de
las cosas que salían apretadas de los huecos a su alrededor. Tan
espinosas como la cáscara de una castaña, tan altas como los hombres, se
movían como los cangrejos con sus raíces como patas de cangrejo, lento
al principio, luego hundiéndose de repente \ldots{}

El árbol estaba inquieto esta noche.

Nadie escuchó sus gritos.

---¿Venganza? ---preguntó la juez, volviéndose de sus prisioneros para
enfrentarse al enojado recién llegado que la había interrumpido--- Sí,
pero se debe hacer con honor, Presidente. Yo soy el Juez, y digo que
debemos tener un juicio. Debemos asegurarnos de que ese es realmente el
Doctor, se le debe permitir dar su opinión antes de darle muerte.

---¡No eres digna de ser Juez! ---se burló el Presidente Ratisbon---
Eres como muchos otros hoy en día, piensas que el Doctor es sólo un
monstruo de cuento de hadas con el que asustar a los niños traviesos.

La Juez se sonrojó airadamente.

---¿No se preguntaba eso todo el mundo? ¿Todos con alguna inteligencia?
¿Incluso feroces ancianos como tu, Extirpar-El-Corazón-Vivo-Del-Doctor
Ratisbon? Pero él está aquí, y dice que es el Doctor, y según nuestras
leyes antiguas debe enfrentarse a un juicio.

---No es necesario ---dijo Ratisbon---. El veredicto fue emitido sobre
este traidor hace novecientos años. La sentencia es la agonía y la
muerte, y es mi deber ver que se lleva a cabo. ¡Llevadlo a la Silla!

Y aunque la Juez levantó las manos y les ordenó que se detuvieran, nada
pudo parar a los hombres que entraron en su cámara , quienes se
apoderaron del Doctor y se lo llevaron bruscamente a rastras. Leela,
mientras sus propios guardias estaban distraídos, le arrebató el
cuchillo a uno que se lo había cogido a ella y corrió a rescatarlo, pero
uno de los hombres de Ratisbona la derribó con un golpe con el mango de
la lanza. Aterrizó a cuatro patas, atontada, la sangre le goteaba de un
corte en la frente. Ven corrió hacia ella, y su madre se acercó y se
arrodilló junto a ella, taponando la herida con un paño.

---¡Déjadme en paz! Apenas es un rasguño\ldots{} ---Leela trató
desembarazarse de ellos, de correr tras el Doctor. La contuvieron.
---¿Adónde le llevan? ---preguntó--- ¿Creía que eras la líder aquí?

La Juez dijo:

---Yo también lo pensaba, pero parece que no. Ratisbon es nuestro
verdugo. Parece que está impaciente por ir a trabajar.

La sala se estremeció. Todo el árbol parecía agitarse sin cesar, como un
gran animal inquieto en sueños. Desde fuera de la sala se produjo un
crujido, como si alguien arrastrase un pesado fardo de ramitas.

Uno de los hombres que se había rezagado en la puerta abierta, inseguro
de si quedarse con la Juez o seguir al Presidente Ratisbon, de repente
gritó de miedo.

---¡Juez!

Tropezó de vuelta a la habitación y trató de cerrar la puerta, pero algo
lo empujó con violencia hacia fuera. El susurro era muy fuerte, y la
habitación se llenó de repente con el olor a estiércol que Leela
recordaba de abajo. Miró a Ven y a su madre, vio el miedo y la
incomprensión en sus rostros, y se puso de pie, cuchillo en mano, lista
para encontrarse con este nuevo peligro cara a cara\ldots{}

Excepto que no tenía rostro. Una cáscara verdosa y dura, plagada de
espinas afiladas, un ramo de raíces ajetreadas, hundidas, como garras,
unos delicados zarcillos que las manoseaban y revoloteaban, un tallo
velloso grueso, pero nada en ningún lugar que se pareciese a unos ojos o
una boca.

Detrás de ella una de las mujeres de la Juez chilló, y la criatura se
giró hacia el sonido. ``Es ciega'', pensó Leela ``pero no es
sorda\ldots{}'' Hizo un gesto a los demás para que se callaran. No sabía
si podía luchar contra esta cosa, no sola. Algunos de los hombres en la
habitación tenían lanzas, pero parecían tener demasiado miedo para
usarlos. De todos modos, ¿dónde apuñalar una cosa así? ¿Cuáles serían
sus puntos débiles?

Alguien gimió. La cosa se torció, arrastrándose hacia adelante sobre su
falda de raíces, los zarcillos extendidos para sentir el aire por
delante. Leela contuvo la respiración, tratando de no temblar cuando una
punta de zarcillo se quedó a un palmo de su cara.

Entonces, desde algún lugar en el exterior, hubo otro grito. ``Debe
haber más de ellos'' pensó Leela, y la criatura se dio la vuelta y se
escabulló. Más gritos en el pasillo; susurros desde el grupo de
asustadas personas en la habitación.

---¿Qué ha sido eso? ---siseó Leela.

---No lo sé ---susurró Ven---. Nunca he visto nada igual.

---El Doctor ---dijo---. Él sabrá lo que son, lo que hacer.

---¡Pero Ratisbon tiene al Doctor! ---dijo la Juez.

---¡Entonces tenemos que salvarlo!

La Juez la miró por un momento, luego asintió lentamente. A las personas
en la habitación les dijo:

---El que tenga armas, que venga conmigo, el resto, reuniros en el Salón
de la Justicia. Tened cuidado con esos \ldots{} esos
lo-que-sea-que-sean.

Leela estaba ya en la puerta. Actuó como si hubiera olvidado que alguna
vez había sido su prisionera, y no trató de recordárselo. En el
exterior, los pasillos de madera estaban llenos del crujido de raíces y
del olor a vegetal húmedo de esas cosas repugnantes. Leela agarró con
más fuerza el cuchillo.

---¿Dónde lo han llevado? ---preguntó.