\chapter*{CAPÍTULO CINCO}
\addcontentsline{toc}{chapter}{CAPÍTULO CINCO}
Cuando el Lynx salió por encima de las copas de los árboles, la capitana
Jo Phillips observó con incredulidad la escena desarrollándose ante
ella.\\
--¡Santa Madre de Dios! --Leigh Brewster abrió la puerta trasera de un
tirón para conseguir una mejor vista de un enorme escarabajo negro y
naranja moviéndose atropelladamente por el campo bajo ellos-- ¡Mirad el
tamaño de esa cosa!\\
--Capitana, parece que tenemos civiles ahí abajo --su copiloto apuntó a
cuatro figuras intentando escapar del monstruo que avanzaba. Phillips
podía ver los restos de una destrozada furgoneta y una indistinta, pero
aún reconocible forma humana en la hierba junto a ellos.\\
--¡Brewster, prepara esa arma! --Phillips llevó al Lynx a dar vueltas en
círculos, alineando la compuerta abierta lateral con la cosa por debajo
de ellos-- Este es el Army Air 179 al cuartel de la JHC. Hostilidad
localizada. Posibles bajas civiles. Nos vamos a involucrar.\\
Cualquier respuesta por parte de sus superiores fue ahogada por el
sonido del Brewster abriendo fuego con la M3M.\\[2\baselineskip]El
Doctor y los otros se lanzaron contra el suelo cuando el helicóptero
abrió fuego. El sonido era increíble. Unos proyectiles de gran calibre
comenzaron a destrozar el suelo entre ellos y el escarabajo, enviando
nubes de tierra en cascada por los aires. Hubo un ensordecedor aullido
de furia y dolor cuando las balas impactaron en la espalda de la
criatura, destrozando su caparazón armado y enviándolo dando vueltas a
otro lado por causa del impacto.\\
Aprovechando el momento, el Doctor ayudó a Charlie a ponerse en pie,
empujándola hacia el santuario de una iglesia visible a través de los
árboles en el extremo alejado del campo.\\
--¡Ahora! ¡Mientras se recupera! ¡Corred!\\
Peleándose contra la corriente descendiente de los rotores del
helicóptero en movimiento, Charlie se tambaleó mientras el Doctor
buscaba a su alrededor frenéticamente a Clara y a Angela. Estaban a unos
metros de distancia, todavía cubriéndose por la arremetida de la
rugiente metralleta.\\
--¡Clara! --gritó el Doctor, apenas capaz de oírse a sí mismo por encima
del ruido-- ¡La iglesia! ¡Intentad llegar hasta la iglesia!\\
Clara asintió, agarrando a Angela por la mano y poniéndose en pie. Las
dos mujeres comenzaron a correr. Atraído por el movimiento, el
escarabajo emitió un silbido de furia y comenzó a moverse
atropelladamente hacia ellos una vez más, con las partes bucales
chasqueando hambrientamente. Otra arremetida de disparos de metralleta
salió del helicóptero, pero el escarabajo ahora estaba demasiado por
debajo del disparador como para apuntar decentemente, y el piloto viró
hacia atrás, intentando encontrar una posición mejor.\\
Con horror, el Doctor se dio cuenta de que la criatura ahora estaba
entre él y Clara, bloqueando su ruta de escape.\\
--¡Clara! ¡Quédate quieta! --las dos mujeres frenaron en seco, todavía
agarrándose entre ellas. El Doctor comenzó a gritar, zarandeando los
brazos-- ¡Ey! ¡Por aquí!\\
El escarabajo le ignoró, aun concentrándose a su presa. Mirando a su
alrededor frenéticamente el Doctor agarró una roca y la lanzó contra la
espalda de la criatura. Hubo un paf seco cuando rebotó inofensivamente
contra el caparazón. El Doctor observó impotentemente como la criatura
proseguía inexorablemente hacia Clara y Angela, con sus antenas
moviéndose rápidamente al intentar localizarles.\\
Con un rugido de motores, el helicóptero bajó en picado una vez más,
desatando otra arremetida de disparos. Esta vez, las balas tuvieron
efecto, destrozando una sección de la quitina del caparazón y amputando
la punta de una de las antenas del monstruo.\\
Furioso con la cosa que se cernía sobre él, que le había herido y le
había distraído de su comida, el escarabajo se giró hacia el
helicóptero. En cuanto su atención se alejó de ellas, Clara y Angela se
giraron y corrieron en la dirección contraria. El Doctor observó con
alivio cómo se tambaleaban a través del seto en el extremo alejado del
campo. Puede que no llegaran a la iglesia, pero por ahora estaban a
salvo.\\
El escarabajo ahora silbaba y chillaba al helicóptero, poniéndose sobre
sus patas traseras cuando las balas comenzaron a destrozarle. Sin
advertencia, se giró de repente, levantando su cuerpo y, con un
repentino escalofrío al darse cuenta, el Doctor finalmente reconoció qué
tipo de escarabajo era.\\
--Oh, no\ldots{}\\
Con un silbido como si fueran miles de extintores, un espray de líquido
lechoso irrumpió del abdomen de la criatura, golpeando el helicóptero
por la parte lateral.\\[2\baselineskip]En cuanto el líquido les golpeó,
Jo Phillips supo que estaban en problemas. Con un severo crujido, el
parabrisas acrílico se rompió formando una telaraña de líneas de
fractura, reduciendo su visibilidad a cero.\\
Desde detrás de ella, pudo oír a Brewster gritar cuando el líquido se
introdujo a través de la puerta abierta, salpicándole la cara y el
cuerpo. Empujó los controles hacia atrás, mandando al helicóptero hacia
arriba, pero toda la cabina comenzaba a llenarse de un asfixiante humo
acre a medida que la descarga comenzaba a comerse los cables y las
conexiones. No podía comenzar a imaginarse qué le estaría haciendo a
Brewster.\\
La nave afectada se ladeó cuando el escarabajo envió otro chorro de
ácido escupiendo por el fuselaje. Unas luces de advertencia comenzaron a
brillar por todo el panel de control ante ella. La palanca de mando de
repente se le quedó quieto en la mano, y con una repentina y fría
certeza Jo supo que todas las líneas de control se habían apagado.
Segundos más tarde, las líneas de combustible también se habían apagado
y, con una tos de protesta, los motores trastabillaron y murieron. La
nave ahora no era nada más que varios cientos de kilos de metal y
plástico con nada con lo que mantenerse en el aire.\\
Cuando el helicóptero se abalanzó contra el suelo, Jo vio la torre de la
iglesia de Ringstone y se preguntó cómo habría sido casarse
allí.\\[2\baselineskip]El Doctor corrió cuando el helicóptero cayó en
picado desde el cielo, con su piel de metal silbando y humeando por la
cáustica mezcla de hidroquinina y perióxido de hidrógeno comiéndose el
armazón.\\
Se estrelló contra el suelo en el extremo alejado del campo con un
impacto que le tiró de espaldas. Hubo un breve momento de feliz
silencio, y entonces el tanque de combustible explosionó, enviando una
bola de fuego naranja ardiendo hacia el claro cielo azul. El Doctor se
cubrió la cabeza con sus brazos cuando el metal y el plástico ardiente
llovían a su alrededor. A medida que una onda de aire abrasador le
barría, el Doctor se puso en pie.\\
El monstruoso escarabajo estaba aplastado bajo la pared de llamas,
chirriando con furia ante su presa que se le había escapado. Con una
última mirada de impotencia ante el desecho ardiente del helicóptero, el
Doctor se giró y corrió campo a través, entonces se abrió camino entre
los árboles y el pulcro camposanto bien colocado de la iglesia del otro
lado.\\
Cerró la antigua puerta de madera tras él y se apoyó contra ella, usando
la fría calma del interior de la iglesia para intentar poner en orden
sus pensamientos. Las cosas se estaban descontrolando. Al menos media
docena de personas habían muerto por ahora y él no estaba ni cerca de
descubrir por qué un pueblo estaba infestado con estos monstruos.\\
Una cosa se estaba volviendo clara: alguien o algo quería que este
pueblo se aislara del mundo exterior.\\
Abrió sus ojos para ver la cara asustada de Charlie Bevan observándole
desde detrás de una hilera de bancos. Intrigado, el Doctor abrió su boca
para preguntarle qué demonios estaba haciendo, pero Charlie negó con la
cabeza frenéticamente y apretó un dedo contra sus labios. Entonces
apuntó hacia arriba.\\
El Doctor observó hacia adelante, escrutando la oscuridad del techo de
la iglesia. El techo era un lío de redes, pesadas con cuerpos de ovejas
envueltas y, colgando entre ellas, había la silueta gigantesca silueta
amenazante de una araña enorme.\\[2\baselineskip]La caída del Lynx lo
cambió todo. De repente, Ringstone era el lugar más importante del país.
Casi cada vez que el coronel Dickinson colgaba su teléfono volvía a
sonar de nuevo. Ya había recibido un par de llamadas de la secretaría
privada del Primer Ministro y de la Secretaría del Estado de Defensa,
ambas pidiendo saber qué estaba pasando. El coronel tuvo que contenerse
de señalar que, si dejaban de interrumpirle con inútiles llamadas y le
dejaban hacer su trabajo, puede que tuviera una mejor oportunidad de
averiguarlo.\\
Mientras tanto, sus llamadas a UNIT eran respondidas con palabras
educadas pero inútiles. Al parecer todas las tropas de UNIT estaban
ocupadas con una crisis en las islas Canarias. Unos robots con cabezas
afiladas estaban emergiendo del reciente volcán en erupción de El
Hierro.\\
Dickinson soltó una risotada por el sarcasmo. Unos robots afilados de
roca\ldots{}. Algunas veces no pensaba que la gente de UNIT viviera en
el mundo real.\\
Hubo una llamada en la puerta de su despacho, y levantó la mirada
impacientemente cuando el cabo Jenkins entró:\\
--¿Sí, cabo?\\
--Hay un Land Rover esperándole fuera, señor.\\
--Bien --el coronel asintió, satisfecho. Agarró su gorra del escritorio
y el teléfono comenzó a sonar de nuevo--. Tendrá que dirigir el fuerte
aquí, Jenkins. ¡Dígale a esos malditos políticos que volveré con lo que
quiera que esté pasando en cuanto lo sepa, maldita sea!\\
Dejando a su ayudante para lidiar con la lista de llamadas en curso, el
coronel Dickinson salió hacia el vehículo que le esperaba. Se introdujo
en el Land Rover y éste arrancó con una sacudida.\\
--¿Alguna idea de que va todo esto, señor? --preguntó el conductor
cuando pasaron a través de las puertas del cuartel.\\
--Ojalá lo supiera, soldado --dijo el coronel Dickinson de forma
lúgubre--. Ojalá lo supiera.\\[2\baselineskip]Clara y Angela habían
estado observando desde el otro lado del seto, esperando que el
helicóptero ahuyentara el escarabajo y tuvieran oportunidad de volver
con el Doctor y Charlie. El impacto había terminado con aquello. Clara
había observado con horror cómo la oleada de ácido engulló la nave,
silbando e hirviendo mientras ardía a través del metal y el plástico.
Segundos más tarde, los rotores dejaron de girar y la máquina dañada
comenzó a caer hacia ellas.\\
Agarrando a Angela por el brazo, Clara la arrastró hacia el terraplén al
lado de la vía, agachándose en él cuando el helicóptero golpeó el suelo
y estalló con un impacto que hizo que las orejas de Clara le pitaran.\\
Cuando el combustible de aviación ardiendo hizo que se encendiera el
seto, ella y Angela huyeron, esperando que pudieran volver al centro del
pueblo y llegar a la iglesia desde allí.\\
Salieron por la carretera y Clara echó una última mirada atrás hacia la
columna de acre humo negro que ascendía hacia el cielo matutino.\\
--Será mejor que no estés debajo de eso, Doctor --murmuró ella para
sí.\\
Las dos mujeres caminaron en silencio. El silencio del campo británico
que Clara al principio había encontrado relajante ahora era ominoso e
inquietante. Cada susurro del sotobosque o aleteo le hacía botar,
temerosa de otro insecto monstruoso emergiendo de los setos. Pasaron
otra carretera cortada por hebras de telaraña.\\
--Algo ha estado ocupado --dijo Clara. Comenzaba a darse cuenta de que
todo el mundo en aquel pueblo estaba efectivamente acorralado.\\
Angela miraba con preocupación donde estaba un coche atrapado en la red,
con una de sus puertas traseras abierta de par en par. Un bolso estaba
tirado en medio de la estrecha carretera, con sus contenidos extendidos
por el hormigón.\\
--¿Crees que\ldots{}?\\
--Será mejor que no, ¿eh? --Clara la asió del brazo-- Vayamos a la
iglesia y encontremos al Doctor.\\
--Pones mucha fe en él --Angela la miró a los ojos--. ¿Está bien
colocado?\\
--Sí --Clara le mantuvo la mirada--. Lo está. Él arreglará esto.\\
Después de un momento de pausa, Angela asintió.\\
--Entonces volvamos con él.\\
No les llevó mucho llegar al centro del pueblo, pero incluso allí, las
calles estaban desconcertantemente desiertas. Cuando iban a cruzar por
la plaza, un repentino ruido le hizo a Clara pegar un bote. Algo pequeño
se movía detrás del memorial de guerra, medio escondido en las sombras
que proyectaban los dos grandes olmos a cada lado de ello.\\
--Hay algo ahí --susurró ella, arrastrando a Angela hacia el jardín de
una de las casas que bordeaba la plaza. Agachándose en el lecho de
flores se asomaron por encima de la pared baja cuando una silueta
pequeña salió hacia la luz del sol. Angela hizo un suspiro de
alivio--.Todo está bien. Es Emily Nichols. Pero no puedo ver a su
madre\ldots{}\\
Angela comenzó a levantarse desde su escondite, pero Clara la asió del
brazo, echándola hacia atrás.\\
--No, espera un momento, algo va mal.\\
Angela se liberó.\\
--¿Qué hay de malo? Una niña pequeña está ahí sola. Voy a ayudarla.\\
Antes de que pudiera moverse, la puerta de la Oficina de Correos al otro
lado de la plaza se abrió de par en par y Simon George dio un paso
nerviosamente hacia la calle.\\
--¡Emily! --silbó él-- ¡Por aquí! ¡Rápido!\\
La pequeña no se movió. Obviamente agitado, Simon se apresuró a cruzar
la calle hacia ella. Estaba constantemente comprobando el cielo por
encima de él. Clara siguió su mirada. ¿Qué estaba buscando? Había casi
llegado a Emily cuando la niña de tres años se giró de repente,
levantando su mano hacia él, y soltó un terrorífico grito casi
inhumano.\\
Casi de inmediato, el aire se llenó con un profundo zumbido y una media
docena de enormes mosquitos cruzaron la plaza del pueblo, con sus alas
siendo un borrón de movimiento bajo la luz del mediodía.\\
Simon se giró para huir, pero no tuvo ninguna oportunidad. Los mosquitos
se acercaron en enjambre hasta colocarse a su alrededor mientras él les
daba azotes con las manos, pero siempre se mantenían fuera de su
alcance. De repente uno de ellos se lanzó hacia abajo, aterrizando en su
espalda y Clara le oyó gritar de dolor cuando la criatura clavó su
probóscide afilada como una aguja a través de su camisa.\\
El director de la oficina de correos se derrumbó de rodillas cuando el
mosquito se alzó desde su espalda, volviéndose a unir al resto del
enjambre, alzándose alto en el cielo y desvaneciéndose por encima de los
tejados. Durante lo que pareció una eternidad, Simon se agachó allí, con
la cabeza bajada, y Clara comenzó a pensar que estaba muerto. Entonces,
con unos movimientos casi de títere, su cabeza se enderezó, y se puso
desequilibradamente de pie.\\
Clara y Angela se agacharon, apretándose contra la pared, cuando Simon y
Emily comenzaron a moverse lentamente hacia ellas. Por toda la plaza,
los vecinos comenzaron a aparecer, atraídos por el grito de Emily.
Arrastraron los pies lentamente hacia adelante, con los brazos colgando
sin rigidez, con sus caras grises e hinchadas, sus ojos sin estar
centrados en ningún punto y vidriosos. Los efectos como de zombie que
había en las caras de los habitantes del pueblo ya era bastante malo, y
en los niños como Emily simplemente era terrorífico. Clara miró
desesperadamente a Angela.\\
--¡Tenemos que encontrar un lugar donde escondernos!\\
Angela asintió hacia un estrecho callejón que llevaba a la parte trasera
de una casa.\\
--Por ahí.\\
A gatas, las dos se estrecharon hacia el pasaje, emergiendo en un
jardincito pulcro. Al final del jardín había una inclinación pronunciada
de hierba que daba al terraplén de la vía. Mientras corrían por el
jardín, otro terrible grito vino de la plaza del pueblo, seguido de un
reconocible chillido de terror humano.\\
Angela se giró, con su cara llena de incertidumbre. Clara no podía
comenzar a imaginarse cómo debía de estar sintiéndose, viendo su hogar
invadido por monstruos, sus amigos y vecinos atacados, vulnerados.\\
--No hay nada que podamos hacer --dijo ella, amablemente--. Si volvemos
atrás, si intentamos ayudar, entonces también estaremos perdidas.\\
--Lo sé, solo me siento impotente --Angela se limpió las lágrimas de los
ojos con el revés de la mano. Angela asintió, y las dos mujeres subieron
el terraplén, cruzaron la vía y fueron hacia los campos desconocidos de
más allá.\\
