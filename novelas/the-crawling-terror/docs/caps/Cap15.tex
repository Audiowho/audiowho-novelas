\chapter*{CAPÍTULO QUINCE}
\addcontentsline{toc}{chapter}{CAPÍTULO QUINCE}
--Así que, el experimento británico acabó en desastre. Eso sigue sin
explicar por qué tienes esto --Clara señaló hacia la Campana con sus
marcas de esvásticas, esperando una explicación.\\
--Las bombas alemanas destruyeron completamente la máquina británica
--le dijo Clearfield--. La Campana, las mesas de control, las notas de
investigación, todo. Comenzar de nuevo habría tomado años, y el fin de
la guerra puso fin a cualquier pensamiento de ello. Pero el aparato
alemán nunca había sido usado como se pretendía. La Campana alemana
seguía ahí fuera.\\
--Así que la encontraste --Clara forzó su memoria--. ¿No escaparon
muchos de los criminales de guerra nazis a Sudamérica? ¿Paraguay,
Argentina, lugares como esos?\\
Clearfield asintió.\\
--Y yo he pasado mucho tiempo en esos países, encontrando gente que
había sido cercana a ese proyecto, intentando averiguar qué le había
pasado a la Campana nazi. Al final, descubrí que tanto ella como el
general de las SS Kammler habían sido retirados de los complejos de
pruebas en Polonia por submarino y les habían llevado a la base
submarina secreta de Neuschwabenland.\\
Clara se encogió, sin sentirse familiar con la palabra.\\
--La Antártida.\\
Ella frunció el ceño.\\
--Pero esto debe de haberte costado una fortuna. Encontrar la Campana,
transportarla aquí, montar este complejo. ¿De dónde has sacado el
dinero?\\
Clearfield lanzó una sonrisa irónica.\\
--Sigue habiendo mucha gente a las que les gustaría ver al imperio nazi
alzarse de nuevo. Gente con acceso al dinero que recaudaron al final de
la guerra.\\
--Así que estás financiando la recreación de los experimentos de la
Campana usando el oro nazi.\\
--Sí.\\
--Y has vuelto aquí para montarlo en el mismo lugar de nuevo. Usando el
círculo.\\
--Eso esperaba. Por desgracia, el daño en el círculo de piedras fue tan
enorme que las energías que creó ya no están disponibles para controlar
bien la Campana.\\
--Y por eso esto --Clara señaló a los monolitos negros, las espirales en
las pantallas aún arremolinándose en los infinitos patrones complejos--.
Pero ciertamente hay docenas, si no cientos de círculos de piedras como
este esparcidos por el campo. ¿No podrías cambiarlo a uno de ellos?\\
Clearfield negó con la cabeza.\\
--Hay muy pocos con las propiedades particulares que hicieron el de aquí
en Ringstone tan especial. Nuestros ancestros entendieron la naturaleza
especial de las piedras, el hombre moderno solo lo ve como unas
curiosidades de una época pasada. La mayoría de los círculos de energía
han sido destruidos por granjeros ansiosos de usar los campos para hacer
crecer los cultivos\ldots{} Lo llaman progreso.\\
--Vale --Clara cruzó sus brazos--. Recuperas una máquina nazi que
permaneció escondida desde la guerra, has conseguido financiar la
creación de un círculo de piedras de alta tecnología, y estás
alimentando insectos gigantes. Así que ahí va la pregunta del millón de
dólares: ¿todo para qué?\\
Clearfield se inclinó hacia adelante con urgencia.\\
--Los Wyrresters que intentaron abrir el portal aquí durante la guerra
son los mismos con los que estoy trabajando ahora. Son unos parias de su
propia especie, cazados, perseguidos\ldots{} Buscando refugio. Esperaba
crear una versión digital del círculo, una forma de abrir el portal
artificialmente, pero\ldots{} --suspiró-- Las energías de las líneas ley
requieren condiciones muy específicas. Mis experimentos solo han tenido
éxito parcialmente. No puedo habilitar una transferencia física, solo
una mental.\\
--Y por eso los insectos --Clara miró las siniestras siluetas merodeando
en las jaulas en el extremo alejado del laboratorio--. ¡Les has creado
cuerpos!\\
Clearfield asintió.\\
--He estado experimentando durante años, intentando descubrir una forma
que les permita sobrevivir aquí en la Tierra. Las criaturas que has
visto, las arañas, los bichos segadores, solo eran pruebas, intentos
para ver si los Wyrresters podían habitar y controlar estos
cuerpos\ldots{} Estas pruebas no han sido del todo exitosas.\\
Clara estaba aterrorizada.\\
--¿Así que esos insectos tienen mentes alienígenas?\\
--Parcialmente. Esos experimentos tempranos tienen una mente compartida,
una conciencia compartida, pero la parte Wyrrester es incapaz de
dominarla totalmente, algunas veces el animal toma la iniciativa. Cuando
lo hace, las criaturas revierten a su naturaleza básica\ldots{}
Finalmente eso lleva a la locura, luego a la muerte. Han servido a su
propósito, sin embargo, como guardias, como distracciones mientras
preparamos la fase final.\\
--Eso es horrible. Horrible --Clara le miró con disgusto--. ¿Sabían los
Wyrresters que pasaron por este\ldots{} procedimiento que iban a ser
solo\ldots{} sujetos de pruebas?\\
--Los Wyrresters que accedieron a estos experimentos estaban totalmente
entregados a su causa --dijo Clearfield con frialdad--. Si nuestras
pruebas con sujetos humanos han sido igual de exitosas\ldots{}\\
--¿Has intentado hacerlo con gente?\\
--¡Por supuesto! --Clearfield la miró como si fuera lo más natural del
mundo-- Si la mente Wyrrester era capaz de ser injertada en una forma
humana entonces eso sería la solución más adecuada. Por desgracia, la
mente humana no es compatible.\\
--¡O eso habíamos pensado!\\
--¿Perdón? --Clearfield levantó la mirada cuando una voz sibilante
volvió a hablar --Vosotros habéis dicho que era imposible. ¡Que no
podíais habitar una mente humana!\\
--Eso era cierto. Pero esta chica es distinta. Su mente es algo con lo
que no nos hemos encontrado antes. Sus conexiones neuronales han sido
abiertas por tecnología que no es de este planeta. Ella será receptiva
con nosotros. Colócala en el círculo.\\
--¿Qué? ¡No!\\
--¡Haz lo que se te ha dicho, Clearfield!\\
--Pero, Gebbron, los cuerpos híbridos de insectos que he creado para
vosotros son un éxito sin cualificar. Todo el trabajo que he
hecho\ldots{}\\
--¡Tú has tenido años, y sin embargo sigue habiendo problemas! Las
criaturas que has diseñado siguen siendo primitivas. Esta chica ofrece
una nueva oportunidad. No la desperdiciaré. Necesitas mi guía, y ahora
tenemos las formas de proveerla. Ella será mi recipiente.\\
Clara levantó la mirada, alarmada.\\
--Eh, esperad un momento\ldots{} No estoy segura de que me guste ser el
recipiente\ldots{}\\
--¡Haz lo que se te ha dicho!\\
Clearfield pareció completamente ofendido.\\
--No me dejáis opción.\\
Clara hizo un ademán nervioso.\\
--Hola\ldots{} como he dicho\ldots{} ¡no me gusta!\\
Clearfield alargó su mano dentro de su chaqueta y sacó su revólver.\\
--Lo siento, Clara. Yo\ldots{} tú\ldots{} debes hacer lo que pide
Gebbron --señaló al círculo--. Mantente en el centro.\\
Clara no se movió. El tono de Clearfield se endureció.\\
--El proceso funcionará igual de bien si está inconsciente, señorita
Oswald.\\
La amenaza era muy clara. Dándose cuenta de que no tenía opción, Clara
dio un paso dentro de los obeliscos negros.\\
Clearfield se deslizó hasta el asiento frente una de las muchas consolas
de control y comenzó a toquetear interruptores. Lentamente un flojo
zumbido de energía comenzó a oírse y la base de la Campana comenzó a
brillar con una profunda luz morada.\\[2\baselineskip]--¡Señor!\\
El capitán Wilson levantó la mirada cuando el soldado a cargo de
monitorizar la estación de comunicaciones entró en el vehículo de
mando.\\
--Hay positiva confirmación de una firma de energía del complejo, señor.
Muy débil, pero definitivamente la distancia de onda que nos dijo que
buscáramos.\\
Wilson lanzó una mirada preocupada a su oficial de mando. La cara del
coronel Dickinson permaneció impasible, pero algo sobre sus formas, la
forma en la que se mantenía de pie, cambió abruptamente y de repente
pareció mucho mayor, mucho más débil. Cuando Dickinson sacó el teléfono
para dar la orden que barrería Ringstone del mapa, Wilson se levantó de
la silla.\\
--Señor, un momento.\\
La mano de Dickinson vaciló en el receptor.\\
--¿Qué ocurre, capitán?\\
--Pido permiso para llevar un pequeño equipo dentro del perímetro,
señor.\\
--¿Con qué propósito?\\
--Localizar la Campana, incapacitarla o destruirla, evacuar cuantos más
civiles sea posible.\\
El coronel le miró de cerca, con su mano aun descansando en el
teléfono.\\
--¿Y los hostiles?\\
--Tenemos armas pesadas disponibles y dos de las Armas de la Siguiente
Generación Antitanque. Además de ocho galones de pesticida experimental.
Si todo eso fracasa, entonces haremos explotar las cosas con granadas.\\
El coronel no dijo nada.\\
--Deme dos horas --dijo el capitán con suavidad--. Inteligencia apunta
al hecho de que quienquiera que opere el aparato esperará hasta el
equinoccio de primavera antes de activarlo a toda energía. Deme dos
horas antes de que pida ese bombardeo aéreo.\\
Dickinson apartó su mano del teléfono.\\
--De acuerdo, capitán. Pero, si vemos esa firma de energía comenzar a
entrar en un pico, les sacaré a usted y a su equipo y haré esa
llamada.\\
Wilson saludó.\\
--Señor.\\
Cuando el capitán se giró para marcharse, Dickinson le llamó.\\
--Capitán Wilson\ldots{}\\
--¿Señor?\\
--Buena suerte.\\
Los dos hombres mantuvieron su mirada durante un momento, ambos
conscientes de lo que estaba en juego. Entonces el capitán Wilson se
apresuró a reunir su brigada.\\[2\baselineskip]--Oh-oh.\\
El Doctor observó la lectura en la consola de la TARDIS que de repente
se había encendido.\\
--¿Qué es eso? --Charlie Bevan levantó la mirada alarmado.\\
--Una lectura de energía. De tu tiempo. La Campana acaba de ser
activada.\\
--¿Pero creía que habías dicho que teníamos hasta el\ldots{} equinoccio
de primavera?\\
--Eso he dicho --el Doctor se paseó por la consola, ajustando los
controles--. Esto es algo distinto --frunció el ceño--. Y no viene del
círculo, viene del parque científico.\\
--¿Clearfield?\\
--¿Quién sino? --el Doctor apretó el control de materialización-- Creo
que es hora de que descubramos exactamente cómo es que nuestro profesor
sobrevivió, y por qué está tan determinado en iniciarlo todo de nuevo.\\
El carraspeo y el rugido chirriante de los motores de la TARDIS
resonaron por la sala de la consola, y los enormes rotores que daban
vueltas comenzaron a frenar. El Doctor cruzó la sala de control y salió
por las puertas antes de que incluso se hubiera dejado de mover. Charlie
se apresuró tras él.\\
Charlie emergió hasta la brillante luz del sol. La TARDIS había
aterrizado en el aparcamiento del complejo industrial, acurrucada contra
la pared de uno de los grandes edificios prefabricados. El Doctor ya
estaba junto a las puertas, usando su destornillador sónico para
entrar.\\
Las puertas de cristal se abrieron a un lado con un silbido y el Doctor
se desvaneció en su interior. Cerrando las puertas de la TARDIS, Charlie
corrió para atraparle.\\
El interior del edificio estaba oscuro y desierto, pero había un suave
zumbido eléctrico que hacía que todas las puertas y ventanas temblaran
ligeramente. Estaba haciéndole castañetear los dientes a Charlie.\\
Al abrirse camino lentamente por el ancho pasillo que les llevó desde la
zona principal de recepción de repente oyeron un ruido, un golpe
acompañado por una voz ahogada.\\
--Eso suena como Angela --dijo Charlie, escuchando los lejanos gritos de
ayuda.\\
Siguieron el ruido hasta un armario cerrado al final del pasillo que
salía desde la pasarela principal. El Doctor apretó su destornillador
sónico contra el cerrojo y hubo el ruido de un cerrojo al ser liberado.
Inmediatamente la puerta se abrió de golpe y una frenética figura salió
del armario, gritando y golpeando con sus puños.\\
El Doctor se apartó a tiempo, pero Charlie fue demasiado lento,
recibiendo un golpe en el costado de la cabeza cuando Angela intentó
abrirse paso a través de ellos.\\
--¡Ey! ¡Somos nosotros! --gritó él.\\
La joven veterinaria frenó en seco.\\
--¡Oh, gracias a Dios! --su alivio era palpable-- Creía que los zombis
de Clearfield habían vuelto.\\
--¿Dónde está Clara? --preguntó el Doctor, agudamente.\\
--Se la ha llevado por ahí --Angela señaló hacia las anchas puertas
dobles al final del pasillo.\\
Con la mandíbula apretada, el Doctor giró en redondo y salió hacia
ellas.\\
--¡Ten cuidado, Doctor! --le gritó Angela-- Tiene una pistola.\\
A pesar de su advertencia, el Doctor abrió las puertas y corrió
desvergonzadamente hasta la sala del otro lado. Charlie y Angela le
siguieron precavidamente. El ruido de dentro de la sala ahogó el resto
de sonidos, un flojo zumbido que Charlie podía sentir a través del
suelo. Una luz morada brillaba desde el centro de la sala enviando unas
sombras que bailaban por las paredes. Un círculo de trece siluetas
negras monolíticas dominaba el centro de la sala, cada una emitiendo
chispas de energía.\\
En el centro se alzaba Clara. El Doctor levantó su destornillador
sónico.\\
--¡Clearfield! --gritó-- ¡Apágalo! ¡Ahora!\\
Las cabezas de todo el mundo en la sala se giraron de golpe para
mirarle. Clearfield se había levantado de su asiento y estaba observando
al Doctor con incredulidad.\\
--Tú\ldots{}\\
--Lo digo en serio --dijo el Doctor, peligrosamente--. O lo apagas tú o
lo haré yo.\\
Clearfield alargó la mano y apretó una serie de interruptores.
Lentamente la luz de la Campana y el ruido de la maquinaria se
desvanecieron. Las luces se encendieron. Unos técnicos en otras consolas
comenzaron a levantarse, pero Clearfield les señaló que se quedasen
quietos. Cruzó lentamente la sala hacia el Doctor.\\
--Bueno, bueno, bueno\ldots{} el Doctor y su detector de idiotas.
Dijiste que nos encontraríamos setenta años después y aquí estamos. Me
gustan los hombres de palabra.\\
El Doctor le ignoró.\\
--¿Y tú estás bien, Clara?\\
Durante un momento, no dijo nada, luego asintió levemente. Clearfield
ahora estaba a unos pocos metros del Doctor, observándole con una
curiosidad profesional.\\
--Extraordinario. Setenta años desde la última vez que te vi y sin
embargo no has cambiado nada.\\
--Podría decir lo mismo de ti --el Doctor estudió la máscara que cubría
una de las mitades de la cara de Clearfield--. A parte de eso, por
supuesto.\\
--¿Esto? --Clearfield levantó una mano para tocarse el plástico
translúcido-- Déjame hablarte de esto. Es un recordatorio constante de
una noche de 1944, de una noche en la que intenté contactar con otro
mundo --su voz se mantuvo hasta agrietarse con emoción cuando comenzó a
observarle mejor--. Esto es el recordatorio de una plegaria que hice a
un hombre en busca de ayuda. Un hombre que dijo que no podía, que no me
ayudaría --su voz temblaba con furia--. ¡Este es un constante
recordatorio de ti!\\
Se arrancó la máscara de la cara y la habitación resonó con el grito de
horror de Angela.
