\chapter*{CAPÍTULO TRES}
\addcontentsline{toc}{chapter}{CAPÍTULO TRES}
Poco tiempo después, el Doctor y Clara estaban en la boca del paso
subterráneo una vez más mientras Charlie Bevan echaba un vistazo más de
cerca al cadáver. Angela se había quedado en la clínica, diciendo que
quería continuar con el examen, pero Clara diría que era solo una excusa
para evitar ver de primera mano que la araña se había ido. No podía
decir que la culpara.\\
--¡Ten cuidado de no tocar la telaraña! --gritó el Doctor-- No quiero
estar intentando liberarte cuando vuelva esta cosa.\\
Con un pañuelo cubriéndole la boca contra el olor que comenzaba a
asentarse, Charlie se abrió camino de vuelta a ellos, con su cara muy
grave.\\
--¿Alguna idea de quién es? --preguntó Clara, suavemente.\\
Charlie asintió.\\
--Alan Travers.\\
--¿De por aquí?\\
--Es el dueño\ldots{} era el dueño de una granja de lácteos a unas
kilómetros de distancia. Conociendo a Alan, probablemente estaba
volviendo a casa después de una noche en el Wheatsheaf.\\
El agente Bevan aferró su radio.\\
--Debería avisar de esto, intentar organizar una partida de búsqueda de
algún tipo para encontrar a esta criatura.\\
El Doctor asió del brazo al policía.\\
--Antes de que hagas eso, creo que deberíamos quitar el cuerpo de la
telaraña.\\
Charlie negó con su cabeza.\\
--El equipo forense va a querer examinarlo donde está.\\
--No se quedó ahí colgado como decoración --dijo el Doctor, fríamente--.
Esta es una despensa, y si nuestra amigable araña del vecindario se pone
hambrienta y vuelve no habrá nada para que lo examine el equipo forense
--se detuvo--. Además, me gustaría hacer una autopsia.\\
Charlie le miró precavidamente.\\
--¿Eres doctor de algún tipo?\\
--De algún tipo, sí.\\
--Bueno, lo siento. ~Eso está fuera de toda cuestión. Tendremos que
llevar el cuerpo al hospital de Chippenham. El doctor Goodchild puede
llevar a cabo la autopsia.\\
La expresión del Doctor se oscureció, y durante un momento Clara pensó
que iba a discutir con él. Entonces, obviamente pensándoselo mejor,
liberó el agarre del brazo del agente y echó un vistazo al cuerpo
colgado en el túnel otra vez.\\
--Vamos a necesitar ayuda para bajarle.\\
Charlie señaló al pub justo visible a través de los árboles en el otro
lado del campo.\\
--Iré y conseguiré ayuda en el Wheatsheaf.\\
--Bien --el Doctor asintió--. Diles que traigan guantes y una
escalerilla y tijeras de algún tipo. Podaderas o cizallas de jardín.\\
Cuando Charlie se apresuró por el camino hacia el pub, el Doctor le
llamó.\\
--¡Y encuentra algo con lo que envolverle!\\
Se giró para ver a Clara mirándole intrigantemente.\\
--¿Autopsia?\\
El Doctor asintió.\\
--No había veneno suficiente en la herida de punción como para que
hiciera un análisis decente. Quiero ver si hay los mismos indicadores de
alteración genética.\\
--¿Así que quieres comprobar que quienquiera que diseñase el insecto
segador sea la misma persona responsable de la araña?\\
--Exactamente.\\
--¡Bueno, mejor que sea así, sino sería una buena coincidencia!\\
El Doctor sonrió lúgubremente.\\
--¡Tampoco puede ser una coincidencia que la TARDIS nos traiga a este
punto exacto siguiendo el rastro de una misteriosa firma de energía!\\
--¿Pero qué demonios podría conectar insectos gigantes, círculos de
piedra y líneas ley?\\
--Supongo que vamos a averiguarlo.\\
Los ojos del Doctor brillaron y Clara sintió un escalofrío de
anticipación recorrerle la espalda. A pesar del peligro, a pesar de la
muerte, eran esos momentos los que le daban la vida, haciendo que cada
segundo de su extraordinaria vida mereciera la pena. Ella y el Doctor (a
pesar de los cambios de su apariencia exterior aquel seguía siendo su
Doctor), lado a lado, enfrentándose a lo que fuera que el universo
pudiera lanzarles.\\
Unos momentos más tarde, el sonido del motor de un coche le hizo
levantar la mirada, y se giró para ver una maltrecha furgoneta frenando
al principio del paso subterráneo. Charlie Bevan y otro hombre se
bajaron y caminaron hacia ellos.\\
--Doctor, Clara, este es Bert Mitchell --dijo Charlie--. Es el dueño del
Wheatsheaf.\\
Con una educación de la vieja escuela, Bert les dio la mano a ambos,
pero a pesar de eso, no podía apartar los ojos de la forma envuelta que
colgaba en la boca del paso.\\
Clara sintió un impacto de remordimiento de su anterior emoción. El
hombre muerto era uno de sus amigos, sus vecinos. Aquello no era
emocionante para aquella pobre gente, era terrorífico. Intentó darle una
sonrisa reconfortante.\\
--Lo siento. Esto debe de ser difícil de creer\ldots{}\\
--¿Difícil? --Bert negó con la cabeza-- Imposible, más bien.\\
El Doctor, Charlie y Bert comenzaron a desenredar el cuerpo de Alan
Travers de la pegajosa telaraña. Clara se sentó en lo alto del terraplén
de la vía, vigilando por si había alguna señal de una araña. Incluso con
las pesadas cizallas que Bert había traído, las hebras de telaraña
demostraron ser increíblemente difíciles de cortar, y les llevó varios
minutos de trabajo duro antes de que finalmente pudieran liberar el
cuerpo. Envolviéndolo con una sábana, los tres hombres se esforzaron por
el camino hasta la furgoneta y cargaron el cuerpo con cuidado en la
parte trasera.\\
Respirando pesadamente, Charlie se limpió el sudor de su frente.\\
--Espera aquí un momento, Bert. Voy a dejar un mensaje en el hospital,
que el doctor Goodchild sepa que estáis yendo.\\
Cuando Charlie se apresuró a cruzar la plaza hacia la oficina de la
veterinaria, Clara pudo ver a la gente comenzando a ir a sus asuntos en
el pueblo, todavía inconscientes del horror que comenzaba a
desarrollarse entre ellos. Bert estaba obviamente pensando lo mismo.\\
--Nadie va a creerlo --dijo él, en voz baja--. Después de todas esas
bromas de que criaban monstruos en el parque científico.\\
Las orejas del Doctor de repente se levantaron.\\
--¿Parque científico? ¿Qué parque científico?\\
--Al otro lado de la vía --Bert asintió hacia el terraplén lejano--.
Investigación científica agrícola. Nuevos fertilizantes, cosechas
resistentes a la meteorología, biodiesel, ese tipo de cosas. Ha habido
bromas de que Jason Clearfield ha estado criando monstruos ahí dentro
--negó con la cabeza--. Nunca pensé que resultara ser cierto.\\
Clara intercambió una breve mirada con el Doctor, pero antes de que
pudieran preguntar a Bert nada más, Charlie Bevan volvió a salir de la
oficina de la veterinaria.\\
--¿Funciona tu radio de banda ciudadana, Bert?\\
El dueño del pub se subió a la cabina de su furgoneta, encendió el motor
y encendió la radio de comunicación en su salpicadero.\\
--Eso parece, ¿por qué?\\
--Las líneas de teléfono no van del todo bien. Los móviles igual.
Conéctalo con la frecuencia de la policía. Llámame cuando llegues al
hospital. Tengo un par de cosas que terminar por aquí.\\
Bert asintió, cerrando la puerta y poniendo la primera marcha de la
furgoneta. Tras observarle irse del pueblo, Clara se giró hacia Charlie
con un ceño intrigado y fruncido.\\
--¿Banda ciudadana? ¿La gente sigue usando eso?\\
Charlie asintió.\\
--La cobertura del móvil no es muy buena aquí fuera, así que la banda
ciudadana sigue siendo popular en la comunidad agraria. Es fácil de
usar, fácil de arreglar\ldots{}\\
El Doctor le interrumpió.\\
--¿Qué es lo que pasaba con los teléfonos?\\
Él se encogió de hombros.\\
--Interferencias de algún tipo. No podía encontrar señal móvil para
nada. Pude usar la línea terrestre de Angela para ponerme en contacto
con el hospital, pero la línea murió cuando estaba al teléfono con el
coronel.\\
--¿Coronel? --la voz del Doctor se volvió fría.\\
--Sí, con el coronel Dickinson en la base militar de Warminster.\\
El Doctor cerró sus ojos.\\
--No, no, no, no, no, no\ldots{} ¿por qué has tenido que hacer eso?\\
--¿Perdón?\\
--¡Involucrar al ejército!\\
--Porque tenemos una araña gigantesca merodeando por los campos y
matando gente. ¿A quién, sino, pensabas que iba a llamar? ¿Al zoo
local?\\
--¡Sí! --gritó el Doctor-- ¡Esa puede que hubiera sido la idea sensata!
¡Al menos podrían querer capturarla en vez de dispararla! ¿De qué nos
sirve muerta?\\
--¿De qué nos sirve de ninguna manera? --la cara de Charlie enrojeció
por la furia-- Ahora, escucha, no sé quién eres, pero si crees
que\ldots{}\\
--¡Parad! --Angela Drabble se alzaba en la entrada de la clínica
veterinaria lanzando miradas de odio a ambos hombres-- ¡Parad, los dos!
¡Estáis parloteando como niños!\\
Tal fue el tono de la voz de la joven que ambos hombres se callaron de
inmediato. Clara tuvo que ocultar una sonrisa. Era raro que alguien
tuviera la autoridad como para cortar al Doctor de esa manera. Ella iba
a llevarse bien con la veterinaria del pueblo.\\
Angela se acercó a ellos.\\
--Eso está mejor. Ya tenemos bastantes problemas sin que vosotros dos
actuéis como una pareja de gatos de granja dando vueltas para tener una
pelea. Ya hemos perdido un hombre, asegurémonos de que no perdemos
ningún otro.\\
Señaló a la gente abriéndose camino de un lado a otro de la plaza del
pueblo.\\
--Vamos a tener que informar a la gente de lo que está pasando. Hacer
que protejan a sus hijos en sus casas, que no salgan, que se armen si es
necesario.\\
Tímidamente, el Doctor asintió, estando de acuerdo con ella.\\
--Tienes bastante razón --observó la torre del campanario de la
iglesia--. Quizá pueda armar algún tipo de transmisor de radio de onda
corta. Si lo estimulo con mi destornillador sónico, puedo hacer que la
campana de la iglesia sea capaz de emitir algún tipo de advertencia para
que todo el mundo pueda oírla.\\
--O --dijo Charlie--, podemos convocar una reunión del
pueblo.\\[2\baselineskip]--¡Cabo Jenkins! ¿Ya ha conseguido reconectar
mi llamada con Ringstone?\\
--No, señor. La línea sigue muerta.\\
El coronel Paddy Dickinson se volvió a sentar en su silla, con un ceño
aumentando en sus rasgos desgastados por el tiempo. Lo que Charlie Bevan
le acababa de decir sonaba completamente absurdo, y Dickinson, sin
embargo, se había encontrado con el hombre en varias ocasiones, más
recientemente en el partido de rugby de policía contra ejército del mes
pasado, y siempre le había parecido ser un policía profesional sólido y
fiable. Ciertamente no le parecía un hombre dado a soñar despierto.\\
¿Pero insectos gigantes?\\
Después de unos pocos minutos de deliberación, alargó la mano al
teléfono otra vez.\\
--¿Jenkins? Póngame con la Comandancia de Coordinación de Helicópteros
en Andover.\\[2\baselineskip]Diez minutos más tarde, el Lynx estaba en
el aire, con sus dos poderosos motores Rolls Royce Gem 42 levantándolo
sin esfuerzos del hormigón y mandándolo precipitadamente por encima del
campo británico a 180 nudos.\\
La tripulación había recibido las más vagas instrucciones, hacer un
reconocimiento en la zona e informar de cualquier cosa fuera de lo
común, pero el hecho de que hubieran sido mandados en un AW159 Lynx
Wildcat equipado con una metralleta M3M de calibre 50, le había dado a
la misión el escalofrío de la emoción inesperada.\\
La capitana Jo Phillips, la piloto, no podía entenderlo. Ella había
estado en Ringstone el mes pasado con su prometido, un día de ``vamos a
conocernos'' con sus futuros suegros. El pueblo no era nada más que
tiendas de antigüedades y teterías: nada que pudiera provocar una
respuesta militar de aquel tipo.\\
Echó una mirada a su copiloto, Mike Vickers.\\
--¿Qué piensas?\\
Vickers se encogió de hombros, desenvolviendo una tira de chicle y
metiéndosela en la boca.\\
--Podría ser cualquier cosa --dijo él, de forma indistinta. Vickers era
medio bahameño e invariablemente despreocupado de todo.\\
--¿Recordáis el mes pasado? --le gritó Leigh Brewster desde la cabina
detrás de ella-- Nos movilizaron porque un granjero pensaba que había
visto un OVNI, el muy paleto estúpido.\\
Phillips sonrió. El OVNI había resultado ser un dirigible de anuncios
que se había soltado de una convención exterior de ciencia ficción en
Devizes. Lo habían pintado para que pareciera ser el Halcón Milenario y
había parecido bastante bueno, así que no era de extrañar que hubiera
provocado tal revuelo. Pero Brewster no tenía mucho tiempo para los
locales, quería volver al campo. Phillips pensó que estaba loco. Ya
había visto bastante de los desiertos de Afganistán para toda su vida.\\
Viendo la torre de la iglesia de Ringstone asomándose por encima de los
bosques lejanos, Phillis bajó el helicóptero, poniéndolo cercano a los
campos.\\
--Vale, veamos si podemos averiguar de qué trata todo
esto.\\[2\baselineskip]Gabby Nichols estaba en la parte trasera del
salón del pueblo escuchando con creciente pánico como el agente Bevan,
Angela y un hombre de cara delgada y de aspecto terrorífico informaban
de la muerte de Alan Travers, y comenzaban a explicar la infestación de
insectos gigantes que parecía estar afectando al pueblo.\\
Si Gabby no hubiera ya sido testigo de las criaturas de primera mano,
podría haber pensado que era algún tipo de broma práctica y macabra.
Varios de sus vecinos a su alrededor estaban pensando aquello.\\
--¡Vamos, venga! --una voz que sonó como Simon George, el jefe de la
oficina de correos, sonó desde la parte delantera del salón-- Debéis
pensar que somos estúpidos.\\
Hubo un murmullo de acuerdo por la sala.\\
--Es un poco temprano para el Día de los Inocentes --gritó otra
persona.\\
--Soy consciente de que esto debe de sonar increíble --Charlie Bevan
parecía nervioso--. Pero hay un peligro muy real\ldots{}\\
--De que tú parezcas un estúpido total --acabó Simon, forzando risas en
la multitud--. Alguien ha estado gastándote una broma mala, Charlie
Bevan, y has caído del todo.\\
--Entonces quizá te gustaría venir con nosotros al hospital y examinar
por ti mismo el cuerpo --la voz del hombre de cara delgada, el Doctor,
se había llamado, cortó a través de la habitación--. Nadie está gastando
bromas. Estas criaturas son muy reales y peligrosas, y si no comenzáis a
escucharnos entonces alguien más en esta habitación probablemente acabe
muerto.\\
Ahí sí que hubo silencio. Bevan lanzó un asentimiento agradecido al
Doctor.\\
--Gracias, Doctor. Bien, esto es lo que propongo que hagamos\ldots{}
probablemente ya os hayáis dado cuenta de que tenemos un problema con
los teléfonos. Esto parece estar afectando tanto a las líneas terrestres
y a los móviles\ldots{}\\
Gabby escuchó aturdida. Eso explicaba por qué no había sido capaz de
contactar con Roy. Echó una vista a la pantalla de su teléfono. Llevaba
toda la mañana igual: todavía buscando cobertura.\\
Hubo un repentino estirón en sus pantalones.\\
--¿Cuándo va a dejar de hablar ese hombre, mamá? --Emily tenía la mirada
levantada hacia ella, suplicantemente-- Quiero irme.\\
Esa simple declaración era como una burbuja ligera subiendo en la cabeza
de Gabby. Su coche estaba aparcado en el otro lado de la plaza. Podrían
marcharse sin más.\\
Subiéndose a su hijo dormido al hombro, agarró la mano de su hija y se
abrió camino por el abarrotado salón. Apresurándose a cruzar la plaza,
rebuscó en su bolso en busca de las llaves de su coche. Las había
perdido más veces de las que le gustaba recordar desde que se mudaron,
así que hasta que se acostumbrase a sus nuevos alrededores, siempre las
había llevado encima.\\
Agradecida por su plan, abrió el coche, metió a Emily en su sillín
trasero y comenzó a abrochar a Wayne en su cuna de viaje.\\
--¿A dónde vamos? --Emily tenía su expresión intrigada.\\
Gabby lo pensó durante un momento. Tampoco es que hubiera pensado mucho
a dónde ir.\\
--A casa de la abuela --dijo después de pensarlo durante un momento.
Podían llegar a casa de su madre en Wolverhampton en un par de horas.\\
--¿Vamos a ir primero a casa?\\
Gabby vaciló. ¿Debería hacer un viaje a casa a por cosas esenciales?
Wayne necesitaría pañales y comida. Y Emily solo llevaba un ligero
vestido de algodón.\\
Ella negó con la cabeza. Siempre podían comprar lo que necesitasen en
ruta. Se pararían en el supermercado de Devizes. Tenía consigo su
tarjeta de crédito y su móvil. En cuanto pudiera tener cobertura,
llamaría a Roy y le diría a dónde había ido.\\
--No, cielo. Tenemos que ir directamente --aseguró a Emily en su silla,
luego se deslizó en el asiento del conductor, abrochándose su propio
cinturón y metiendo la llave en su hueco. Para su alivio, el coche
arrancó a la primera.\\
Cuando salió a la carretera, todo parecía tan tranquilo y tan normal que
por un segundo se preguntó si se había vuelto loca. Entonces el recuerdo
de los gritos de Emily por el insecto segador, y la horrible forma en la
que se había removido cuando lo había apartado de un manotazo de la
ventana y había intentado aplastarlo, desvaneció cualquier duda que
hubiera podido tener.\\
Cambiando de marcha, giró por el salón del pueblo y se dirigió hacia el
exterior del pueblo a toda velocidad. Normalmente trataba las estrechas
calles del pueblo con cuidado pero hoy solo quería alejarse lo antes
posible de Ringstone.\\
Cuando pasaron a través de las carreteras alineadas con setos, Emily
hizo un pequeño ``¡Wi!'' de emoción. Gabby no pudo evitar sonreír.
Niños.\\
Su sonrisa se desvaneció cuando giró una curva estrecha para encontrar
una carretera bloqueada por delante por un grueso lío de fibras blancas
como la leche. Gabby frenó en seco, pero iba demasiado rápido como para
detenerse a tiempo. El pequeño Fiat se estrelló contra la telaraña.\\
El impacto dejó a Gabby sin aliento. El interior del coche fue una
cacofonía de gritos. Recuperándose, comprobó rápidamente la cuna de
viaje a su lado. Wayne estaba bien, solo que no estaba contento. Miró
por el retrovisor; Emily también estaba atada de forma segura.
Satisfecha de que ambos niños estuvieran sanos y salvos, Gabby puso el
coche marcha atrás. Los neumáticos chirriaron ruidosamente en el
hormigón y el coche se tambaleó, pero la telaraña les contuvo. Cambiando
a una marcha adelante, intentó abrirse camino avanzando, pero el coche
no se movía ni un ápice.\\
Casi gritando con frustración, Gabby pegó un golpe poniendo el coche
otra vez marcha atrás y apretó el acelerador completamente hasta el
fondo. Unas nubes de humo blanco comenzaron a alzarse de los neumáticos
a medida que rodaban de forma inútil y, con una tos titubeante, el motor
se caló.\\
Gabby pegó un golpe al volante, intentando ahogar los gritos de Emily y
Wayne. ¿Qué iba a hacer? Un ruido repentino le hizo pegar un bote.\\
Levantó la mirada para ver las hebras de la telaraña justo fuera del
parabrisas comenzando a moverse, a vibrar. Frunció el ceño. ¿Qué
demonios podría estar causándolo? Entonces, con un escalofrío que le
caló hondo, se dio cuenta de que lo que hubiera tejido la telaraña
estaba viniendo a investigar qué había atrapado.\\
El miedo impulsó a Gabby a entrar en acción. Desabrochándose, intentó
abrir la puerta del conductor, pero la telaraña evitó que se abriera del
todo. Comenzó a bajar la ventanilla, luego se detuvo. Si la red era lo
bastante pegajosa como para contener el coche, entonces ciertamente
sería capaz de atraparla si la tocaba.\\
Se giró en su asiento. El coche se había deslizado en la red en ángulo,
y la puerta trasera estaba libre de cualquier hebra. Ignorando los
lloros de su hija, se estiró entre ambos asientos delanteros, intentando
llegar al manillar de la puerta trasera. La puerta se abrió y Gabby se
tambaleó contra la carretera.\\
Desabrochando el cinturón de Emily, la sacó de su asiento dejándola en
el hormigón sin ninguna ceremonia mientras extraía a Wayne de su cuna.
Desde algún lugar tras ella podía oír algo grande acercándose
sigilosamente a través de los campos cercanos. Luchó contra el deseo de
mirar a su alrededor, concentrándose en deshacer los broches y las tiras
que mantenían seguro a su hijo.\\
Tres más\ldots{} dos más\ldots{} el último cierre se abrió y sacó a
Wayne del coche. Cuando se giró para agarrar a Emily, se dio cuenta de
que su hija se había quedado callada y estaba observando aterrorizada
algo por encima de su hombro.\\
Apenas capaz de respirar, Gabby se giró lentamente. Dos patas largas
cubiertas de pelo estaban asomándose por encima del seto. A medida que
observaba, algo enorme y oscuro comenzó a impulsarse hacia la telaraña.
Gabby alargó la mano hacia la mano de su hija, lentamente retrocediendo
de la araña monstruosa a medida que sus patas delanteras comenzaban a
corretear por encima del tejado metálico del coche, intentando
determinar si era comida o no.\\
Gabby sintió un repentino arrebato de esperanza. ¡No les había visto!
Estaba más interesada en el coche. Sus ojos bajaron hasta encontrarse
con los de su hija y se puso un dedo sobre los labios.\\
Para su alivio, su hija asintió y juntas comenzaron a retroceder de la
telaraña. A medida que Gabby se tensaba para comenzar a correr hubo un
duro zumbido a su espalda y algo aterrizó duramente sobre ella.\\
Todos los pensamientos de sigilo fueron abandonados y Gabby gritó y se
giró, intentando deshacerse de lo que fuera. Entonces hubo un repentino
y afilado dolor entre sus omoplatos, y sus piernas le fallaron.\\
Al caer contra el suelo, fue consciente del monótono zumbido de docenas
de pares de alas ahogando los hijos de su hija y una nube de siluetas
acechantes y zumbantes cerrando a su alrededor.\\
Solo cuando perdió la conciencia, Gabby se acordó de las doce latas de
insecticida en el maletero de su coche.\\
