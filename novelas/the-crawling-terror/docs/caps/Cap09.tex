\chapter*{CAPÍTULO NUEVE}
\addcontentsline{toc}{chapter}{CAPÍTULO NUEVE}
La puerta se cerró detrás de ellas y Clara se quedó quieta durante
varios segundos, esperando a que sus ojos se acostumbraran a la
oscuridad del interior del edificio.\\
La zona justo dentro de la puerta estaba abarrotada con cajas de
empaquetar y palés desechados. Un toro mecánico estaba aparcado contra
la pared, junto con estantes y estantes de cables eléctricos pesados.
Por encima de sus cabezas, el techo estaba poblado de montacargas de
cadenas, y había una pasarela metálica que recorría el interior del
edificio, con escalerillas de acceso posicionadas a cada par de
metros.\\
Clara se acercó a pesar de las cajas. El edificio parecía desierto, pero
por encima de ella podía distinguir un círculo de grandes siluetas con
formas de bloques iluminadas por una pálida luz azul. Manteniéndose en
las sombras cercanas a la pared, las dos mujeres comenzaron a abrirse
camino hacia la luz.\\
A medida que se acercaban comenzaron a oír ruidos, un extraño zumbido
eléctrico y bajo y otros sonidos más familiares. Chasquidos y zumbidos.
Ruidos de insectos.\\
Contra la pared más alejada del edificio se alzaban docenas de cajas de
cables y cristal. Algunas estaban vacías, con sus puertas abiertas;
otras todavía contenían unas enormes figuras monstruosas que se
aferraban a una red de mallas o zumbaban y se agitaban en el medio de
las celdas. Angela retrocedió aterrorizada.\\
--¿Qué son?\\
Clara tuvo que admitir que no lo sabía. A primera vista, había pensado
que las criaturas eran arañas enormes, pero tras obligarse a mirar más
de cerca podía ver alas y escamas, y demasiadas patas. Era como si
alguien hubiera desmembrado tres o cuatro tipos de insecto y luego
hubiera pegado las partes juntas en un orden distinto. Una de las
criaturas le hizo un chasquido vago y ella pegó un bote hacia atrás.\\
Por detrás de las cajas había bancos llenos de complejos equipos
científicos, y una gran ventana de cristal que miraba hacia otra
habitación que contenía un laboratorio sofisticado.\\
--¿Quién está haciendo esto? --susurró Angela, furiosamente-- ¿Y para
qué demonios están aquí?\\
--No lo sé --dijo Clara--, pero me sorprendería si no tuviera algo que
ver con eso.\\
En el medio de la habitación se alzaban trece grandes cajas negras, cada
una del tamaño de un armario, cada una con un gran conjunto de LED
colocados en la parte frontal. Unos cables gruesos como brazos desde la
base de cada uno se conectaban a una gran máquina enorme de color gris
con la forma de una campana, rodeada por consolas, varias CPU y
generadores de tamaño industrial.\\
Soltando la mano de Angela, Clara comenzó a abrirse paso hacia los
monolitos que hacían zumbidos. Al acercarse, pudo ver imágenes
parpadeando a través de las pantallas de LED montadas en las cajas,
patrones que parecían de alguna manera familiares. Solo cuando estuvo de
pie en el medio del círculo hizo la conexión. Los patrones en las
pantallas eran los mismos que los patrones inscritos en las piedras de
fuera del pueblo.\\
--Es el círculo de piedras --susurró ella--. Alguien ha construido una
versión tecnológica del círculo de piedras.\\
--Muy inteligente por tu parte.\\
La voz cortó a través de la sala como un disparo. Clara se escudó los
ojos cuando las luces a su alrededor se encendieron de un chasquido,
inundando el almacén con una brillante luz blanca. Cuando las luces se
encendieron, los insectos híbridos en las jaulas gritaron por el
disgusto, llenando el aire con una cacofonía de gritos y silbidos.
Angela corrió al lado de Clara cuando las figuras de bata blanca las
rodearon.\\
--Cubrid esas cosas.\\
Cuando dos siluetas corrieron a las jaulas, un hombre alto con un
inmaculado traje gris dio un paso hacia el centro del círculo. Lo
primero que Clara vio era que sujetaba lo que parecía un anticuado
revólver de la Segunda Guerra Mundial. Lo segundo fue que la mitad de su
cara estaba cubierta por una máscara de plástico blanca.\\
Por las descripciones que ya había oído solo había una persona que
pudiera ser.\\
Aquel era el misterioso Jason Clearfield.\\[2\baselineskip]Solo unos
reflejos rápidos como un relámpago salvaron la vida del Doctor. Cuando
la escopeta disparó se lanzó a un lado, aterrizando fuertemente sobre su
hombro y rodando hasta ponerse en pie con un movimiento suave y fluido.
Una maldición ahogada vino desde dentro de la casa.\\
Corriendo hacia adelante, el Doctor agarró el cañón de la escopeta y
estiró duramente. Hubo un fuerte ruido sordo y un gruñido de dolor desde
el otro lado de la puerta. De inmediato, el Doctor liberó la escopeta y
apretó su destornillador sónico contra el cerrojo, golpeando su hombro
contra la puerta al hacerlo.\\
Se lanzó hacia dentro, y hubo otro grito de dolor cuando su asaltante
fue lanzado hacia el suelo. El Doctor se abrió camino a empujones. Un
hombre mayor estaba en el pasillo de baldosas agarrándose el hombro. El
Doctor le arrancó la escopeta y le miró con desagrado.\\
--Bueno, está bien ver que sigues el modelo militar tradicional de
disparar primero y hacer las preguntas después.\\
Charlie y Kevin se abrieron paso por la puerta detrás de él, con el
oficial apresurándose hacia adelante para ayudar al hombre a ponerse en
pie.\\
--¿Qué demonios has pensado que estabas haciendo, Robin? ¡Podrías
habernos matado!\\
--Creía que erais ellos --el anciano hizo una mueca de dolor--. Creía
que habíais venido para convertirme en otro zombi.\\
--Oh, bueno, entonces no pasa nada --soltó el Doctor, riendo
desdeñosamente--. Está perfectamente bien intentar matar a tus amigos y
vecinos a pesar del hecho de que están siendo usados en contra de su
voluntad.\\
Robin le lanzó una mirada furiosa.\\
--No hay amigos míos ahí fuera.\\
Los dos hombres se miraron con odio durante un momento, luego Robin
respiró hondo.\\
--Será mejor que paséis --se giró hacia Kevin--. Tú, chico, cierra la
puerta. Y asegúrate de cerrarla bien. No quiero que esos sigilosos
escalofriantes entren.\\
Robin les llevó vestíbulo abajo hacia una gran cocina hogareña. Una caja
abierta de cartuchos de escopeta se hallaba sobre una antigua mesa de
roble que dominaba la sala. El Doctor le pasó la escopeta a Charlie.\\
--Ten, por mucho que me desagraden esas cosas, me parecería tonto
guardarla. Solo por si acaso.\\
Charlie la tomó con un asentimiento, desbloqueándola y quitándole los
cartuchos usados. Robin se sentó pesadamente en una de las sillas de la
cocina, frotándose el hombro, obviamente aún herido por la caída que
había soportado. El Doctor cruzó la sala hasta su lado.\\
--Será mejor que me dejes echarle un vistazo a eso.\\
--¿Por qué? ¿Eres algún tipo de doctor?\\
--Sí, algo así --el Doctor comenzó a comprobar los músculos del hombro
de Robin, ignorando los gruñidos de dolor que emitía--. Nada parece
estar roto.\\
--No gracias a ti --gruñó Robin--. Sin embargo, supongo que es algo.
Pero no va a endurecer algo podrido.\\
--No mucho --el Doctor reajustó el destornillador sónico y recorrió la
punta por el hombro de Robin, llenando la cocina con un suave ruido
burbujeante. --. Tendrás un moratón horroroso durante un tiempo, pero
aparte de eso, estás como nuevo.\\
Robin movió su hombro de forma experimental, y luego levantó la mirada
hacia el Doctor con asombro.\\
--Bueno, ¡caramba!\\
El Doctor colocó otra silla y se sentó enfrente de él, con los codos en
la mesa, con la barbilla entre sus manos y los ojos brillando con
energía.\\
--Y ahora, señor Sanford, quizá te gustaría hablarme de tus experiencias
luchando contra insectos gigantes durante la
guerra.\\[2\baselineskip]--¿Disculpe, señor?\\
El cabo Palmer se alzaba en la entrada del centro de comando móvil.\\
--¿Sí, Cabo? --el capitán Wilson levantó la mirada de forma agradecida
desde su portátil, sintiendo los tendones en su cuello chasquear al
hacerlo. Odiaba el trabajo de oficina.\\
--Se han oído tiros de escopeta dentro del perímetro, señor.\\
El coronel Dickinson miró de inmediato desde la pantalla de su
ordenador.\\
--¿Tiros? ¿Dónde?\\
--En el lado sur del pueblo, señor. Una escopeta, probablemente.\\
--Entonces será mejor que echemos un vistazo. Que preparen mi Land
Rover, Palmer.\\
--Señor.\\
Cuando el cabo se apresuró, Dickinson se recostó en su silla, apretando
sus labios pensativamente.\\
--¿Qué piensas, capitán? ¿Resistencia dentro del pueblo?\\
Wilson se encogió de hombros.\\
--Hay suficientes granjeros con escopetas que es difícilmente
sorprendente que alguien suelte una de esas cosas al cabo del tiempo
--se detuvo--. Si te soy sincero, me sorprende no haber oído más.\\
--Lo que sugiere que ha habido algún tipo de supresión de los
lugareños.\\
--Posiblemente. Asumiendo que el resto de ellos no estén muertos.\\
El coronel estuvo callado durante un momento, luego se levantó y cerró
la puerta del centro de mando.\\
--Capitán, lo que estoy a punto de contarle es confidencial. Concierne a
la Wunderwaffe.\\
--Lo siento señor --Wilson negó con la cabeza--. Las lenguas nunca me
han sido un punto fuerte.\\
--Una Arma de las Maravillas. Un aparato tecnológico nazi altamente
secreto conocido como Die Glocke.\\
El capitán Wilson le miró pálidamente.\\
--¿Die Glocke?\\
--La Campana.\\
--¿La Campana? --Charlie Bevan le miró extrañado-- ¿Qué demonios es
eso?\\[2\baselineskip]Robin Sanford se puso en pie desequilibradamente,
cruzó la cocina y sacó una gastada taza de apariencia antigua de una
estantería, dejando caer una bolsa de té en su interior y se colocó
junto a una tetera de igual apariencia.\\
--Durante la Segunda Guerra Mundial, el ejército británico interceptó
una señal de radio codificada. En ese tiempo intentaban descifrar los
códigos de los submarinos alemanes, y todo el mundo asumió al principio
que solo era una encriptación ultra compleja nueva. El problema era que
la señal no venía del Atlántico Norte, se originó en el espacio
profundo.\\
--¡Tienes que estar bromeando! --bufó Charlie, despectivamente--
Propaganda alemana, seguramente.\\
--Mucha gente pensó lo mismo en ese momento, pero luego comenzaron a
tener informes de los agentes dentro de Europa indicando que los
alemanes habían detectado también esta señal, y que estaban a punto de
traducirlo.\\
Robin observó la tetera hervir, perdido en sus recuerdos durante un
momento. El Doctor se sentó y esperó, bastante feliz de dejarle contar
la historia a su propio ritmo, pero Charlie estaba impaciente.\\
--¿Y bien? ¿Qué decía?\\
--Les llevó a las mejores mentes meses poder conseguir una burda
traducción. Turing, Welchman, Knox\ldots{} Todos lo probaron. Incluso
tuvieron a Judson trabajando en ello durante un tiempo.\\
El Doctor se enderezó ante la mención de ese nombre.\\
--¿Judson?\\
--Sí, antes de que lo mandaran a trabajar en el proyecto ULTIMA --la
tetera se apagó con un chasquido y Robin sirvió el agua hirviente en su
taza--. Era un conjunto de instrucciones para construir una máquina, una
gigantesca campana de metal de unos tres metros y medio de altura. Nadie
tenía la mínima idea de para qué era, pero los alemanes ya habían
comenzado su construcción, así que no tuvimos más elección que construir
una nosotros también.\\
--Debías de ser muy joven en esa época --dijo el Doctor.\\
Robin asintió, sorbiendo su té.\\
--Tenía dieciocho. Se me negó el servicio activo por una dolencia
cardíaca hereditaria. Pero me hicieron soldado en la Guardia Local aquí
en Ringstone.\\
--Creía que la Guardia Local solo servía para proteger las playas y las
costas --comentó Kevin.\\
--Muy cierto, jovencito, pero también protegían instalaciones militares
importantes.\\
--Como Ringstone --dijo el Doctor en voz baja.\\
--Sí, Doctor. Aquí es donde el ejército británico construyó y probó su
versión de la Campana.\\[2\baselineskip]El Coronel Dickinson se inclinó
sobre su ordenador.\\
--Una vez comencé a hacer preguntas sobre Ringstone y comenzaron a sonar
todo tipo de campanas de alarma en cadena --soltó una risa irónica--. No
pretendía bromas.\\
Sacó una serie de archivos que mostraban planos, heliografías y unas
fotografías granuladas blancas y negras.\\
--Die Glocke era el más secreto de los programas de armas nazi,
conducido durante los meses de decadencia de la guerra. Codificado con
el nombre de Proyecto Cronos estaba bajo el comando de un general de las
SS llamado Hans Kammler. Los nazis estaban desesperándose en ese
momento, y estaban convencidos de que aquello era algún tipo de
superarma o un nuevo tipo de nave aérea o un aparato de antigravedad.\\
Aumentó una serie de imágenes aéreas de un complejo de fábricas en una
estéril zona montañosa del norte de Europa.\\
--Construyeron su aparato en lo que ahora es Polonia, en la mina
Wenceslas en los Sudetes. Una presa local se usaba para generar energía
hidroeléctrica para los experimentos y unos presos de un campo de
concentración cercano eran usados como mano de obra para la construcción
del complejo de pruebas.\\
Wilson se reclinó hacia adelante y señaló una forma circular indistinta
en la imagen aérea.\\
--¿Qué es esto? Parece, como\ldots{} --se interrumpió, sintiéndose de
repente estúpido.\\
--¿Sí? --el coronel le miró de forma expectante.\\
--Bueno, parece Stonehenge.\\
--Muy cierto, capitán Wilson --Dickinson pulsó otra tecla y otra imagen
apareció en la pantalla, una imagen contemporánea de una gran estructura
circular de piedra, alzándose en el centro de una expansión de hormigón
descuidada.\\
--Los alemanes lo llamaban ``el Henge'' o ``la Trampa para Moscas''. Ahí
fue donde Die Glocke fue probada y es todo lo que queda del proyecto
Cronos.\\
El capitán Wilson se recostó en su silla, su cara siendo una máscara de
confusión.\\
--No estoy seguro de entenderlo, señor. ¿Superarmas nazis y círculos de
piedra?\\
El coronel Dickinson cerró el portátil.\\
--Los nazis habían malentendido fundamentalmente la naturaleza de la
máquina. Hitler era un iluso, convencido de que era algún tipo de arma
perfecta, concedida por los alienígenas para ayudarle a ganar la guerra.
Mucha gente murió por esa creencia. Varios científicos murieron en la
primera operación del Die Glocke. Incluso con los refinamientos de la
máquina y la ropa protectora, cinco de los siete científicos que
llevaron los últimos experimentos murieron por su exposición a ello --su
expresión se endureció--. Y eso ni siquiera comienza a cubrir el número
que fueron expuestos a ello deliberadamente, o los trabajadores que
fueron asesinados por las SS al final de la guerra para mantenerlo en
secreto.\\
--Asumo que los experimentos británicos tuvieron más éxito.\\
El coronel asintió.\\
--Al parecer, no era solo cuestión de construir la máquina, era cuestión
de encontrar el tipo de energía correcta para hacerla funcionar. Al
final fue Aleister Crowley quien solucionó el problema para los
Aliados.\\
--¿El ocultista? --Wilson se esforzó para alejar la incredulidad de esa
voz.\\
--Estaba trabajando en la contrainteligencia en ese momento, y sugirió
que nuestros experimentos estaban teniendo más éxito que los alemanes
porque se llevaban a cabo en la proximidad de líneas ley.\\
--Lo siento, señor --Wilson se frotó la barbilla--. Estoy teniendo
problemas en creer todo esto.\\
--Muy de acuerdo, capitán. Me llevó bastante llegar a comprenderlo un
poco. La cosa es que Crowley tenía razón. Los resultados experimentales
mejoraron significativamente cuando la máquina fue colocada dentro de
las líneas ley. Los alemanes debieron de haber llegado a esa conclusión
hacia el final, pero por la localización que habían escogido para su
complejo de pruebas no tenían forma de comprobar esa hipótesis. Hay
algunas pruebas de una posible incursión hecha por un comando de las SS
en un círculo neolítico en Escocia con un prototipo de Campana mucho más
pequeño en algún momento de 1944, pero no estoy autorizado para ver esos
archivos.\\
--¿Así que nuestra máquina se estableció en Stonehenge?\\
El coronel negó con la cabeza.\\
--Demasiado obvio. Los Aliados sabían que los nazis estaban al tanto de
los experimentos, así que necesitaban encontrar algún lugar un poco más
fuera de las líneas de fuego. Ringstone fue escogido como el lugar de
pruebas británico. Nombre codificado ``Proyecto Big Ben''.\\
--¿Usando el círculo de piedras local? ¿Los Guardias del Rey?\\
--Exacto --el coronel se inclinó hacia adelante--. La máquina no era un
arma, capitán. Era un aparato de tele transporte. Y funcionó.\\
