\chapter*{CAPÍTULO DIECEISÉIS}
\addcontentsline{toc}{chapter}{CAPÍTULO DIECISÉIS}
El Doctor observó con tristeza la cara desfigurada que se hallaba bajo
la máscara. Virtualmente no quedaba nada del lado izquierdo de la cabeza
de Clearfield, solo hueso y una máscara fundida de una cicatriz. Él
contuvo la mirada del ojo que le quedaba a Clearfield.\\
--Lo siento.\\
--Lo siente\ldots{} --Clearfield soltó una risa sin humor y se volvió a
colocar la máscara plástica-- Setenta años de ver esta cara en el espejo
cada día, y dice que lo siente.\\
--Ya era parte de la historia. Lo que pasó esa noche\ldots{}\\
--¡Déjame explicarte exactamente qué pasó esa noche! --escupió
Clearfield-- Tenías razón: la combinación de veneno de Wyrrester y los
campos residuales mutagénicos generados por la Campana se combinaron de
una forma inesperada. ¡Me hizo invulnerable, fuerte, estimuló mi
capacidad de aprender, me hizo prácticamente inmortal! --respiró hondo--
¡Debería haber sido mi mayor momento, mi apoteosis! Si tan solo hubiera
podido alejarme un par de metros de donde me dejaste cuando las bombas
comenzaron a caer. Si me hubieras ayudado a salir de allí, entonces
habría sobrevivido ileso, ya que\ldots{} --cerró sus ojos, reviviendo el
momento-- Tomé refugio bajo el vehículo de control, por desgracia una
bomba aterrizó cerca, haciendo explotar el tanque de gasolina. Debería
haber muerto, pero el veneno de Wyrrester lo evitó. Cuando las llamas se
apagaron, salí gateando de los destrozos siendo un\ldots{} hombre
cambiado. No he envejecido ni un día desde entonces.\\
--¿Y esto? --el Doctor señaló las máquinas, el círculo.\\
--Este es el trabajo de una vida\ldots{}\\
--El error de una vida --el Doctor dio un paso adelante, pretendiendo
razonar con él--. Los Wyrresters\ldots{}\\
--Los Wyrresters son la mayor bendición que el hombre podría desear. Sus
voces han estado dentro de mi cabeza durante la mayor parte de mi vida
adulta --Clearfield sacó la pistola de dentro de su chaqueta--. Pretendo
ayudarles en su deseo de crear una colonia aquí en la Tierra y no voy a
permitir que te interpongas en mi camino.\\
Apuntó la pistola a la cabeza del Doctor.\\[2\baselineskip]Kevin
Alperton estaba en la cocina de la casa de Robin Sanford, limpiando el
cristal roto, y ayudando al anciano a clavar una lámina de madera en la
ventana, cuando se dio cuenta del movimiento en el campo de fuera.\\
Su primer pensamiento fue que todavía otro tipo de insecto iba a atacar,
pero cuando miró más de cerca se dio cuenta de que el movimiento venía
de cuatro soldados abriéndose camino a través del campo detrás de la
casa.\\
--¡Señor Sanford! --le llamó-- ¡Mire!\\
Robin Sanford se unió a él en la ventana.\\
--¡Ya era hora, maldita sea! --abrió la puerta trasera, tomando una gran
llave de un gancho y pasándosela a Kevin-- Ve y abre la puerta al final
del jardín. Se volverá un maldito jaleo si intentan pasar por encima.\\
Kevin tomó la llave nerviosamente.\\
--¿Fuera?\\
--Claro. Te cubriré desde la ventana --Robin tomó la escopeta--. ¡Bueno,
adelante!\\
Con el corazón acelerándose, Kevin dio un paso fuera hacia el bien
cuidado jardín. La puerta en el extremo alejado estaba probablemente a
tres metros de distancia, pero de repente parecía estar a un kilómetro.
Agarrándose a la llave con fuerza, comenzó a abrirse camino lentamente
por el jardín, asegurándose de que sus zapatos hacían el menor sonido
posible sobre las losas de piedra.\\
El jardín era una maraña de plantas en tiestos, intercalados con sillas
de plástico y comedores para pájaros. Kevin estaba simplemente
agradecido de que al señor Sanford le gustasen los patios en lugar de
los jardines muy decorados. Seguía esforzándose en olvidar la imagen del
zorro peleándose salvajemente en su propio jardín la noche previa. Al
menos, los insectos no podían hacer madrigueras a través de la piedra.\\
Estaba casi en la puerta cuando una alta silueta casi escondida por el
follaje le llamó la atención. Había algo familiar en ella\ldots{}\\
Con la curiosidad superando su miedo, apartó las hojas a un lado. Era
una piedra, de un metro veinte de alto, cubierta con musgo y líquenes.
Kevin podía ver los remolinos y los patrones excavados en la roca. Hubo
un silbido de detrás de él que hizo que Kevin pegara un bote. El señor
Sanford le miraba desde la ventana.\\
--¿A qué estás esperando? ¡Abre esa puerta!\\
Apresurándose, Kevin toqueteó un gran cerrojo, dolorosamente consciente
del ruido que estaba haciendo al hacerlo. El cerrojo se abrió con un
chasquido y Kevin abrió la puerta de madera. Se encontró a sí mismo
mirando el cañón de una pistola.\\
--Esperad un momento --dijo una voz ahogada--. Es solo un niño.\\
Cuatro soldados armados hasta los dientes se abrieron paso a través del
jardín, metódicamente comprobando cada rincón y trozo de sombra hasta
que tuvieron la certeza de que la zona estaba segura. Kevin podía ver
que dos de los soldados tenían sus rifles apuntando a la ventana de la
cocina donde el cañón de la escopeta del señor Sanford estaba claramente
visible.\\
--Baje el arma, por favor, señor --ladró uno de ellos.\\
Hubo un tintineo de cristal roto cuando el cañón del arma fue retirado
por la ventana rota, y una voz ahogada salió de dentro de la casa.\\
--Estoy de vuestro maldito lado, ya sabéis.\\
Los soldados acompañaron a Kevin hasta la puerta trasera y todos
entraron en la casa.\\
--Arnopp, Palmer, comprueben el frente. Hawkins, el primer piso --uno de
los soldados, obviamente el líder, se quitó el casco--. Soy el capitán
Wilson. ¿Hay alguien más en esta casa?\\
Robin Sanford negó con la cabeza.\\
--Solo el chico y yo.\\
--¿Y usted quién es?\\
--Sanford. Robin Sanford. Este es Kevin.\\
--Alperton --añadió Kevin--. Mi apellido es Alperton.\\
--¿Qué hay de antes? Los dos de la moto, ¿quiénes son?\\
Robin levantó una ceja.\\
--Lo habéis visto, ¿eh? Entonces supongo que sois vosotros a los que
tenemos que agradecer por disparar a esa araña.\\
Wilson asintió.\\
--Siempre estamos contentos de ayudar. Bien, esos otros dos\ldots{}\\
--Eran el oficial Bevan --comentó Kevin--. Y el Doctor.\\
--¿El doctor? ¿Quién es? ¿El médico local?\\
--La verdad es que no lo sé --dijo Robin bruscamente--. Es un científico
de algún tipo.\\
Wilson intercambió una mirada con uno de sus hombres.\\
--Señor Sanford, entiendo que usted estuvo estacionado aquí durante la
guerra\ldots{}\\
--Eso es, la Guardia Local. Me habría alistado, pero tengo problemas en
mi pecho, verá\ldots{}\\
--Señor, necesito que me cuente todo lo posible sobre el `Proyecto Big
Ben'.\\
Robin Sanford se enderezó.\\
--¿Qué sabéis de eso?\\
--Tenemos razones para pensar que alguien está recreando esos
experimentos.\\
--¡No! --Robin negó su cabeza con furia--¡No, eso es imposible!\\
--Señor, nuestros instrumentos están recogiendo una firma de energía
distintiva de algún lugar en las cercanías. Alguien está operando un
aparato Campana y necesitamos encontrarlo y destruirlo --Wilson señaló a
una de las sillas--. Siéntese, por favor.\\
Robin vaciló, pero la expresión de Wilson dejaba bastante claro que no
estaba de humor para juegos.\\
--Por favor.\\
Robin se sentó. Colocando su casco en la mesa de la cocina, el capitán
Wilson acercó una silla y se sentó mirándole.\\
--Bien, necesito que me cuente todo lo que sabe sobre esta máquina. Y
este misterioso Doctor.\\[2\baselineskip]Cuando Clearfield apuntó el
revólver ante la frente del Doctor, hubo un repentino grito sibilante y
el choque del metal cuando todas las jaulas de la pared más alejada del
edificio se abrieron. Una marea furiosa de insectos surgió a través del
suelo. Momentos más tarde, las luces se apagaron, dejando al edificio
entero en total oscuridad.\\
--¡Doctor! ¡Corre! --gritó Charlie Bevan.\\
El Doctor no necesitó más ánimo. Al lanzarse a sí mismo contra un lado
hubo una explosión ensordecedora cuando la pistola de Clearfield se
disparó. En el brillante destello del brillo del disparo, el Doctor pudo
ver a Clara de pie en shock en el centro del tecnocírculo cuando unos
enormes insectos pasaron a su alrededor.\\
La pistola se disparó de nuevo, y el Doctor usó la momentánea
iluminación para orientarse. Charlie y Angela se apretaron contra la
pared, esforzándose por abrir una de las puertas de emergencia. Clara
seguía sin haberse movido.\\
Cuando uno de los insectos híbridos salió disparado hacia él, el Doctor
salió corriendo, pegándole una patada para apartarlo y dirigiéndolo
hasta Clearfield, haciéndole perder el equilibrio. Cuando el profesor
chocó contra el suelo, la pistola se disparó una tercera vez, la bala
rebotando salvajemente por las pasarelas metálicas en el tejado.
Prácticamente a gatas, el Doctor se tambaleó por el suelo hasta la
oscura y abultada silueta de la Campana. Podía sentir el vello de sus
brazos estar de punta por la electricidad estática que seguía enganchada
en su superficie. Trabajando rápidamente, y en una casi total oscuridad,
localizó un panel de acceso colocado en la base de una máquina y lo
abrió.\\
Un pálido brillo violeta cruzó sus manos. El espacio dentro de la base
de la Campana estaba ocupado con tubos de cristal, cada uno de ellos
conteniendo un brillante líquido morado. El Doctor vaciló. Sin tiempo
para hacer un decente estudio de los funcionamientos de la Campana, no
tenía ni idea de qué componentes la desconectarían de forma efectiva. Lo
que era más preocupante, no tenía ni idea de qué componentes podían
contener una carga. Con el tiempo rápidamente acabándosele, usó la
técnica que le había servido bien en varias de sus previas
encarnaciones.\\
--Pito, pito, gorgorito\ldots{} --agarró uno de los tubos de cristal y
empujó fuerte. Unos cables y unas tuberías salieron de sus encajes, y
hubo una ducha de chispas cuando se salió de su posición-- ¿dónde vas tú
tan bonito?\\
Metiéndose el tubo en el bolsillo de su chaqueta, el Doctor se puso en
pie. Podía seguir distinguiendo la vaga silueta de Clara en el centro
del círculo. Se apresuró a ponerse a su lado.\\
--¡Vamos!\\
Agarrándola por la mano, corrió por el oscurecido almacén hasta donde
Charlie y Angela estaban esperando. Mientras corrían, las puertas dobles
de repente se abrieron y un haz de luz brilló por el suelo. Unos
insectos se escabullían corriendo en busca de cobertura cuando la luz
inundó la sala.\\
A duras penas frenando, el Doctor emergió en la luz solar, todavía
arrastrando a medias a Clara consigo. Cuando Charlie y Angela salieron
tras ellos, el Doctor soltó la mano de Clara, cerrando las puertas tras
él y apretando la punta del destornillador sónico contra el cerrojo.
Hubo un silbido de vapor, y una gota de metal fundido goteó del cerrojo
cuando éste se fundió en una masa sólida. Angela le lanzó una mirada
severa.\\
--Bloquear las puertas de emergencia es una infracción del Acta Laboral
de Salud y Seguridad, ¿sabes?\\
--Lo mismo es dejar las mascotas sueltas en el lugar de trabajo --le
sonrió el Doctor--. ¡Muchas gracias por eso, por cierto!\\
--¿Qué eran esas cosas? --Charlie estaba esforzándose por encontrar una
porción de su pañuelo de tela que no estuviera cubierta de barro, grasa
o baba.\\
--Recipientes, receptáculos, cuerpos para los Wyrresters --el Doctor
observó su reloj--. Probablemente les hayamos frenado, pero no lo
suficiente, no del todo\ldots{} --les alejó del edificio-- Tenemos que
volver a la TARDIS. Vosotros tres podéis usar la moto para volver a la
casa de Robin Sanford.\\
--¿Qué vas a hacer? --preguntó Charlie.\\
--Tengo que destruir la máquina.\\
De repente se dio cuenta de que Clara no estaba con ellos, el Doctor se
detuvo y miró a su alrededor, buscándola. Seguía de pie fuera del
edificio, con las manos apretadas a cada lado de su cara. El Doctor se
apresuró a volver a su lado.\\
--¿Estás bien?\\
Clara le lanzó una sonrisa débil.\\
--Solo un poco desorientada, eso es todo.\\
El Doctor le observó los ojos, preocupado.\\
--No estoy sorprendido, la energía en esa habitación sería suficiente
para inquietar la constitución más robusta --le apretó el hombro de
forma reconfortante--. Estás bien.\\
Los dos se apresuraron a rodear el edificio hasta dónde Charlie y Angela
estaban esperándoles a la sombra de la TARDIS. El Doctor desbloqueó la
puerta y entre ellos se las arreglaron para arrastrarla fuera a través
de las puertas dobles. Al hacerlo, Angela miró a través de las puertas
abiertas, y luego dio un paso atrás de forma abrupta, lanzando una
mirada al Doctor que era medio asombro y medio puro terror.\\
--Estoy seguro de que el oficial Bevan te dirá todo lo que necesitas
saber --dijo el Doctor. Se giró a Clara--. Será mejor que conduzcas tú.
Aquí, Charlie ya ha causado suficiente daño de por sí.\\
--¡Ey! --Charlie le miró de forma indignada.\\
Clara retrocedió, negando con la cabeza.\\
--No lo creo\ldots{}\\
--De acuerdo\ldots{} --los ojos del Doctor se estrecharon-- Parece que
vas a conducir tú después de todo, oficial.\\
Charlie se subió a la motocicleta, Angela se colgó del asiento detrás de
él.\\
El Doctor ayudó a Clara a subirse al sidecar.\\
--¿Estás segura de que estás bien?\\
--De verdad, solo tengo dolor de cabeza, eso es todo.\\
El Doctor dio un paso atrás cuando Charlie encendió la gran Norton de
una patada.\\
--Nos veremos en la granja más tarde. Cierra las puertas y quédate
dentro.\\
Charlie asintió y la Norton salió rugiendo por el aparcamiento. El
Doctor observó cómo se desvaneció por la carretera, luego sacó con
cuidado el frasco de cristal de su chaqueta. Lo sujetó a contraluz,
observando el espeso líquido morado removerse y arremolinarse dentro del
cristal.\\
--Solo un dolor de cabeza\ldots{} --murmuró, luego se giró y entró en la
TARDIS.\\[2\baselineskip]Clara abrió sus ojos. Todo a su alrededor
estaba oscuro. Intentó abrir su boca, pero su lengua estaba espesa y
pesada. Intentó moverse, pero sus miembros parecían vagos, sin
respuesta.\\
Abruptamente una luz destellante se encendió, cegándola, y fue
consciente de dos formas moviéndose cautelosamente hacia ella a través
del brillo, una de ellas más grande que la otra.\\
--Remarcable. No esperaba que su conciencia volviera tan pronto.\\
--Hemos menospreciado las habilidades de estos primitivos. Quizá
deberíamos reconsiderar nuestro plan.\\
Con un escalofrío de reconocimiento, Clara se dio cuenta de que las
flojas voces burbujeantes eran las mismas que había oído hablando antes.
Eran las voces de los alienígenas.\\
Al intentar ver a través del brillante destello, una de las formas se
acercó, y ella se estremeció de terror cuando la luz iluminó cada
diminuto detalle de la criatura ante ella.\\
La bestia era como un enorme escorpión, de unos cuatro metros de largo.
Su caparazón negro y brillante estaba cubierto de pelos afilados, y se
movía con aprensión sobre seis patas larguiruchas. Dos grandes garras,
con sus superficies cubiertas con arcanos símbolos de remolino, se
abrían y cerraban lentamente, y una cola curva, moteada con una púa de
aspecto malvado, que sacudía agitadamente.\\
La criatura se acercó y Clara comenzó a entrar en pánico cuando se
cernió sobre ella. Intentó retroceder, pero una fuerza invisible la
mantuvo firmemente en el lugar. Unos ojos fieros y negros brillaban en
los pliegues de la piel que se formaba en la cara del monstruo y unas
gruesas mandíbulas carnosas se movieron con humedad al examinarla.\\
Clara levantó sus manos para protegerse, pero al hacerlo también se dio
cuenta de que no eran manos lo que estaba levantando, sino unas garras
negras de quitina, con los bordes afilados como cuchillas, su superficie
cubierta en espirales y patrones.\\
Se congeló, desesperada de rechazar la evidencia de sus propios ojos.\\
Sus propios ojos\ldots{}\\
A medida que el horror de lo que había pasado le caló hondo intentó
gritar, pero el único sonido que podía hacer era una tos malvada y
burbujeante.\\
La transferencia había tenido éxito.\\
Su mente estaba en el cuerpo de un Wyrrester.\\
