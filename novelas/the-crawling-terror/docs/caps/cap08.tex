\chapter*{CAPÍTULO OCHO}
\addcontentsline{toc}{chapter}{CAPÍTULO OCHO}

{--O nos sofocamos aquí, o aterrizamos en un planeta donde el aire es
tóxico --dijo Marco Spritt--. ¿Se supone que eso es una elección?}

{El Alejandría estaba en ruta al planeta perdido, viajando a toda
 velocidad. Laker había estimado que les llevaría no más de una hora el
llegar allí.}

{--Tenemos trajes espaciales --dijo Balfour. Su voz estaba temblorosa,
 como si estuviera esforzándose para aceptar la enormidad de la
 situación--. Unos trajes de última tecnología de supervivencia
 ambiental. Los ordené especialmente. Los mejores del mercado. Nos
mantendrán con vida.}

{--Sí, ¿pero durante cuánto? --preguntó Tanya Flexx.}

{--Cada traje tiene el bastante aire y filtros como para durar al menos
 setenta y dos horas --dijo Laker--. La comida y el agua son el
problema.}

{--Así que tenemos un máximo de tres días viviendo en un traje espacial
--dijo Tanya, alzando sus manos en el aire--. Fantástico.}

{--No puedo pensar en nada más --le espetó Balfour.}

{Clara lo sintió por él. Ni siquiera Trugg podría ayudarle a salir de
 esta, y ninguna cantidad de dinero podría salvarles. Pero había algo.
 Ella y el Doctor tenían un as en la manga. Ella se giró hacia él y le
lanzó una mirada significativa. Él levantó las cejas y dijo:}

{--¿Qué?}

{Clara le dijo sin hablar:}

{--TARDIS.}

{El Doctor no parecía feliz. Raramente parecía feliz, pero ahora tenía
 una particularmente mirada extraña en sus ojos bajo su ceño furioso. El
 Doctor no le abría las puertas de la TARDIS a nadie, pero Clara sabía
 que, en una cuestión de vida y muerte, no vacilaría. La TARDIS era el
 último santuario, después de todo. Había suficiente sitio y energía como
 para desparecer en un parpadeo. Les podría llevar a todos al planeta
 oscuro, o de vuelta a la estación espacial, o de vuelta a la misma
 Tierra. Durante un mareado momento, Clara incluso recordó el montón de
 correcciones que le esperaba en casa. Pero si el Doctor había parecido
 reacio a mencionar la TARDIS, Mitch Keller no tenía tales aprensiones.
Ya había llegado a la misma conclusión que Clara.}

{--¿Qué hay de tu nave, Doctor? --preguntó él-- La cabina azul.}

{--¿Estás loco? --dijo Hobbo--. No es para nada lo suficiente grande.
 Somos once, recuerda. Esa cosa no podría contener más de cuatro o cinco
como máximo.}

{--Y eso si no incluimos a Trugg --dijo Balfour. El robot hizo un
 chirrido mudo pero no dijo nada. En ese momento, todos los ojos estaban
puestos en el Doctor.}

{--Bueno, estoy seguro de que todos nos podríamos apretar dentro --dijo
con un desdeñoso ademán de sus dedos--. Pero ese no es el tema.}

{Marco explotó, furiosamente.}

{--¿Ese no es el tema? ¡Si tú tienes un salvavidas a bordo, entonces
llévanos con él! No me importa lo mucho que tengamos que estrujarnos.}

{--Estoy de acuerdo con Marco --dijo Cranmer--. Salgamos ahora si
podemos.}

{--¿Ves? --dijo Marco, triunfantemente.}

{Hubo un breve clamor mientras todos intentaban hablar a la vez, pero la
voz del Doctor se levantó por encima del resto.}

{--Callaos, todos vosotros. Especialmente tú, Marco.}

{--Pero, en serio\ldots{} --comenzó Marco, pero el Doctor le lanzó una
 mirada de odio y levantó un rígido dedo en busca de silencio. Marco se
quedó callado.}

{--Sí, tengo una cápsula espaciotemporal a bordo del Alejandría --dijo el
 Doctor--. Se llama la TARDIS y es milagrosa, pero hay una razón por la
que no podemos usarla ahora. Dígaselo, señor Cranmer.}

{Cranmer parecía confundido mientras todo se giraban hacia él.}

{--Vamos, hombre, has estado usando tu ordenador para ejecutar escáneres
 en el planeta desde que lo hemos encontrado. Densidad, atmosfera,
 radiación, todo eso --el Doctor alzó sus cejas--. Diles a todos lo que
has encontrado.}

{--Bueno --dijo Cranmer, mirando hacia abajo, hacia las muestras de su
 ordenador--. Es bastante remarcable, de verdad. Confieso que no lo
 entiendo del todo, pero parece haber unos fenómenos espaciotemporales
extraordinarios en o alrededor del planeta.}

{--¿Qué se supone que debe significar? --preguntó Marco.}

{--Hay una capa, o una concha, de chronons y antichronons rodeando el
 planeta en un estado altamente agitado, lanzando una tormenta a los
taquiones sueltos y radiación Hawking.}

{--No lo entiendo --dijo Tibby--. ¿No te puedes explicar un poco mejor?}

{--Las lecturas son poco comunes y difíciles de interpretar --dijo
 Cranmer--, pero parece haber esferas superpuestas de tiempo discreto
rodeando el planeta.}

{--Oh, eso lo hace todo más claro --dijo Marco con un fuerte sarcasmo.}

{--Quiere decir que no es solo un viejo y tóxico planeta perdido en el
 profundo espacio --dijo el Doctor--. No podría materializar mi TARDIS
 aunque quisiera. Las interferencias temporales actuarían como una
 barrera impenetrable para una máquina del tiempo --se detuvo,
 asegurándose de que tenía la completa atención de todo el mundo--. Solo
 hay una forma de llegar a ese planeta\ldots{} y es el con el
Alejandría.}

{--Pero, ¿por qué tenemos que ir ahí? --preguntó Cranmer.}

{--Es para eso para lo que vinimos, ¿no? --dijo Tibby, enfadada.}

{--¡Es una muerte segura! --dijo Cranmer-- Setenta y dos horas de aire,
recordad.}

{--Se puede hacer mucho en setenta y dos horas --dijo Tibby.}

{Los dedos del Doctor chasquearon fuertemente y señaló a Tibby.}

{--Correcto.}

{No parecía que ésta fuera la cosa más reafirmante que hubieran oído en
 todo el día, pero el Doctor no pareció darse cuenta porque en ese
 momento el Alejandría se zarandeó ligeramente y todo el mundo tuvo que
agarrarse a algo.}

{--Hemos sido atrapados en algún tipo de campo gravitacional --dijo
 Laker. Sus manos estaban ocupadas en los controles de vuelo, ajustándose
a las repentinas turbulencias.}

{--Es la capa de chronons cargados --dijo Cranmer, toqueteando su
 ordenador--. Está creando un tipo de vórtice gravitacional, empujándonos
 a\ldots{}}

{El planeta suspendido en el holovisor parecía de repente muy cercano.
 Las nubes se arremolinaban por la superficie, rastreando los caminos de
tormentas continentales parpadeando con cargas eléctricas.}

{--Lo que ahora estamos viendo son imágenes a tiempo real a escala
 completa --dijo Laker, haciendo otro ajuste en los controles--. Sea lo
que sea que esté pasando en ese planeta, lo estamos viendo ahora.}

{El planeta estaba aumentando cada segundo que el Alejandría se ponía en
 órbita, zarandeándose de un lado a otro, latiendo como venas arteriales,
 parpadeando por la superficie y llevando grandes nubes de oscuridad en
la atmosfera.}

{--Será mejor que os pongáis los trajes espaciales --anunció Laker--.
Esto se puede poner serio.}

{Durante un momento nadie se movió. Era casi como si nadie pudiera
 creerse que realmente habían llegado a ello. De repente, la idea de
 llevar un traje espacial no era muy apetecible, pero más
 estremecimientos a través del Alejandría provocaron que se decidieran
muy rápidamente.}

{--Todos los trajes están aquí --dijo Balfour, tocando un control de la
 pared. Una serie de paneles se deslizó sin hacer ruido a un lado para
 revelar paquetes de plástico en unidades de receso. Trugg se movió hacia
 adelante y comenzó a repartir los paquetes a todo el mundo en la
cubierta de vuelo.}

{--Los trajes espaciales están hechos de la última malla ajustable de
 acero con yelmos de supercristal de moléculas encadenadas y unas
varillas totalmente comprimidas de oxígeno --dijo Trugg.}

{--Ahora no necesitamos que nos lo vendas --dijo Balfour--. Simplemente,
dánoslo.}

{Clara tomó uno de los paquetes de Trugg y lo encontró mucho más ligero
 de lo que esperaba. Materiales del futuro, supuso ella, presumiblemente
 más duraderos pero sin mucho peso. Le quitó el sello al paquete y sacó
 un mono azul metálico parecido a un traje de baño. El material era suave
 y extremadamente flexible, nada como el pesado traje que una vez había
llevado en la Luna.}

{--Póntelo por encima de tu ropa --le aconsejó el Doctor, quien ya estaba
 subiéndose las piernas de su propio traje--. El material se expandirá
 para cualquier talla que sea necesaria. Debería ser capaz de adaptarse,
incluso para ti.}

{--Gracias --dijo Clara, secamente. Ella se subió el traje y descubrió
 que se había ajustado automáticamente y se había sellado sin que ella
tuviera que hacer nada de nada.}

{El Doctor parecía incluso más delgado de lo normal en su traje espacial.
 Ya se estaba poniendo las delgadas botas blancas y un par de guantes a
 juego. Clara hizo lo mismo, y encontró que aquellos también se habían
sellado con el resto del traje. Era sorprendentemente cómodo y cálido.}

{--Cascos --dijo Balfour mientras Trugg les repartía un número de globos
 transparentes acompañado con collares flexibles. A Clara le parecieron
más bien frágiles y vaciló al ir a ponérselo.}

{--No te preocupes --dijo el Doctor, rascando con su nudillo contra su
 propio yelmo espacial--. Es irrompible, un tipo de polímero de cadena de
 moléculas con un filtro de luz polarizado. Podrías lanzar un ladrillo
contra él y ni siquiera se rayará.}

{--La etiqueta del nombre se activará según vuestras señales vitales
 --dijo Balfour--. Doctor, señorita Oswald, puede que tengáis que poner
los vuestros manualmente.}

{Había una muestra digital acoplada al cristal. Después de un momento o
dos, Clara averiguó cómo usarlo y tecleó CLARA.}

{--¿Dónde están los tanques de oxígeno? ¿Qué respiramos?}

{El Doctor señaló un pequeño aparato pegado a la parte trasera de su
yelmo.}

{--Una unidad compacta de soporte vital con cañas sólidas de oxígeno. Es
 aire suficiente como para tres días, dos si te los pasas corriendo
colina arriba.}

{--Esperemos que no haya colinas, entonces --dijo Clara, poniéndose el
 yelmo. Se encajó en el sitio el collar de su traje y oyó el silbido
 neumático de su sello. Durante un momento parecía amortiguado hasta que
 el sistema de aire se activó y algún tipo de transceptor permitió sonido
dentro y fuera del yelmo.}

{--Ahora tenéis un aumento de los sonidos atmosféricos --les aconsejó
 Balfour--. En condiciones de vacío necesitaréis conectar las
 comunicaciones por radio. Cada traje tiene su propio transpondedor. Los
 controles están en vuestras mangas, podéis ajustar los ajustes
 ambientales si queréis, a pesar de que todos los trajes son
termostáticos.}

{--¿Dónde nos sentamos? --preguntó Tibby. Estaba ayudando a Luis Cranmer
con su traje.}

{--Hay un conjunto de asientos de emergencia situados tras la silla del
 capitán --dijo Balfour--. Es una característica estándar de seguridad en
 todas las naves tipo Heracles, pero estos han sido actualizados
 con\ldots{}}

{--Sí, sí --dijo Marco--. ¡Vamos con ello y ya!}

{Trugg operó un control y dos hileras de sillas se levantaron suavemente
 de la cubierta tras la silla del capitán y el sillón de astrogación. Los
 asientos estaban colocados en cierto ángulo y tenían unas barras de
 seguridad acomodadas. Le recordaron incómodamente a Clara de los
asientos de las montañas rusas de los parques temáticos.}

{La nave estaba temblando bastante en ese momento y se estaba poniendo
 difícil mantenerse en pie. Clara se sorprendió al ver que la vista
 holográfica del planeta se había aplanado en nada más que un vasto y
 curvado horizonte y un cielo negro. Ya habían llegado a la atmosfera
superior y el Alejandría comenzaba a percibirlo.}

{--¿Todo preparado? --les gritó Laker-- Será mejor que os sentéis y os
atéis.}

{--Ha pasado un tiempo desde la última vez que me tuve que atar para un
aterrizaje --dijo Mitch.}

{--Seguro que teníais que hacerlo siempre en los viejos tiempos --dijo
Hobbo--. Tú y Neil Armstrong.}

{--Qué mona --respondió Mitch--. Realmente mona.}

{Marco ya había tomado su lugar en la hilera detrás de Laker y Jem y, ya
 que no había ningún otro lugar al que ir, Clara y el Doctor se sentaron
 con él. No parecía feliz, su cara estaba sudando profusamente dentro de
 su casco. Clara intentó pensar en algo verdaderamente tranquilizante que
 decir pero no pudo. Escogió dedicarle una breve sonrisa, la cual
sospechaba que parecía más una mueca de preocupación.}

{--Espero que esto termine rápido --dijo él, cerrando sus ojos.}

{--Claro --dijo Laker, levantando su voz por encima del ruido de las
 turbulencias--. Estoy cambiándome a la radio, Canal 1 en vuestros
trajes.}

{Todos inspeccionaron los controles en sus antebrazos y seleccionaron el
 canal adecuado. Al instante, la voz de Laker sonó dentro de sus cascos,
tan clara como si estuviera de pie a su lado.}

{--Nos estoy llevando lo mejor que puedo, pero no va a ser bonito. El
 piloto automático no puede hacer mucho más, así que el resto depende de
 mí. Si nos chocamos, será culpa del ordenador. Si lo conseguimos, será
mía.}

{--Lo vas a hacer genial --dijo Jem--. A por ello.}

{El Alejandría estaba atravesando la atmosfera planetaria, botando y
 rebotando cada vez que la parte inferior de la nave tocaba el borde. Un
 grito comenzó a aumentar de la nave a medida que ésta se introducía más
 en el aire. Aún quedaba bastante hasta llegar a la capa de nubes, pero
ya parecían estar yendo en un ángulo muy pronunciado.}

{Clara alargó la mano y le agarró la mano al Doctor. Fue puramente
instintivo.}

{--No había hecho esto antes --dijo ella--. Ya lo puedo tachar de mi
lista.}

{--No hay nada de lo que preocuparse --dijo el Doctor.}

{--¿Estás seguro? --preguntó Luis Cranmer, nervioso.}

{Clara de repente recordó que estaban en canal abierto, y todo el mundo
podía oír su conversación.}

{--De hecho, hay mucho por lo que preocuparse --dijo el Doctor--. La
 gravedad por un lado, por supuesto. Ese es el problema principal. Si
 tienes eso mal, ¡PLAF!\@ Se acabó el viaje.}

{Cranmer tragó sonoramente y volvió la cara al holograma del planeta. Las
 nubes estaban corriendo para chocarse con ellos, con finas hileras de
vapor sacudiendo la nave mientras ésta se hundía en la superficie.}

{--La gravedad nos arrastrará hasta la superficie, claro, pero con suerte
 hay una atmosfera, la cual provocará fricción si mantenemos el ángulo de
reentrada correctamente. Eso ayudará a ralentizarnos.}

{--¿No arderemos ni nada? --preguntó Cranmer.}

{--Bueno, ese es el problema con la fricción. La temperatura se disparará
 y si se pone demasiado alta, la nave entrará en ignición y todos seremos
incinerados en una gigantesca bola de fuego.}

{Cranmer gruñó y cerró sus ojos.}

{--Creo que voy a vomitar.}

{--No lo hagas --le advirtió el Doctor--. Traje espacial. No es una buena
idea.}

{--¡Apenas me importa si voy a arder vivo!}

{--Ah, pero la borrosa forma del Alejandría creará una ablativa onda
 expansiva mientras pase por la atmosfera --explicó el Doctor--. Esa onda
 expansiva mantendrá lo peor del calor lejos. Además de que el casco está
 hecho de un material especial que absorberá todo el calor. Así que no te
preocupes por eso.}

{--¿Quieres decir que vamos a estar bien?}

{--Sí --dijo el Doctor--. Vamos a estar bien.}

{Clara sintió algo moverse bajo su mano y bajó la mirada. El Doctor había
cruzado los dedos.}
