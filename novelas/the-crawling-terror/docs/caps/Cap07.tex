\chapter*{CAPÍTULO SIETE}
\addcontentsline{toc}{chapter}{CAPÍTULO SIETE}
Clara siguió a Angela de cerca mientras ella se abría camino a través
del brillante amarillo de uno de los muchos campos de semillas de colza
que acolchaba el campo, intentando poner tanta distancia entre ellos y
el pueblo como era posible.\\
Cuando el sol se alzó más alto en el cielo matutino, el aire comenzó a
volverse vivo con los insectos, y cada zumbido, aleteo o susurro hacía
que Clara mirara por encima de su hombro con miedo de lo que pudiera
haberlo provocado.\\
Por mucho que lo intentara, no podía dejar de pensar sobre el póster de
insectos británicos que decoraba la pared del laboratorio de biología en
el colegio Coal Hill, en particular en la sección de insectos con
aguijones. La araña, el escarabajo y los mosquitos ya eran bastante
malos, pero, ¿qué pasaba si había avispas gigantes o abejas por algún
lugar? Recordó el dolor de ser picado por una avispa cuando era una niña
pequeña, y solo pensar en el daño que una picadura gigante podía
hacerle, le hizo tener náuseas.\\
A medida que se abrió camino entre las hileras de plantas ligeramente
ondeantes, Clara de repente vio una serie de edificios bajos y
prefabricados destacando por encima de los setos que se hallaban ante
ellos, y se dio cuenta de que Angela había estado dirigiéndose hacia el
parque científico del que todo el mundo no dejaba de hablar.\\
Estaban acercándose al borde del campo cuando Angela se detuvo,
agachándose por entre las flores amarillas y se asomó a través de una
puerta de madera desvencijada en los terrenos del complejo industrial
ante ella. Clara se agachó a su lado.\\
--Debería haber supuesto que te estabas dirigiendo a algún lugar
específico.\\
Angela se encogió de hombros.\\
--No íbamos a durar mucho si nos quedábamos en el pueblo, pero esas
cosas puede que aún no hayan llegado ahí, y tiene que haber habitaciones
seguras donde podamos escondernos, y un teléfono que podamos usar.\\
--A menos que `esas cosas' vengan de aquí en primer lugar --le recordó
Clara.\\
Angela frunció el ceño.\\
--¿Los científicos locos criando insectos gigantes en un laboratorio
secreto en el desolado campo británico? ¿En serio que te lo crees?\\
--Te sorprendería la de cosas raras que he comenzado a creer
recientemente.\\
Angela la observó con precaución.\\
Clara negó con la cabeza.\\
--Ignórame. Solo estoy diciendo chorradas. Tienes razón, deberíamos
encontrar algún lugar seguro. ¿Ves a alguien?\\
Angela echó otro vistazo por la carretera.\\
--Me parece muerto.\\
Clara respiró hondo.\\
--Bueno, vamos pues.\\
Acercándose a la puerta, ambas corrieron por la carretera hacia el
complejo industrial. Cuando cruzaron el aparcamiento desierto, Clara se
dio cuenta de repente de lo expuestas y vulnerables que estaban ahora
que habían dejado la cobertura del campo. Se apresuraron a llegar al
edificio más cercano y se asomaron a través de la puerta de cristal al
vestíbulo de más allá.\\
--¿Puedes ver a alguien? --preguntó Angela, empujando sin resultado la
puerta cerrada.\\
--No. Está vacío. Tendremos que intentarlo con la otra.\\
Acercándose a las sombras junto a la pared, las dos mujeres se abrieron
camino hasta el siguiente edificio, pero aquel también estaba cerrado y
vacío.\\
--¡Esto es estúpido! --gritó Angela apretando su puño contra la puerta
de cristal por frustración-- Hay coches aparcados aquí así que, ¿dónde
está todo el mundo?\\
Clara miró por el parque científico desierto con pánico creciente.
Cuanto más se quedasen ellas dos en abierto, más oportunidades tenían de
ser vistas. Tenían que entrar rápidamente.\\
Un repentino borrón de movimiento justo en el rabillo de su ojo hizo que
Clara pegara un bote. Había algo moviéndose en el edificio del otro
lado. Giró, apretando su espalda contra la piel de aluminio del
edificio. Los ojos de Angela se abrieron por horror.\\
--¿Qué es? --graznó ella, poco dispuesta a girarse y verlo por sí
misma.\\
Cuando Clara se armó de valor para ahuyentar lo que quiera que les
persiguiera, de repente se dio cuenta de qué era lo que había visto.\\
--¡Todo va bien! --dijo ella, jadeando de alivio-- Es solo una cámara de
seguridad.\\
Angela se puso una mano sobre su boca, conteniendo una risa casi
histérica.\\
La cámara era una de las varias montadas en las esquinas de los
edificios. Mientras observaban, se giró para apuntar directamente hacia
ellas. Al hacerlo, Clara entendió que no era un movimiento automatizado.
Estaba siendo controlada. Alguien estaba observándoles. Dio un paso
fuera de la sombra del edificio, mirando directamente hacia la cámara,
zarandeando los brazos frenéticamente.\\
--¡Ey! ¿Podéis oírnos? ¡Estamos atrapadas aquí! ¡Tenéis que dejarnos
entrar!\\
Durante lo que pareció una vida, la cámara les miró impasiblemente,
entonces un conjunto de puertas dobles en un edificio opuesto se abrió
con un golpe y varias figuras emergieron.\\
--Gracias a Dios --respiró Angela, comenzando a girarse hacia ellos.\\
--No, espera --Clara le asió del brazo--. Algo no va bien.\\
Las figuras dirigiéndose hacia ellos estaban vestidas de blanco, con
unos monos y unas batas de laboratorio. Se movían con el mismo paso
monótono y desordenado que la gente en el pueblo.\\
--Oh, no --Angela negó con la cabeza con desesperación--. Aquí también
no\ldots{}\\
Clara la asió por la mano. Angela comenzaba a creer que no había
esperanza, y si empezaba a pensar así ya estaban perdidas.\\
--Angela --la voz de Clara era severa--. Tenemos que irnos. ¡Ahora!\\
Con las figuras zombis persiguiéndolas, las dos mujeres comenzaron a
correr.\\[2\baselineskip]--¿Kevin? --Charlie Bevan observó al chico con
asombro-- ¿Kevin Alperton? ¿Qué demonios estás haciendo ahí abajo?\\
El chico salió de debajo de la mesa de comunión.\\
--Escondiéndome de los insectos, igual que tú.\\
El Doctor se agachó para poder estar cara a cara con él.\\
--Pero no es solo de los insectos de lo que te escondes, ¿no?\\
Kevin negó con la cabeza.\\
--De todos. Bueno\ldots{} --miró nerviosamente al Doctor y a Charlie--
De todos menos de vosotros dos, espero.\\
--Bueno, no hay nada de lo que preocuparse --Charlie intentó darle al
chico una sonrisa reconfortante, y le puso una mano sobre el hombro--.
¡Puaj! --apartó la mano con disgusto. Estaba cubierto con una gruesa
baba.\\
Kevin hizo una mueca y le dio al policía una sonrisa de disculpa.\\
--Lo siento por eso. Los insectos pueden detectarse por su olor, así que
me estaba disfrazando --le dedicó un encogimiento de hombros tímido al
Doctor--. Lo vi en una película de miedo llamada Mimic. Pensé que
merecía la pena intentarlo.\\
--Protección de feromonas --el Doctor le hizo un asentimiento de
aprobación--. Muy emprendedor. Y, ¿de dónde, si puedo preguntar, has
obtenido secreciones para intentar esto?\\
Kevin abrió su mochila, sacó un magullado tupper y se lo pasó al Doctor,
subiendo la tapa de plástico.\\
--Está un poco apalizado.\\
El Doctor alargó la mano hacia la caja y extrajo amablemente los restos
de uno de los mosquitos. Charlie Bevan hizo un pequeño chillido de
alarma y retrocedió unos pasos.\\
--Aterrizó en mi espalda, pero me caí de espaldas y lo aplasté antes de
que pudiera picarme --explicó Kevin--. Creo que son los mosquitos los
que están convirtiendo a todos en zombis.\\
Sacando su destornillador sónico de su bolsillo, el Doctor dejó el
insecto en el trapo de terciopelo de la mesa de la comunión y comenzó a
examinarlo:\\
--Creo que tienes razón --dijo, levantando la probóscide afilada como
una aguja del mosquito y olisqueó el claro líquido que perlaba su
punta.\\
--¡Con cuidado! --silbó Charlie-- Si eso es algún tipo de
veneno\ldots{}\\
--Escopolamina.\\
--¿Perdón?\\
--Es escopolamina. Es una droga que bloquea los neurotransmisores que
llevan la información a la parte del cerebro que almacena la memoria a
corto plazo. También hace que la gente sea más abierta a la sugestión.
Josef Mengele la usaba en interrogaciones como un tipo de suero de la
verdad. La CIA administró dosis de escopolamina durante sus
controvertidos experimentos de rediseño del comportamiento en la década
de 1960.\\
Charlie observó al insecto con incredulidad.\\
--Pero eso es\ldots{}\\
--Es todo tipo de cosas --espetó el Doctor, con furia--. Es
irresponsable. Inmoral. Peligroso. Criminal --se giró para mirar a
Charlie--. Y es deliberado. Alguien está usando estos insectos como
herramientas.\\
Cualquier otra discusión fue abruptamente interrumpida por el traqueteo
de las losas del techo de algún lugar por encima de ellos. La araña
volvía a su morada.\\
El Doctor se giró para mirar a Kevin.\\
--Este túnel. ¡Muéstranoslo! ¡Rápido!\\
Sin estar dispuesto a perder las formas con las que ocultarse, Kevin
volvió a meter el mosquito en el tupper, lo cerró con la tapa y se lo
metió en su mochila.\\
--Por aquí.\\
Les llevó hasta una puerta de madera en el extremo alejado del
presbiterio. En el otro lado había una pequeña sacristía con otra puerta
que llevaba por el edificio hasta la parte trasera de la iglesia. Kevin
se apresuró a travesarla. Las habitaciones de detrás estaban oscuras;
las paredes estaban cubiertas con fotografías de niños y noticias sobre
el condado. Kevin se escurrió dentro de una pesada puerta de madera por
detrás de la que se hallaba un conjunto de escaleras de apariencia
ligeramente débil.\\
--Está aquí abajo.\\
La puerta se abrió con un crujido casi teatral y Kevin rebuscó en la
oscuridad el interruptor de la luz. Una desnuda bombilla de ahorro
eléctrico se encendió, iluminando flojamente una estrecha escalerilla y
unas húmedas paredes de piedra. El Doctor y Charlie siguieron a Kevin
por la oscuridad.\\
La bodega era un revoltijo de cajas, muebles viejos, equipos de acampada
y unas banderas cuidadosamente dobladas de scouts. Kevin se abrió camino
hasta la pared más alejada y apartó un sofá raído para revelar una
pequeña apertura oscura.\\
--Descubrimos esto el pasado octubre --explicó Kevin--. Hubo una fiesta
de Halloween de scouts aquí abajo y Baz Jones y yo queríamos asustar a
Akela. Debe de haberse entablado hace años.\\
El Doctor se asomó por la apertura oscura.\\
--¿Cómo de lejos va?\\
--Va hasta la vieja bomba de agua al lado de la Oficina de Correos.\\
Charlie Bevan hizo una mueca.\\
--Eso no es muy lejos.\\
--Bueno, es mejor que nada --el Doctor encendió de un golpe su
destornillador sónico, iluminando el túnel con la luz verde esmeralda--.
Tú sabes dónde vamos, Kevin, será mejor que guíes el camino.\\
Cuando Kevin se escurrió dentro del túnel, el Doctor se giró para mirar
a Charlie.\\
--Creo que es hora de que hagamos que el señor Sanford nos cuente toda
la historia sobre estos bichos gigantescos del espacio
exterior.\\[2\baselineskip]Clara y Angela se agazaparon juntas a la
sombra de un enorme contenedor de reciclaje, observando cómo las
siluetas enfundadas en batas blancas se separaban por toda la zona
industrial.\\
Rápidamente se había vuelto aparente que no tenían oportunidad de
escapar si se quedaban en terreno abierto. Angela había estado
retrocediendo de vuelta al pueblo, pero Clara había señalado que no
sería mucho mejor allí y, además, ahora estaba segura de que fuera lo
que fuera que estuviera pasando allí tenía algo que ver con el parque
científico. No había ninguna otra razón por la que alguien debería estar
observándolas, sin hacer nada para ayudarlas.\\
Ahora que su atención se había girado hacia ellas, Clara había avistado
docenas de cámaras colocadas alrededor del lugar. Fue entonces cuando
tuvo su idea. La mayoría de los edificios parecían desiertos. Todos
excepto uno. Por lo tanto tenía sentido que la clave para descubrir lo
que estaba pasando se hallara dentro de aquel edificio. Tenían que
entrar. Angela la había mirado como si estuviera loca.\\
--¿Quieres entrar dentro del edificio del que acaba de salir esa gente
que parecen zombis?\\
Clara sencillamente había asentido.\\
--¿Tienes una idea mejor?\\
Usando los coches aparcados como cobertura, habían comenzado a abrirse
camino hacia el centro del complejo industrial, pero había demasiadas
siluetas de bata blanca como para que se pudieran mover libremente. Sólo
acababan de apañárselas para llegar a los contenedores de reciclaje sin
ser vistas. Necesitan mandar a sus perseguidores en otra dirección.\\
Clara había estado observando las cámaras de seguridad, intentando
determinar algún tipo de patrón en sus movimientos. Estaba segura de que
la mayoría de ellas eran puramente automáticas, rastreando de un lado a
otro en una secuencia cronometrada previamente. Presumiblemente ese
movimiento pre controlado podía ser controlado manualmente, como había
sucedido cuando habían visto avistadas, pero estaba comenzado a creer
firmemente que había demasiadas cámaras esparcidas por el lugar para que
alguien las controlara a todas de esa manera.\\
Desde su escondrijo podía ver tres de las cámaras. Había una ventana de
doce segundos en el que las tres apartaban las miradas de los
contenedores donde ella y Angela se ocultaban. Debía de haber sido pura
chiripa que hubieran llegado allí sin ser vistas. Esperando que su buena
suerte siguiera manteniéndose, Clara respiró hondo, agarró el abrigo de
algodón que Angela había estado llevando, y esperó al siguiente hueco en
el ciclo.\\
--En cualquier segundo\ldots{} --susurró ella.\\
--¡Siguen sin haber moros en la costa! --dijo Angela, escaneando el
aparcamiento.\\
Las tres cámaras estaban colocándose en la posición perfecta. Clara se
tensó. En cualquier segundo\ldots{}\\
Ahora.\\
Lanzándose hacia adelante desde la cobertura de los contenedores de
reciclaje, Clara corrió hacia el seto de espino blanco que rodeaba el
complejo industrial. Mientras corría, contó bajo su aliento:\\
--Doce, once, diez\ldots{}.--si no llegaba de vuelta cuando hubiera
llegado a cinco\ldots{}\\
Frenó en seco en el seto, dejando el abrigo para que se quedara atrapado
en las ramas superiores. Su corazón se frenó cuando su boca comenzó a
abrirse. ¡No se iba a quedar enganchado! Para su alivio, las puntiagudas
ramas del espino finalmente se enredaron en el algodón, pero su cuenta
mental estaba en cuatro.\\
Clara se giró y corrió todo lo rápido que pudo. Podía ver la cara
azotada por el pánico de Angela observándola desde los contenedores. Por
el rabillo del ojo, pudo ver las cámaras comenzando a girar otra vez
más.\\
No iba a lograrlo.\\
