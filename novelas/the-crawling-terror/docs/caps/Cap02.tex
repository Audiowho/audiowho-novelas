\chapter*{CAPÍTULO DOS}
\addcontentsline{toc}{chapter}{CAPÍTULO DOS}
La cirujana veterinaria Angela Drabble estaba justo abriendo la puerta
de su clínica cuando oyó gritar su nombre y el sonido de un bebé
llorando.\\
Se giró para ver a Gabby Nichols corriendo a través de la plaza del
pueblo. Tenía a su hijo en un brazo y algo liado en un mantelito en el
otro. Su hija la seguía justo detrás, arrastrando sus pies y llevando un
petulante mohín que todos los niños de tres años parecían tener cuando
se les pedía hacer algo que no querían.\\
--Buenos días, Gabby --Angela miró al mantelito liado--. Espero que no
sea otro mirlo malherido, ¿no?\\
--Esta vez no, Ang. No sabía dónde más llevarlo.\\
Angela pudo ver que Gabby estaba atolondrada y asustada. Tenía el ceño
fruncido. Ella no era así. En el poco tiempo que la conocía, le había
parecido ser una joven muy entera, que no entraba en pánico
fácilmente.\\
--Bueno, será mejor que entres y me enseñes qué tienes --llevando el
bulto con ella, Angela metió apresuradamente a Gabby y a Emily en la
clínica.\\
Una vez dentro, dejó el bulto en una de las mesas de exploración y pasó
los siguientes pocos minutos asegurándose de que Emily estuviera
adecuadamente entretenida con un libro de La Oveja Shaun y una taza de
zumo. Siempre había niños en la clínica insistiendo que querían quedarse
cerca de sus mascotas, y Angela siempre se aseguraba de que tenía
material con el que distraerles.\\
Dejando a Emily en la sala de espera, Angela cerró la puerta de la sala
de exploración y comenzó a desenvolver el mantel.\\
--A ver, veamos qué tenemos aquí\ldots{}\\
Cuando la última tapa de tela se echó para atrás dio una afilada
bocanada de aliento.\\
--¡Oh, Dios mío!\\[2\baselineskip]Esforzándose para evitar el contacto
con la telaraña pegajosa, el Doctor hizo un rápido examen del cuerpo.
Era un hombre, y llevaba muerto cierto tiempo. El Doctor recorrió su
destornillador sónico por encima de dos gigantescas heridas de punción
en los hombros del hombre. La carne alrededor de la herida estaba verde
y de aspecto enfermizo, pero no era eso lo que le había matado. Por lo
que el Doctor podía decir, había muerto de un ataque al corazón, y luego
algo le había arrastrado allí y le había metido en un capullo. Era ese
`algo' lo que preocupaba de verdad al Doctor.\\
Tocó con el dedo ligeramente contra una de las hebras de la telaraña que
sostenía al cadáver contra el tejado. Incluso con los más ligeros
toques, tenía que hacer un esfuerzo considerable para liberar su dedo.\\
El paso subterráneo estaba lleno de eso. Había sido el lugar perfecto en
el que atrapar una presa.\\
Dio un paso hacia atrás, hacia la luz del sol, con su mente
acelerándose. En algún lugar de aquel campo pacífico había una araña
muy, muy grande. Clara estaba a una cierta distancia, mordiéndose
nerviosamente las uñas. El Doctor no podía culparla. El cadáver no era
una imagen agradable. Visible a través de la telaraña, la cara del
hombre estaba contorsionada de dolor y miedo, con los labios echados
para atrás con una mueca terrible. Caminó hacia donde ella estaba
esperando.\\
--¿Está\ldots{}?\\
--Sí. Bastante --el Doctor le colocó una mano sobre el hombro y la alejó
amablemente del paso subterráneo.\\
--¿Verdad que no vamos a dejarle ahí colgando?\\
--Esa telaraña es increíblemente fuerte, y realmente no me apetece
quedarme ahí atrapado con lo que sea que la haya tejido por si vuelve.\\
Clara miró a su alrededor de forma nerviosa.\\
--¿Crees que sigue por aquí, por algún lugar?\\
--Es posible --admitió el Doctor--. Y en esta ocasión, no voy a sugerir
que nos las apañemos solos. Necesitamos ir a ese pueblo y conseguir
ayuda.\\
Clara asintió.\\
--De acuerdo --le lanzó una débil sonrisa al Doctor--. Nunca voy a
volver a quejarme de que me traigas a lugares aburridos.\\
Recorrieron el camino hasta el pueblo. Ringstone era bonito, como sacado
de una postal. Unas casas de campo construidas de piedra, algunas con
unos profundos tejados de paja, alineando ambos lados de una ancha y
abierta plaza dotada de árboles y bancos. En el centro había un alto
monumento de piedra caliza, un memorial de guerra de algún tipo. Una
brillante cabina de policía roja se alzaba en el exterior de la pequeña
tienda del pueblo, unas bajas paredes de piedra bordeando jardines que
deslumbraban con flores y en la distancia la torre de piedra baja y
fornida de una iglesia normanda se asomaba por encima de los tejados.\\
El Doctor miró a su alrededor, valorando rápidamente sus alrededores.\\
--No hay nada aquí --murmuró para sí mismo--. No hay nada aquí\ldots{}\\
--Pero eso es bueno, ¿verdad? --Clara estaba comenzando a preocuparse--
Quiero decir, tiene que ser mejor que una araña gigante esté aquí que en
medio del centro de una ciudad llena de gente, ¿no?\\
--Eso depende de lo que seamos capaces de encontrar para detenerla
--dijo el Doctor con remordimiento--. No es que vea exactamente a la
tienda del pueblo equipada para lidiar con una invasión de arañas
gigantes, ¿y tú?\\
El sonido de la puerta de un coche siendo abierta hizo que el Doctor
mirara a su alrededor. No muy lejos, una mujer había abierto la puerta
trasera de un Range Rover y estaba cargando una gran bandeja de metal en
el maletero.\\
--Vamos --dijo el Doctor--. Será mejor que comuniquemos las malas
noticias a los lugareños.\\[2\baselineskip]A Angela le seguía doliendo
la cabeza con las implicaciones de la criatura que Gabby había traído a
la clínica. Había asumido de forma bastante razonable que le había
traído algún tipo de pájaro o roedor que se habría abierto camino hasta
la casa o había sido golpeado por un coche; algo no demasiado improbable
en aquellos lugares. Nada la había preparado para lo que había estado
dentro del mantelito.\\
Era una típula común de jardín, un insecto segador, pero era gigantesco,
medía casi cuarenta centímetros de envergadura. Por desgracia, había
sido machacado bastante por Gabby en sus esfuerzos llenos de pánico para
matarlo. Sin embargo, quedaba lo suficiente como para que Angela
realizara un riguroso examen.\\
Prometiendo a Gabby que la informaría en cuanto tuviera más información,
había acompañado a la joven mujer y a sus hijos fuera de la clínica.
Gabby estaba aterrorizada de que pudiera haber más de esas cosas en su
casa, pero Angela le había asegurado que aquello tenía que ser una
casualidad de algún tipo. No era posible que hubiera más de esos.\\
Solo tranquilizada parcialmente, Gabby había salido para esperar que
abriera la tienda de herramientas, pretendiendo comprar tantas latas de
insecticida como le cupieran en las manos.\\
En cuanto se hubo ido, Angela había descartado el mantelito y había
extendido los restos del insecto gigante en la impoluta bandeja de acero
sobre la mesa. Después de una buena media hora diseccionando y haciendo
pruebas tuvo que admitir que estaba sin palabras. Por lo que ella podía
decir, todo en el insecto era normal, todo excepto su tamaño.\\
Dándose cuenta de que había un límite de cuánto podía averiguar en una
investigación sobre los orígenes de la criatura por su propia cuenta,
había decidido que lo llevaría al doctor Goodchild en el hospital rural
de Chippenham. Con suerte, él sería capaz de ayudarla a llevar a cabo
unas pruebas decentes, posiblemente usando su escáner de ultrasonido.
Había decidido no llamar por adelantado ni decirle que iba. Si comenzaba
a hablar de insectos gigantes, era posible que pensara que ella
estuviera volviéndose loca. Era mejor presentarse con la cosa monstruosa
cara a cara. Estaba cargando la bandeja en la parte trasera de su coche
cuando se dio cuenta de dos siluetas: un hombre y una mujer, caminando
hacia ella.\\
--Buenos días --el hombre tenía un tono escocés en su voz.\\
Angela puso con cuidado una tela por encima del insecto en el
maletero.\\
--Buenos días.\\
--¿Me puede decir si hay una comisaría en el pueblo?\\
--No. La más cercana está en Wyndham. Pero Charlie Bevan, el agente
local, vive justo al otro lado de la plaza --Angela frunció el ceño--.
¿Va todo bien? ¿Ha habido un accidente?\\
--No exactamente\ldots{} --el hombre y la mujer se miraron entre ellos--
Puede que tengáis un ligero problema de insectos.\\
Angela sintió que se le iba la sangre de la cara.\\
--Oh, no. Por favor, dígame que no hay más\ldots{}\\
Las pobladas cejas del hombre se alzaron, intrigado.\\
--¿Más?\\
--Será mejor que eche un vistazo a esto --Angela levantó la tela de la
bandeja de acero--. Dudo que jamás haya visto usted un insecto más
grande.\\[2\baselineskip]Clara se subió en el taburete en un rincón de
la clínica veterinaria observando cómo el Doctor y Angela se inclinaban
sobre la mesa de exploración, mirando de cerca al enorme insecto que
estaba bajo las brillantes luces.\\
Nunca dejaba de sorprenderle lo rápido que el Doctor conseguía tomar el
control completo de la situación, fuera en un planeta alienígena o en un
pueblo en el campo británico.\\
En cuanto Angela les había mostrado el insecto, el Doctor había
comenzado a proferir todo tipo de preguntas científicas y biológicas, la
mitad de las cuales Clara ni siquiera había empezado a entender. El
alivio de Angela tras darse cuenta de que tenía a alguien con quien
compartir las preocupaciones había sido tangible. Habían vuelto a traer
el lioso cuerpo y el Doctor había escuchado mientras ella no había
dejado de hablar durante diez minutos sobre teorías de mutaciones y
contaminación química.\\
Solo cuando había tenido su completa confianza, el Doctor final y
amablemente le había dado las noticias sobre el cuerpo que habían
encontrado en el paso subterráneo.\\
Angela se había puesto muy callada y pálida. Le llevó varios momentos a
Clara darse cuenta de que en una comunidad tan pequeña era inevitable
que el hombre muerto fuera alguien que ella conocía, y que probablemente
le conociera bien.\\
--Lo siento --el Doctor colocó una mano sobre su hombro--. Si sirve de
consuelo, no creo que la araña le matara. Por lo que he podido ver,
murió de un ataque al corazón.\\
Angela asintió.\\
--Creo que será mejor que vaya a por el agente Bevan --se puso la
chaqueta, temblando a pesar del cálido día de primavera--. ¿Os
importaría esperar aquí?\\
El Doctor asintió.\\
--Por supuesto --en cuanto Angela se hubo ido, sacó el destornillador
sónico de su chaqueta y comenzó a examinar el insecto muerto una vez
más.\\
Clara se había resbalado del taburete y se había acercado al hombro del
Doctor.\\
--Entonces, ¿tiene razón?\\
--¿Razón? --el Doctor no levantó la mirada-- ¿Razón sobre qué?\\
--Mutaciones.\\
Se enderezó, observando las lecturas del destornillador.\\
--Sí y no.\\
Clara se cruzó de brazos y le lanzó una mirada de odio.\\
--Bueno, eso es de gran ayuda.\\
El Doctor la observó con una mirada afilada.\\
--Sí que es una mutación, pero no es natural. Alguien ha usado mucho
tiempo y muchos equipos caros para diseñar esta criatura.\\
--¿Diseñarla? --Clara la había observado de forma incrédula-- ¿Quieres
decir que estas cosas han sido construidas?\\
--Modificadas. La fisionomía básica parecería ser una especie de
ocurrencia natural, pero hay trazas de ADN recombinante, hormonas de
crecimiento y potenciadores sinápticos.\\
--¿Pero por qué? ¿Por qué nadie querría crear estas\ldots{} cosas?\\
El Doctor se dio golpecitos con el destornillador sónico contra los
labios.\\
--No estoy seguro\ldots{}\\
Su conversación fue interrumpida por la vuelta de Angela, seguida por un
bajo y fornido hombre pelirrojo con uniforme de policía.\\
--Doctor, Clara, este es el agente Bevan.\\
Cualquier duda o pregunta que Charlie Bevan pudiera haber tenido, había
muerto en su garganta cuando atisbó el insecto segador en la mesa.\\
--Oh, Señor. Creía\ldots{} creía que esto era algún tipo de\ldots{}\\
Sacó un gran pañuelo blanco de su bolsillo y se limpió el sudor que se
acumulaba en su frente.\\
--Creo que será mejor que me mostréis dónde habéis encontrado ese
cuerpo, ¿no?\\
