\chapter*{Prólogo}
\addcontentsline{toc}{chapter}{Prólogo}

{Peter Dickson conoció la verdad sobre los gatos negros de su madre.}

{---Si un gato negro se te acerca ---le decía---, entonces eso
	significa buena suerte, claro. Pero si se queda a medio camino y
	entonces se gira\ldots{} Si tiene unos ardientes ojos
	verdes\ldots{}---solía sorberse la dentadura y negar con la
	cabeza---. Dicen que tu padre vio un gato negro esa mañana, de
	camino a su barco. Supongo que tenía ojos verdes. Supongo que debería
	haber vuelto a casa en ese momento, como cualquier marinero sensato.
	Estaría aún aquí hoy si hubiera prestado atención a ese gato negro. Son
	unos animales volubles, los gatos. No confíes en ellos. Sólo piensan en
	sí mismos. Si te traen suerte, buena o mala, puedes estar seguro que es
para su propio beneficio.}

{El gato negro que Dickson vio casi treinta años más tarde ni se le
	acercó ni giró en redondo. Le observó desde el otro lado de la calle con
	unos reflectantes ojos vidriosos. Era imposible saber de qué color eran
	realmente; ¿entonces significaba suerte o no? Dickson respiró hondo una
	bocanada del neblinoso aire londinense. Ni lo sabía ni le importaba. No
	era supersticioso, a diferencia de su anciana madre, una mujer
	victoriana en todos los sentidos, pensó. Y de cualquier modo, ni
	siquiera se podría saber el color del mismo gato, sólo parecía negro
	porque estaba oscuro. Había un amasijo de coloración pálida bajo su
	barbilla, un triángulo blanquecino en la oscuridad que se arremolinaba
	bajo el brillo de sus ojos. Entonces, en un instante, el gato se hubo
ido. Como si los ojos se le hubieran apagado.}

{Dickson dejó salir una humareda de su cigarrillo. Una última aspiración
	y se volvió a meter en la casa. Los invitados llegarían pronto, y
	necesitaba asegurarse de que todo estaba listo. Lanzó con un movimiento
	rápido el extremo del cigarrillo y le observó brillar brevemente antes
	de desvanecerse y apagarse. Como los ojos del gato. Tosió por el frío
aire de octubre, y se giró para entrar de nuevo.}

{Rose bajó la mirada para mirarse, preguntándose lo ridícula que
	parecería. ¿Se vestían de verdad así en 1920? ¿Con una fina capa de
	algodón hasta las pantorrillas? ¿Y con un tono verde menta? Había
	encontrado una larga y oscura capa con capucha, la cual paseó
pesadamente por la sala de la consola de la TARDIS.}

{El Doctor se detuvo para mirarla. Toqueteaba algún medidor u otra cosa.
	Satisfecho, asintió y se movió al siguiente controlador, que resultó
	estar cubierto por la capa de Rose. Un breve fruncimiento de ceño y el
	Doctor se apartó. Rose vio aquellos fieros e intensos ojos reflejando la
	luz de la consola mientras se concentraba en el siguiente controlador.
	Le gustaba la forma en la que se erguía allí tan quieto y confiado de sí
	mismo, y aún así sabía que en cualquier segundo podría irrumpir en una
	amplia sonrisa. Cuando pareció darse cuenta de que estaba siendo
observado, la volvió a mirar de nuevo.}

{---¿Qué?}

{---¿Falta mucho?}

{---Suenas como una niña de excursión.}

{---Soy una niña de excursión. En una excursión al pasado ---no
pudo evitar sonreír ante la idea, y él le devolvió la sonrisa.}

{---Sí. Es genial, ¿no es así? Ahí fuera están en el 1924. O lo
estarán en un momento ---toqueteó un control, animado.}

{---¿Y ahí es cuando tiene lugar eso de la exposición?}

{---La Exposición del Imperio Británico, sí. Conseguís tener un poco
de cultura de tanto en tanto.}

{Rose rió.}

{---Es como una excursión del colegio. Repítemelo, ¿por qué quiero
verlo?}

{Él parpadeó en un amago de incredulidad.}

{---Porque tu mejor amigo va.}

{Eso le hizo sonreír.}

{---¿Y entonces por qué no se ha vestido para la ocasión?}

{Entonces se sorprendió, estando tras la consola y señalándose la ropa.
	Una chaqueta de cuero encima de una camiseta de cuello redondo color
marrón oscuro, unos pantalones desgastados y unos zapatos raídos.}

{---Perdona ---dijo, señalándose---. Esta camiseta es nueva.}

{Sin esperar su veredicto sobre la camiseta, se giró hacia el escáner.
	La imagen estaba oscura, demasiado oscura para ver nada. Entonces la
	oscuridad se convirtió en siluetas cuando el contraste y el brillo se
ajustaban.}

{---Podríamos intentar con el infrarrojo ---murmuró el
Doctor---. Pero no creo que haya mucho calor ahí fuera.}

{Rose apenas podía distinguir algunas formas: piezas de hierro y algunas
placas de madera; un viejo marco de cama y un montón de cubos.}

{---Hace frío y estamos en el patio de algún chatarrero.}

{El Doctor se encogió de hombros.}

{---Me gustan los patios de chatarreros. Nunca se sabe lo que se puede
	encontrar ---comprobó otra lectura---. Necesitarás esa capa
	---dijo, como si la viera por primera vez. Las puertas se abrieron y
una ligera brisa de niebla se introdujo desde el patio.}

{---¿Crees que conoceremos a alguien famoso? ---preguntó Rose.}

{---¿En el octubre de 1924?}

{---¿Tenían gente famosa, no?}

{Su voz flotó desde el neblinoso exterior.}

{---No había televisión, pero sí que tenían de eso.}

{Rose se apresuró a seguirle, con emoción hacia lo desconocido.}

{En un primer momento pensó que era el gato, luchando con algo. Haciendo
	un terrible ruido aullador. Pero había algo rítmico y mecánico en el
	sonido que irrumpió el aire nocturno. No era un sonido hecho por ningún
	animal. Era un chirriante y rechinante sonido como si algún gran motor
	se estuviera encendiendo y entonces se apagara. Una y otra vez. Venía de
	todas partes y a la vez de ningún lugar, daba igual el lado hacia el que
se girara, el sonido estaba resonando a su alrededor.}

{Un haz de luz detrás de la puerta del patio de Gibson. Por un momento,
	Dickson vio el brillo por encima de la puerta de madera, y la luz
	iluminando entre las placas. Entonces se apagó y el sonido finalizó con
un zumbido satisfecho.}

{---¿Quién anda ahí? ---gritó Dickson. Pero su voz era rasposa y
	quebradiza. A penas pudo oírse a sí mismo. Miró de nuevo a la casa y
	consideró volver a entrar. Pero sintió curiosidad por el sonido y la
	luz. Dickson bajó los escalones de la puerta trasera y se dirigió a la
puerta del patio de Gibson.}

{Cruzó la calle, sin ver el gato negro que giraba en redondo por la
	calle, con la cola levantada. Caminó con cuidado hacia las puertas de
	madera, ignorando cómo las sombras tras él parecieron ser más profundas
	y grandes. ¿Era aquél el sonido de una puerta abriéndose? ¿Eran aquello
voces?}

{La sombra tras él aceleró el paso, su presa ahora a su alcance. Sus
	dedos inhumanos se alargaron, temblando rítmicamente, chasqueando hacia
la parte trasera del cuello de Dickson.}

{En la distancia Dickson podía oír el Big Ben marcando la media hora con
	las campanadas. Vaciló, con el vello en la punta de la nuca erizándose
	como si hubiera una ligera brisa. De repente todos sus sentidos se
	tensaron. Podía ver un pálido brillo de luz tras la puerta. Podía sentir
	el frío aire nocturno en su piel. Podía oler el agua estancada del
	Támesis siendo arrastrada por la brisa. Por alguna razón también pudo
	percibir el férreo sabor de la sangre en su boca, como si se hubiera
mordido la lengua.}

{Y bajo su asombro, cuando las campanadas acabaron, estuvo seguro de oír
el tictac del Big Ben, marcando los segundos restantes de su vida.}
