\chapter*{DOS}
\addcontentsline{toc}{chapter}{DOS}

{Llevaban caminando por las calles empapadas durante lo que parecía una
	eternidad. La humedad que había en el ambiente era tal, que resultaba
	difícil decir si había niebla o caía una fina llovizna. En un primer
	momento, Rose pensó que el Doctor tenía un plan definido, que tenía
	alguna idea de dónde buscar la TARDIS.\@ Pero después de bajar por otra
	calle se dio cuenta de que él tenía exactamente la misma idea que ella.
O sea, ninguna.}

{---Piensa, piensa, piensa ---decía entre dientes para sí mismo, parados
	en la esquina de una calle cualquiera junto a un buzón de correos, cuyo
color rojo era la única nota de color dentro de ese mundo gris oscuro.}

{---Tal vez alguien se encaprichó de ella ---sugirió Rose.}

{---No lo creo. Demasiada coincidencia.}

{---Entonces alguien nos vio llegar. O sabe lo que es la TARDIS.}

{---Puede ---movió los dedos alentándola a seguir---. Más ideas, más
pistas.}

{---Alguien atacó a Dickson, ¿no? Lo salvamos. Tal vez eso los
ahuyentó.}

{---Podría ser. ¿Más?}

{---Tiene que haber una conexión, ¿no? ---añadió ella.}

{El Doctor asintió varias veces con rapidez.}

{---Es probable.}

{---Y Sir George tenía miedo de alguien o de algo. Pensó que era un
ataque deliberado.}

{---Bastante deliberado. Y motivado.}

{---¿Y ahora qué?}

{El Doctor se lamió un dedo y lo alzó al aire como si comprobara la
fuerza y la dirección del viento.}

{---¡Por aquí! ---y señaló el camino por donde habían venido.}

{---¿Seguro?}

{---Afirmativo ---concluyó, poniéndose en marcha a paso ligero.}

{---¿A la TARDIS?\@ ---para Rose era lo mismo que haberla encontrado.}

{Pero su respuesta la desanimó tanto o más que la incipiente lluvia.}

{---Nop. Volvemos a casa de Sir George. Es la única conexión, la única
pista que tenemos.}

{---Espero que recuerdes el camino.}

{La llovizna pronto dio paso a una intensa lluvia y tuvieron que sortear
	los charcos que se iban formando. Llegaron a la casa al tiempo que un
	enorme coche negro salía marcha atrás. El conductor era una figura
	recortada por la luz que provenía de la casa. Y se vislumbraba una
silueta femenina sentada en la parte posterior.}

{Dickson apareció como por intuición, desplegando un paraguas mientras,
	apresurado, bajaba las escaleras. Sus ojos se abrieron, como
	sorprendidos al ver al Doctor y a Rose, si bien era una sorpresa bien
disimulada.}

{---Hemos decidido aceptar la oferta de la cena, después de todo ---le
anunció el Doctor.}

{---Si es que sigue en pie ---añadió Rose.}

{---Por supuesto, señor. Por favor, entren. Estaré con ustedes en un
	momento ---Dickson centró esa hospitalidad tan profesional en sujetarle
	el paraguas a la mujer del coche, mientras ésta caminaba sobre la
acera.}

{---Podría habernos ofrecido el paraguas ---se quejó Rose, mientras se
	sacudía el agua del pelo y se lo peinaba tras haberse quitado la
capucha.}

{---¿Y dejar que se corra la pintura?}

{---¿A qué te refieres?}

{Por toda respuesta, el Doctor hizo un gesto con la cabeza, señalando a
	la mujer que ahora entraba al pasillo tras ellos. Dickson se detuvo en
	la puerta detrás de ella, bajando el paraguas. Pero Rose sólo prestaba
	atención a la mujer. A su cara. Parecía haber salido de un baile de
	máscaras. Su vestido era de un tono pálido y de seda brillante, y se
	mecía a su alrededor con la brisa que soplaba a través de la puerta
	abierta. El cabello, de un intenso pelirrojo, caía en cascada sobre sus
	hombros desnudos. Pero su rostro lo cubría una exquisita máscara con
	forma de mariposa, por lo que tan sólo se le veía la boca. La máscara,
	salpicada de lentejuelas, estaba pintada de tonos vibrantes: amarillos,
	rojos, azules y verdes. Una delicada pluma azul enmarcaba cada lado,
	contrastando con el tono rojizo del pelo. Y unos ojos
	extraordinariamente azules miraban, sin pestañear, a través de unos
agujeros con forma de almendra.}

{---¿Qué tal? ---dijo con una voz suave y melosa---. No creo que nos
	hayamos visto antes ---extendió la mano hacia el Doctor, y Rose vio que
	el guante blanco le llegaba hasta el codo. Por el ángulo en que dobló la
	muñeca hacia él, era obvio que esperaba que el Doctor la besara. En
cambio, éste la tomó gentilmente y la estrechó delicadamente.}

{---Soy el Doctor ---dijo---. Y esta es mi amiga Rose.}

{La mujer asintió con la cabeza, cualquier decepción quedó escondida
tras la máscara.}

{---Melissa Heart ---se presentó. Asintió levemente hacia Rose. Un
	reconocimiento, nada más---. Supongo que usted, como yo, está aquí por
la conspiración.}

\mbox{}

{Pese a la presencia de Melissa Heart, quien se disculpaba profusamente
	por no haber asistido a la cena, eran pocos los que se sentaban en el
	comedor. Ya habían retirado la cena y los comensales bebían vino blanco
	en pequeñas copas de vidrio facetado. El Doctor, Rose y Melissa se
	sentaron en las sillas que habían quedado vacantes después de que los
Koznyshevs y Lord Chitterington se hubieran marchado.}

{Por lo menos había menos nombres que recordar, pensó Rose, aunque ya no
quedaba nada que comer salvo un triste trocito de tarta de manzana.}

{El Doctor se había disculpado con Sir George y había aceptado la
	renovada oferta para cenar. O, al menos, para tomar el postre. Le contó
	que los habían ``defraudado'' y que se habían quedado sin alojamiento.
	Sir George les ofreció enseguida alojarse en la casa, pero su esposa le
	recordó amablemente que ya tenían invitados en casa y que estarían muy
apretados.}

{--- Tranquilo ---dijo el Doctor---. Encontraremos un hotel, o algo.}

{---Hay habitaciones en el Club Imperial ---anunció Repple---. Seguro
	que podemos responder por ustedes allí, al menos durante un día o dos
hasta que encuentren otro alojamiento.}

{--- Qué bien que se haya solucionado ---dijo Melissa Heart, aplaudiendo
	con aparente deleite---. Me acabo de mudar a mi propia casa, la antigua
	vivienda de Anthony Hubbard, junto al río. A lo mejor la conoce. Pero,
como decía,}

{apenas acabo de deshacer las maletas, así que me temo que darles
alojamiento sería difícil.}

{El Doctor capeó las diversas y obvias preguntas que acompañaron la
	llegada de la tarta de manzana. Estarían en Londres unos días para ver
	la Exposición del Imperio Británico. Oh, sí, lo estaban deseando. Sí,
	conocían la ciudad, pero habían estado fuera un tiempo. Viajando. El
	rostro inexpresivo de Melissa Heart, la Dama Pintada, según recordaba
	Rose que alguien la había llamado, observó con mucha atención al Doctor
mientras hablaba, pareciendo absorber cada palabra.}

{--- Bueno ---dijo el Doctor mientras hundía la cuchara en la tarta---,
¿y de qué va esta conspiración?}

{El repentino silencio quedó roto por el jadeo involuntario de alguien.}

{---¿Es que no quieren hablar de ello? ---el Doctor se encogió de
	hombros y asintió con simpatía. Se levantó, se quitó la chaqueta de
	cuero y la colgó del respaldo de la silla. Luego se volvió a sentar---.
¿Saben qué? ¿Por qué no lo adivino?}

{Rose echó un vistazo a la mesa para ver la reacción de los asistentes.
	Sir George estaba recostado en su silla. Si acaso, parecía ligeramente
	divertido. Su esposa, por el contrario, parecía nerviosa e inquieta. El
	Coronel Oblonsky se había puesto rojo y sus labios temblaban de ira.
	Aske, Repple y la Dama Pintada se mostraban impasibles por igual y era
difícil leer sus rostros.}

{El Doctor inhaló.}

{---O también podemos acabarnos el postre y dejar que sigan ustedes
	adelante con ella. Gracias por el tentempié. No quisiera molestar o
entrometerme.}

{---¡Qué intrigante! ---fue Melissa Heart la que habló---. Como recién
	llegada a este pequeño grupo, me interesaría oír los detalles. Y también
	ver si lo que el Doctor ha averiguado se acerca lo más mínimo a la
verdad.}

{---¿Y cómo sabemos que no es un agente bolchevique? ---bramó Oblonsky,
	su ira finalmente sacando lo mejor de él---. Yo digo que lo echemos a la
	calle ---se inclinó considerablemente hacia delante, esparciendo la
	cubertería---. Cuando hayamos determinado lo mucho que sabe y para quién
trabaja.}

{---No soy agente de nadie ---dijo el Doctor con calma.}

{---Señores, por favor ---Sir George se puso de pie, arrojando la
	servilleta al lado de su plato. Pero Oblonsky no prestó atención y
siguió mirando malévolamente al Doctor y a Rose.}

{Fue el Comandante Aske quien calmó la situación. Se aclaró la garganta
y dijo en voz baja:}

{---Dudo que un agente bolchevique, o cualquier otro tipo de agente,
	fuera tan osado como para auto invitarse a la cena y ofrecerse a
	explicar sus planes, El Coronel.Repple y yo estamos en alerta constante
ante la posibilidad de espías, infiltrados, agentes y asesinos.}

{Repple levantó la mano según Aske terminaba de hablar.}

{---El Doctor, obviamente, no es ninguno de ellos. Él y su acompañante
podrían ser de ayuda. Mantengamos la mente abierta.}

{Oblonsky se echó hacia atrás cruzando los brazos, todavía enojado.}

{---Aun así, sigo sin estar convencido.}

{---Bueno, es un comienzo ---dijo el Dector alegremente. Levantó su copa
	en un remedo de brindis y después tomó un sorbo de vino---. Mmm, 1917
---declaró.}

{---Ni se le acerca ---dijo Sir George---. Es un burdeos de 1921.}

{---No me refería al vino ---dijo severamente el Doctor con tono
	firme---. Aunque si lo hiciera, diría que las uvas provienen de un
	pequeño viñedo de las afueras de Briançon. No ---continuó con la
	suficiente rapidez como para que Rose adivinara que se lo había
inventado---, me refería a la Revolución Rusa.}

{---No es difícil de adivinar ---dijo Rose al ver la sorpresa en sus
rostros. No es que ella lo hubiera descubierto antes.}

{Ni que tuviera una pista de verdad de lo que él estaba hablando---. Hay
muchos rusos aquí. El Coronel, los}

{Koznyshev, antes.}

{---Y Lady Anna ---añadió el Doctor.}

{Anna asintió, levantando las cejas como único indicio de su sorpresa.}

{---Me fui en octubre de 1917. Con mi marido y mi hijo pequeño.}

{---Su primer marido ---dijo Rose, y se alegró de ver que el Doctor
elevaba una ceja mientras Anna asentía.}

{---Yo conocí a Sir George cuando estaba en la embajada británica en
	Moscú. Él era la única persona a la que conocía lo suficientemente bien
	como para pedir ayuda cuando llegué a Londres ---se inclinó sobre la
mesa y le tomó la mano.}

{---Entonces ---dijo Rose, dispuesta a sacar el máximo provecho de su
	éxito hasta el momento---, tenemos algunos rusos desposeídos y a Repple,
	un hombre que ha perdido su título y lo quiere recuperar. Todos ustedes
	quieren echar a Lenin y compañía, y reclamar las tierras que han
perdido, ¿es eso? ---sonrió, satisfecha consigo misma.}

{La Dama Pintada dio unas palmadas con aparente admiración.}

{---No ---intervino el Coronel Oblonsky.}

{---Vaya.}

{---Pero se acerca bastante ---apuntó el Doctor, sonriéndole---. No está
mal.}

{---Gracias ---murmuró Rose.}

{---Ha acertado con respecto a mí ---concedió Repple. Se puso en pie y
	miró a su alrededor. Aske suspiró y volvió el rostro. Pero Repple lo
	ignoró---. No descansaré hasta que haya recuperado mi patrimonio. No, no
	en Rusia. Hasta el golpe de Estado que me arrebató el poder, hasta que
	me tildaron de criminal y me enviaron al exilio, yo era el príncipe
	elector de Dastaria. El Rey, si lo prefiere. Cuando regrese, el pueblo
	se levantará y expulsará a los opresores que han asolado nuestra
patria.}

{--- Señor ---dijo Aske en voz baja---, triunfaremos. Pero hemos de
	hacerlo con calma y despacio. Hay que pisar con cuidado. Sacar el máximo
	rendimiento a los apoyos y aliados que tenemos. No llamar la atención
indeseada.}

{---Tenemos que ayudar a nuestros amigos también ---dijo Repple---. Me
	temo que poco podemos hacer, salvo prestar nuestro apoyo y nuestro
	nombre a su empresa, amigos míos. Pero Dastaria comparte frontera con
	Rusia. Su causa es noble. Toda la ayuda que podamos ofrecer, la
ofreceremos incluso desde el exilio.}

{---Me temo que será poca ---dijo Aske en voz baja.}

{---Parece ---continuó Oblonsky---, que de alguna forma obtiene usted
	información, Doctor. Puede que no sea un agente de Lenin o de Trotsky y
sus lacayos. Pero ahora ya lo sabe todo.}

{El Doctor asintió.}

{---Casi todo. Para tener una posibilidad de éxito, habiendo pasado
	tanto tiempo tras la revolución, tendrán una carta de triunfo. Algo que
puedan utilizar para conseguir apoyos. Para entusiasmar a la gente.}

{--- Siga ---le urgió Sir George.}

{---Creo que pretenden volver a Rusia con el heredero al trono ---y
	sonrió de repente---. ¿Tengo razón o tengo razón? El silencio fue
	confirmación suficiente. Todos los ojos estaban ahora puestos en el
	Doctor. Salvo Rose. Ella observó a los otros comensales y para su
	sorpresa, vio que si bien la máscara de Melissa Heart miraba al Doctor,
sus ojos lo hacían hacia Repple.}

{---Ahora ---continuó el Doctor---, el Coronel aquí presente bien podría
	ser el legítimo Zar de la Madre Rusia. Pero tiene más pinta de militar.
	Un soldado leal, ¿me equivoco? La sucesión no incluye a las mujeres por
	todo tipo de razones medievales, todas ellas inexactas. Así que, creo
	que el Zar es\ldots{} El Conde Koznyshev, aunque no le agrada el
	pastel---. Se recostó, como un prestidigitador a la espera de aplausos.
	Sólo hubo silencio---. ¿En la sala de baile? ---añadió esperanzado---.
¿Con el huevo de Fabergé?}

{Rose lo vio enseguida, sin embargo. Un extraño fragmento de una
	conversación, un comentario sin sentido, le vino a la cabeza: ``No se
	atrevería''. Debió de jadear más alto de lo que pretendía, pues ahora
todo el mundo se había girado hacia ella.}

{---Es Freddie, ¿no? ---dijo---. Freddie es el legítimo Zar de Rusia.}

{El resto de los detalles de la historia y de los cabos sueltos fueron
	apareciendo mientras terminaban la cena. Anna, Anastasia, era prima del
	Zar Nicolás II y también estaba emparentada con la reina Victoria. Su
	primer marido había sido un primo de la última Zarina. Con el Zar y su
	familia directa muertos, junto con un sinnúmero de otros parientes,
Frederick, a sus diez años, era el siguiente en la línea de sucesión.}

{El Coronel Oblonsky había sido Jefe de la Guardia personal del Zar y
	parecía culparse por el éxito de la revolución. Los Koznyshevs eran
	leales partidarios del Zar. Lord Chitterington había estado allí para
	ofrecer el apoyo clandestino del gobierno británico, un apoyo que no se
	extendería a la intervención militar, según había subrayado, sino que se
	limitaba a la ayuda financiera y a diversas presentaciones
diplomáticas.}

{Repple volvió a dejar claro que él podría ofrecer poco más que palabras
	de apoyo hasta que recuperara su propio trono. Tal vez él tenía la
	esperanza de volver a Dastaria con la ayuda y la intervención de un Zar
	restaurado. Incluso sin siquiera saber cómo estaba destinada la historia
	a continuar, a Rose le pareció que lo único que los ``conspiradores''
podían hacer era hablar y hacer planes.}

{---¿Por qué estás aquí? ---preguntó Rose a Melissa Heart tras la cena
	mientras se dirigían hacia el salón principal para continuar con sus
debates.}

{---Oh, querida ---le respondió ella---, por la diversión. Y he de
conocer a conocer a mucha gente desde que llegué a Londres.}

{Fuera por diversión o no, Melissa Heart rehusó reunirse con los demás
en el salón principal. Se disculpó y los dejó en el pasillo.}

{---Sé dónde está la salida ---le aseguró a Dickson, quien empujaba un
carrito con dos licoreras de Oporto.}

{Rose aún remoloneó un rato en el pasillo antes de seguir a los demás.
	Melissa Heart la observó desde detrás de su máscara, como esperando a
	que Rose se marchara antes que ella. El efecto era inquietante. Rose se
	giró para seguir al Doctor hasta el salón. Mientras lo hacía, vislumbró
	algo en la escalera, un tenue movimiento tras la balaustrada que
	recorría el rellano de las escaleras. Hizo una pausa, escudriñando la
	distante oscuridad. Una mano apareció momentáneamente por encima de la
	barandilla. Y saludó. Rose echó un vistazo hacia donde estaba Melissa
	Heart para asegurarse de que no estaba mirando y devolvió el saludo
rápidamente.}

{---Buenas noches, Freddie ---murmuró Rose mientras se giraba para
irse.}

{Una vez que Melissa se hubo ido y Anna se hubo retirado a dormir, sólo
	quedaban en el salón Sir George, el Coronel Oblonsky, Aske y Repple,
además de Rose y el Doctor.}

{---No digo que vaya a ser una tarea fácil, caballeros, señorita Tyler
	---declaró Oblonsky. El vino y el oporto hacían que arrastrara las
	palabras, marcando aún más su acento---. Va a ser un proceso largo y
	difícil, y de ninguna manera estamos preparados para embarcarnos en una
invasión de la madre patria a gran escala.}

{Sir George asintió y le dio una palmada amistosa en el hombro al
coronel.}

{--- No nos hacemos ilusiones ---reconoció---. Creo que nuestro Freddie
	habrá alcanzado la madurez antes de que podamos ayudarle a reclamar lo
que le pertenece por nacimiento.}

{---No tienen esperanza, ¿no es así, Doctor? ---comentó Rose en voz
	baja, mientras admiraban un oscuro retrato de una señora muy seria que
colgaba en el otro extremo de la habitación.}

{--- No ---respondió. Su voz sonaba realmente triste---. Pero soñar es
gratis. No hacen ningún daño.}

{---¿Y qué pasa con el ataque a Dickson?}

{---Algo completamente distinto, creo ---y le devolvió el ceño fruncido
a la mujer del cuadro---. Pero no sé qué.}

{En el otro extremo de la habitación, Repple y Oblonsky estaban
	enfrascados en una conversación seria. Aske llevó a Sir George a un
lado, más cerca del Doctor y Rose. Ella les oyó hablar. }

{--- Me preguntaba, Sir George, si podría concederme unos momentos en
	privado. Hay algo que me gustaría hablar con usted. Es \ldots{} ---hizo
una pausa y miró a Repple y a Oblonsky---. Es un tema delicado.}

{---¿En la biblioteca? ---sugirió Sir George. Ambos asintieron
cortésmente al Doctor y a Rose al irse.}

{Dickson había vuelto y estaba recogiendo las copas vacías. El Doctor lo
detuvo al pasar.}

{---¿Señor?}

{---Lo de esta noche, díganos otra vez lo que ocurrió exactamente. Tan
detalladamente como le sea posible.}

{Si estaba sorprendido o no, no lo exteriorizó.}

{---Escuché un ruido extraño, vi una luz procedente del patio. Así que
fui a mirar.}

{---¿Y después qué? ---preguntó Rose.}

{El hombre se encogió de hombros.}

{---Una mano me agarró por detrás. Me tapó la boca, obligándome a
	volverme. Y otra mano me cogió de la garganta. Estaba fría, eso lo
recuerdo. Muy fría.}

{---Fría como metal ---murmuró el Doctor.}

{Dickson asintió.}

{---Luché, pero eran demasiado fuertes. No me podía deshacer de ellos.
	Entonces escuché una voz serena, casi melódica\ldots{} ---frunció el
	ceño mirando al vacío mientras lo recordaba---. Decía que debía
	responder unas preguntas. Preguntaba sobre Sir George y los invitados de
	esta noche, pero antes de que pudiera responder, llegaron ustedes.---se
encogió de hombros y cogió la copa que le ofrecía el Doctor.}

{---¿Nada más? ¿Ningún otro detalle que haya pasado por alto?}

{---Había algo raro, sí. Un sonido.}

{La puerta volvió a abrirse antes de que pudiera continuar. Sir George y
	Aske habían vuelto. El primero parecía muy serio, el segundo,
compungido.}

{---Lo entiendo ---dijo Sir George mientras cruzaban la sala---.
Lamentable, pero no se puede evitar.}

{---Es usted muy amable, señor ---le respondió Aske---. Por supuesto,
	cualquier cosa que podamos hacer para ayudar\ldots{}}

{---Es hora de marcharnos ---anunció Repple.}

{El Coronel Oblonsky saludó y Repple asintió en reconocimiento.}

{---Doctor, señorita Tyler ---dijo Repple acercándose a ellos---. Hay un
	corto paseo hasta el Club Imperial. O podemos llamar a un coche, si lo
prefieren.}

{---Un corto paseo suena genial ---dijo el Doctor---. Cogeré mi
	abrigo.---se quedó inmóvil, a mitad de camino hacia la puerta---. ¿Oyen
eso?}

{---¿El qué? ---preguntó Sir George, inclinando la cabeza hacia un
lado.}

{---Me pareció\ldots{}---el Doctor frunció el ceño---. Sí, ahí está otra
vez. Tic tac.}

{Rose también lo oía, ahora que el Doctor lo había mencionado. Un sonido
sordo, apenas audible.}

{---Es un reloj ---dijo.}

{---Aquí no hay relojes ---respondió el Coronel Oblonsky en voz baja.}

{---Cierto ---convino Sir George---. No hay relojes en el salón
	principal. Había uno. Se rompió ---se encogió de hombros, como
disculpándose---. Yo no oigo nada.}

{---Es casi imperceptible ---dijo el Doctor.}

{Aske y Repple intercambiaron miradas. Ambos se encogieron de hombros,
no muy convencidos.}

{Pero Dickson estaba alerta, inmóvil.}

{---Eso es, señor ---dijo, su voz un susurro ronco---. Eso es lo que oí.
Cuando me atacaron.}

{---Debe de venir del vestíbulo ---opinó Sir George---. Ahí está el
reloj de pie.}

{---El vestíbulo ---murmuró el Doctor---. Por supuesto ---se llevó un
	dedo a los labios y se acercó rápidamente hasta la puerta, sin hacer
ruido. Se detuvo un momento y luego la abrió de repente.}

{No había nadie.}

{---Tempus fugit ---dijo el Doctor.}

{El gato, que se
	estiraba en el sofá, parpadeó abriendo los ojos por el ruido. Rodó sobre
	su espalda y Rose vio como sus garras se extendían, se enroscaban y
	después se retraían mientras bostezaba. Era un gato negro, con un pálido
	triángulo de pelo claro bajo su barbilla. Rose se agachó y le acarició
	el pelo claro y fue recompensada con un ronroneo y la intensa mirada de
	los profundos y vidriosos ojos del gato. Después de un rato se estiró de
	nuevo, después saltó del sofá y se escabulló bajo la silla donde se
había sentado Wyse de nuevo.}

{---¿Cómo se llama? ---preguntó Rose.}

{---¡Dios mío, ahí me has pillado ---sonrió Wyse---. Yo sólo lo llamo
	``el gato''. Ese gato ha estado aquí más tiempo que yo. Pero hablando de
	nombres\ldots{}}

{---Me llamo Rose. Él es el Doctor.}
