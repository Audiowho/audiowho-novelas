\chapter*{Capítulo 5}
\addcontentsline{toc}{chapter}{Capítulo 5}

--¡Oh, mira eso! --dijo el Doctor--. Vosotros los humanos tenéis un montón de recursos. En medio de toda esta nieve y hielo construís esto. 

Agitó los brazos abarcando la totalidad de la huerta. Estaban a los pies de la colina, mirando a la enorme cúpula de cristal.

--¿Te das cuenta, Theol, del logro de ingeniería que es esto? Construir un invernadero con temperatura controlada del tamaño de todo un huerto, en una colina nevada, ¡usando tecnología primitiva!



Theol se rió ante el entusiasmo del Doctor.

 --Las manzanas no saben nada mal, tampoco --dijo.



--¡Apuesto a que no! --dijo el Doctor-- ¡Vamos!



Caminó por la colina tan rápido como sus piernas se lo permitían, rociando repentinamente a Theol con la nieve.



Theol se precipitó tras él, con todos los pensamientos acerca de sus chapoteantes botas ya olvidados.

--Conozco un camino en el interior --dijo, mientras alcanzaba al Doctor un momento después. Él estaba de pie ante la puerta principal de la cúpula, estudiándola valorativamente--. Al otro lado, a la vuelta hay otra entrada y Pieter rara vez la deja cerrada con llave.



El Doctor le lanzó una mirada de soslayo. Theol no podía decir si era aprobación o desaprobación lo que podía leer en los ojos del Doctor. Tal vez no debía saberlo.

 --Aplaudo tu ingenio, Theol, pero parece como si esta vez --empujó la puerta y ésta se abrió en las bisagras de rejilla--. Pieter no hubiera cerrado la puerta de entrada, tampoco.



Theol frunció el ceño.

--Eso no es algo que haga él, Doctor. Deja la otra abierta, ya que está cerca de la casa de campo, y puede mantener un ojo en sus cosas.



Algo estaba claramente fuera de lugar, y a Theol no le gustó. El Doctor, sin embargo, no parecía particularmente desilusionado por esta noticia, y ya se había escabullido por la puerta. Parecía estar admirando los manzanos de más allá. Theol lo siguió y cerró la puerta tras de sí.



En el interior, hacía calor, pero no estaba tan caliente como esperaba Theol. Algo definitivamente no andaba bien. Su inquietud crecía.



--¡Mira este lugar! --dijo el Doctor, riendo. Dio una vuelta, abarcando todo el lugar. Avenida tras avenida de maduros y majestuosos árboles llenaban el interior de la cúpula, con las manzanas colgando como adornos de Navidad de sus frondosas ramas. En lo alto, las brillantes luces eléctricas iluminaban sobre el dosel, dando la impresión de luz diurna, a pesar del crepúsculo perpetuo del mundo exterior.



Theol tuvo que admitirlo: era impresionante, aunque algo del lustre se había perdido después de tantas visitas repetidas a conseguir manzanas con Fral. Estaba seguro de que Pieter sabía de sus robos, pero el granjero era un tipo amable, y probablemente hubiera optado por hacer la vista gorda con sus fechorías. Ellos nunca tomaban más de lo que podían comer.



Vio cómo el Doctor sacó una manzana de una rama cercana, lo frotó vigorosamente en su chaqueta, y luego le dio un mordisco. Comió con admiración, sonriendo como un loco.



--Pensé que habías dicho que robar no estaba permitido --dijo Theol.



--Ah, sí. Hmmm. Bueno, no diré nada si tu no dices nada --replicó el Doctor en un susurro conspirador. Extendió su otra mano hacia Theol, e inexplicablemente, había otra manzana en ella, brillante y de color rojo. Theol no podía averiguar cómo los juegos de manos del Doctor habían logrado producir esta segunda manzana, pero la tomó de todos modos, guardándola en el bolsillo del abrigo.



--Parece como si tu amigo Pieter todavía estuviera por aquí. O eso, o se fue con un poco de prisa --dijo el Doctor señalando un abrigo, un gorro y unos guantes que habían sido tirados al azar en un banco de madera junto a la puerta.


Arrojó el corazón de la manzana en la maleza, y a continuación, hizo bocina con las manos alrededor de su boca y gritó a todo pulmón:

 --¿Hola?



La voz del Doctor hizo eco entre las copas de los árboles, asustando a un par de pequeñas aves, que salieron de su escondite entre las hojas y revolotearon en la distancia hacia el otro extremo de la huerta, graznando ruidosamente. Por otra parte, no hubo respuesta.



--¿Hola? --Llamó el Doctor por segunda vez.



Una vez más, no hubo respuesta.



-Bueno, supongo que eso da la respuesta --dijo el Doctor.



--Tal vez está en la granja --aventuró Theol.



--Sí. Estoy seguro, Theol. En la granja. Bien --El Doctor no parecía en lo más mínimo convencido. Juntó las manos, como si se preparara para la acción--. Recuérdame de nuevo, ¿dónde viste exactamente al Hombre Verde?



Theol suspiró.

--Fuera. Por el bosque --Señaló a través de la pared de paneles de cristal hacia la línea de árboles en el borde de la granja. Sus ramas parecían irregulares y sin vida en comparación con los vibrantes y frondosos manzanos del interior de la cúpula. El Doctor se agachó y observó con los ojos entrecerrados. A Theol le resultaba dificil poder distinguir cualquier cosa entre las sombras.



Tras unos instantes, el Doctor se puso en pie.

--Bueno, no parece particularmente atractivo, ¿verdad? --dijo con forzada jovialidad--. Echemos un vistazo rápido por aquí y --Se detuvo a mitad de la frase cuando vio algo por encima de su cabeza--. Ah. Eso sí que es interesante...



--¿Qué es interes...? --Theol siguió su mirada y de repente se quedó estupefacto--. ¡Hay un agujero en el techo!



--Sí --dijo el Doctor--. Uno bastante grande --Parecía preocupado--. Bien. Venga. Será mejor que echemos un vistazo.



Se apresuraron a lo largo de la avenida de árboles. Sus botas resbalaban en el suelo blando y fangoso. Unos minutos más tarde, estaban de pie justo debajo del agujero. Gotas de lluvia les mojaban la cabeza y los hombros ya que los copos de nieve que danzaban en el aire cálido de la cúpula se trasformaban en llovizna.



--Te preocupa que aquí sea donde haya aterrizado la segunda vaina, ¿verdad? --preguntó Theol.



El Doctor parecía impresionado. Asintió con la cabeza.

--Hace bastante calor aquí para que germine --Dio una patada a unos fragmentos de vidrio roto--. Ahora, debe haber aterrizado en algún lugar... ¡aquí! --Corrió hacia donde la rama de un manzano colgaba rota y floja de la rama principal--. El árbol debe haber sufrido la peor parte del impacto.



Theol miró donde la rama se había partido, exponiendo la madera fibrosa y fresca del interior. Casi había sido arrancada por completo del tronco. Por encima, otras ramas también estaban astilladas y rotas, describiendo la trayectoria de la cápsula que había caido a la tierra.

--Llegó con un poco de fuerza --dijo--. Tal vez rebotó en la dirección opuesta.



El Doctor no respondió. Estaba ocupado inspeccionando el mantillo de alrededor, en busca de evidencias de la propia cápsula. Theol decidió poner su teoría a prueba. Retrocedió, ampliando la búsqueda. Si la vaina había rodado...



No había dado más de dos pasos antes de verla, a pocos metros detrás de donde el Doctor estaba buscando. Se acercó a la base de otro árbol, se puso sobre ella durante un momento, mirandola nerviosamente. Parecía similar a la otra vaina que ya habían encontrado en la nieve, pero ésta había florecido en un patrón en forma de estrella, dejando al descubierto su interior hueco y carnoso. Parecía como si hubiera expulsado algo.



Theol miró hacia el manzano dañado, luego otra vez a la vaina, trabajando mentalmente en los ángulos. Ciertamente, no parecía que la vaina hubiera rodado hasta ahí por si sola después de aterrizar.

--Umm, Doctor...



--¡Shhh! --dijo el Doctor-- Estoy tratando de concentrarme.



Theol le dio un momento. Consideró recoger la vaina y llevársela al Doctor, pero decidió no hacerlo.



Un minuto más tarde, el Doctor apareció a su lado.

--Oh --dijo. Sonaba un poco desanimado--. La encontraste. ¿Por qué no me lo dijiste?



Theol suspiró. No creía que valiera la pena comentar que lo había intentado.



El Doctor se puso de cuclillas delante de la vaina abierta.

--Esto no es bueno --dijo--. De hecho, yo diría que es definitivamente malo. Muy malo. Tenemos que encontrar a tu amigo Pieter.



--¿Porque la vaina está abierta? --preguntó Theol.



--Porque la vaina está abierta --confirmó el Doctor.



--¿Y eso significa...?



--Significa que alguien ya ha sido infectado. Significa que el hombre verde que viste es definitivamente el Krynoid, y significa que tenemos que detenerlo pronto antes de que comience a dispersar vainas de semillas en el resto del planeta.



--Oh --dijo Theol, sin saber qué decir. Entendió las implicaciones de lo que el Doctor había dicho: Pieter era probablemente el que había sido infectado.



Theol sintió algo rozarle en la mejilla, y levantó la mano para apartarlo a un lado. Era una rama de árbol, probablemente agitada por el viento.



Hizo una pausa por un momento. El viento. Se encontraban en un invernadero gigante. No había viento. Sintió una sombra arrastrándose sobre él, tapando la luz eléctrica superior.



Poco a poco, Theol volvió la cabeza. Intentó tragar saliva, pero tenía la boca seca. Uno de los árboles estaba inclinado. Mientras lo miraba, una de sus ramas se movió en su dirección, con sus largos y delgados dedos acariciando la parte superior de su brazo. Oyó un desgarrador crujido de madera justo detrás de él, y se dio cuenta, con horror, que muchos de ellos estaban en movimiento. Las hojas crujían mientras se acercaban.



--¡Doctor! --dijo Theol, en un susurro aterrado--. ¡Los árboles se están moviendo!



--Ah, sí --El Doctor parecía algo avergonzado--. Puede que haya olvidado mencionar eso --Poco a poco empezó a ponerse en pie, poniendo a Theol a su lado--. Cuando un Krynoid alcanza un punto determinado en su germinación, tiene la capacidad de controlar otras plantas en la misma zona. Eso significa, Theol, que hay buenas y malas noticias.



Theol se agachó para evitar el balanceo de una rama baja hacia su cabeza.



--La buena noticia es que el Krynoid obviamente todavía cerca --prosiguió el Doctor.



--¿Y la mala noticia? --jadeó Theol, intuyendo cuál podría ser la mala noticia.



--La mala noticia es que este huerto ha desarrollado de repente intenciones muy deshonrosas contra nosotros. ¡Corre!



Theol no necesitó oirlo dos veces. Se agachó bajo las ramas del manzano más cercano, tropezando y rodando sobre el camino de tierra entre las dos hileras de árboles. Se puso en pie justo cuando el Doctor aparecía tras él, esparciendo las hojas y gritando.

--¡Correeeeee!



Detrás de ellos, los árboles ya habían comenzado a cerrar filas, inclinándose sobre la ruta para bloquear su camino hasta la puerta, y el Doctor y Theol se dieron cuenta rápidamente de que no había posibilidad de escapar por donde habían venido.



--Vaya suerte --mascuyó el Doctor con la respiración entrecortada--. Atrapados en una cúpula sellada con un huerto lleno de árboles que quieren comerme. Está claro que es el karma por todas las manzanas que he disfrutado en los últimos años.



--¡Quieren comernos! --gritó Theol, luchando contra una creciente ola de pánico.



--Bueno, ahora estamos entrando en semánticas --dijo el Doctor--. Aún así, no creo que debamos esperar a averiguarlo. ¿Habías mencionado otra puerta?



Theol respiró hondo.

--Por aquí --dijo, agarrando la manga del Doctor y tirando de él en la dirección opuesta--. Está más adelante. Pasando los, um, árboles...
