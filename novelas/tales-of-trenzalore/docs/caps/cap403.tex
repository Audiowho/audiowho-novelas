\chapter*{Capítulo 3}
\addcontentsline{toc}{chapter}{Capítulo 3}

Cuando la gente del pueblo salió de sus casas un par de horas más tarde y alzaron la cara ante el dorado sol naciente que se alzaba momentáneamente por encima de las lejanas montañas, fueron recibidos no con el murmullo habitual de la caída de la nieve, sino con sonidos metálicos, golpes y lo que podrían ser palabrotas en Gallifreyan que bajaban desde la parte superior de la Torre del Reloj.

Distraídos por los sonidos, muchos volvieron sus ojos hacia la Torre antes de que el sol se pusiera de nuevo. Los que todavía estaban allí cinco minutos más tarde vieron al Doctor, agobiado bajo el peso de lo que parecía ser una mochila, trepando por la pared exterior, balancearse un momento, y luego, con un grito de ``¡Geronimo!'' lanzarse hacia el suelo.

La gente del pueblo gritó y gimió. Algunos incluso se precipitaron hacia delante con los brazos extendidos, como si pretendieran coger al Doctor mientras caía.

Pero no cayó. No durante más de unos segundos. La gente que estaba abajo oyó el zumbido inconfundible de su destornillador sónico, y volvieron a quedarse sin aliento de nuevo al ver como, con un ruido metálico y con un zumbido, alas enormes, o palas, o remos, se abrieron y comenzaron a dar vueltas sobre la cabeza del Doctor.

--'¡Uau!'--gritó el Doctor cuando su cuerpo salió disparado hacia arriba. En el minuto siguiente se abalanzó y movió por encima de la gente del pueblo llamado Navidad, como una marioneta de la que el titiritero hubiera perdido el control.

Sin embargo, con el paso del tiempo, parecía ir cogiéndole el tranquillo a la caída del artefacto, y sus maniobras se hicieron menos espasmódicas, más controladas. Se estabilizó y, al ganar confianza, empezó a hacer ochos en el cielo con elegancia y precisión, gritando con deleite. Por último bajó la velocidad, flotó un momento y, como un mensajero despeinado enviado desde lo alto, descendió a tierra.

Tan pronto como aterrizó las cuchillas se plegaron sobre sí mismas, como los pétalos de una flor en la noche, y lo rodearon los niños que le exigían saber cómo funcionaba su artilugio, o le rogaban que los llevara por el cielo. El Doctor se defendió de ellos, riéndose, mientras Fergin se abría paso con dificultad hacia él.

--¿Qué es eso, Doctor?--Fergin le dirigió una mirada no muy convencida a las plegadas ``alas''.

--Es mi molinete --contestó el Doctor orgullosamente-- ¿Te gusta?.

Fergin frunció el ceño. 

--¿Para qué sirve?

--¿Es un juguete? --preguntó uno de los niños.

El Doctor revolvió el pelo del niño.

--No, no es un juguete --adoptó una expresión seria mientras miraba las caras de los adultos que se habían congregado allí.

--Sé que muchos de vosotros estáis preocupados por vuestros seres queridos --dijo-- sé que os estáis preguntando qué les está pasando, y por qué, al levantarse esta mañana, habéis encontrado vacío su lugar en la cama.

Hubo un murmullo de consternación, y la gente intercambió miradas, algunas personas asintieron.

--¿Es otro, Doctor? --preguntó alguien.

Otro. Lo nombraban con terror y, sin embargo, con la agotada aceptación de aquellos que están bajo constante asedio.

El doctor asintió en tono de disculpa. 

--Sí, me temo que sí.

--¿Quién es esta vez? --preguntó Fergin.

--No estoy seguro todavía ... aunque tengo una desagradable sospecha --frunció el ceño y arrugó la frente--. Sean quienes sean, son furtivos. Un montón de flanes cobardes. Atacan desde el interior, infiltrándose como parásitos.

--Y, ¿cómo va a ayudarnos este … molinete? --Fergin lanzó otra mirada cautelosa a las plegadas alas, como si esperara que cobraran vida en cualquier momento y empezaran a cortar a la gente indiscriminadamente.

Los ojos del Doctor se iluminaron. 

--Ah, bueno, con esto voy a convertirme en un aguafiestas.

Algunos de los niños rieron. Con evidente alivio, uno de ellos le preguntó: 

--¿Qué es un aguafiestas?

--En este caso, un aguafiestas es alguien que mete la nariz donde no le llaman --el rostro del Doctor se convirtió en una masa de arrugas al sonreír--. Lo cual, aunque sea yo quien lo diga, es una de las cosas que mejor se me dan.



\mbox{}


Aunque sabía que era un comportamiento infantil, el Doctor obtenía un enorme placer al ver la boca abierta por el asombro y la rabia en la cara de Tomalin Pilke, o más bien, de la cosa que estaba controlando a Tomalin Pilke, mientras el le zumbaba por encima.

--No te preocupes --dijo con un gesto alegre-- sólo estoy de paso.

Hizo un pequeño ajuste en su destornillador sónico y el motor, que había construido a partir de trozos inertes y piezas de tecnología alienígena que había ido recuperando en secreto durante años, aumentó la intensidad. Ese incremento hizo que las hojas que había por encima de su cabeza zumbaran más rápido y se inclinaran en un ángulo que, aunque leve, lo levantó por encima de los árboles más altos del bosque, cuyas espinosas ramas parecían alargarse en la luz de la luna y agarrarle las piernas.

Había utilizado su destornillador como un sensor de calor para averiguar donde se habían congregado los soñadores, y se dirigía hacia allí. Sabía que no sería capaz de acercarse sigilosamente hasta ellos (el molinete era demasiado ruidoso) pero también sabía que era poco probable que los soñadores tuvieran los medios para acabar con él. Teniendo en cuenta lo que había visto, los soñadores estaban armados únicamente con herramientas de excavación.

No, su mayor problema eran los árboles. Si los soñadores estaban en una parte particularmente densa de la selva, la copa de los árboles sobre el área de excavación podría ser demasiado gruesa para que la penetrara. Pero eso era un obstáculo con el que lidiaría a su debido tiempo.

Al final no hizo falta que lidiara con ello. Aunque el bosque era el hogar de ocasionales grupos de densas coníferas y pinos, debido a las temperaturas bajo cero y la limitada cantidad de luz propias de Trenzalore, los árboles que lo poblaban estaban, en su mayor parte, enfermos, larguiruchos y ampliamente espacidos. La mayor parte de la vegetación crecía al nivel del suelo y estaba comprendida por especies de plantas que se habían adaptado a las duras condiciones: musgos, helechos y hongos. Éstos recubrían el suelo del bosque formando una gruesa alfombra que a veces hacía difícil el avanzar; como consecuencia los ciudadanos que de vez en cuando tenían razones para adentrarse en el bosque, se limitaban a seguir los caminos probados y de confianza.

La sección del bosque donde se habían reunido los soñadores, a unos tres kilómetros desde donde Tomalin Pilke montaba guardia, estaba fuera del camino aunque la poblaban unos raquíticos y desnudos árboles. Mientras el molinete giraba sobre su cabeza, el Doctor miró hacia abajo para ver a los soñadores que se habían reunido, los cuales, a su vez, se habían puesto de pie para mirarlo al tiempo que sujetaban sus herramientas de excavación. Resultaba extraño, pero todos tenían la misma expresión en sus habitualmente inexpresivos ojos, una rabia furiosa y totalmente ajena. Por un momento fue como si los rostros de la gente del pueblo se hubieran vuelto transparentes, como si pudiera ver a través de ellos al ser que se había infiltrado en ellos.

Como había hecho antes con Tomalin, levantó la mano en un alegre saludo.

--¡Hola!

Hubo un sonido como el de un escape de gas cuando los soñadores, al unísono, le enseñaron los dientes y sisearon.

--No es muy amable --murmuró, aunque a decir verdad la recepción ni le sorprendió ni le preocupó. Él ya estaba evaluando el terreno, escaneando con sus ojos el área que los soñadores habían excavado.

En la tierra había un gran cráter rectangular, más negro que el follaje que lo rodeaba, pero no podía ver lo que había dentro. 

--Malditos ojos viejos-- murmuró, e hizo un ajuste al destornillador que disminuyó la oscilación de las hojas, descendiendo lentamente hacia tierra. Mientras descendía, trazando cuidadosamente una ruta a través de los pocos árboles que se extendían hacia el cielo de la noche, los soñadores empezaron a inquietarse. Se agruparon bajo él, el más ágil saltaba y las manos se deslizaban arañando el aire como gatos enfrentados a una bola de lana colgante.

El Doctor se mantuvo fuera de su alcance. Descendió lo suficiente como para poder ver la fosa que habían cavado.

Algo blanco brillaba en el oscuro suelo. Un esqueleto humano con trapos aferrados a sus moteados huesos. Por su aspecto podría decirse que llevaba allí muchos años. ¿Qué interés podía tener un cadáver de hace mucho tiempo para un invasor alienígena?

Sin embargo, el Doctor estaba más preocupado que perplejo. Por sus viajes él sabía que los restos mortales podían ser, a veces, mucho más de lo que parecían. Se había encontrado con muchos cadáveres (y unos cuantos cráneos) a lo largo de los siglos, que habían resultado ser receptáculos de poderosas energías extraterrestres.

--¿Quién es vuestro amigo? --gritó, aunque sin esperar realmente una respuesta.

No consiguió ninguna. Los soñadores se limitaron a continuar silbando y a tratar de agarrarlo.

El Doctor consideró tratar de hablarles para romper la influencia extraterrestre que los controlaba, advertirles de que los esqueletos enterrados en bosques misteriosos nunca eran un buen presagio.

Pero no lo hizo, no tenía ningún sentido. En su lugar habló directamente, y en tono burlón, a lo que se escondía detrás de sus ojos.

--Todavía estás demasiado asustado como para enfrentarte conmigo, ¿verdad? ¿Prefieres esconderte en el bosque, jugando con tus marionetas y perdiendo el tiempo con viejos huesos? --se encogió de hombros--. La verdad, estoy decepcionado. De hecho, eres un enemigo tan decepcionante que casi me avergüenzo de estar hablando contigo. 

Miró la hora en su reloj. 

--Y, ¡ooh! mira, es casi la hora del té con galletas. No me lo pierdo por nada del mundo, así que cuando decidas salir de tu escondite ven a buscarme y hablaremos, pero te lo advierto, puede ser que todas las galletas de mermelada estén...

En ese momento le golpeó la primera piedra.

El Doctor no vio quién se la había tirado, simplemente había subido y le había golpeado en la pierna.

Afortunadamente dio en su pata de palo, y no hizo más que un fuerte ruido antes de rebotar.

--¡Oye! --dijo indignado--. Ten cuidado, vas a estropear el barniz.

Otra piedra pasó zumbando hacia él, y esta vez hizo una apresurada maniobra en pleno vuelo para evitarla, lo que le hizo balancearse. Esta segunda piedra pasó tan cerca que sintió el aire cuando la misma pasó cerca de su nariz.

Decidió que era hora de irse, sabiendo que si una piedra le daba en la cabeza y lo dejaba inconsciente se precipitaría hacia el suelo. Por desgracia, el molinete no era tan ágil. Antes de ascender el Doctor tuvo que estabilizar el movimiento del aparato, que se balanceaba, lo cual requería varios complicados ajustes a realizar con el destornillador... y para cuando se puso a ello más misiles (no sólo piedras, sino espadas, azadones y picos) volaban hacia él.

La mayor parte fallaban, o no llegaban a darle por poco, pero no todas. Hubo un ruido sordo cuando otro proyectil golpeó la pierna de madera. Y a continuación, una roca o un trozo de madera rebotaron en su, afortunadamente muy acolchado, hombro, provocando un ``uf'' de dolor.

Aún así, estas sólo eran heridas menores, se preocupó mucho más cuando un azadón, lanzado como una jabalina, le pasó por encima y le dio a las alas que hacían girar el molinete.

Con un crujido horrible, las alas desaceleraron, se sacudieron, y se atascaron. 

--¡Whoa! --gritó el Doctor al sentir que caía en picado. Un segundo después, las hojas trituraban y escupían los trozos de madera y metal, y el Doctor volvió a subir como si fuera una goma elástica. Trató desesperadamente de estabilizar la máquina con el destornillador y, aunque lo consiguió, estaba claro que el molinete estaba dañado. Hacía un ruido alarmante, entre un ruido de molino metálico y un débil puf-puf-puf, y aunque seguía subiendo y moviéndose en la dirección en la que, más o menos, quería ir, no podía evitar sentir que cojeaba por el aire.

--Vamos, pequeño --murmuró con el mismo tono persuasivo que hacía tiempo había utilizado con su TARDIS-- ¡Puedes hacerlo!

Apretó fuertemente el destornillador, presionando su cabeza verde con forma de burbuja a la hebilla del grueso cinturón circular en el que se alojaba el motor. El molinente respondió lentamente, titubeando, abriéndose camino a través de las copas de los árboles, volviendo hacia las luces parpadeantes de Navidad.

Cuando dejó atrás la hilera de árboles, el Doctor respiró con más calma. Aunque el molinete todavía reducía velocidad y perdía altura, a partir de aquí se trataba simplemente de lograr abrirse camino entre los pocos campos y pistas locales que había entre el bosque y la ciudad. La máquina había descendido a no más de seis metros sobre el suelo, y resoplaba como un coche en la reserva cuando, con un chirrido metálico, dejó de funcionar.

A pesar de que no se estrelló, el Doctor hizo un descenso más rápido de lo que le hubiera gustado. Preparó sus viejos huesos para el impacto mientras el suelo se le acercaba, y un segundo después se tambaleaba, giró y quedó tendido en el suelo mientras las ligeras palas unidas al rotor se fragmentaban a su espalda.

Finalmente se detuvo y quedó tendido, de cara al camino por el que había venido, sobre lo que quedaba de la mochila, que había en cierta medida había amortiguado su caída. Le dolían varias partes del cuerpo, pero creía que no estaba especialmente malherido. Aún así, tendría que limpiar un poco los pantalones y el abrigo cuando volviera.

Se sentó con un gruñido y de inmediato se dio cuenta de que algo se movía hacia él por la pista. Aunque estaba oscuro,venía rápidamente y con una clara intención. El Doctor giró su destornillador y lo usó como una linterna.

A la luz verde del destornillador vio el rostro ávido de Tomalin aproximándose hacia él. Los ojos del joven se habían convertido en orbes amarillos con rendijas negras en lugar de las pupilas, sus labios habían retrocedido y sus dientes estaban al descubierto, goteando saliva. Agarraba el rastrillo con tanta fuerza que tenía los nudillos tan blancos como el hueso.

Utilizando toda la autoridad que podía sacar de su maltrecho cuerpo, el Doctor rugió 

--¡Alto!

Pero Tomalin no se detuvo, y siguió aproximándose. El Doctor trató de ponerse en pie mientras el hombre poseído levantaba el rastrillo por encima de la cabeza para asestar el golpe mortal.

Entonces, detrás del Doctor, una voz resonó en la oscuridad:

--Ya has oído al Doctor, muchacho. ¡Retírate!

De pronto unas manos alzaban al Doctor mientras otros cuerpos pasaban junto a él para enfrentarse a Tomalin. El doctor miró la cara redonda y bigotuda de Fergin Eggleton.

--He reunido a unas cuantas personas. Pensaba que necesitarías algo de ayuda.

El Doctor le dio las gracias con un gesto y se volvió hacia Tomalin, al que habían detenido una pandilla de aldeanos cargados con una selección de armas improvisadas.

--No le hagáis daño --dijo el Doctor-- No es responsable de sus acciones.

Tomalin giró la cabeza con brusquedad y fijó los ojos en el Doctor. Sus ojos todavía estaban amarillos.

--Tomalin --dijo el Doctor en un tono más o menos amigable-- ¿Por qué no nos muestras tu brazo?

Tomalin retrocedió como si le hubieran picado mientras le siseaba al Doctor.

--¿Qué pasa con su brazo, Doctor? --preguntó Fergin.

--Creo que lleva una marca --El Doctor elevó la voz--. ¿Tengo razón, Tomalin?

En lugar de responder, el joven siseó una vez más, dio media vuelta y huyó internándose en la oscuridad.

Un par de aldeanos se disponía a darle caza, pero el Doctor levantó una mano.

--Dejad que se vaya.
