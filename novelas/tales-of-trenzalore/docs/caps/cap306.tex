\chapter*{Capítulo 6}
\addcontentsline{toc}{chapter}{Capítulo 6}

Una hora más tarde, con el viento nocturno del sur aullando con toda su fuerza, la tripulación ya no estaba tan entusiasmada en cuanto a desembarcar del Salvavidas.
 
Quizás era comprensible. Aunque todos eran jóvenes y rebosaban energía, y a pesar de que estaban bien equipados y tenían a Yoshua para guiarlos por el estrecho camino de la Ruta de la Cabra, todavía les quedaban dos o tres días extremadamente difíciles por delante. Y por supuesto, cuando finalmente regresaran a Navidad, ninguno de ellos sabía lo que se iban a encontrar allí.
 
El Doctor estaba ocupado al timón mientras el Salvavidas avanzaba a trompicones hacia el oeste por el extremo sur de la Cresta de Fafnir, con su chaqueta ondeando al viento. El refugio que les proporcionaba la cordillera redujo la cantidad de viento de forma significativa, así que viajaron como unas dieciocho millas hasta que, uno por uno, los hombres, ahora enrollados con tanta piel y lana como pudieron, tiraron sus esquís y sus equipos de supervivencia por la borda, y saltaron inmediatamente después, con las piernas pegadas contra sus pechos para que cayeran y rodaran por la nieve con un impacto mínimo. Todos parecían indemnes, poniéndose de pie, y corriendo para recuperar su equipo.
 
Caleb fue el último, pero se giró cuando estuvo a punto de saltar.
 
--¿Estás seguro que puedes manejar esta nave solo, Doctor? --gritó.
 
El Doctor estaba agarrado con las dos manos al timón, cuya manipulación le pareció más difícil de lo que había esperado. Este simple elemento controlaba fuerzas enormes, observó. Vibraciones profundas que resonaban a través de los complejos mecanismos que había bajo la cubierta: los ganchos, las poleas, las barras y las cuerdas del timón, por no mencionar el astil de dirección al que el eje frontal de los esquíes estaba conectado.
 
--¡Estaré bien! --replicó.
 
--Creo que te conozco lo bastante bien, Doctor --respondió Caleb--. No nos harías caminar de vuelta a casa si la ruta que hubieras elegido no fuera más peligrosa aún.
 
--¡Sólo llegad a la ciudad, Caleb! Olvídaos de mí, olvidaos del Salvavidas... y ni se os ocurra tratar de llegar por la Cresta a los Codos del Diablo. Evitad ese lugar como la peste, ¿me oyes?
 
--Sin duda, es mejor que luchemos contra los Autons aquí que en la ciudad.
 
--Si todo va según lo previsto, no tendremos que luchar... pero incluso los mejores planes pueden fallar si la gente no se adhiere a ellos. ¡Ahora vete!
 
Caleb se fue, dejando caer su mochila y sus esquís a un lado antes de desaparecer entre una estela de copos de nieve.
 
El Doctor se concentró en el paisaje blanco y brillante que le quedaba por delante, aunque su extremo derecho permanecía oscuro bajo la sombra de la Cresta. Esperaba poder ver la entrada a los Codos del Diablo cuando llegara por fin allí, para lo cual, según los cálculos de Caleb, tardaría alrededor de veinte minutos. Una vez en el lugar, seguiría depositando más fe en la suerte que en su habilidad para atravesar el angosto pasillo en forma de Z, aunque era un consuelo que sólo tuviera que detenerse a mitad de camino, en el Degolladero, donde casi seguro que los que lo esperasen para emboscarlo intentarían subir a bordo.
 
El Doctor sonrió sombríamente.
 
En cuanto se dio cuenta de contra quién estaba luchando, entendió su plan.
 
El planeta Trenzalore no tenía ningún valor para el Nestene. Era una naturaleza virgen, casi sin contaminación alguna. No había toxinas o productos químicos en el aire, ni ácidos, ni metales, ni humo. Por ende, el Nestene no tenía un interés real en Navidad o en la gente que viviera allí. Este escuadrón Auton estaba aquí por la misma razón que los innumerables intrusos que habían intentado entrar en el asentamiento durante los últimos tres siglos.
 
Por el Doctor.
 
Al principio, el Nestene planeó que sus descerebrados Autons se tiraran en paracaídas sobre Navidad y simplemente agarraran a su objetivo y lo mataran. Con los Autons prácticamente invencibles contra las armas fabricadas a partir de herramientas agrícolas, no hubiera habido nada que se hubiera podido hacer. Pero cuando los furiosos vientos cruzados los desviaron de su curso, el Nestene, siempre capaz de improvisar, se inventó otro plan al darse cuenta de que el Doctor intentaría descubrirlos y puso una elaborada trampa.
 
Sus pensamientos se dispersaron cuando se aventuró en el cañón, un procedimiento más sencillo de lo que había anticipado, mientras la nave maravillosamente diseñada se deslizaba con gracia por la brecha, y la vela mayor se sacudía alrededor de su cuello de cisne y capturaba todo el impacto del viento del sur, el cual se canalizaba e impulsaba el barco con una velocidad todavía mayor.
 
--Eres una nave preciosa --dijo el Doctor, dándole cariñosamente palmaditas a los radios barnizados del timón--. Lo siento muuucho.
 
Durante un breve instante, las anchas laderas de cañón se iluminaron con el doble de brillo que Soror y Frater juntos, mientras que los pinos de los ascendientes terraplenes no parecían más que conos de nieve congelada. Estaría disfrutando más de la escena si no estuviera tan nervioso de lo que le esperaba delante. Entonces la velocidad del Salvavidas se redujo y el viento del cañón comenzó a disiparse sobre los flancos de las colinas circundantes. Para cuando llegaran al Degolladero, a unos diez minutos del primer ``codo'', tendrían que aminorar la marcha. Ahí tenía que ser el punto donde los agresores vendrían a atacar, era el único lugar entre la ciudad y la tundra donde unos bípedos del tamaño de hombres no tendrían ninguna posibilidad de invadir la nave.
 
--Bueno... veamos cómo lo hacéis, chicos --dijo el Doctor--. Sobre todo, veamos cómo lo hago yo, porque tengo que llegar primero.
 
Al rozar la primera curva extrema del cañón, el Salvavidas se deslizó brevemente sin control, virando hacia babor en el mismo momento en el que el Doctor se aferró al timón y los esquíes delanteros cambiaron de dirección, antes de que se enderezara y se volviera a despegar, por el canal más interior, y los flancos del valle se cerraran sobre el barco, amortiguando toda clase de sonido. Las paredes invasoras del valle se convirtieron en rostros de roca maciza, hendidos con las fisuras por las que transcurría el hielo y sujetados por pinos recubiertos de nieve blanda. La luz de las dos lunas fue progresivamente disminuyendo, creando una atmósfera oscura y tunelesca, pese a haber todavía suficiente pasillo para que la vela mayor condujera a la nave unas cuantas millas más.
 
Puede que fuera estrecho, pero a partir de aquí, durante algo más de un kilómetro, era completamente recto. Aun así, el Doctor sabía que tendría que trabajar con rapidez.
 
--Vale, ve constante... --dijo, bloqueando la rueda, cogiendo su bastón, bajando del puente a la escotilla delantera, camino que descendió torpemente, y cerrando con fuerza las puertas de esta al pasar.
 
Se estaba mucho mejor en la bodega, donde había varias literas envueltas en mantas y la estufa burbujeaba calor. Había un rico aroma a café, el cual desafortunadamente no tenía tiempo de probar ahora. La puerta a su izquierda daba a la Habitación de Tercera Clase, una cavidad enorme y en forma de campana hecha de hierro fundido solidificado, en cuyo medio descendía el eje central hasta conectarse con la compleja masa de varillas, resortes y otros ejes que formaban el tren de aterrizaje del barco, y del cual provenía ahora un ruido de fricción con el hielo y un constante surtidor de nieve recién batida, la cual se adhería a las paredes interiores y caló instantáneamente al Doctor de los pies a la cabeza.
 
Se abrió paso hasta el final de la plataforma llena de cables, desde donde podría alcanzar la parte superior del eje. Pese a tener las manoplas puestas, los dedos del Doctor estaban demasiado entumecidos como para llevar a cabo la delicada operación en cuestión. Tenía que procurar tener especial cuidado al coger el tubo de boronita del bolsillo interior de su chaqueta, sólo había uno, y sería un desastre que se cayese al suelo. Era una lástima que no tuviera más, reflexionó mientras insertaba el extremo de la mecha en la base maleable. Pero incluso aunque tuviera media docena, pese a ser inmensamente potente, lo bastante potente como para demoler un bloque entero de pisos en la Tierra, sería demasiado poderoso para obtener el efecto deseado. No obstante, si detonaba un único explosivo, el hierro en forma de campana de la sala contendría gran parte de la explosión y la dirigiría hacia abajo.
 
Sólo así tendría el éxito asegurado, o al menos eso esperaba.
 
Se inclinó desde la plataforma, colocó el explosivo en el eje de metal y ató la mecha alrededor hasta que quedó fija en su lugar. A continuación, enrolló el otro extremo de la mecha alrededor de la punta de su bastón y estirándose, consiguió meterlo por el estrecho hueco circular de encima del eje. Cuando retiró el bastón, la mecha se despegó como un calcetín y se quedó enredada en el hueco de arriba.
 
Miró al reloj: estaban a punto de llegar al Degolladero. A toda prisa, saltó directamente a la bodega, donde se puso a darle a las alfombras y mantas de las literas la forma de una persona y a cubrirlas de edredones.
 
¿Se lo tragarian los Autons?
 
--Son cacharros de plástico sin cerebro... ¡claro que se lo tragarán! --se respondió.
 
Pero siempre había ese elemento de duda. El control del Nestene no era tan débil. Como solía ocurrir antaño, el Doctor se dio cuenta de que estaba improvisando, aprovechando una oportunidad. Pero en realidad, esta era la única oportunidad, no sólo para él sino para toda la población de Navidad. Si el Nestene lo derrotaba, sus brutales soldados invadirían la ciudad y matarían a cualquier hombre, mujer y niño que se entrometiera en su camino.
 
Ninguna resistencia es fútil, le había dicho a Luca. Pero a veces no se podía evitar lo inevitable, no por mucho tiempo.
 
Podía sentir cómo el Salvavidas iba progresivamente decelerando. El barco comenzó a sacudirse mientras atravesaba obstáculos en la nieve que de no haber estado éste hubiera pasado por encima sin problemas. Estaba claro que habían llegado al Degolladero. Visualizó la densa vegetación cubierta de nieve y enmarañada de encima. Afortunadamente, el Salvavidas siguió hacia adelante, deslizándose en lugar de patinar, pero sin detenerse.
 
Y haciéndose un blanco irresistible.
 
--Tiempo, chicos --dijo, dirigiéndose a los pies de la escalera--. Tiempo.
 
Se los imaginó saltando desde arriba, uno detrás de otro, como comandos, aterrizando entre ramitas y hojarasca y duchas de nieve polvo.
 
Y justo sonaron los primeros impactos, los golpes de unos pesados cuerpos desangrados al caer sobre la cubierta.
