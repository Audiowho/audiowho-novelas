Lo estaban estrangulando. Notaba algo alrededor del cuello que cada vez le apretaba más y más. Pero cuando se llevó las manos a la garganta, no encontró nada. Sólo los dedos que arañaban su propia piel.

¿Dónde estaba? En un lugar oscuro, eso era todo lo que sabía. La oscuridad tenía una textura, como la del aceite. Le dio la sensación de que si se movía rápido, esta se deslizaría entre sus carnes, fría y espesa.

No es que pudiere moverse rápido. De hecho, apenas era capaz de moverse. Su cabeza daba vueltas como un bombo; le ardían los ojos; sus pulmones luchaban por obtener aire.

Se volvió a agarrar el cuello, y luego se dejó caer de rodillas. Su instinto le decía que este lugar era horrible. Un lugar para el miedo y la locura y la crueldad y la dolorosa pérdida eterna.

¿Alguien le había soplado en la nuca? ¿O tal vez le había rozado con la punta de los dedos? Ayudadme, trató de decir, pero de su garganta oprimida no salió ni un graznido.

La cabeza le daba vueltas. Pronto se quedaría inconsciente. La oscuridad de fuera se estaba filtrando a través de su piel, arrasando con sus pensamientos…

Entonces, de repente, la presión que le oprimía la garganta se desvaneció. Fue como si algo se le desenroscara del cuello. Cayó hacia adelante, inspirando aire con la boca, atragantándose, brusca – y maravillosamente – capaz de respirar una vez más.

--Qué desagradable es que te recuerden lo frágil que eres, ¿a que sí? ¿La facilidad con la que se te puede reducir a la nada?

La voz era fría, rencorosa, y sonaba como un tintineo desagradable, como si las palabras que dijera vinieran acompañadas por el leve sonido del cristal al estallar.

Se masajeó la garganta. Cauteloso y todavía de rodillas, viró la cabeza.

Allí se encontró con un hombre. De tez clara. Sepulcral. Y de ojos ardientes.

Iba vestido completamente de negro, como un director de funeraria.

--¿Quién eres?

La pregunta salió como un ronco susurro, pero el hombre de cara blanca dio una zancada como una araña de largas patas para colocarse a su lado.

--Lo importante no es quien soy yo, sino quién eres tú. ¿Cómo te llamas?

--Aliganza. --Intentó aclararse la garganta--. Aliganza Torp.

El hombre de rostro blanco enarcó las cejas.

--¿Estás seguro de eso?

Aliganza vaciló. Ya no sabía decir con seguridad si estaba seguro de algo.

--Pues… Sí.

--Interesante --dijo el hombre de rostro blanco.

--¿El qué?

--Tu nombre. O sea, así no es cómo te llama él, ¿verdad?

--¿Quién?

--¡Quién va a ser!

La rabia repentina del rostro pálido se intensificó, y Aliganza se asustó. Los ojos del hombre brillaban. Viciosamente soltó:

--¡Él! Él a quien obedecen. El hombre de la torre. El sabio de Navidad.

--Te refieres a… ¿el Doctor?

--El Doctor.

La expresión del hombre se plegó, y sólo durante un segundo, Aliganza vislumbró lo que había debajo. Algo espantoso. Entonces la imagen se desvaneció antes de que quedara grabada en su mente.

Cuando el hombre volvió a hablar, su voz salió suave, su boca se transformó en una sonrisa que parecía ser más larga de lo que debiera.

--¿Cómo te llama? --preguntó--. O debería decir… ¿cómo te llamaba? ¿Cuándo eras joven?

Aliganza sonrió ante el recuerdo.

--Me llamaba Barnable. Barnable el cuadragésimo tercero.

--¿Y por qué?

--Porque… le gustaba. --A pesar de la situación, Aliganza se llenó de orgullo--. A todos los niños que le gustaban, a todos sus favoritos, los llamaba…

Su voz se desvaneció. El rostro pálido estaba sacudiendo la cabeza.

--No. No te llamaba Barnable porque le gustaras. Te llamaba Barnable porque no se acordaba de tu nombre. ¿No lo ves? No eres nada para él. Un insecto. Un mero parpadeo ante sus ojos. Ya se ha olvidado de ti. Es anciano, y frío, y no le importa. Todo le importa un rábano excepto su propio orgullo, su propia gloria.

--No --murmuró Aliganza--. No lo conoces. Es un buen hombre. Un gran hombre. Es amable y…

--¿Cuál es su nombre? --preguntó el rostro pálido.

--¿Qué?

--¿Su nombre? ¿Cuál es?

--Pues es… es el Doctor.

El hombre del rostro pálido soltó un bufido.

--Ese no es un nombre. Es un título. ¿No lo ves? Piensa tan poco en vosotros que ni siquiera se ha molestado en contaros su nombre. Y por eso es por lo que estáis todos en peligro, ¿no? ¿Por eso vivís constantemente aterrados? Porque es demasiado grande, demasiado importante, para contaros su nombre. Tanta soberbia. Tanta arrogancia.

Aliganza sacudió la cabeza.

--No… no es cierto.

--¿Entonces qué es cierto?

Pero a Aliganza no se le ocurrió nada que decir. La voz del hombre, sus heladoras palabras, estaban jugando con sus pensamientos como la bruma alrededor de las lápidas del cementerio.

¿Y no tenían sentido? ¿No le estaban desvelando una verdad que debería haber descubierto – que todos deberían haber descubierto – hacía generaciones?

--Tengo algo para ti --dijo el rostro pálido.

--¿El qué?

--Un regalo.

--Extiende la mano.

Aliganza vaciló. El hombre sonrió. Su boca era amplia, y roja, y estaba empapada.

--No te voy a hacer daño --dijo melosamente--. Extiende la mano.

Aliganza obedeció.

\mbox{}

\centerline{ \Huge *}

\mbox{}


--¡Padre!

El niño saltó de la cama en cuanto el hombre pegó un chillido. Durante un instante se quedó aterrorizado. Su padre, normalmente amable, parecía un monstruo.

Pericles Trop sabía de monstruos. No paraban de venir a Navidad. Venían para matar, para destruir, y aunque daban miedo, como el Doctor siempre los derrotaba, todo acababa bien al final.

Sin embargo, Pericles sabía que a veces el Doctor no podía salvar a todos. Cuando Pericles era un bebé, unos monstruos llamados Krotones invadieron Navidad, y mucha gente murió cuando los invasores destruyeron el silo donde se estaban refugiando.

El Doctor se enfadaba y se entristecía cuando la gente moría, e iba a donde las familias y se disculpaba como si él tuviera la culpa, pero nadie lo culpaba por ello. Sabían que si no fuera por él, habría muerto más gente. Sabía que si no fuera por él, ya habrían destruido Navidad más de mil veces.

La gente decía que el Doctor tenía cientos de años. Cientos y cientos. Pericles no sabía si eso era cierto, pero sí sabía que el Doctor era anciano, y que pasaba cada vez más y más tiempo en la Torre del Reloj, vigilando la grieta rara de la pared. Y sabía que el pueblo de Navidad estaba preocupado. Preocupado de qué pasaría si el Doctor…

--¿Qué quieres?

Durante un instante, la voz del padre de Pericles no sonó como la suya. Sonó fría y dentada y gélida. Y sus ojos en la penumbra le parecieron extraños, como si brillaran con una luz amarilla.

De forma instintiva, Pericles apuntó la linterna que colgaba de su mano hacia el rostro de su padre. Su padre se encogió y gruñó:

--¿Qué estás haciendo? ¡Aparta eso! --Pero no pasaba nada, porque la luz reveló que sus ojos estaban bien, que Pericles se había equivocado.

--Lo siento, Padre --dijo Pericles--. Te oí gritar, así que vine para ver qué pasaba. ¿Estabas… teniendo una pesadilla?

Aliganza Torp se quedó mirando a su hijo durante un momento, y luego sus rasgos se suavizaron un poco.

--¿Una pesadilla? --murmuró--. Sí… debe de haber sido eso.

--¿Ya pasó?

Aliganza asintió.

--Sí, ya pasó. Ahora vuelve a la cama. Necesitas dormir. Mañana hay cole.

Pericles bajó la linterna. Asintió aliviado.

--Sí, Padre. Buenas noches, Padre.

La voz de Aliganza sonó como un murmullo desde las sombras que ocultaban su rostro barbudo.

--Buenas noches, hijo.



\mbox{}

\centerline{ \Huge *}

\mbox{}



Cuando se aseguró de que Pericles se había dormido, Aliganza caminó a tientas hasta el baño. Al lado del espejo encima de la palangana había una lámpara de aceite. La encendió y miró a su reflejo.

Sus ojos no eran los suyos. Eran amarillos, y las pupilas no eran más que franjas negras y verticales. Retrocedió conteniendo la respiración. Parpadeó y volvió a mirar. Entonces dejó escapar el aire con un resuello de alivio.

Se había equivocado. Sus ojos eran los suyos. La imagen momentánea debe de haber sido una consecuencia de su sueño, una que se quedó atrapada en su mente, como una astilla de hielo.

Le entró un escalofrío. Hoy hacía frío. Más frío de lo habitual. Pero aun así su mano estaba caliente. Y palpitaba. La levantó en el aire, bajo la luz.

La piel de su antebrazo derecho estaba enrojecida e hinchada y llena de ronchas. Alguna especie de sarpullido. Se la rascó, pero eso sólo provocó que su extraño hormigueo se le extendiera por las venas y le trepara por el brazo y por el hombro y por el cuello, hasta inundar las cuencas de sus ojos y la caverna de su mente…

Oyó una fría y plateada carcajada. ¿De dónde venía?

Se volvió a mirar al espejo.

--Soy Aliganza Torp --susurró.

Volvió a la cama.

