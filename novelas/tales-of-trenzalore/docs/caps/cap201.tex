\part*{An Apple a Day... \\ \vspace{2 mm} {\large George mann}}
\addcontentsline{toc}{part}{An Apple a Day}

\chapter*{Capítulo 1}
\addcontentsline{toc}{chapter}{Capítulo 1}

En toda Navidad, la nieve era profunda, nítida y uniforme.



Aquí, en las afueras de la ciudad, se había formado una gruesa corteza que cubría el paisaje, dándole un aspecto tranquilo, prístino. Copos de nieve descendían de los cielos como azúcar glass espolvoreado sobre el mundo, empolvando los techos de los edificios, ocultando los caminos.



A Pieter apenas le importaba. Él nunca había conocido nada más que la nieve y, a pesar del intenso frío, se sentía alegre y optimista. Se acercaba el Festival de la Cosecha. Él amaba la cosecha, no sólo porque representaba el fin de un arduo año de trabajo (el fruto de su trabajo, literalmente), sino también por la manera en que la gente del pueblo se reunía para compartir las celebraciones.



El festival duraría dos días durante los cuales la gente del pueblo erigiría efigies del Hombre Verde en la plaza del pueblo, tejidas laboriosamente a partir de hebras de hiedra. La gente bailaría hasta que sus piernas cedieran, jugarían a juegos desenfrenados con los niños y, lo que más le gustaba a Pieter, se sentarían a escuchar historias contadas por el Doctor..



Cada año, desde que Pieter era un niño, el Doctor se sentaba en las escaleras de la Torre del Reloj en las horas finales de la fiesta y les obsequiaba con cuentos de sus aventuras entre las estrellas. Había algo embriagador en estas historias de otros mundos lejanos y, para Pieter, que apenas había salido de los límites de la ciudad en sus treinta y siete años, aquellas historias resultaban exóticas, extravagantes y maravillosas. Por supuesto, no dudaba de que fueran verdad (el Campo de la Verdad hacía que nadie en la ciudad, el Doctor incluido, pudiera contar mentiras) Y así, la gente del pueblo se reunía, con humeantes tazas de sidra caliente apretadas entre sus manos, para escuchar al Doctor contar sus historias.



Pieter se preguntaba acerca de qué extrañas criaturas podrían escuchar este año (más de los Krotons, los Nimon, los Squall, ¿o algo nuevo y aún más extraño?) Desde luego, así lo esperaba.



Consideraba esto mientras caminaba a través de la nieve hacia el huerto. Aún era temprano, pero ya era hora de trabajar. Si tenía suerte, el Hombre Verde lo habría bendecido este año y sus cultivos serían abundantes y estarían listos para la cosecha. Llamaría a los chavales del pueblo para que vinieran y ayudaran con la cosecha, y para mañana y el inicio de la fiesta, el trabajo estaría hecho.



Dobló un recodo en el camino, ahora apenas perceptible debido a la nieve, pero marcada por un brasero humeante, y allí, sentado en cuclillas en la colina, estaba la cúpula de cristal del huerto. Parecía una joya brillante, acurrucada entre los ventisqueros, llena de luz y calor. Sintió una oleada de orgullo. Había heredado esta parcela de su padre diez años antes, y la había cuidado bien, cada año le daba un mayor y mejor rendimiento que el anterior. Este año, esperaba, no sería la excepción. Si las cosas iban según lo previsto, tal vez incluso podría ser capaz de pagar las reparaciones que necesitaba hacer a la casa de campo.


Caminó a lo largo del camino, con las botas levantando terrones de nieve. A medida que se acercaba, sin embargo, se dio cuenta de que algo había manchado el generalmente liso exterior del invernadero. En la mayor parte de su superficie, los copos de nieve golpeaban contra el vidrio caliente e inmediatamente se convertían en gotas de agua, cayendo por los lados de la cúpula para congelarse de nuevo alrededor de la base. Pero en el techo, el vapor estaba formando una tenue espiral, como si estuviera saliendo del pico de una tetera hirviendo.



El calor se escapaba en el frío aire. Tenía un agujero. Algo había roto el cristal reforzado de la cúpula. Si no se daba prisa, toda su cosecha se arruinaría. Peor; si la nieve se metía dentro, sus propios árboles podrían morir congelados.



Presa del pánico, Pieter tropezó apresuradamente el resto del camino hasta la cúpula, buscó la llave en su bolsillo y abrió la puerta. Entró, cerrando la puerta tras de si. Para su alivio, el interior aún estaba cálido, y sus temores iniciales parecieron infundados. Las manzanas en los árboles parecía saludables y en buen estado a pesar de las heladas. ¿Pero que podría haber causado un agujero de ese tipo?



Despojándose del sombrero, el abrigo y los guantes y arrojándolos sobre un banco de madera junto a la puerta, Pieter se apresuró a lo largo de una avenida de altos árboles, con el rostro vuelto hacia arriba, mientras buscaba la brecha en el techo.



No fue difícil de encontrar. A unos cientos de metros en la cúpula se hizo evidente que algo había golpeado el cristal con una fuerza razonable. El agujero era del tamaño de la cabeza de un hombre, irregular y desigual, y las fracturas finas trazaban telas de araña en los paneles circundantes. Remolinos de nieve bailaban alrededor de la herida, volviendo a derretirse, la lluvia repiqueteaba mientras los copos de nieve pasaban al cálido interior de la cúpula.



Pieter sintió crujir los cristales rotos bajo sus botas, y miró hacia abajo. Los fragmentos se extendían por todo el mantillo, brillando como agua derramada en la luz artificial. Y allí, a pocos metros de distancia, estaba lo que debía de ser el culpable, una vaina esférica verde, aproximadamente del doble del tamaño de su puño, situada junto a la base de un árbol. Se acercó a ella con cautela. Parecía orgánica, a pesar de estar cubierta de una gruesa piel escamosa. La piel exterior estaba rota y arrugada, mostrando un interior carnoso y suave. Parecía una semilla, pero si lo era, no se parecía a cualquier semilla que hubiera visto antes, y era mucho más grande.



¿Cómo había caído a través de la azotea de su huerto? Alguien la tenía que haber tirado. Era la única explicación. Uno de los otros agricultores, envidioso del éxito de Pieter e intentando sabotearlo, o bien uno de los chicos de la ciudad tratando de impresionar a sus amigos.



Bueno, él no iba a dejar que eso lo parase. Decidiría sobre la vaina, o lo que fuera, y arreglaría el agujero en su techo. Todavía estaría listo para el festival, si se centraba en ello. Era frustrante, pero no tenía miedo de un poco de trabajo duro.



Pieter se agachó y recogió la vaina con ambas manos. Era pesada y caliente al tacto. De hecho, ahora que la tenía en la mano, podía sentirla temblando ligeramente, como si algo dentro de ella se moviese. Observó, fascinado, como la piel carnosa comenzaba a pelarse como los pétalos de una flor que se abre en primavera.



Por un momento, no pasó nada. Pieter se dio cuenta de que estaba conteniendo el aliento y lo dejó escapar justo mientras una gruesa enredadera verde salía del interior de la vaina y golpeaba la palma de su mano.



Lanzó un grito de sorpresa y dolor, dejando caer la vaina al suelo. Echó un vistazo a la parte de atrás de la mano. Estaba sangrando. ¡Esa cosa, fuera lo que fuese, le había atacado! Se tambaleó hacia atrás, sintiéndose de repente mareado. ¿Lo habría envenenado?



Todo sucedió repentinamente, y ahora no podía pensar con claridad. Sus pensamientos eran como melaza; lentos y pesados. Pidió ayuda, sabiendo en el fondo que no serviría de nada, ya que no había nadie cerca para escucharlo, y entonces el mundo cambió de repente y se volvió negro.

