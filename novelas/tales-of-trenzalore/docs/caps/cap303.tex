\chapter*{Capítulo 3}
\addcontentsline{toc}{chapter}{Capítulo 3}

El Salvavidas de Trenzalore había existido mucho antes de que él Doctor hubiese fijado su residencia en Navidad. Tan arduas eran las condiciones en las Tierras Lejanas que no lo sacaban del granero más de lo estrictamente necesario, pero bastaba para el trabajo porque era un barco muy bien construido.
 
Más que una versión en miniatura de uno de aquellos anticuados bergantines de vela de la Tierra, el Salvavidas media veinte metros de eslora, con una cubierta superior, dos mástiles de aparejo y un foque en la parte delantera, adornado con velas, que su tripulación manejaría constantemente ascendiendo los aparejos colgados como telas de araña. La bodega era lo suficientemente amplia para almacenar cualquier cantidad de madera y maleza, por no hablar de pieles y cuerpos de animales, cuya captura era el propósito habitual del Salvavidas, aunque también salía en casos de emergencia. Además, la bodega también servía de camarote para la tripulación, que necesitaba calor regular y refugio en los viajes largos. Por supuesto, el Salvavidas no era un bote salvavidas como los de la Tierra. No había agua superficial en Trenzalore que no estuviera permanentemente congelada, el cuerpo de agua más cercano a Navidad era el Lago Lagda, quizás de unos ochenta kilómetros cuadrados de ancho y enterrado bajo el hielo. Pero eso apenas importaba, ya que el barco estaba situado en un tren de aterrizaje con resorte con esquís especiales, de tamaño gigante, fijados de manera que cuando las ventiscas quedaban atrapadas en las velas de la embarcación, irían a una velocidad terrorífica a través de los campos nevados.
 
Como siempre, eran necesarios dos equipos de seis perros de nieve cada uno para remolcar el Salvavidas desde la ciudad e ir hacia el este a lo largo del Valle de Halva, la única entrada al corazón del protegido corazón de las Colinas Halva, la cadena montañosa circular en medio de la cual estaba enclavada Navidad, y a lo largo de la cual el sol brillaría durante esos escasos momentos en que se alzaba sobre el horizonte en verano. Los equipos de perros estaban enganchados a babor y estribor en lugar de en la parte delantera, para que cuando las colinas se aplanaran a ambos lados, los vientos cíclicos de la llanura abierta llenaran las velas de la embarcación. Los arneses se aflojarían, y los equipos se desplegarían de forma segura mientras el Salvavidas iría rápido por la tormenta.
 
A partir de ahí, todo dependía del timonel.
 
--Estamos con viento del norte en contra --gritó Caleb al resto de la tripulación mientras estaba de pie en el puente, con ambas manos en el timón, solo sus ojos sobresalían de su bufanda--. ¡Cambio de estribor a babor!
 
La tripulación asintió, y se fue rápidamente a trepar por las cubiertas de alquitrán para ajustar las velas. En cuanto el viento cambió de dirección, cosa que hacía normalmente en las Tierras Lejanas, pero de forma predecible, razón por la que Caleb tenía una carta eólica recubierta de vidrio a su derecha, pudieron meterse en la bodega, donde una marmita de café caliente burbujeaba sobre la estufa de carbón.
 
El Doctor, por su parte, se quedó en la proa perdido en sus pensamientos, ignorando los copos de nieve que lo azotaban como puntas de flecha pese a  haberse abrigado para el frío: un gorro, una bufanda y guantes de lana, y por supuesto su inmenso abrigo de piel. Es cierto que no necesitaba ``sobreempaquetarse'' como muchos excursionistas que se aventuraban en las Tierras Lejanas, ahora mismo llevaba el abrigo desabrochado por delante y se le estaba sacudiendo con el viento, pero hubo un tiempo en el que era casi completamente ajeno al frío de Trenzalore.
 
--¡Doctor, te vas a morir de frío! --Gritó Caleb, mientras viraba el timón y efectuaba la lenta curva en la misma dirección que el viento.
 
El Doctor parecía no escuchar. Miró a la blancura, borrosa por culpa del silencio, mientras surcaban las nieves con un desliz continuo y un ondeo regular y con el golpeo de las cuerdas y las velas. La resistencia lateral de la superficie congelada les permitía mantener un curso continuo y constante, pero a treinta nudos no era precisamente un paseo por el parque. La nave se sacudía, giraba y se balanceaba con frecuencia mientras las corrientes de vientos lo envolvían como si fueran una tela. De vez en cuando, les pasaban trozos de hielo rotos que eran como unos espectros torturados y retorcidos por el viento. También había otros obstáculos: cuestas y montículos, formaciones rocosas imponentes, pero Caleb conocía el negocio. Atravesaron cada una de las dificultadas sin ni siquiera un meneo o una estela de polvo ondulante en su parte trasera.
 
--Un lugar extraordinario, Trenzalore --dijo finalmente el Doctor, mientras recorría cojeando la sucia cubierta--. Las gallinas que entran por las que salen.
 
--No comprendo --gritó Caleb.
 
--Bueno... no necesitas entenderlo todo para saberlo. La ley del ying y el yang, positivo y negativo, luz y oscuridad --el Doctor miró al vacío iluminado por la luna--. Todo está bien equilibrado aquí, es extraño.
 
--Bien equilibrado... Poco más y no vemos la luz del día.
 
--Ahí es donde quiero llegar --el Doctor hundió en el hombro de Caleb uno de sus dedos enguantados--. Casi no hay luz solar, ¿y qué hace la Naturaleza? Os da dos lunas, Soror y Frater, y eso produce una síntesis lunar que permite florecer a la flora y a la fauna. Mira estas temperaturas... --rascó una capa de escarcha que había sobre el cristal del termómetro--. No podrían sobrevivir bajo ningún estándar, pero vuestros ancestros encontraron un pequeño hueco en las montañas. No sólo eso, tenéis todas esas superficies planas, y una marea de viento en continuo cambio que os transporta suavemente por encima de ellas --el Salvavidas se sacudió como si se hubiera dado contra algún tipo de obstrucción--. Bueno, la mayoría de las veces. Supongo que un viaje sin golpes sería un poco aburrido. La cosa es que sois capaces de vivir aquí, de viajar, tenéis todo lo que necesitáis. Pero no es un lugar atractivo, nadie se atrevería a venir a tratar de... --sus palabras se apagaron.
 
--¿Arrebatárnoslo? --sugirió Caleb--. Supongo que esa es la palabra que te falta, ¿no?
 
El Doctor esbozó una cara triste.
 
--Sí, la verdad es que sí.
 
--Esa es la otra parte del equilibrio. Nadie en su sano juicio querría vivir aquí. Pero hay toda clase de enemigos congregándose allí fuera... según tú.
 
El Doctor le dio una palmadita en el brazo.
 
--Esos no son de lo nos que tenemos que preocupar, Caleb, son de los de aquí abajo... de los extraños.
 
Caleb giró el timón para cambiar la dirección de la nave.
 
--¿Alguna idea de quiénes son?
 
--Es bastante obvio, la verdad --el Doctor parecía alicaído otra vez--. Pero no te va a gustar. Me temo que vuestras lanzas y ballestas no serán de mucha utilidad.
 
Además de cargar la bodega con sus equipos de supervivencia, esquís, raquetas de nieve, cuernos, etc, la tripulación del Salvavidas también trajo algunas armas. La mayoría de ellas habían sido adaptadas a partir de herramientas de labranza, pero eran poco más que juguetes, ninguna de ellas estaba diseñada propiamente para el combate. Las ballestas antes mencionadas, de las cuales sólo había dos, eran prácticamente inútiles: pequeñas y ligeras, fabricadas principalmente para cazar.
 
--Espero que los extraños estén en tablas con nosotros --dijo Caleb con un ligero tono de preocupación.
 
--Esperemos que no en las mismas tablas, Caleb --el Doctor volvió a ahondar en su brazo con un aire de cordialidad forzada--. Como por ejemplo en las nuestras. Eso nunca funcionaría.
 
--Ya sabes a lo que me refiero... con respecto a las armas.
 
--Sí... bueno, en teoría. Aunque ellos todavía tienen algo de ventaja.
 
--Bueno, en seguida lo sabremos. Mira lo que tenemos delante.
 
La larga escarpa de la Cresta de Fafnir se extendía más allá de los extremos de su visión, una pared de rocas y piedras cubiertas de hierro que ascendía lentamente hasta terminar en una gruesa línea blanca sobre una nube violeta de nieve. Nadie, ni siquiera el Doctor, sabía si se podía rodear, nunca nadie había llegado tan lejos,  así que normalmente la atravesaban. Lejos hacia el este estaba la Ruta de la Cabra, un zigzagueante sendero que llevaba hacia la cima, aunque sólo los más fuertes o los más tontos se atrevían a transitarla. La principal ruta hacia el otro lado eran los Codos del Diablo. A lo lejos, no era más que un mordisco en forma de V sobre la escarpa, una diminuta dentellada, pero de cerca era colosal, como el Desfiladero de Cumberland, ese antiguo paso de montaña en las Montañas Apalaches de Norteamérica a través del cual los primeros exploradores recorrían la Ruta Salvaje.
 
Casi inconscientemente, el Doctor comenzó a canturrear.
 
--Da-vyy... Da-vyy Crockett, rey de la frontera salvaje... --Suspiró--. Pobre Tiberio --aunque, por supuesto, si a lo que se enfrentaban ahora era real, sería más bien ``Pobre Navidad''.
