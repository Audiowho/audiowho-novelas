\chapter*{Capítulo 5}
\addcontentsline{toc}{chapter}{Capítulo 5}

--No me mires con esa cara, Caleb --dijo el Doctor mientras se inclinaba para rebuscar en la mochila de Tiberio Gluck y sacaba un surtido de curiosidades, del cual sólo se reconocían un montón de latas, herramientas para cavar, arpones de pesca y varias cajas de cerillas--. Te lo dije, la Conciencia Nestene no tiene forma natural. Dondequiera que se oculte, es muy probable que esté en órbita en alguna parte. Quizás en una nave espacial secuestrada por los Autons. Y ellos son todo lo que podrías esperar.
 
Caleb no parecía quedar tranquilo con eso. Volvió a mirar el objeto mutilado que colgaba del árbol. Rubin lo estaba custodiando, sujetando nerviosamente la ballesta, mientras Yoshua merodeaba por el perímetro del claro, y Josef y Jerema se encargaban de trasladar el cuerpo de Tiberio, ahora envuelto en mantas, a la bodega del Salvavidas.
 
--¿Estás seguro de que está muerto? --preguntó Caleb.
 
--Estrictamente hablando, nunca ha estado vivo. Pero es evidente que ya no es viable, así que lo han abandonado. Ya no es más que un trozo de polímero inerte.
 
--Creía que habías dicho que esas cosas eran invulnerables.
 
El Doctor distrajo brevemente su atención mientras sacaba de la bolsa de Gluck un tubo hecho de un material pálido y maleable. Estaba envuelto en papel marrón con algo grasiento. Cuando lo olió, detectó nitrato de amonio y posiblemente un toque de nitroglicerina.
 
--Boronita, ¿eh?
 
La boronita era un explosivo industrial habitual en las colonias más alejadas de la Tierra, era como nueve veces más potente que la dinamita tradicional, pero con un contenido de nitroglicerina reducido para contener su volatilidad. Incluso alguien que rutinariamente parecía haber tomado malas decisiones en la vida, como Tiberio Gluck, podía manejarlo con relativa seguridad.
 
--Teníamos una reserva durante los primeros días del asentamiento --dijo Caleb--. Lo utilizábamos para penetrar el lecho de roca de debajo de la ciudad y llegar a las aguas termales. Lo que sobró lo guardamos. Más tarde, Tiberio cogió algo para sí.
 
--Bueno... La boronita es siempre la bomba --el Doctor guardó el explosivo bajo su abrigo, y hurgó en el paquete, sacando fusibles enrollados, de unos tres metros y medio--. Sobre el tema de los Autons, Caleb... no recuerdo haber dicho que eran invulnerables. Pero seamos sinceros, nada puede caer en picado a través de la atmósfera de un planeta sin que se haga algo de daño. Aunque es interesante... --levantó la vista, pensativo--. Tal vez el campo de fuerza está interfiriendo de alguna manera. Tal vez eso y las condiciones extremas de Trenzalore estén limitando las capacidades del Nestene. Quiero decir, ya me he topado con un montón de Autons. La mayoría de ellos podían cambiar sus características sobre la marcha, reparar daños físicos extensos...
 
--Entonces, ¿por qué esta criatura no se está reconstruyendo? --preguntó Caleb.
 
--¡He ahí la cuestión! Y esa es probablemente una buena noticia.
 
--Lo dices como si también hubiera malas noticias.
 
--Por supuesto que las hay --el Doctor se puso de puntillas--. El ying y el yang, recuérdalos. No se puede tener uno sin el otro...o al menos yo no --hizo otra pausa--. La mala noticia es que la joven Yalala Gluck tenía razón. Los Autons siempre han sido más eficaces cuando se les usa como facsímiles, disfrazados de personas de autoridad. Una idea realmente original... si no la utilizaran de una forma tan vil. Lo que hacen es reemplazar a esa gente en la comunidad en la que se estén intentando infiltrar. Una vez instalados, su potencial para provocar el caos es bastante alto, así que imagínate. Si uno de esta partida de aterrizaje llega a Navidad pareciendose a mí, tendremos serios problemas.
 
--¿Y ahí es a dónde crees que han ido?
 
--Fijo. La pregunta realmente importante es... ¿cuántos hay?
 
-- No más de cinco, diría yo -- interrumpió Yoshua, acercándose a ellos--. Hay huellas por aquí --le siguieron hasta el otro lado del claro, donde un sendero relativamente reciente, ahora poco más que un conjunto de marcas redondeadas, los alejó de los árboles hacia la lejana cordillera de la Cresta de Fafnir.
 
--¿Puedes deducir que son cinco con eso? --preguntó el Doctor.
 
Yoshua asintió. Era el miembro más fornido de la tripulación, y el de la barba más poblada. Además de sus deberes a bordo del barco, había heredado el manto de cazador jefe de su padre, y en momentos en el que los suministros de alimentos eran bajos, era él a quien mandaban a las Tierras Lejanas para traer mandriles y ciervos de la tundra. Si él decía que estas marcas las habían hecho cinco individuos, no había razón para no creerle. De hecho, hasta tenía sentido.
 
--Si el control del Nestene es limitado en este planeta, tiene que haber un límite en el número de unidades Auton que pueda dirigir --dijo el Doctor--. Quiero decir, ese es el caso en los planetas donde las condiciones son buenas. Todavía no he visto legiones de Autons en acción... aunque una vez conocí a un Auton que era legionario --frunció el ceño, confuso--. Los recuerdos se desvanecen, por desgracia. En fin, puede que Yalala Gluck y tú tengáis razón, Yoshua, sólo nos enfrentamos a un puñado de ellos. Y eso tiene que ser bueno. ¿Cuánto tiempo antes de la siguiente racha de viento, Caleb?
 
Caleb miró su dial de tiempo.
 
--Doce minutos.
 
--Eso no es bueno. Andaremos cortos de tiempo.
 
--¿Por qué se dirigen al sur? --preguntó Luca a nadie en particular, señalando las huellas, que, gracias a la potente luz de las lunas, pudieron distinguir perfectamente cómo se habían dirigido hacia la distante cordillera, sin ni siquiera virar al oeste en dirección a los Codos del Demonio o al este hacia la Ruta de la Cabra--. ¿No es el camino equivocado?
 
--No, si te diriges a Navidad por la ruta más rápida --explicó el Doctor, volviendo cojeando por el claro.
 
Los demás lo siguieron.
 
--¡Pero tendrán que escalar por la Cresta de Fafnir! --argumentó Luca.
 
--¿Qué he dicho sobre los Autons que no hayas entendido todavía, Luca? --respondió el Doctor--. Te lo dije, nunca se cansan y no tienen sentimientos. Pueden atravesar cualquier obstáculo que se interponga en su camino.
 
Caleb y Yoshua se miraron con inquietud. Se les había pasado por la mente la misma y breve imagen: unas poderosas criaturas de metro y medio, cada cual más maltrecha, rota y parcialmente derretida que la anterior, vestidas con los harapientos remanentes de sus ropajes y, pese a las circunstancias, aventurándose en la nieve de una rodilla de espesor, sin detenerse o planificar un rumbo, y una pared de roca que se va acercando hasta alcanzarlos y obligarlos a escalarla sin descanso, nivel tras nivel, repisa tras repisa.
 
--¿Qué tipo de armas tienen? --preguntó Yoshua.
 
--Por lo general matan con rayos de energía concentrada --dijo el Doctor--. Claro que eso no les funcionará en Trenzalore. No podría haber pasado a los escáneres del Ordenador Central Papal.
 
--A juzgar por el aspecto de Gluck, tuvieron que golpearlo y estrangularlo --dijo Caleb.
 
--Sí, bueno... --El Doctor intentó sonar menos incómodo de lo que realmente estaba--. Al menos eso no quita que podáis levantaros y luchar contra ellos. Al llegar a la ciudad, aseguraros de que todo el mundo bloquee sus puertas y ventanas. No será fácil atravesarlas siendo plástico vivo.
 
--¿Cuando volvamos a la ciudad? --preguntó Luca--. ¿No vienes con nosotros?
 
El Doctor se volvió hacia ellos.
 
--Por supuesto que sí --esbozó media sonrisa--. Después. Primero quiero darme una vuelta con el Salvavidas. Siempre he querido hacerlo --empezó a dar zancadas.
 
El grupo intercambió unas miradas que parecían más perplejas con cada paso que daban al seguirlo.
 
--Así que, ¿cuándo y dónde nos separamos? --preguntó Caleb.
 
--Cuando lleguemos a la Cresta de Fafnir --respondió el Doctor--. Tomaré el timón del Salvavidas y os dejaré al pie de la Ruta de la Cabra. A partir de ahí, iréis andando --le dio a Caleb una palmadita en el hombro--. Sois jóvenes y robustos. Tenéis a Yoshua como guía. Tenéis vuestras técnicas. Estoy segurísimo de que podréis arreglároslas. Hasta entonces, una vez que estemos a bordo, quiero que sólo icéis la vela mayor, ¿entendido?
 
--¿Y si no lo conseguimos? --preguntó Caleb.
 
--Bueno... --la sonrisa del Doctor vaciló--. Me temo que vais a tener que hacerlo. Es muy posible que vosotros seáis la única esperanza del pueblo.
 
Volvieron a intercambiarse miradas de preocupación. El equipo sabía que no debían cuestionar al Doctor. Los habían educado así desde pequeños, a pesar de sus ocasionales momentos de locura. Y ahora, ¿ahora les estaba abandonando?
 
--Esos Autons tienen ventaja --dijo Yoshua--. Aunque navegemos hacia la Ruta de la Cabra, todavía queda muy al este de aquí. No les ganaremos mucho terreno, por no decir ninguno. Y ellos no se cansan, recordad. Llegarán a Navidad mucho antes que nosotros.
 
--No lo harán, Yoshua --respondió el Doctor--. Haré todo lo posible para asegurarme de ello.
 
--Entonces, ¿no vas a dejarnos? --Luca le preguntó con un tono quejumbroso--. ¿No vas a huir?
 
--Yo nunca huyo, Luca. Nunca.
 
Luca sonrió, pero tragó nerviosamente saliva.
 
--No lo entiendo --dijo Caleb, mirando al Doctor con sospecha--. Después de "dejarnos", como dices, ¿dónde vas a llevar el barco?
 
El Doctor se encogió de hombros.
 
--¿A dónde si no? A los Codos del Diablo.
 
--Con una sola vela te va a ser imposible atravesar el Degolladero. Ni con una racha fuerte del sur.
 
--Lo sé --el rostro del Doctor explotó con una radiante sonrisa y agarró al receloso timonel por las costillas--. ¿Y eso no te parece lo más genial de todo?
