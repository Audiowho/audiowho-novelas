\chapter*{Capítulo 2}
\addcontentsline{toc}{chapter}{Capítulo 2}

--¿Extraños en Tierras Lejanas? --preguntó el Doctor mientras bajaba lentamente la escalera de la Torre del Reloj, con su pierna artificial crujiendo--. ¿A que se refiere exactamente con eso?
 
--Ha sido difícil conseguir algo coherente de ella --respondió Caleb desde abajo. Era un joven alto y delgado, con un esbelto rostro y una mata de pelo de color rojo pajizo, sólo sus bordes eran visibles alrededor de los bordes de su sombrero de lana--. Becca dice que está en estado de shock profundo... El problema es que balbuceaba un par de cosas que no tenían mucho sentido, y luego dejó de hablar. Sé bueno si quieres hablar con ella.
 
--¿Quién es ella?
 
--Su nombre es Yalala Gluck. No sabemos mucho de ella, para ser honestos. Tiberio Gluck es su padre. Algo ermitaño, se mudó a Terrallende después de que su esposa muriera. Siempre en busca de oro y plata. Viene a la ciudad de vez en cuando. Trae pieles, ámbar y similares, que intercambia por comida y ropa.
 
--¿Un trampero? --dijo el Doctor, esbozando una sonrisa infantil--. Qué pintoresco.
 
--Tú lo conociste.
 
--¿De verdad? --el Doctor enrolló su bufanda alrededor del cuello--. ¿Gorro de mapache? ¿Muerto por un mosquete? No, espera... Ese era Davy Crockett. ¿O era Daniel Boone? Lo siento... --se encogió de hombros--. No lo recuerdo. Vives todos estos siglos, y empiezas a olvidar cosas que sucedieron hace un par latidos.
 
--Serían más como unos cinco o seis años.
 
--Eso es lo que quería decir --el Doctor se volvió a su mesa de trabajo--. ¿Qué opinas, Handles? Extraños en Tierras Lejanas.
 
La maltrecha y vieja cabeza cibernética volvió a la vida.
 
--No hay suficientes datos para hacer un análisis correcto de estos llamados ``extraños''. Las Tierras Lejanas es una zona subártica, capaz de soportar sólo las formas más resistentes de vida vegetal boreal, que en consecuencia...
 
--Sí, si --exclamó el Doctor--. ¡Ya sabemos todo eso!
 
--No hay formas de vida nativas bípedas inferiores de ningún tipo que estén registradas en el catálogo de fauna de Trenzalore. Sin embargo, dicho catálogo puede ampliarse si los descubrimientos adicionales...
 
--¡No, gracias! --el Doctor se encogió de hombros bajo su inmenso abrigo de pieles--. No estamos en el negocio de la exploración de Trenzalore. Podemos dejar eso a los Tiberio Gluck de este mundo. ¿Qué haces ahí parado, Caleb? No podemos perder el tiempo.
 
--Lo siento, Doctor.
 
--Mi bastón, por favor.
 
Caleb entregó el bastón tallado al Doctor.
 
--¿Cuántos extraños vio esta niña?
 
Caleb reflexionó.
 
--Cree que cuatro o cinco.
 
--¿Y cuántos vió su padre?
 
--De eso se trata. No regresó con ella.
 
--Ah-ha... --el rostro del Doctor cambió a un ceño fruncido mientras se dirigía hacia el exterior, su abrigo de piel era tan grande que lo arrastró tras de sí--. Eso es algo a lo que le puedo hincar el diente. Toma una actitud poco seria respecto a sus responsabilidades de criar a su hija, ¿verdad? Ciertamente tendré unas palabras con él, duro y vejestorio luchador indio, o no.
 
Se detuvo en seco fuera de la Torre del Reloj, donde una fila de cinco jóvenes fornidos le esperaban, barbudos y curtidos por la intemperie, sus corpulentos físicos acentuados por sus coloridos atuendos.
 
--¿Y qué es esto? --El Doctor desfiló delante de ellos como un sargento mayor--. ¿Yoshua... Rubin... Josef... Jerema... Luca...? --levantó una ceja inquisidora hacia Caleb--. Todos los miembros de la tripulación del Salvavidas, si no me equivoco. ¿Nos vamos a navegar?
 
--Pensamos que podría ser necesario --dijo Caleb.
 
--Lo que sea por una aventura, ¿eh? --el Doctor se encaminó a través de la nevada calle hacia la taberna. La tripulación del Salvavidas se pegó detrás.
 
--¿No lo crees? --preguntó Caleb.
 
--Creo que la crianza negligente era un triste, pero regular hecho en la Tierra --respondió el Doctor--. Y que solo era cuestión de tiempo antes de que siguiera al hombre a las estrellas. Gluck vive en las Tierras Lejanas, donde cree que puede escaparse de todo.
 
--Pero. ¿qué pasa con los extraños?
 
--¿Cómo sabemos que son extraños para Tiberio Gluck?
 
--La chica huyó sesenta kilómetros para alejarse de ellos. Difícilmente pueden ser amigos.
 
El Doctor se dio la vuelta apoyándose en su bastón con el ceño fruncido.
 
--¿Sesenta kilómetros?
 
--Tenía dos perros de nieve. Bestias menores nunca lo habrían conseguido.
 
--¿Sesenta kilómetros, Caleb?
 
--Vino todo el camino desde la tundra a través de los Codos del Diablo. Estimamos sesenta kilómetros por lo menos.
 
El Doctor reflexionó sobre esto mientras caminaba hacia la taberna, que estaba abarrotada como siempre, aunque el ambiente era bastante menos jovial. Las filas de los aldeanos se separaron para dejarlo pasar. En el centro, una chica joven que llevaba ropa casera estaba sentada en un taburete, con las manos vendadas recientemente, cada dedo estaba vendado individualmente. La enfermera Becca, el médico más calificado del pueblo, estaba arrodillada junto a ella, masajeando suavemente un par de diminutos, desnudos y blancos dedos en un cuenco de agua tibia. A pesar de esto, la niña tembló, sus rasgos angelicales eran pálidos incluso en la rosácea luz del fuego. Sus labios eran delgados y de color gris perla y sus ojos húmedos brillaban aunque miraban a la nada. Su pelo rubio colgaba en rizos húmedos y con con hebras.
 
--Leve hipotermia y congelación de primer grado --contestó Becca a la pregunta inicial del Doctor--. Aparte de eso, está en un sorprendente buen estado de salud.
 
--Tal vez Tiberio Gluck no era un padre tan negligente después de todo --el Doctor se mordió la comisura de la boca--. Pobre Tiberio...
 
--¿Qué crees que le pasó, Doctor? --preguntó alguien.
 
--Perdona si no respondo a esa pregunta delante de la niña.
 
--Está demasiado traumatizada para saber siquiera dónde está --dijo Becca, pasando una mano delante de los vidriosos ojos de su paciente, sin obtener respuesta.
 
El Doctor seguía sin decir nada, y todos ellos comprendieron su reticencia. El filtro de verdad era un aspecto dolorosamente revelador de la vida en la Navidad. Si, por ejemplo, el Doctor sospechaba que Tiberio Gluck había sido asesinado, y que su hija de 8 años era extremadamente afortunada de haber escapado con vida, y él expresaba sus conjeturas en voz alta, eso no ayudaría a la moral de los miembros más nerviosos de la comunidad.
 
Felix, el hermano menor de Caleb, estaba de pie , escuchando con asombro. Y luego estaba la niña.
 
El Doctor se arrodilló para mirarla.
 
--¿Dices que su nombre es Yalala?
 
--Creemos que sí --dijo Becca--. Tiberio era muy reservado cuando se trataba de asuntos familiares.
 
--¿Yalala? --preguntó el Doctor con cuidado.
 
Ella lo miro directamente.
 
--No tienes necesidad de estar asustada. Estas a salvo ahora. Has recorrido un largo camino por tu cuenta. ¿Quieres decirnos el por qué? ¿Quiénes eran esos extraños, Yalala? --preguntó el Doctor-- ¿Qué aspecto tenían?
 
Sólo después de un minuto agónico, ella realmente pareció verlo. Sus labios grises arrugados fruncidos formaron una 'O' perfecta. Temblorosa, ella le señaló con un dedo vendado y gritó.
 
Fue un grito estridente e intenso, ininterrumpido hasta que la enfermera Becca y otras madres de la comunidad se movilizaron para calmarla. Pero todo ese tiempo, ella gritó y señaló por encima del hombro al Doctor, que por fin salió afuera, a la nieve.
 
Caleb y el resto de la tripulación del Salvavidas lo acompañaron, desconcertados.
 
El Doctor no era sólo su jefe, su alcalde, su asesor en todas las cosas, era su salvador. Aunque no lo adoraran de verdad, como últimamente le daban arrebatos de sequedad e impaciencia, su respeto por él era tan profundo como el permafrost.
 
--¿Doctor? --Caleb finalmente se aventuró a preguntar-- ¿Qué significa esto?
 
--Yo habia pensado que era perfectamente obvio --respondió bruscamente el Doctor--. A quien Yalala vio por ahí, o lo que sea que ella viera o creyera ver, se parecía a mí.
