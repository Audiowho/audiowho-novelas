\chapter*{Capítulo 6}
\addcontentsline{toc}{chapter}{Capítulo 6}

--¡Deprisa, Doctor!

Fergin estaba dentro de las enormes puertas de la granja de nieve, con los brazos enrollados alrededor de dos de sus postes de metal. Con una puerta ya cerrada, se estaba preparando para empujar la otra, en cuanto todos los del pueblo hubieran pasado.

El último fue el Doctor, quien, a pesar de su edad y debilidad, todavía se las arregló para alcanzar al grupo.

Vino a zancadas por el suelo nevado, con la cabeza gacha, su pierna de madera rotando con cada paso, su bastón para caminar operando más o menos como un miembro extra.

--¡Ahora, Fergin! --exclamó mientras se escurría por la rendija.

Cuando un par de personas se adelantaron para atrapar al Doctor antes de que se cayera de morros, Fergin cerró la puerta con un golpe. Taskia inmediatamente apareció de un salto con una pesada cadena, la cual enrolló alrededor de las dos puertas y luego aseguró con un candado del tamaño de un puño.

Pericles Torp, que encabezaba al grupo de niños, se agolpó contra la puerta con un siseo furioso e intentó escalarla, pero patinó y cayó al suelo.

Reincorporado gracias a la gente del pueblo que le había cogido, el Doctor se acercó cojeando a la puerta. Él y Pericles se miraron durante un momento a través de las barras. Fue entonces cuando el Doctor habló.

--Muy bien, Mara, escúchame. No tolero las amenazas, pero estoy preparado para ofrecerte una proposición siempre y cuando dejes en paz a mis amigos. Encuéntrate conmigo en la calle principal dentro de diez minutos, solos tú y yo, cara a cara. ¿Qué dices?

Pericles parpadeó sus ojos amarillos de serpiente.

--¿Qué clase de truco es este?

--Ninguno. Sólo una charla. Pero no metas a los niños en esto. Devuélvelos a la escuela. Quiero hablar con el titiritero. No con la marioneta.

El chico lo miró sin parpadear mientras la Mara consideraba su propuesta. Finalmente dijo:

--Muy bien. Pero recuerda, Doctor, los niños están bajo mi mando. Puedo ordenarles que se despedacen entre ellos sin previo aviso.

Los ojos del Doctor se entrecerraron.

--Diez minutos --repitió.

Entonces sin soltar ninguna otra palabra, se dio la vuelta y se fue cojeando.

~

El Doctor yacía de pie sobre la nieve, aparentemente ignorante de los copos que se le pegaban al pelo y a los hombros de su abrigo. Las casas que se alineaban a ambos lados de la amplia calle estaban en silencio, aunque una suave luz ámbar radiaba de algunas ventanas. La postura del Doctor era casual; no parecía aprehensivo ni decidido, tan sólo paciente, como si pudiera esperar durante horas. Un fragmento de una vieja rima corrió por su mente: No se oía ni un ruidito, ni siquiera chillar a un ratón…

Pero sí que se oía algo. En la blanca oscuridad, al final de la calle, hubo indicio de movimiento. Un destello de color rojo.

La Mara se estaba acercando.

Aunque se movía a dos patas, parecía reptar desde la oscuridad. Parecía más alargada, más como una serpiente, de lo que había sido cuando el Doctor la vio en el granero, pero todavía se la reconocía como humano. Eso era bueno, pues significaba que la criatura no se había hecho del todo manifiesta, que su asidero a este mundo seguía siendo tenue. Miró al Doctor con unos ojos que relucían como lámparas amarillas, sacando una lengua negra y delgada. Sus escamas rojas parecían ser intermitentes, haciendo refulgir su capucha de cobra como una vela.

Tranquilamente, el Doctor dijo:

--Hola, Mara.

La Mara lo rodeó, con unos movimientos ágiles, de bailarina, como si deseara examinarlo desde todos los ángulos. El Doctor se quedó quieto, mirando al frente.

Cuando la Mara dio una vuelta completa a su alrededor y se volvieron a mirar, dijo:

--¿Aún no me temes, Doctor?

El Doctor se encogió de hombros.

--Me temo que no. Lo siento. Sólo sienten miedo aquellos que tienen algo por lo que vivir.

La Mara sonó casi sorprendida.

--¿De verdad estás tan cansado de la vida?

--He visto el futuro --El Doctor puso una mueca--. Termina en llamas.

--Qué delicioso --siseó la Mara. Su tono cambió, se volvió menos juguetón--. ¿Tienes algo que ofrecerme?

El Doctor asintió.

--Sí. Última oferta. Lárgate de este planeta, vuelve por donde has venido, y no te destruiré.

La Mara se echó a reír.

--Oh, qué pena. Me esperaba mucho más.

--Así es la vida --dijo el Doctor.

De pronto la Mara se echó sobre él, con la boca abierta y sus colmillos relucientes. El Doctor mantuvo la calma. Cuando estuvieron casi frente a frente, la Mara se detuvo, mirándolo directamente a los ojos. Silbantemente dijo:

--No puedes ganar esta batalla, Doctor. Si no dices tu nombre adrede, te obligaré. Esta gente me pertenece. Puedo invadir sus sueños, controlarlos a voluntad. Y pronto conseguiré invadir también los tuyos. Me abriré paso por tu mente y te haré escupir tu nombre… y luego será el caos.

El Doctor puso los ojos en blanco.

--Sí, ya, ya he oído eso antes. Aburrido, aburrido. --Alargó un dedo y tocó a la Mara en su escamoso pecho--. Si lo piensas hacer, ¿por qué no lo haces ahora, eh? ¿Por qué no acabas con esto? Te lo voy a decir yo, ¿vale? Es porque no puedes. Es porque no eres lo suficientemente fuerte. Ni siquiera has conseguido entrar todavía ni en la mitad de estos insignificantes humanos, ¿verdad? Estás atrapada entre la oscuridad y la luz, entre la espada y la pared.

La Mara retrocedió, ofendida por sus comentarios.

--Cada vez somos más fuertes, Doctor. Pronto nada será un obstáculo para nosotros.

--Pronto, pronto, pronto. Ya, sí, tal vez no tenga tanto tiempo para esperar.

--No tienes elección. No puedes destruirnos. No puedes escapar de nosotros.

--No sois los únicos.

La Mara entrecerró los ojos.

--¿Qué significa eso?

En vez de responder, el Doctor levantó una mano y la sacudió en círculos por encima de su cabeza.

--¿Sabes lo que es esto? ¿Este sitio? Es una granja de nieve. Fabrica nieve de más y la reparte a otras colonias repartidas por el planeta. A los humanos les encanta la nieve. ¿Has construido alguna vez un muñeco de nieve, Mara? ¿Has estado alguna vez en una batalla de bolas de nieve? ¿Cogido un copo con la lengua y dejado que se derritiera?

--Esto es irrelevante --se burló la Mara.

El Doctor soltó un bufido.

--¡Irrelevante! Vale. Eso es lo que decís siempre. Todos los déspotas locos por el poder que sólo quieren conquistar y destruir. Cualquier cosa trivial, divertida, es irrelevante. ¿Pues sabes qué? Apuesto a que Jalen Fellwood no piensa así.

La Mara pareció durante un momento haber sido cogida por sorpresa.

--Jalen Fellwood está muerto.

--Estrictamente hablando, sí --dijo el Doctor, levantando la voz--. Y pese a ello sigue aquí, ¿no? ¿En tu interior? La vil entidad en la que estaba metido hacía todos esos siglos guardó una pequeña parte de él vivo. Atrapada en el interior de esos viejos huesos. Una furiosa pepita de odio. Porque por eso estás aquí, ¿no? ¿No es eso lo que dependes? El mundo los cría y ellos se juntan. Puede que Jalen Fellwood esté muerto, pero sigue siendo tu punto focal. Tu corazón. --Se detuvo. Su voz se redujo de repente--. Tu talón de Aquiles.

Estaba claro que la Mara se estaba poniendo nerviosa. Retrocedió con unos ojos que se movían de un lado al otro, como si se esperara una emboscada, un ataque. El Doctor dio un paso al frente, todavía hablando, tomando ventaja.

--¿Sabes lo que hizo la gente de Navidad después de enterrar el cuerpo de Fellwood en el bosque? Sembraron la tierra con sal. Lo hicieron para que el espíritu de Fellwood no volviera, y así no pudiera vengarse.

»Sal. Qué cosa tan simple. Tan inocua. Y aun así consiguió atrapar a Fellwood y al mal que lo controlaba. Sepultado en el suelo durante siglos. ¿Y sabes qué Mara? ¿Sabes por qué Fellwood quedó atrapado? Es porque creía. Creía que la sal era poderosa y creía que esa sal lo destruiría…

La voz del Doctor se redujo a un murmullo.

--Y todavía crees en ello, ¿no, Jalen? Todavía crees en ello.

La Mara lo volvió a decir, aunque no sonó tan segura esta vez.

--Jalen Fellwood está muerto.

Como respuesta, el Doctor alzó la vista hacia la nieve y sacó la lengua.

Un copo aterrizó en ella. Se la tragó. Sonrió.

--¿Has cogido alguna vez un copo con la boca y dejado que se derritiera en la lengua? --le volvió a preguntar--. Deberías intentarlo, Mara. Es uno de los pequeños e irrelevantes placeres de la vida. --Entonces se relamió los labios--. Pero fíjate, menudo sabor que tiene esta nieve. Sabe… mmm, ¿cómo se decía? --Su rostro se encendió--. ¡Oh, sí, ya! ¡Sabe salado!

De repente la Mara se alarmó. Su cabeza se alzó para mirar cómo la nieve giraba, vagaba, caía del cielo.

--¡Mi propia receta! --exclamó el Doctor triunfal, señalando con su bastón a la distante chimenea que, desde dónde estaban ellos, era lo único que se veía de la granja de nieve--. Toda la sal a la que la gente de Navidad pudo ponerle las manos encima. Se la mezcla con el agua de las cámaras de nieve, se ajustan unos cuantos controles, y ¡tachán! ¡Nieve salada! --Sonrió--. ¿Qué te parece? ¿Crees que tendrá éxito?

La Mara gritó. Su piel roja y escamosa estaba comenzando a evaporarse y a hervir a medida que los copos de nieve la tocaban. Giró, rodó, pero no había escapatoria. Se encaró al Doctor, con unos ojos amarillos llenos de furia.

--¡Los destruiré a todos! ¡A todos los niños! ¡A toda la gente que quieres! ¡Les ordenaré que se despedacen entre sí!

El Doctor sacudió la cabeza.

--Oh, me parece que no. Creo que ya tienes bastante con mantener el cuerpo y el alma de una pieza. Mírate, Mara. Te estás rasgando.

Era cierto. Cada uno de los copos de nueve que se habían posado sobre la Mara estaba haciendo que su piel hirviera, ardiera, que se licuara. Tiempo después la criatura serpiente comenzó a parecerse a una vela de cera roja expuesta a un inmenso calor, y su forma física se distorsionó, se disolvió. Y cuando se disolvió, se secó, se desvaneció, como si nunca hubiera existido, excepto quizás en un sueño.

La Mara forcejeó y se retorció y siseó cuanto pudo, pero sus esfuerzos se redujeron, se hicieron más débiles…

Hasta que finalmente todo lo que quedó de ella fue una cáscara de huesos grises y quebradizos en un suelo cubierto de nieve recién caída.



Aliganza Torp se levantó de un grito de lo que parecía haber sido el sueño más largo y profundo que hubiera tenido en la vida. Creía que había tenido sueños raros, pero no se acordaba bien. La primera cosa que vio cuando abrió los ojos fue el rostro de un anciano escudriñándolo. Conocía al anciano, pero tardó un rato en situarlo; y entonces Aliganza cayó en la cuenta: era el Doctor. El Doctor estaba sujetando un extraño collar en la mano, el cual parecía haber retirado de la cabeza de Aliganza.

--Hola --dijo el Doctor en voz baja--. Bienvenido de nuevo.

Aliganza parpadeó. Tenía la sensación de que no estaba en su cama.

--¿Me he ido a alguna parte?

--Por así decirlo --dijo el Doctor, y dejó caer el curioso collar en el bolsillo de su chaqueta.

--¿Qué es eso? --preguntó Aliganza.

--¿El qué? --Y entonces el Doctor se dio cuenta de a qué estaba mirando Aliganza--. Oh, eso. Es un inhibidor de sueño.

--¿Un inhibidor de sueño? ¿Qué significa eso?

--Significa --dijo el Doctor-- no más pesadillas.

Instintivamente, Aliganza se fue a mirar el antebrazo. Pensó que tenía algo allí, pero no había nada.

Un recuerdo repentino de unos ojos amarillos pasaron por su mente en un abrir y cerrar de ojos. Su mano se alargó y agarró al Doctor del brazo.

--Doctor --soltó antes de saber lo que iba a decir--, ¿estamos ya a salvo? ¿Se ha ido?

--Sí --dijo el Doctor con dulzura, y le dio una palmadita en la mano--. Se ha ido.

--¿Para siempre? --preguntó Aliganza.

El Doctor se detuvo. Ladeó la cabeza.

--Ah, pues --murmuró, y sus arrugadas facciones se convirtieron en una sonrisa torcida--. ¿Quién sabe? Para siempre es mucho tiempo.

