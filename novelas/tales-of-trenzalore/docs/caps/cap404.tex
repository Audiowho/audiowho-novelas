\chapter*{Capítulo 4}
\addcontentsline{toc}{chapter}{Capítulo 4}

--Háblame del esqueleto.

Sentado junto al fuego en la atestada cabaña de Vida Clatterly, el Doctor se inclinó hasta apoyar la barbilla sobre la mano que se cerraba alrededor de la cabeza de su bastón. Vida, la residente de Navidad más antigua, aparte de sí mismo, y la primera guardiana de la historia de la ciudad (una tradición oral transmitida de generación en generación) evitó su mirada. En cambio, centró su atención en la taza de té de hierbas que había en la mesa baja situada entre ellos, murmurando para sí misma mientras se agitaba con un vigor innecesario.

Era una mujer pequeña, arrugada, con la piel color castaño y ojos grandes, suaves y de color grisáceo. Llevaba un chal de colores brillantes y un pañuelo en la cabeza, los cuales brillaban a causa de la cantidad de lentejuelas y botones que los adornaban. El Doctor la conocía desde que era un bebé, de hecho él había ayudado a que naciera. Ella cogió un plato lleno con trozos irregulares y pastosos salpicados de pasas y lo empujó hacia él.

--¿Pastel de roca?

--Vida --dijo el Doctor a modo de advertencia.

Ella suspiró y su mirada se centró en algo que colgaba junto a la chimenea. Era un disco de metal en el que habían grabado unas letras unido a un deshilachado trozo de cinta.

--¿Recuerdas haberme dado eso?

--Recuerdo haberlo hecho --dijo el Doctor--. Campeona de Patinaje de Navidad Sub-16. No recuerdo el año, pero nunca olvidaré tu Triple Salto. Bueno, sobre este esqueleto...

Esta vez Vida suspiró profundamente. 

--Es la mayor vergüenza de Navidad.

--Cuéntamelo.

Por primera vez, lo miró a los ojos. 

--¿Cuánto tiempo llevas aquí, Doctor?

Él arrugó la nariz. 

--Setecientos años, más o menos, una o cinco décadas arriba o abajo.

--El esqueleto lleva allí novecientos años, soy una de las tres únicas personas que saben, sabían, de su existencia.

--¿Quién era?

Vida miró al fuego, como si pudiera ver la historia desarrollándose allí. 

--Su nombre era Jalen Fellwood. Fue uno de los primeros pobladores, un verdugo del pueblo en los días previos al Campo de Verdad. Era ... un hombre malo. Algunos dicen que estaba aliado con las fuerzas oscuras. Tomó como esposa a una chica joven llamada Summerly Treece, y la asesinó como sacrificio para el mal que adoraba, con la esperanza de que a cambio se le concedieran poderes incalculables. Pero el padre de Summerly, Rolan, otro verdugo, reunió a un grupo de ``hombres justos'' para dar caza a Fellwood. Lo persiguieron hasta el bosque, lo mataron y enterraron su cuerpo en una tumba sin nombre. Salaron el suelo para evitar que su espíritu regresara y se vengara de ellos.

--Sal --murmuró el Doctor.

--Es una defensa mágica.

--Sí, lo sé.

Vida lo miró con el ceño fruncido. 

--¿No creerás que...?

--¿Qué?

Soltó una risa nerviosa. 

--Bueno... la magia no existe, ¿verdad?

El Doctor se echó hacia atrás con un gruñido y dijo suavemente 

--Eso depende de cómo la definas. Hay algunas razas tan antiguas que su ciencia parece ser una forma de magia, los Daemons, los Osirans, los Carrionites, los Hervoken... --sonrió con un deje de nostalgia-- y la creencia es muy importante, la mente es algo muy poderoso.

--De todos modos… --empezó a decir Vida, pero fue interrumpida por un golpe en la puerta.

--Adelante --gritó el Doctor, antes de que Vida pudiera decir nada.

La puerta se abrió y apareció la cara regordeta de Fergin.

--Siento interrumpir...

--¿Qué pasa, Fergin?

--Sólo pensé que deberías saberlo, Doctor. Han vuelto. Los que entraron en el bosque, y han traído un condenado esqueleto con ellos.



\mbox{}



El doctor entró en la taberna de Navidad y se encontró con que todos los miembros de la pandilla de Fergin habían respondido a su llamada a las armas. Con su bastón en una mano y un saco de arpillera en la otra, miró a las filas de rostros que había a su alrededor y en ellas vio una gran variedad de expresiones: esperanza, ansiedad, miedo, determinación.

Eran buena gente, gente noble. El Doctor quería a todos y cada uno de ellos.

Se sentó y comenzó a hablar.

--A lo que nos enfrentamos --dijo-- no está ahí fuera... --agitó su bastón hacia la ventana-- está dentro de nosotros, o más bien dentro de vuestros seres queridos, de nuestros seres queridos. Está dentro de los que han estado soñando. Está utilizando sus cuerpos como escudo, y eso significa que la única manera en que podemos derrotarlo es sacándolo de ahí, y la única forma en la que podemos hacerlo es incapacitando a sus huéspedes.

Hubo un murmullo de consternación y alguien hizo una pregunta desde el fondo de la habitación:

--¿Cómo lo hacemos, Doctor?

--Con ésto.

El Doctor dejó la bolsa en el suelo y la parte superior, que había apretado firmemente con el puño, cayó y se abrió. La gente del pueblo miró el saco con cautela mientras su contenido cambiaba y repiqueteaba suavemente. Fergin se agachó y sacó un adorno navideño de plata, como los que decoraban los muchos árboles repartidos por la ciudad.

--¿Qué son... aparte de lo que parecen?

El fantasma de una sonrisa asomó a los arrugados labios del Doctor.

--Bombas.

Fergin miró horrorizado la bola, como si fuera a estallar en cualquier momento. La incredulidad se extendía e incrementaba por toda la habitación.

--¿Qué?

--¡Está de broma!

--Si cree que voy a tirarle una bomba a nuestra Tareena...

El Doctor levantó las mano. 

--Venga ya, me conocéis muy bien para pensar eso. ¿Qué es lo que deberías estar preguntando?

Una chica de pelo corto, cuyo rostro mostraba una de las expresiones más ferozmente determinadas de la habitación, dio un paso adelante. 

--¿Qué tipo de bombas son, Doctor?

El Doctor sonrió. 

--¡Brillante! ¡Dadle a esa niña una manzana de caramelo! --Hizo un esfuerzo y con un gruñido se inclinó hacia delante, cogió una de las bolas de la bolsa y la lanzó al aire.

--Son bombas de sueño. Éste es el plan, entramos, las tiramos, las bolas se rompen y el gas sale, y cuando todos estén dormidos robamos el esqueleto.

Extendió las manos, como si fueran a aplaudirle.

--¿Y eso es todo? --dijo Fergin.

El Doctor lo miró, perplejo. 

--¿Qué quieres decir con ``eso es todo''?

--Bueno, es un poco... simple.

El doctor levantó un dedo. 

--La primera regla para salvar el universo, Fergin es: ``nunca te inventes un plan complicado.'' Si lo haces y la gente olvida algo, las cosas salen mal.

--¿Para qué quiere esta... cosa el esqueleto, Doctor? --preguntó la chica de pelo corto.

--¡Ja! --dijo el Doctor-- ¡Otra excelente pregunta! ¡Estás que te sales, Clara!
La chica puso los ojos en blanco. 

--Me llamo Taskia, Doctor.

--Claro que sí, no dejes que nadie te diga lo contrario --El Doctor barrió con la mirada aquel mar de rostros, como si quisiera asegurarse de que estaban escuchando--. El esqueleto es todo lo que queda de un hombre muy malo llamado Jalen Fellwood. Si estoy en lo cierto (y por lo general tengo razón) la Mara lo utilizará como el punto focal que necesita antes de manifestarse.

--¿La Mara? --dijo Taskia-- ¿Es ese el nombre de la cosa a la que nos enfrentamos?

--Sí.

--¿Qué quieres decir con ``manifestarse'', Doctor? --preguntó Fergin--. ¿Significa que le devolverá la vida al esqueleto?

--Lo utilizará como un marco, pondrá carne en los huesos y, con el tiempo, ganará suficiente poder como para adoptar su verdadera forma.

--¿Cuál es su verdadera forma? --preguntó Taskia.

El Doctor agitó una mano en el aire como si estuviera dirigiendo una orquesta. 

--Bueno, estrictamente la Mara es una entidad gestalt, tiene muchas formas, y ninguna. Proviene de los lugares oscuros del interior, o eso es lo que los Kinda solían decir.

--¿Los Kinda?

--No importa. El hecho es que cuando quiere hacer una entrada espectacular, cuando realmente presume, la Mara se convierte en una enorme serpiente roja... bueno, más bien de un color cereza, la verdad --levantó las cejas, nada impresionado--. Personalmente creo que tiene complejo de inferioridad.

Taskia se inclinó hacia adelante y le tocó la mano, como para mantenerlo concentrado. 

--Entonces, ¿qué hacemos con el esqueleto cuando nos hayamos hecho con él, Doctor? ¿Y cómo derrotaremos a la Mara?

El Doctor refunfuñó. 

--Preguntas, preguntas, resolvamos los problemas de uno en uno, ¿de acuerdo?

--¿Quieres decir que no lo sabes?

--Bueno... todavía no --admitió y, a continuación, su flácida y arrugada cara se iluminó con una sonrisa encantadora--. Pero cuando llegue el momento estoy seguro de que se nos ocurrirá algo.
