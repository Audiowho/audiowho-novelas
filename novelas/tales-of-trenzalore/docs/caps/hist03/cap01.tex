Después de tantos años de prospección en las zonas más salvajes de Terrallende, Tiberio y Yalala estaban acostumbrados al frío extremo. Pero en las profundidades del invierno, la temperatura del aire podría caer aún más bajo, y las frecuentes ventiscas acentuaban esto con una aterradora sensación térmica.
 
Tiberio a menudo pensaba que debería llevar a su hija de vuelta a Navidad, al menos hasta que llegara la primavera, sin embargo esto nunca fue una prioridad. En los cuarenta años que llevaba buscando en Terrallende, aún no había encontrado la veta madre, lo que hacía que estuviera más que decidido a seguir intentándolo. Sabía que había metales preciosos aquí porque había encontrado fragmentos en los depósitos aluviales, donde habían discurrido una vez los antiguos arroyos, por lo que como una regla autoimpuesta él no se retiró ni aún en los más amargos vientos del invierno. Y si la joven Yalala de 8 años alguna vez se quejó de ello, que rara vez lo hizo para su inmenso crédito, su respuesta habitual era envolverla más profundamente en pieles y lana. En última instancia, por supuesto, y a pesar de su tierna edad, Yalala disponía de un par de manos útiles de las que no podía prescindir, ni siquiera temporalmente. Era lo suficientemente fuerte como para manejar un picahielo y una pala, y él le había enseñado cómo empacar y colocar una carga de boronita.
 
No es que tener una asistente tan joven y dispuesta significara que la vida aquí fuera menos que una implacable historia de privaciones.
 
Tiberio a menudo los imaginaba como un par de exploradores de los viejos tiempos coloniales de las Américas. Nunca había visitado la Tierra, por supuesto, pero había visto imágenes en los libros de historia. Una gran diferencia era la noche casi perpetua de Trenzalore, que nunca habría sido un problema para los antiguos pioneros americanos, aunque ellos a menudo habían gozado de la ventaja de tener caballos o mulas para llevarse a sí mismos y a sus equipos. Tiberio y Yalala no se beneficiaban de tales bestias de carga, sólo tenían a Howzi y Mowki, sus infatigables perros de nieve, aunque ninguno parecía especialmente canino cuando retozaban a través de la nieve en sus pantalones, túnicas y capuchas especiales con forma de perro, que Tiberio había hecho para ellos con dos alfombras bellamente estampadas.
 
Fueron Howzi y Mowki quienes vieron por primera vez los objetos cayendo del cielo.
 
Estaban empujando el trineo lentamente a lo largo del Cañón de los codos del Diablo, un pasaje en forma de Z que dividía en dos la congelada cresta rocosa llamada Cresta de Fafnir, y la apertura al sur del gran bosque de pino-luna conocida como la Tundra-Vald. Habían salido del punto más estrecho del cañón, el Ahorcador, cuando los perros se pararon en seco. El cálido aliento les salía por sus enguantados y coloreados hocicos y sus redondos ojos no pestañeaban.
 
--¡Vamos, muchachos! ¡Vamos! --gritó Tiberio tensando las riendas.
 
Los perros estaban demasiado distraídos como para obedecer, lo que hizo que Tiberio siguiera el destino de sus miradas hacia el cielo negro de seda, donde, muy fugazmente, seis objetos curiosos se estrellaban más allá de la cara de plata de Soror, la más pequeña de las lunas gemelas de Trenzalore.
 
--¿Y qué podría ser esto? --murmuró.
 
Hablaba para sí mismo, pero Yalala, al tanto como siempre, lo escuchó. Se puso de rodillas en medio de las mantas y bufandas agolpadas en la parte trasera del trineo, y pudo ver fugazmente la última silueta descendente antes de que esta desapareciera de la vista.
 
Había habido seis de ellos en total, pensó Tiberio. Cosas extrañas, sin forma reconocible, pero por su dirección y el ángulo de su descenso, bajaban por la Tundra-Vald. Se apartó la bufanda con un tirón para poder rascarse su barba rala de color gris, un hábito común cuando estaba agitado.
 
Otra razón para mantener a su hija en las Tierras Lejanas era el peligro que parecía acechar los alrededores de la ciudad. A pesar de sus coros y hogueras, su vino caliente y alegres reuniones, Navidad era segura únicamente para visitarla fugaz y esporádicamente en aquellos días. No importaba mucho lo que sus habitantes pensaran del bicho raro, reparador-de-juguetes, de la Torre del Reloj, o lo exitoso que supuestamente era en deshacerse de estas amenazas inexplicables e incontables provenientes de las estrellas, en algún momento fallaría. Aunque llevaba residiendo en Trenzalore por más tiempo del que Tiberio llevaba vivo, el Doctor (como se le llamaba) se iba marchitando entre la edad y el estrés, y en un conflicto u otro ya había perdido la pierna izquierda. Por el contrario, las Tierras Lejanas, con todas sus bestias salvajes y fríos peligros, era un lugar más seguro en muchos aspectos, y aún así, ahora había un misterio.
 
Navidad se encontraba a unos cuarenta kilómetros al sur de allí, más allá de las tierras salvajes. A pesar de esto, la lógica sugería que estos nuevos objetos celestiales debían estar conectados con todos aquellos que caían ocasionalmente, y aún así, el instinto de Tiberio le llevaba la contraria. Y él confiaba en su instinto (aunque le hubiera llevado de un risco a una planicie y a una cueva llena de hielo, sin ofrecerle nada más que unas baratijas brillantes como fruto por todos sus esfuerzos).
 
--¡Vamos! --gritó una vez más tirando de las riendas.
 
Los perros de nieve prosiguieron la marcha. Medían cinco pies de alto hasta los hombros y siete pies de largo, desde el hocico hasta la cola, eran una raza específica de Trenzalore, mucho más resistente que cualquiera de sus primos de la Tierra, y su doble capa de pelo les permitía resistir los peores estragos de los peores inviernos. Un kilómetro y medio más adelante, antes de que el trineo hubiera dejado atrás los Codos del Diablo, los matorrales terminaron y una tormenta de nieve los golpeó. Intensas ráfagas de copos de nieve tapaban las matas de pinos-luna y las imponentes rocas talladas por el viento que allí se alzaban, pero el equipo siguió valientemente hacia delante.
 
Estaban aún más expuestos en la llanura más allá del cañón, donde soplaba un feroz viento del oeste. Yalala se enterró bajo más mantas y colchas. Tiberio se inclinó sobre el manillar, empapado de los pies a la cabeza. Este era el tipo de resistencia que se veían obligados a mostrar a lo largo de su exilio auto-impuesto a las Tierras Lejanas, y en poco menos de una hora habían llegados a las afueras de la Tundra-Vald, donde el oeste disminuía hacia otros matorrales (nombre dado en Trenzalore a los breves momentos de calma entre tormentas), y aunque copos de nieve del tamaño de plumas de ganso seguían cayendo a su alrededor, el frío más intenso ya había pasado. Condujeron a través de varios claros, donde las rachas de nieve llegaban a la mitad de la altura de los pinos.
 
--¡Whoa, por allí, whoaaa!
 
Howzi y Mowki aminoraron hasta detenerse.
 
Tiberio miró hacia arriba sin comprender, donde tiras de tela colgaban en las ramas que había a su derecha. Quitandose el guante de cuero crudo, palpó la pieza más cercana. Era suave y flexible. Seda, pensó. En pocas palabras, estaba excitado. Podría intercambiarla en Navidad cuando volviera de visita. Normalmente lo que solía intercambiar en la ciudad eran pieles, lo que hizo que convirtiera la caza en una más de sus actividades habituales. Aquello le proporcionaba los suministros necesarios, pero la seda era un bien escaso en este planeta y tendría mucho más valor. Podría permitirse adquirir algo de boronita nueva.
 
Entonces, sintiendo una presencia, miró hacia la izquierda.
 
Nada se movía más allá de la superficie del claro iluminada por la luna. Los árboles más cercanos eran puntales oscuros, las otras matas formaban un laberinto silencioso en la perenne noche.
 
--Yalala --gruñó--. ¡Quédate aquí!
 
Su pálido rostro se asomó a través de un desgarrón en el haz de las mantas, mientras se ataba las raquetas de nieve a los pies forrados de piel y se alejaba laboriosamente en la distancia.
 
Yalala siempre obedecía. En primer lugar, porque no había conocido ninguna otra opción más que seguir las secas instrucciones de su padre. En segundo lugar, porque sabía que era vital si quería sobrevivir. Este frío reino prohibía la vida a todo el mundo salvo a los más inteligentes y robustos. La población de Navidad sólo se aventuró a salir allí cuando era necesario, y en pequeños grupos de elegidos. Incluso entonces había veces en las que no regresaban, perdidos en tormentas de nieve, caídos en zanjas o barrancos escondidos. En una ocasión, ella y su padre se habían tropezado con sus restos, conservados en hielo, con las heridas heladas más terribles que cualquier otra cosa que hubiera visto en sus peores pesadillas.
 
Si ella no quería correr la misma suerte, era crucial que siguiera las reglas de su padre. Pero tras casi dos horas, todavía no había señales de él. La ventisca había cesado por completo, y había un silencio mágico con la luna brillando en el blanco profundo y fresco, trazando ejes a través del negro enrejado de los pinos.
 
Howzi y Mowki yacían a su lado, escarbando bajo la superficie, prácticamente envueltos el uno en el otro. La miraron con curiosidad cuando finalmente se puso de pie en la parte posterior del trineo, y se detuvieron a escuchar. De nuevo, nada se movió. El único sonido era el silbido lejano del viento sobre las capas de hielo. Pensó en llamar a su padre, pero cada parte de su cuerpo estaba envuelto, salvo sus ojos, y no quería quitarse el silenciador de la boca, ya que sus labios se secarían y cuartearían a una velocidad indecente.
 
Por fin, se puso sus propias raquetas y salió. Lo menos que podía esperar era una regañina de su padre. Pero había otros temores más acuciantes.
 
¿Y si se había hecho daño? ¿Y si se había quedado atrapado?
 
No se había aventurado más de cinco metros, cuando los perros comenzaron a gemir y gruñir. Al mismo tiempo, hubo una especie de bandazo a su derecha. Se dio la vuelta y sus ojos casi se salen de sus cuencas cuando un gran bloque de nieve sólidamente compactada apareció ladeandose lentamente hacia arriba para encararla, alcanzando finalmente una altura de quizás dos metros. Mientras lo miraba, vió partes que se desprendían, dejando al descubierto el esquema básico de un hombre: un torso, extremidades y una cabeza, aunque estos eran tan apelmazados que eran casi indistinguibles. Solo cuando avanzó pesadamente, su pesado dibujo hizo crujir la congelada superficie, los vestigios finales de nieve gotearon, permitiéndole ver el rostro de debajo.
 
Yalala no se molestó en quitarse la bufanda. A pesar de que se mantuvo firme en su lugar, sus aterrorizados chillidos podían ser escuchados lejos, a lo largo del páramo invernal.
