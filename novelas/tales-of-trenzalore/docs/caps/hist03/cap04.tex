Una hora más tarde, la dirección del viento cambió, según la carta de navegación de Caleb, lo que les permitió virar hacia los Codos del Diablo a un mayor ritmo.
 
A medida que atravesaban el cañón en S, los miembros de la tripulación que no se necesitaban en cubierta fueron saliendo de todos modos para ver con asombro cómo las paredes de la empinada montaña se extendían hacia todas direcciones y cómo sus flancos nevados se veían sólo interrumpidos por las manchitas oscuras de los pinos. Al llegar al Degolladero, un pasillo particularmente estrecho que recorría medio kilómetro en una línea tan recta como una flecha cuyo punto más profundo no tenía más que cuarenta metros y del cual sobresalían salientes de roca recubierta de densa y blanca maleza, la velocidad del viento descendió significativamente y el Salvavidas aminoró su velocidad.
 
A veces en el Degolladero, como en esta ocasión, el progreso se ralentizaba tanto que la tripulación tenía que desembarcar, correr a la parte delantera con sus raquetas y remolcar la embarcación. Esta tarea no era tan dura como podía parecer en un principio, la nieve estaba suave y el barco, que por lo general llevaba impulso, se deslizaba con facilidad. Pasado el Degolladero, recogieron las cuerdas de nuevo y los hombres se volvieron a subir a bordo, justo a tiempo para que el viento del sur los bajara a toda velocidad por las cuestas del norte de la Cresta, volviera a inflar las velas y a empujar la nave hacia adelante, ahora que el oscuro bastión de la tundra se extendía varios kilómetros ante ellos.
 
Se dirigieron hacia ella en una línea recta delineada por las huellas que Yalala Gluck imprimió al volver a casa, las cuales, aunque estaban parcialmente borradas por la nevada reciente, todavía eran visibles desde la proa del Salvavidas como unos surcos redondos a la luz de la luna. Cuando llegaron a las afueras de la arboleda, el viento de cola aminoró y cubrieron los últimos cientos de metros a una velocidad menor.
 
--Nos detendremos pronto --gritó Caleb mientras comprobaba su carta eólica--. El suelo es inestable. Así que a partir de aquí tendremos que ir a pie.
 
--No nos pasará nada --respondió el Doctor, escaneando la oscuridad invasora de los árboles y añadiendo entre dientes--. No creo que nos quede mucho.
 
--¡Arriad las velas! --gritó Caleb, y la tripulación obedeció.
 
Justo en frente de ellos, la tundra se abrió como un par de cortinas verde vejiga, mostrando un claro cubierto de nieve más o menos del tamaño de un campo de fútbol. A medida que avanzaban lentamente hasta la zona central, el Doctor ya intuía que las cosas no eran como deberían ser.
 
El barco, al plegar las velas, aminoró la velocidad hasta detenerse. El equipo fue a cubierta armado con las pocas armas que traían. Bajaron la pasarela, y en una sola fila, descendieron escaneando con los ojos la blancura lisa que se extendía en todas direcciones. Del mismo modo que la llanura entre la tundra y la Cresta de Fafnir había revelado rastros del viaje de Yalala, este claro también revelaba el rastro de algo más, varios cráteres en la superficie daban indicio de un largo altercado. Unas huellas débiles pero firmes daban a entender que el trineo de Tiberio había volcado de forma abrupta, supuestamente cuando Yalala estaba tratando de escapar.
 
El Doctor miró hacia los curiosos objetos que había visto sobre un árbol antes de que llegaran: varios hilos de seda deshilachada, junto a un borujo de cuerdas colgando sobre las ramas. Cuando miró más lejos, divisó más harapos colgando de otros árboles.
 
--¿Qué son? --preguntó Luca, fascinado y asustado al mismo tiempo.
 
De cabello rubio y rostro fresco, Luca era el miembro más joven de la tripulación del Salvavidas. Había entrado ese mismo año y no tenía experiencia, hasta ahora, con la vida de fuera de Navidad.
 
--Paracaídas --explicó el Doctor, mientras se reunía con la tripulación--. Ingenioso --se toco la barbilla--. Como no pueden penetrar la fuerza Papal con sus motores interestelares, se dejan caer. ¿Quién lo hubiera pensado? Tened en cuenta que de la Exosfera de Trenzalore a la superficie del planeta hay poco más de seiscientos cincuenta kilómetros. Eso es todo un viaje, la mayoría en caída libre, yo no me atrevería a intentarlo. Ajá...
 
Había descubierto algo una docena de metros a su izquierda, y, cojeando, se desplazó hasta allí. Sólo había una parte visible, sobresaliendo sobre la nieve, aunque cuando se pararon a observar, divisaron otras piezas medio cultas. Se asemejaban a los fragmentos de una gran cáscara de huevo, los cuales ante la pálida luz de las lunas parecían estar cubiertas de baldosas de cerámica cuya mayoría estaban rotas y ennegrecidas. El Doctor se puso en cuclillas y apartó algo más de nieve.
 
--Un metro de longitud y poco más de diámetro cuando estaba intacto... --se puso de cuclillas para echarle un vistazo al interior, pero no vio nada excepto una superficie metálica e impoluta--. No hay controles, ni aislamiento... ni comodidades de ningún tipo --olisqueó el aire--. Es una noche fría por supuesto, pero no capto ningún aroma que me sugiera la presencia de componentes químicos o biológicos. Por lo tanto es un recipiente ignífugo. Un simple paquete... entregado por un cartero cósmico muy desagradable.
 
--¡Hay más por aquí! --gritó alguien.
 
--Habrá unos cuantos más, me imagino --dijo el Doctor--. Esta será una de las cápsulas en las que se pusieron en órbita al principio.
 
--¿Quiénes se pusieron en órbita? --preguntó Caleb.
 
El Doctor se metió las manos en los bolsillos del pantalón.
 
--Los que desembarcaran aquí. Necesario, supongo... especialmente después de que la órbita comenzara a decaer y penetraran en la Ionosfera, de lo contrario se habrían calcinado por completo. Me parece que les dolió, las cápsulas probablemente se separaron al entrar en la Termosfera, y a partir de dicho punto los pasajeros quedaron completamente expuestos.
 
--¿Nos estás diciendo que esos seres literalmente cayeron del espacio? --preguntó Caleb.
 
El Doctor asintió.
 
--Increíble, ¿a que sí? Los métodos más simples suelen ser los mejores. También explica cómo se desviaron tanto de su curso. Se tarda algún tiempo en acostumbrarse a los vientos cruzados de Trenzalore.
 
--Pero eso es imposible.
 
--Para ti o para mí, sin duda. Pero es asombroso lo que se puede lograr cuando no se padece ningún dolor.
 
El pequeño grupo observó al Doctor en completo silencio.
 
--Así que... ¿sabes quiénes son? --preguntó finalmente Caleb.
 
--Tengo algunas teorías --respondió el Doctor--, pero no nos desviemos del tema, ¿eh? Estamos aquí para encontrar a Tiberio Gluck, no para entretenernos con conjeturas. A ver... llegamos por el sur, así que sabemos que actualmente no hay nadie allí que no debería estar. Eso sólo deja los otros tres puntos cardinales de la brújula. Así que... Yoshua y Rubin, iréis al norte. Josef y Jerema, al oeste, y Caleb y yo iremos al este. Luca, eso te deja a ti protegiendo el Salvavidas. Es mejor que te quedes en lo alto de la pasarela. Así tendrás una ventaja si alguien viene hacia ti.
 
El joven, que sólo iba armado con una lanza casera que usaba normalmente para pescar en el hielo, parecía más que un poco nervioso.
 
--Doctor... si estas personas, sean quienes sean... ¿si no pueden sentir dolor?
 
--Eso no suena bien, lo admito --dijo el Doctor--, pero al contrario de la creencia popular, Luca, ninguna clase de resistencia es fútil.
 
--Pero... es que... --el chico no paraba de tartamudear--. Si tienen tus pintas...
 
--No pasa nada --el doctor esbozó una sonrisa infantil--. Esos tíos tan sólo se han sumergido varios cientos de kilómetros en un montón de gases densos y radioactivos, por no mencionar las tormentas. Así que sólo puedes estar seguro de una cosa, Luca: ninguno de ellos va a ser igual que yo. Bien... --se frotó las manos--. Buscaremos abriéndonos en círculo, pero si llegamos al punto donde no nos veamos ninguno de nosotros, nos mantendremos en contacto soplando nuestros cuernos, ¿entendido?
 
Ellos asintieron sombríamente.
 
--Tenemos aproximadamente cuarenta y nueve minutos antes de que termine el punto muerto y el viento del norte de la tarde nos impulse rápido a casa. No queremos perder el barco, ¿verdad?
 
Ellos negaron con la cabeza.
 
--Entonces, ¿a qué estamos esperando? Encontremos a Tiberio Gluck.
 
Lo encontraron.
 
Diez minutos después.
 
En el claro siguiente.
 
Estaba apoyado contra un tronco de pino, parcialmente cubierto de nieve y su mochila a un par de metros. Afortunadamente, fue Yoshua, el miembro más viejo y robusto de la tripulación, quien encontró al hombre y le sacudió el hombro para ver si volvía en sí. La cabeza colgaba a un lado sobre un cuello tan roto que parecía más goma que músculo, y la cara había sido golpeada hasta quedar irreconocible.
 
Nadie más del resto de la partida se acercó, salvo el Doctor. Su rostro estaba grabado con un profundo y enfadado ceño fruncido. Encontraron otro cuerpo. Esta segunda figura adoptaba la forma básica de un hombre, y de hecho llevaba prendas de ropa, junto con un arnés de cuero del cual colgaban varias cuerdas. Pero parecía haberse roto al entrar en contacto con las ramas superiores del pino, porque había numerosos fragmentos suyos esparcidos por los tallos inferiores. El cuarto superior izquierdo de su torso, que todavía conservaba el brazo y la cabeza,  estaba suspendido al revés cerca de la parte de abajo, su cabeza horriblemente deformaba pivotaba de un lado a otro y todo el pelo se le había chamuscado de raíz.
 
El Doctor dio un paso al frente para mirarle la cara, que vagamente se parecía a la suya, aunque los ojos le habían explotado dentro de las órbitas y toda su mitad estaba horrorosamente destrozada.
 
--¿Es... es uno de ellos? --preguntó Caleb horrorizado.
 
--Me temo que sí --respondió el Doctor.
 
--Parece... artificial.
 
--Es artificial --dijo el Doctor--, aunque tal palabra no le hace justicia a un frío e insensible maniquí al que en sus buenos tiempos los poderes telekinéticos de una forma de vida alienígena desagradable y vengativa habrían animado para obrar asesinatos.
 
--Pero, ¿qué es?
 
--Tienen muchos nombres, Caleb, depende del sistema solar. Pero en el mundo de tus antepasados, los conocíamos simplemente como... Autons.
