--Uno, dos, tres... --los ojos del Doctor se movían de izquierda a derecha por el techo arqueado--. Cuatro.
 
No había ninguno más.
 
Sintió una punzada de inquietud. ¿Sólo cuatro? Yoshua había deducido cinco.
 
Un golpe.
 
El Doctor se dio la vuelta. El último había venido de proa. Sonó como si se hubiera quedado enredado primero en un aparejo. Percibió más movimiento: un arrastrar y golpear de pies, los Autons revolviéndose, inspeccionando la cubierta desierta y adentrándose de uno en uno en la escotilla que todavía seguía abierta, la escotilla de popa. Avanzaron por ella arrastrando los pies sin vida.
 
La madera crujía bajo sus pies. El Doctor subió a toda prisa la escalera que le llevaba a la escotilla de proa. Mientras lo hacía, vislumbró por el rabillo del ojo que algo contorsionado y destrozado estaba apareciendo al otro extremo de la bodega.
 
Con mucho esfuerzo, sacó los pernos, levantó la escotilla y consiguó salir, cerrando las puertas tras él en silencio. A partir de aquí, la escotilla de popa estaba oculta por el puente, pero todavía podía ver el último de la partida de persecución, una forma quebrada y coja en la penumbra glacial, agachándose mientras descendía a la bodega. La última parte de éste que perdió de vista fue su mano izquierda, con la cual agarraba una rama nudosa y llena de espinas.
 
Una porra.
 
Autons con porras.
 
Parecían más aterradores que Autons con pistolas láser automáticas.
 
Se quedó agazapado, conteniendo la respiración, hasta que escuchó el martilleo frenético y estrepitoso bajo la cubierta, los Autons atacando a los bultos de las camas.
 
Sobresaltado por la acción, lo primero con lo que tuvo que lidiar fue con la escotilla de proa. No se podía bloquear desde arriba, pero poseía dos manillas anilladas que atrancó metiendo una barra de acero inoxidable entre ellas. Para asegurarse, abrió uno de los barriles de suministros que había sobre una repisa a lo largo de toda la borda. Suministros de emergencia en caso de que el Salvavidas quedara encallado, estos estaban hasta arriba de cerdo y pescado salado o galletas, así que necesitó la misma determinación y fuerza para remolcarlo por la cubierta y colocarlo encima de la escotilla cerrada.
 
El caos abajo era todavía ensordecedor. Para ser criaturas de plástico y sin sentimientos, los Autons desataban una furia casi demoníaca contra sus posibles víctimas. Desconcertado, el Doctor recorrió la cubierta cojeando hasta llegar a la escotilla de popa, que también cerró, y luego desenredó una línea de dirección desde la botavara colgante, la ató alrededor de las manillas lo máximo que pudo, y la pasó por un ojo de buey hasta llegar a la cubierta de estribor, donde la anudó.
 
Por supuesto, las escotillas estaban hechas de madera, nada más. Al final, los Autons las acabarían destruyendo para abrirse camino. Cuánto tiempo les tomaría era una incógnita.
 
El Doctor cojeó hacia el puente de mando, y soltó el timón.
 
Entonces se dio cuenta de la lentitud con la que viajaban. Un par de nudos como máximo. La vela mayor apenas se sacudía. Al menos, con el Degolladero finalmente ensanchandose, comenzaba a llegar la luz de las lunas. Miró con envidia las otras velas, lo bien que estaban recogidas. Aunque supiera cómo desenrollarlas bien, sólo era uno, no podría hacerlo.
 
En su lugar, se puso a los mandos del timón y apretó los dientes.
 
Todavía se podía oír una cacofonía viniendo desde abajo, pero se había reducido notablemente en cuanto los Autons se dieron cuenta de que los habían engañado. Tal vez un minuto más tarde, las porras comenzaron a atacar la parte interior de las escotillas.
 
Afortunadamente, la embarcación continuaba deslizándose.
 
--Cuando estéis listos --gritó el Doctor, levantando la vista hacia la vela mayor caída.
 
Pero pasaron otros cincuenta metros antes de que los laterales del desfiladero comenzaran a replegarse de nuevo, y la vela mayor comenzara a ondear otra vez, y a hincharse. El Doctor no cantó victoria. El Salvavidas estaba acelerando, pero muy lentamente. Mientras tanto, seguían dándole golpes a los enveses de las escotillas, cada vez con más fuerza y ferocidad. Pero se alivió al sentir esa vibración en la base del mecanismo de dirección. Cuando el timón viró a estribor, tuvo que contenerla con fuerza. La velocidad del barco se incrementó y el paisaje comenzó a moverse junto con los copos de nieve. De doce a quince nudos, estimó. Una aceleración uniforme, de quince a veinte, esto ya estaba mejor.
 
Rodearon el segundo codo a un ritmo cada vez mayor, la embarcación se estaba arrastrando otra vez, la botavara se estaba balanceando sin parar y el Doctor estaba intentando girar a babor, cuando, a menos de un kilómetro de distancia, avistó la gran brecha en forma de V que había al final sur del cañón de los Codos del Diablo.
 
A partir de allí eran llanuras abiertas, en cuyo extremo opuesto, a unos cincuenta kilometros de distancia, yacía la ciudad. Pero no iban a llegar tan lejos, ni siquiera en esa dirección.
 
Su destino, aunque no le había revelado esto a los demás (uno no podía estar del todo seguro de que no había facsímiles a su alrededor), quedaba hacia el oeste, y el barco apenas había salido a campo abierto, cuando el Doctor lo vio: un resplandor de luz de estrellas sobre una extensión enorme y completamente plana que abarcaba hasta donde alcanzaba la vista.
 
El lago Lagda.
 
Todavía quedaban unos minutos, pero el viento del sur estaba soplando cada vez con más fuerza y desplazando a la nave con furia. El Doctor miró hacia el timón. A pesar del barril, la escotilla se estaba resquebrajando y astillándose. Un puño ennegrecido la perforó por el medio.
 
--¡No, gracias! --gritó el Doctor. Saltó del puente, agarró una de las lanzas caseras de la tripulación y atacó la apertura una y otra vez, con una fuerza violenta, y repetidamente, sin infligir dolor y con la esperanza de que los contuviera. Sólo necesitaba un par de minutos más, nada más. Pero su enemigo no dio su brazo a torcer y forcejeó enérgicamente con el arma hasta que pilló la punta y la partió en dos.
 
Un crujido de madera similar resonó desde la escotilla de popa.
 
El Doctor se dio la vuelta y se quedó pálido cuando el silbido de la nieve de debajo de los patines del Salvavidas se transformó en una fuerte sacudida, y los ruidos que hacían al avanzar cesaron de forma abrupta. De repente comenzaron a correr, y a una increíble velocidad.
 
El Doctor se asomó por la borda. Una superficie brillante se extendía en todas direcciones. Le echó un vistazo a la parte trasera, donde la luz de la luna se desvanecía enseguida.
 
Soltó una carcajada mientras volvía cojeando al timón, se dejó caer de cuclillas y ató la mecha que había enhebrado alrededor de la base del timón. A continuación, abrió una de las cajas de cerillas de Tiberio Gluck y descubrió que no era fácil prenderlas con esta ventisca. Una detrás de otra, se le rompieron tres cerillas y no prendieron, cuatro, cinco seis. Justo en ese momento, con un estruendo, el barril se hizo a un lado, y el Doctor vio cómo una inmensa figura intentaba avanzar más allá del mastil de acero.
 
--¡Vamos, vamos! --murmuró.
 
Las cerillas séptima, octava y novena también se rompieron, y ahora sólo le quedaba un par. Pero la décima prendió y comenzó a emitir un chorro azul que acercó al final de la mecha. La cuerda empezó a silbar, mientras el Doctor giraba sobre sus pies y se precipitaba por la borda de estribor, impulsando primero su pierna buena y, con un grito salvaje de ``¡Geronimoooo!'', arrastrando la de madera después.
 
Aterrizar en hielo a tal velocidad no era algo que tuviese muy planeado.
 
Lo golpeó como un mazo, enviándole sacudidas de dolor y nauseas por todo su cuerpo. Rodó durante unos treinta metros, como una masa de miembros desgarbados, chocando y rebotando contra la superficie de roca sólida, y deslizándose bocabajo durante otros veinte, antes de alzar la vista aturdido y ver cómo el Salvavidas se alejaba rápidamente.
 
El resplandor que se produjo cuando la sección inferior estalló fue cegador.
 
Entonces detonó, mandando astillas ardiendo en todas direcciones y envolviendo al Doctor mientras se cubría la cabeza con las manos. Bajo su cuerpo, el lago congelado se estremeció, y luego se fracturó por todas partes. De hecho, lo siguiente que supo el Doctor era que estaba tumbado sobre un bloque flotante que rápidamente se fue alejando, y que el agua congelada le estaba empapando el brazo izquierdo. Clavó los dedos en la superficie glacial, antes de que se volviera a enderezar.
 
A su alrededor yacía un mar de trozos de hielo inestables que reflejaban con ferocidad las llamas naranjas que salían de la cáscara carbonizada del Salvavidas, el cual se había detenido en seco a ochenta metros más allá y se estaba hundiendo ahora por la proa.
 
El Doctor lo observó en tensión, rezando por que desapareciera.
 
¿Y si no funcionaba?
 
¿Y si el peso de ese tren de aterrizaje, esa masa de hierro enmarañado y retorcido no lo hundía?
 
Con un fuerte burbujeo, la embarcación comenzó a hundirse de verdad, su popa se elevó en el aire, dejando a la luz los trozos destrozados de madera que una vez fueron su quilla.
 
La esperanza resurgió en el Doctor, pero había una cuestión que lo fastidiaba, ¿los Autons sabían nadar? Supuso que eso dependía de lo quemados y desmembrados que estuviera tras la explosión. Incluso aunque no estuvieran en tal estado, estaban hechos de plástico macizo y eran demasiado pesados para flotar, y el Lago Lagda era monstruosamente profundo, una falla geológica en vez de un valle inundado que descendía cientos y cientos de metros y estaba contenido por empinadas paredes de barro. Mientras observaba los restos del naufragio, las llamas de su interior se fueron apagando, empapadas por el agua, hasta que no quedó más que el humo, reduciendo la nave a un silueta rota y ennegrecida. Entonces las lunas volvieron a inundar el paisaje con su luz plateada. El Doctor se arrodilló, su piel se erizó cuando la popa del barco se detuvo encima de la superficie. Su barra de hielo se tambaleó precariamente, pero solo durante unos segundos. Finalmente los restos del naufragio se hundieron de nuevo.
 
Casi se permitió relajarse un instante, hasta que vio algo más.
 
Una figura apareció en la popa, encaramado como un mono en la borda.
 
A pesar de que el último fragmento de madera desapareció bajo la superficie turbulenta, la forma desgarbada saltó lejos, aterrizando ágilmente sobre un fragmento de hielo, zarandeándola de lado a lado. A continuación, saltó de nuevo. Ágil como una rana, aterrizó a cuatro patas en un segundo fragmento.
 
--No puedes estar hablando en serio --dijo el Doctor lentamente.
 
En tres saltos, se había acercado casi treinta metros a él.
 
Saltó una cuarta vez, otra vez aterrizando con éxito. Ahora estaba mucho más cerca. Se preguntó qué clase de parodia pesadillesca de sí mismo representaría éste. Primero, había caído de la atmósfera, y ahora había quedado chamuscado por las llamas. No quedaría mucho de sus características Gallifreyanas.
 
Con un quinto y verdaderamente prodigioso salto, alcanzó otro fragmento, esta vez golpeando la superficie congelada con un golpe seco audible y aterrizando a un lado de la losa, inclinándola. La forma negra del Auton se aferró a ella, pero la losa continuó inclinándose, deteniéndose brevemente en posición vertical, antes de darse la vuelta completamente, su parte inferior marrón brillando bajo la luz de las estrellas.
 
--Ahhh, bueno... menuda suerte --el Doctor se apoyó en su bastón para enderezarse--. Bailar sobre hielo no es para todos. Los mejores planes y todo eso.
 
Se dio la vuelta. El viento azotaba violentamente a su alrededor y la escarcha comenzó a materializarse en la piel empapada de su manga izquierda de la chaqueta. Sintió como se entumecía su mano izquierda. ``Sabañones'', pensó con asombro. Siempre hay una primera vez para todo en Trenzalore. En el lado positivo, el terrible frío significaba que no pasaría mucho tiempo antes de que la superficie destruida solidificara de nuevo, lo que le permitiría cojear de vuelta a la orilla. Golpeó un par de veces el hielo con su bastón. Definitivamente estaba fusionándose de nuevo.
 
Fue entonces cuando el Auton saltó del agua a su derecha.
 
Era una incógnita cómo había atravesado los últimos cincuenta metros bajo la superficie. Posiblemente trepando por otros trozos de hielo hundidos o tal vez enganchando sus dedos inertes en su parte inferior coagulada. Pero salpicando aguanieve y barro, aterrizó con los brazos abiertos en el lado derecho de la propia isla privada del Doctor, poniéndose de lado de una manera tan espectacular que éste tuvo que arrojarse sobre su parte trasera para reequilibrarla.
 
De cerca, el maniquí era totalmente el horror que había imaginado: una efigie contorsionada y destrozada. No quedaba ni un trozo de ropa, ni un mechón de pelo, ni una pizca de rasgos faciales en medio de las masas brutales de plástico quemado. Nada salvo un solo ojo, en parte desalojado de la única órbita restante pero fijo en él con una intensidad similar al láser.
 
El ojo del Nestene.
 
Varias veces luchando contra los Autons, se había enfrentado con el ojo del Nestene. Por lo general era artificial y sin embargo siempre había proporcionado una ventana transparente a la maldad sin alma que se escondía debajo.
 
Fue al ojo donde el Doctor atacó.
 
En verdad, no sabía si el órgano de imitación era sólo un efecto o en realidad servía para transmitir visión al horrible intelecto en el reino del más allá. Pero parecía un lugar tan bueno como cualquier otro, así que se deslizó hacia delante sobre sus rodillas y clavando su bastón cual espada, el órgano se sacudio hasta convertirse en una especie de tejido sintético. El Auton, todavía sujeto con las dos manos, no pudo responder. Clavó el bastón por segunda vez, y el ojo se desprendió por completo, pero esto no impidió que el Auton elevara su rodilla derecha en la losa, y al cambiar el centro de su equilibrio, pudo arrebatarle el palo y capturarlo.
 
El Doctor tiró de su instrumento de nuevo. Al estar cubierto de hielo, se deslizó fácilmente de las manos del Auton. Cuando golpeó de nuevo, lo hizo con ambas manos, impactando en el lado de la cabeza del monstruo, primero por la izquierda y luego por la derecha.
 
Pero estos seres no sienten dolor.
 
Con ambas rodillas sobre la superficie, volvió a ponerse de pie. El hielo se inclinó de nuevo. El Doctor se deslizó hacia un lado. Había un espacio entre aquel fragmento de hielo y el siguiente, pero estaba lleno de agua fangosa y era lo suficientemente amplio como para tragarlo entero. Retrocedio como pudo a toda prisa, apoyando todo su cuerpo sobre la pierna sana, pudiendo así impulsarse hacia arriba, golpeando a su oponente con todo el cuerpo.
 
El Auton era lo suficientemente resistente como para soportar el golpe y asfixiarlo en un abrazo de oso, pero de repente, la losa se inclinó hacia el otro lado y ambos cayeron.
 
El Auton le soltó, pero ya era demasiado tarde. El Doctor consiguió impulsarse y aterrizar de cuerpo entero en la losa de al lado, clavando el bastón en ese mismo lugar para tener algo de estabilidad. El impacto separó aún más los dos losas, ampliando enormemente la fisura entre ellas. El Auton cayó en ella violentamente.
 
Pero aún no lo había derrotado completamente.
 
Su garra retorcida se deslizó por su pernera izquierda, sujetándole alrededor del tobillo. Durante unos segundos eternos, siguieron unidos; el Auton medio sumergido y pataleando en el gélido líquido y el Doctor acostado de medio lado sobre el hielo. La inclinación cada vez más pronunciada de la losa hizo que comenzara a deslizarse.
 
--¡Esto va a ser genial! --jadeó, hurgando con su mano derecha a través de sus múltiples capas de ropa--. Justo lo que necesitaba.
 
Al mismo tiempo, golpeó una y otra vez la cara quemada del Auton con el pie que tenía libre con la esperanza de distraerle y poder localizar y abrir las hebillas. Y lo consiguió.
 
Poco a poco, su pierna artificial se separó de la pernera del pantalón.
 
El Auton, aún aferrandose a ella, se hundió.
 
El Doctor escarbó en el borde hasta que unas cuantas de tiras de cuero salieron del brazalete y cayeron a la superficie. Una fracción de segundo después, las placas de hielo golpearon la parte superior de la misma.
 
--¿Crees que quiero volver a pasar por todo eso... otra vez? --gimió al tiempo que sus dos corazones latían fuertemente en su pecho--.¿Hacerme una nueva pierna?
 
Casi en el último momento, se dio la vuelta sobre su estómago y se abrió paso hasta el borde de la losa, mirando hacia abajo a través del pequeño espacio que había reabierto. Casi esperaba tener un último vistazo de la cabeza sin pelo del Auton mientras descendía hacia el vacío. Pero no vio nada, sólo la oscuridad, y segundos después ni siquiera eso debido a los nuevos cristales de hielo que se formaban rápidamente.
 
A su alrededor, el lago se volvía a congelar, el viento polar hinchaba copos frescos a través de su superficie, volviéndolo blanco. Iba a ser una larga caminata a casa, pensó mientras conseguía ponerse de pie, más bien, un largo cojear.
 
Las cosas que tenía que hacer, pensó con resignación, en nombre de Navidad.
