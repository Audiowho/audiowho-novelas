\section*{Acerca del libro}

Como se había predicho, los ejércitos del Universo se reunieron en Trenzalore. Sólo una cosa se oponía entre el planeta y la destrucción: el Doctor. Durante 900 años, defendió el planeta, y el pequeño pueblo de Navidad, contra las fuerzas que lo destruirían. 

Nunca supo cuánto tiempo podría mantener la paz. Nunca supo qué criaturas emergerían de la nevada noche de nieve para amenazarlo. Sólo sabía que al final moriría en Trenzalore. 

Algo de lo que sucedió durante esos terribles años está bien documentado. Pero la mayoría permanece envuelto de misterio y oscuridad.

Hasta ahora.

Esta es una ojeada de algunos de los terrores que la gente afrontó, las monstruosas amenazas que el Doctor derrotó. Estos son los relatos de los monstruos que se encontraron con miedo, y de aquel hombre que no.

\mbox{}

\textbf{Relatos de Trenzalore} documenta cuatro de las aventuras del Doctor en diferentes periodos durante el Asedio de Trenzalore y la subsiguiente batalla:

\mbox{}

\emph{Deja que Nieve} - por Justin Richards

\emph{Una manzana al día} - por George Mann

\emph{Extraños en las Tierras Lejanas} - por Paul Finch

\emph{Los sueños} - por Mark Morris

\newpage

\section*{Acerca de los autores}

\subsection*{Justin Richards}

Un célebre escritor y consultor creativo de BBC Books de la gama de libros de Doctor Who, Justin Richards vive y trabaja en Warwick con su esposa y dos hijos. Cuando no está escribiendo, puede ser encontrado satisfaciendo su pasión por inventar, leer y ver demasiada televisión.

\subsection*{Mark Morris}

Mark Morris se convirtió en un escritor a tiempo completo en 1988, y un año más tarde vio el lanzamiento de su primera novela, Toady. Desde entonces ha publicado unas nuevas dieciséis novelas, entre las que se encuentran Stitch, La Inmaculada, El Secreto de la Anatomía, Fiddleback, El Diluvio y cuatro libros de la popular gama de Doctor Who.

Sus historia cortas, novelas, artículos y reseñas han aparecido en una amplia variedad de antologías y revistas, y es editor del aclamado Cine Macabre, un libro de cincuenta ensayos de películas de terror por influyentes del género, por el cual ganó el British Fantasy Award 2007. También escribe bajo el nombre de J.M. Morris. Para saber más acerca de Mark Morris visita su página web en www.markmorriswriter.com.

\subsection*{George Mann}

George Mann es el autor de la serie de misterio steampunk Newbury y Hobbes, así como de otras numerosas novelas, relatos y audio-libros originales. Ha editado varias antologías incluyendo.

El libro Solaris de Nueva Ciencia Ficción, El Libro de Solaris de Nueva Fantasia y una retrospectiva colección de historias de Sexton Blake, Sexton Blake, Detective. Vive cerca de Grantham, Reino Unido, con su esposa, hijo e hija.

\subsection*{Paul Finch}

Paul Finch es un ex policía y periodista. Se fue metiendo en el tema literario escribiendo episodios del drama para la televisión The Bill, y ha escrito extensamente en el campo de la animación infantil. Sin embargo, probablemente es más conocido por su trabajo en la fantasía y el horror. Su primera colección, Aftershocks, ganó el British Fantasy Award en 2002, mientras que ganó el premio de nuevo en 2007 por su novela corta, Kid. Más tarde, en 2007, ganó el International Horror Guild Award por The Old North Road. Ha escrito dos audio dramas de Doctor Who para Big Finish; Leviathan y Centinelas del Nuevo Amanecer. Paul vive en Lancashire, con su esposa y sus hijos.

\newpage

\section*{Doctor Who: Historias de Trenzalore}

\centerline{Justin Richards, Mark Morris, George Mann y Paul Finch}

\mbox{}

\subsection*{Historias de Trenzalore}

Como se había predicho, los ejércitos del universo se reunieron en Trenzalore. Sólo una cosa se opone entre entre el planeta y la destrucción: el Doctor. Sólo una cosa se interponía entre el Doctor y la próxima Gran Guerra del Tiempo: su nombre. Durante 900 años, defendió el planeta y la pequeña ciudad de Navidad contra las fuerzas que lo destruirían.

Detrás de la barrera de la tecnología mantenida por la Iglesia de la Orden Papal, en el corazón del Filtro de Verdad, cerca de la grieta entre este universo y el siguiente, el Doctor se mantuvo firme entre la vida y la muerte. Nunca supo cuánto tiempo podría mantener la paz. Nunca supo qué criaturas podrían emerger de la nevada noche para amenazarlo. Sólo sabía que al final moriría em Trenzalore.

Algo de lo que pasó durante esos terribles años está bien documentado. Pero la mayoría de ello
ha permanecido envuelta en misterio y oscuridad. 

Hasta ahora…

Reunidos en este volumen son sólo cuatro incidentes del tiempo en que el Doctor pasó en Trenzalore. Cuatro historias de heroísmo y peligro. Cuatro historias que documentan lo lejos que el Doctor iría a fin de proteger el lugar que él había convertido su hogar. Cuatro de cientos, tal vez, miles.

Con el tiempo, seguramente más historias surgirán sobre cómo el Doctor protegió el pueblo de Navidad, y de cómo la gente del pueblo lo llevaron a su corazón y amaron el tiempo que les compró. Pero por el momento, solo tenemos rumores, y leyendas, mitos e historias.

Relatos de Trenzalore...