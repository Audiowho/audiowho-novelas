El Doctor estaba tallando. No había nada como ponerse a tallar para concentrarse. Cada poco le echaba un ojo a la grieta brillante de la pared, pero seguía siendo la misma de siempre. La misma que había visto hacía más de setecientos cincuenta años, la primera vez que llegó a Navidad.

Su cuerpo – su último cuerpo, y para ser honesto, uno de sus favoritos – había sido joven de aquella. Lleno de vigor y vitalidad.

--Vigor y vitalidad --murmuró, disfrutando del sonido de las palabras--. Vigor. Y vitalidad.

Claro que en aquel momento tenía dos piernas, antes de ese horrible accidente con una Serpiente Tsunami ciega. Algo patituertas, puede que sí, pero se completaban entre sí. Y su mente era aguda. Aguda como un… como un…

Miró al cuchillo de su mano. Su frente se frunció confusa. No le gustaban los cuchillos. Qué cosas tan desagradables. Al menos valían para untar mantequilla. Y mermelada. Encima de las tostadas. Y también valían para…

¡Ah! ¡Para tallar! Claro. Había estado tallando. No había nada como ponerse a tallar para… eh…

--¿Doctor?

El Doctor alzó la cabeza. Había una chica a su lado. Parecía nerviosa, al igual que la mayoría de la gente cuando se aproximaban a la grieta. De hecho, rara vez venían aquí los adultos. Sólo los niños eran los que a veces venían a sentarse, y a hablar, y a beber cacao, y a traer sus juguetes rotos.

Contempló los ojos nerviosos de la chica, su gorro de lana rojo y su largo y grueso abrigo, que estaba punteado con tiras de un material colorido.

--Hola, Amelia --dijo sin levantar la voz.

--Es Mellandine, Doctor --lo corrigió.

--Mellandine. Claro. ¿Qué puedo hacer por ti? ¿Un cohete para tu muñeca favorita? --De repente levantó un brazo y extendió un dedo de advertencia--. Si son gafas de rayos X, me temo que la respuesta es no. Me metí en un montón de embrollos con la Sra. Pilke por eso.

--Hay problemas en el pueblo, Doctor --dijo ella--. En la taberna.

--Oh, caray. ¿No serán los Guerreros de Hielo otra vez? Mira que son grandes, pero no lo pueden ni con su cerveza.

--Es Sylvian Capple, Doctor. Está… actuando de forma extraña. Padre dijo que fuera a avisarte.

Entonces la mente del Doctor se asentó. Así es cómo parecía ocurrir hoy en día. Se pasaba días a la deriva, con pensamientos vagos y confusos, pero a la mínima señal de problemas – ¡bang! Volvía a ser el mismo de siempre. El Doctor. El gran protector. Listo para defender el pueblo de Navidad desde todos los flancos.

En momentos como éste, le entraban ganas de saltar de la silla, sacar su sónico y bajar disparado por las escaleras – pero aunque quisiera, estaba tristemente débil.

Hizo fuerza para levantarse, sacudiendo su escaso pelo blanco como una tela de araña, y se volvió hacia Mellandine, tensando sus flácidos rasgos para formar una sonrisa.

--¿Me alcanzas esa caja, Mellandine? --dijo, señalando con la cabeza hacia una maleta de viaje destartalada, cuya parte superior descansaba contra un pilar de madera--. Me da que hoy voy a necesitar mi sombrero especial de sheriff.



\mbox{}



\centerline{ \Huge *}



\mbox{}



Navidad era un lugar feliz – cuando los invasores no andaban volando en pedazos partes de la ciudad, claro – y la Taberna de Navidad, cuyo dueño era el padre de Mellandine, Fergin Eggleton, era el lugar más feliz de todos. Sus paredes vibraban por el bullicio y la risa que generaban los que entraban en ella; rezumaban cotilleos y trolas y obscenidades. Cada noche, la mayoría del pueblo se congregaba aquí para beber cerveza y a charlar sobre los chanchullos locales.

Pero cuando el Doctor entró por la puerta, con un Stetson agujereado en la cabeza, no fue recibido con una bonhomía regocijante sino con una tensión silenciosa. Todos excepto dos ocupantes de la taberna se habían agolpado contra la pared a mano izquierda, preocupados, incluso con expresiones de terror en sus rostros.

Las únicas dos personas que no se habían acurrucado contra la pared eran Fergin Eggleton y Sylvian Capple. Fergin, un barril alegre y calvo con un gran bigote rojo, estaba en medio del espacio casi vacío, con las manos levantadas en un gesto tranquilizador. La forma con cabeza de fregona de Sylvian Capple, a quien estaba mirando Fergin, estaba de cuclillas contra la pared derecha, plegando su larguirucho cuerpo como un papaíto piernas largas.

A pesar de su posición encorvada, Sylvian parecía de todo menos sumiso. Al contrario, se asemejaba a un muelle, con los músculos de sus delgados miembros completamente tensos, sus ojos, inyectados en sangre mirando hacia todas direcciones, y sus dientes, lo suficientemente prietos para que rechinaran con rabia. En sus manos sujetaba un picahielos cuya curvada cabeza resplandecía tenuemente ante la luz de la lámpara.

Cuando el Doctor entró, Fergin le lanzó una mirada que fue medio de advertencia y medio de gratitud.

El Doctor asintió una única vez, y le sonrió a Sylvian.

--Bueno, Sylvian --dijo sin levantar la voz--, ¿qué pasa aquí, eh? --Se acercó al hombre con cautela, y luego, estremeciéndose de dolor cuando la rodilla que le quedaba crujió, se acuclilló delante de él. Haciendo un gesto hacia el picahielos, dijo--: Más vale que tengas cuidado con eso. Te vas a hacer daño.

La única respuesta de Sylvian fue encogerse más, y un gruñido animal comenzó a emerger por su garganta.

El Doctor enarcó los ojos.

--¿Por qué estás tan asustado? ¿Qué has visto?

--Ha estado teniendo los sueños, Doctor --murmuró Fergin tras él.

--¿Los sueños?

--Últimamente los han tenido muchos tíos. Sueños horribles. Los ha estado trastocando muchísimo.

--¿De verdad? --dijo pensativamente el Doctor. Apoyó su bastón sobre el suelo, se metió la mano en el bolsillo de su abrigo viejo y sacó su destornillador sónico. Lo encendió y jugó con los ajustes hasta que comenzó a parpadear con una luz suave. La levantó en el aire para que la luz se viese reflejada en los ojos de Sylvian.

--Eso es, Sylvian --murmuró--. Mira hacia la luz. Contémplala profundamente y deja que tus preocupaciones se desvanezcan. --Viró la cabeza, hacia donde estaba Fergin, con ojos brillantes--. Tú no, Fergin.

El Doctor, volviéndose de nuevo hacia Sylvian, vio cómo al hombre se le escurría la tensión de su huesudo rostro. 

--Allá vamos --murmuró--. A ver, Sylvian, ¿qué me puedes decir de estos sueños que has estado teniendo?

De pronto, los ojos de Sylvian se abrieron de par en par, y sólo durante un instante fue como si sus ojos cambiaran y se volvieran amarillos. Con un grito saltó hacia adelante, empuñando su picahielos hacia la cabeza del Doctor.

Momentáneamente y contradiciendo su debilidad, el Doctor retrocedió. Su Stetson salió despedido y aterrizó entre sus pertrechas piernas. La punta curvada del picahielos aterrizó encima de él y lo clavó al suelo de madera. Sylvian, gruñendo y babeando, intentó liberar el instrumento, pero estaba tan metido para dentro que no fue capaz de sacarlo. Se levantó de golpe, curvando los dedos a modo de garras como si intentase desgarrar al Doctor con sus propias manos.

El Doctor, levantando el sónico, ajustó los controles y un chirrido agudo inundó la sala. Mientras los habitantes del pueblo se llevaban las manos a los oídos, el rostro de Sylvian se contorsionó del dolor. Con un lamento saltó por encima del Doctor, se precipitó hacia la puerta semiabierta y se desapareció entre la nieve.

El Doctor apagó el sónico. Durante un rato nadie se movió. Entonces todo el pueblo se apresuró para ayudarlo a levantarse. Uno de ellos le recogió el bastón y se lo entregó. Los demás le sacudieron el polvo de la chaqueta.

El Doctor miró con remordimientos su Stetson, el cual seguía clavado en el suelo junto al picahielos. Entonces su rostro se arrugó y sacudió las manos en el aire.

--¡Estoy enfadado con vosotros! --gritó--. Enfadadisísimo. --Señaló acusadoramente a Fergin--. Unos cuantos días, dijiste. ¿Por qué nadie me habló de estos sueños?

El pueblo retrocedió ante él. Algunos agacharon la cabeza, con el rostro lleno de culpa.

Fergin murmuró:

--No queríamos molestarte, Doctor. Eres un hombre importante.

El Doctor puso los ojos en blanco.

--¿Importante? Yo no soy importante. Soy el hombre menos importante de esta ciudad. --Señaló con el bastón al montón de gente--. Vosotros sois los importantes. ¿No os he dicho al menos unas tropecientas millones de veces que si ocurre algo mínimamente raro, me lo hagáis saber? ¿Sin rodeos?

La gente asintió sin decir ni mu. 

--Bien. --El Doctor le lanzó otra mirada de remordimiento a su Stetson. En un tono triste, dijo--: Antes tenía el pulso tomado. Ahora casi ni tengo pulso. --Dejó caer su sónico en el interior del bolsillo y luego colocó una mano nudosa en el hombro de Fergin--. Me estoy haciendo viejo, Fergin.

Fergin trató de reírse, pero en sus ojos había una mirada de preocupación.

--Chorradas, Doctor. Aún te quedan años. Vas a vivir más que todos nosotros.

El Doctor sonrió con ironía.



\mbox{}



\centerline{ \Huge *}



\mbox{}



A Aliganza le picaba el brazo. Se despertó para rascárselo, a pesar del hecho de que recorrer las yemas de los dedos por la piel no le propiciaba ningún alivio. De hecho, cada vez que se había rascado, el hormigueo gélido de sus venas se había convertido en astillas de cristal.

Sin embargo, extrañamente, cuando las astillas alcanzaron su mente, no le dolieron nada; era como si fueran pensamientos. No eran sus pensamiento, pero al mismo tiempo parecía que sí. Era como si estuvieran reemplazando su viejo yo – aburrido, lento, desinteresado – por un nuevo él – implacable, confiado, decidido.

Se levantó de la cama, sintiéndose ligero, casi ingrávido, como un joven de nuevo. Se estiró y aspiró aire, y fue como si hubiera renacido.

Había algo que necesitaba hacer. Algo urgente. No sabía el qué, pero aun así le seguía atosigando. Se sentía como una marioneta, como si alguien estuviera moviendo sus hilos. Pero al mismo tiempo, no era algo malo; estaba bien. Se sentía tan aliviado al liberarse de las preocupaciones de su día a día, de su constante necesidad de poner comida sobre la mesa, sus responsabilidades hacia… hacia…

Pericles.

Durante un instante, la palabra le sonó alienígena; no tuvo ni idea de lo que significaba. Entonces la información inundó su mente. Pericles. Su hijo. Por supuesto.

Una frialdad lo recorrió. Una especie de desprecio. Salió de la habitación como si lo estuvieran forzando, y caminó por el suelo crujiente de madera de su casita hacia una puerta que se hallaba al otro lado del pasillo.

Su mano rodeó el pomo. Abrió la puerta.

El chico estaba durmiendo pacíficamente, y su cabello rubio estaba desparramado por encima de su almohada.

Aliganza lo observó durante un rato con frialdad. Entonces cerró la puerta y bajó las escaleras.

Aquí era, en el estrecho armario de al lado de la cocina, donde guardaba sus herramientas. Abrió el armario, asomó la cabeza y empuñó el mango de su pala.

De repente, Aliganza supo a dónde necesitaba ir, lo que necesitaba hacer.

Su brazo no le paraba de picar y enviar nuevos pensamientos al cerebro.

Abrió la puerta de su casa y salió al frío.



\mbox{}



\centerline{ \Huge *}



\mbox{}



El Doctor estaba en lo alto de la Torre de Vigilancia sujetando un telescopio de metal viejo sobre su ojo. A través de él vio un continuo torrente de gente saliendo de sus casas y llevando una pala o un pico o algún que otro instrumento para cavar. Había hombres y mujeres, pero no niños. Se extrañó. Si a esta gente la estaba controlando alguna influencia maligna es que los niños, o eran inmunes a ella (¿demasiado libres? ¿Demasiado imaginativos? ¿Demasiado rebeldes?), o el filtro los había descartado por su falta de fuerza física. A menos que, ya que era tan sólo una fracción de la población total la que se estaba desplazando, fuera una simple coincidencia. Pero el Doctor no creía en las coincidencias. No del todo.

La gente poseída se estaba desplazando en la misma dirección – hacia la densa arboleda al norte de la ciudad. El Doctor se preguntó qué iban a desenterrar. ¿Un meteoro? ¿Una nave espacial? ¿Un arma? 

Sólo había una forma de averiguarlo.



\mbox{}



\centerline{ \Huge *}



\mbox{}



--¡Alto!

El hombre que había en la entrada al bosque era Tomalin Pilke, el hijo del profesor. Trabajaba en uno de los grandes invernaderos a las afueras de Navidad, los cuales producían comida y verduras para la comunidad. El Doctor había conocido a Tomalin desde que este había sido un niño – conocía a toda la gente de Navidad desde niños, incluso a la residente más anciana de la ciudad, Vida Clatterly.

--¡Hola! --gritó el Doctor, meneando en círculos y con entusiasmo una pala en el aire--. He venido a ayudar con la excavación.

Tomalin era normalmente un joven alegre, pero hoy no parecía estarlo. Su boca era una línea recta. Y la forma en la que blandía su rastrillo, con los pinchos apuntando hacia el Doctor como una garra en alto, parecía de todo menos amistosa.

--¿Qué sabes de la excavación? --preguntó.

--Oh, bastante --mintió el Doctor--. Sé que está ocurriendo. En el bosque. Y que es importante. Importantísimo. Vital, de hecho.

Tomalin se pensó sus palabras. Finalmente dijo:

--Tú eres… el Doctor.

--Un diez en observación. Siempre dije que eras un chico brillante. «Sra. Pilke --le dije a tu madre--, ese chico tuyo llegará lejos. Reconoce a la gente sólo con mirarlos». Agudo como una… cosa afilada. Entonces me cuelo, ¿vale?

El Doctor avanzó cojeando, pero Tomalin blandió su rastrillo con más agresividad.

--No eres bienvenido aquí --dijo.

El Doctor entrecerró los ojos.

--¿Quién lo dice?

Durante un instante Tomalin lo miró arrepentido, como si su personalidad real estuviera emergiendo, pero luego su rostro se volvió a quedar una vez más como el granito.

--Márchate, antes de que te arrepientas.

Lentamente, el Doctor levantó el bastón y señaló al joven.

--¿Por qué no muestras tu verdadera identidad? --dijo--. Demasiado asustado, ¿es eso? Vaya gallina. ¿Prefieres esconderte detrás de la de otro?

--No te temo, Doctor --siseó Tomalin.

--¿No? Pues deberías.

--Eres viejo. Ya no eres una amenaza para nosotros.

El Doctor puso los ojos en blanco.

--Bla, bla, bla. Ya he oído todo eso antes. He leído el libro, he visto la película y tengo la camiseta. No hay duda de que eres demasiado arrogante para escuchar a lo que te dicen, pero aquí va de todas formas: márchate de este planeta. Corre o repta o huye tan rápido como tus apéndices puedan impulsarte. Puede que sea viejo, ¿pero sabes qué? Eso me hace el doble de peligroso. Porque ya no tengo nada que me importe. Me queda poco tiempo, y si me voy a ir, quiero hacerlo luchando.

Bajó su báculo.

Tomalin ni se inmutó.

--Piensa en ello, grandullón --dijo el Doctor sin levantar la voz--. Piénsalo bien.

Entonces, balanceando la pala en el aire una vez más, se dio la vuelta y se alejó renqueando.

