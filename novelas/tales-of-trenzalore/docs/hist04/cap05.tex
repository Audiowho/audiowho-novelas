--Sólo son dos --susurró Fergin al oído del Doctor--. Podemos con ellos.

Se refería a los guardias, que estaban vigilando, uno armado con una azada, el otro con un hacha, fuera de un gran granero. El granero estaba adosado a una granja de la esquina oeste de Navidad, apartado de la comunidad por los campos cubiertos de nieve que la rodeaban. La granja pertenecía a la familia Svorsen, y de hecho, Clem Svorsen, el mayor de los tres hijos del granjero, era uno de los guardias.

El Doctor, asomándose desde detrás de un seto denso, y con el resto de la cuadrilla de Fergin apiñados en las sombras por detrás, sacudió la cabeza.

--Esos chicos son amigos tuyos; amigos vuestros. Ni se os ocurra hacerles daño a ellos; o a cualquiera.

Una tímida voz se alzó entre la multitud:

--Pero si están ayudando a la Mara.

La voz fue pisoteada, aunque suavemente, por otra, y rápidamente se levantó un murmullo. El Doctor respondió con un feroz gesto de silencio.

--No están ayudando a la Mara, están siendo controlados por la Mara. Esos chicos, y quien esté en ese granero, son víctimas, y cualquiera de vosotros podría haber ocupado su lugar. No lo olvidéis.

Se dio la vuelta y observó. Varias personas miraron hacia el suelo avergonzados.

--¿Entonces cómo los pasamos, Doctor? --preguntó Fergin--. ¿Los distraemos?

El Doctor se tocó un lateral de la nariz.

--Déjame eso a mí.



En cuanto le vieron los guardias, saliendo tranquilamente de la oscuridad invernal, se pusieron rígidos y adoptaron una posición de combate, con las rodillas flexionadas, las piernas separadas, las armas apuntando hacia él.

El Doctor levantó una mano.

--Parad el carro, chicos, vengo en son de paz. He venido a hablar con el jefe.

De cerca pudo ver cómo la influencia de la Mara estaba afectando a los dos hombres. Su piel era roja y moteada, como el blanco de sus ojos, e incluso sus dientes. No veía las marcas delatoras de la Mara en sus brazos, por culpa de las inmensas chaquetas que llevaban, pero sospechaba que sí las tenían: rayas curvas de piel inflamada y enrojecida, las cuales, con el tiempo, se marcaban más, asimilándose a una serpiente, hasta que sobresalían de la superficie y se despegaban, brotes físicos de la misma Mara, infectando al que la tocara.

Los ojos de Clem Svorsen cambiaron de repente. Se volvieron amarillos, como los de una serpiente, y sus pupilas se estrecharon. Abrió la boca y habló con una voz que no sonaba como la suya; débil y sinuosa, carente de amabilidad.

--Hablaremos, Doctor, a su debido tiempo; pero aún no.

--Oh no --dijo el Doctor, sacudiendo la cabeza--, ni se te ocurra venir aquí, y comenzar a soltarme el rollo. Este es mi hogar, mis reglas.

La Mara soltó una fría carcajada.

--Para ser el hombre más temido de este universo eres una criatura inútil e insignificante. No me puedes hacer daño, Doctor. Si lo intentas, acabarás dañando a los que amas.

--¿Amar? --se mofó el Doctor--. ¿Qué sabes tú de amar?

--Lo leo en la mente de esta criatura. Tiene cierto amor por ti. Todos lo tienen. Y tú por ellos.

--Sí, ya, eso es lo que pasa cuando se juega al Twister a temperaturas bajo cero. Nos apegamos. A veces literalmente.

--Tus balbuceos no tienen sentido, Doctor.

--Sólo si piensas que es un balbuceo. Yo, lo llamo una distracción verbal ingeniosa.

Con eso, el Doctor retiró la mano que había metido en el bolsillo de su chaqueta, la abrió y sopló lo de dentro. Una nube de polvo blanco, como una ventisca de nieve, envolvió las cabezas de los dos guardias. Tosieron y farfullaron durante un momento, y luego, recuperándose, miraron al Doctor y levantaron sus armas.

El Doctor retrocedió, con la esperanza de que no hubiera calculado mal la fuerza de la dosis…

Y tenía razón. Al mismo tiempo, las caras de los hombres se adormecieron, sus ojos se cerraron y se desplomaron sobre el suelo.

Volviéndose hacia el seto que bordeaba el camino que daba al granero, el Doctor extendió los brazos.

--¡Tachán!

En ese momento, unas figuras emergieron y se acercaron poco a poco hacia él, oscuras en contraste con la nieve. Fergin miró a los guardias inconscientes.

--¿Están bien?

--Durmiendo como unos bebés. No está mal lo que se puede hacer con el elemento sorpresa.

Apartaron a los guardias del camino, y luego el Doctor se retiró cuando la gente del pueblo, liderada por Fergin, se arrojaba contra las grandes puertas dobles del granero. Las puertas se doblaron y se astillaron, pero no cedieron. Obviamente las habían asegurado desde dentro, probablemente con un taco de madera entre las puertas.

--¡Deprisa! --insistió el Doctor, aunque sabía que la gente estaba haciendo lo que podía.

De repente hubo un fuerte crujido y las puertas se combaron hacia dentro. 

--Ya casi está --gritó Fergin, con su cara roja y sudorosa a pesar del frío--. ¡Una vez más! Todos juntos…

La gente del pueblo se precipitó contra las puertas. Hubo otro chasquido, y las puertas se abrieron de golpe. Cogidos por sorpresa, algunos se cayeron, y rápidamente se incorporaron. El Doctor, apartado a un lado, pudo ver un brillo parpadeante de color azul claro brillar desde el interior del granero, y oír un leve sonido. Alguna gente retrocedió, con intimidación y temor en sus caras. Ahora que habían conseguido entrar, parecían indecisos de qué hacer a continuación, dirigiendo las cabezas hacia el Doctor. Tan rápido como sus chirriantes articulaciones y su pierna de madera pudieron llevarlo, el Doctor se arrastró hacia adelante.

Como la mayoría de los graneros, éste era un gran espacio abierto lleno de paja y herramientas. Al otro lado, en un ataúd improvisado, estaba el esqueleto de Jalen Fellwood. A su alrededor en un corro, tumbados como las muescas del final de la cara de un reloj, estaba la gente del pueblo que había sucumbido a la influencia de la Mara. Parecían estar teniendo un sueño colectivo, en el que sus cuerpos temblaban y se sacudían, y sus ojos se movían por debajo de sus párpados cerrados.

La luz azul claro venía de los crepitantes hilos de energía que parecían fluir directamente de las mentes de los soñadores al esqueleto. Desde lejos el efecto era el de una enorme araña brillante, en el que el esqueleto era el cuerpo de la criatura y los hilos de energía, sus arqueadas y peludas patas.

Fergin miró al Doctor, perplejo.

--¿Tiramos ya las bombas somníferas, Doctor?

Más enfadado consigo mismo que con cualquier otro, el Doctor soltó:

--¿Qué sentido tendría eso? ¡Llegamos demasiado tarde!

El esqueleto irradiaba luz parpadeante. El Doctor se sacó el destornillador sónico y lo encendió, con la esperanza de que perturbara las ondas de energía, pero fue inútil. La energía del sueño ya estaba demasiado sustentada, era demasiado poderosa. Cubriéndose los ojos, el Doctor se acercó cojeando a los soñadores y al esqueleto, mientras todavía intentaba ajustar la configuración de su sónico.

Pero pudo ver a través del resplandor que varios hilos de membrana se estaban empezando a arrastrar por encima de los huesos, que se estaban formando músculos y venas y órganos de la nada. Acercándose más, sintió un desagradable cosquilleo en la piel. Respirando hondo, entró de una zancada en el campo de energía e inmediatamente comenzó a gritar. Hubo un crujido, una sensación súbita de dolor, y fue lanzado hacia atrás.

La siguiente cosa que supo fue que la gente se estaba arremolinando a su alrededor, ayudándolo a ponerse de pie, preguntándole si estaba bien. El Doctor asintió impacientemente, volviendo a dirigir su atención hacia el esqueleto. Sus ojos se abrieron como platos. La gente del pueblo soltó un grito ahogado; algunos comenzaron a alejarse de espaldas.

En el ataúd, el esqueleto de Jalen Fellwood comenzó a levantarse lentamente. Volvió su cabeza para mirarlos.

Todavía estaba incompleto, sus ojos brillaban en el interior de sus cuencas, su musculatura se hacía visible bajo una fantasmagórica pátina de piel a medio formar. Sin embargo, estaba claro que el cuerpo que estaba tomando forma alrededor de los huesos no era humano; al menos no del todo. Los ojos eran los de la Mara, amarillos y reptilianos. Y la piel que se estaba formando encima de la carne era escamosa y rugosa, enrojeciéndose cada vez más mientras la observaban.

--¿Qué hacemos, Doctor? --preguntó Fergin, acercándose a él furtivamente.

El Doctor consideró las opciones; no había muchas.

--En cuanto la Mara se haga manifiesta, tirad las bombas de sueño. Pasa la bola.

Fergin asintió.

--¿Eso la detendrá?

El rostro del Doctor fue solemne.

--Más vale intentarlo.

No tuvieron que esperar mucho. El híbrido humano-Mara ya estaba casi completo. Se levantó del ataúd y se estiró, contemplando su nueva forma. Su cabeza era plana, su hocico alargado; una capucha de cobra sobresalía de cada lado de su pálida y rugosa garganta. Abrió la boca para sisearlos, revelando una lengua bífida negra y colmillos curvados.

En cuanto los hilos azules de energía entre los soñadores y la Mara se consumieron y se quebraron, el Doctor exclamó:

--¡Ahora!

Una lluvia de esferas plateadas se precipitó hacia la Mara, explotando en cuanto tocaron el suelo. Un polvo blanco se condensó en una nube, oscureciendo a la criatura. El Doctor entrecerró los ojos y mantuvo la respiración, esperando que contra todo pronóstico…

Pero entonces la escuchó reptar a toda velocidad, y al segundo siguiente vio un destello rojo al mismo tiempo que la Mara salía disparada como una bala y se dirigía hacia la pared de la izquierda.

Hubo gritos, chillidos, dedos señalando. Más similar a una araña que a una serpiente, la Mara empezó a ascender por la pared vertical. Se movía como un rayo. Antes de que nadie pudiera reaccionar para interceptarla, ya había recorrido toda la pared y salido por la puerta.

El Doctor dio una vuelta de 360 grados sobre su pierna de madera y se dispuso a seguirla. Pero cuando consiguió abrirse paso a través de la multitud y salir, escaneando el paisaje permanentemente nevado, la Mara se había esfumado.



--¿Alguna novedad?

Samanda Glyde apartó la vista del paciente que había estado examinando. El Doctor estaba de pie en la puerta, apoyado en su bastón, mirando sombríamente de arriba a abajo las dos filas de camas de hospital.

La enfermería de Navidad era un edificio de madera, no mucho más grande que un granero de tamaño normal, y todas las camas de su única sala estaban actualmente ocupadas. Aunque no era inusual (después de cada incursión alienígena casi siempre había pacientes que atender), lo que era raro es que estos pacientes en particular estaban muy tranquilos. La mejor amiga de Samanda, Taskia, había ido ayer al granero de los Svorsen y le había contado lo que había pasado. Desde entonces nadie había visto al alienígena, a la Mara. La tensión en la ciudad era palpable, la comunidad se sumió en una atmósfera de taciturna aprehensión. Mientras tanto, era trabajo de Samanda cuidar de las víctimas de la Mara.

Tampoco es que necesitaran mucha atención. Dormían plácidamente. Parte de ello se debía a que el Doctor les había fabricado a todos “inhibidores del sueño”, unos curiosos aparatitos de metal y cables, que había improvisado de cachos y piezas que guardaba en la Torre del Reloj. Aunque los “inhibidores del sueño” que había improvisado eran diferentes, todos tenían una cosa en común: activados por un toque del destornillador sónico del Doctor, daban un pulso continuo y parpadeante que el Doctor le había asegurado a Samanda que ahuyentaría a la Mara.

Sacudiendo la cabeza en respuesta a la pregunta del Doctor, dijo:

--No, siguen todos dormidos. Eso es bueno, ¿no?

--Sí. Y no. --El Doctor recorrió cojeando la nave central, escudriñando una por una a cada forma durmiente. Los rostros de las víctimas de la Mara estaban rojos y llenos de ronchas, y en ambos brazos sobresalía la marca de una serpiente.

--¿Por qué no? --preguntó Samanda.

--Porque son los ojos y oídos de la Mara, y hasta que no la destruyamos no puedo asumir el riesgo de levantarlos. Pero cuanto más duermen, más débiles se vuelven, hasta que…

--Mueran --murmuró Samanda.

El Doctor no dijo nada.

--¿Cómo la vas a encontrar? --preguntó Samanda.

El Doctor suspiró.

--Oh, fijo que me encuentra ella a mí. No quiero ir por ahí de chulo, pero está aquí por mí. --Sus líneas de expresión se transformaron en una débil sonrisa y sus ojos se ablandaron--. Mientras tanto, ¿qué tal un poco de chocolate caliente? Con seis azucarillos para mí.

Antes de que Samanda pudiera responder, la puerta de la sala se volvió a abrir y Taskia apareció. Tenía los ojos como platos y sus hombros estaban jadeando, como si hubiera estado corriendo.

--Está aquí --dijo de manera entrecortada--. Esa cosa… Está en la escuela.

Instantáneamente, el rostro del Doctor se oscureció. Levantó el bastón y lo dejó caer con tanta fuerza que Samanda pensó que las tablas del suelo se habían astillado.

--¡Ni se te ocurra! --rugió--. ¡Ni se te ocurra!



Para ser un anciano con una pierna de madera, el Doctor podía moverse increíblemente rápido cuando se le sacaba de quicio. Dejó que la gente lo siguiera mientras recorría a pisotones la calle principal, en cuyos hombros y escasa melena se iba posando la nieve. Se detuvo delante de la escuela; un edificio de madera pintado de blanco con una puerta y ventanas azules.

--¡Sal y enfréntate a mí, Mara! --exclamó, sacudiendo el bastón--. ¡Ven y métete con alguien de tu tamaño!

Durante un instante nada ocurrió. Entonces un grito se elevó de entre el gentío a diez metros por detrás.

--¡Allí arriba, Doctor! ¡En el tejado!

El Doctor alzó la vista. En efecto, era el híbrido Mara-humano, agachado en el tejado de la escuela, con su roja piel sobresaliendo por encima de los árboles cargados de nieve que tenía por detrás. Con la boca abierta y los colmillos al descubierto, parecía estar sonriendo maliciosamente.

--¿Qué pasa, Mara? --gritó el Doctor--. ¿Tienes miedo de enfrentarte a mí, de hombre a reptil?

La puerta de la escuela se abrió de golpe y lo que parecían ser los niños de toda la población de la ciudad salieron a borbotones. Hubo chillidos y gritos de shock provenientes de la gente aglomerada detrás del Doctor, muchos de los cuales eran padres de los que estaban entre la multitud emergente.

El rostro del Doctor permaneció frío y lleno de ira mientras escaneaba las expresiones de los niños. Todos y cada uno de ellos eran ahora sirvientes de la Mara. Sus rostros estaban enrojecidos, sus facciones en blanco, sus ojos mirando al vacío. Aquellos cuyos brazos estaban expuestos portaban la marca de la Mara en su piel.

--¿Ahora usas a niños para librar tus batallas? --gruñó, mirando hacia la criatura serpiente--. Incluso para algo que pasa la mayor parte del tiempo en su barriga, es rastrero.

Los niños se pararon formando un semicírculo irregular. Uno de ellos dio un paso hacia adelante. El Doctor lo reconoció como el hijo de Aliganza Torp. ¿Cómo se llamaba? ¿Barnable? No, Pericles, es verdad.

El chico parpadeó y de repente sus ojos se volvieron amarillos como los de una serpiente.

--Di tu nombre, Doctor --dijo con una voz gélida.

--¿Por qué? --soltó el Doctor.

--Porque si no, le haré daño a las personas que amas.

El Doctor se encogió de hombros.

--No es un argumento muy persuasivo. Verás, si te digo mi nombre, descenderá un infierno, literalmente, y la gente que amo morirá de todas formas, barrida como una viruta en la tormenta más grande que este universo haya visto. --Señaló con su bastón no a Pericles, sino a la criatura del tejado--. Pero eso es lo que quieres, ¿no, Mara? Te mueres por ello. Quieres rellenar tus mejillas de reptil con toda esa barbaridad y degradación y crueldad. Porque eso es lo que eres, te alimentas de lo que cae al fondo del mar. No supones un beneficio para nadie más que para ti misma, así que ¿por qué no vuelves por dónde te has venido, antes de que me enfade de verdad?

Escupió esas últimas palabras sacudiendo su cuerpo con tal rabia que la gente del pueblo de detrás suya retrocedió como un metro.

La Mara, sin embargo, ni parpadeó. Lo miró fríamente a través de los ojos del chico que había seleccionado como portavoz.

Finalmente dijo:

--Al menos, en la guerra las muertes de aquellos que amas serán rápidas. ¿Pero qué difícil será ver a los niños destruir a sus padres?

Simultáneamente los niños, antes como estatuas, comenzaron a moverse. Sus ojos se volvieron amarillos, y actuando como una única entidad, abrieron sus bocas y sisearon.

A medida que avanzaban, con garras en vez de dedos, el Doctor se dio la vuelta y chilló:

--¡Retirada! ¡A la granja de nieve, como os dije!

La gente del pueblo no necesitó ni que se lo dijeran. Aunque muchos de ellos dudaron en correr hacia sus hijos, recogerlos, salvarlos de algún modo, confiaban en el Doctor implícitamente y habían atendido cuidadosamente a lo que les había dicho el día anterior.

La granja de nieve de las afueras de la ciudad, con sus grandes puertas de hierro y sus paredes de piedra maciza, iba a ser su refugio, su santuario.

Como si fueran uno, y con el Doctor cubriéndoles las espaldas, se dieron la vuelta y echaron a correr.

