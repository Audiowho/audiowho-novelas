El cielo ardió. Crantle estaba acostumbrado a las luces que salpicaban las largas noches,las estrellas y las innumerables naves espaciales que orbitaban Trenzalore y que lo habían hecho desde antes del nacimiento de su abuelo. Pero esto era algo diferente. Un rastro de fuego ardiente a través del cielo y que se estrellaba al otro lado de la cresta alta que rodeaba la ciudad de Navidad.



Crantle vivía fuera de la comunidad principal. Hacía los trineos cargados de nieve que iban a las comunidades periféricas. El parque de nieve de Navidad era la principal fuente de agua para muchos de los nuevos asentamientos de Trenzalore. Crantle cosechaba la nieve, embalándola en las bodegas de aislamiento de sus trineos. Viajaban en un pequeño convoy, Crantle iba en el trineo de delante, gritando a los perros, a pesar de que conocían la ruta tan bien como él. Los trineos le seguían, atados entre ellos, cogiendo velocidad sobre el suelo helado.



Cuando la nieve dio paso al hielo, y el hielo poco a poco dio paso a un paisaje más verde, Crantle deslizó las ruedas debajo de los trineos para continuar en horas de la noche casi perpetua. Un viaje de ida y vuelta le llevaba más de una semana. Más de una semana sin ninguna compañía, excepto los perros y su propio canto desafinado. Algunos hombres se hubieran vuelto locos, pero Crantle amaba cada minuto de ello .



Pasaba el tiempo entre los viajes ocupándose de los perros, y preparando los pequeños invernaderos donde cultivaba sus propias verduras, cualquier cosa que disminuyera su dependencia de los demás. Recibía carne y fruta a cambio de la nieve con la que negociaba.



No tenía pensado hacer otro transporte de nieve durante varios días, por lo que el cielo ardiente era una curiosa distracción. Crantle se sentó en su silla de madera preferida del porche, y observó el rastro de fuego abrasador de las estrellas y las naves distantes. Desapareció detrás de la cresta en una repentina lluvias de chispas. En las profundidades de la noche, dos líneas más de fuego grabaron un camino hacia el planeta. Pero Crantle no se dio cuenta de aquello. Su atención estaba en el moribundo resplandor donde había caído la primera bola de fuego.



``Meteorito'' era un nombre en algún lugar en el fondo de su memoria. Una roca cayendo desde el cielo. Una de las comunidades con las que Crantle hacía negocio era un pequeño pueblo minero. Cavaban en el suelo, en la roca, en busca de valiosos minerales. Quizás Crantle no necesitara molestarse en excavar. Tal vez la roca que había visto cayendo se hubiera roto en una ráfaga de escombros a través del paisaje. Era poco probable, pensó Crantle, pero no tenía nada mejor que hacer que echar una mirada. Vivía cerca de la cima de la cordillera, por lo que sería probablemente menos de una hora de caminata. Incluso si no merecía la pena, podría ser interesante.



\mbox{}



La luz de las lunas gemelas de Trenzalore reveló una cicatriz ennegrecida por donde había pasado la estela de fuego en el cielo. La nieve ya cubría de nuevo el suelo quemado, los copos sibilando y derritiéndose donde caían sobre los parches calientes. Crantle caminaba junto a la tierra desnuda, usándola como un camino que lo guiara al meteoro,tanteando la nieve de delante de él con su largo bastón de madera. Solo podía distinguir la forma oscura e irregular del meteoro empujado contra el borde de una zona boscosa.



Los pinos lunares se balanceaban suavemente en la brisa fresca, esperando pacientemente a los pocos minutos de luz del día que los mantendrían a través de la siguiente noche. Crantle escudriñó las sombras a la luz de las lunas, casi esperando ver la forma desgarbada del Doctor esperándole entre los árboles. Si alguien de Navidad llegó a ver lo que había caído más allá de la cordillera, sería el Doctor. Si había alguien en Navidad con quien Crantle realmente disfrutaba hablando, era el Doctor. Había algo en el hombre que inspiraba confianza. De alguna manera Crantle sentía que podía estar a solas con el Doctor, no habría entrometimientos, ni preguntas corteses, ni conversaciones sin sentido simplemente porque sí.



Pero al llegar al final del camino chamuscado a través de la nieve, Crantle no vio a nadie. Frente a él, el meteoro estaba echando humo como si todavía estuviese ardiendo. Su lado dentado brillaba bajo la luz de la luna, emitiendo vapor, derritiéndose. El agua se agrupaba alrededor de la base del mismo, corriendo por la herida que había causado en la tierra. Era casi el doble de la altura de Crantle, e igual de profundo, una esfera áspera redondeada por el calor de su llegada. Y, para sorpresa de Crantle, estaba hecha de hielo. Mientras se acercaba, no era calor lo que sentía en su rostro, pero sí un escalofrío.



Crantle tocó el hielo con su bastón de madera. Acercándose más, alargó una mano vacilante, acariciando un lado del hielo. Podía sentir el frío a través de su grueso y acolchado guante. Pero también algo más, un temblor débil, una vibración. Como si el hielo estuviera temblando de su propio frío. Frotó la superficie con el guante, aclarando la escarcha y dejando una superficie lisa y vidriosa. Reflejaba las lunas y las estrellas, su luz plasmada y distorsionada sobre la superficie ondulada.



Pero bajo las luces brillantes, en lo profundo del propio hielo, había otra forma, oscura y borrosa. ¿Una figura? Temblando, como si estuviera luchando por moverse dentro de su tumba de hielo. Un truco de la luz de las lunas, pensó Crantle. Nadie podría sobrevivir dentro de un bloque de hielo. Y este hielo se había caído del cielo, nadie podía estar en su interior.



El sonido era como el chasquido de un árbol mecido por el viento. Una grieta repentina, y toda la sección de hielo delante de Crantle se partió, de arriba a abajo. Una sección se separó, estrellándose contra el suelo y rompiéndose como el cristal. Instintivamente, dio un paso atrás. Momentos después, un puño atravesó el hielo cerca de donde había estado su cabeza. Afiladas astillas transparentes salieron disparadas hacia la cara de Crantle, irritándole las mejillas y alojandose en su barba.



Si gritó de la sorpresa o el miedo, el sonido se perdió en la explosión de hielo a la vez que la criatura en el interior destrozó este en su salida y se puso delante de él. Una figura enorme, que se elevaba por encima de Crantle, encerrado en una armadura de color verde oscuro como las escamas de un reptil. La cara estaba escondida detrás de un casco que le cubría la cabeza, con los ojos protegidos por cristales oscuros que reflejaban el rostro asustado de Crantle. Tenía finos labios exangües fruncidos en una mueca que podría ser de desprecio. O de diversión.



La única defensa de Crantle era el bastón de madera que sostenía. Lo blandió frente a él, esperando a ver la reacción de la criatura, tratando de decidir si era más seguro quedarse ahí o correr. No esperó que hablara.



--Primitivo --dijo la figura con voz áspera, su voz era un susurro ronco--. Te rendirassss a nosotrosss--dando un torpe paso hacia adelante, extendió su mano hacia Crantle.



Crantle reaccionó sin pensar, golpeando con la pesada vara de madera el pecho de la criatura. La criatura se tambaleó ligeramente bajo el impacto. Crantle retrocedió el bastón para atacar de nuevo, pero la criatura se movió más rápido de lo que había previsto. Una mano como una tenaza agarró el bastón entre los rechonchos dedos de su guante, arrancándolo de las manos de Crantle. La criatura lo lanzó lejos, la enorme mano se disponía ahora a agarrar a Crantle.



Dio un paso atrás, volviéndose para correr, pero era demasiado tarde. La mano cubierta de armadura de la criatura se cerró sobre la nuca de Crantle. Sintió que lo levantaban en el aire. El mundo era un borrón en movimiento, árboles, hielo, el rostro impasible de la criatura, suelo nevado dirigiéndose hacia él. Entonces llegó la oscuridad.



\mbox{}



El Guerrero de Hielo examinó por un momento el cuerpo inerte del humanoide, buscando señales de vida. No había ninguna. Dio un silbido de satisfacción. Arriba, el cielo era recorrido por el camino ardiente de otro meteorito de hielo. Un tercero lo seguía de cerca. Impactaron uno detrás del otro a lo largo de la cresta, justo dentro de la línea de árboles.



El guerrero pasó sobre el cuerpo de Crantle y se abrió camino hacia el más cercano de los meteoros.
