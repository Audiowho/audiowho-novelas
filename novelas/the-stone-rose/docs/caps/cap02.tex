\chapter*{Capítulo Dos}
\addcontentsline{toc}{chapter}{Capítulo Dos}

El Doctor, Rose y Gracilis se abrieron camino por la Vía Lata, pasando
por debajo de la Columna de Trajano misma, que se erguía en el cielo con
sus historias en relieve de la victoria de Trajano sobre los dacios.
Parecía más impresionante desde abajo, con los paneles de mármol
saliendo en espiral de un tipo de templo en la base (--Tiene las cenizas
de Trajano --dijo el Doctor). A esa altura, Rose tuvo que doblegar el
cuello para ver la estatua del emperador de pie en lo alto, a unos
treinta metros por encima de ella. Había una plataforma--mirador en lo
alto de la columna y pudo ver cómo el Doctor se acercaba para subir
hasta arriba, pero Gracilis era un hombre con una misión y se vieron
obligados a apresurarse.

Al final llegaron al lugar nombrado por el dueño de la cafetería. Un
apartamento en un bloque, no era el más salubre de los lugares, pero
tenía mejor pinta que el lugar en el que habían aterrizado.
Verdaderamente, no se diferenciaba demasiado de Powell Estate, varios
bloques de apartamentos construidos alrededor de un patio, y había
incluso algunas tiendas en las plantas bajas, pero vendiendo aceite de
oliva y menaje de cocina en vez de cigarrillos y comida china.

Subieron por unas escaleras hasta el apartamento en cuestión, donde el
Doctor tomó la iniciativa y llamó a la puerta.

Tras un momento se abrió ligeramente y un hombre con la mirada fija en
una túnica mugrienta les observó.

--¿Y bien? ¿Qué queréis?

El Doctor le sonrió.

--Nos gustaría ver a la joven que vive aquí. Ya sabe, ¿la profeta? ¿La
astróloga?

Los modales del hombre cambiaron al instante. De repente se comportó de
manera servil y entusiasmadamente, mientras empujaba la puerta y se
apartaba para dejarles entrar.

--Ah, es un placer, dama y caballeros, un verdadero placer. Permítanme
presentarme. Me llamo Balbus, y conocerán a la joven, la lectora de las
estrellas, la interpretadora de los planetas, aquella que sabe lo que va
a venir. Sólo por el menos de los precios, la conocerán.

--Siempre pidiendo dinero--murmuró Rose--. Nada ha cambiado--esperó que
Gracilis regateara la suma que le pidió, pero era obvio que estaba
demasiado ansioso con su hijo para regatear el dinero, y le dio al
hombre un puñado de monedas sin protestar.

El hombre desaliñado les guió hacia una sala trasera donde alguien
estaba acurrucado en un rincón.

--Tienes visitantes, Vanessa--dijo, acariciándose las manos en la manera
de alguien que acaba de hacer un buen trato--. Diles lo que quieren
saber.

La silueta levantó la mirada y Rose se sorprendió. Había esperado
inconscientemente una gitana de la feria, una mujer mayor con las
mejillas sonrosadas, con una sonrisa intrigante mientras le hablaban de
unos altos viajeros oscuros y viajes a través del mar. Pero era sólo una
niña: una niña delgada de piel morena con los ojos asustados.

--Sí, maestro.

Rose se giró al Doctor, con una mirada perpleja.

--Es una esclava--le dijo sin hablar.

Gracilis se sentó delante de la niña.

--¡Debes decirme dónde encontrar a mi hijo! --le imploró--. Puedo darte
su fecha y su lugar de nacimiento, todo lo que necesites saber.

La chica parecía asustada.

--Responde al caballero, Vanessa--dijo el dueño, cuya sonrisa parecía la
de un lobo.

En una voz dulce, comenzó a hacer a Gracilis preguntas sobre Optatus,
entonces sacó un pedazo de pergamino y comenzó a realizar unas extrañas
fórmulas. No significaban demasiado para Rose, nunca había sido
demasiado buena en matemáticas cuando fue el momento y mucho menos
entonces viendo los símbolos subir y bajar, pero se dio cuenta de que la
atención del Doctor había sido captada. Observó los figuras de forma
congelada durante unos momentos, antes de zarandear la cabeza como para
vaciarla y girarse hacia Gracilis.

Gracilis parecía ansioso, expectante. Rose lo sintió por él: no sólo por
su hijo, sino porque estaba tan desesperado que se había visto llevado a
medidas ridículas como aquella. La chica parecía bastante simpática, no
del tipo que se aprovecha de la gente, pero Rose no pudo decir lo mismo
de su dueño. Aprovecharse de los débiles y de los heridos, aquel era
obviamente el juego allí, como si descubrir dónde estuvieran unas
estrellas en el momento del nacimiento de alguien fuera a decir dónde se
habían ido dieciséis años después.

La sonrisa de Balbus se volvía más y más forzada.

--Responde al caballero--dijo de nuevo, después de que hubieran pasado
unos minutos.

--Vamos. Déjanos tener nuestro dinero bien gastado--le dijo el Doctor--.
No se puede calcular el movimiento de los cielos en dos minutos.

La chica pareció agradecida y comenzó a escribir unas cuantas sumas más.
De repente Rose se dio cuenta de algo. ¡La chica estaba sacando tiempo!
Por supuesto que no podía darle a Gracilis una respuesta verdadera, por
lo que intentaba pensar en qué decirle.

Quizá el Doctor también se había dado cuenta. Se sentó justo delante de
la chica.

--Obviamente no infravaloro tus habilidades, pero supongo que es
bastante difícil descubrir algo como esto con tan poca información.
Necesitas descubrir más información acerca del chico, Optatus. Y supongo
que necesitarás conocer el lugar dónde desapareció.

Ella asintió desesperadamente, sus ojos parecían rogarles.

--Sí, sí. Necesito ver el lugar en el que desapareció.

--Bueno, estoy seguro de que a tu\ldots{}--el Doctor se detuvo,
saboreando la desagradable palabra-- dueño no le importará que te quedes
un rato con nosotros. No en pos de un objetivo como el que perseguimos.

Pero extrañamente su dueño no parecía feliz por la idea.

--Me temo que no lo puedo considerar\ldots{}--comenzó, pero no avanzó.

Gracilis golpeó su puño contra la mesa, haciendo que la tinta de la niña
se corriera por el pergamino en el que tenía sus cálculos.

--¡Entonces déjame comprártela! --dijo--. No lo entiende, caballero, ¡es
mi única esperanza!

--¿Cómo? ¿Dejar a mi pequeña mina de oro, quiero decir--dijo Balbus, con
la sonrisa sumisa de nuevo en juego--, abandonar mi deuda sagrada de
encargarme de ella?

--Oh, nosotros la podemos proteger, no hay ningún problema--dijo el
Doctor alegremente--. Creo que puede ser una verdadera buena idea. Aquí
Gracilis es un hombre rico. Estoy seguro de que no tendréis ningún
problema en llegar a un acuerdo.

Balbus se encogió de hombros.

--Se acerca el Quinquatrus. Todas esas mujeres, los turistas, les
encantará oír sus futuros. Si no tengo a Vanessa perderé mucho
dinero\ldots{}

Los dedos de los pies de Rose se cerraron con incomodidad al escucharles
discutir por un precio para la chica, un ser humano siendo comprado y
vendido como si fuera una mesa o una bolsa de manzanas o un abrigo de
plumas.

Aún así Vanessa no parecía tan aterrorizada; parecía feliz, deseosa,
incapaz de creer su propia suerte. Su vida allí no sería demasiado
divertida y obviamente predecía tiempos mejores sirviendo a Gracilis.

Finalmente, una vez completadas las negociaciones, Gracilis, el Doctor y
Rose abandonaron el apartamento con Vanessa escoltándoles.

--Así que, ¿qué pasa ahora? -- preguntó Rose.

--Justo lo que he dicho--replicó el Doctor--. Creo que nos ayudaría a
todos si volviéramos a la villa de Gracilis y examináramos el lugar
dónde Optatus fue visto por última vez. Si la invitación sigue en pie,
¿verdad?

Se giró hacia el anciano, que asintió fervientemente.

--Sí, sí, si creéis que es lo mejor--suspiró--. Podría buscar en Roma
durante un año y nunca le encontraría, aunque esté aquí.

--Sí, la verdad es que tampoco no tendrás fotografías de él para
pasarlas--dijo Rose sin pensar. El Doctor le lanzó una mirada--. Quiero
decir, bueno, esperaremos hasta llegar a la villa.

El carro de Gracilis esperaba fuera de las puertas de la ciudad y todos
se subieron. El Doctor indicó con un gesto que quería que Rose se
mantuviera cerca de Vanessa, pero ella lo habría hecho de todas maneras.
La chica había dicho a penas una palabra desde que habían dejado el
apartamento, pero Rose estaba centrada en establecer una conversación
con ella.

--Así que ¿cuándo viniste a Roma? --intentó, comenzando con una buena
pregunta. Pero aquello pareció alarmar a Vanesa, que se mantuvo
silenciosa. Probó otra:

--¿Cuántos años tienes?

Aquella vez la chica respondió.

--Dieciséis-- ella suspiró.

--¿Y cuánto llevas haciendo esto de la astrología?

De nuevo Vanessa no respondió, pero Rose se sorprendió de ver lágrimas
comenzando a caer por sus mejillas. Ella agarró impulsivamente a la
chica con un abrazo.

--Eh, no llores. Lo siento, no te preguntaré nada más, no si no me lo
quieres decir--pero ahora la chica había comenzado a llorar y parecía no
poder parar. Rose la sujetó mientras sus sollozos hacían temblar su
cuerpo, cogiéndola con cariño, reconfortándola. Se preguntaba qué le
habría pasado a aquella chica para que estuviera tan asustada.

El viaje fue lento y Rose pensó en los tan anhelados trenes y coches.
Aún así, supuso que un carro tirado por caballos (bueno, de hecho,
burros) era mucho más amable con el medio ambiente, aún así siendo un
viaje largo y lleno de baches. Se sorprendió al saber que no llegarían a
la villa aquel día y tendrían que pasar la noche en una posada en el
camino. Esperó que al menos eso pudiera darle una oportunidad de hablar
con Vanessa sin nadie cerca, pero los esclavos iban en una distinta
parte del edificio. Rose se preguntó cómo serían las estancias de los
esclavos, pensando en lo incómoda que era la cama que le habían dado a
ella, se pasó la noche a medias durmiendo, a medias despierta
preocupándose sobre la higiene romana y la infestación potencial,
intentando decirse a sí misma que cualquier picor era sólo objeto de su
imaginación\ldots{}

Se marcharon a la mañana siguiente cuando el sol estaba a penas en lo
alto. Probablemente les llevaría un día entero llegar a la villa, por lo
que Gracilis quiso comenzar pronto. Rose estaba feliz, aún así, aquello
significaba que no pasarían otra noche en una posada.

El anciano no mostró interés en desayunar, pero mientras el sol se
alzaba en el cielo, el Doctor saltó del vehículo y les cogió unos higos
de un árbol salvaje cerca de una carretera.

--He averiguado la fecha. Es el 120 dC--le susurró a Rose mientras le
pasaba la fruta--. Adriano es el emperador. No te preocupes. Estoy
poniéndome al día.

Gracilis estaba obviamente deseoso de volver y dejar que Vanessa
comenzara a rastrear a Optatus. Rose sentía que la chica temblaba con
ello, estaba muy convencida de que Vanessa no tenía dones, ni poderes
místicos, y se preguntaba qué haría el encantador anciano romano cuando
descubriera que se había gastado todo el dinero en una esclava para
nada. Por ahora, aún así, parecía de buen humor, quizá un poco
preocupado, sentado a un lado del carro hablando tranquilamente al
Doctor. Aún así, Rose comenzó a planear estrategias de rescate en su
cabeza. Por si acaso.

A pesar de las expectaciones de Gracilis, Vanessa parecía más alegre
tras la noche de descanso e incluso respondía vacilantemente a algunos
comentarios de Rose sobre el paisaje. Animada, Rose aprovechó:

--Soy de Bretaña\ldots{} ¿Britania? --dijo--. Ya sabes, dónde el tal
emperador Adriano construyó el muro, ¿no?

--Ah, el muro de Adriano. Construido para mantener a raya a los
bárbaros--dijo Vanessa.

--¿Quiénes son? ¿Los fans del \emph{Celtic}? --dijo Rose, riendo.

Vanessa parecía sorprendida, pero también rió. Por un momento, Rose
pensó que iba a decir algo más, revelar algo sobre ella misma, pero no
lo hizo.

Atardecía ya cuando llegaron a la villa, pero había la suficiente luz
para que Rose pudiera ver cómo era. Se había estado esperando algo
parecido a un chalé, pero era más como una granja, como una granja
increíblemente pija. Había un número de edificios cubiertos por una
decoración estucada más bien fea rodeando un patio con fuentes y
estanques de peces. Había mosaicos elaborados y estatuas elegantes.
Había campos de trigo, y huertos de melocotoneros y almendros llenos de
flores, establos para los burros y patios para los pollos y los gansos.

--No es un mal lugar, este--murmuró el Doctor, mientras entraban. Una
mujer regordeta y bajita llegó corriendo a su encuentro. Parecía que su
estado natural era amabilidad y jovialidad, pero en aquel momento su
cara era ojerosa y ansiosa.

--¿Le has encontrado? --gritó, ignorando a todo el mundo menos a
Gracilis.

Gracilis negó con la cabeza tristemente, entonces presentó a la mujer
como su mujer, Marcia, la madre de Optatus.

--¡Pero esta buena gente han venido a ayudar! --le dijo--. Esta
esclava\ldots{}--le indicó a Vanessa--\ldots{} es una profetisa de gran
poder. Una vez haya sabido más de Optatus, le encontrará por
nosotros--se detuvo un segundo--. ¿Se sabe algo de Ursus?

--Me ha asegurado de que estará lista para mañana, tal y cómo prometió--
respondió Marcia.

Marcia les ofreció comida, pero ya habían comido en el camino. La luz
estaba yéndose lentamente y unas cuantas lámparas de aceite no daban la
suficiente luz, parecía ser que era costumbre irse a dormir pronto y
levantarse con el amanecer.

--Mañana os mostraré a mi hijo--les prometió Gracilis mientras llamaba a
unos esclavos para enseñarles al Doctor y a Rose sus habitaciones--.
Estoy seguro de que seréis las respuestas a mis plegarias.

Pero Rose, mientras se echaba y se removía en la cama desconocida, pero
afortunadamente limpia, no estaba segura de todo aquello.

A la mañana siguiente, el Doctor ya estaba ya listo y preparado cuando
Rose, adormilada, se abría camino al piso de abajo. Finalmente ella le
encontró en un huerto de árboles frutales, sentando bajo un árbol
mientras unas flores de melocotonero le bañaban el pelo como una lluvia
de nieve.

--Listos para un trabajo de detectives--dijo ella.

--Hércules Poirot podría resolver cualquier caso sólo con sentarse y
pensar--le dijo a ella.

--¡Tú con un bigote retorcido! --se rió--. ¡Te iría muy bien con las
patillas!

--Supongo que me haría parecer incluso más sofisticado--dijo
altivamente.

Rose sonrió.

--Hazlo entonces. Déjate un bigote retorcido. Te reto.

--¡De acuerdo! --dijo, señalando a su labio superior--. Me estoy
dejándolo crecer uno ahora, ¡mira!

Ella miró más de cerca, haciendo que le creía, pero se tuvo que detener
al estallar a carcajadas al cabo del rato, y el Doctor se le unió.

--Quizá no--dijo él.

--Así que, ¿cuál es el plan? --le preguntó Rose después de que se
calmaran.

--Gracilis está preparando algo--le dijo el Doctor--. Tenemos que
encontrarnos en media hora.

No hablaron de nada en particular hasta que un esclavo llegó a irles a
buscar. Les llevó a un bosquecillo cerca de la entrada principal de la
villa. Unos pavos reales orgullosos se paseaban cerca de la hierba y a
través de los pulcros y ordenados lechos de flores, y el agua goteaba de
las bocas de unas ninfas y faunos de piedra en un pequeño estanque. La
única nota discordante la daba un ser humano; un hombre alto, regordete
y con el ceño fruncido, ajeno a aquel medio ambiente de riqueza y
belleza. Estaba agachado contra la base de una estatua, al menos Rose
supuso que era una estatua, con una sábana tapándola por encima.

Gracilis, Marcia y Vanessa se acercaron por el bosquecillo de una
dirección distinta y el hombre de apariencia extraña se incorporó
mientras les vio llegar.

--Ah, Ursus, estimado amigo--dijo Gracilis--. Confío en que todo esté
preparado para la revelación, ¿verdad?

El hombre asintió bruscamente.

--¡Excelente! --Gracilis se giró hacia el Doctor y Rose. Rose se dio
cuenta de que Vanessa apenas existía a sus ojos, a no ser que él
estuviera hablando de ella o con ella.

--Este es Aulus Valerius Ursus. ¡Es un artista local pero se está
volviendo rápidamente en la comidilla de todo el Imperio! Creo que puedo
decir con un poco de miedo a contradecirme de que es uno de los grandes
escultores de nuestros días. Raramente acepta pedidos de ciudadanos
privados, así que me sentí gratamente honrado cuando accedió a crear un
trabajo para mí para felicitar a mi querido hijo.

--Con el dinero que me ofreciste a duras penas pude rechazarlo--dijo el
hombre, con una sonrisa rapaz que le recordó a Rose al antiguo dueño de
Vanessa, Balbus.

Gracilis le dedicó una risita triste.

--Es cierto que una de las ventajas de ser un hombre muy rico es que las
cosas que no están a la venta las puedo comprar. Y aún así no puedo
comprar la única cosa que deseo por encima de todas las otras: la vuelta
de mi hijo.

Dio un paso adelante y cogió la sábana que tapaba la estatua.

--Aún así, es mi deseo que esto nos acerque a un desenlace más próspero.

Con un agudo giro de su muñeca, la sábana cayó y la estatua fue
revelada. Era de un joven atacando en una pose noble. La estatua era de
un mármol blanco brillante, pero sus labios, ojos y pelo habían sido
pintados con colores brillantes: Rose pensó personalmente que aquello
era de un poco de mal gusto, un psao por detrás de dibujarle un bigote y
unas gafas con un bolígrafo.

Gracilis suspiró sonoramente.

--Se suponía que este día iba a ser de celebración--dijo, girándose al
Doctor y a Rose--. El Liberalia, el día en el que mi hijo se ponía al
fin la \emph{toga virilis} y se convertía en un hombre a los ojos del
mundo--señaló hacia el chico de piedra--. Esto era para conmemorar ese
día tan trascendental. Pero al menos espero que nos pueda ayudar en
nuestro apuro.

--Este es Optatus--dijo Marcia, irrumpiendo en lágrimas, y arrojándose a
sí misma a la estatua, abrazando sus rodillas fuertemente como si
aquello pudiera impedir que su hijo se fuera de nuevo--. Y ahora así
podréis encontrarle.