\chapter*{Capítulo Ocho}
\addcontentsline{toc}{chapter}{Capítulo Ocho}

El Doctor estampó su puño contra las barras de la celda.

--¡Todo esto por tomar prestado un caballo! ¡Es ridículo!

Ringo rió.

--¿Crees que eso es algo? Anda que yo, tenía un bonito negocio vendiendo
obras de arte, nunca dije que eran griegas de verdad, no puedo evitar
que la gente lo piense y de repente aquí estoy yo, listo para ir a
conocer a Plutón.

Obviamente le había tocado la fibra sensible.

--Por lo que me pillaron a mí es bastante serio--dijo George,
misteriosamente--. Lo único es que yo no lo hice--se detuvo para
pensarlo un momento, y entonces siguió--. Hacía los mejores pasteles en
la ciudad, sí señor. La gente venía de millas lejos por uno de mis
pasteles. Entonces un día vino un chico. Parecía rico, pero se le veía
que acababa de estar en una pelea, y a penas se podía mantener en pie
por la bebida. Iba zarandeando su bolsa del dinero, así que le di un
pastel y le pregunté si quería que fuera a buscar un médico. Lo
siguiente que sé es que está en el suelo. Muerto.

--Ya supongo lo que vino después--dijo el Doctor.

George se sorbió los mocos.

--Todos dijeron que le había envenenado. No importaba que apenas hubiera
tocado su pastel y que estuviera medio muerto cuando entró. Dijeron que
lo hice para coger su dinero. Algún renacuajo salió corriendo con su
monedero mientras yo intentaba reanimarle. Todos mis amigos dicen que no
lo hice. Pero su familia era rica. Así que no había esperanza.

--Lo siento--murmuró el Doctor.

Se sentaron en silencio por un momento. Entonces todo el mundo se quedó
quieto mientras unos pasos se acercaban a la celda de la mazmorra. Todo
el mundo excepto el Doctor.

--Intentad recordar qué os he dicho--les dijo a los demás--. Si nos
mantenemos juntos, ¿quién sabe lo que podremos conseguir?

--Bueno, para comenzar podremos conseguir la muerte\ldots{}--comenzó
Paul--. Sí, sí, de acuerdo, merece la pena intentarlo.

--¡De acuerdo! --gritó una voz mientras alguien abría la puerta de la
mazmorra.

Una antorcha se hizo camino por entre las barras, ligeramente iluminando
las caras de los cuatro amigos del Doctor: John tenía barba, Paul era
fornido, George tenía ojos tristes y Ringo era delgaducho. También
iluminaba la burlona cara de Thermus, que la estaba sujetando. Pero sus
siguientes palabras fueron de sorpresa.

--¿Dónde está el tipo que no deja de quejarse de estar aquí ilegalmente?

El Doctor se levantó y dio un paso adelante.

--¿Sí? --dijo, secamente--. ¿Qué quieres?

Pudo ver que Thermus levantaba las cejas.

--Pues bien, ¿es esa la manera de hablar con alguien que ha venido a
sacarte?

--Yo no diría que lo es, Thermus--dijo Flaccus, estando detrás de él,
sujetando la llave de la mazmorra--. Yo diría que más bien es la manera
de un tipo desagradecido que no aprecia todo lo que hemos estado
haciendo por él.

--¿Qué? --dijo el Doctor.

--Debe tener amigos influyentes, señor--siguió Thermus, mientras giraba
la llave en la cerradura--. Nos han dicho que todo ha sido un gran error
y que será liberado. Creo que Rufus intenta disculparse con usted
personalmente.

¡Gracilis! Pensó el Doctor. Conocía gente en Roma, debía de haber
arreglado las cosas. Sintió una ola de gratitud hacia el anciano.

--Claro. Bueno, bien. Debo tener unas palabras con mis amigos\ldots{}

--No hay tiempo para ello, señor--le dijo Flaccus--. No debemos mantener
a un hombre como usted entre estos criminales--Thermus le cogió del
brazo al Doctor y le escoltó firmemente, con mucho más respeto que
antes, hacia el exterior de la celda.

--¡Recordad lo que os he dicho! --les gritó el Doctor por encima de su
hombro--. ¡Trabajad juntos!

Los dos guardas le llevaron por los oscuros pasadizos. Pasaron por una
alcoba en la que había una mesa con varios frascos de vino y unos
cuantos dados, obviamente la sala de los guardas. Había algo más en la
mesa, una pequeña bolsa de tela. El Doctor se acercó y la agarró.

--Esto es mío, creo--dijo.

Thermus se encogió de hombros. El Doctor adivinó que habrían cogido la
mayoría de las monedas que llevaba, pero no iba a comenzar una lucha por
aquello. Habían dejado el destornillador sónico, que era lo principal:
se preguntó qué demonios habrían hecho con ello.

No seguían los pasos de la noche anterior; el Doctor estaba siendo
llevado por una dirección distinta. El ruido de la arena se iba haciendo
más fuerte y recordó con una sacudida de culpa sobre los hombres a los
que acababa de dejar atrás. A pesar de sus palabras de ánimo, sabía que
era improbable de ninguno de ellos volviera a ver nunca a sus familias.

--¿Quién ha arreglado esto? --les preguntó a los guardas--. ¿Ha sido
Gracilis?

--¿Gracilis? --dijo Thermus--. Sí, creo que ese era el nombre, ¿no es
así, Flaccus?

--Creo que sí que lo es--dijo su colega--. Ha sido un buen amigo con
usted, ese Gracilis.

--Así es--dijo el Doctor.

--De hecho, creo que está fuera esperándole. Justo ahí, señor-- Flaccus
señaló a una rampa. Había una puerta en lo alto.

Los tres hombres subieron la rampa. Thermus abrió la puerta y, en un
desagradable eco de la noche anterior, el Doctor se encontró a sí mismo
siendo empujado. Cuando la puerta se cerró de golpe detrás de él, oyó
risotadas viniendo de los dos guardas.

Gracilis no le estaba esperando para saludarle.

Nadie le estaba esperando para saludarle, a no ser que contaras a los
cientos de miles de ciudadanos romanos gritando.

El Doctor estaba en la arena.

La arena era enorme, más grande que un campo de fútbol. El suelo estaba
cubierto de una fina arena blanca, para absorber la esperada sangre.
Tenía cuatro hileras de asientos estaban llenas de romanos gritones y
una pared de mármol coronada por una verja protegiendo a los
espectadores más cercanos de los eventos teniendo lugar delante de
ellos. El Doctor atisbó la satisfecha cara de Lucius Aelius Rufus en la
hilera más baja de las gradas.

Los ojos del Señor del Tiempo parpadearon por la arena, buscando algo,
cualquier cosa, que pudiera usar para ayudarse. Podría usar una espada
como el mejor de ellos, pero no tenía espada. No tenía ningún tipo de
armas. Los árboles se habían colocado en el suelo de la arena y corrió
hacia uno de ellos, tampoco es que esperara que mejorar algo de
protección fuera lo que fuera a lo que tuviera que enfrentarse.

Hubo un grito de placer de la multitud. Una trampilla se había abierto
en el borde de la arena, seguida por otra y otra. Lentamente y a
regañadientes, los animales eran forzados a pasar por los agujeros.
Leones, tigres y osos.

--¡Oh! --gritó el Doctor, cuando las trampillas se cerraron de nuevo.

Los animales parecían delgados y aletargados, medio muertos de hambre.
No querían atacar, aún no. Pero el Doctor sabía que eso no duraría.
Había un aroma a sangre en la arena de la anterior masacre, y eso les
haría verle a él de una forma diferente en un minuto. Y por si había
alguna resistencia, George le había dicho cómo los entrenadores (los
\emph{bestiarii}) pronto estarían animándoles con fuego, armas y carne
cruda.

El Doctor hizo una reverencia a la multitud. Les gusto aquello y
aplaudieron. Se giró para mirar a Rufus.

--\emph{Nos morituri te salutamus}--gritó, aunque el saludo que hizo no
fue probablemente reconocido por el magistrado. Aún así, la multitud
también aplaudió a aquello.

Un león se acercaba, un macho adulto cuya magnificencia de antaño ahora
parecía andrajosa y sosa.

--Pareces cansado--le dijo el Doctor con suavidad--. ¿Qué vida es esta
para ti, rey de la jungla? --se le ocurrió algo--. Creo que es hora de
que vayas a dormir--agarró su bolsillo del cinturón y sacó el
destornillador sónico. Un par de giros y señaló al león que se acercaba
y ahora comenzaba a rugirle, desde lo más profundo de su garganta.

Sabía que no debería usar su destornillador sónico, no delante de
cientos y cientos de humanos primitivos. Pero iba a hacerlo de cualquier
forma. No demasiado porque se fuera a salvar a sí mismo, sino porque aún
tenía que salvar a Rose. Y nadie le iba a detener de hacer aquello.

Hubo un ligero ronroneo del destornillador sónico, pero sólo el Doctor
sabía que estaba haciendo otro pitido, una onda de sonido inaudible para
el oído humano, pero el cual el león definitivamente iba a pillar. Y así
fue, pues el león giró en redondo y se alejó. Unos momentos más tarde,
se tumbó en el suelo, cual gato viejo cerca de una chimenea.

Hubo risas y quejas desde la multitud, unas cuantas risas por aquel
hombre extraño alejando a un león con lo que parecía un pequeño palo,
pero principalmente quejas de los que solicitaban sangre.

--¡Vamos! --le gritó el Doctor al público--. ¿De verdad queríais que
terminara todo tan rápido? ¿No es la anticipación algo parecido al
placer?

Ahora un enorme oso negro se le acercaba.

--Es difícil creer--le dijo el Doctor al oso--, que los ositos de
peluche sean tan monos y tú\ldots{} no lo seas. No te ofendas.

Levantó de nuevo su destornillador sónico, pero el oso no se detuvo.

--Ah--dijo el Doctor--. Tomar nota. Los osos necesitan una frecuencia
distinta.

El oso estaba acelerando, obviamente sintiendo su presa. El Doctor se
apoyó contra el árbol más cercano y comenzó a escalarlo, acercándose a
la copa cuando el oso le alcanzó.

El oso se levantó sobre sus patas traseras y agarró el árbol,
zarandeándolo.

--Segunda nota--dijo el Doctor--. A los osos no les asustan los árboles.

Sabía que si saltaba, el oso estaría sobre él en un momento. Agarrándose
fuertemente al árbol con una mano, intentó ajustar los niveles del
destornillador sónico con la otra, pero el oso le daba al árbol unas
sacudidas demasiado fuertes y el dispositivo se le cayó al suelo. Y
ahora el oso comenzaba a escalar el árbol. No era un árbol grande,
probablemente no aguantaría su peso demasiado, pero o bien el oso le
atraparía cuando llegara a su altura o cuando el árbol se derrumbara y
eso no significaría mucha diferencia para el Doctor.

Lenta y cuidadosamente, dio la vuelta por el árbol hasta que se hubo
situado encima del oso. La criatura sacó una furiosa garra hacia arriba,
aún fuera de su alcance. Pero en cualquier segundo llegaría\ldots{}

El Doctor saltó. No al suelo, sino al lomo del oso. Confundido, cayó de
nuevo al suelo, a cuatro patas, e intentó deshacerse de él. El Doctor se
agarró firmemente. El oso volvió a ponerse sobre sus patas traseras,
rugiendo con agitación.

--Relájate, Peluchito--dijo el Doctor--. No es fácil ser un jinete de
oso, ¿sabes?

A la multitud le encantaba aquello. No era tan bueno como una muerte,
pero les entretenía de cualquier forma, el joven delgaducho tratando
aquella fiera como si fuera un burro o una mula.

El oso volvió a ponerse a cuatro patas. El Doctor se tensó,
preguntándose cuál sería su siguiente movimiento. De repente, se
inclinó, listo para rodar y liberarse de la irritación de su espalda.
¿Qué iba a hacer? Si se seguía sujetando, sería aplastado, pero si se
dejaba ir, el oso estaría sobre él en un segundo\ldots{}

El oso comenzó a caer, y el Doctor vio un brillo en el suelo, por el
rabillo del ojo. Rodó con la criatura, saltando en el último momento y
agarrando el brillo. Se levantó con el destornillador sónico en la mano,
y cuando el oso se incorporaba y se preparaba para atacar, lo apuntó
hacia adelante.

Y el oso se detuvo. Dio un quejido y comenzó a retroceder, mirando al
Doctor con odio.

--Lo siento--murmuró el Doctor.

Pero a la multitud no le gustó aquello. Dos animales vencidos y ni una
gota de sangre derramada. Si no podían tener la sangre del Doctor se las
apañarían para tener la de las bestias, pero no iban a conseguir ninguna
de las dos.

Aquellos a cargo obviamente percibieron el humor de la multitud y sabían
que algo tenía que pasar pronto. Una puerta se abrió y salieron dos
\emph{bestiarii} a la arena. Uno sujetaba una antorcha encendida,
mientras otro sujetaba un tridente tan alto como él, el cual bajó para
apuntarlo hacia adelante. Se acercaron a un tigre merodeando con la
confianza de aquellos que van armados y que tenían la mano superior. Uno
hombre asustó a las bestias atigradas con las llamas y el otro usaba las
puntas del tridente para pincharlo y llevarlo en la dirección del
Doctor.

El Doctor no se quedó en el mismo sitio, por supuesto. Los
\emph{bestiarii} comenzaron a apuntar hacia una y otra dirección,
intentando mantener al animal en el camino. Aunque el Doctor tenía una
firme confianza en que podría estar así todo el día, sospechó que no se
le permitiría hacerlo. Mejor acabar con aquello. Se paró en seco y se
apoyó casualmente contra la pared de mármol.

--¡Vamos, pues! --gritó a los hombres que se acercaban, que sonrieron
ante la idea de poder tener a su presa al fin.

--¿Estáis teniendo un buen día? --les gritó a los de los asientos más
cercanos, consiguiendo una risa en respuesta, a excepción del cercano
Rufus, que le frunció el ceño--. ¡Ey, sonríeme! --le gritó el Doctor--.
Deberías estar feliz, has provisto a la multitud del mejor espectáculo
en años, mejorando lo presente.

Pero Rufus seguía frunciendo el ceño. Y mientras tanto, el tigre se
acercaba, gruñendo a media por los que le atormentaban y a medias por el
Doctor.

El Doctor de repente entró en acción, tomando a todo el mundo por
sorpresa, incluido el tigre. Saltó por encima de la cabeza de la bestia,
con las manos hacia adelante como quién salta al potro. Con una pirueta,
estaba en pie en la cola del animal, con los brazos en el aire para
marcar un aterrizaje perfecto. El que sujetaba la antorcha estaba más
cerca y el Doctor le agarró la antorcha al hombre sorprendido, usándola
para arrancarle el tridente al otro.

--¡No intentéis esto en casa, amigos! --le gritó a la multitud, mientras
los \emph{bestiarii} seguían allí en pie en silencio, incapaces de creer
la forma en la que las tablas se acababan de girar. Siguieron así por un
solo segundo. Ahora el Doctor volvía a moverse y se giraron para
seguirle.

Pero el tigre también se giró. Aquellos hombres eran los más cercanos, y
eran los que le habían estado persiguiendo, haciéndole daño\ldots{}

La multitud estuvo, momentáneamente, satisfecha.

Pero el Doctor sabía que pronto estarían aullando por su sangre de
nuevo. Podría seguir esquivándoles y podría luchar, pero ellos seguirían
mandándole más y más cosas: animales y hombres. Tenía que salir de allí.

Pero nadie había escapado de la arena.
