\chapter*{Capítulo Quince}
\addcontentsline{toc}{chapter}{Capítulo Quince}

Rose y Vanessa estuvieron sentadas en silencio, intentando tramar un
plan. Desganadamente, Rose cogió una patata ya fría.

--Probablemente se han necesitado ambas piernas de Ursus para hacer
aparecer por arte de magia este par de patatas--dijo, dejando el tenedor
de nuevo.

--Oh, no--dijo el GENIO--. Ese ha sido un deseo muy simple.
Probablemente no ha valido más que un globo ocular.

Rose apartó la bolsa firmemente y se levantó.

--Mira, no sirve de nada estar aquí todo el día deseando, quiero decir,
esperando--se corrigió apresuradamente--, que algo pase de repente.

--¿Entonces tenemos que ir y matar mucha gente? --dijo Vanessa,
tristemente.

--Bueno, es Roma--dijo Rose, pretendiendo considerarlo--. Podríamos
montar una arena falsa y dejar que aquí el GENIO se disfrazara de
león--negó con la cabeza--. No, lo que necesitamos es una fuente de
energía alternativa. Ya sabes, como granjas eólicas, paneles solares o
algo.

--¿Y cómo vamos a vivir mientras tanto? --preguntó Vanessa--. ¿Nos
quedamos aquí y vamos pidiendo patatas?

Rose se encogió de hombros.

--Bueno, tú te las has arreglado para vivir hasta ahora--dijo--. Voto
para que volvamos con Gracilis. El Doctor dijo que iba a traer a todo el
mundo a la vida en un par de días, pero si no está más aquí--sujetó el
frasco--, bueno, será mejor que seamos las heroínas del momento.

--¿El Doctor es el tío que vino aquí contigo hasta que deseaste que no
lo hubiera hecho? --dijo Vanessa.

--Ese es--dijo Rose.

Vanessa asintió.

--Pero si vamos, ¿qué pasa con\ldots{}?--señaló al GENIO.

--Bueno, no voy a dejarlo aquí--dijo Rose, decisivamente--. Mira, GENIO,
¿sabes todo eso de pretender ser la diosa Minerva y tal?

--Meramente obedecía los deseos de mi por aquel entonces
controlador--dijo el GENIO.

--Sí, da igual, bueno, si te llevamos con nosotras, tendrás que tener la
apariencia de otra cosa. Un perro o algo--se giró a Vanessa--. ¿Los
romanos tienen perros?

--Eso creo--dijo Vanessa, sin sonar muy segura.

--Bueno, ¿sabes de alguna mascota que seguro que tengan? Preferiblemente
algunas que se puedan llevar encima.

Vanessa lo pensó.

--He visto a un par de personas con monos--dijo al final.

--Brillante--dijo Rose--. Perfecto. GENIO, conviértete en un mono.

El GENIO chasqueó la lengua.

--Tienes que desearlo\ldots{}

--Claro. De acuerdo. Da igual. Genio, d\ldots{}¡espera un momento!
--Rose se tapó la boca de golpe y esta vez era la suya propia--. Si te
hubiera ordenado que te convirtieras en un mono, tendrías que
convertirte en un mono, ¿verdad? Un pequeño mono peludo, obsesionado con
los plátanos y sin ninguna habilidad para conceder deseos.

--Oh, comienzas a pensar--dijo el GENIO--. Destripas mi diversión, ¿por
qué haces eso?

Rose se lo quedó mirando y entonces dijo con mucho cuidado:

--Desearía que adoptaras la apariencia de un mono cuando haya cualquier
romano cerca para verte, y que mantengas todas las habilidades de un
GENIO.

Resonó un trueno.

--Tus deseos son órdenes--dijo el GENIO.

--Pero sigues pareciendo el mismo.

El GENIO suspiró condescendientemente.

--No hay romanos cerca para verme. Me he aferrado al pie de la letra del
deseo.

--Vamos a tener que confiar en él--dijo Rose, cogiendo la caja de
cartón. Ç

De cerca, el GENIO olía ligeramente a metálico, y sus escamas brillaban
con un toque broncíneo al la luz del exterior. De repente podía creerse
que fuera una cosa hecha, una construcción, más que un extraño
alienígena, y sintió un ataque de pena por él. No era realmente
responsable por todo aquel horrible lío, estaba haciendo para lo que
había sido hecho o quizás para lo que creía que había estado hecho.
Nunca entendería el jaleo de la inteligencia artificial, no estaba
segura si aceptaba por completo la idea de un ordenador pensando por sí
mismo, teniendo esperanzas y sueños (aunque hubiera visto esa película
de Spielberg sólo porque salía Jude Law)\ldots{} Pero quizás podría
aceptar que la IA pensara por sí misma, aunque no fuera así. O\ldots{}
no, mejor sería dejarlo.

Rose había estado preocupada de que nunca pudieran encontrar la forma de
salir del bosque, pero por suerte su camino había creado bastante camino
a través de los matorrales para que lo siguieran sólo perdiéndose un par
de veces. Para su alivio, cuando finalmente llegaron al carro de Ursus,
el burro seguía pastando plácidamente, totalmente inconsciente de
cualquier drama de muerte, viajes temporales o ser secuestrado en un
lugar 2000 años antes de tu nacimiento que estuviera teniendo lugar allí
cerca. Rose colocó el GENIO en la parte trasera del carro con Vanessa y
subió al frente para intentar llevar al burro de vuelta a la villa de
Gracilis.

Marcia les vio llegar y se apresuró para saludarle.

--¡Rose, estás bien! ¡Gracias a los dioses! Estábamos muy preocupados
por ti. Oh, y qué mono tan bonito es ese.

Rose dejó de lado rápidamente las historias preconcebidas sobre juguetes
infantiles o animales importados de lugares extraños y suspiró aliviada
cuando recogió la caja del GENIO. El mono más adorable y con unas
mejillas sonrosadas de color marrón chocolate le devolvió la mirada con
sus enormes ojos oscuros. Así que le había concedido el deseo.

--¿Lo tenías antes? --preguntó Marcia, curiosa--. No lo recuerdo\ldots{}

--Eh, sí, sí que lo tenía--dijo Rose. Después de todo, sus recuerdos ya
habían estado mezclados, así que una mentirijilla piadosa no haría
daño--. Pero ha estado durmiendo mucho.

Vanessa salió del carro y se puso detrás de ella en silencio. En cuanto
hubieron entrado en las propiedades había vuelto a sus modales tímidos,
aunque no es que hubiera sido muy charlatana en el recorrido.

--De cualquier forma--dijo Marcia, girándose para entrar--. Debo volver,
unos cuantos amigos han venido de visita. Les hemos invitado para ver
nuestra estatua de Optatus, no sería educado rechazar la invitación, a
pesar de las circunstancias. Vosotros dos tenéis que entrar y unirse a
la fiesta.

--Guau--le susurró Rose a Vanessa--, las cosas deben de estar bien. Te
está tratando como un ser humano.

Vanessa puso una sonrisa seca:

--Creo que hablaba de ti y del mono\ldots{}

Rose le pasó la caja de cartón. El GENIO estaba asomándose por el borde,
colgado como un perro con la cabeza fuera de la ventanilla del coche.

--Mira--dijo Rose--, puedes ser mi portadora oficial del mono. No se
podrán quejar de que estés en una fiesta de esa manera.

Vanessa aceptó la tarea.

--Así que\ldots{} ¿seguías desaparecida, entonces? Si Marcia estaba
preocupada por ti.

Rose se encogió de hombres, sin respuestas.

--Eso supongo.

--Pero si tu amigo, el Doctor, nunca estuvo aquí\ldots{}

--Si el Doctor nunca hubiera estado aquí, yo tampoco habría estado
aquí--señaló Rose--. ¡Oh, Dios mío! ¡No debería estar aquí! ¡No debería
ni estar aquí! Así que\ldots{} debe de ser una de esas cosas
paradójicas. Quizá el tiempo esté intentando curarse a sí mismo
manteniéndome aquí.

Cuando siguieron a Marcia hacia el interior, Rose volvió a murmurar:

--No debería estar aquí\ldots{}

Un grupo de gente estaba en la villa, aparentemente los vecinos más
cercanos de Marcia. Había varias parejas, una claramente a penas de
acuerdo formal; una chica que parecía ser la hija de la susodicha
pareja; una desgarbada mujer de mediana edad cuyas brillantes ropas
amarillas de seda no le quedaban bien; una mujer mayor, ataviada de
joyas, cuyo vivido pelo rojo no era claramente el suyo; un joven apuesto
con una túnica verte; y tres o cuatro hombres sin descripción que ya
habían bebido demasiado vino, a juzgar por su incesante risa.

Rose había esperado que estuvieran de pie con las bebidas charlando,
como los sosos guateques en su tiempo, pero en lugar de eso todo el
mundo estaba tumbado en camas como en la cena, mientras un grupo de
chicas africanas ligeras de ropa bailaba a su alrededor.

--¿Dónde está Gracilis? --preguntó Rose, cuando se tumbó rápidamente en
la cama que indicó Marcia.

Vanessa se puso de pie detrás de Rose, a la manera de los demás esclavos
de la sala, aún sujetando la caja que contenía al GENIO.

--Oh, cielos, ha ido a Roma a buscarte. Estábamos tan
preocupados\ldots{}

Rose frunció el ceño. Detectó la mano del Doctor en todo aquello.

--¿Ha ido solo, entonces?

Marcia pareció confusa por un momento.

--Pues\ldots{} por supuesto. No, no, creo\ldots{} Por supuesto, la
esclava Vanessa fue también a buscarte, pero no volvió dijo mi marido,
bueno, no me acuerdo exactamente lo que dijo\ldots{} Oh, sí que tenía
que mandar un mensaje si Vanessa volvía. Supongo que será mejor que haga
eso\ldots{}

--¿Un mensaje a quién? --preguntó Rose.

--A\ldots{} mi marido, por supuesto.

Marcia parecía tan insegura de sí misma que Rose lo sintió por ella.
Obviamente el tiempo no se había estado curando tan bien. Parecía que
simplemente había colocado una tirita encima de la herida y había
esperado que fueran bien las cosas. Las ocasiones en las que el Doctor
había estado presente ahora eran un poco borrosas para todo el mundo,
excepto para ella. ¡El estúpido GENIO y sus deseos! Si nunca volviera a
oír aquel ridículo sonido de trueno otra vez no lo lamentaría\ldots{}

¡Crash!

Rose pegó un bote. Ya no estaba sentada en la cama junto a Marcia, sino
junto al joven de la capa verde.

Él también pegó un bote.

--¡Oh! --dijo.

--No me digas--dijo Rose con una sonrisa--, estabas deseando conocerme
un poco mejor.

--Bueno, ahora que lo mencionas\ldots{}--dijo.

--Soy Rose--le dijo mientras le gustaba la mirada de sus ojos azul
oscuro y su ligeramente avergonzada sonrisa--. Rose Tyler.

--Crispus. Quintus Junius Crispus.

Un esclavo le pasó a Rose una copa de vino, la cual aceptó y entonces se
lo pensó mejor, recordando lo último que le pasó con la bebida. Así que
cuando Crispus de repente dijo ``Ursus'', casi se cayó de la cama.

Obligándose a relajarse, Rose dijo:

--¿Ursus? ¿Qué pasa con él?

--He oído que estaba haciendo una escultura de ti. Me encantaría verla.

--Sí, bueno, eso no va a pasar probablemente--le dijo--. He decidido que
no voy a ser una modelo.

--Oh--dijo, claramente sin entenderlo--. Es una lástima. Creo que
Cornelia tenía muchas ganas de hablar contigo.

Rose frunció el ceño.

--¿Cornelia?

Señaló hacia la mujer con las ropas amarillas.

--Cornelia. La madre de Ursus.

Rose se quedó congelada. No era alguien con quien quisiera hablar. Pero
era demasiado tarde. La mujer había visto su mirada y aprovechó la
oportunidad para entablar contacto. Se acercó, con sus andares muy poco
gráciles recordándole a Rose a los de un vaquero dirigiéndose al rodeo.
La torpeza de Ursus claramente le venía de familia.

--Debes de ser Rose--le dijo, alargándole la mano.

Rose miró a los gruesos dedos rosas y recordó aquellas manos regordetas
que se le acercaban en el taller. No podía tomarle la mano a aquella
mujer, simplemente no podía.

Tras un segundo, la mano fue retirada. Rose quería que se la tragara el
suelo. Por supuesto, si decía aquello en voz alta, probablemente acabara
cayendo de repente hacia Australia, no ¿qué estaba opuesto a Italia?
¿Nueva Zelanda?

Cornelia habló y devolvió a Rose a la realidad.

--Me decepciona que mi hijo no esté aquí--dijo la mujer--. Entre tú y
yo, fue una decepción para nosotros durante años. Me trae alegría que
haya encontrado el éxito al fin, aunque sea como un artesano--sonrió
apreciando a Rose--. Y es encantador que haya estado haciendo una
estatua tuya. Estoy segura de que no pudo resistirse de inmortalizar a
alguien tan joven y bella.

Rose puso una mueca de ``Oh, no, no creo'', sin aún confiar en decir
nada.

--Desearía que me lo contaras todo--dijo Cornelia.

Un sonido recorrió las orejas de Rose. ¡El GENIO lo había oído! Abrió su
boca para protestar, pero lo que salió en lugar de eso fue:

--Tu hijo me drogó y entonces me convirtió en piedra usando un poder que
le había dado un ArtiluGio Establecido para Necesidades Imaginativas
Operativas del siglo XXIV, disfrazado de la diosa Minerva. Mis amigos,
el último de los Señores del Tiempo que ahora resulta que nunca ha
estado aquí y una chica del futuro, me devolvieron a la vida y
perseguimos a Ursus a un templo en ruinas dónde petrificó a ambos y yo
hice que cayera encima de su daga, hiriéndole fatalmente. Su cuerpo fue
entonces absorbido por el GENIO, el cual está ahí pero ahora te parecerá
que estás viendo un mono.

Cornelia parecía que fuera a desmayarse.

Rose estaba desesperadamente intentando hacer algo, cuando\ldots{}

¡CRASH!

Rose sabía que no había deseado, al menos no en voz alta, una
distracción, pero llegó una. Hubo alientos contenidos y risitas de los
romanos reunidos. Las bailarinas africanas se tambalearon en medio de su
rutina diaria, justo cuando sus ya ligeros vestuarios desaparecieron de
golpe. Las bailarinas se apresuraron a salir de la sala, avergonzadas.
Rose sospechó que los dos caballeros con las expresiones confusas e
incrédulas fueran los deseosos culpables en ese caso.

¡CRASH!

La chica con los padres descontentos gritó. Su padre había desaparecido,
eliminado de la existencia como si nunca hubiera existido. Su madre se
quedó boquiabierta, también sorprendida pero, así le pareció a Rose,
también aliviada.

¡CRASH!

Dónde había estado la anciana ahora había un pequeño bebé, cuyos lloros
infantiles quedaban ahogados debajo de la peluca pelirroja que le había
caído por encima.

--Creo que deseó haber sido joven de nuevo--murmuró Rose--.
Probablemente no haya planeado lo de los pañales.

Vanessa parecía alarmada, sujetando al GENIO con toda la envergadura de
sus brazos.

Rose saltó y se apresuró a acercarse a ella.

--Tenemos que salir de aquí antes de que haga más daño--dijo, cogiendo
la osa. Miró al mono--. ¿Podrías detener esto, traficante de deseos?

--Es mi función--dijo el GENIO, con poco ánimo de ayudar--. Tengo que
cumplir cualquier deseo que oiga, mientras tenga el poder de hacerlo.
¡Ajá!

El sonido del trueno no dejó a Rose duda alguna del significado de la
exclamación del GENIO. Miró a su alrededor a toda prisa, intentando ver
qué deseo acababa de conceder.

Le llevó un momento verlo. Las ropas del joven Crispus se habían
convertido de repente en un morado brillante y una corona de laural
había aparecido en su cabeza. Pero por supuesto, así es cómo debía ir
vestido. Él era el emperador, después de todo. Imperator Caesar Quintus
Junius Crispus Augustus, princeps de Roma, y ella, Rose, ella era
su\ldots{} ¿concubina?

La gente en la sala se estaba inclinando, algunos postrándose en el
suelo. Rose estuvo a punto de hacer lo mismo.

Pero entonces se detuvo. Era una chica del siglo XXI. Inclinarse y
postrarse no era algo natural para ella. de cualquier manera, no eran
así como debían ser las cosas. Algo iba mal\ldots{}

Miró hacia la caja que estaba sujetando.

Dentro había un mono.

No, el GENIO. El que concedía deseos. Aquello no era real.

--¡Tú no eres el emperador! --gritó en voz alta.

Un gran error. Hubo grititos de sorpresa por toda la sala.

Crispus se puso en pie de golpe.

--¿Qué? Te cortaré la cabeza por ello. ¡Cogedla! --gritó,
imperiosamente.

Rose agarró a una Vanessa inclinada, arrastrándola por la puerta.

--¡Déjame ir! --gritó Vanessa, removiéndose, pero Rose no podía
abandonarla en toda aquella locura.

Los hombres borrachos se pusieron en pie y varios esclavos fornidos ya
estaban acercándose a las dos chicas. Rose apretó el paso, llegando casi
a la salida, con Vanessa tambaleándose detrás de ella, pero los esclavos
les estaban alcanzando.

--¡Vamos! --le gritó a Vanessa.

--¡El emperador nos matará por huir! --se quejó Vanessa.

--¡Él no es el emperador! --le gritó Rose, sin aliento.

Vanessa se quejó más fuerte:

--¡Te matará por decir eso!

--Así que voy a morir dos veces. Típica muerte imperial. ¡Vamos!

Vanessa gritó cuando un esclavo le agarró de la túnica. Rose se giró
para ayudarla pero la agarraron a ella también, otro esclavo que le
cogió de los brazos fuertemente. Dejó caer la caja que contenía el GENIO
cuando el esclavo comenzó a arrastrarla hacia Crispus. Era ahora o
nunca. Desear algo quizá provocaría salir de la sartén y caer en las
brasas, pero tenía que tener esperanza, porque la sartén ya estaba
demasiado caliente en ese momento\ldots{}

No pudo pensar qué decir, de una frase segura que no fuera
malinterpretada. Segura\ldots{} aquello era lo principal, lo único que
importaba en aquel momento.

--¡Desearía que Vanessa y yo estuviéramos a salvo! --gritó Rose.

¡CRASH!

Y todo desapareció.
