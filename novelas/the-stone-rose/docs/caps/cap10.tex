\chapter*{Capítulo Diez}
\addcontentsline{toc}{chapter}{Capítulo Diez}

Los corazones del Doctor se detuvieron cuando vio a Tiro petrificado.
Con tristeza, gritó al hombre cerca de la estatua.

--Perdóneme, ¿no está Ursus en persona para ver su obra de arte en su
sitio?

El hombre se encogió de hombros.

--No. Ni siquiera está en Roma. Tuvimos que mandar a un carro para coger
la estatua de una villa campestre en la que se hospeda.

--¿Seguro que no está en Roma? --insistió el Doctor-- ¿No sabes de
ninguna otra nueva estatua suya que tuviera que ser colocada o algo?
Algo un poco más femenino, quizás. Algo un poco más de Fortuna.

--Te doy mi palabra de que no--dijo el hombre--. Confía en mí, lo
sabría.

--¿Qué hacemos ahora? --preguntó Gracilis cuando el Doctor se giró--.
¿Volvemos a la villa?

El Doctor negó con la cabeza.

--No--dijo--, hay trabajo que hacer aquí primero. Comenzando por esta
misma estatua. No quiero hacer cundir el pánico, así que será mejor que
esperemos al anochecer\ldots{}--miró hacia el sol, comprobando su
situación--. Oh, ¿a quién quiero engañar? No puedo esperar tanto.
Hagamos cundir el pánico. Oh, y esperemos que haga lo que creo que hace,
o puede que en su lugar haya un linchamiento.

Se giró y corrió hacia la estatua, colocándose en un plinto detrás de
ésta. La gente, que había comenzado a alejarse, percibió que habría más
entretenimiento y giraron sus pasos. Si alguno de ellos reconocía aquel
hombre con patillas como el fugitivo de la arena de antes, se callaron,
y como los ciudadanos de Roma no eran conocidos por su taciturnidad o su
benevolencia, las opciones fueron buenas si no había sido reconocido.

El Doctor hizo una reverencia, con cuidado de no caer del plinto.

--¡Damas y caballeros! --gritó, imitando el anuncio anterior--. ¡Os
presento al dios Mercurio! --nadie parecía dispuesto a reírse de su
repetición, así que siguió--. Como sabéis, esta es una época de
festividad. Hemos celebrado y honrado a la divina Minerva, la del escudo
y la lanza. Nacida ya crecida y armada salida de la cabeza de su padre,
creámoslo o no. Seguro que tuvo que doler. Ahora estamos llevando a cabo
las celebraciones para el justo y divino y aún más bélico Marte. ¿Y
sabéis qué? Están muy agradecidos por vuestras celebraciones, sin
mencionar vuestras ofrendas y vuestras borracheras en sus nombres. ¿Y
quién mejor para traer su mensaje de agradecimiento a vosotros que
Mercurio, el mensajero de los dioses?

El Doctor sacó el frasco a escondidas y quitó cuidadosamente el tapón.
Dejó caer una pequeña gota del liquido del color del jade y ésta cayó en
el mármol de la estatua.

--¡Damas y caballeros--dijo de nuevo--, niños y niñas, os presento al
dios Mercurio!

Por un momento, la multitud observó expectante. Uno o dos se giraron.

Y entonces una mujer gritó. Y otra. Dónde había estado mármol blanco,
ahora había una mancha rosada. Se extendió por toda la estatua, la
piedra siendo comida por la mancha de la carne. Los labios pintados se
volvieron suaves y prominentes, los vacíos ojos fueron reemplazados por
unos orbes verdes brillantes. Ante la asombrada multitud ahora había un
hombre vivo, vestido con el sombrero y las sandalias aladas de Mercurio,
sujetando el caduceo de Mercurio, su vara con dos serpientes enrolladas
a su alrededor. Para la sorpresa del Doctor, las serpientes de piedra
sisearon de repente, con sus escamas volviéndose amarillas mientras se
desenrollaban de la vara y se arrastraban reptando. Causaron incluso más
gritos entre la multitud.

--Diles que les traes un mensaje de paz y amor--le silbó el Doctor al
confuso Tiro, agarrándole por la cintura para evitar que se cayera hacia
adelante--. Confía en mí. Entonces podremos irnos de aquí.

Tiro, cuyos ojos alocados apenas podían enfocarse en la multitud delante
de él, gritó:

--¡Os traigo un mensaje de paz y amor!

La multitud enloqueció, gritando y chillando. El Doctor bajó a Tiro y se
lo pasó a Gracilis, que les esperaba abajo, y le susurró:

--Sácalo de la vista.

--¡Diversión para el festival! --gritó el Doctor, intentando conseguir
la atención de la multitud y permitiendo a Gracilis y a Tiro abrirse
camino--. ¡Y aquí hay un poco más! --sacó una pequeña moneda de
bronce--. ¡Un as! No vale mucho. Pero no querríais perderlo--abrió ambas
manos, con las palmas hacia arriba--. ¡Y yo voy y lo pierdo! --señaló
hacia la multitud--. Usted, señora, ¿ha visto mi as? Le ruego que me
disculpe, señor, no le había visto el doble sentido. ¿Usted sí?
Entonces, señora, ¿qué es eso que tiene en la oreja?

Sacó la moneda de la oreja de la mujer, para su regocijo.

--¡Ahora yo! ¡Yo! ¡Yo! --gritaron varios niños, que parecían más
impresionados con el truco callejero del Doctor que con el espectáculo
de una estatua de mármol volviendo a la vida. El Doctor siguió con ello
un par de momentos más hasta que consideró que les había dado a Gracilis
y a Tiro el tiempo suficiente, entonces hizo su propia escapada, dejando
atrás un grupo de niños emocionados que ya se estaban planteando cómo
gastarse su recién ganado botín.

Encontró a los demás escondiéndose discretamente tras un pilar en una
calle tranquila. Ambos parecían conmovidos.

--Si no lo hubiera visto con mis propios ojos\ldots{}--murmuró Gracilis.

--Un poco de sorpresa para ambos--dijo el Doctor--. Quizá sea más fácil
si hubiera explicado antes que Ursus es\ldots{} un brujo malvado que ha
estado por ahí convirtiendo a la gente en piedra para sus propios fines.
Y aún así, no lo ha conseguido. De cualquier manera, tenemos trabajo que
hacer.

Comenzó a caminar, dejando a la pareja de sorprendidos que le siguieran.

--Quieres decir--dijo Gracilis, corriendo para alcanzarle--, que
Optatus\ldots{}

Todo el horror de la situación repentinamente pareció golpearle y se
habría caído al suelo de no ser por Tiro que le cogió.

El Doctor se detuvo.

--Sí--dijo--, creo que has entendido la idea. Lo lamento mucho. Pero
vamos a traerle de vuelta, en cuanto lo hayamos hecho con el resto.

Lo que quedó de día lo pasaron dando vueltas por Roma en busca de todas
las estatuas de Ursus. Gracilis, como amante del arte, sabía la gente
correcta con la que hablar, así que fue capaz no sólo de descubrir todas
las localizaciones sino para estar seguros de que, por lo que sabía la
gente, las estatuas del escultor sólo se exhibían en la misma Roma. A
parte de la de Gracilis, nadie sabía de otras comisiones privadas fuera
de los muros de la ciudad.

Una por una, las estatuas fueron restauradas a sus estados de vida.
Diana se convirtió en una hermosa mujer de color sujetando un arco. El
Doctor le pasó la capa de Gracilis a una avergonzada Venus. Los gemelos
Castor y Pólux se abrazaron el uno al otro, y entonces al Doctor por
emoción y alivio. Desconcertado esclavo tras otro se bajaron de sus
podios, para que les dijeran que habían estado bajo un hechizo. Para el
alivio del Doctor, esta inadecuada explicación parecía satisfacerlos.

--¿Pero quiénes son toda esta gente? --preguntó Gracilis en un momento
determinado.

--Imagino--dijo el Doctor--, que son esclavos comprados por Ursus para
este propósito en particular.

Gracilis frunció el ceño.

--Aún pertenecen a Ursus--dijo--. No tenemos el derecho de quitárselos.

El Doctor observó a Gracilis con una mirada muy severa.

--Cierto--dijo--. Según la ley romana, un amo puede tratar a su esclavo
como le plazca. Puede azotarlo, torturarlo, matarlo, convertirlo en una
novedosa estatua de mármol si cree que es una buena idea. Y tú eres un
buen romano, eso lo sé. Pero mírame y dime que crees que lo que Ursus ha
hecho aquí está bien.

Gracilis le apartó la mirada.

--Creo que Roma puede convertirse en un lugar un poco peligroso para
Ursus en el futuro cercano--siguió el Docto--. Sin contar lo que le va a
pasar cuando le coja. Soy, es cierto, conocido por mi naturaleza
permisiva, pero aún así\ldots{}--levantó una mano para evitar que
Gracilis hablara--. No quiero oír qué les debería pasar a estos
esclavos. Sólo quiero oír lo que les pasará. Creo que eres un buen
hombre. Así que creo que lo que va a pasar es que te vas a asegurar de
que van a estar bien.

Abatido, Gracilis asintió.

El Doctor le dio golpecitos en la espalda.

--¡Buen hombre!

Al cabo del rato quedaba una pequeña cantidad de poción brillante verde,
y el Doctor la guardaba como si fuera la cosa más preciada del universo.
Justo en ese momento, lo era.

--Nunca creí que podría ver tal magia--dijo Gracilis, sorprendido,
mientras el Doctor devolvía el tapón a su sitio otra vez, y mandaba a
Juno, y a su desorientado pavo real, a que esperara el carro de Gracilis
en las afueras de la ciudad, donde los demás esclavos se estaban
reuniendo.

--No es magia--dijo el Doctor, más para sí mismo que para el anciano--.
Sino ciencia-- pero no admitió, ni siquiera para sí mismo, que no tenía
ni idea de cómo la ciencia había creado ese líquido maravilloso, o de
dónde había venido.

Mientras recorrían las calles oyeron hablar de trucos de festivales, de
magia, de dioses caminando entre el mundo de los hombres. El Doctor
sonrió ante las palabras, pero Gracilis se volvió más y más nervioso,
convencido de que les arrestarían en cualquier momento, pero fijado en
seguir hasta el final.

--¿Qué nos van a hacer? --dijo el Doctor, intentando tranquilizarle--. A
no ser que sea el Día al Revés, no pueden multarnos por hacer que la
gente vuelva a la vida.

Finalmente sólo quedó una estatua, y según los contactos de Gracilis, se
encontraba en un jardín de arboles cerca del teatro de Pompeyo. Pero
había algo allí cuando llegaron. El jardín estaba totalmente rodeado de
guardas armados.

El Doctor se acercó:

--¿Qué está pasando? --dijo, con una inocente curiosidad brillando en su
cara.

--Alguien ha estado robando las estatuas del escultor Ursus--le dijo un
guarda--. Pero no van a consguir esta. ¡Les veremos cuando pasen!

El Doctor puso los ojos en blanco.

--¿En qué se está convirtiendo el mundo? --pensó mucho cuando camino de
vuelta casualmente hasta Gracilis. Sólo quedaba una estatua,
comparándola con las docenas de liberadas. Si se arriesgaban en aquella
y se metían en líos, ¿qué sería de Rose y de Optatus?

Pero a través de los árboles vio el brillo del mármol. La estatua de una
joven en un pedestal. Una chica de la edad de Rose, con toda su vida por
delante.

Por supuesto no podía dejarla.

--¿Cómo vas a pasar por entre todos esos guardas? --preguntó Gracilis,
preocupado.

El Doctor lo pensó por un momento. Entonces se le iluminó la cara.

--¡Entrar no es el problema! --dijo--. Se preocupan por alguien que
salga con una estatua. Bueno, no estaba planeando hacer tal cosa. Ahora,
necesito que entretengas a un par de ellos en una conversación,
distraerles mientras yo me escabullo\ldots{}

El Doctor se arrastró hacia el jardín. La mayor parte de los guardas
rodeaban el perímetro, pero había uno justo al lado de la estatua misma.
Afortunadamente, le daba la espalda, pero aún así\ldots{}

Con cuidado y en silencio, el Doctor se acercó. La estatua era de la
diosa de la Tierra, una regordeta joven que radiaba confort y solicitud
incluso en su forma pétrea. El Doctor sacó el tapón del frasco, alargó
una mano por la boca de la estatua. Una sola gota y volvió a ser humana
de nuevo. El pánico llenó sus ojos cuando el Doctor la bajó del plinto,
pero algo en su expresión debió de tranquilizarla. Puso un dedo en sus
labios y la cogió de la mano, cuando salieron juntos de los árboles.

--¿Qué hay, Gaya? --le susurró el Doctor al oído de la que fuera la
diosa de la Tierra cuando se acercaban al tronco de un árboles--. No te
preocupes. Todo va a ir bien--tuvo problemas para no reírse cuando vio
al guarda armado aún de espaldas al pedestal vacío--. Sígueme el juego.

--¡Ey! --gritó un guarda cuando el Doctor y la chica salieron de los
árboles y cruzaron la línea de guardas.

El Doctor le sonrió.

--Hola.

--¡No se permite a nadie estar ahí!

--No estamos ahí--señaló el Doctor, razonablemente--. Estamos fuera.

--¿Qué estabais haciendo ahí? Lo hemos registrado todo.

--Bueno--dijo el Doctor--, obviamente os habéis olvidado de nosotros. No
es difícil. No te culpo, seguro. Mi, eh, amiga y yo\ldots{}--el guarda
asintió, suponiendo--, debimos caernos dormidos por el sol del
atardecer. Aún así, ya estamos del todo despiertos, así que si nos
permitís\ldots{}

--¡Ey! --un gritó vino del interior del jardín--. ¡La estatua ha
desaparecido!

El Doctor de repente se vio rodeado de guardas. Puso su mejor expresión
de sorpresa.

--¿Dónde está? --preguntó un guarda.

--¿Dónde está qué? --preguntó el Doctor.

--¡La estatua! ¡Debes haberla robado, no se puede estar dentro!

El Doctor levantó los brazos.

--¡Por favor, cacheadme! --dijo--. Si creéis que tengo una estatua
escondida en mi túnica\ldots{}

--Entonces ya la has sacado.

--Claro, ¿acabo de pasar a vuestro lado con una gran y brillante estatua
y entonces he vuelto, sólo por diversión?

Los guardas se miraron los unos a los otros, sorprendidos pero
reticentes a rendirse su única esperanza de su ignominioso error. de
repente el hombre que había estado vigilando la estatua dijo:

--Ey--dijo, señalando a la acompañante del Doctor--,se parece un poco a
estatua. Incluso esa es la ropa que llevaba.

El Doctor se golpeó la frente.

--¡Por supuesto! Eso es. Me has pillado del todo. Lo que he hecho en
realidad ha sido escabullirme aquí dentro, transformar la estatua en
esta joven mujer y entonces intentar salir casualmente a su lado. ¡Qué
tonto he sido al pensar que ustedes estaríais vigilándonos! Quería salir
en paz, ¿no es así?

En ese momento, Gracilis se acercó.

--¿Hay algún problema, Doctor? --dijo--. Soy Gnaeus Fabius Gracilis--les
dijo a los guardas, con aires de superioridad--, y este hombre es mi
amigo. ¿Debo preguntar por qué le estáis importunando?

Murmurando con sospechas, los guardas dejaron pasar al Doctor y a
``Gaya''.

--Gracias--dijo el Doctor, acariciándole la espalda a Gracilis--. Te
dije que podíamos hacerlo. Ahora, salgamos de aquí.

Gracilis arregló un transporte para los esclavos, y entonces él y el
Doctor fueron a por su propio medio de transporte.

--¡Volver a tener a mi hijo, al fin! --gritó Gracilis, feliz. El Doctor
era menos feliz. Tenía mucha confianza e su habilidad de encontrar a
Rose, pero eso no cambiaba el hecho de que no tuviera, en ese instante,
ni idea de dónde estaba. Un pensamiento de repente se le ocurrió.

--Y podemos comprobar a ver qué sabe Vanessa--dijo--. ¿No ha habido
ninguna palabra de su parte, no? Espero que esté bien.

Pero en ese momento en todo lo que podía pensar Gracilis era en su hijo
que salvar.

Mientras caminaban hacia las puertas de la ciudad, el Doctor vio una
calle de aspecto familiar.

--Espera un segundo--dijo--. ¿Supongo que habrá espacio en ese excelente
carro tuyo para una más que estilosa cabina azul?

Era por la tarde del día siguiente. El carruaje que llevaba al Doctor y
a Gracilis se acercó a la villa, seguido de cerca por un carro alquilado
llevando la sólida silueta azul de la TARDIS (el Doctor tenía razón, no
había habido espacio en el carro).

Gracilis ordenó que el carruaje se detuviera antes de llegar a la villa.

--Quiero devolver Optatus a mi mujer--dijo--. No quiero que vea su
restauración. Me temo que el conocimiento de lo que de verdad ha
sucedido pueda inquietar su mente.

Caminaron hasta el huerto y Gracilis observó la estatua, sin hablar.
Quizá, había llegado el momento, estaba demasiado asustado para
apresurarse, sabiendo que sus esperanzas podrían destrozarse. Pero
finalmente asintió al Doctor.

El Doctor dio un paso adelante y dejó caer una de las pocas restantes
preciosas gotas de líquido verde caer en la piedra.

Ni siquiera el Doctor respiró durante el segundo en el que esperaron,
microsegundos siendo horas.

Entonces sucedió. Se extendió la carne, parpadeó un ojo, un brazo bajó
de su noble pose.

Y entonces Optatus estaba en los brazos de su padre y ambos lloraban.

El Doctor miró desde la distancia mientras Optatus se reunía con su
madre. Sus lágrimas caían libremente, pero no podía dejar de sonreír.
Finalmente, después de que todo el mundo se hubo calmado, se acercó. No
pudo evitar que Marcia volviera a llorar, abrazándole y agradeciéndole
tantas veces que las palabras comenzaron a sonar en sus oídos como un
cántico sin sentido, pero finalmente fue capaz de preguntarle lo que
quería. ¿Había visto a Rose? ¿O a Ursus? ¿Había habido algún mensaje de
Vanessa?

La respuesta a todas las preguntas fue ``No''.

El Doctor se escabulló de las celebraciones. Desesperado no era el
sentimiento que querría admitir que sentía, pero justo en ese momento,
el mundo romano se alzaba ante él de una forma imposiblemente enorme y
descorazonadora. Rose era una pequeña aguja de mármol en un gigante
granero romano. ¿Cómo iba a encontrarla?

Y entonces se le ocurrió. Una verdadera maravilla de pensamiento, un
pensamiento de los que te golpean entre los ojos.

Sabía exactamente dónde podría encontrar a Rose.
