\chapter*{Capítulo Once}
\addcontentsline{toc}{chapter}{Capítulo Once}

El azul brillante de la TARDIS chillaba incongruentemente en medio de
las clínicas paredes blancas y el anémico mármol de la sala de
escultura. De cualquier forma, el Doctor, aún vistiendo su túnica
romana, se mezcló entre las exposiciones de una manera que ningún otro
visitante podía hacer, y aún así era el que recibía las miradas
extrañadas de los turistas con cámara en mano que vestían camisetas con
eslóganes y los polvorientos académicos con chaquetas de tweed.

Por lo que le importaba al Doctor, aquellas personas ni siquiera
existían, ni siquiera los niños acercándose a la TARDIS, suponiendo que
era algún tipo de exposición interactiva, merecían una rápida mirada.
Era un hombre con una misión y no iba a ser distraído.

Pero cuando llegó a la estatua de Rose, algo le distrajo.

Posado sobre el gran dedo de un gigantesco pie gigante, en una descarada
desconsideración por los carteles prohibiendo a todo el mundo tocar las
exposiciones, había una silueta familiar. Mickey Smith.

--¡Doctor! --dijo Mickey cuando el Doctor se acercó.

Miró por encima del hombro del Doctor, pero éste iba solo.

--Oh. Hola.

El Doctor ralentizó su paso acelerado.

--Hola--respondió--. Esto es\ldots{} ¿antes o después de la última vez?

Mickey se encogió de hombros.

--¿Cómo sé cuál ha sido la última vez para ti? La última vez para mí fue
hace una noche, cuando tú y Rose os fuiste para convertirla en el centro
del mundo del arte.

El Doctor parpadeó.

--Y supongo que llegasteis bien--siguió Mickey, mirando al Doctor de
arriba a abajo--, ¿o es que las faldas masculinas se han puesto de moda?

El Doctor le ignoró, concentrándose en la estatua ante él.

La joven belleza de Rose estaba capturada para siempre. Incluso
petrificada, la fuera brillaba en su cara. Nadie podría mirarla y no
darse cuenta de lo especial que era. Inconscientemente alargó una mano
para sujetar la suya. Pero por supuesto, no estaba allí.

De repente, una oleada de duda le recorrió.

Mickey se levantó y se puso detrás de él.

--Sabes, quienquiera que hiciera esto debió de conocerla bien--dijo--.
Es como\ldots{} como si la entendiera de verdad--se detuvo, y entonces
tuvo un pensamiento repentino--. Ey, ¿no estaría, ya sabes, viéndose con
este tipo o algo?

El Doctor rió secamente, de forma inhumana, y Mickey dio un paso atrás.

--¡Guau! No quería ofenderte, tío.

--Esta no es una estatua de Rose--dijo el Doctor.

Mickey parecía confuso.

--¿De qué estás hablando? Por supuesto que sí. ¿Crees que no reconocería
a Rose cuando la viera?

--Por supuesto que no--dijo el Doctor--. Porque la estás mirando ahora
mismo. Esta no es una estatua de Rose. Es Rose misma. Rose se ha
convertido en piedra.

Mickey tuvo que sentarse en el pie y se cogió la cabeza con las manos.

--No es cirto--decía, con sus palabras bien altas y camufladas, con su
cuerpo temblando con los sollozos que intentaba reprimir, intentando
evitarlos.

Un oficial sin uniforme se les acercó.

--Disculpe, señor--le dijo a Mickey, al parecer ignorando las
lágrimas--. Me temo que tengo que pedirle qu no se siente en las
exposiciones.

Mickey le ignoró, y probablemente ni le escuchó.

--Está un poco molesto ahora mismo--señaló el Doctor.

El hombre no se inmutó.

--Lo siento, señor, no puedo hacer una excepción.

El Doctor se acercó y le picó.

--Lo siento, sólo estaba comprobando si eras humano de verdad. Porque un
verdadero humano vería lo molesto que está\ldots{} mi amigo y mostraría
un poco de compasión.

El guardia ignoró la furia del Doctor. Probablemente estaría
acostumbrado a ese tipo de cosas, incluso en un lugar tan refinado como
un museo. Hablaba tan razonablemente que el Doctor, no en el mejor de
los humores justo en ese momento, sintió que su ira crecía aún más.

--Tenemos un deber para proteger estos objetos. No habrían durado
durante incontables generaciones si a todo el mundo se le permitiera
sentarse en ellos, ¿no es así?

El Doctor estaba a punto de lanzar un número de contraargumentos que
conllevaban usos pasados de trabajos de piedra, igual que los usos que
se le habían ocurrido y que involucraban al guarda, todos los cuales
quizá probablemente le habrían hecho enloquecer, cuando Mickey se
levantó de golpe. Giró la cara hacia el guarda de seguridad.

--¡No me importan tus estúpidas estatuas o tu estúpido deber! --gritó.

Todo el mundo en la sala se giró para observar. Un turista hizo una
fotografía.

--¡Está muerta! ¿No lo entiendes? ¡Está muerta! He estado viniendo aquí
cada día, cada estúpido día, para sentir que estaba cerca de ella, para
poder sobrellevarlo hasta que la pudiera volver a ver. Pero no
sabía\ldots{} ahora no la voy a volver a ver nunca, nunca jamás!

--Lo siento, señor, pero\ldots{}

El Doctor se metió antes de que la situación empeorara.

--Como te he dicho, está un poco molesto--dijo severamente, y cogió a
Mickey por el brazo.

Con unas lágrimas silenciosas aún cayendo por las mejillas de Mickey, el
Doctor le llevó a la sala contigua y hacia las escaleras. Parecía
confuso y enfadado.

El Doctor le hizo sentarse en una mesa en el restaurante del Museo y fue
a una barra cercana. Volvió con dos tazas de plástico de refresco de
grosella negra y puso uno delante de Mickey, colocando una caña en lo
alto.

Estuvieron sentados en silencio unos pocos minutos. Ninguno de ellos
estaba de hecho allí, estaban en el pasado, con Rose. Hablando con ella.
Riendo con ella. Mirándole a la cara.

--Siempre fue demasiado buen para mí--dijo de repente Mickey--. No me la
merecía, vaya que no. Hubo una vez que yo tuve la gripe, y ella me
cuidó, cada día. Me sentía como si me quisiera morir, entonces ella me
cogió de la mano y recordaba cómo podía ser de buena la vida--casi
sonrió--. Creía que era el hombre más afortunado del planeta por
tenerla. Éramos solo niños, pero sabía que era especial. No dejé de
pensar que me dejaría. Y así lo hizo, llegado el momento. Volvió, aún
así. Creía que se habría recuperado, que vería la verdad en un par de
semanas. Pero no fue así. Nunca creí estar con ella una segunda vez.
Sabía que había algo mejor para ella y que se acabaría dando cuenta al
final. Sólo tenía que aprovechar cada día con ella. Quiero decir, me
cabreé cuando se fue contigo. Me cabreé contigo, pero me cabreé con ella
también, me cabreé que hubiera visto a través de mí al final. Porque
ella se merecía algo mejor que yo. Se merecía alguien que le pudiera dar
el universo entero--el dolor en su voz se convirtió en enfado--. Pero tú
la has matado.

--Lo sé--dijo el Doctor, y fue como si se odiara a sí mismo.

--¡La has matado y yo no la volveré a ver! ¡Ella pensó que querría
peligro y emoción, pero podrías haberla detenido! Ella no era, una
Señora del Tiempo, era sólo una chica normal y tú la has matado.

--Rose no era ``normal'' --dijo el Doctor. Dejó de sonar enfadado
consigo mismo, y ahora se dirigía a Mickey--. ¿Qué se suponía que tenía
que hacer? ¿Envolverla en un montón de algodón? ¿Decirle, ``Mira, te
podría dar el universo, pero no lo voy a hacer por si te haces daño''?
¿''Hay todas esas cosas ahí fuera, todos esos planetas, todas esas
maravillas, pero quiero que te quedes en casa y trabajes en una
tienda''?

Mickey se levantó y gritó.

--¡Yo habría cuidado de ella!

El Doctor le devolvió el grito:

--¡Lo sé!

Mickey se volvió a sentar.

--Deberías haberlo hecho--repitió en silencio. De repente se
estremeció--. ¿Cómo se lo voy a decir a su madre. Ella me crucificará.

--Creo que quieres decir que lo hará conmigo--dijo el Doctor. Y se rió a
medias--. Lo divertido es que casi me crucifican esta mañana. Por
suerte, en lugar de eso, me tiraron a los leones.

Mickey se concentró en lo primero que había dicho, demasiado envuelto en
lo que estaba pasando en ese momento como para que le importaran lo más
mínimo las aventuras del Doctor.

--¡Cómo que te vas a quedar por aquí! ¡Y Jackie tendrá que pagarlo con
alguien! ¡Ya no tiene a nadie más! --su cara se derrumbó--. ¡Y ni
siquiera tiene una tumba!

El Doctor se calló durante unos minutos, dejando que las lágrimas de
Mickey hicieran su recorrido. Entonces dijo, casi vacilante.

--Puedo traerla de vuelta.

Mickey levantó la mirada, sorprendido.

--¿Que puedes qué?

El Doctor habló de nuevo, más seguro esta vez.

--Puedo traerla de vuelta.

Mickey se puso en pie, casi más enfadado que antes.

--Tú\ldots{}¿tú puedes? Bueno, ¿por qué no lo has dicho antes? ¿Esto era
divertido para ti, verme así? El idiota de Mickey, no entiende de estas
cosas, ¡vamos a reírnos de él! --parecía que fuera a pegarle un puñetazo
al Doctor, que se levantó rápidamente.

--No\ldots{} estaba seguro de si era lo correcto--hizo un ademán,
acallando la siguiente protesta de Mickey--. Pero ahora lo estoy. ¿No
importa que pueda hacer lo más importante aquí?

Mickey pareció querer discutírselo, pero entonces asintió.

--Sí. Claro. Bueno, ¿a qué estamos esperando, entonces? --comenzó a
caminar.

--Para que ese guarda se vaya, por ejemplo--le gritó el Doctor
colocándose tras de él.

Mickey se detuvo y se volvió a sentar.

Ambos se callaron durante unos momentos. El Doctor tomó un largo sorbo
del refresco de grosella negra.

Entonces Mickey dijo, un tanto nervioso.

--Pero\ldots{} ¿no tendrá como unos 2000 años o algo?

--Casi unos 1900, tomes o no ese raro cambio en el calendario--respondió
el Doctor--. Pero\ldots{} eso no debería importar. No es consciente
donde está. No ha envejecido.

--¿Y estás seguro de que no es consciente? --preguntó Mickey--. ¿Estás
seguro de que no ha estado viendo todo lo que ha pasado?

El Doctor levantó una ceja.

--Bueno, si es así, debe de haberte visto cada día durante las últimas
noches. Eso debería hacerte ganar unos cuantos puntos Brownie--quería
haberlo dicho muy amablemente, pero Mickey parecía como si un cachorro
acabara de recibir un patada. Suspiró --Adelante--dijo, levantándose--.
A ver si ya no hay moros en la costa.

Pero Mickey se quedó sentado.

--Rose puede que no haya envejecido, no la Rose de dentro. Pero la
estatua sí. Tiene trozos rotos. Tiene manos rotas. ¿Eso también volverá
cuando la devuelvas a la vida?

El Doctor no respondió.

--¿No, verdad? --dijo Mickey, furioso--. No crecerá mágicamente como
hizo la tuya. ¡Vas a traerla de vuelta con trozos de menos y una mano
ausente!

El Doctor golpeó la mesa.

--¡Es mejor que no tener a Rose! --gritó.

Mickey parecía un tanto asustado. Pero tras unos segundos, asintió.

--Sí--dijo--. Supongo que sí.

Y se abrieron camino hasta la sala de las esculturas, y el Doctor oyó a
Mickey susurrar.

--Espero que ella esté de acuerdo.

Era el final del día y la gente comenzaba a salir del museo. Había un
par de turistas paseándose por entre una hilera de bustos en la sala de
esculturas, pero nadie estaba cerca de Rose.

El Doctor sujetó el pequeño rasco con sus pocas gotas restantes del
precioso líquido inspirador de vida. Con la mano firme, respiró hondo. Y
entonces su mano se giró y la poción cayó en la estatua.

Nada sucedió.

No se sonrojaron las mejillas. No se movieron las ropas o parpadearon
las pestañas.

El Doctor se la quedó mirando.

--¿Cuánto tarda en funcionar? --preguntó Mickey.

--No va a funcionar--dijo el Doctor, con tristeza--. Es demasiado tarde.
Debe de haber sido una estatua demasiado tiempo--se detuvo--. Ya ha
acabado.

Mickey no lo aceptaba.

--Eso es una tontería. Tienes una máquina del tiempo. Oh, ya sé eso de
las leyes del tiempo y esas cosas, pero no puedes evitar que pase, pero
puedes encontrarla antes. Cambiarla entonces.

El Doctor negó con la cabeza, frustrado y enfadado.

--¿No lo ves? Si la cambiara entonces, entonces esta--señaló hacia la
estatua--, ¡no estaría aquí ahora! Es por eso por lo que no la pude
encontrar en Roma. Nunca se suponía que tenía que encontrarla. ¡No hay
nada que pueda hacer--bajó el brazo, con la mano rozando la cara de
Rose.

Miró de nuevo a la estatua y entonces enloqueció.

Mickey vio alarmado cómo el Doctor corría hacia la TARDIS, gritando.

--¡Oh, por favor! ¡Oh, por favor! --gritaba a pleno pulmón. Un momento
más tarde, salió de la nave, llevando la chaqueta tejana de Rose.

El Doctor le dio la vuelta y la sacudió.

Cayeron un monedero, un pañuelo, una cajita de caramelos de menta, un
teléfono móvil y un pendiente.

Recogió el pendiente y lo colocó frente a la estatua.

--Es el mismo--dijo Mickey.

--Se olvidó de ponérselo otra vez--dijo el Doctor--. Así que Rose, la
verdadera Rose, sólo lleva puesto un pendiente. Pero la estatua tiene
dos. Eso significa que\ldots{}--dejó que tomara cuerpo--\ldots{} esta no
es Rose. Es sólo una estatua-- de repente se animó--. Tengo que volver y
encontrarla--miró al frasco de cristal. Había el más mínimo resto de
líquido aún en el fondo--. Esto tiene que ser suficiente\ldots{}

La cara de Mickey brillaba de alivio. Pero un pensamiento se le ocurrió.

--Espera un momento. ¿Entonces cómo ha llegado esta estatua hasta aquí?

El Doctor sonrió.

--Tengo una idea sobre ello. ¿Crees en los dioses?

Mickey hizo una mueca.

--No.

--Bueno, en este mismo momento, yo sí--dijo el Doctor--. Creo que
Fortuna nos está sonriendo. Vamos. Te necesito que me eches una mano
antes\ldots{}

Acabaron justo cuando el guarda de seguridad irrumpió en la sala.

--Es hora de hacer una retirada rápida, creo--dijo el Doctor. Empujó a
Mickey a las escaleras y fue corriendo hacia la TARDIS.

--¿La traerás de vuelta, no? --le gritó Mickey.

--¡Te apuesto lo que quieras! --gritó el Doctor. Pero justo cuando se
cerraron las puertas de la TARDIS detrás de él, se murmuró para sí--.
Aunque antes tengo que hacer una paradita\ldots{}

Y entonces, un poco más tarde, el Doctor llegó a Roma un tiempo antes.
