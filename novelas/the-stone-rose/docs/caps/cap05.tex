\chapter*{Capítulo Cinco}
\addcontentsline{toc}{chapter}{Capítulo Cinco}

Rose fue arrancada de un sueño sobre gatos parlantes la próxima mañana
cuando Vanessa le zarandeaba el hombro.

--Es hora de despertarse--dijo la chica, mientras Rose bostezaba e
intentaba recordar dónde estaba. Le llevó un par de minutos a salir de
la cama, bostezando todo el rato.

--¿Crees que Ursus será capaz de capturar las ojeras de mis ojos a la
perfección? --dijo mientras se miraba a sí misma en el circulo de bronce
pulido que usaba de espejo--. ¿Qué hora es?

--Dos horas tras el amanecer--le dijo Vanessa--. Tienes una hora antes
de ir al estudio.

--Pues ya podría Ursus pensar un poco más en dejarme tener mi sueño
revitalizante--gruñó Rose, pero comenzó a prepararse de todas formas.

Vanessa le ayudó a hacerse el pelo, lo que les llevó la mayor parte del
tiempo a su disposición. Finalmente Rose estuvo lista para irse.

--Mira, ¿por qué no vienes conmigo? --le sugirió a Vanessa--. No digo
que vaya a ser divertido, pero no me importaría la compañía. Apartarte
de todos los demás al mismo tiempo. Quiero decir, Ursus tiene un esclavo
allí dentro, no puede realmente objetarme por traer yo una, diga lo que
diga. Sin embargo, con la forma por la que se trata a los esclavos como
al mobiliario por aquí, probablemente ni siquiera se dé cuenta de que
estás.

--Sí, me gustaría venir.

Cruzaron el patio, yendo hacia el taller cerca de los establos. El
Doctor ya estaba despierto y le saludaron al pasar.

La puerta del estudio estaba cerrada así que Rose le pegó un puñetazo.

--Recuerda, abre tus ojos y orejas--le susurró a Vanessa mientras
esperaban--. Ya sabes, si es que es de los malos.

Vanessa iba a responder, pero justo entonces la puerta del estudio se
abrió de golpe por Ursus. Hizo una mueca al verla, quizá no el tipo de
saludo que alguien como Kate Moss tendría, pero podría aceptarlo. Dio un
paso al interior, con Vanessa bien cerca.

--Sácala de aquí--gruñó Ursus, apuntando con la cabeza hacia la
esclava--. Os lo he dicho a ti y a tu amigo doctor muchas veces: no
permito público mientras trabajo, ni siquiera esclavos.

Rose puso tono imperioso.

--Entonces, ¿quién va a cuidar de mí? ¿Qué pasa si necesito que me
arreglen el pelo, o darme una bebida o algo?

Ursus se acercó a la mesa y sacó una taza. Sirvió un poco de vino en un
cáliz y se lo dio a Rose.

--Aquí tienes. Tu bebida. Tu pelo está bien. ¡Ahora que se vaya!

Aquella fue la orden de salida para Vanessa, que salió rápidamente por
la puerta. Rose estuvo medio tentada de seguirla. Pero aquello era parte
de su destino, ¿no era así? Tenía que posar para aquella estatua. Y el
Doctor también confiaba en ella para investigar a Ursus, en caso de que
supiera algo de la desaparición de Optatus. Se tragó el agrio vino
(seguía sin poder aguantar el sabor), pero le podría ayudar a relajarse
un poco.

Ursus fue hacia la sala contigua y Rose le siguió. Aquel día no estaba
Tiro, lo que le dio a Rose una ligera sensación de decepción.

--Así que, ¿cómo quieres que me ponga? --preguntó ella, pero Ursus la
ignoró. Se acercó a una mesa y comenzó a remover las herramientas.

Rose se sentó en un banco, esperando instrucciones. Miró alrededor de la
sala, había una cosa distinta de cómo había estado el día anterior, una
alta silueta cubierta en una esquina. ¿Un trabajo en progreso? Esperaba
que Ursus no intentar abrazar más de lo que podía hacer porque quería
acabar con aquello y esperaba de verdad que no se alargase demasiado.
Tenía el terrible presentimiento, sin embargo, que se así sería. Días,
quizá semanas. ¿Merecía la pena, aunque fuera por un tipo de
inmortalidad?

Ella sonrió ligeramente. Ya tenía algo parecido a la inmortalidad, de
una forma cíclica. Incluso si moría, allí y entonces, lo que,
obviamente, no estaba planeando hacer por el momento, justo en unos
2.000 años después, estaría en una estación espacial, venciendo a los
Daleks. Y muchos más años de los que podría comprender, estaría
observando morir a la Tierra.

Pero aunque aquello era el futuro, era su pasado.

Y justo entonces debería estar concentrándose en su presente. ¿Qué le
estaba ocurriendo?

De repente se despertó a sí misma. No había estado precisamente dormida,
pero se había quedado tan ensimismada en sus propios pensamientos que
había comenzado a divagar. ¡Aquel vino debía de haberla relajado
demasiado! Recordaba aquel rato que compartió con Shareen, cuando ambas
eran adolescentes, cuando planearon escaquearse de la fiesta de Danny
Fennel y se habían tomado medio sorbo de una botella de vino del armario
de la cocina de su madre para emborracharse. Con la excepción de que
después de un vaso cada una, ambas se habían quedado dormidas, y no sólo
se perdieron la fiesta sino que también les echaron la bronca del siglo
de parte de Jackie.

Rose parpadeó. Ya no tenía diez años, y se había tomado más que un vaso
de Lambrusco para hacerle pegar cabezadas en aquellos momentos. ¿Y por
qué de repente se sentía tan somnolienta?

Se forzó a despertarse y miró a Ursus.

Estaba sentado en un banco mirándola, y su expresión le hizo sentir
vacía por dentro. De repente, era la hora del \emph{Silencio de los
Corderos}. ¿Cómo podía haber sido tan estúpida? Habían tenido la
sospecha de que era un chalado, y aún así había ido alegremente hacia su
trampa para moscas por una estúpida idea de tener que evitar la paradoja
de lo que ocurriría si nunca se hacía la estatua.

Ursus se levantó y se acercó. Le quitó el vaso de vino de su mano sumisa
y Rose se dio cuenta de la verdad mientras él se la ofrecía en bandeja
delante de sus narices. Le había drogado el vino. Y ella se lo había
puesto tan fácil. De repente sintió un terrible pánico en su interior.

Ursus bajó el vaso y agarró una lanza del montón de objetos divinos en
el rincón. ¿Era todo aquello? ¿Iba a clavársela hasta matarla? Rose hizo
el desesperado esfuerzo de moverse, pero sus miembros no respondían.

Pero él no se la clavó. En vez de eso, abrió su mano y puso la lanza en
su interior. ¿Qué?

Rose intentó tomar ventaja de aquella sorprendente situación: ¡su captor
le había dado un arma! Intentó de nuevo moverse, intentar apuntar la
lanza hacia la fea y presuntuosa cara.

Pero de nuevo, falló.

Usus la empujó del banco. Intentó resistirse pero no pudo. Ahora él
movía sus inútiles miembros, manipulándola como si fuera el peluche de
una tienda de Henrik.

A parte de estar aterrorizada, era totalmente humillante. Rose Tyler, la
muñeca de Barbie.

¿Había una Barbie guerrera? Porque ahora, añadiendo confusión al horror
y la humillación, Ursus había cogido un casco metálico del montón y se
lo colocaba con cuidado en la cabeza.

Aquello no estaba bien. Su estatua no llevaba un casco, no había
sujetado una lanza. ¿Qué estaba pasando?

Ursus finalmente dejó de tratarla como si estuviera hecha de plastilina.
Rose no podía verse a sí misma, pero sentía que su cabeza se sujetaba en
sus hombros, su lanza agarrada heroicamente en un a mano mientras estaba
de pie, orgullosa. Y\ldots{} ¿qué iba a pasar a continuación?

El escultor dio un paso atrás, admirándola. Era una mirada cínica e
impersonal; no había nada en ella que dijera que fuera un ser humano.
Para él no era nada más que arcilla que pudiera moldearse en una forma.

Intentó hablar de nuevo. Su miedo, su desesperación, debieron de haberle
dado su fuerza, porque el más ligero de los ligeros sonidos salió, algo
que debió de haber sido reconocido como un ``Noooo''.

Ursus frunció el ceño.

--No hagas eso--dijo.

Aquello era bueno, ¿no? Al menos le habló. Cualquier cosa que hiciera
que pareciera una persona, importaba.

De repente le dio la espalda a Rose, y sintió un golpe de esperanza.
Había cambiado de idea, no iba a hacerle nada\ldots{}

Pero se acercó a la silueta del rincón. Agarrando la sábana, la arrancó,
revelando lo que había debajo.

Era una estatua, tal y como Rose había sospechado. Un hombre con alas en
su sombrero y en sus zapatos, como el logo de Interflora. Pero había
algo familiar en él\ldots{} El pelo rizado, los rasgos atractivos, ¿era
Tiro? Pero él había dicho que ni siquiera había comenzado a modelarse.

Ursus sonrió a la estatua.

--No me puso las cosas difíciles--dijo--. Sabía que la belleza es más
importante que la vida.

El estómago de Rose pareció desvanecerse en su interior. Aquello no
podía estar pasándole a ella. Era un sueño, uno de esos donde tus
piernas no te obedecen, dónde no puedes correr, por mucho que lo
intentes. Los gatos parlantes habían sido reales y todo desde entonces
había sido un sueño: una pesadilla.

Pero no parecía como si la pesadilla fuera a terminar.

El Doctor se había pasado la mañana haciendo de Sherlock Holmes y no es
que esperase que hubiera algo más que descubrir por allí. Rose, esperó
él, estaría usando sus propios instintos de detective para descubrir
cosas de Ursus, mientras él estaba poniendo punto y aparte en su
búsqueda por la zona. Gracilis le había sugerido torturar a los esclavos
para asegurarse de que le dijeran la verdad, pero el Doctor se las había
apañado para persuadirle de que no lo hiciera.

Con cuánta más gente hablaba, más estaba convencido de que sólo Ursus
podría darle respuestas. Después de compartir una comida de pan y queso
con Gracilis, el Doctor decidió que necesitaba ir a ver si Rose había
descubierto algo. Llevándose un poco de comida como excusa, después de
todo, seguro que incluso a la modelo de un artista se le permitiría
comer, fue directo al taller de Ursus.

Mientras el Doctor se acercaba al establo, un carro se alejaba. En la
parte trasera de él podía ver un gran objeto envuelto colocado en un
lecho de paja. Sin siquiera ignorando a la más ligera de las sospechas,
corrió detrás del carro y saltó en la parte trasera antes de que se
hubiera alejado cien yardas.

--¡Ey! --gritó el carretero.

--¡No me hagas caso! --le gritó el Doctor--. Sólo quiero echar un
vistazo rápido a lo que llevas aquí atrás-- y comenzó a desenvolver el
objeto, un objeto de tamaño humano.

El carretero frenó y se bajó.

--¿Qué crees que estás haciendo? Me dijeron que viniera aquí, que
recogiera unos objetos y los repartiera--dijo, yendo hacia la parte
trasera--. Si llega dañada, seré yo quien cargue con la culpa.

--Eso asume de que voy a dañarla--dijo el Doctor--. Y no tengo la más
mínima intención de dañar esta\ldots{} ¡ajá! Esta encantadora estatua de
Mercurio, mensajero de los dioses, con su sombrero alado que no es para
nada del otro mundo.

Volvió a envolver la escultura, le dio un golpecito en su cubierta
cabeza y se bajó de la carreta, justo cuando el cartero se acababa de
subir.

--Y bien, no te voy a entretener mucho más--dijo el Doctor--. Estoy
seguro de que es un hombre ocupado, por supuesto que lo es, y las
estatuas no se reparten solas, ¿verdad? --hizo un gracioso ademán en la
dirección de la carretera y el carretero sonrió, a pesar de lo que
acababa de pasar.

Pero justo cuando el Doctor volvió a bajar, él no sonreía. Había
reconocido la estatua. Sí, era Mercurio, pero estaba modelado a partir
de un hombre, y ese hombre era el esclavo Tiro.

Pero no había manera en la que Ursus pudiera haber ya completado la
estatua de Tiro. No había forma posible en la Tierra.

El más terrible de los pensamientos golpeó al Doctor. Un pensamiento que
explicaba que Ursus fuera capaz de completar las estatuas con tanta
rapidez. Y la razón por la que eran tan realistas. Y la razón por la que
sus herramientas estaban sin usar y su taller impecable de polvo de
mármol.

El Doctor comenzó a correr.

Ursus dio un paso atrás alejándose de la congelada Rose. Con las puntas
enguantadas en una mano, agarró las puntas enguantas de la otra. Y las
empujó. Lentamente, el guante salió y lo dejó caer al suelo, un
striptease terrorífico y repugnante. Entonces agarró las puntas de su
guante restante con los dientes y también se lo quitó.

--Toda mi vida, todo lo que he querido hacer era crear belleza. Pero los
dioses me maldijeron con estas\ldots{}

Levantó su gigantesca mano de dedos regordetes. Eran blancas y fofas, no
las que tendría un artesano llena de callos.

--Fui perseguido y se mofaron de mí durante años, pero no me rendí. Hice
votos a los dioses, les prometí sacrificios si ella me daba lo que
deseaba. Y entonces, un día, ella me lo dio. Le pedí lo que más deseaba:
la habilidad de crear belleza en la piedra. Y ella me concedió el deseo.

Lentamente, oh, demasiado lentamente, se acercó a Rose, con las manos
levantadas.

--Ahora tengo fama, renombre. Ya no me avergüenzo, ni lo haré. Mi madre
levanta la cabeza y habla con orgullo de ``mi hijo''. Tengo dinero,
dinero para comprar todo lo que quiero, dinero para vengarse de aquellos
que se mofaron de mí. Pero sobre todo\ldots{} tengo belleza, puedo crear
belleza.

Alargó una mano hacia Rose como si le fuera a acariciar la mejilla.

Tenía un recuerdo: un hombre descansando en una cama de hospital.
``Regresión \emph{pétrea-fold}'', lo había llamado el Doctor. ¿Había
pillado la Regresión \emph{Pétrea-fold}? ¿Se la había pasado Ursus? ¿Qué
le estaba pasando?

Lo último que vio era el cuerno de la abundancia, aún descansando sin
sujetar ni sin tener ninguna intención de ser usado en el rincón de la
habitación.

El Doctor frenó en seco ante los establos. Había otro carr allí, uno
vacío esta vez, preparado para recibir su carga. Estaba contemplando la
puerta cerrada cuando oyó un sonido al otro lado de ella: jadeos y
gruñidos junto con el chirriar de unas ruedas. La puerta se abrió de par
en par y allí estaba Ursus, arrastrando una carretilla con ruedas con
una figura de mármol en ella. la estatua descansaba en horizontal y el
Doctor no podía ver qué era, pero tenía una idea bastante acertada.

Se lanzó contra el escultor:

--¿Qué le has hecho a Rose?

Ursus era tan fuerte como indicaba su nombre, pero la furia del Doctor
le hizo ser competencia para cualquier hombre. Forcejearon, cayendo al
suelo y rodando uno encima del otro. Mientras el Doctor estaba tumbado
en el suelo, atisbó a Vanessa acercándose hacia ellos, con una lámpara
de bronce en sus manos. La levantó por encima de sus cabezas\ldots{}

--¡Eso es, Vanessa! --gritó el Doctor mientras giraba en seco,
poniéndose ahora encima\ldots{}

Y todo se volvió negro.

El Doctor se rascó su cabeza dolida y se incorporó. Estornudó con un
pedazo de heno pegado en su nariz. Un burro le miraba con curiosidad.
Estaba en un establo. Alguien le debía haber arrastrado allí. Miró a su
alrededor.

Agachada tras un pilar, temblando ligeramente, podía ver a Vanessa
observándole. Se alejó cuando él se puso en pie.

--Lo\ldots{} lo siento--gimió, sonando a un ratón más que nunca.

--Me has golpeado--dijo el Doctor, con frialdad y enfado--. ¡Has evitado
que salvara a Rose!

--¡Yo no quería! --casi estaba llorando--. ¡Te moviste! ¡Quería darle a
Ursus!

El Doctor entrecerró los ojos.

--Digamos que te creo\ldots{} por ahora. ¿Dónde está Rose? ¿A dónde se
la ha llevado?

--¡No vi a Rose! --tosió--. Sólo a una estatua.

El Doctor dejó pasar eso.

--Bueno, ¿dónde está ahora Ursus? --preguntó.

--No\ldots{} no lo sé--tartamudeó la chica.

--Creo que sí lo sabes--dijo el Doctor--. Creo que sabes más de lo que
nos haces creer, Vanessa. Como por ejemplo, ¿qué es el Muro de Adriano,
Vanessa?

Ella le miró a él, con los ojos abiertos y aterrorizada.

--¿Y bien? --dijo él.

A penas le salían las palabras de la boca.

--Es una muralla, que divide Inglaterra de Escocia.

El Doctor levantó una ceja.

--¿Un muro que aún no se ha construido, dividiendo dos lugares que no se
llamarán así en siglos?

Vanessa rompió a llorar.

--Ya sabes, tuve mis sospechas en el instante en el que nos
presentaron--siguió el Doctor--. ``Vanessa'', suena bastante romano,
debo admitir. Marcia, Claudia, Julia, Vanessa\ldots{} Pero resulta que
sé, porque soy extremadamente inteligente, que ese nombre se inventó en
el siglo XVIII por un tipo escritor llamado Jonathan Swift. Y en cambio,
allí estabas tú, una niña con un nombre del futuro, sentada en una mesa
haciendo el teorema de Merik. Oh, sé a lo que se parecen los cálculos
astrológicos, y sé cómo es el teorema de Merik, y eso era lo último, no
lo primero. Así que serías tan amable de decirme qué hace una chica de
al menos el siglo XXIV en el siglo II, y --se inclinó y le gritó--,
¡¿qué le ha pasado a Rose!?

Vanessa le miró a través de las lágrimas. Entonces, lentamente y
temblorosa, sacó un pequeño tubo negro de un pliegue de su túnica de
lana. Su dedo temblaba encima de un botón rojo en un extremo y señaló
con él directamente al Doctor.

--¡Déjame ir\ldots{} o disparo!