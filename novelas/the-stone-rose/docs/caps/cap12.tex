\chapter*{Capítulo Doce}
\addcontentsline{toc}{chapter}{Capítulo Doce}

Rose contuvo el aliento, como si alguien le hubiera lanzado un cubo de
agua fría por encima. Se removió al despertarse, confusa y mareada.

Había cerrado los ojos durante un segundo y cuando los abrió de nuevo
estaba en un lugar totalmente distinto. Aquel no era el taller de Ursus,
aquello era\ldots{} hojas. Podía ver hojas. Ramas. Árboles. Estaba en un
bosque. Y estaba de pie ante algo con ruedas. ¿Un coche? No. ¿Un
velocípedo? No. ¡Un carro de madera! Y justo delante de ella, algo alto
y delgado: una persona, y definitivamente no era Ursus. Una gran sonrisa
brillante se enfocó. ¡El Doctor!

Se tambaleó hacia adelante y le envolvió con un gran abrazo.

--¡Chico, no sabes cuánto me alegro de verte!

Éste gritó.

--¡Au!

Se había olvidado de que estaba sujetando una lanza.

--Lo siento--dijo, sonriendo. Se quitó el incómodo casco y zarandeó su
cabeza para aclararla.

Él levantó una ceja.

--¿Rose Tyler, la princesa guerrera?

--Sí--dijo--. Estaba pensando en pasarme por casa y sembrar el caos por
Colchester.

--Ah, sabía que no dirías que eres toda una Boadicea de
pacotilla--respondió el Doctor, y ella sonrió.

--Sí, pero ¿cómo ha pasado esto? --dijo--. Lo último que recuerdo
es\ldots{}-- y se calló.

El Doctor parecía un poco triste.

--Te convertiste en piedra--dijo--. Lo siento.

Las memorias volvieron de golpe.

--Lo supe--dijo--. Cuando me mostró la estatua de Tiro, lo supe--se
estremeció.

El Doctor sonrió con tristeza, mostrando solidaridad.

--No pienses en eso más. Y Tiro va a estar bien. Voy a rescatarle
en\ldots{} oh\ldots{} un día o dos--Rose arrugó una ceja y él se
explicó:

--Esa es la maravilla de los viajes en el tiempo. He llegado unos días
antes de que me fuera--le pasó el ahora vacío frasco--. Ahí tienes. Un
restaurador milagroso. ¡Acérquense, acérquense! ¿Siente que tiene la
cabeza llena de piedras? Una gota de nuestra poción se lo arreglará.
Señoras, ¿es que acaso su marido recibe todos sus comentarios de amor
con un silencio pétreo? ¡Dele nuestro remedio increíble y será un hombre
nuevo en un instante!

Rose sonrió.

--¿De dónde ha salido esto? ¿Y dónde está Ursus? ¿Te has encargado de
él?

--No lo sé, la verdad. En algún lugar por aquí y no aún.

--¡Estás perdido sin mí!

Le cogió del brazo.

--¿Crees que no lo sé? Si alguien me preguntara qué tipo de amiga eres,
les diría: ¿Rose Tyler? Estoy perdido sin ella. Es de piedra sólida, eso
es lo que es.

Rose le pegó un puñetazo en el brazo, pero acabó riéndose.

--Hay otra cosa que no entiendo, sin embargo--dijo cuando dejó de
reírse--. ¿Cómo encaja todo esto con la estatua del Museo Británico?
Quiero decir, ¡mírame!

Así hizo el Doctor.

--Casco, lanza, ¡oh cuán noble perfil! Minerva, a no ser que me
equivoque--dijo--. Y verás, ese disfraz iría perfecto para las fiestas.
O podrías ser un premio con forma de Minerva. A cualquier hombre de
sangre roja, sangre azul o incluso sangre verde le encantaría una
Minerva. ¡Y lo genial es que, si alguno se sobrepasa, tienes un arma a
mano!

Rose le detuvo.

--Sí pero, ¿quién es Minerva?

--¿Quién, dices? Algunos dicen\ldots{} --el Doctor vio la mirada de Rose
y decidió cortar su más que elaborada explicación--. La diosa de la
guerra como táctica y de las artes, patrona de artesanos.

--¿''Guerra y artes''? --dijo Rose--. ¿Algo como ``Di que te gusta mi
cuadro o invado tu país''? Da igual, lo que digo es que, ¿qué le ha
pasado a lo de Fortuna?

--Bueno\ldots{}--comenzó el Doctor.

No tuvo oportunidad de explicarse. Hubo un repentino grito de detrás de
ellos. Rose hizo un ademán de echar a correr, pero el Doctor la sujetó.

--Está bien--dijo--. Es sólo Vanessa.

La chica caminaba hacia ellos, con una mirada de shock total en su cara.
Miraba al Doctor como si fuera un fantasma.

--¿Cómo\ldots{} cómo has llegado aquí? --dijo--. No puedes haber llegado
antes que yo.

--Tú entre todas las personas deberías saber que cualquier cosa es
posible--dijo el Doctor--. Rose--siguió, girándose hacia ella--, déjame
presentarte a Vanessa, quien no es una astróloga ni una esclava romana,
sino una chica del año 2375.

--Cielo santo--dijo Rose--. Y yo pensaba que estaba lejos de casa--se
giró a Vanessa--. ¿Qué estás haciendo aquí, entonces?

--Intentando volver a casa--dijo Vanessa.

--¿Pero cómo has llegado aquí en primer lugar? --preguntó Rose.

--Sí, yo también estaría interesado en esa parte--dijo el Doctor--, pero
Vanessa no deja de evitar la pregunta. Parece tener algo que esconder.

Vanessa parecía como si se fuera a echar a llorar.

--¡No lo hago! Sólo\ldots{} que no me creeríais. De verdad que no.

--Venga, inténtalo--dijo Rose--. Seguro que no es peor que nosotros
vayamos pensando que tienes algo que ver con lo que está pasando aquí.

Vanessa se sonrojó.

--Pero\ldots{} creo que puede ser así.

Rose retrocedió.

--¡Le he estado diciendo a todo el mundo que eras de los buenos!

--¡Pero lo soy! Es sólo que\ldots{} Oh, vale. ¡Os voy a contar todo!
--gritó Vanessa, con tristeza.

Se apoyó contra un árbol. Rose y el Doctor se sentaron a su lado.

--Soy Vanessa Moretti, hija gamma de Salvatorio Moretti, del Bureau
Tygon.

--De repente todo está claro--dijo el Doctor, alucinando.

Rose le mandó callar.

--El Bureau Tygon es el principal parque de investigación
científica--siguió Vanessa.

Y entonces las orejas del Doctor se levantaron.

--¿Y tu padre estaba trabajando en los viajes en el tiempo? --dijo--. ¿Y
por qué nunca he oído\ldots{}?

--¡No! --Vanessa interrumpió su interrupción--. Trabajaba en la IA.

--¿En qué? ¿En esa película del chico de El sexto sentido? --dijo Rose.

--Inteligencia Artificial--le dijo el Doctor.

--Sí, lo sé--dijo, e hizo un ademán para que Vanessa siguiera.

--Nunca mencionó nada que ver con los viajes en el tiempo. Quiero decir,
no es posible, todo el mundo lo sabe--y sonrió con tristeza,
corrigiéndose a sí misma--. Todo el mundo pensaba que lo sabían. Mi
padre trabajaba en un proyecto, un proyecto de IA. No comenzó muy
entusiasmado, dijo que era solo un juguete, algo para sacar dinero. Pero
comenzó a emocionarse más. Creo que estaba a punto de acabarlo
cuando\ldots{} cuando me fui. Se trajo algo a casa del trabajo ese día,
no sé qué. Pero entonces se tuvo que ir. Yo estuve viendo un vidcast de
la Antigua Roma--rió con arrepentimiento--. Me solía encantar la
historia. Pero mi caster se apagó. Creí que iba algo mal con la
alimentación de energía porque las luces no dejaban de parpadear, pero
bajé a ver si el caster de mi padre seguía funcionando y así era. Había
una caja a un lado y pensé que contendría lo que fuera en lo que mi
padre estuviera trabajando, pero no miré para verlo. De cualquier
manera, me senté para ver el cast. Era sobre el reino de Adriano. Sobre
la construcción del Panteón y el muro y todo. Entonces recibí una
telellamada de mi amiga Ariane. Le estaba diciendo lo que estaba viendo
y dije, recuerdo esto, porque fue lo último que dije, que desearía haber
vivido en esa época--comenzó a reírse histéricamente--. ¿Os lo podéis
creer? ¿Podéis creeros que dijera eso? ¡Vivir en una época como esta!
¡Debí de haberme vuelto loca!

Rose alargó y le estrechó la mano, intentando calmarla.

--Sí, bueno, las cosas nunca van como te las imaginas, ¿no es así? Eso
es lo que hace que los abogados matrimoniales tengan trabajo.

--¿Y qué pasó después? --preguntó el Doctor, más preocupado sobre la
historia de Vanessa que por sus sentimientos.

Vanessa se tragó la histeria con un par de hipos. Después de un par de
intentos, se las arregló para seguir.

--Entonces\ldots{} entonces se cortó la llamada. El caster se apagó. Las
luces se apagaron. Me sentí como si me hubiera mareado\ldots{} y
entonces estaba aquí.

El Doctor no estuvo satisfecho.

--Tiene que haber más que eso.

--¡Bueno, no lo está! --insistió Vanessa--. No tengo ni idea de lo que
pasó. Creí\ldots{} creí que estaba soñando. O alucinando. O que alguien
me estaba gastando una broma muy elaborada. Pero después de un par de
semanas me rendí de esa idea.

--¿Y sigues sin saber cómo llegaste aquí? --dijo Rose.

Vanessa negó con la cabeza.

--Así que, ¿por qué crees que tienes algo que ver con lo que está
tramando Ursus?

--No hay tal cosa como los viajes en el tiempo de donde yo vengo y aún
así estoy aquí, miles de años antes de mi nacimiento. Es imposible
convertir a la gente en piedra y aún así está pasando aquí. Dos
imposibles\ldots{}

--\ldots{}no necesariamente hacen un posible--completó el Doctor--. Y de
cualquier manera, los viajes en el tiempo es perfectamente posible, mas
demasiado avanzado para tu sociedad.

Rose se encogió de hombros disculpándose a Vanessa.

--Pero convertir a la gente en piedra\ldots{}--dijo.

--Tampoco es imposible. Estamos hablando de algo extremamente complejo a
un nivel molecular, no de algo que un antiguo romano normal pudiera
haber conseguido, cierto, pero no imposible.

--Así que entonces no es magia--dijo Rose.

--No seas tonta, Rose--dijo el Doctor.

--¿Ni regresión \emph{pétreafold}?

Levantó una ceja.

--Cielos, ¿has estado prestando atención? No, eso tarda semanas.

--Así que\ldots{} ¿Ursus también es del siglo XXIV?

El Doctor negó con la cabeza.

--Gracilis le conoce desde que era un niño--de repente saltó--. ¡Ursus!
¿Dónde está?

Rose se encogió de hombros.

--¿Cómo podría saberlo?

--Te trajo aquí\ldots{}

--Bueno, no estaba prestando mucha atención en ese momento--dijo Rose--.
Estaba un poco ocupada quedándome muy quieta y teniendo palomas
haciéndose caca encima de mí.

El Doctor le hizo un señal para que se callara.

--Sí, sí, lo sé\ldots{} Pero no te ha traído aquí por ninguna razón, y
dejarte\ldots{} Y yo estaba justo detrás, así que le habría visto si
hubiera vuelto a la vía\ldots{}--comenzó a dar vueltas, examinando el
suelo--. ¡Pisadas! --gritó tras un momento--. ¡Vamos!

Rose se levantó para seguirle cuando comenzó a andar, con Vanessa
fregándole los tacones.

--No pudo llevar más allá el carro, es por eso por lo que te dejó
aquí--dijo el Doctor después de un rato.

--No me digas--dijo Rose--. ¿Qué crees que le detuvo?

Se paseaban por entre árboles, yendo por un camino que apenas estaba
allí, o no estaba allí. Intentó aclarar el camino con su lanza, pero los
arbustos seguían fregando sus ropas y su piel. El Doctor, sin hablar, se
las apañaba para esquivar todas ellas.

--No estoy bien vestida para esto--murmuró, pensando en unos tejanos y
unas botas altas--. ¡Ou! Gritó, cuando una rama golpeó en su
anteriormente elaborado peinado. Deseó haberse quedado con el casco de
Minerva--. Aún así, por el lado bueno, nadie va a pedirme que pose con
estas pintas. Ni que sea una copia de Minerva.

El Doctor siguió con su propio flujo de pensamientos.

--Probablemente estaba planeando volver a por ti.

--Si podía encontrar el camino de vuelta--dijo Rose--. Porque no estoy
segura de que yo pueda--parecieron seguir el rastro durante millas,
aunque probablemente no había sido tanto, pues lo único que podía ver
eran árboles y para ella eran todos iguales.

--¿Pero dónde estaba yendo? --preguntó Vanessa.

El Doctor, que no parecía estar demasiado fuera de lugar a pesar de un
pelo que tenía fuera de su sitio, se paró en seco.

--Ahí, creo.

Rose se asomó por un árbol. Había un claro delante de ellos, sólo uno
pequeño, pero lo bastante grande como para romper la espesura de
árboles. Rose no se había dado cuenta de lo oscuro que había sido encima
de los árboles hasta que la luz del sol la golpeó y le cegó los ojos. Y
al enfocarse, se dio cuenta de lo que había estado hablando el Doctor.
En el claro había un pequeño edificio de piedra, una ruina con agujeros
en las paredes.

--¿Qué es? --susurró ella--. ¿Algún tipo de altar?

El Doctor asintió.

--Eso creo--dijo--. Uno muy antiguo. Abandonado, obviamente, bueno, por
la mayor parte de la gente. Sabes, mañana es el Quinquatrus. Creo que
Ursus planea llevar su propio festival aquí.

Hubo un ruido del interior del altar: un ruido de pelea.

--Quizá no esté tan abandonado--dijo Rose.

Se arrastraron hacia allí, silenciosos como ratones, y se asomaron a
través de un agujero en la pared más cercana. Rose y Vanessa tuvieron
que contener grititos cuando el Doctor les dijo que se callaran con la
mirada.

Ursus estaba en el interior, pero también había una mujer. Estaba de
espaldas, así que no podían verla bien, pero Rose pudo ver que
claramente llevaba un casco y llegaba un escudo y una lanza. Le recordó
a la imagen de Britania en los anversos de las monedas de cincuenta
peniques, pero había algo más que eso\ldots{} Ursus había vestido a Rose
con un casco como aquel, le hizo llevar una lanza como aquella. Aquello
significaba que la mujer también estaba vestida como la diosa Minerva.
Rose hizo una mueca, obviamente tenía algo con la diosa de las artes y
la guerra, y se sintió inquieta al haber formado parte de ello.

Pero entonces la mujer se giró y esta vez Rose no pudo evitar contener
el aliento. ¡La luz! ¡La luz que brillaba de los ojos de la mujer! ¡El
hermoso y etéreo brillo de su cara! La forma en la que su pelo ondeaba
en su cabeza en un halo, como si estuviera bajo el agua.

Aquella mujer no estaba vestida como la diosa Minerva.

Sino que era la diosa Minerva.

Ursus hablaba:

--He creado la más perfecta obra para honrarla, en este momento para su
festival--dijo.

El Doctor le pegó un codazo a Rose.

--¡Esa eres tú! --susurró, bastante desconsideradamente.

Le devolvió el codazo y siguieron oyendo.

Minerva asintió.

--¡Y serás recompensado por tu devoción! --dijo ella, con una voz que
sonaba a miel y pétalos de rosas--. Mientras me hagas ofrendas, te daré
lo que deseas.

--Creo que no quiero ver esto\ldots{}--murmuró Rose.

Hubo otro sonido de dentro del templo, un balido asustado. Ursus llevaba
un corderito.

--¡Creo que no quiero ver esto! --dijo Rose, cuando el escultor sacó un
cuchillo--. ¡Ey! ¡Tú! ¡Detente! --estaba a medio camino del templo antes
de que el Doctor pudiera reaccionar. Ningún animal de granja mono iba a
ser asesinado en su guardia\ldots{}

El cuchillo colgó en el aire cuando éste cayó. Descendió cada vez
más\ldots{} Rose sintió como si estuviera a cámara lenta.

Y es que de hecho lo estaba. La diosa le miraba con aquellos ojos
brillantes sobrenaturales, y no podía correr.

A penas fue consciente de Doctor y Vanessa siguiéndola hacia el templo.

A penas fue consciente del doloroso y desgarrador balido final del
cordero cuando su sangre manchó el suelo, y del grito triunfante de
Ursus.

Pero todo lo que podía ver era la diosa, con la sangre encharcándose por
sus pies.

Y entonces, de una forma terrible, la sangre comenzó a desvanecerse,
como si la diosa fuera una esponja, absorbiéndolo. Y entonces el cuerpo
del cordero, que ya era pequeño, comenzó a encogerse. Su esencia se
estaba fundiendo, derritiéndose hacia donde había estado la sangre y
siendo absorbido en su turno hasta que no quedó ni un lanudo fragmento.
