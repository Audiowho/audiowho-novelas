\chapter*{Capítulo Dieciocho}
\addcontentsline{toc}{chapter}{Capítulo Dieciocho}

--Eso será el fin de la aventura, pues--le dijo Rose al Doctor, cuando
la TARDIS despegaba de nuevo--. Todo va a pasar cuando debía. El viejo
tú conseguirá el frasco de líquido y devolverá a todo el mundo y
entonces me dará el frasco vacío para dártelo a ti y todo funcionará.

--¡Gracias a los dioses! --dijo Vanessa.

Había intentado ayudar al Doctor a determinar el exacto momento y lugar
en el que debía volver. Se giró a Rose, como si se estuviera
despidiendo.

--Gracias, por todo.

--No pasa nada--dijo Rose--. Sólo ten cuidado con lo que deseas en el
futuro, ¿vale?

Vanessa sonrió.

La TARDIS aterrizó y Rose abrió las puertas. Vanessa se apresuró a
salir, ansiosa de estar en casa. el Doctor y Rose la siguieron más poco
a poco.

Llegaron a un pequeño estudio. La TARDIS se alzaba encima de una hermosa
alfombra persa y tapices de seda colgaban por las paredes. Una pantalla
mostraba un documental.

--Esta era la Edad de Oro de Roma\ldots{}--decía la voz en off.

--Entonces ya ha vuelto la luz--dijo el Doctor.

Miró hacia un lado. En un escritorio una difuminada marca cuadrada se
podía ver a contraluz provocada por el polvo acumulado, dónde debió de
haber estado una vez una caja de cartón.

--A Padre no le gusta ser molestado por los robots limpiadores--explicó
Vanessa, avergonzada.

--No le culpo--dijo el Doctor--. Y hablando de tu padre\ldots{} Su
investigación va a ser destruida y lo mismo pasará con su laboratorio.
Pero su cerebro funcionará aún. Hagas lo que hagas, no le dejes
construir otro GENIO. El destino del mundo, Vanessa. El. Destino. Del.
Mundo--hizo un ademán.

--Eh, sí--dijo Rose, sin estar segura de seguir con aquello--. Cuídate,
¿vale?

Volvieron a la TARDIS, dejando a una chica con una mirada muy preocupada
tras ellos.

Las puertas de la TARDIS se cerraron y volvieron a volar de nuevo.

--La pregunta ahora es--dijo el Doctor--, qué va a ser de ti --miraba al
GENIO--. Eres un poco peligroso, ¿lo sabes? Aun habiendo allanado el
terreno de baches, no te ofendas.

La criatura parecía preocupada, y el corazón de Rose de repente se
conmovió. Sí, había causado algunos problemas por los que preocuparse,
todo eso de haber convertido a gente en piedra, para comenzar. Pero no
había sido culpa del GENIO, como el Doctor había dicho, era la gente a
la que se tenía que culpar.

--Tengo una idea--dijo--.

--Soy todo oídos--dijo el Doctor.

Rose le dio un codazo entre las costillas.

--¡Ya no las tienes tan grandes!

--¿Su idea, señorita Tyler? --dijo, frunciendo el ceño.

--Vale. Bueno, estaba pensando en lo que dijiste tiempo atrás--le dijo.

--Si lo dije, debió de ser brillante. ¿Qué dije?

--Sobre los esclavos--dijo--. Y sobre cómo pueden comprar su libertad, o
ser liberados. Y el GENIO, bueno en la historias, ¿no es a veces llamado
el Esclavo de la Lámpara? Lo sé todo sobre Aladín. Bueno, al menos he
visto la película de Disney. La cual es brillante, por cierto. Robin
Williams, es tan divertido y\ldots{} vale, sí--añadió rápidamente
después de una mirada del Doctor--. El caso es que, el GENIO no puede
pedir deseos para sí mismo. Pero yo puedo desearlo por él, es cómo le
convertí en un mono. Y el último deseo de Aladín fue liberar al genio.
Hacerlo para que no tenga que conceder más deseos nunca más. Y así no es
un esclavo. Podría hacer eso.

El GENIO parecía asustado.

--¡Pero conceder deseos es para lo que he sido construido! ¡Es todo lo
que he conocido!

Rose negó con la cabeza.

--¿No lo ves? Podrás seguir concediendo deseos, si es lo que deseas.
Pero sería tu elección. No tendrías que hacer cosas como destruir gente,
hacerles daño o cosas como esa.

--¿Elección\ldots{} mía? --dijo el GENIO.

--¡Sí!

--¿Eso es\ldots{} la libertad?

--Sí, eso es la libertad.

--Entonces quizás\ldots{} me podría gustar--dijo el GENIO--. Me podría
gustar la libertad.

Rose respiró hondo.

--Allá va, entonces--miró al Doctor, que asentía en aprobación--.
Deseo\ldots{} que el GENIO sea libre. Que no tenga que conceder más
deseos a no ser que él quiera. Que no sea esclavo nunca más.

Un rayo de luz salió de la consola y golpeó al GENIO, el cual pareció
absorberlo como un espagueti. Hubo un trueno, un sonido triunfante.

--¿Ha cambiado algo? --dijo Rose.

----¿Por qué no lo intentas y lo vemos? --sugirió el Doctor.

--Deseo\ldots{}--dijo Rose, pensando--. Deseo\ldots{} que la nariz del
Doctor se vuelva verde.

--¡Ey! --dijo.

Rose abrió los ojos de par en par, aterrorizada.

--¡Oh, no! Parece que el GENIO no es libre después de todo\ldots{}

El Doctor salió corriendo en busca de un espejo y Rose se partió de
risa.

--¿Te está bien la libertad, entonces? --le preguntó al GENIO.

La pequeña criatura se incorporó en toda su envergadura, la cual no era
mucha, pero de repente parecía conllevar una dignidad que no había
habido antes.

--La libertad me está, de hecho, bien--dijo.

Rose se agachó.

--Sabes, no tienes que conceder deseos nunca más. Pero si hay algo que
quieres que desee para ti, sabes, te podría ayudar.

El GENIO levantó una pequeña garra escamosa.

--Me gustaría--dijo--, ir a algún lugar\ldots{} Un lugar dónde no haya
gente que codicie mi poder. Un lugar sencillo. Un lugar donde pueda
ser\ldots{} feliz--una lágrima cayó por su ojo y acabó al final de su
pico.

--Entonces deseo eso para ti--dijo Rose.

¡Crash!

Y el GENIO desapareció.

Pero Rose creyó oír la palabra ``gracias'' en un eco a través del aire
al desaparecer.

--Así que--dijo Rose-- este es de verdad el final de la aventura. Y
nunca tenemos que volver a Roma de n\ldots{}--de repente contuvo el
aliento y se hundió en la consola. Comenzó a pulsar botones al azar--.
¡Tenemos que volver! ¡Tenemos que volver y deshacer todo!

El Doctor abrió los ojos de par en par.

--¿Ah, sí?

--¡Sí! --se le quedó mirando, apremiándole para que se diera cuenta de
la importancia de aquello--. ¿No lo ves? Ursus nunca hizo la estatua,
¡la estatua del museo! ¡Tenemos que volver y hacer que la haga de alguna
manera, o cuando volvamos al siglo XXI la realidad explotará!

--Bueno, no queremos que pase eso--el Doctor se estaba riendo mientras
le apartaba amablemente las manos de la consola.

--¡No seas tan condescendiente! --dijo ella, enfadada--. ¡Ríe lo que
quieras, pero intento salvar el mundo!

Dejó de reír, pero no parecía ser capaz de dejar de sonreír.

--No me río de ti--dijo--. De hecho sí que tenemos que pasarnos por
Roma, pero no por esa razón. Vamos.

Le cogió de la mano y la llevó por la sala de control a una sala más
pequeña. Allí, entre un montón de parafernalia de esculturas, estaba su
estatua. La estatua del museo. La estatua de Fortuna, nueva y
reluciente.

Rose la miró boquiabierta.

--Pero nunca he posado para esto.

--No hace falta--dijo el Doctor, dándole golpecitos en el brazo, un
brazo que aún tenía una mano pegada a él.

--¿Qué quieres decir?

--Quiero decir--explicó--, que no tendrás que posar para esto. Como dijo
Mickey--el Doctor sonrió para sí--, fue esculpida por alguien que te
conoce bien.

Se pasó una mano por el pelo y parecía estar esperando un aplauso.

Rose dio una vuelta por la estatua.

--¿De verdad tengo el culo tan\ldots{}?

--Sí--le interrumpió el Doctor--. Esta estatua es precisa en cada
detalle. Culo. Brazos. Piernas. Uña rota en la mano izquierda.

Rose miró hacia abajo.

--Ey, ni siquiera yo me había fijado en ello. ¿Y bien?

--¿Y bien qué?

--Bien, ¿de dónde ha salido?

Él suspiró exasperadamente.

--Yo la he hecho.

Rose rió.

--No, en serio.

--Sí, en serio.

--Espera, ¿vas en serio? Pero, quiero decir, ¿cómo? No sabía que
esculpieras. Dijiste que no esculpías. Dijiste que no eras un maestro
escultor. Te oí decirlo.

--Aprendí--dijo el Doctor.

Estaba impactada.

--¿Cuándo?

--Esto es una máquina del tiempo--dijo, y le explicó todo. Cómo perdió
su rastro. Cómo volvió al Museo Británico y cómo se dio cuenta de la
verdad--. Los pendientes me dieron la primera pista--dijo--. pero cuando
Mickey y yo nos giramos y encontré mi firma en la parte trasera\ldots{}

--Será mejor que no hayas firmado mi trasero--dijo Rose.

--\ldots{} de la base--siguió el Doctor--, bueno, aquello fue bastante
esclarecedor.

--¿Quieres decir que nunca nadie se fijó en que esta estatua tenía
escrito ``El Doctor'' en ella antes? --preguntó Rose--. ¿Y no se han
preguntado por qué?

--Ah--dijo el Doctor--. La firma está en gallifreyan. No tenían ni idea
de lo que era. De cualquier manera, entonces supe que tenía que
encontrar la tú verdadera, y encontrar una estatua que tomara tu lugar.
Pensé durante un minuto en robar la estatua y traerla de vuelta aquí,
pero bueno, entonces sería una estatua de unos 4.000 años de antigüedad,
lo que habría confundido a la gente y también habría iniciado paradojas
de todo tipo, y creo que ya hemos tenido bastante de ellas por el
momento. Así que, bueno, introduje rápidamente las coordenadas, de
vuelta al Renacimiento, tomé unas clases de escultura--sacó el móvil de
Rose--. Mickey me pasó unas imágenes para hacerlo bien y Miguel Ángel me
ayudó con las partes más difíciles. Como tus orejas, fueron una
pesadilla para hacerlas bien. Y entonces, cuando hube terminado, volví a
Roma un par de días antes de que me fuera, y me escondí fuera de casa de
Gracilis, listo para seguir a Ursus cuando se fuera con\ldots{} contigo.
Una vez efectuado el rescate, el mundo ya estaba bien de nuevo.

Rose casi se quedó sin palabras durante un momento.

--¿Te fuiste a callejear durante meses y meses con Miguel Ángel mientras
yo estaba ahí como algo a lo que un perro podría acercarse y levantarme
la pata?

--¡Fuiste una estatua durante un par de horas! --dijo el Doctor con
indignación--. Y fue idea tuya en primer lugar. Bueno. En parte. Y no te
creerías lo explotador que es Miguel Ángel. Tiene que estar todo
perfecto.

Rose observó la estatua durante un rato más.

--Es perfecta--dijo al fin.

--Estaba inspirado.

Se sonrieron el uno al otro. El mundo estaba bien de nuevo.

--De cualquier forma--siguió el Doctor--¸ ¿sabes qué? Creo que me traes
suerte. Mi Fortuna, esa eres tú.

--Quieres decir como un tipo de mascota--dijo Rose--. Como tu trébol de
cuatro hojas. O como cuando llevas tus pantalones de la suerte a una
entrevista.

--Exactamente--le dijo el Doctor--. Eres mis pantalones de la
suerte--entonces dijo, más serio--. Me di cuenta cuando pretendiste ser
Fortuna en ese altar. Sabía que tenía que representarte de esa manera.

Rose frunció el ceño.

--Pero sólo fuiste a ese altar porque habías visto la estatua de mí como
Fortuna.

--Y sólo había una estatua de Fortuna porque había visto lo buena
Fortuna que eras.

--¿Otra paradoja?

El Doctor sonrió.

--Sólo una pequeña. Más bien lógica circular. Como que nadie, por
ejemplo, averiguó esa fórmula complicada para devolver a la vida a la
gente que estaba convertida en piedra.

--¡Así que nadie lo hizo! --se dio cuenta Rose. Entonces pensó en algo
más--. Dijiste que a veces Fortuna era ciega. Así que no sabe a quién
favorece.

Él asintió.

--Así que a veces le da la espalda a la gente que confiaba en ella. Y
algunas veces la suerte\ldots{} se va.

--Y los pantalones de la suerte son solo pantalones, y los tréboles de
cuatro hojas son solo plantas y las patas de conejos significa que
deberías llamar a la protectora de animales. No te preocupes.

Rose ayudó al Doctor a llevar la estatua a la sala de controles.
Brillaba como un jade verde con la luz de la columna central.

--No estaba mal ahí, ¿no crees? Parece que pega con la decoración.

--Creo que una Rose por TARDIS es suficiente--dijo el Doctor, que ahora
estaba agachado en la consola--. Algunos dirán que incluso es demasiado.

Rose puso una mueca triste.

--De cualquier forma, sabes que no puede quedarse ahí. Tenemos que
encontrarle un nuevo hogar.

La TARDIS aterrizó y Rose salió de ella, nerviosa. Pero supo dónde
estaban al instante.

--¡Estamos de vuelta en la villa! --dijo cuando el Doctor se le unió.

--Sí--dijo él--. Creí que te apetecerían unas pequeñas vacaciones a la
romana.

Ella se le quedó mirando.

--O quizás no. Vamos. Tenemos trabajo que hacer.

Un esclavo les vio y corrió a la villa. Segundos más tarde, Gracilis y
Marcia salieron corriendo, seguidos por un chico que Rose nunca había
visto antes, al menos no de aquella manera. Pero supo quién era.

--Debes de ser Optatus--le dijo, sonriéndole.

Él asintió con timidez.

--Y tú Rose. Debo agradecerte todo lo que has hecho por mí.

Intentó parecer modesta.

--Oh, en realidad no fue nada.

Marcia la abrazó fuertemente.

--¡Dices que no ha sido nada! ¡Lo que tú y el Doctor habéis hecho por
nosotros no se puede pagar! Oh, temí por tu seguridad\ldots{} no te
había visto desde\ldots{} desde\ldots{}

--Oh, no te preocupes de eso--dijo Rose apresuradamente, dándose cuenta
de que su último encuentro dirigido por el GENIO sería, al menos esperó
que fuera, un tanto borroso.

Pero Marcia seguía frunciendo el ceño.

--Tú y la esclava Vanessa\ldots{}

--¡Ah! --el Doctor puso un brazo alrededor del hombro de Gracilis--
Mira, quería tener unas palabras contigo sobre ella.

--¿Ah, sí?

--De hecho, sí. La cosa es, sé que te pertenece.

Rose soltó una risotada, pero el Doctor le lanzó una mirada.

--Sé que te pertenece, pero no va a volver.

Gracilis abrió la boca para hablar pero el Doctor le hizo callar.

--Sé que pagaste mucho dinero por ella. Pero míralo de esta forma. Tú la
compraste para que te ayudara a traer a tu hijo de vuelta, y tú tienes a
tu hijo de vuelta. No necesitas más esclavos. Te acabas de llevar un par
de docenas. Y\ldots{} me gustaría darte algo a cambio de su libertad. Si
me pudieras prestar un par de manos\ldots{}

Con la ayuda de unos esclavos, el Doctor trajo la estatua de Fortuna
fuera de la TARDIS y se la llevó a Gracilis.

--Creí que quizá tuvieras un hueco para esto--dijo el Doctor--. Después
de todo, ahora te falta una estatua\ldots{}

Así que la Rose de piedra fue llevada al pequeño huerto justo a la
entrada de la villa.

--¡Con cuidado! --gritó Gracilis, cuando la estatua chocó contra una
pared durante un giro abrupto.

--¿Está bien? --preguntó Rose.

--Oh, sí--dijo el Doctor--. Bueno, quizá haya una ligera grieta. Justo
en la muñeca--y sonrió.

El Doctor y Rose se sentaron en el jardín. El sol se reflejaba en el
estanque, emitiendo reflejos brillantes por el mármol blanco de la
estatua. El Doctor acarició un pavo real, el cual hizo un ruido de
maullido como un gato. El Doctor le devolvió el maullido.

Habían estado sentados a solas durante un rato cuando Gracilis se les
unió de nuevo. Les pidió perdón, pero quería preguntarles algo.

--Esa chica, Vanessa--dijo--. Leía de verdad los astros, ¿no?

Rose no estaba segura de qué decir, pero el Doctor asintió.

--Supongo que podrías decir que así es.

Gracilis estuvo callado durante un momento, pensativo. Entonces siguió:

--Creo que fue mandada por los dioses para ayudarnos. Y creo que
vosotros también fuisteis mandados por los dioses.

Rose rió.

--No, en serio, no lo fuimos. Te soy sincera.

--En ese caso--dijo Gracilis, mirando a la estatua--, debéis ser dioses.

--¡No, no lo somos! --comenzó Rose, pero Gracilis se había levantado y
ya se estaba yendo.

--Os honraré toda mi vida--dijo.

--¡Gracilis! --Rose de repente tuvo el recuerdo de algo.

--Sólo\ldots{} no más sacrificios, ¿vale?

Gracilis sonrió y le hizo una reverencia.

Rose miró por última vez su estatua cuando se levantaron, listos para
volver a la TARDIS y a nuevos lugares.

Casi 1.900 años más tarde, una granulosa imagen de la misma estatua
estaba enganchada con cinta adhesiva en el armario de la cocina de
Jackie Tyler. Era una lástima que no hicieran una postal en condiciones
de ella, pero Mickey le había hecho una foto con su teléfono y se la
había imprimido para ella, y era mejor que nada. Su hija. Su hermosa
hija, Rose. Jackie comenzó a cantar para sí misma mientras abría el
armario para sacar una comida de microondas para uno.

En el Museo Británico, Mickey Smith estaba en la sala de esculturas.

--Esta es la diosa Fortuna--le decía a un grupo de niños a los que
estaba guiando--. Daba suerte, o la quitaba. Pero aguantas lo que quiera
que haga. Porque cuando decide favorecerte, hace que todo merezca la
pena.

Y los niños, que habían estado inquietos y se habían estado empujando
los unos a los otros y retándose a robar algo en la tienda de regalos,
oyeron algo en su voz que les hizo prestar realmente atención durante un
momento.

--Es guapa--dijo uno de los niños.

--¡Ja! ¡Es tu novia! --le recriminó uno de los otros--. ¡La quieres!

Y Mickey sonrió cuando llevó a los niños bromistas hacia la siguiente
exposición.

Alrededor de unos 370 años después de aquello, Vanessa Moretti pasó otro
día solitario en su casa, mientras su padre estaba supervisando la
construcción de su nuevo laboratorio. Pensaba mucho en el tiempo que
había pasado en Roma. ¿Cómo podía haberlo odiado tanto? Seguro que
cualquier lugar era mejor que aquel sitio. Por cierto, todo parecía un
sueño en ese momento. Recordaba al Doctor explicándole una larga
historia, algo sobre su padre. Pero ahora estaba en casa, parecía que no
podía recordarlo. Deseó que pudiera ver de nuevo al Doctor y a Rose,
preguntárselo.

Pero no tenía más al GENIO, así que los deseos no se cumplían tan
fácilmente.

Deseó tener un GENIO.

Quizá su padre podría construirle otro.

Y quién sabe cuán lejos en el espacio o el tiempo, una pequeña criatura
escamosa con las garras de un dragón y el pico de un pequeño estaba
sentado admirando sus alrededores. La hierba era verde y el sol
brillaba. Un animal parecido a un conejillo de indias se acercó hacia el
recién llegado y le examinó con interés.

El GENIO miró al conejillo de indias.

--Ah, mi peludo amigo--dijo--, si tuvieras un deseo, ¿qué desearías?

El conejillo de indias pió. Hubo el ruido de un trueno resonando y
entonces una verdura naranja parecida a una zanahoria apareció. El
conejillo de indias pió otra vez.

--De nada--dijo el GENIO. A pesar de su pico, casi pareció estar
sonriendo--. Creo que seré feliz aquí.
