\chapter*{Capítulo Siete}
\addcontentsline{toc}{chapter}{Capítulo Siete}

El Doctor llegó a Roma durante la mañana del 19, el Quinquatrus, y se
abrió camino por las calles hasta la Colina del Aventino. No había
pasado el carro de Ursus durante la noche, ni había encontrado algún
rastro de él en los hostales por el camino, pero no iba a dejar que eso
le desanimara, esperaba que el hombre ya estuviera en Roma.

La primera parada fue el templo de Minerva, patrona de los artistas.
Había una multitud en el exterior cuando llegó él y el Doctor se movió
por entre ella, hablando mientras paseaba.

--Minerva es genial, ¿no es así? --no dejaba de decir a distintos
adoradores--. Y por cierto, tú que eres del mundillo de las artes, ¿no
conocerás casualmente al escultor Ursus?

Pero ninguno de ellos lo conocía. Oh, habían oído hablar de él, pero
claramente Ursus no era el centro de la vida y la sociedad de la
comunidad artística. No se unía a los cotilleos ni a las reuniones de
consejos, no recomendaba herramientas y acogía aprendices. Sus
habilidades de escultor eran alabadas, pero su subida meteórica hasta la
fama no había tenido la misma acogida. La gloria que había recibido en
menos de un año no era apreciada por aquellos que habían servido sus
aprendizajes durante muchas lunas. Le hablaron de rabietas y amenazas,
desaires y muecas.

Así pues, el Doctor descubrió bastante sobre Ursus, pero no su
localización.

Rechazando a sentirse descorazonado, comenzó un recorrido rápido por
Roma. Cualquier observador habría tenido dificultades para discernir si
era el más devoto de todos, visitando cada templo, o el más irreverente,
no trayendo ofrendas ni mostrando el mínimo respeto por la tradición. El
Doctor también visitó tabernas y tiendas, demostrando un infatigable
apetito por el vino con miel, los pasteles calientes y los cotilleos.

--Yo sé de una estatua de Ursus--le decía alguien, y el Doctor
recorrería la ciudad para encontrarse con una Vesta o una Flora de
mármol, alguna creación increíblemente realista que llenaba al Doctor de
furia.

Estaba en todo lo cierto que podía estar de la naturaleza del verdadero
``talento'' de Ursus. Ningún escultor podría haber creado todas aquellas
obras de arte en menos de un año sin nada más que un martillo y un
cincel.

El anochecer estaba cayendo y el Doctor no estaba más cerca de encontrar
a Ursus, o a Rose, o alguna pista. Pero no podía dejar de mirar.

Entonces vio un altar que aún no había visitado. Era pequeño, no del
tipo de los magníficos templos que había visto antes, pero era un altar
a la misma Fortuna. ¿Dónde mejor que encontrar una estatua de Fortuna
que en su propio templo? ¡Seguramente la diosa de la fortuna le podría
traer buena suerte!

No había sacerdotisas cerca, su primer golpe de suerte. El Doctor
respiró hondo y entró.

Una estatua de Fortuna se alzaba en el extremo del altar y sus corazones
se aceleraron. Pero aunque compartía la pose con la estatua que había
visto en el Museo Británico, aunque llegaba una cornucopia y miraba con
orgullo hacia adelante, no era Rose, ni siquiera era una estatua nueva.
El mármol estaba descolorido y la pintura había desaparecido.

--Rose es más guapa que tú--le dijo el Doctor a la estatua.

--Gracias--dijo la estatua.

Por supuesto no había sido la estatua. Era una voz que venía de algún
lugar detrás de ella. Pero al mismo tiempo, había algo mal en aquello.
El Doctor dio un paso adelante para investigar, pero cuando lo hizo casi
pisó un pequeño frasco de cristal que vino rodando de detrás de la
estatua. Parecía estar lleno de una brillante substancia verde. Se
detuvo para cogerlo y la voz siguió hablando:

--Esto traerá a Rose de vuelta a la vida, y al resto. ¡Adoradme todos,
pues soy Fortuna\ldots{} y todo eso!

El Doctor dio otro decidido paso hacia la estatua, pero una puerta se
abrió de par en par detrás de el y una voz gritó sin aliento.

--¡Doctor! ¡Doctor!

Se giró para ver a Gracilis irrumpiendo en el interior.

--¡Gracias a los dioses que te he encontrado! --resopló--. Vengo para
advertirte\ldots{}

Pero hubo otra interrupción. En el altar entró Lucius Aelius Rufus, el
hombre del hostal, acompañado por diversos guardas armados.

--¡Ahí está! --rugió Rufus, señalando al Doctor. El Doctor miró hacia la
estatua, aún sorprendido, pero entonces se giró para encararse a los
hombres de Rufus cuando se le acercaron y le agarraron.

--Bueno, perdonadme--dijo el Doctor, frunciendo el cejo amablemente--.
Esta no es forma de comportarse en un templo. Creo que podría tratarse
de sacrilegio. O es blasfemia, nunca recuerdo cuál es la diferencia. Una
de ellas, da igual cuál, es lo que está pasando aquí.

Los hombres le ignoraron y comenzaron a arrastrarle hacia la puerta.

--Y bien, ¿dónde estamos yendo? --preguntó el Doctor, iniciando una
conversación.

Rufus sonrió, mostrando unos dientes de oro.

--A la arena--dijo.

El Doctor le devolvió la sonrisa.

--¡Un día de relax! --dijo--. Es muy bonito por tu parte. Pero te voy a
decir algo, aún así, sería igual de feliz con una intima cena para dos,
un poco de charla\ldots{}

Uno de los hombres le golpeó en l acara, y el Doctor se derrumbó. Para
su horror, el frasco de cristal se le cayó de la mano. Intentó recogerlo
pero los hombres eran fuertes y él estaba mareado del golpe.

--¡Gracilis! --intentó llamarle, pero estaban ya fuera del altar y
recibió otro golpe por dar problemas.

No sólo estaba siendo arrastrado hacia un peligro, sino que estaba
siendo arrastrado alejándose cada vez más de lo que podría ser la única
salvación de Rose. Y era incapaz de investigar el mayor misterio del
día: ¿por qué alguien en un templo de la Antigua Roma le hablaba a
través de algo que sonaba distintivamente como una grabadora?

Pronto fue claro dónde estaba siendo el Doctor llevado. Una enorme
estructura se cernía sobre él, un gigantesco edificio redondo tan alto
como unos treinta Doctores, hecho de una brillante piedra blanca que
dolorosamente le recordaba al terrible destino de Rose. Docenas de arcos
se estrechaban en los pisos más bajos, actualmente vacíos de vida. Pero
el Doctor sabía que en cuestión de tiempo, cientos de miles de personas
se apretujarían bajo esas entradas, deseosos de ver el espectáculo
sangriento que les esperaba en su interior.

Aquel era el Anfiteatro Flavio, el cual sería un día conocido como el
Coliseo. Hogar de las luchas de gladiadores, cazas de bestias salvajes,
y cientos entre otros cientos de ejecuciones macabras.

--¿Vamos a ver un espectáculo? --preguntó el Doctor, con interés--. Lo
único que parece que hemos venido el día equivocado. Está un poco
callado todo, así que probablemente será mejor que volvamos en otro
momento.

--No se derrama sangre en el Quinquatrus--le informó uno de sus
captores.

--Ah, claro, me alegra oír eso. Bueno, si me dejáis ir, entonces\ldots{}

El hombre sonrió, con cara de pocas migas.

--Mañana, por otro lado, cuando honremos a Marte\ldots{}

El Doctor suspiró. Se estaba cansando de aquello. De repente frenó,
hundiendo sus tacones y haciendo desestabilizarse a sus sorprendidos
captores. Se escurrió los brazos, les pegó un empujón y se libró del
agarre de los hombres, dejándoles jadeantes con asombro.

--No se preocupen, caballeros, puedo encontrar mi camino de vuelta a
casa--dijo, moviéndose fuera de su alcance\ldots{}

\ldots{} y hacia los hombres de otros dos hombres que habían aparecido
por detrás de él.

Realmente no era aquel su día de suerte después de todo.

Las monedas cambiaban de mano entre los dos grupos de hombres y el
Doctor fue arrastrado de nuevo, esta vez a través de una puerta y hacia
una maloliente y oscura estructura subterránea.

Los dos hombres que le sujetaban encajaban con el lugar. Uno era bajo y
robusto, con una curva cicatriz dividiendo en dos su mejilla desde la
boca hasta el ojo, dándole una mueca torcida de payaso. El otro era más
alto, con una larga cara coronada por una mata grasienta de pelo negro.
Ambos olían a sudor y a miseria.

El Doctor reconoció sus alrededores, no de manera específica, sino que
era el tipo de sitio que visitaba involuntariamente cientos y cientos de
veces. Las paredes húmedas, la oscuridad, el olor a miedo, aquello era
una mazmorra.

--No he tenido un juicio, sabéis--remarcó, animando una conversación
hacia el hombre de la cicatriz, al que sus colegas se le referían por el
nombre de Thermus.

--Te juzgamos en tu ausencia--le respondió el hombre.

--¿De verdad? ¿Sabes? La última vez que miré, la pena por tomar prestado
un caballo no es la muerte. Asumo que debió ser terrible en su día, o lo
será algún día, pero creía que un ``lo siento, creo que ha habido un
malentendido, aquí tienes un denario o dos por tus molestias'' era más
adecuado.

--No hacemos la ley--dijo el hombre alto, Flaccus.

--No, pero sí Lucius Aelius Rufus--señaló Thermus.

Ambos hombres parecieron encontrar aquella observación una buena broma y
se rieron felizmente.

--Ah--dijo el Doctor--. ¿Debo creerme que el caballero en cuestión es un
magistrado de algún tipo? ¿De los corruptos y ávidos de poder con un
inflado sentido de su propia importancia, quizás?

Ambos hombres sofocaron una risa, lo que el Doctor entendió como un
``Sí''.

--Necesito ver a alguien más, entonces--les dijo--. Alguien que puede
gobernar por encima de Rufus. El emperador. Si pudiera tener una
audiencia con el emperador\ldots{}

Y entonces fue cuando los captores del Doctor se reían tan fuerte que
tenían dificultades para mantenerse en pie.

--¡Ver\ldots{} al\ldots{} emperador! --tosió Flaccus--. Sí, claro, le
mandaremos una nota. Siempre se pasa por aquí al anochecer.

--Bueno, eso es útil--dijo el Doctor--. Oh, espera, ¿estabas siendo
sarcástico? Porque obviamente eso es enormemente útil. Dime, ¿has
recibido algún tipo de entrenamiento de aspectos de trabajador social
para tu oficio aquí, o ha venido por naturaleza?

Llegaron al extremo de un pasillo iluminado únicamente por una sencilla
antorcha. Las llamas crepitaban en las barras metálicas cercanas, una
ligera chispa entre la oscuridad. Thermus dejó caer el brazo del Doctor
y se movió hacia adelante, con una gran llave de metal en su mano. La
puerta se abrió de par en par.

Flaccus agarró el bolsillo del cinturón del Doctor y lo arrancó.

--¡No! --gritó el Doctor.

--¡Sí! --dijo Flaccus, sarcástico--. Después de todo, no lo vas a
necesitar más.

Entonces le dio un fuerte empujón y el Doctor se tambaleó hacia la
celda.

--¡Esto es un total y completo mal uso de la justicia!

Pero no hicieron caso. Thermus pegó un portazo y giró la llave.

Normalmente, al Doctor no le habría preocupado. Cualquier cerrojo podría
ser abierto por su destornillador sónico. Pero aquel era el
destornillador sónico que estaba en su bolsillo de cinturón. Y aquel era
el bolsillo de cinturón, el bolsillo de cinturón de Rose, que estaba al
otro lado de las barras, gradualmente retrocediendo fuera de la vista
mientras colgaba de la grasienta y gruesa mano.

El Doctor se giró de las barras y se dio cuenta por primera vez que no
estaba solo. Vio unos ojos en la luz titilante, reflejando unas caras
que no había visto: los típicos dibujos para Halloween. Caminó hacia
adelante y vio los propietarios de los ojos de mejor forma: un mar de
caras desesperadas apenas registrando su presencia.

Se sentó en el frío suelo de piedra y se sonrió, aunque dudaba que a
nadie le importara, aunque pudieran verle.

--Hola--dijo--. Soy el Doctor.

Hubo un silencio por un momento, entonces una voz de entre las sombras
dijo:

--¿Entonces puedes curar la crucifixión?

--Sí, ¿o ser quemado vivo? --dijo otro.

--Prevenir es mejor que curar, ¿no creéis? --dijo el Doctor.

Hubo murmullos de descontento con eso.

Una voz más razonable dijo:

--Mira, todos nosotros vamos a morir mañana. No hay forma de huir. La
mayor parte de nosotros ni siquiera eligió esta forma de irnos.

--Mejor una rápida muerte que una lenta en las minas--añadió una cuarta
voz.

--Sí. Así que perdónanos si no te damos la bienvenida como manda. No
tiene mucho sentido hacerse amigos cuando quizá tengas que matar al otro
mañana.

--Oh, no lo sé--dijo el Doctor--. No creo que hacer amigos sea una mala
cosa. No es que espere que compartamos una lata de judías y comencemos a
explicar cosas divertidas de nosotros. Tengamos una charla. Por ejemplo,
¿por qué os vais a matar los unos a los otros mañana?

--¿Eres tonto o qué?

--Debe ser extranjero--dijo la voz más amable--. Mira, compañero, así
son las cosas aquí. No sabemos exactamente cómo nos vamos a ir, pero nos
vamos a ir. Ser quemados vivos, crucificados, devorados por bestias, o
nos harán luchar contra otros hasta la muerte. Y entonces la única forma
de sobrevivir es matar y matar de nuevo y seguir matando, en una
desesperada esperanza de que la multitud estará tan impresionada que no
querrán que seas asesinado al final. Es la única oportunidad que tenemos
de salir de aquí con vida.

--Parece una oportunidad muy pequeña.

--Lo es. Pero es mejor que no tener nada.

La voz del Doctor estaba llena de tristeza.

--¿Dónde hay vida hay esperanza? ¿Cómo puedo deciros que estáis
equivocados? --se detuvo--. Pero, ¿qué hay de la dignidad? ¿Qué pasa si
no participamos voluntariamente en esta broma sangrienta? ¿Qué pasa si
nos mantenemos juntos y rechazamos luchar?

--Entonces nos matarán justo donde estaremos. No hay vida, y mucho menos
esperanza. Nadie ha escapado nunca de la arena.

El Doctor sonrió, aunque ninguno de sus compañeros pudo verlo del todo.

--Entonces es vuestro día de suerte. Porque hacer las cosas que nadie ha
hecho nunca es mi especialidad.

El Doctor llegó a pensar sobre sus cuatro compañeros de charla y celda
como John, Paul, George y Ringo. Habría otros también, hombres y
mujeres, hombres libres y esclavos, demasiado metidos en las
profundidades de la desesperanza como para poder hablar con nadie.
Varios cientos de prisioneros eran mantenidos allí listos para los
juegos del próximo día. Muchos de ellos habían sido público voluntario
en los juegos previos y sabían lo que les esperaba.

--Nunca quieres que sean liberados--confesó George--. Así es cómo
funciona. Todo el mundo aúlla pidiendo su sangre y no te queda otra que
hacer lo mismo.

--Recuerdo aquel brillante--dijo Paul--. Había un tipo, un músico, y
pensó que estaba allí para tocarle algo a la multitud. ¡Entonces a mitad
de algún ritmo soltaron a los animales! Él creyó que era un error y
comenzó a corretear, intentando hacer que le sacaran de allí, pero por
supuesto no lo hicieron. ¡Así que intentó encantar a las bestias con su
música, igual que hizo Orfeo en el Inframundo!

--¿Funcionó? --preguntó John.

--Nah. Supongo que el león que le cogió no amaría demasiado la música.

John se rió con su propia anécdota.

--Hubo una vez que cogieron a una pareja de hombres ciegos--dijo--. Les
dieron espadas y los soltaron. Pegaron bandazos, sin idea de lo que
estaba pasando, a veces cortando un pedazo de oreja u otra cosa por
suerte. Aquello fue de la risa--se detuvo--. Ahora ya no tiene mucha
gracia.

--¿No, verdad? --coincidió Paul.

George y Ringo murmuraron algo estando de acuerdo.

--Entonces hablemos sobre las alternativas--dijo el Doctor.

Algunos de los prisioneros se tumbaron para dormir, pero el Doctor se
mantuvo despierto toda la noche. A veces los guardas les visitaban, y el
Doctor tomaba cada oportunidad para recordarles que él, en su opinión,
estaba allí ilegalmente. Éstos sólo se reían.

Incluso el Doctor, con su excelente sentido del tiempo, encontró difícil
saber cuándo llegó el siguiente día. La noche reinaba eternamente en la
mazmorra, con la única antorcha en el exterior funcionando tanto como de
sol y de luna. Eran los sonidos más que la luz lo que les alertó:
rugidos y aullidos y berridos.

--Se están preparando para la caza de las bestias salvajes--les explicó
George--. Lo primero del día--asumiendo que el Doctor era un extranjero
en Roma, había asumido explicarle todos los hábitos en la arena--.
Tienen maravillosos animales. Tú que eres de un país extranjero, debes
haber visto ya unas cuantas bestias, en tu hogar.

--Tienen leopardos, y ciervos, y esas increíbles cosas altas llamadas
jirafas.

--Y los elefantes, no los olvidéis.

El Doctor apretó los puños.

--¿Sabéis cuántas especies se extinguirán por estos juegos? --gritó con
furia--. ¡Por todos los cielos, ¿qué os pasa, humanos?! Creéis que sois
lo único en el planeta que merece la pena, que podéis explotar la
naturaleza sólo para mostrar vuestra superioridad. ¿Podéis siquiera
comprender una fracción de lo que está pasando? --entonces se calmó
igual de rápido que se enervó, volviéndose triste en vez de enfadado--.
No, probablemente, no podéis. Y seguro que ni os importaría si lo
hicierais.

--Debe ser extranjero--concluyó Ringo tras un momento--. O eso o
chalado.

--Extranjero--decidieron los otros.

Después de un rato, otros sonidos comenzaron a escucharse, difuminándose
por la distancia. Había música, seguida de la risa de una multitud
expectante, creciendo en volumen a medida que iba llegando la gente.

--Llevan a las bestias a jaulas--explicó George--, y entonces son
subidas hasta la planta baja. La arena está montada con árboles y
colinas y cosas, y los entrenadores usan barras para obligarles a
moverse. Intentan crear un poco de pánico, hacer que los bichos corran
por ahí un rato, y entonces les cazan. Algunas de las bestias se matan
entre ellas, y eso está bien, pero los entrenadores se encargan del
resto. He visto a un hombre matar a un tigre con sus propias manos
desnudas--concluyó melancólicamente.

Se sentaron en silencio durante un rato, escuchando. Gradualmente los
rugidos de los animales crecieron ligeramente y los aullidos de la
multitud llegaron de golpe.

--Ya ha acabado--observó John.

--¿Y qué viene a continuación?

--¿A continuación? Bueno, primero tienen que limpiar los cuerpos. Hay un
poco de trabajo en ello.

--¿Y luego?

--Y luego somos nosotros. Es nuestra hora de morir.
