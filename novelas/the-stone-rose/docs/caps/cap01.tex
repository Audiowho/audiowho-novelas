\chapter*{Capítulo Uno}
\addcontentsline{toc}{chapter}{Capítulo Uno}

Una vez Rose se hubo recuperado de la conmoción principal de encontrar
una estatua de sí misma en el Museo Británico, se emocionó.

--¡Esto es maravilloso! --dijo--, ¿os dais cuenta de lo que significa
esto? Es posible que nos vayamos en nada a\ldots{}--miró el cartel-- la
Roma del siglo II. ¿No veis lo maravilloso que es?

--¡Cielos! --dijo una voz detrás de ellos--. Me recuerda a una chica que
conocí una vez. Me pregunto qué le habrá pasado--el Doctor les alcanzó y
le lanzó a Rose una sonrisa que podría haber derretido incluso a una
estatua de mármol. Ella le devolvió la sonrisa.

Jackie leía el cartel bajo la escultura.

--Aquí dice que es una estatua de la diosa Fortuna--dijo--. No me digas
que he dado a luz a una diosa. Pete nunca se lo creería.

--Fortuna, la diosa romana de la buena suerte--le dijo el Doctor--.
Retratada con una cornucopia.

--Aquí dice que es un cuerno de la abundancia--dijo Jackie.

El Doctor parecía entretenido.

La figura de Rose sujetaba una cornucopia, llena de fruta de piedra y
flores, en el recodo de un brazo. El otro brazo ya no estaba entero, con
un muñón en la muñeca señalando al grupo que tenía a su alrededor. Rose
levantó sus propias manos.

--Espero que eso no se haya hecho a partir del real--dijo.

--¿A que no sabes qué? --dijo Jackie--. Lleva tus pendientes.

Rose se sacó uno para compararlo. Era un disco metálico liso con un
diseño de espiral irradiando de una pequeña flor en el centro. Lo sujetó
cerca de la oreja de la estatua. Idéntico, incluso con la flor.

--¡Esto es increíble! --dijo--. Es muy detallado.

Dejó caer el pendiente verdadero en el bolsillo de su chaqueta tejana y
sonrió.

--¡Parece que tengo un futuro como modelo de artista! Siempre me ha
gustado.

Mickey frunció el ceño.

--Cuando mi colega Vic te pidió que posaras para él, dijiste que no.

Rose suspiró.

--Sí, pero tumbarme en una alfombra de lana en ropa interior mientras tu
amigo Vic me hace fotos no se parece demasiado a posar como diosa para
algún antiguo romano.

El Doctor se puso las gafas y examinó la mano que quedaba de la estatua.

--Mmm\ldots{}--dijo.

--¿Qué pasa? --preguntó Rose.

--La estatua también lleva tu anillo.

Rose miró hacia el anillo de su mano derecha.

--Si lleva mis pendientes, ¿por qué no?

El Doctor frunció el ceño.

--Solían hacer los torsos por separado, los producían en masa, y luego
les clavaban una cabeza. Obviamente el escultor estaba tan enamorado de
tu figura que tuviste que ser modelo para toda la obra.

--¿Y eso es tan difícil de entender? --preguntó Rose, levantando una
ceja.

El Doctor se giró y le lanzó una sonrisa animosa.

--No, no lo es.

Rose encontró muy difícil apartarse de su doble de piedra, pero entonces
nunca sería hecha y todos serían tragados por una terrible paradoja. Así
que se dejó a sí misma atrás, pasando por el pie gigante.

--Ah, es culpa mía--comentó el Doctor--. Es todo lo que queda del Ogro
de Hyfor Tres. Forma de vida basada en silicona. Lo vencí por allá
el\ldots{} debía de ser el 200 y pico aC. Ahí estaba yo: ¡toma esto,
ogro malvado! Y ahí estaba él: ¡ja, ja, ja, nunca me vencerás! Y yo le
respondí: no estés tan seguro de ello\ldots{}

--Dice que es una estatua colosal acrolítica--se apresuró en señalar
Mickey.

--Bueno, eso es lo que dirían--dijo el Doctor, pasando los sarcófagos,
pasando las hileras de cabezas de piedra, cuyas miradas parecían
dignificar a Rose.

Perdieron de nuevo al Doctor en la sección egipcia, y Jackie se fue a
ver si encontraba una postal de su hija de Piedra. Rose y Mickey se
quedaron en la entrada, esperando.

--Así que, ¿cómo has descubierto esto? --le preguntó Rose pasados unos
minutos de silencio--. No es algo normal para buscar, ¿verdad?

Mickey parecía avergonzado, mirando abajo hacia el suelo.

Ella abrió los ojos de par en par.

--¿Qué? No puede ser tan malo, ¿verdad? ¿No has estado robando o algo?
¿O te estás viendo con una de las chicas de la tienda de regalos y no me
lo quieres contar?

Mickey frunció el ceño negándolo, pero seguía pareciendo avergonzado.

--¡Vamos! ¡Dímelo! --dijo.

Mickey puso sus hombros atrás, como intentando parecer un valiente.

--Bueno\ldots{} he estado haciendo un poco de voluntario. Ya sabes, con
los niños y eso.

Rose rió con delicia.

--¡Pero eso es fantástico!

Él se encogió de hombros, avergonzado de nuevo.

--Bueno, tú te vas haciendo el bien por el universo, sólo creí que
podría hacer un poco de lo mismo aquí en casa, eso es todo.

El Doctor se les acercaba.

--No se lo digas--le silbó Mickey.

Rose suspiró, exasperada.

--Sí, claro por que ser buena persona no es muy guay, ¿verdad? --pero no
pudo evitar agarrarle y darle a Mickey un beso rápido en la mejilla--.
Eres un buenazo.

Jackie se les unió, con su caza de postales sin haber tenido éxito, y
los cuatro se abrieron camino hasta la luz del sol.

--Bueno, adiós por ahora. Cuidaos. No hagáis nada que yo no haría--dijo
el Doctor mientras llegaban al final de los amplios escalones de la
entrada del museo, tendiéndole la mano a Mickey.

--¿Qué? ¿Ya os vais? ¡A penas me ha dado tiempo de decirle hola a mi
única hija cuando ya te la estás llevando de nuevo! --se quejó Jackie,
poniendo las manos en la cintura.

--Nos encantaría quedarnos--dijo el Doctor sin preocuparse en intentar
aparentarlo, poniendo una mano alrededor del hombro de Rose--. Nos
encantaría, claro que nos encantaría. Nos encantaría mucho, mucho,
mucho. Pero me temo que tenemos una cita a la que acudir.

--¿Tenemos? --dijo Rose.

--Creía que era obvio--dijo el Doctor--. Tú y yo nos vamos a la Antigua
Roma.

--¡Espera un minuto! --les gritó Jackie--. ¡He visto esa Roma en la
tele! ¡Cuídate, hija mía! Cuídate de lo que pueden llegar a hacer.

Rose rió.

--¡Guárdate la toga, mamá! ¡Puedo cuidarme yo misma!

Rose tropezó hacia la consola de la TARDIS mientras la sala de controles
se tambaleaba hacia un lado. El Doctor toqueteaba los botones de la gran
consola de bronce con forma de seta en el centro, tocando un botón allí,
levantando una palanca aquí, haciendo algo enérgicamente con una bomba
en otra parte.

Dio un paso vacilante hacia adelante mientas la máquina del tiempo
parecía estabilizarse, pero debía de haber estado esperando a ello,
porque al instante se movió justo para el lado contrario. La sábana con
la que se había cubierto el hombro se cayó al suelo, pero al menos frenó
su caída después del próximo zarandeo de la TARDIS.

--Ya encontraremos algún lugar en el que quedarnos--dijo el Doctor,
mirándola hacia abajo desde su posición erguida pero poco
estabilizada--. No hay necesidad de llevar tu propia ropa de cama.

--Es para vestirme, no para dormir--dijo Rose. Suspiró--. Fui a una
fiesta de togas una vez, pero no me acuerdo de cómo envolverme con esta
cosa.

El Doctor sonrió.

--Las chicas guapas no llevan togas--le dijo.

--¿No?

--Nop. Y aunque lo hicieran, probablemente no llevarían un estampado de
Winnie the Pooh.

Rose miró más de cerca la sábana. En una esquina, Winnie the Pooh estaba
sentado comiendo miel con Piglet a su lado.

--No me he dado cuenta--dijo--. Pero qué sábanas más bonitas guardas. Ya
sabes, por si alguna cuna acaba aquí. Así que, ¿qué debería llevar,
entonces, oh gran dios romano de la moda?

Señaló con la mano hacia un lado.

--Oh, habrá algo por ahí atrás. Busca con la letra R de Roma. O A de
Antigua.

--¿Y tú qué? --preguntó ella--. ¿La D de Destacar?

El Doctor vestía firmemente un traje del siglo XXI con una camisa azul y
unas deportivas, no el tipo de cosa que alguien llevara hacía varios
milenios.

--Ya encontraré algo--dijo, inclinándose para girar un disco.

La TARDIS se zarandeó mientras Rose se dirigía al umbral de la puerta,
arrastrando la sábana por detrás de ella.

--Sería mucho más fácil si arreglaras algunos estabilizadores de esta
cosa--le dijo.

--¡Los marineros saben mantenerse en pie con cosas peores que esto! --le
gritó alegremente, bailando algunos pasos de baile para darse
credibilidad.

Rose gruñó.

--Sí, bueno, tú dame unos sorbos de ron y te repito lo mismo--y se
volvió a tropezar.

La TARDIS finalmente aterrizó. Rose vestía un vestido azul pálido hasta
los tobillos, chocando ligeramente con el tono verde de su cara de
náusea, con un chal azul oscuro cubriéndole la cabeza, escondiendo su
pelo que estaba peinado elaboradamente hacia atrás. El Doctor vestía una
lisa túnica blanca que acababa por las rodillas, con su destornillador
sónico colgando absurdamente de su cinturón.

--Espero que estemos en la antigua Roma--dijo Rose--. Te lincharían en
el barrio si te pasearas vestido así.

--Estoy seguro de que me rescatarías--dijo el Doctor.

Abrió las puertas y salieron al exterior, ese primer paso en un mundo o
tiempo extraño que nunca perdía su emoción, sin importar las veces que
lo hicieran.

Estaban una ciudad o en un pueblo, con edificios de apartamentos a ambos
lados. El cielo era azul, pero el tipo de azul frío que indicaba la
primavera o el otoño temprano.

El Doctor observó el horizonte.

--¡Ajá! ¿Ves eso? --indicó a un enorme pilar con la silueta de un hombre
en su punta, visible por encima de los tejados--. La columna de Trajano.
Estamos definitivamente en Roma, entonces. A no ser que tu barrio se
haya expuesto al mundo.

--Huele como mi barrio--dijo Rose, arrugando la nariz. Dio un paso hacia
adelante e hizo una mueca mientras sus sandalias chapotearon contra un
charco profundo--. ¡Y mira estas calles! ¡Están inundadas! ¿Esto es Roma
o Venecia?

El Doctor miró hacia sus pies y alzó una ceja.

--Bueno, esto explica el hedor.

Rose frunció el ceño.

--¿Qué quieres\ldots{}?--entonces se dio cuenta--. Oh. Puaj. Puaj, puaj,
puaj, puaj. Eh, creía que los romanos inventaron las cloacas y las
cañerías y todo eso.

--Lo hicieron--le dijo el Doctor--. Pero no creo que hayamos aterrizado
en la mejor parte de la ciudad\ldots{}

--¡Ya te digo yo que no! --exclamó Rose, mientras un grito sonó de
repente viniendo de una calle cercana.

Inmediatamente, ambos comenzaron a correr hacia el sonido.

Tres jóvenes rodeaban a un hombre mayor con barba y el pelo gris. Estaba
tirado en el suelo, claramente jadeando, mirando con miedo a la daga que
le ponían en la cara.

--¡Ey! --gritó Rose--. ¡Dejadle en paz!

Los hombres ni se giraron para mirarla.

--¡Ayuda! --gritó el anciano--. ¡Por favor, ayudadme!

--Danos todo lo que lleves encima, abuelo. Haz lo que te decimos y todo
estará bien--dijo el hombre con la daga.

--Eh, perdónenme, caballeros--comenzó el Doctor, lleno de confianza,
adelantándose.

Esta vez se giraron para mirar, y Rose aprovechó la ventaja de la
distracción. Había un montón de jarras de cerámica enormes apoyadas
junto una puerta cerca de ella y una de ellas acabó volando hacia la
cabeza del ladrón que llevaba la daga. El Doctor se adelantó y liberó al
hombre de su daga, mientras más jarras se estrellaban contra sus dos
compañeros. Al rato, los tres corrían por la calle, con fragmentos de
cerámica colgando de su pelo y ropas.

--¡Ja! --les gritó Rose, mientras el Doctor ayudaba al anciano a ponerse
en pie. Parecía un poco conmovido, pero bueno, no era de extrañar.

--Mil gracias--dijo débilmente--. Cneo Fabio Gracilis a su servicio.

Las demás introducciones fueron cortadas mientras una puerta cercana se
abría de golpe. Un hombre con la cara roja y la mirada enfadada
contempló el montón de cerámica roto a sus pies.

--¡Eh! ¿Qué les habéis hecho a mis ánforas?

--Eh\ldots{} ¡fueron ellos! --dijo Rose, mintiendo, señalando a los tres
ladrones.

El hombre corrió detrás de ellos, gritándoles:

--¡Ey! ¡Ey! ¡Ey! --mientras el Doctor y Rose hicieron una huida rápida
en la dirección contraria, llevando a Gracilis entre ellos.

--¿Está bien? --le preguntó Rose, cuando estuvieron a una distancia
segura y se hubieron parado--. ¿Le robaron algo esos tipos?

El hombre negó con la cabeza, pero el esfuerzo pareció hacerle perder el
equilibrio.

El Doctor se adelantó para cogerle.

--¡Ep! Quieto ahí. No creo que esté usted bien, ¿verdad? ¿Está herido?

--No, no--dijo Gracilis--. Ha sido el susto, saben\ldots{} Y también
debo confesar que me siento un poco mareado.

El Doctor frunció el ceño.

--¿De verdad? ¿Puede recordar qué día es hoy?

--Ah, tampoco no estoy tan mal--dijo el hombre--. Son los idus de marzo.

Rose casi se atraganta.

--¡Bromea!

Gracilis pareció sorprendido.

--¿Entonces me he equivocado? ¿Sufro de fiebres cerebrales?

El Doctor frunció el ceño a Rose pero sonrió ampliamente a Gracilis.

--No, no, está en lo cierto. Asumo que también sabe en qué año estamos,
¿verdad?

--¿El año? --dijo el hombre incrédulamente--. Por supuesto que sí. De
verdad, señor, aprecio su preocupación, y por supuesto su valiente
intervención, pero debo asegurarle que estoy bien. No hay ninguna
necesidad de esto.

--¡Completamente! Está usted perfectamente--dijo el Doctor, dándole unas
palmadas en el hombro a Gracilis y poniendo una mueca a Rose. Con los
labios le dijo ``Valía la pena intentarlo'' y después un ``Ya lo
descubriré luego'' --. Bueno, completa salud en el frente de la memoria.
Excelente, ahora dígame, ¿hace cuánto que no come?

Gracilis pareció pensativo.

--¿Sabe qué? No tengo ni idea. Ayer quizá. O quizá el día anterior.

--Entonces antes de hacer nada más, pongamos en el menú algo para comer
y un lugar donde sentarnos. Vamos.

--¿Pero no debemos estar alerta? --dijo Rose alegremente--. Ya sabes,
del envenenamiento de la comida y tal\ldots{}

El Doctor volvió a fruncir el ceño.

--De acuerdo. Busquemos algo para comer--dijo ella--. Podemos encontrar
una parte más bonita de la ciudad, por eso.

Pero Gracilis negaba con la cabeza.

--No, no, no. ¡No hay tiempo! ¡Debo seguir mi búsqueda!

El Doctor fue amable pero firme, casi como si fuera un doctor de verdad.

--Comida y reposo. No servirá de nada a nadie hasta que no tenga de eso.
Y entonces, bueno, Rose y yo nos encanta una buena búsqueda, ¿verdad,
Rose?

--Nos encanta, sí--dijo Rose.

--Así que nos dirá qué está buscando y nosotros lo buscaremos con usted,
¿trato hecho?

--Eh\ldots{}--dijo Gracilis. Pero el Doctor ya le había agarrado la mano
y la había estrechado--. Trato hecho.

Una vez hubieron llegado a la parte principal de la ciudad las calles
estuvieron mucho más pobladas.

--¡Es como la calle Oxford en Navidades! --dijo Rose conteniendo el
aliento, mientras la décima o undécima persona la esquivaba.

--Roma tiene una población de un millón de personas--dijo el Doctor.

--¿De verdad?

--Sí--comenzó a contar a todos los viandantes--. Uno, dos, tres\ldots{}

--Sí, de acuerdo. Te creo. Pero parece como si cada uno de ellos fuera a
la dirección contraria a la nuestra--pegó un bote para esquivar a un
caminante persistente--. ¡Y van todos bebidos!

--Es un día festivo--le explicó el Doctor.

--¿Sí? ¡Qué suerte para nosotros!

El Doctor negó con la cabeza.

--Me habría sorprendido si no lo hubiera sido. Para los romanos, casi
cada día es festivo por una razón u otra.

Rose sonrió.

--¡Qué suerte para ellos!

Finalmente el Doctor se las apañó para abrirse camino a través de lo que
Rose llamaría una pequeña cafetería, aunque probablemente tendría algún
otro nombre gracioso en latín. La mayoría de sus clientes compraban
comida para llevar, pero había unas pequeñas mesitas para aquellos que
quisieran sentarse.

--Parece un Starbucks--dijo Rose. El Doctor compró un puñado de fruta y
pastelitos y tres copas de vino especiado, que resultó saber a vinagre
hervido con dientes de ajo, mientras Rose dejaba a Gracilis en un banco.

Rose no se había dado cuenta de lo pálido que estaba el anciano hasta
que vio cómo el color le volvía a la cara con el vino y los pasteles.

--Gracias--les dijo por trigésima vez--. ¿Cómo podría pagároslo? Dejadme
daros una recompensa--comenzó a abrir un saquito en su cinturón e hizo
sonar unas monedas.

--Oh, no aceptamos recompensas--dijo el Doctor, poniendo una mano para
rechazarlo.

--De verdad, hacemos esto por diversión--le dijo Rose a Gracilis,
observando su expresión sorprendida--. Así que, ¿qué está buscando?

La cara del anciano empalideció de nuevo y Rose se alertó bastante. Pero
se relajó y respiró profundamente.

--A mi hijo--dijo--. A mi apuesto e inteligente hijo, Optatus. Ha
desaparecido. ¡Un chico, o debería decir, un hombre, de sólo dieciséis
años!

--¿Y piensas que puede estar en algún lugar de Roma? --preguntó Rose.

Gracilis suspiró.

--No lo sé. Mi familia actualmente reside en nuestra villa del campo,
pero ha sido registrada, y todas las tierras de alrededor. Pensé en
Roma, ya sabéis cómo son los chicos, siempre buscando aventuras en las
calles de la ciudad. Pero he buscado y he preguntado y he rogado de una
manera poco adecuada para mi posición y no he encontrado ni rastro.

El propietario de la cafetería, un hombre barrigón con manchas de comida
en su túnica, no se molestaba en ocultar el hecho de estar escuchando su
conversación con interés.

--Eh, yo sé qué puede hacer--intervino de repente.

Gracilis saltó de su sitio.

--¿Puedes ayudarme a encontrar a mi hijo?

--Bueno, no--dijo el hombre--. No encontrarle exactamente--Gracilis se
volvió a hundir en el asiento--. Pero creo que sé quién podría.

Salió del mostrador y se sentó en el banco cerca de Rose. Su olor a
pescador sobrepasó incluso las oleadas avinagradas del vino y tuvo que
hacer un esfuerzo por no vomitar.

--Bueno, no nos deje con el suspense--dijo el Doctor.

El hombre respiró por la nariz profundamente.

--Hay una chica. Dicen que puede ver el futuro, cualquier cosa, sólo
mirando las estrellas.

--¿Una astróloga? --preguntó Gracilis.

--Exacto--respondió el hombre panzón--. He oído que predijo que Adrián
reconstruiría el Panteón, ¡y lo está haciendo!

--Eso no es nada--añadió un hombre de un banco cercano, a través de un
bocado de pan y queso--. A mí me dijo que iba a tener una discusión con
mi mujer, ¡y se hizo realidad!

--Bueno, sí--dijo el hombre panzón--, pero es que intentaste ligar con
la chica delante de tu mujer. Hasta yo podría haberlo predicho. De
cualquier manera, he oído que dice que el Imperio va a caer en unos
cuantos siglos. Estoy pensando en irme con mi familia, al lado seguro.

Rose chasqueó la lengua.

--Oh, por favor--dijo ella--. ¿A quién intentas engañar? La astrología
es un montón de mentiras.

--Sabía que dirías eso--dijo el Doctor--. Típico de una tauro.

Ella levantó las cejas.

--Vamos. No me digas que crees en todo eso\ldots{}--pero el Doctor la
hizo callar mientras Gracilis se levantaba del asiento.

--Dime, ¿dónde puedo encontrar a esta famosa mujer?

Mientras el dueño de la cafetería le daba las indicaciones, el Doctor y
Rose se pusieron en pie, mientras el Doctor se tragaba el último pedazo
de un pastel preparándose para marchar.

Gracilis se giró hacia ellos.

--Amigos míos, estoy muy agradecido con vosotros por vuestra ayuda, y me
encantaría ofreceros mi hospitalidad en mi villa si os pasarais alguna
vez, pero no voy a abusar de vuestra bondad ni un minuto más.

--Debe de estar bromeando--dijo el Doctor--. No vamos a permitir perder
la oportunidad de conocer a una dama que pueda ver el futuro, ¿verdad,
Rose? --y miró a Rose y sonrió.

Ella le devolvió la sonrisa.

--Ni en sueños.