\chapter*{Capítulo Cuatro}
\addcontentsline{toc}{chapter}{Capítulo Cuatro}

Vanessa seguía llorando levemente cuando el Doctor les siguió por la
arboleda. Tampoco le pareció importar a él. Se dejó caer de golpe en la
hierba a su lado.

--Vale, esto es lo que tenemos. Optatus visitó a Ursus en su taller cada
día durante un par de semanas. La impresión general que dio el chico es
que la estatua seguía aún en proceso de planteamiento, pero como no se
le permitía a nadie entrar a ver qué pasaba no tenemos pruebas que lo
respalde. Entonces hace unos pocos días, se levanta temprano, desayuna
unas galletitas de trigo, ñam, ñam, y se dirige al estudio. Y eso es
todo. Nadie le ve salir, lo que no es necesariamente sospechoso, pero lo
que es raro es que Ursus de repente declara que la estatua está casi
terminada, necesitando solo un par de días más para retoques finales.

--¿Y ha sido rápido, entonces? --preguntó Rose, cuya experiencia en esos
temas no se extendía mucho más allá de un jarrón de cerámica de las
clases de Arte de octavo curso.

--Yo diría que sí-- dijo el Doctor--. No necesitarías un modelo todo el
tiempo, pero para la estatua misma yo habría pensado más bien meses,
incluso años. No soy un maestro escultor, por supuesto\ldots{}

Rose sonrió.

--¿Qué dices? ¿Quieres decir que nunca te ha dado clases Miguel Ángel ni
nada?

El Doctor le lanzó una mirada de advertencia y Rose se dio cuenta de que
se había pasado de la raya.

--Perdón--le susurró.

--No pasa nada--le respondió susurrando y entonces puso una mirada
pensativa--. Quizá debería pasarme por el Renacimiento en algún momento.
El David del Doctor suena bastante bien--volvió a levantar la voz--.
Otro aspecto es, que si la estatua estaba a punto de terminar, Ursus
debió de haber estado trabajando en ella, más que haciendo preliminares.
Y el mármol puede ser blanco y pulido pero no es muy limpio trabajar con
ello. Suciedad o polvo, llámalo como quieras. Y aún así nadie vio a
Ursus o a Optatus con la desesperada necesidad de un lavado.

--Así que\ldots{} ¿no estaba haciendo una estatua? --dijo Rose.

Pero el Doctor señaló a la estatua.

--Ahí es donde sí tenemos una prueba--dijo--. Sea como sea, es
imperativo que entremos en su estudio y descubrir sus prácticas de
trabajo.

--Bueno, yo estoy lista para ser su modelo--dijo Rose. El Doctor
asintió.

--Bien, te voy a decir qué vamos a hacer, vamos a ir a echar un vistazo,
¿de acuerdo?

Rose sonrió.

--No puedes parar quieto dos minutos, ¿no es así? --se giró a Vanessa,
que había dejado de llorar pero aún así no parecía ser una caja de
risas--. Mira, ¿quieres quedarte aquí? Si alguien viene, diles que estás
meditando junto a la estatua o algo. Diles que te hemos dicho que lo
hagas.

Vanessa asintió agradecida, así que el Doctor y Rose se marcharon sin
ella.

--¿Qué le ocurre? --le preguntó el Doctor--. ¿Va a pasar algo malo el
próximo miércoles?

Rose le frunció el ceño.

--No puede ver el futuro de verdad, me lo ha dicho. Le he dicho que la
llevaríamos a casa y entonces me ha dicho que no tenía. Y eso me ha
preocupado.

--Quizá su casa esté a mucha, mucha distancia--dijo el Doctor.

--Y mientras tanto está atrapada aquí como esclava.

Él asintió.

--Me gusta tan poco como a ti. Pero algunos esclavos sí llevan vidas
felices, ¿sabes? O bien se les da la libertad, o la compran.

--Eso no está bien--murmuró Rose.

--No--dijo él--, no lo está-- de repente cogió el brazo de Rose y la
empujó tras unos árboles.

--¿Qué? -- articuló con la boca sin hablar.

El Doctor esperó unos momentos y entonces la soltó.

--Ursus--dijo--, acabo de verle.

--¡Lo que significa que no está en su taller! -- se dio cuenta Rose.

--Así que, ¿qué mejor momento para que nosotros podamos hacer una
visitilla?

Corrieron hacia los establos, vigilando sus espaldas en caso de ver
alguna señal de la vuelta de Ursus. El Doctor intentó abrir la puerta
del taller, pero estaba cerrada.

--¿Y el destornillador sónico? --dijo Rose, mirando el cinturón del
Doctor.

--Me lo he dejado en la villa--le respondió el Doctor--. ¿Tú sabes lo
frustrante que es no tener bolsillos? Me siento como si hubiera perdido
una parte de mí. No dejo de intentar meter las manos en ellos y no están
ahí.

--Necesitas uno de esos pequeños zurrones para colgarte del cinturón--
le dijo Rose. Alargó una mano hacia su cabeza. Uno de sus mechones cayó
por su hombro al quitarse una horquilla de plata, la cual le pasó.

--Habrá que tirar de los métodos anticuados, supongo.

Él asintió.

--Estando en Roma\ldots{}

Rose seguía mirando a sus espaldas, pero no le llevó mucho tiempo al
Doctor de abrir el cerrojo y Ursus seguía sin aparecer. Finalmente, la
puerta se abrió. El Doctor irrumpió en el lugar como si éste fuera suyo.

La primera sala estaba más o menos vacía, conteniendo una mesa pequeña
sobre la que descansaba una jarra pequeña, una copa y una hogaza de pan.
Pero cuando la hubieron atravesado y entraron en la siguiente, vieron un
joven muy atractivo sentado en un banco. Pegó un bote cuando entraron e
inclinó la cabeza.

--Oh, hola--dijo Rose.

--¡Hola! --repitió el Doctor--. Soy el Doctor y ésta es Rose. ¿Quién
eres tú?

El joven parecía nervioso.

--Me llamo Tiro. Perdónenme pero, ¿tienen el permiso de mi maestro,
Ursus, para estar aquí? No permite a nadie entrar en su taller.

--Oh, no le importa. Nos dijo que nos pasáramos. Fíjate si fue así, que
dejó la puerta abierta a propósito. Quiero decir, no es muy amigo de
dejársela abierta, ¿verdad?

Tiro negó con la cabeza.

--Ahí lo tienes, pues.

Rose sonrió.

--Queríamos ver cómo iba todo. Voy a posar para él--hizo una pirueta--.
La próxima \emph{top model}, esa soy yo-- miró a Tiro, apreciando su
figura atlética y esbelta, sus rasgos perfectos y su pelo suavemente
ondulado--. Déjame adivinar, tú estás aquí para posar también, ¿verdad?

Tiro asintió.

--Mi maestro me compró para ese propósito, sí.

Rose reprimió una mueca.

--No creo que me vaya a acostumbrar nunca a eso -- le restó importancia
con un gesto ante la extrañada mirada de Tiro--. No importa. Así que,
¿cómo llevas eso de ser modelo?

Él le sonrió.

--No lo sé. Aún no he comenzado. Estoy un poco nervioso.

--Sé a lo que te refieres--Rose coincidió con él.

--Luchar contra monstruos, explorar lunas y vencer al mal, pero no hay
nada comparado con quedarse quieto unas pocas horas mientras un tipo le
pega golpes a un bloque con un cincel--se burló el Doctor.

Rose le dijo a Tiro que le ignorara.

Se apartó para contemplar una montaña de cachivaches en una esquina: una
lanza, un arco, un cuerno, un gorro con alitas enganchadas.

--Todo lo que un dios o una diosa pudiera necesitar--dijo; cogió el
cuerno e hizo una pose--. ¿Qué pensáis?

--Estoy seguro que me traerás suerte--dijo el Doctor. Caminó hasta
situarse detrás de ella y comenzó a abrir puertas, asomándose--. Qué
raro\ldots{} --comentó.

--¿Qué es raro? --dijo Rose, poniendo los ojos en blanco. El Doctor
encontraba rarezas en todas partes.

Éste frunció el ceño.

--Bueno, se supone que Ursus va a comenzar dos estatuas nuevas: una tuya
y la otra de Tiro. Pero aquí no hay ni una pizca de piedra en ninguna
parte.

Rose se encogió de hombros.

--Quizá aún no la tenga. Probablemente en Roma no hagan repartos de un
día para otro.

Las orejas del Doctor se pusieron de punta.

--Ajá, eso suena a los pasos osunos del hombre mismo. Le preguntaré.

La puerta se abrió y Ursus entró. Su cara se horripiló al ver al Doctor
y a Rose.

--Creí haberos informado que no se permite a nadie entrar en mi
taller--gritó, y se giró para mirar a Tiro, que se encogió con nervios.

El Doctor dio un paso adelante.

--No culpes al chico--dijo--. y tampoco nos culpes a nosotros. Por
desgracia, olvidaste decirle a Rose a qué hora requerías de su presencia
para posar así que, siendo unos tipos amables y agradables, simplemente
nos hemos pasado para averiguar cuando sería conveniente para ti. Eso es
todo.

Ursus pareció relajarse, aunque no lo suficiente para el gusto de Rose.
Tenía la inquieta impresión de que era un hombre peligroso con el que
cruzarse.

--Ven a la tercera hora tras el atardecer--le gruñó, y ésta tuvo que
reprimir la necesidad de saludarle militarmente con sarcasmo.

--¿Pasa algo si el Doctor viene conmigo y mira? --preguntó, sabiendo la
respuesta.

--¡No! --explotó el escultor--. ¡No, no está nada bien! ¡Nadie puede
verme trabajar!

--Sólo he preguntado--dijo Rose.

--Me ha parecido ver que no tienes piedra preparada para tus
esculturas--dijo el Doctor, ignorando a propósito el apabullante deseo
de Ursus de que se fueran inmediatamente--. Resulta que conozco unos
cuantos mercaderes, los mejores marmolistas del mercado, yo\ldots{}

--Está de camino--ladró Ursus, antes de que el Doctor pudiera alargar
más su mentira.

--Vale, vale, no hace falta que me agradezcas por mi tan amable
oferta--dijo el Doctor.

--¿Tu oferta? Dices conocer a mercaderes de mármol. Irrumpes en mi
estudio. Eres un rival, ¡vienes a robarme las ideas!

--No, no lo soy--dijo el Doctor con indignación.

--No es un maestro escultor--añadió Rose--. Eso ha dicho hace menos de
media hora.

--Eso es--dijo el Doctor--. Me alegro de que haya quedado claro-- se
giró y comenzó a examinar una mesa llena de herramientas de esculpir.

--Estás acabando con mi paciencia--dijo el escultor.

--Oh, lo lamento--dijo el Doctor, aún inclinado hacia la mesa y sin
hacer el mínimo esfuerzo de dejar de hacer lo que estaba haciendo.
Levantó un cincel--. ¿Sabes qué? Haces un verdadero esfuerzo en que tus
instrumentos estén en perfectas condiciones. Nadie diría que ninguno de
estos se ha usado antes.

--Soy un artesano muy cuidadoso--dijo Ursus.

--Ya te digo que lo eres--dijo el Doctor, sonando poco impresionado--.
Incluso tu taller está impoluto. No hay ni rastro de polvo de mármol en
ningún lugar. Aún así, supongo que tendrás esclavos para que te lo
limpien por ti.

--No se le permite a nadie la entrada a mi taller--reiteró Ursus--.
Nadie excepto mis sujetos. Gracilis no entra aquí. Su mujer no entra
aquí. Ni siquiera sus esclavos entran aquí. Todos respetan mi necesidad
de privacidad para crear arte. La única persona que no lo respeta eres
tú.

El Doctor parecía sorprendido.

--¿Qué te ha dado esa idea? Nos vamos en un minuto, ¿verdad, Rose?

--Ahora mismo--coincidió ella, maniobrando una despedida con la mano
hacia Tiro, que le sonrió.

--Hablando de ser un artesano cuidadoso--dijo el Doctor, mientras se
arrastraron lentamente hacia la puerta--. Supongo que es por eso por lo
que llevas esos guantes, para protegerte las manos. Un tanto pionero
aquí. No creo que haya visto a nadie más\ldots{}

Pero ya habían cruzado el umbral entonces y les cerraron la puerta de un
golpe en sus narices. Una llave se giró con fuerza en el cerrojo.

--Obviamente no quiere compartir sus consejos de moda--dijo el Doctor.
Miró hacia el sol, midiendo su posición en el cielo con lo que supuso
Rose que era un ojo experto--. Vamos. Creo que es la hora de cenar.

Rose quería intercambiar una palabra con Marcia antes de la cena, pero
no lo quiso mucho rato. Sonriendo ante su éxito, corrió a su habitación
para prepararse. Estaba ligeramente nerviosa por la cena, pero después
de vivir con su madre durante la mayor parte de su vida supuso que
podría aguantar con literalmente todo en la pirámide alimenticia.

Cuando finalmente fue a cenar, la comida no fue lo único con lo que tuvo
que aguantar. Primero un esclavo le lavó los pies, algo que era muy raro
pues nadie le había lavado los pies desde que su madre le había quitado
la arena en Tenby con un cubo de agua de mar mientras estaba sentada en
la silla del despacho comiéndose un helado. Entonces la llevaron a lo
que parecía una cama, y adivinó que se suponía que tumbarse boca abajo,
ya que era lo que Marcia y Gracilis estaban haciendo. El Doctor, sin
embargo, se apoyó a sí mismo por un lado, descansando en un codo, y Rose
pensó que aquello parecía una posición mucho más fácil para comer así
que le copió a él en su lugar.

Tenía la idea de que las comidas romanas eran lirones y flamencos, más
de una tienda de mascotas que de una pizzería. Pero no era así del todo.
A Rose le sirvieron una carne sin identificar bañada con una salsa que
despedía un fuerte olor, con unas ordinarias y reconocibles verduras
como unas judías y unos espárragos. Miró a su alrededor buscando
cubiertos, pero no se veían en ningún lugar.

El Doctor supuso lo que estaba haciendo y sonrió.

--Aún no han inventado los tenedores y nadie se molesta en usar un
cuchillo--le dijo, cogiendo un trozo de comida con los dedos y
metiéndoselo en la boca.

Ella hizo una mueca.

--¡Asqueroso! Está todo cubierto de salsa.

--\emph{Garum}--dijo el Doctor--. Tripas de pescado fermentadas durante
un par de meses y obtienes algo lo bastante fuerte como para tapar el
sabor de carne podrida. Ya ves, no tienen congeladores.

Rose pensó que quizá estaría enferma.

--Creo que me quedaré sólo con las verduras--dijo--. ¿Cuándo inventan la
pizza?

--Si quieres una Cuatro Estaciones o una Vegetal, te puedes esperar unos
pocos siglos--le dijo el Doctor.

--¿Qué tal si las pido ahora? Les doy todo el tiempo del mundo.

El Doctor sonrió.

--Oh, y se considera educado eructar después de comer--añadió.

--Entonces van a pesar que soy muy maleducada--anunció Rose firmemente.

Después de que hubieran acabado, un esclavo vino para lavarles las
manos, mientras otro esclavo les llevaba fruta y nueces y pequeños
pasteles de miel, los cuales estaban más hechos al paladar de Rose.

A Vanessa no se la vio por ninguna parte durante la cena.

Rose se preguntó si aún estaría fuera, sentada en la oscuridad cerca de
la estatua de Optatus, o si estaría en algún lugar de la casa con el
resto de los esclavos. Pensó que sería mejor buscar a la niña tras la
cena. Dejando al Doctor en una conversación con Marcia y Gracilis,
escuchando todo tipo de cotilleos aburridos que esperaban que el Doctor
conociera, se escabulló al exterior de la casa y hacia el jardín.
Seguramente, Vanessa seguiría allí.

Rose se sentó detrás de ella en la oscuridad y se sacó una hogaza de pan
que se había apañado para esconderle.

--Ten--dijo--. Me preocupaba que no tuvieras nada que comer. Lo siento
porque no es mucho.

Vanessa le sonrió, agradecida.

--Eres muy amable.

Rose se encogió de hombros.

--De nada. Mira, ¿por qué no vienes adentro?

--No sé qué hacer--dijo Vanessa, desesperada--. Nunca he estado en una
casa como esta antes. No sé cómo se supone que me tengo que comportar.
Castigan a los esclavos que no hacen lo correcto, estoy segura.

--Pero tú estás aquí como una astróloga, no como una esclava
normal--dijo Rose, intentando animarla--. No te castigarán, no les
dejaré. Mira, ¿qué tal si hacemos una cosa? Le decimos que necesitas
entrar en comunión con las estrellas en una meditación, o algo, que es
una parte esencial de tus rituales de astróloga. Te dejarán entonces
hacer cualquier cosa. Creerán que les vas a devolver a su hijo. No te
castigarán.

--Excepto cuando no consigan a su hijo y se den cuenta de que les he
engañado.

Rose le lanzó una mirada de ofensa.

--¡Ey! Te he dicho que el Doctor y yo estamos en el caso. ¡Le vamos a
encontrar! --suspiró--. Pero primero, tengo que posar para esa estatua.
Se supone que tengo que estar ahí tres horas después del amanecer.
¿Cuándo demonios es eso? ¿Se supone que tengo que sentarme a mirar el
sol y entonces comenzar ``un hipopótamo, dos hipopótamos'' durante tres
horas?

Vanessa sonrió finalmente.

--Te despertaré--dijo ella--. No duermo mucho, ya no lo hago.

--Hay trato--dijo Rose--. Mira, hay algo que quiero hacer antes de que
la luz se vaya por completo, así que será mejor que vuelva.

Pero no se movió. El sol estaba bajando y no podía apartar los ojos de
él.

Todos aquellos dramas a su alrededor. Gracilis y Marcia, desesperados
por el retorno de su hijo. Ursus, con su lujuria por la fama artística.
Las preocupaciones y miedos de Vanessa. Los esclavos, ¿quién sabe qué
estarán deseando y soñando? Y aún así, en dos mil años, todos serían
olvidados. Las cosas que eran vida y muerte en el presente no significan
nada ni siquiera para la próxima generación, imagínate a aquellos que
viven en el siglo XXI. Cuando ella naciera, la gente de allí sería polvo
y la villa ruinas. La única cosa que sobreviviría sería la estatua de
una diosa, ¿y quién sabe qué habrá sufrido hasta entonces?

Para el sol poniente, el tiempo entre el que ahora estaba Rose y del que
había venido no había más que un parpadeo. Pero para Rose, que había
estado desde los albores de la humanidad hasta el mismo final de la
Tierra, de repente pareció una eternidad.

Vanessa se acabó su hogaza de pan.

--Vamos, pues--dijo Rose, y se levantó.

Caminaron juntas hacia la casa y no miraron atrás.
