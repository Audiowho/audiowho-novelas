\chapter*{Capítulo Diecisiete}
\addcontentsline{toc}{chapter}{Capítulo Diecisiete}

Rose estaba impactada.

--¿Por qué te tendría miedo? ¿Ha pensado que podrías aplastarle como a
tantos Slitheen?

El Doctor asintió.

--Quizá. Creo que miró dentro de mi cabeza, sólo un poco, cuando me di
cuenta de qué era. Un GENIO, del año 2375, creado por Salvatorio
Moretti.

--Mi padre--añadió Vanessa.

El Doctor asintió apreciando su comentario.

--Los GENIOS se suponían que iban a ser una ayuda--siguió el Doctor--.
Una gran ayuda, una brillante ayuda. Esta era una sociedad dónde todo
estaba disponible y cuyos ciudadanos llegaron a esperar que podrían
tener una cosa en cuanto la pensaran. Los GENIOs se supone que lo
facilitaban todo. No había más paseos hasta las tiendas, dile al GENIO
lo que quieres. ¿Te apetecen unas vacaciones? No esperes más, estarás
allí antes de que parpadees. ¿Envidias el hovermóbil de tu vecino?
Puedes tener uno igual.

Se detuvo e hizo una sonrisa triste.

--Los humanos siempre la fastidian con las cosas, ¿no es así? Solo con
los prototipos de los GENIOs, las cosas comenzaron a desmadrarse. La
gente desearía que también tuvieran un GENIO y\ldots{} ¡pop! Había uno
para ellos. Se extendieron por todo el planeta como unos pequeños
conejitos escamosos. Los inventores no tenían ni idea de lo poderosos
que eran, porque se habían equivocado al no tener en cuenta la I de IA,
que los GENIOS eran inteligentes. Podían pensar por sí mismos, descubrir
cómo absorber sistemas energéticos para conceder deseos cada vez más y
más grandes. Y también se equivocaron al no tener en cuenta la
naturaleza humana. ¿Envidias el hovermóbil de tu vecino? Bueno, ¿por qué
no deseas que sea tuyo, y que tu vecino caiga en la pobreza y se vea
obligado a envidiarte? ¿Te apetecen unas vacaciones? ¿Por qué no desear
que el sol siempre brille, conseguirás un bronceado, y a quién le
importa que el planeta se seque lentamente? ¿Por qué no desear que tus
enemigos se debiliten y que tu molesta esposa pierda de verdad la
lengua? Humanos, nunca satisfechos, vengativos, siempre poniendo el
placer por encima de las necesidades del futuro.

--Pero te seguimos gustando, ¿no? --dijo Rose.

--Os quiero--dijo el Doctor, sonriéndole--. Pero a veces liáis un poco
las cosas. La mayor parte del tiempo, de hecho.

--¿Y qué pasó luego? --preguntó Rose--. Con la Tierra y los GENIOS y
esas cosas.

El Doctor la miró con la cara pálida.

--No tengo ni la más remota idea.

Rose frunció el ceño.

--¿Qué?

Alzó las cejas.

--Lo siento. ¿De qué estábamos hablando?

--¿Doctor? -- ¿estaba bromeando o algo? --. Esto es algo, como, ¿un
trauma de regeneración de nuevo? ¿Me estás tomando el pelo? ¿O alguien
ha deseado algo?

El Doctor se encogió de hombros.

--Lo siento. No tengo ni idea de lo que hablas.

--¡Los GENIOs destruyendo el mundo!

La miró, entonces al GENIO que se escondía en su caja de cartón. De
nuevo a Rose, de nuevo al GENIO. Entonces se apretó el puño contra su
cabeza y la zarandeó con fuerza.

--Lo siento. Gajes del oficio de un Señor del Tiempo. ¿Dónde estábamos?
--respiró hondo--. Gracias a los GENIOs, no eso está mal, gracias a la
gente que usaba a los GENIOs, la Tierra estuvo al borde de la
destrucción. No podía haber estabilidad, porque un deseo podía cambiarlo
todo. Se aprueba una ley en contra de los deseos y otra persona podrá
desear que se cancele. Se arregla el planeta, y la siguiente persona
deseará que se destruya de nuevo. Y la energía que se estaba usando para
conceder todos los deseos, ni te la imaginarías.

--Yo creo que sí--dijo Rose, recordando cómo la criatura había absorbido
el cuerpo de Ursus.

--Hubo algunos seguros construidos dentro de los GENIOs. No puedes
desear que nadie muera o que no exista, por ejemplo, y eso incluía a los
mismos GENIOS. Pensaron intentar desear que los GENIOs nunca hubieran
sido creados en primer lugar, pero no pudieron hacerlo porque crearían
una poderosa y realmente implosionante paradoja, ¿quién habría concedido
el deseo?

--¡Yo lo pensé! --dijo Rose de repente--. Eso es lo que le dije a
nuestro GENIO.

--A nuestro GENIO, de hecho--salió un murmullo indignado del interior de
la caja de cartón. Pero Rose pudo adivinar que estaba prestando mucha
atención a todo lo que decía el Doctor.

--Hubo varias formas a su alrededor. Alguien ideó un plan. Encontraron
al GENIO más anterior que pudieron, un GENIO creado en mayo de 2375.

Vanessa abrió su boca, pero el Doctor levantó una mano para hacerla
callar.

--Lo sé--dijo--. Pero era el primero que habían hecho. Y sabían que la
criatura necesitaría una tremenda cantidad de energía para hacer lo que
tenía que hacerse y le conectaron con el Sol.

--¿Qué hicieron qué? --dijo Rose.

Pero el Doctor no detuvo su historia.

--Y así que lo desearon, desearon volver al día en el que aquel GENIO
había sido creado. En mayo de 235. Y con la enorme cantidad de energía a
su disposición, el GENIO concedió el deseo.

--¿Así que todo los GENIOS posteriores nunca habrían existido, sólo el
que había concedido el deseo? --preguntó Rose.

El Doctor asintió.

--Sí. Es un poco de hacer trampa con la realidad, porque es imposible
cambiar la naturaleza de las cosas sin crear una paradoja, pero era
mejor que las alternativas. Pero, como sabían que pasaría, como lo
planearon, incluso un GENIO no pudo soportar absorber el poder del Sol.
Al forzar a la pobre criatura a cometer genocidio en su propia especia,
también le habían hecho cometer suicidio. Cada átomo de aquel GENIO
ardió, y el incendio resultante destruyó el Bureau Tygon. El laboratorio
donde los GENIOs se crearon en primer lugar, cada pedazo de
investigación, todo ceniza y humo.

Hubo un sonido de la caja de cartón.

--¿Qué día dejaste tu casa, Vanessa? --preguntó el Doctor.

--Era el 17 de abril de 2735--dijo.

--Y es por eso por lo que tu GENIO sigue existiendo--le dijo--. Porque
es el primero.

--Pero si me lleva de vuelta a casa\ldots{}--dijo.

El Doctor se encogió de hombros.

--Entonces la Tierra será destruida.

Llevó un momento para que calara hondo. Rose había intentando entender
las cosas, pero eran demasiado giros y vueltas y no se le estaba dando
bien.

--Así que el GENIO es el único que existe--dijo ella--. Porque llevó a
su planeta a un tiempo antes de que fuera creado, perdón se lo hicieron
hacer. Pero espera un minuto, si desearon volver al principio, entonces
nada de las cosas malas destructoras de planetas nunca pasó. Así que no
hay garantía de que la Tierra se destruya esta vez, aunque tuvieran de
nuevo al GENIO.

--Está la naturaleza humana--dijo el Doctor, y Rose no pudo discutir con
él--. Sé qué pasaría si este GENIO volviera. Y no pasa. Así que no puedo
dejar que el GENIO vuelva. El tiempo tiene que estar de vuelta por su
camino correcto-- sonrió--. Superpoderes del Señor del Tiempo.

--¿En serio?

--Bueno, más o menos. El tiempo es, digámoslo de la forma más
impresionante y algunos dirían egocéntrica, mi dominio. Puedo ver cosas
que una vez pasaron, aunque nunca volvieran a pasar. Bueno, si me
concentro. La nueva realidad, la realidad real, mantiene
estableciéndose, incluso conmigo. Pero las otras líneas temporales dejan
ecos, huellas, si observo el rato suficiente. Por ejemplo, hay algo
interesante: adivina de dónde obtuvo el GENIO del futuro su nombre.

--¿El inventor era un gran fan de una pantomima que protagonizaron
actores y actrices de culebrones australianos con pantalones de harén?

--Casi. Las fantasías de las \emph{Mil y una noches} y todo eso. Hubo un
poco de \emph{revival} arábigo en marcha, todo el mundo tenía alfombras
persas y los turbantes eran la última moda, ¿verdad, Vanessa?

Vanessa asintió.

--Nunca he ido muy a la moda--dijo.

--Así, ¿qué crees que inspiró a los genios que inspiraron a los GENIOs?

Rose lo entendió.

--¡Los GENIOs que habían viajado atrás en el tiempo con sus dueños a los
días de las \emph{Mil y una noches}!

El Doctor se dio un golpecito en la nariz.

--¡Premio para la dama!

Vanessa frunció el ceño.

--Pero nada de eso pasó nunca, así que los GENIOs nunca fueron atrás en
el tiempo para convertirse en\ldots{} genios.

--Cierto--dijo el Doctor--. Pero el mágico Este, lleno de impactantes
místicas y hombres sabios y tales que podían percibir cosas. Ecos y
trazos. Nadie recordaba que los genios fueran reales, pues nunca lo
habían sido. Pero dejaron una huella.

--Cielos--dijo Rose--. Ey, ¿todo esto son historias basadas en líneas
temporales desaparecidas?

--Oh sí--le dijo--. Los elfos, las hadas, los gnomos, los Mumin,
Chorlton y los caballitos en las motos, Bob Esponja, todos intentaron
invadiros en algún momento. Hubo un revuelto galáctico cuando
\emph{Robocop} salió. Y por lo que respecta a los cinco famosos
justicieros del futuro que se disfrazaron de cuatro niños y un perro
(aunque creo que el perro era un error) para evitar que los criminales
secuestraran y robaran durante toda la eternidad, bueno, creo que siguen
atrapados en un bucle temporal con nada más que cerveza de jengibre y
sándwiches de carne enlatada para sobrevivir. Sin mencionar a la
señorita Marple, la señorita Marciana diría yo. Usaba su rayo de la
verdad para conseguir todas esas confesiones hasta que la Policía
Temporal la encontró. Hizo que ella y todo St Mary Mead dejaran de
existir. Lo cual es una lástima porque había una pequeña cafetería
encantadora en la calle principal dónde hacían unas tartas de vainilla
brillantes.

--¿Eso es cierto? --dijo Vanessa, conteniendo el aliento.

--No--dijo Rose--. Aprendes a ignorarle una de cada cinco palabras que
dice. Quiero decir, estaba pretendiendo ser Poirot antes. Está de ese
humor.

El Doctor arrugó la nariz mirándola.

--De todas formas--siguió ella--, ¿qué vamos a hacer ahora?

Nadie respondió por un momento. Entonces Vanessa dijo:

--No puedo volver a casa.

--¿Por qué no? --preguntó Rose.

--¡Ya has oído lo que ha dicho el Doctor! ¡Estaré condenando a la Tierra
a muerte!

--Tú no--dijo el Doctor--. El GENIO. Es el GENIO el que no puede volver.

--Quieres decir que yo puedo\ldots{}--se detuvo, mirando a la pequeña
criatura escamosa con sus orejas de perro puntiagudas desde la caja de
cartón--. ¿Pero cómo? Y el GENIO no se puede quedar aquí. Mira el
problema que ha causado.

Toda la arrogancia desapareció del GENIO.

--Por favor--dijo con tristeza--, por favor haced otro deseo. Desead que
sea borrado de la existencia. Si no sirvo a ningún propósito\ldots{}

--Pero no podemos hacer eso--le dijo Rose--. Construyeron un mecanismo a
prueba de errores. Lo ha dicho el Doctor.

--Y aunque pudiéramos, no querríamos--dijo el Doctor rápidamente--. ¿No
sirves a ningún propósito? ¡Le das a la gente sus mayores deseos! ¡Todo
lo que tenemos que hacer es encontrar gente cuyos deseos sean\ldots{}
menos destructivos!

--¿Pero cómo? --dijo Vanessa de nuevo.

Rose interrumpió.

--De la misma manera que te llevaremos a casa--dijo--. Con nuestra útil
máquina del tiempo.

--Y hablando de tiempo--el Doctor miró hacia el cielo, juzgando la
posición del Sol--. ¿Qué día es?

--Eh\ldots{} ¿viernes? --dijo Rose, insegura.

--Quiero decir la fecha. ¿Cuánto he sido piedra?

Rose pensó.

--Es el día después de Ursus me hiciera lo que ya sabes.

--Entonces es 19. El Quinquatrus. Eso significa que estoy llegando de
Roma ahora mismo--frunció el ceño--. Tengo que tener cuidado. Podría ser
catastrófico si me encuentro conmigo mismo.

--¿Algo catastrófico? --comentó Rose--. Eso sería toda una novedad.
¿Volvemos a Roma, entonces? --¿No es ahí dónde está la TARDIS?

--No. Bueno, sí. Ambas. Pero para evitar no destruir las líneas
temporales, sin embargo, la TARDIS que queremos está en el exterior de
la villa de Gracilis. Pero tenemos que llegar a Roma en las próximas,
oh, ocho horas.

--¿Por qué?

No respondió directamente.

--¿Tienes el frasco de líquido? --preguntó.

--Sí--se lo mostró. Aún estaba lleno--. Pero no lo necesitamos más, ¿no?
Todo el mundo está listo, y Ursus está muerto.

--``Todo el mundo está listo'', ¿eh? ¿Qué pasa con Optatus y con todas
las demás víctimas?

--Pero dijiste que te habías encargado de ellas.

El Doctor se inclinó, resaltando sus palabras.

--No me ``he encargado'' de ellas aún. De hecho, ni siquiera tengo la
cura milagrosa--indicó al pequeño contenedor de líquido--. Se me dará un
frasco lleno de este líquido en unas ocho horas.

Entonces Rose lo entendió.

--Oh. Claro. ¿Cuánto tardamos en llegar hasta Roma?

--Unas veinte horas.

--Y si no llegamos allí en ocho horas, toda la causalidad implosionará o
algo.

--Pero--añadió Vanessa--, ¿no acabáis de decir que teníais una máquina
del tiempo en la villa?

--Oh, sí--dijo Rose--. Eso es cierto.

La TARDIS se materializó en una alcoba en la parte trasera del altar de
Fortuna.

--Aquí estamos--dijo el Doctor--. Es el 19 de marzo, 120 dC, sobre las
seis de la tarde.

Rose frunció el ceño.

--¡Pero vas a estar ahí fuera en un minuto! Dijiste que sería
catastrófico encontrarte contigo mismo.

--Oh, la explosión que destruiría el planeta pronto se olvidaría cuando
el universo se partiera a pedazos si no conseguimos dar el frasco a
tiempo--le dijo el Doctor--. Pero para evitar la otra posibilidad, me
voy a quedar aquí y tú vas a ir ahí fuera y vas a hacer de la diosa
Fortuna--sonrió--. Creo que debo de haber tenido la idea original de
este amigo tuyo haciéndose pasar por Minerva. Justo ahora me daré la
idea a mí mismo, lo que hace todo un poco demasiado complicado como para
preocuparse de ello.

Rose observó el escáner.

--¿Por qué algunas de las pequeñas modelos tienen una tela en los ojos?
¿Se supone que todos sus adoradores son feos o algo?

--La fortuna es ciega. No juzga a quien se merece los favores,
simplemente los reparte al azar. Como una novia tirando su ramo. --el
Doctor hizo una mueca--. Una vez cogí un ramo y casi acabo casándome con
un elefante.

--No era de buen ver, ¿verdad? --preguntó Rose.

--No, un elefante de verdad, la mascota preferida del emperador Golibo.
¿Puedes imaginarte el trozo de ``Puede besar a la novia''? ¡Esas
trompas!

--¿Y qué pasó?

--Por suerte mi prometida se comió el ramo, invalidando el contrato. Y
lo que hice fue técnicamente una ``huida''. Ahora, ¿podemos ponernos en
serio?

Pero Rose estaba riéndose a carcajadas:

--¿Ella\ldots{} ella\ldots{}?

--¿Qué? ¡Ja, ja, ja! El Doctor casi se casa con un elefante\ldots{}
¿Nunca has estado a punto de casarte con alguien con quien no debías?
Pongámonos manos a la obra.

--¿Ella\ldots{} ella\ldots{}?

--¡Rose!

Rose casi explotó.

--¿Ella había ya preparado su trompa para la noche de bodas? --volvió a
doblegarse de la risa.

El Doctor la observó, de brazos cruzados, mirándola con una expresión
pétrea que le hizo reír aún más.

--¿Ya te has cansado?

Asintió, aún riendo por debajo de su nariz.

--¿Entonces podemos ponernos con lo de evitar que el tiempo y el espacio
se partan en pedazos? ¿Podemos? Entonces pongámonos manos a la obra.

Rose se recompuso y levantó sus manos pidiendo perdón.

--Vale. Eres Fortuna, estás escondida detrás de la estatua, y nunca,
repito, nunca me dejarás que te vea; y esperarás hasta que Gracilis haya
cogido el frasco y se haya ido antes de irte tú. Oh, toma esto,
distorsionará tu voz. ¿Vale? --le pasó un pequeño aparato metálico y le
empujó a la puerta--. ¡Vamos, vamos, vamos!

--Espera un momento--dijo Rose, intentando frenar--. ¿No puedo ensayar o
algo?

El Doctor miró al escáner.

--¡No hay tiempo! Voy a estar ahí en cualquier segundo. Ah, y
Rose\ldots{}

Ella se giró.

--¿Sí?

--Sí, me temo que sí. Había preparado su trompa. Y le había dicho adiós
al circo--le dedicó una amplia sonrisa y le empujó por la puerta.

Rose se encontró tambaleándose en el altar, con las puertas de la TARDIS
cerrándose tras de sí. La máquina del tiempo desapareció entre las
sombras cuando se dirigía hacia la estatua que el Doctor le había
indicado. Tuvo que apretujarse para meterse detrás y sólo esperaba que
la oscura luz la camuflara también, sintió que se encajaba en todas
partes.

Se acababa de acomodar y se había ajustado el aparato a la boca, cuando
la puerta del altar se abrió. Echó un vistazo a través de las piernas de
Fortuna y vio que sí, era el Doctor. Él vio la estatua. Ella se hundió
cuando él se apresuró hacia ella\ldots{} y entonces él se dio cuenta de
que no era ella.

Rose se sobrecogió. No lo había sabido, ¿cómo podía haberlo sabido? Lo
que su desaparición le había hecho. Aquel Doctor tenía una mirada con
tanta soledad en sus ojos que su corazón casi se paró de lástima. No
quería nada más que saltar, ir hacia él, decirle que todo iba a ir bien.

Pero, al poder dividirse el tiempo y el espacio, aquello probablemente
era una mala idea.

--Rose es más guapa que tú--dijo de repente el Doctor.

--¡Gracias! --dijo ella, antes de poderse detener.

Se mordió la lengua. Rápido, mejor seguir adelante con el resto antes de
que sospechara demasiado. La caja cubriendo su boca le hacía sonar más a
Cher que a Rose, pero puso su mejor voz de ``diosa'' y dijo:

--Esto traerá a Rose de vuelta a la vida, y al resto. ¡Adoradme todos,
pues soy Fortuna\ldots{} --y aclaró vagamente--, y todo eso!

Se agachó todo lo que pudo y con cuidado, con mucho cuidado, mandó el
pequeño frasco de cristal rodando hacia el Doctor, y su pasado.

El Doctor lo recogió y comenzó a ir hacia ella.

Ella se tensó, de repente dándose cuenta de que quizá podría descubrirla
después de todo, pero justo como el Doctor había descrito, fue
interrumpido por Gracilis.

Rose no pudo soportar ver la captura del Doctor, aunque sabía que
tendría un final feliz. Se obligó a mirar el momento en el que dejaba
caer el frasco. Aquello era importante. Ahora tenía que esperar que
Gracilis lo cogiera\ldots{}

Gracilis alzó las manos con desesperación.

--¿Qué debo hacer? ¿Qué debo hacer? --le oyó murmurarse--. Debo
encontrar a alguien que pueda ayudar.

El anciano caminó hacia la salida. En cualquier momento\ldots{}

Gracilis pasó el frasco. Rose esperó que lo viera y se detuviera, pero
no lo hizo.

Abrió las puertas y salió al exterior.

Rose sintió una sensación de mareo en su estómago, y no supo si era
miedo o si la historia acababa de cambiar e iba a ser borrada de la
existencia. Si Gracilis nunca encontraba el frasco\ldots{}

Y entonces tuvo una idea. Si era sensata o no, no tuvo tiempo para
decidirlo, probablemente no lo fue. Pero lo dijo, esperando que el GENIO
estuviera bastante cerca como para oírla:

--Deseo que Gracilis venga ahora y encuentre el frasco.

Hubo un trueno en su cabeza. Y Gracilis entró de nuevo por la puerta.
Negaba con la cabeza, frunciendo el ceño como si intentara localizar
algo. Miró al suelo. ¡Ajá! Recogió el frasco del líquido que devolvía a
la vida y se lo puso en un bolsillo de la cintura.

Rose dejó escapar un suspiro de alivio.
