\chapter*{Capítulo Seis}
\addcontentsline{toc}{chapter}{Capítulo Seis}

El Doctor levantó sus manos hacia el cielo.

--¡No dispares! ¡Por favor, no dispares! ¡Te lo ruego!

Vanessa puso una expresión confusa cuando éste se lanzó a sus rodillas
ante ella.

--¡Por piedad, no dispares! --gritó él.

Su mano tembló, y el Doctor la alargó y le arrebató el aparato.

--Gracias--dijo él--. Confunde al enemigo, ese es el truco. No es que
esté diciendo que seas necesariamente una enemiga. Quiero decir, no creo
de verdad que me quisieras disparar--miró hacia abajo, al aparato en su
mano--. Especialmente no con el mando a distancia de un vid-caster.

Le lanzó una ligera sonrisa temblorosa.

--Es lo único que tengo de casa. Lo tenía en las manos
cuando\ldots{}--dejó de hablar mientras una lágrima le caía por la
mejilla.

El Doctor se puso en pie y Vanessa se apartó de él. Éste se sentó en el
heno a su lado y puso un brazo amable encima de su hombro.

--Verás, de hecho sé que vienes del siglo XXIV--dijo, sujetando el mando
a distancia--. Esto lo prueba. Así que, ¿me vas a hablar de ello?

Ella negó con la cabeza y el Doctor percibió cómo se tensaba. Dejó caer
los brazos a ambos lados.

--Mira, siento haberte gritado--siguió--. Sólo estoy preocupado por
Rose. Después de todo, sospecho que sea lo que sea que le haya pasado es
lo que le pasó a Optatus la semana pasada, y tú no has tenido nada que
ver en ello, bueno, a no ser que formes parte de un plan muy elaborado
con muchos cómplices. Si esto fuera una historia de detectives,
probablemente tendría que considerarlo, pero no lo es, y mientras no
tenga mi bigote retorcido, puedo reconocer una persona genuinamente
asustada cuando la veo. Y tú estabas muy asustada en aquella casa, ¿no
es así?

Ella asintió, pero la tensión seguía siendo evidente en su cuerpo. Por
lo que pareció la primera vez, comenzó a hablar, hablar de verdad como
una persona humana y no como una oveja asustada.

--He estado asustada durante tanto tiempo\ldots{} Tenía tanto miedo y
estaba tan confusa cuando llegué aquí. Había un tipo de festival en
marcha y había gente por todas partes. Balbus me rescató, me dio comida
y bebida. Pero dije algo, ni siquiera recuerdo qué, y la fastidié. Llegó
a la conclusión de que podía predecir el futuro.

--Lo que puedes hacer--dijo el Doctor--. Y con bastante exactitud.

--Bueno, sí. Y entonces me acusó de ser una esclava fugitiva y me dijo
que me ejecutarían, a no ser que trabajara para él. Y así tuve que
hacerlo. Me esforcé de lo lindo. He leído horóscopos en las revistas y
he estudiado astronomía. Creía que me mantendría a salvo, hasta\ldots{}

El Doctor la miró, analizándola.

--¿Hasta qué?

Ella hizo una sonrisa a medias.

--Hasta que pudiera volver a casa--la sonrisa desapareció--. Pero era
horrible. No sé muy bien de historia. Alerté a la gente de alejarse de
Pompeya, pero al parecer eso pasó hace años. Y me preocupaba que si
decía una palabra sobre emperadores futuros me ejecutasen por traición.
Y entonces había gente como tu amigo Gracilis, que desesperadamente
necesitaban esperanza y ayuda, y yo les daba mentiras, alimentando sus
miserias. Y a todo esto, Balbus me observaba, enriqueciéndose cada vez
más de las cosas que yo decía.

--El provecho del profeta--citó el Doctor.

Ella volvió a sonreír.

--Y entonces tú y Rose llegasteis. Supe a primera vista que no erais
romanos ordinarios.

El Doctor se encogió de hombros con modestia.

--Bueno, debo admitir que hago que las cabezas se giren a mi paso allí
donde vaya. Es una carga con la que tengo que vivir.

--Y tuve miedo de nuevo, de quienes pudierais ser, pero estaba tan
desesperada de alejarme de Balbus\ldots{} Y entonces Rose habló conmigo,
y parecía tan amable, pero no podía permitirme decirle nada, solo por si
acaso\ldots{}--miró al Doctor con ojos solícitos--. ¿Puedes llevarme a
casa? -- dijo--. ¿Por favor?

--¿Dónde está tu casa? --le preguntó el Doctor--. ¿Dónde y cuándo?

Vanessa respiró hondo.

--Soy de la Tierra, de hecho Cerdeña, así que estoy bastante cerca, en
todo excepto en años. Cuando me fui, era el 2375.

El Doctor la miró. Sabía que en la Cerdeña de la Tierra, en el año 2375
no tenían viajes en el tiempo. Y ese era aún otro misterio.

--¿Y cómo exactamente llegaste a la Roma del 120 dC? Algo más que una
pequeña vuelta de camino a casa del colegio, supongo.

La chica no hizo mucho más para evitar la pregunta que ignorarla.

--¿Puedes llevarme a casa?

El Doctor decidió no volver a sacar el tema por ahora. Negó con la
cabeza.

--Aún no--dijo--. No hasta que encuentre a Rose.

--Pero debe seguir en el estudio--le dijo Vanessa--. He estado
observando toda la mañana y no se ha ido.

El Doctor sabía que ese no era el caso. Pero se permitió tener
esperanza, solo por un momento.

--Vale--dijo--. Comprobémoslo.

La puerta del estudio seguía abierta. Buscaron en cada sala, pero no
había nada. Ni Rose, ni Tiro, ni estatuas, ni polvo de mármol. El Doctor
observó una bola de papiro que descansaba en una mesa, pero todo lo que
decía, en una letra cursiva difícil de leer, era: ``Una estatua para
Roma'' seguido de una figura: ¿era un recibo par el cartero? Lo dejó
caer.

--No lo entiendo--dijo Vanessa--. ¿Dónde está?

--¡No lo sé! --los peores miedos del Doctor se habían confirmado, y no
podía esconderse más de la conclusión--. La estatua que tenía Ursus, ¿la
viste?

--Sí--dijo--. Era una estatua de Rose.

--Lo cual no es posible--casi estaba temblando de sorpresa y furia.

Vanessa le alargó una mano para reconfortarle, pero el Doctor se la
apartó.

--¿Qué ocurre? --dijo, asustada.

--No era una estatua de Rose. La Rose de piedra era ella misma. Era Rose
la del museo--el Doctor se petrificó un momento, como si también se
hubiera convertido en piedra. Entonces se levantó, acercándose a
Vanessa. No podía rendirse ante el miedo, nunca se rendiría.

--Tenemos que encontrarla--dijo.

Le dijo a Vanessa que esperara mientras iba a su habitación, dónde tenía
algo que necesitaba. Justo en el momento no sabía cómo podría ayudarles
el destornillador sónico, pero necesitaba cualquier posible ventaja que
pudiera obtener.

El destornillador sónico estaba donde lo había dejado, pero había algo
más allí, algo hecho de tela. Lo cogió; era un pequeño bolsillo, del
tamaño perfecto para el destornillador sónico. Había una nota.

``Querido Doctor, ¡feliz no cumpleaños! Seguro que no sabías que podía
coser. Marcia me enseñó qué hacer. Con cariño, Rose.''

El Doctor aplastó la nota en su mano, sobrecogido con tristeza, enfado e
impotencia. Entonces cuidadosamente ató el bolsillo a su cinturón y
partió en busca de su amiga.

Vanessa se apresuró para alcanzar al Doctor mientras éste salía
corriendo de la casa y se abría camino por el camino que llevaba de la
entrada de la villa a la carretera. Por todas partes, el camino estaba
embarrado y se podían ver las huellas de un carro. El Doctor se apresuró
en seguirlas.

--¿No deberíamos\ldots{} decírselo a alguien? --preguntó Vanessa, sin
aliento por el ritmo que marcaba el Doctor.

--No hay tiempo--dijo el Doctor abruptamente, sin bajar el ritmo.

Pero al rato tuvo que detenerse. El camino se unía a la vía principal.
Era una típica calzada romana, larga y recta. Y como todo el mundo
sabía, todos los caminos llevaban a Roma. Lo que la gente no siempre
mencionaba es que aquello significaba que todos los caminos también
salían desde Roma.

El Doctor se puso de cuclillas, buscando rastros.

--Aquí--dijo, tras un minuto--. Un carro giró a la izquierda.

Pero Vanessa también estuvo buscando.

--Creo que uno giró a la derecha--dijo.

El Doctor se levantó y se le acercó.

--¡Maldita sea! --dijo--. Tenemos que encontrar el sitio dónde se cruzan
los rastros. Hay que descubrir cuál de ellos fue primero.

Pero su búsqueda fue infructuosa.

--¡De acuerdo! --dijo de repente el Doctor, poniéndose de golpe en pie
mientras examinaba un seco pedazo de tierra sin ninguna pista por
tercera vez--. Tenemos que dividir nuestros recursos.

Vanessa parecía preocupada.

--Quiero decir--le explicó el Doctor--, tú vas por un camino y yo voy
por el otro. Busca pistas. Habla con la gente. Descubre todo lo que
puedas--miró hacia el cielo, tapándose los ojos--. Espera hasta que el
sol se haya puesto por\ldots{} ahí--dijo, señalando--. Entonces vuelve
atrás y encuéntrate conmigo aquí.

--Pero, ¿qué camino tomo?

El Doctor lo pensó por un segundo.

--Rose y yo llegamos en los idus de marzo. Eso significa--contó con los
dedos--\ldots{} que casi son las Quinquatrus. Sí, Balbus lo mencionó. Un
festival que celebra el nacimiento de Minerva el 19 de marzo. Lo que la
convierte en una piscis--añadió con una sonrisa--. Y Minerva es\ldots{}

--Diosa de las artes y de los artesanos--completó Vanessa,
entendiéndolo.

--Es el momento en el que todos sus devotos le hacen ofrendas--siguió el
Doctor--. Y no me refiero a un ramo de flores o una caja de bombones de
chocolate con leche. Si no recuerdo mal, se reúnen en su templo en la
colina Aventina. Así que Ursus puede estar llevándola allí.

Vanessa asintió.

--Sí, eso tiene sentido. Así que se ha llevado a Rose a Roma.

Pero el Doctor negó con la cabeza.

--No tan rápido. Había dos carros y se han ido por dos direcciones
opuestas. Y hemos descubierto un recibo de un carretero: llevar una
estatua a Roma. Así que podría ser el otro carro el que ha ido por
ahí--señaló hacia el lugar, con los ojos cerrados y un brazo alargado.
Deteniéndose, abrió los ojos y se encontró señalando a la derecha, hacia
Roma--. Yo voy por aquí y tú vas por ahí, ¿de acuerdo?

Pero no le dio tiempo a responderle. Ya se había marchado, corriendo por
el camino que llevaba a Roma.

El sol brillaba en el cielo y el Doctor no encontró a nadie. El clima
era cálido para ser marzo, quizá todo el mundo se estaba pegando una
siesta de tarde. De vez en cuando avistaba rastros de carretas, pero eso
no significaba nada, pues podrían pertenecer a cualquier carro. Rendirse
no estaba en su naturaleza, pero los planes sin esperanza tampoco lo
estaban. Volvería a la villa, vería si Vanessa había tenido suerte y
hablaría con Gracilis, rogándole alquilar un carro o un carruaje, o
incluso sólo un burro.

Miró sin resultado alguno hacia la lejanía por un segundo final, como si
Rose se le fuera a revelar de alguna manera. Entonces se giró y volvió.

El Doctor ganó a Vanessa: no había señales de ella ni en el borde de la
vía, ni el camino que llevaba a la villa. Estaba entrando por la entrada
principal cuando vio un brillo de luz blanca por entre los árboles. La
estatua de Optatus.

Se le pararon los corazones. La ``estatua'' de Optatus.

Se hizo camino hasta el jardín y no se sorprendió de encontrar a
Gracilis y a Marcia allí. Le sonrieron con tristeza.

--Debes pensar que somos unos tipos sentimentalistas, Doctor--dijo
Gracilis--. Pero nos ayuda a sentir que sigue con nosotros.

El Doctor asintió. Se acercó a la estatua y la miró con detenimiento.
Entonces recorrió con amabilidad su mano por el brazo derecho.

--Hay un bulto--dijo--. Justo aquí.

Marcia lo miró.

--Optatus se rompió el brazo--dijo--. Se cayó de un árbol. El cirujano
se lo arregló, pero no se le curó del todo.

--Es increíbles que Ursus prestara tal atención al detalle\ldots{}--dijo
el Doctor, dejándolo caer. Pero no siguió. No les podía dejar saber la
verdad.

Se detuvo un momento, observando en silencio la estatua. Entonces dijo
abruptamente:

--Rose ha desaparecido. Está en un peligro mortal. Puede que ya sea
demasiado tarde.

Gracilis contuvo el aliento y Marcia parecía como si estuviera a punto
de desmayarse de golpe de nuevo.

El Doctor siguió diciendo:\\--Creo que va de camino a Roma. Gracilis,
¿puedes ayudarme a llegar allí?

Gracilis asintió fervientemente.

El Doctor se giró a Marcia.

--Vanessa me está ayudando a buscarla. Pero no puedo esperar más por
ella. cuando vuelva, ¿podrías mandarme un mensajero para hacerme saber
lo que ha descubierto?

Marcia también asintió.

--No hay tiempo que perder--dijo el Doctor--. Vamos.

No por primera vez, el Doctor maldijo el hecho que la TARDIS estuviera
en algún lugar de un callejón trasero romano, tan lejos como un día de
viaje en burro. Gracilis había insistido en acompañarle, así que el
Doctor tuvo que esperar impacientemente mientras se despedía de su
esposa y reunía provisiones. Alguna gente parecía ser totalmente
inconsciente de que el más mínimo segundo podría significar la
diferencia entre la vida y la muerte.

Gracilis seguía teniendo la esperanza de encontrar una pista en Roma
sobre el paradero de Optatus. El Doctor no le dijo que la mejor de las
pistas estaba justo en su propia villa. Gradualmente acabó diciendo que
creía que Ursus era el responsable de la desaparición de Rose. Pero no
reveló más que eso.

De tanto en tanto, el Doctor saltaba del carromato para comprobar las
huellas de carros, o para preguntar a un campesino agricultor, pero
ninguna de las respuestas le sirvió. Cuando cayó la oscuridad, se
acercaron a un hostal, dónde el Doctor no tenía la más mínima intención
de pasar la noche.

Gracilis protestó. Estaba tan deseoso como el Doctor de llegar a Roma,
pero los burros tenían que descansar.

--Entonces yo seguiré a pie--declaró el Doctor.

--Pero quizá Ursus también esté aquí descansando--sugirió Gracilis,
deteniendo al Doctor por un momento.

El Doctor volvió atrás, con sus huellas giradas hacia el hostal.

--De acuerdo. Lo haremos a tu manera--dijo.

Entraron y el Doctor inmediatamente arrinconó al propietario y le
describió a Ursus. El hombre reclamó no haber visto al escultor, todas
las veces que le preguntó el Doctor.

Gracilis pidió vino mientras el Doctor se paseaba por la sala.

Al Doctor de repente se le ocurrió una idea.

--Regentas un hostal. Debéis tener burros aquí. O incluso mejor,
caballos--dijo al propietario.

El hombre se inclinó, sumiso.

--Así es, señor.

--Entonces me gustaría alquilar tu mejor corcel, por favor, de
inmediato.

El hombre murmuró el perdón del Doctor, pero temió que tal cosa no se
podía realizar.

--Me temo que nuestras bestias no están frescas y no serían capaces de
sobrellevar un largo recorrido.

El Doctor frunció el cejo, controlando su impaciencia y finalmente
accedió a sentarse dónde compartió una cena con Gracilis.

Justo cuando estaban acabando, hubo un sonido de cascos del exterior y
unos momentos más tarde la puerta se abrió de par en par. Un hombre de
apariencia arrogante de unos cuarenta años entró, con su nariz romana
apuntando al horizonte. Chasqueó los dedos y, cuando el propietario
corrió a acercarse, empujó a un esclavo la copa de vino que le servía a
Gracilis.

Gracilis comenzó a quejarse, con su pecho henchido de indignación, pero
el Doctor levantó una mano para acallar sus enfadadas palabras.

--Hola--dijo alegremente, saltando y ofreciéndole una mano al recién
llegado--, pareces sediento. ¿Ha sido un viaje largo?

El hombre miró al Doctor con desdén, observando su túnica lisa, unos
rasgos muy poco romanos y una sonrisa mucho menos servil. No le apretó
la mano.

--Soy Lucius Aelius Rufus. He viajado desde la Galia en un negocio con
el emperador--dijo, sin inmutarse.

Gracilis botó ligeramente. Claramente, había oído hablar del hombre.

--La Galia, ¿eh? --dijo el Doctor--. Oh, un lugar muy duro. Aunque es un
bonito paisaje.

Rufus le ignoró.

--Espero aquí meramente mientras mi caballo es cambiado y entonces
volveré a emprender la marcha.

El Doctor alzó las orejas.

--¿Tu caballo? --dijo--. Me temo que no tienes suerte. No hay caballos
frescos aquí.

--Tonterías--dijo Rufus--. Ya ha sido todo arreglado.

El Doctor se giró para contemplar al propietario. El desafortunado
hombre se encogió como Uriah Heep y le rogó perdón de nuevo al Doctor.
Habían recibido un mensaje de que aquel caballero venía. El último
caballo se había reservado para él. El propietario lamentaba que sus
anteriores palabras pudieran haber confundido al Doctor, no pretendía
haberle faltado al respeto.

--Ya veo, ya veo--dijo el Doctor, amablemente, sentándose en frente de
Rufus e ignorando su mirada hostil--. Debe de tratarse de negocios
importantes, si ni siquiera puedes descansar una sola noche--dijo.

--Por supuesto--el recién llegado sonaba aburrido.

--Vida o muerte, grandes beneficios para la humanidad, ¿ese tipo de
cosas?

--Mis negocios son cosa mía--dijo Rufus, y giró la cabeza en un vano
intento de evitar que el Doctor se dirigiera a él.

--¿Pero me podrías decir si se trata de vida o muerte? --insistió el
Doctor.

La mano del hombre se retorcía con furia alrededor de su copa mientras
bebía el vino. No dijo nada.

--Bueno, entenderé eso como que me has oído, entonces--dijo el Doctor.

Se levantó y se acercó a Gracilis.

--Cógeme si puedes y asegúrate de que no le hace nada al hostelero--le
susurró a su amigo en el oído. Entonces casualmente se paseó hasta la
puerta principal, dejando a Gracilis mirándole admirado.

Unos minutos más tarde, el sonido de unos cascos se oían en la calzada,
gradualmente desapareciendo en la distancia. Le llevó un momento a Rufus
entenderlo, y cuando se hubo dado cuenta y hubo seguido al Doctor hacia
fuera, caballo y jinete eran ya un punto en la distancia.
