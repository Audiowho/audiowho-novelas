\chapter*{Capítulo Tres}
\addcontentsline{toc}{chapter}{Capítulo Tres}

Gracilis llevó a su esposa llorosa hacia la casa principal. A Vanessa le
dijeron que les siguiera; Marcia le diría más sobre el nacimiento de
Optatus. Ursus parecía estarse a punto de ir, pero el Doctor le detuvo
con un gesto.

--Es algo bastante impresionante--dijo, señalando la estatua.

--Todos mis ``algos'' son impresionantes--respondió Ursus.

--Ah, ya veo. ¿Lo veo? Sí, creo que sí lo hago--dijo el Doctor,
asintiendo. Comenzó a girarse, dejando ir al hombre, pero entonces le
detuvo con una pregunta--, debe de haber visto mucho a Optatus,
trabajando en esto. ¿Qué crees que le ha pasado?

Ursus se encogió de hombros.

--¿Cómo podría saberlo? Los hijos de los hombres ricos son secuestrados.
Pasa mucho.

--Pues menudos secuestradores--añadió Rose--. Le han tenido durante unos
días y ni se han molestado en pedir rescate.

--Entonces debe de haber huido a algún lugar. Quizá le hayan atacado
algún vagabundo, quizá esté bebido en alguna taberna por ahí. No es que
me incumba.

El Doctor pareció considerarlo.

--Sí, puede ser\ldots{} por supuesto, estamos a kilómetros de la ciudad
y no cogió ningún transporte\ldots{} Aún así, puede que tengas razón.

Rose estaba segura de que no se lo creía ni lo más mínimo.

O quizá no quería creérselo, o no quería una solución mundana. Quizá,
para el Doctor, cualquier pequeño misterio que investigar era mejor que
no hacer nada.

--¿Cuándo viste por última vez al chico? --preguntó el Doctor, con su
mejor voz de detective.

Ursus pareció responder a regañadientes.

--Hace cuatro días--murmuró al cabo del rato--. Mi trabajo ya estaba a
punto de terminar, pero me visitó en mi estudio para observar los
últimos retoques. Tengo montado un estudio cerca de los establos para
poder trabajar aquí.

--Debe de ser interesante--dijo el Doctor--. Me gustaría ir y visitarte
allí.

Ursus negó con la cabeza firmemente.

--No permito que nadie me vea trabajar. A nadie.

--Ah--dijo el Doctor. Entonces comenzó a contar con los dedos--. Espera
un momento, ¿cuatro días? Ese es el día exacto en el que el joven amo
Optatus desapareció. Rayos, podrías haber sido la última persona en
verle.

--Es posible--el escultor se encogió de hombros--. ¿Cómo puedo saberlo?

--Bueno, creo que de verdad tendremos que visitar ese estudio excelente
tuyo. Para seguir el rastro y todo eso.

Ursus se enfureció.

--Te he dicho que no, que no permito que nadie entre en mi estudio. ¡Soy
un artista y no lo permito!

--No es tu estudio, ¿no? --dijo Rose--. Este lugar pertenece a Gracilis,
y me apuesto lo que sea a que nos dejará entrar. Ya que estamos buscando
a su hijo y todo eso.

Ursus dio un paso adelante y por un momento Rose pensó que iba a
pegarla. Ella se puso tensa. Pero en vez de eso él alargó una mano y
tocó ligeramente su mejilla. Sus manos eran enormes, con unos grandes
dedos con forma de salchicha enfundados en unos ligeros guantes de
cuero, no del tipo de mano que ella asociara con los artistas. Pero su
física apariencia regordeta obviamente desmentía su habilidad,
comprobada por su magnífica escultura realista de Optatus.

--Tú, puedes venir a mi estudio--le dijo el escultor a Rose, y ella de
repente se dio cuenta de lo inevitable que se acercaba a continuación.

--Tengo un pedido: una estatua de una diosa. Tú eres joven y hermosa. Tú
serás ella. Es la única forma en la que puedas ver mi estudio: si te
hago una diosa.

Todo les había estado guiando a ello. Llegar a Roma, rescatar a
Gracilis, ser invitados a su villa, conocer a Ursus, siempre iba a
acabar pasando, porque todo les iba a llevar a aquel momento, a la
creació de la estatua que viajaría a través de los siglos y media Europa
para terminar en la planta baja del Museo Británico en el siglo XXI.

--De acuerdo--dijo Rose.

El Doctor se había sentado en el borde de un estanque y colgaba su pie
en el agua, sonriendo a los pequeños pececillos brillantes que nadaban
para mordisquear los dedos de sus pies. Rose se sacó las sandalias y se
le unió.

--Así que vas a ser modelo, ¿entonces? --dijo--. Rose, la modelo. La
modelo Rose.

Ella asintió tristemente.

--Sí, eso parece. Ya sabes, creía que sería más glamuroso que esto:
posar para un hombre de las cavernas en una toga en el establo de
alguien.

--Puede que no será glamuroso, pero es importante. Necesitamos ver ese
estudio.

--¿Crees que Ursus es sospechoso? --preguntó Rose.

El Doctor se encogió de hombros.

--No lo sé. Puede ser. Quiero decir, no me gusta, pero eso no significa
que sea sospechoso. Aunque puede parecerlo, viendo que soy tan buen juez
de personalidades. De cualquier manera eso mejor no ignorar cualquier
posibilidad.

Rose suspiró.

--Me pregunto cómo le estará yendo a Vanessa con la señora de Gracilis.
Pobre chica. Creo que el tal Balbus la metió en el timo de la astrología
y ahora está atrapada en ello.

--¿Crees que no puede ver el futuro? --preguntó el Doctor.

Rose rió incrédula.

--¡Por supuesto que no! Nadie puede. Quiero decir, sé que hay gente que
puede ver cosas, pero es por culpa de, ya sabes, de alienígenas y de las
brechas temporales y eso. Vamos, ¡tú sabes que es un fraude!

El Doctor levantó una ceja.

--Recuerda las cosas que han dicho de ella en la cafetería. Las cosas
que predijo.

--Debes de haber estado viendo demasiada televisión veinticuatro horas,
sí señor--dijo Rose, a penas creyendo que tendría que explicárselo al
Doctor de entre todas las personas--. Debió de haber supuesto lo del
edificio, o haber oído rumores. Y sólo porque sepamos que el Imperio
romano vaya a caer de verdad, no significa que cualquiera que haya
hablado de ello sea una profetisa. Simplemente debe de ser, ya sabes,
pesimista.

--Buen uso de la lógica--dijo el Doctor.

--Gracias--dijo Rose. Pensó por un minuto--. Ya que lo comentas, hay
algo raro sobre ella. Me gustaría que me dijera algo más sobre ella. Ni
siquiera me ha dicho de dónde es. Esta cosa de la TARDIS en mi cabeza,
de acuerdo, vale, puedo entender todos los idiomas, ya sabes, y eso está
bien, pero con ello me cuesta averiguar acentos. Supongo que debe de
hablar latín, porque todo el mundo la entiende, pero no creo que sea de
Roma. Conoce el Muro de Adriano, pero no creo que sea británica. Seguiré
intentándolo, pero\ldots{}

El Doctor se irguió de repente igual que el fauno de piedra a su lado.

--Así que--dijo después de un momento--, no crees que esta chica pueda
predecir el futuro.

--No, no lo hago--dijo Rose.

--Así que, ¿podrías explicarme exactamente cómo esta hablante de latín
de dieciséis años es consciente del muro de Adriano que ni siquiera
comenzará a pensar en construirlo hasta el año siguiente? Eso es lo que
yo llamo una buena suposición.

Rose le miró boquiabierto.

Se encontró a Vanessa en la casa principal con Marcia. Marcia tenía un
pedazo de bordado en su regazo, pero lo ignoraba a favor de explicarle a
Vanessa la historia entera de la vida de Optatus.

--Y entonces en su quinto verano se cayó del melocotonero y se hizo daño
en el brazo--ella se detuvo, expectante.

Vanessa dijo:

--Ah, la, eh\ldots{}, típica intrepidez de los Capricornio, bajo la
influencia hostil de\ldots{} Júpiter.

--Por supuesto, por supuesto--coincidió Marcia--. Acércate, querida, y
siéntate--le dijo a Rose. Llamó con la mano a un esclavo, que salió y
volvió al cabo de pocos segundos con una copa de vino para Rose--.
Querida, tengo que confesarte lo mucho que me gusta tu pelo--continuó--.
¡El rubio está tan de moda!

--Eh, gracias--dijo Rose--. Bueno, ya sabes lo que dicen: las rubias se
lo pasan mejor\ldots{}

--¿Lo dijo una esclava?

--No--dijo Rose, confundida--. Es mío.

--Oh, ¡te lo has teñido! --dijo Marcia, asintiendo compresiva--. Bueno,
te pega bastante.

Rose decidió que sería sensato dejar el tema de la peluquería antigua
antes de que se viera metida en problemas. Se giró a Vanessa--. Así que,
¿has cogido alguna pista?

La joven le dio una sonrisa nerviosa.

--Presiento que Optatus ha sido favorecido por los dioses--dijo--. Las
estrellas de su nacimiento fueron\ldots{} auspiciosas. Estoy segura de
que está a salvo.

--Genial--dijo Rose, sentándose.

--¡Sí, me deja mucho más tranquila! --dijo Marcia, sonriendo.

--Bueno, mientras Vanessa, eh, considera su destino astrológico, estamos
intentando averiguar quién le vio por última vez. Dígame, Marcia, ¿qué
sabe de este tal Ursus?

Los ojos de Marcia se abrieron.

--¿Crees que Ursus pueda estar conectado con la desaparición de mi hijo?
¡Haré que mi marido lo expulse de estas tierras ahora mismo!

Rose se apresuró en calmarla.

--No, mire, sólo estoy preguntando. Gracilis dijo que era de por aquí,
¿no es cierto? Así que debe de conocerlo. E incluso si estuviera
conectado, que no digo que lo esté--añadió rápidamente, al ver que la
boca de Marcia se abrió de golpe--, no queremos que se vaya a toda
prisa, ni nada. Mantengámosle dónde lo tengamos controlado, eso es lo
que hay que hacer.

Marcia asintió a regañadientes. Pensó durante un momento y entonces
dijo:

--Tampoco sé tanto de él. Recuerdo que era un niño torpe y desagradable.
Recuerdo sorprenderme al escuchar que había decidido ser artista.

Era el tipo de historia que nunca cambiaba, pensó Rose. Un chico
impopular, objeto de burlas y ridiculizado pero centrado en conseguir su
ambición y demostrar que sus detractores se equivocaban. Pero el final
tenía lugar con menos frecuencia: el chico se volvía bueno en contra de
lo que podía parecer. Porque, ella lo había visto por sí misma, es lo
que había pasado. Después de años de humillación, el chico, convertido
en un hombre, se había encontrado con el éxito que tanto había buscado,
quizá más de con el que había soñado nunca.

--Fue, oh\ldots{} no estoy seguro de hace cuanto--le dijo Marcia--. De
repente comenzamos a escuchar su nombre por todas partes. Las estatuas
aparecieron en los templos y en los altares de la misma Roma, y las
admiraban en todas partes. Creímos que quizá fueran de otros artistas
del mismo nombre, pero no, era el Ursus que conocíamos. Mi marido le
persiguió durante un tiempo hasta que accedió a aceptar nuestra
petición--suspiró--. Y no sabes lo agradecida que estoy. Me reconforta
saber que aún puedo seguir observando la cara de mi niño, incluso en
esta época de oscuridad.

Rose no dijo nada, ni sugirió que si no hubieran convencido a Ursus de
hacer aquella estatua de Optatus, toda aquella ``época de oscuridad'' no
habría tenido lugar y hubiera podido ser evitada.

De repente Marcia golpeó con su puño en su regazo.

--¡Por supuesto! --dijo--. La primera vez que vimos una estatua de Ursus
fue cuando estuvimos en Roma, ahora lo recuerdo. Era el festival de
Fortuna. Eso significa que es hace casi unos diez meses.

Vanessa pegó un respingo.

--¿Estás bien? --dijo Rose.

La chica asintió, pero miraba en la lejanía.

--Hace casi diez meses--dijo bajo su respiración.

--Creímos que Fortuna nos había sonreído cuando aceptó el regalo para
nuestro hijo--dijo Marcia con tristeza--. Ahora creo que pudimos haberla
ofendido.

El nombre de ``Fortuna'' resonó en la cabeza de Rose. Era aquello lo que
decía bajo su estatua en el Museo Británico. Y quizá, pronto, ella
descubriera exactamente cómo había llegado la estatua allí.

--Ursus quiere hacer una estatua de mí--les dijo.

Marcia pareció sorprendida durante un momento, aunque rápidamente lo
ocultó.

--¡Pero eso es encantador! --dijo--. Sé que mi marido le ha dado permiso
para trabajar en el estudio todo lo que quiera--ella sonrió, con
indulgencia--. Creo que le gusta verse como patrón de las artes. De
cualquier manera, debes quedarte con nosotros hasta que el trabajo esté
acabado.

--No quiero poneros en un\ldots{}--dijo Rose, pero Marcia ya estaba
actuando de buena anfitriona y acallando sus protestas.

--Será un placer tener alguien joven por aquí mientras mi
hijo\ldots{}--sus maneras quebraron durante un segundo--, mientras mi
hijo está desaparecido. Apreciaré la compañía.

--Vamos a traerlo de vuelta--le dijo Rose, incómoda--. El Doctor y yo, y
Vanessa--añadió a toda prisa--. Ya sabes, con sus predicciones y todo
eso.

La joven se enrojeció.

Rose recordó lo que el Doctor había dicho de hacerla hablar, pero era
incómodo con Marcia allí. Así que dijo:

--De hecho, creo que pueda necesitar su ayuda ahora mismo. Mientras el
Doctor está rastreando los movimientos de Optatus, Vanessa puede\ldots{}
mirar las vibraciones y todo eso.

Marcia asintió, consciente de ello.

--Sí, por supuesto--movió un mano para despedir a Vanessa y la chica
siguió a Rose fuera de la habitación.

Caminaron por el patio. Los esclavos pasaron a su alrededor, llevando
cestas de frutas o de pan recién horneado.

--Oh, adoro ese olor--dijo Rose, mientras una bandeja de hogazas pasaba
cerca de allí.

--Creo que Optatus puede haber andado a través de este patio--dijo
Vanessa, nerviosamente.

Rose rió sonoramente.

--¡No me digas! Mira, todo esto está bien. No voy en busca de nada de
esto de abracadabra. Sólo creo que necesitas un descanso. Vamos--llevó a
Vanessa a la arboleda donde estaba la estatua de Optatus y se sentaron
juntas en la hierba.

--Parece tan joven--murmuró Rose, mirando la estatua--. Dulces
dieciséis. De la misma edad que tú.

--Eso creo--asintió Vanessa.

Rose le lanzó una mirada.

--¿No estás segura?

--Es\ldots{} difícil seguir el rastro a veces.

--Así que, ¿de qué signo eres?

--Escorpio--dijo Vanessa--. Decidida y fuerte.

Rose rió.

--Bueno, ¡ahora me has convencido! No te crees nada de eso.

No era una pregunta, sino una afirmación.

La chica parecía asustada, así que Rose se apresuró a tranquilizarla.

--Está bien. Sé que todo ha sido un jaleo, y me apuesto lo que quieras a
que te han arrastrado a ello. ¿Verdad?

Vacilante, Vanessa asintió.

--Así que esto es lo que va a pasar. El Doctor y yo vamos a encontrar a
Optatus. Eso es lo que hacemos, resolver las cosas. Te puedes quedar con
nosotros. Le diremos a todo el mundo que te necesitamos. Con suerte
nadie te preguntará sobre las lunas de Saturno ni nada, pero tú
obviamente sabes cómo dar la charla si es necesario. Entonces, cuando
esté todo arreglado, te llevaremos a casa de alguna manera. Te sacaremos
de todo esto.

Rose pensó en su propia vida a los dieciséis. La secundaria, enamorarse,
perderse clase, irse de casa. Todo acababa en desastre y en un corazón
roto, por supuesto, y nunca habría querido volver, pero fuera lo que
fuera, era una vida. Lo que Vanessa estaba pasando, bueno, no conocía
los detalles, pero se habría apostado cualquier cosa a que no era para
nada parecido.

Rose había creído que Vanessa se alegraría. Pero cuando miró a la chica,
vio que había comenzado a llorar.

--Eh, ¿qué pasa? --dijo, abrazándola.

--Ya no tengo hogar--dijo la chica--. Y nunca más lo tendré.
