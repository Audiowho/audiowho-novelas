\chapter*{Capítulo Nueve}
\addcontentsline{toc}{chapter}{Capítulo Nueve}

De repente unas puertas se abrieron en el extremo opuesto de la arena.
Alguien había obviamente decidido cambiar de táctica. Los hombres fueron
arrastrados a la arena, docenas de ellos obligados a punta de espada.
Unas voces llamaron al Doctor y él reconoció a John, Paul, George y
Ringo. Conocidos hace pocas horas, le parecieron tan cercanos para él
como si fueran hermanos.

Unas estacas estaban clavadas en la arena y era allí dónde estaban
siendo arrastrados, tras ser atados por los brazos a los postes
verticales. El Doctor observó con horror como un \emph{bestiarius}
corría con una cesta de carne cruda, pedazos de la cual dejó caer a los
pies de los hombres atados. No había forma de malentenderlo. El Doctor
había luchado demasiado, la multitud necesitaba matanza.

En cuanto los \emph{bestiarii} hubieron huido, el Doctor corrió hacia
los hombres. El pequeño Ringo era el más cercano y el Doctor le liberó
con unas cuantas sacudidas de las puntas del tridente. Le pasó al hombre
la antorcha encendida para protegerse, y corrió hacia la siguiente
estaca. Para su sorpresa -- y delicia -- vio cómo Ringo iba a otro
hombre. Usó la antorcha para encender las cuerdas -- las llamas
obviamente le dolieron al hombre, pero al rato estuvo libre.

El Doctor cortó las cuerdas del siguiente prisionero. Señaló a la
siguiente estaca, dónde Paul estaba atado.

--Intenta desatarle--le instruyó el Doctor.

Tembloroso, el hombre corrió hacia allí y comenzó a hacer lo que le
habían mandado.

La multitud gritó en desaprobación. Aquello no era lo que se suponía que
tenía que pasar.

Se abrieron más trampillas. Media docena de leopardos salieron de ellas.
Les llevó un momento capturar el aroma, pero entonces corrieron hacia
los hombres condenados. El Doctor gritó para que todo el mundo se
quedara dónde estaba, pero un par de prisioneros, demasiado
aterrorizados como para escuchar, corrieron en dirección opuesta. El
movimiento atrajo a los felinos y una vez más la multitud tuvo algo con
lo que contentarse.

Los hombres restantes se reunieron en un grupo, Ringo al frente mientras
zarandeaba frenéticamente la antorcha a los animales. El grupo viajó de
una estaca a otra, mientras el Doctor liberaba los prisioneros de uno en
uno y los demás recogían los pedazos de carne cruda y se los lanzaban a
los leopardos para atraer su atención.

El último hombre al que llegó el Doctor fue George. Era un hombre de
rostro marcado y piel oscura de unos cuarenta años, pero en aquel
momento su cara brillaba como la de un ángel.

--¿Está pasando de verdad? --dijo--. ¿O ya he muerto?

El Doctor sonrió.

--Cooperación--dijo--. Menuda hermosa palabra. Vamos a conseguir salir
de aquí, ¿vale?

--No pido un milagro--le dijo George.

--De igual manera--dijo el Doctor--, quién no llora, no mama. O eso
dicen. Pero supongo que un milagro es lo que está de camino.

Cuando George fue libre, el Doctor gritó:

--¡A la pared!

Guió al grupo en un ataque contra el perímetro de la arena. Una puerta
cerca se abrió y unos hombres armados aparecieron, las siluetas fornidas
y sudorosas de los carceleros Flaccus y Thermus al final. Los
prisioneros, sin embargo, estaban con demasiada adrenalina como para
pararse. Los sorprendidos guardas de repente se encontraron a sí mismo
cayendo en una marabunta de antorchas, tridentes, puños y sencilla
rabia. En cuanto la lucha hubo acabado, había unos cuantos hombres
condenados que habían muerto, pero muchos otros ahora sujetaban espadas
y estaban ante los cuerpos de los que una vez fueran sus captores.

Flaccus y Thermus se habían protegido durante la lucha, zarandeando sus
espadas sin efecto alguno. Fue Paul quién les vio y alertó a los otros.
Los dos guardas retrocedieron cuando los hombres furiosos se giraron
hacia ellos.

--¡Sólo estábamos siguiendo órdenes! --gritó Thermus.

--¡Hicimos lo mejor por vosotros! ¿No os acordáis? --dijo Flaccus,
tragando saliva--. ¡Os tratamos como nuestros propios hijos!

--¡Nos tratasteis como basura, más bien! --gritó Paul, sujetando una
liberada espada en el aire.

Flaccus y Thermus se giraron y salieron corriendo.

Y se tropezaron justo encima del león que el Doctor había mandado a
dormir anteriormente.

Y el león se despertó.

Cuando los gritos de los guardas acabaron, los hombres del Doctor se
abrieron por fin paso a la pared. Los caídos en batalla estaban
sirviendo de distracción para las bestias, pero todo el mundo estaba
incómodamente consciente de que su atención podría volver a ser atraída
en cualquier momento.

--¿Y ahora qué? --tosió George, mirando hacia la pared de mármol.
Incluso si pudieran escalarla, algo que no podían hacer, la verja en lo
alto de ella les detendría de ir más allá.

El Doctor también miró hacia arriba. No muy lejos por encima de él podía
ver la furiosa cara de Rufus, aún con ganas de la sangre del Doctor. Al
lado de Rufus, para la delicia absoluta del Doctor, vio a Gracilis,
empujando la capa del magistrado. Sonrió. El triunfo del hombrecito, era
sobre lo que iba aquello.

Miró a la verja por encima de él. Entonces miró al tridente que
sujetaba. Volvió a mirar hacia arriba. Comenzó a retroceder de la pared.

Sonrió a sus camaradas.

--Siempre me he considerado a mí mismo un atleta--dijo--. Este es el
momento para descubrir si el salto de pértiga está hecho para mí\ldots{}

Corrió hacia adelante, hundiendo el tridente en el suelo y usándolo para
impulsarse a sí mismo en el aire. La multitud contuvo el aliento. Nadie
podía hacer aquello. Iba a empalarse en las puntas de la verja\ldots{}

Pero no fue así. El Doctor soltó una carcajada de alegría cuando esquivó
lo alto y aterrizó en el regazo de dos senadores confusos.

--¡Juegos Olímpicos, allá voy! --aún riéndose, se incorporó--. Pasadme
una espalda--gritó hacia abajo.

Una vino por el aire cuando George la lanzó. El Doctor la cogió cuando
cayó a su lado de la verja.

--Bien, no queremos ningún problema--dijo, dirigiéndose a los senadores
que le rodeaban--. Y no quiero herir a nadie. Pero si hace falta, lo
haré. Así que por vuestro propio interés haced lo que digo. Tú--dijo,
girándose al hombre más cercano--, dame tu toga.

El hombre obedeció a regañadientes, arrancándose su tela morada y
dándosela.

--Tú también--dijo el Doctor, y el siguiente hombro también cumplió.

El Doctor ató las largas telas y lanzó la cuerda resultante al lado de
Paul, que la cogió y comenzó a escalar la pared de la arena.

Unos guardas armados comenzaron a aparecer en las gradas, pero la
multitud era tan espesa que no se podían acercar demasiado. Pronto
varios prisioneros armados con espadas habían cruzado a la zona de
asientos y estaban amenazando a todo el mundo.

Un grito llegó de abajo. Los leopardos se habían cansado de su festín
inmóvil y estaban acercándose a su presa viva. Ringo seguía zarandeando
su fuego y George tenía el tridente, pero los felinos parecían tomar sus
esfuerzos como un desafío.

--Esperad--gritó el Doctor, toqueteando su destornillador sónico--.
Tengo que encontrar otra frecuencia\ldots{}

La distracción del Doctor casi le provocó la muerte.

--¡Cuidado! --gritó una voz de repente. ¡Era Gracilis!

El Doctor se giró en redondo. A una docena de asientos de distancia,
Gracilis había saltado mientras Rufus levantaba un arco para disparar.
Sus dedos se soltaron de la cuerda\ldots{}

\ldots{} y el Doctor levantó su destornillador sónico, usándolo como un
pequeño propulsor. La flecha salió disparada y se clavó sin hacer daño
alguno en el suelo.

Rufus grito de furia y se adelantó, intentando buscar mejor puntería. Un
par de prisioneros armados comenzaron a ir a por él, pero el Doctor les
gritó que se detuvieran, demasiada gente inocente había por el camino.

Hubo un grito del suelo del estadio. Un leopardo había cruzado las
defensas y pedía una víctima.

--¡Seguid subiendo! --les gritó el Doctor, apremiante.

Pero eso dio a Rufus una idea. Sonriendo malévolamente, se acercó a la
verja. Ahora su flecha apuntaba hacia abajo, justo a John, que estaba a
medio camino de las togas atadas. Se movió para disparar, pero el
disparo nunca se produjo. El tridente de George se hundió en el pecho
del magistrado y Rufus se precipitó sobre la arena. El leopardo que
había estado a punto de atacar a George giró su atención a su nueva
delicia.

La multitud gritaba y se encogía de horror, delicia y miedo.

Gracilis se puso en pie y se tambaleaba en dirección del Doctor, pero la
multitud no le dejaba pasar.

--¡Te veo fuera! --le gritó el Doctor, indicando al anciano para que
cambiara de rumbo.

Se giró hacia la pared. Ahora George subía por la cuerda. Ringo fue el
último, lanzando su antorcha al leopardo cuando éste se abalanzó contra
él. En cuanto los pies de Ringo tocaron el suelo, el Doctor se abrió
camino hasta la salida más cercana. No fueron los únicos. Todo el mundo
cerca quería salir de allí ya, y eso seguía deteniendo a los hombres
armados de llegar a los prisioneros.

El Doctor zarandeó una espada, amenazante.

--¡Fuera! ¡Fuera! ¡Fuera! --gritaba.

De repente, George estaba a su lado. Parecía conmovido.

--He matado a un magistrado--conteniendo el aliento--. Si me iban a
matar entonces, ¡ahora no van a mostrar piedad!

El Doctor le dedicó una sonrisa de confianza.

--Has salvado una vida--dijo sencillamente--. ¡Ahora, date prisa! ¡Sal
de aquí, vete lo más lejos posible! ¡Haz buenos pasteles! ¡Y vive!

George le lanzó una media sonrisa nerviosa y corrió.

Mientras el Doctor le observaba irse, se dio cuenta de que ni siquiera
sabía su nombre verdadero.

Reinaba el caos en las calles. Afortunadamente muchos de aquellos que no
habían estado en la arena intentaban desesperadamente obtener los
detalles de lo que había pasado de aquellos que se iban. Y lo otro
afortunado era, pues como la arena era tan enorme, pocos sabían cómo
eran los hombres fugados.

El Doctor se agachó y se estrechó entre la multitud, soltando
ocasionales ``¿Has visto lo increíble que era?'' a los viandantes, o un
``¡Nunca había visto nada parecido''. Finalmente vio a Gracilis entre
las masas y corrió hacia él.

--¡Gracilis!

El anciano se giró.

--¡Doctor! --se saludaron el uno al otro alegremente. Gracilis casi
parecía estar llorando.

--¡Creía que te había perdido! Primero mi hijo y luego mi amigo. He
intentado asegurarme de tu liberación de tu más ilegal castigo. He
visitado mis contactos e intentado pedir favores, pero ninguno iba a ir
en contra de Rufus. Así que seguí al hombre a la arena, intenté razonar
con él, pero no me escuchó y entonces, entonces\ldots{}

El Doctor le detuvo.

--No nos preocupemos por ello. Ya ha acabado todo. Bueno, casi. Mira,
¿puedes dejarme tu capa?

Sin preguntar, Gracilis se desabrochó su larga capa y se la pasó al
Doctor.

--Me disfrazo--le dijo el Doctor. Y frunció el ceño--. Será mejor que te
alejes de mí. Quiero decir, no saben quién soy yo y no hay ningún
testimonio de mi detención, pero eso no creo que les detenga de
perseguirme. No quiero meterte en problemas.

El anciano se estiró todo lo que pudo.

--Me has estado ayudando, Doctor, con gran inconveniencia para ti. Es
por muchas razones culpa mía que estés metido en este embrollo. No
abandonaré a un amigo.

--Gracias--dijo sencillamente el Doctor, pero su cálida sonrisa dijo más
que sus palabras.

Caminaron por la calle, intentando llegar lo antes posibles al altar de
Fortuna, dónde el Doctor había recibido la cura para Rose, y había oído
la voz misteriosa. No podían ir todo lo rápido que hubiera querido por
miedo de atraer demasiado la atención, pero al fin llegaron, abriéndose
camino por entre una muchedumbre de niños que estaban de vacaciones. El
Doctor, dándose cuenta de que aún llevaba una espada, se la pasó a un
joven sorprendido, con instrucciones de no herir a nadie.

Finalmente llegaron. El Doctor entró en el templo corriendo, golpeando a
un joven que estaba preparándose para presentar una ofrenda.

--¿Hola? --gritó el Doctor, ignorando la presencia del hombre--. ¿Hay
alguien aquí?

Fue directo a la estatua de Fortuna, en pie en su cubículo en uno de los
extremos. Se asomó detrás de ella, pero no había nadie allí. Quien fuera
que hubiera sido, ¿de verdad esperaba que hubiera estado allí todo el
día?

--¿Hola? --intentó de nuevo, pero con menos convicción.

Se giró. Una cosa no estaba, por favor que la segunda sí que estuviera.
Siguió sus pasos del día anterior. Aquí fue dónde estaba cuando los
hombres armados le habían agarrado. Aquí es dónde le habían golpeado.
Aquí es dónde se le había caído el frasco.

Y no había allí un frasco. Se puso de rodillas, buscando frenéticamente.

--¿Qué ocurre, Doctor? --preguntó Gracilis, preocupado.

--Alguien, quiero decir, Fortuna me dio algo que dijo que traería a
Rose--dijo el Doctor--. Y también a Optatus.

Los ojos de Gracilis se iluminaron.

--¿Te refieres a esto? --preguntó, sacando un frasco de cristal de un
brillante liquido verde--. ¿Esto me devolverá a mi hijo?

El Doctor pegó un salto, con su cara llena de alegría.

--¡Eso es! --gritó--. ¡Oh, gracias, gracias, gracias!

Cogió el frasco y lo besó, y se detuvo a sí mismo de besar también a
Gracilis.

--Lo encontré en l suelo aquí tras tu captura--le explicó Gracilis--, y
me preguntaba si pudiera ser algo importante.

--Creo que puede serlo--dijo el Doctor--. Quiero decir, ¿me mentiría
Fortuna?

Lo pensó por un segundo, preguntándoselo seriamente. No era como si las
circunstancias fueran sospechosas. Pero tenía un total presentimiento de
que podía confiar en la extraña voz. De hecho, es como si le sonara
familiar\ldots{}

--Vamos. Lo primero es encontrar la estatua de Ursus.

--¿Por qué? --preguntó Gracilis--. Si esta poción puede devolverme a mi
hijo\ldots{}

El Doctor se rascó la nariz.

--Confía en mí--dijo--. Vamos a llegar a ello. Pero tenemos que
encontrar primero esta estatua.

El joven con la ofrenda había estado observando todo aquello con
admiración y sin preocupación alguna. De repente se aclaró la garganta.

--¿Estáis buscando una estatua del escultor Ursus? --preguntó,
nerviosamente.

El Doctor se giró para mirarle.

--Pues vaya que sí. ¿Conoces alguna?

El hombre asintió.

--Creo que una nueva estatua de Ursus va a revelarse hoy en el foro.

El Doctor sonrió.

--¡Sí! Gracilis, viejo amigo, parece como si todo fuera de rosas, y no
hago referencia a Rose--levantó el frasco--. Una cura milagrosa. Una
estatua de Rose va a ser revelada. Rescatar a Rose, rescatar a Optatus,
rescatar a todos los demás y estar en casa para el té. Bueno, para el té
de mañana, sea lo que sea. O posiblemente al desayuno del día de
después. Da igual, ¡la vida es bella!

Aún sonriendo, se abrió paso hacia el exterior del altar y ambos fueron
hacia el foro.

Entraron por el magnífico Arco de Augusto, pero el Doctor no estaba de
humor para apreciar la arquitectura. Escaneó la muchedumbre ante él, los
cientos de personas que iban a sus negocios diarios, encontrándose,
comprando, orando, y los cientos de estatuas, ninguna moviéndose de un
lado para otro, que les estaban observando hacerlo. Había movimiento
cerca de la basílica, cuyas paredes ya estaban pobladas de estatuas, le
llamó la atención y corrió hacia allí. Una multitud se había reunido
allí y preguntó a una mujer qué estaba pasando.

--Una nueva estatua--le dijo--. Del tipo ese, Ursus, ese del que habla
todo el mundo.

El Doctor dedicó un segundo para agradecérselo y entonces comenzó a
abrirse camino a través de la masa, con Gracilis no lejos detrás de él.

--Señoras y señores--gritó una voz cuando el Doctor llegó al frente--.
¡Os presento, al dios Mercurio!

Hubo comentarios emocionados, pero el Doctor no dijo nada. Ahora podía
ver la estatua. No era Rose. Era el esclavo Tiro.
