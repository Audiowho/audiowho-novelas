\chapter*{Capítulo Trece}
\addcontentsline{toc}{chapter}{Capítulo Trece}

Rose se quedó mirando el lugar en el que había estado el cordero. Sintió
arcadas.

--No te preocupes, Rose--dijo Minerva--. Incluso los dioses debemos
comer. No es distinto de tu forma de consumir\ldots{}--se detuvo, como
si estuviera buscando las palabras adecuadas--, una costilla de cordero
o un kebab.

--Claro--dijo Rose, mareada--. Creo que quizá nunca vuelva a comer de
eso.

--Adelantaos, Doctor, Vanessa--siguió la diosa--. Ningún daño se os hará
en este lugar.

--¿Estás segura? --dijo el Doctor--. Porque de hecho, este seguidor
tuyo, Ursus ha estado por ahí haciendo bastante daño.

Ursus dio un paso adelante.

--¡Vigila tu lengua cuando hables con la diosa! --le espetó.

El Doctor frunció el ceño.

--Creo que eso haría que al hablar tuviera bastantes problemas--dijo.
Sacó la lengua y bizqueó para mirársela--. \emph{Tertermente eléjeque}.
--dijo.

Ursus gruñó, y entonces el Doctor se encogió de hombros y habló con
normalidad.

--¿Estás negando que has hecho daño, entonces? Porque creo que sí lo has
hecho. Sabes, con todo eso de convertir a la gente en piedra y tal.
Llámame filisteo si quiero, aunque los filisteos de verdad no fueran tan
malos como decían. Puede que no supieran ver un buen cuadro cuando se
les enseñaba, pero sabían cómo montar una buena fiesta al estilo
clásico\ldots{} ¿Dónde estaba? Oh, sí, llámame filisteo si quieres, pero
no veo la justificación para todo el negocio ese de la petrificación
como arte.

--El arte lo justifica todo--dijo sencillamente Ursus.

--Eh, no, no es así--respondió el Doctor--. Punto para mí. ¿Siguiente?

--¡Sin el arte, la vida no tendría significado!

--Mmm. De hecho no puedo decir que esté en total desacuerdo contigo en
ese punto, créeme o no, pero, y creo que ese es el punto en el que
diferimos, pues ya hay mucho arte en el mundo. Mosaicos, cuadros,
música, incluso muchas estatuas esculpidas en roca bastante bonitas. Así
que, resumiendo, la vida es feliz, no hace falta ir por ahí con tus
poderes mágicos.

Ursus se sacó los guantes y levantó las manos, mostrando aquellos
regordetes y deformes dedos.

--¿Sabes lo que es--dijo--, sentirse que tienes el cuerpo equivocado?

--Bueno, de hecho\ldots{}--comenzó el Doctor, mirándose sus propios
dedos.

--Se suponía que tenía que crear arte--siguió Ursus, y Rose de repente
tuvo un flash de memoria, de las palabras que dijo cuando el mundo se
oscureció a su alrededor.

--¿Así que hiciste tus ofrendas a Minerva y le pediste--y se esforzó
para recordar--, que te diera la habilidad de hacer belleza en piedra?

Él asintió.

--Minerva respondió mis súplicas, me permitió hacer lo que nací para
hacer, ser lo que estaba destinado a ser al nacer.

--Me apuesto que debió ser impactante cuando descubriste su don--dijo el
Doctor--. Por un lado, tienes a los verdaderos artesanos esculpiendo con
sus martillos y sus cinceles, y por el otro, tienes a un asesino en
serio acosando gente con su dedo mortal. Por cierto, ¿también te dio
ella los guantes mágicos? Quiero decir, imagino lo que te debe de pasar
cada vez que te quieres rascar la nariz.

--¡No puedo creer que\ldots{} una diosa hiciera algo así! --dijo Rose.
Se giró a Minerva, con dificultad de creerse estar hablando con una
deidad--. ¿Eres verdaderamente feliz de que él vaya yendo por ahí
matando gente de esta manera?

La sobrenatural sonrisa volvió a brillar.

--Pues por supuesto. Él lo hace para glorificarme. Y me trae muchas
ofrendas en su lugar.

--Los dioses romanos estaban a favor de las ofrendas--murmuró el
Doctor--. No les importaba lo que hicieras, mientras cuidaras los ritos
correctamente. Y tenían una relación con sus adoradores que era ``tú me
rascas la espalda y yo te rasco la tuya''. Les das un cerdo, y te
destrozarán al enemigo por ti. Ese tipo de cosas.

--Pero toda la gente\ldots{}

Ursus parecía contrariado.

--Sólo eran esclavos comprados para ese propósito. Los hombres son
comprados para ser asesinados en la arena. Seguro que convertirse en
belleza es una muerte mejor que ser hecho pedazos para un espectáculo de
gladiadores.

Rose abrió y cerró la boca un par de veces, con cada respuesta muriendo
en su lengua. Era gracioso cómo los lugares en la Tierra a veces podrían
serle más extraños que otros plaetas.

--Optatus no era un esclavo--dijo finamente, abandonando el tema de
``matar está mal'' al mismo tiempo.

Ursus rió.

--Ese estúpido de Gracilis no dejó de molestarme con un tributo para su
patético hijo. No pude responder que no. No quiso oír nada de mí cuando
era un fracasado, un don nadie. Y me persiguió solo porque se oía hablar
de mi nombre en Roma--sonrió, maravillado--. Además, me ofreció mucho
dinero, ¿cómo podría negarme?

--Bueno, sabes, si hubiera sido yo habría encontrado una manera--le dijo
Rose--. Lo que has hecho es\ldots{} malvado. ¡Toda esa gente está
muerta!

--De hecho--le interrumpió el Doctor con un susurro--, voy a devolverles
a la vida en\ldots{}--miró a su muñeca, como si estuviera consultando un
reloj imaginario--, en unos dos días, ¿recuerdas? Usando la increíble
cura milagrosa.

El Doctor se detuvo. Rose se giró para mirar, estaba de pie con una
sonrisa en su cara, una sonrisa que conocía bien. Era la sonrisa de
haber descubierto algo.

--¿Qué? --dijo ella.

--Ha habido muchos milagros por aquí, ¿no ha sido así?

Ella estuvo de acuerdo.

--Sí, pero es lo que hacen los dioses, ¿no?

--He visto un montón de cosas extrañas en mis días--dijo el Doctor--.
Cosas que la mayoría de la gente no cree que existan. Abominables
hombres de las nieves. Hombres lobo. Demonios. Vampiros. ¿Pero dioses
romanos con poderes místicos? No lo creo.

Ursus dio un paso adelante.

--¡Vigila tu boca!

--¡Erre que erre! --respondió el Doctor--. ¡Siempre pidiéndome que haga
cosas bastante incómodas, por no decir físicamente imposibles. Mira,
déjame decirlo claramente. La diosa Minerva apareció un buen día, ¿no?

Ursus asintió.

--¿Tras años y años de adorarla, haciendo ofrendas y todo eso?

--Sí.

--Doctor, cuidado--le susurró Rose--. Está ahí.

--No está diciendo muchas cosas en este momento, ¿no? --dijo el Doctor
en alto--. Está ahí de pie, con toda su divinidad. De hecho, solo parece
responder cuando se le habla. Lo que no suena una forma demasiado divina
de comportarse. --sus ojos brillaban tanto como los de la diosa--.
¡Vanessa!

La chica se adelantó.

--¿Sí?

--¿2375?

--Sí--dijo, extrañada.

--¿Cerdeña?

--Sí.

--¿El Bueau Tygon?

--Si.

--¿Salvatorio Moretti?

--Sí.

--En ese caso\ldots{}--el Doctor se giró a Minerva que aún estaba allí,
beatifica y majestuosa--. Desearía ver tu verdadera forma.

Rose pensó que oyó un sonido, algo chasqueando cerca de su oído. Y
entonces Minerva se desvaneció. Sencillamente se desvaneció,
sencillamente así, como si hubiera sido apagada. En el suelo dónde había
estado de pie había una caja de cartón con una SM a un lado. Y en lo
alto de la caja se asomaba una pequeña criatura escamosa, un cruce entre
un dragón bebé y un ornitorrinco.

Todo sucedió muy rápido.

Ursus gritó, casi chillando.

--¿Qué has hecho? --sacó su cuchillo de sacrificio aún lleno de sangre y
corrió hacia el Doctor.

Vanessa gritó.

--¡Esa es la caja! ¡Del estudio de mi padre! --y se apresuró hacia ella,
poniéndose entre medio del Doctor y el furioso escultor.

El Doctor se abalanzó hacia adelante, gritando:

--¡Vanessa! ¡Quédate quieta!

Llegó a su altura cuando Ursus alcanzaba a Vanessa.

El Doctor tocó el brazo de Ursus, alejando el cuchillo de la chica,
justo cuando Ursus le pegaba a ésta un empujón. Vanessa cayó al suelo,
convertida en sólida piedra.

Rose saltó hacia Ursus, gritando:

--¡No! --saltó en su espalda, intentando alejarse de sus manos, y perdió
el equilibro, cayendo de cara. Esperó a que se intentara deshacer de
ella, intentando agarrarla, pero no lo hizo. Y entonces vio el charco
carmesí saliendo de debajo de él.

Había aterrizado encima de su propia draga de sacrificio. Y estaba
muerto.

Lenta y cuidadosamente, se levantó, esperando que todo hubiera sido un
sueño, deseando que se hubiera imaginado lo que había pasado en el calor
del momento.

Pero no era así. Estaba el Doctor, con las manos levantadas,
desesperadamente intentando salvar a Vanessa. Aquel era el Doctor en su
estado más magnánimo. Y sería así para siempre.

Rose se escurrió una lágrima buscando el frasco que el Doctor le había
dado, la cura milagrosa que había usado para devolverle a la vida. Lo
encontró. Estaba total y completamente vacío.

Lo destapó aún así, vaciando el frasco de cristal por encima de su
inmóvil cabeza de piedra. Pero no había ni una sola gota del líquido
restante.

Entonces no pudo evitar las lágrimas.

--¡Doctor! --gritó--. ¡Oh, Doctor! ¿Por qué tuvimos que venir aquí? ¡Es
todo culpa mía! ¡Es culpa mía que estés aquí! Todo eso sobre
posar\ldots{} desearía que no me hubieras escuchado. Desearía que nunca
hubieras venido aquí\ldots{} Yo\ldots{}

Se giró, conmovida. Había oído aquel ruido de nuevo, algo como un
¿trueno? Pero el cielo parecía claro y calmo. Escuchó atentamente pero
no lo oyó.

Cuando se giró de nuevo, tuvo la más ligera de las impresiones de que
faltaba algo.

Había lo que parecía la estatua de una chica en el suelo. Había una
peculiar y extraña forma en una caja de cartón. Y había el cadáver de un
hombre. Aquello no\ldots{} era bueno, pero sabía que se suponía que
tenían que estar allí. No había nada, ni nadie más.

Nunca hubo habido nada. Obviamente tenía que estar equivocada. No
faltaba nada.